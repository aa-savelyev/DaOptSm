\documentclass[11pt,a4paper]{article}

    \usepackage[breakable]{tcolorbox}
    \usepackage{parskip} % Stop auto-indenting (to mimic markdown behaviour)
    
    \usepackage{iftex}
    \ifPDFTeX
      \usepackage[T2A]{fontenc}
      \usepackage{mathpazo}
      \usepackage[russian,english]{babel}
    \else
      \usepackage{fontspec}
      \usepackage{polyglossia}
      \setmainlanguage[babelshorthands=true]{russian}    % Язык по-умолчанию русский с поддержкой приятных команд пакета babel
      \setotherlanguage{english}                         % Дополнительный язык = английский (в американской вариации по-умолчанию)
      \newfontfamily\cyrillicfonttt[Scale=0.87,BoldFont={Fira Mono Medium}] {Fira Mono}  % Моноширинный шрифт для кириллицы
      \defaultfontfeatures{Ligatures=TeX}
      \newfontfamily\cyrillicfont{STIX Two Text}         % Шрифт с засечками для кириллицы
    \fi
    \renewcommand{\linethickness}{0.1ex}

    % Basic figure setup, for now with no caption control since it's done
    % automatically by Pandoc (which extracts ![](path) syntax from Markdown).
    \usepackage{graphicx}
    % Maintain compatibility with old templates. Remove in nbconvert 6.0
    \let\Oldincludegraphics\includegraphics
    % Ensure that by default, figures have no caption (until we provide a
    % proper Figure object with a Caption API and a way to capture that
    % in the conversion process - todo).
    \usepackage{caption}
    \DeclareCaptionFormat{nocaption}{}
    \captionsetup{format=nocaption,aboveskip=0pt,belowskip=0pt}

    \usepackage{float}
    \floatplacement{figure}{H} % forces figures to be placed at the correct location
    \usepackage{xcolor} % Allow colors to be defined
    \usepackage{enumerate} % Needed for markdown enumerations to work
    \usepackage{geometry} % Used to adjust the document margins
    \usepackage{amsmath} % Equations
    \usepackage{amssymb} % Equations
    \usepackage{textcomp} % defines textquotesingle
    % Hack from http://tex.stackexchange.com/a/47451/13684:
    \AtBeginDocument{%
        \def\PYZsq{\textquotesingle}% Upright quotes in Pygmentized code
    }
    \usepackage{upquote} % Upright quotes for verbatim code
    \usepackage{eurosym} % defines \euro
    \usepackage[mathletters]{ucs} % Extended unicode (utf-8) support
    \usepackage{fancyvrb} % verbatim replacement that allows latex
    \usepackage{grffile} % extends the file name processing of package graphics 
                         % to support a larger range
    \makeatletter % fix for old versions of grffile with XeLaTeX
    \@ifpackagelater{grffile}{2019/11/01}
    {
      % Do nothing on new versions
    }
    {
      \def\Gread@@xetex#1{%
        \IfFileExists{"\Gin@base".bb}%
        {\Gread@eps{\Gin@base.bb}}%
        {\Gread@@xetex@aux#1}%
      }
    }
    \makeatother
    \usepackage[Export]{adjustbox} % Used to constrain images to a maximum size
    \adjustboxset{max size={0.9\linewidth}{0.9\paperheight}}

    % The hyperref package gives us a pdf with properly built
    % internal navigation ('pdf bookmarks' for the table of contents,
    % internal cross-reference links, web links for URLs, etc.)
    \usepackage{hyperref}
    % The default LaTeX title has an obnoxious amount of whitespace. By default,
    % titling removes some of it. It also provides customization options.
    \usepackage{titling}
    \usepackage{longtable} % longtable support required by pandoc >1.10
    \usepackage{booktabs}  % table support for pandoc > 1.12.2
    \usepackage[inline]{enumitem} % IRkernel/repr support (it uses the enumerate* environment)
    \usepackage[normalem]{ulem} % ulem is needed to support strikethroughs (\sout)
                                % normalem makes italics be italics, not underlines
    \usepackage{mathrsfs}
    

    
    % Colors for the hyperref package
    \definecolor{urlcolor}{rgb}{0,.145,.698}
    \definecolor{linkcolor}{rgb}{.71,0.21,0.01}
    \definecolor{citecolor}{rgb}{.12,.54,.11}

    % ANSI colors
    \definecolor{ansi-black}{HTML}{3E424D}
    \definecolor{ansi-black-intense}{HTML}{282C36}
    \definecolor{ansi-red}{HTML}{E75C58}
    \definecolor{ansi-red-intense}{HTML}{B22B31}
    \definecolor{ansi-green}{HTML}{00A250}
    \definecolor{ansi-green-intense}{HTML}{007427}
    \definecolor{ansi-yellow}{HTML}{DDB62B}
    \definecolor{ansi-yellow-intense}{HTML}{B27D12}
    \definecolor{ansi-blue}{HTML}{208FFB}
    \definecolor{ansi-blue-intense}{HTML}{0065CA}
    \definecolor{ansi-magenta}{HTML}{D160C4}
    \definecolor{ansi-magenta-intense}{HTML}{A03196}
    \definecolor{ansi-cyan}{HTML}{60C6C8}
    \definecolor{ansi-cyan-intense}{HTML}{258F8F}
    \definecolor{ansi-white}{HTML}{C5C1B4}
    \definecolor{ansi-white-intense}{HTML}{A1A6B2}
    \definecolor{ansi-default-inverse-fg}{HTML}{FFFFFF}
    \definecolor{ansi-default-inverse-bg}{HTML}{000000}

    % common color for the border for error outputs.
    \definecolor{outerrorbackground}{HTML}{FFDFDF}

    % commands and environments needed by pandoc snippets
    % extracted from the output of `pandoc -s`
    \providecommand{\tightlist}{%
      \setlength{\itemsep}{0pt}\setlength{\parskip}{0pt}}
    \DefineVerbatimEnvironment{Highlighting}{Verbatim}{commandchars=\\\{\}}
    % Add ',fontsize=\small' for more characters per line
    \newenvironment{Shaded}{}{}
    \newcommand{\KeywordTok}[1]{\textcolor[rgb]{0.00,0.44,0.13}{\textbf{{#1}}}}
    \newcommand{\DataTypeTok}[1]{\textcolor[rgb]{0.56,0.13,0.00}{{#1}}}
    \newcommand{\DecValTok}[1]{\textcolor[rgb]{0.25,0.63,0.44}{{#1}}}
    \newcommand{\BaseNTok}[1]{\textcolor[rgb]{0.25,0.63,0.44}{{#1}}}
    \newcommand{\FloatTok}[1]{\textcolor[rgb]{0.25,0.63,0.44}{{#1}}}
    \newcommand{\CharTok}[1]{\textcolor[rgb]{0.25,0.44,0.63}{{#1}}}
    \newcommand{\StringTok}[1]{\textcolor[rgb]{0.25,0.44,0.63}{{#1}}}
    \newcommand{\CommentTok}[1]{\textcolor[rgb]{0.38,0.63,0.69}{\textit{{#1}}}}
    \newcommand{\OtherTok}[1]{\textcolor[rgb]{0.00,0.44,0.13}{{#1}}}
    \newcommand{\AlertTok}[1]{\textcolor[rgb]{1.00,0.00,0.00}{\textbf{{#1}}}}
    \newcommand{\FunctionTok}[1]{\textcolor[rgb]{0.02,0.16,0.49}{{#1}}}
    \newcommand{\RegionMarkerTok}[1]{{#1}}
    \newcommand{\ErrorTok}[1]{\textcolor[rgb]{1.00,0.00,0.00}{\textbf{{#1}}}}
    \newcommand{\NormalTok}[1]{{#1}}
    
    % Additional commands for more recent versions of Pandoc
    \newcommand{\ConstantTok}[1]{\textcolor[rgb]{0.53,0.00,0.00}{{#1}}}
    \newcommand{\SpecialCharTok}[1]{\textcolor[rgb]{0.25,0.44,0.63}{{#1}}}
    \newcommand{\VerbatimStringTok}[1]{\textcolor[rgb]{0.25,0.44,0.63}{{#1}}}
    \newcommand{\SpecialStringTok}[1]{\textcolor[rgb]{0.73,0.40,0.53}{{#1}}}
    \newcommand{\ImportTok}[1]{{#1}}
    \newcommand{\DocumentationTok}[1]{\textcolor[rgb]{0.73,0.13,0.13}{\textit{{#1}}}}
    \newcommand{\AnnotationTok}[1]{\textcolor[rgb]{0.38,0.63,0.69}{\textbf{\textit{{#1}}}}}
    \newcommand{\CommentVarTok}[1]{\textcolor[rgb]{0.38,0.63,0.69}{\textbf{\textit{{#1}}}}}
    \newcommand{\VariableTok}[1]{\textcolor[rgb]{0.10,0.09,0.49}{{#1}}}
    \newcommand{\ControlFlowTok}[1]{\textcolor[rgb]{0.00,0.44,0.13}{\textbf{{#1}}}}
    \newcommand{\OperatorTok}[1]{\textcolor[rgb]{0.40,0.40,0.40}{{#1}}}
    \newcommand{\BuiltInTok}[1]{{#1}}
    \newcommand{\ExtensionTok}[1]{{#1}}
    \newcommand{\PreprocessorTok}[1]{\textcolor[rgb]{0.74,0.48,0.00}{{#1}}}
    \newcommand{\AttributeTok}[1]{\textcolor[rgb]{0.49,0.56,0.16}{{#1}}}
    \newcommand{\InformationTok}[1]{\textcolor[rgb]{0.38,0.63,0.69}{\textbf{\textit{{#1}}}}}
    \newcommand{\WarningTok}[1]{\textcolor[rgb]{0.38,0.63,0.69}{\textbf{\textit{{#1}}}}}
    
    
    % Define a nice break command that doesn't care if a line doesn't already
    % exist.
    \def\br{\hspace*{\fill} \\* }
    % Math Jax compatibility definitions
    \def\gt{>}
    \def\lt{<}
    \let\Oldtex\TeX
    \let\Oldlatex\LaTeX
    \renewcommand{\TeX}{\textrm{\Oldtex}}
    \renewcommand{\LaTeX}{\textrm{\Oldlatex}}
    % Document parameters
    % Document title
    \title{
      {\Large Лекция 2} \\
      Матрицы и действия над ними. Ранг матрицы
    }
    \date{8 сентября 2021\,г.}
    
    
    
% Pygments definitions
\makeatletter
\def\PY@reset{\let\PY@it=\relax \let\PY@bf=\relax%
    \let\PY@ul=\relax \let\PY@tc=\relax%
    \let\PY@bc=\relax \let\PY@ff=\relax}
\def\PY@tok#1{\csname PY@tok@#1\endcsname}
\def\PY@toks#1+{\ifx\relax#1\empty\else%
    \PY@tok{#1}\expandafter\PY@toks\fi}
\def\PY@do#1{\PY@bc{\PY@tc{\PY@ul{%
    \PY@it{\PY@bf{\PY@ff{#1}}}}}}}
\def\PY#1#2{\PY@reset\PY@toks#1+\relax+\PY@do{#2}}

\expandafter\def\csname PY@tok@w\endcsname{\def\PY@tc##1{\textcolor[rgb]{0.73,0.73,0.73}{##1}}}
\expandafter\def\csname PY@tok@c\endcsname{\let\PY@it=\textit\def\PY@tc##1{\textcolor[rgb]{0.25,0.50,0.50}{##1}}}
\expandafter\def\csname PY@tok@cp\endcsname{\def\PY@tc##1{\textcolor[rgb]{0.74,0.48,0.00}{##1}}}
\expandafter\def\csname PY@tok@k\endcsname{\let\PY@bf=\textbf\def\PY@tc##1{\textcolor[rgb]{0.00,0.50,0.00}{##1}}}
\expandafter\def\csname PY@tok@kp\endcsname{\def\PY@tc##1{\textcolor[rgb]{0.00,0.50,0.00}{##1}}}
\expandafter\def\csname PY@tok@kt\endcsname{\def\PY@tc##1{\textcolor[rgb]{0.69,0.00,0.25}{##1}}}
\expandafter\def\csname PY@tok@o\endcsname{\def\PY@tc##1{\textcolor[rgb]{0.40,0.40,0.40}{##1}}}
\expandafter\def\csname PY@tok@ow\endcsname{\let\PY@bf=\textbf\def\PY@tc##1{\textcolor[rgb]{0.67,0.13,1.00}{##1}}}
\expandafter\def\csname PY@tok@nb\endcsname{\def\PY@tc##1{\textcolor[rgb]{0.00,0.50,0.00}{##1}}}
\expandafter\def\csname PY@tok@nf\endcsname{\def\PY@tc##1{\textcolor[rgb]{0.00,0.00,1.00}{##1}}}
\expandafter\def\csname PY@tok@nc\endcsname{\let\PY@bf=\textbf\def\PY@tc##1{\textcolor[rgb]{0.00,0.00,1.00}{##1}}}
\expandafter\def\csname PY@tok@nn\endcsname{\let\PY@bf=\textbf\def\PY@tc##1{\textcolor[rgb]{0.00,0.00,1.00}{##1}}}
\expandafter\def\csname PY@tok@ne\endcsname{\let\PY@bf=\textbf\def\PY@tc##1{\textcolor[rgb]{0.82,0.25,0.23}{##1}}}
\expandafter\def\csname PY@tok@nv\endcsname{\def\PY@tc##1{\textcolor[rgb]{0.10,0.09,0.49}{##1}}}
\expandafter\def\csname PY@tok@no\endcsname{\def\PY@tc##1{\textcolor[rgb]{0.53,0.00,0.00}{##1}}}
\expandafter\def\csname PY@tok@nl\endcsname{\def\PY@tc##1{\textcolor[rgb]{0.63,0.63,0.00}{##1}}}
\expandafter\def\csname PY@tok@ni\endcsname{\let\PY@bf=\textbf\def\PY@tc##1{\textcolor[rgb]{0.60,0.60,0.60}{##1}}}
\expandafter\def\csname PY@tok@na\endcsname{\def\PY@tc##1{\textcolor[rgb]{0.49,0.56,0.16}{##1}}}
\expandafter\def\csname PY@tok@nt\endcsname{\let\PY@bf=\textbf\def\PY@tc##1{\textcolor[rgb]{0.00,0.50,0.00}{##1}}}
\expandafter\def\csname PY@tok@nd\endcsname{\def\PY@tc##1{\textcolor[rgb]{0.67,0.13,1.00}{##1}}}
\expandafter\def\csname PY@tok@s\endcsname{\def\PY@tc##1{\textcolor[rgb]{0.73,0.13,0.13}{##1}}}
\expandafter\def\csname PY@tok@sd\endcsname{\let\PY@it=\textit\def\PY@tc##1{\textcolor[rgb]{0.73,0.13,0.13}{##1}}}
\expandafter\def\csname PY@tok@si\endcsname{\let\PY@bf=\textbf\def\PY@tc##1{\textcolor[rgb]{0.73,0.40,0.53}{##1}}}
\expandafter\def\csname PY@tok@se\endcsname{\let\PY@bf=\textbf\def\PY@tc##1{\textcolor[rgb]{0.73,0.40,0.13}{##1}}}
\expandafter\def\csname PY@tok@sr\endcsname{\def\PY@tc##1{\textcolor[rgb]{0.73,0.40,0.53}{##1}}}
\expandafter\def\csname PY@tok@ss\endcsname{\def\PY@tc##1{\textcolor[rgb]{0.10,0.09,0.49}{##1}}}
\expandafter\def\csname PY@tok@sx\endcsname{\def\PY@tc##1{\textcolor[rgb]{0.00,0.50,0.00}{##1}}}
\expandafter\def\csname PY@tok@m\endcsname{\def\PY@tc##1{\textcolor[rgb]{0.40,0.40,0.40}{##1}}}
\expandafter\def\csname PY@tok@gh\endcsname{\let\PY@bf=\textbf\def\PY@tc##1{\textcolor[rgb]{0.00,0.00,0.50}{##1}}}
\expandafter\def\csname PY@tok@gu\endcsname{\let\PY@bf=\textbf\def\PY@tc##1{\textcolor[rgb]{0.50,0.00,0.50}{##1}}}
\expandafter\def\csname PY@tok@gd\endcsname{\def\PY@tc##1{\textcolor[rgb]{0.63,0.00,0.00}{##1}}}
\expandafter\def\csname PY@tok@gi\endcsname{\def\PY@tc##1{\textcolor[rgb]{0.00,0.63,0.00}{##1}}}
\expandafter\def\csname PY@tok@gr\endcsname{\def\PY@tc##1{\textcolor[rgb]{1.00,0.00,0.00}{##1}}}
\expandafter\def\csname PY@tok@ge\endcsname{\let\PY@it=\textit}
\expandafter\def\csname PY@tok@gs\endcsname{\let\PY@bf=\textbf}
\expandafter\def\csname PY@tok@gp\endcsname{\let\PY@bf=\textbf\def\PY@tc##1{\textcolor[rgb]{0.00,0.00,0.50}{##1}}}
\expandafter\def\csname PY@tok@go\endcsname{\def\PY@tc##1{\textcolor[rgb]{0.53,0.53,0.53}{##1}}}
\expandafter\def\csname PY@tok@gt\endcsname{\def\PY@tc##1{\textcolor[rgb]{0.00,0.27,0.87}{##1}}}
\expandafter\def\csname PY@tok@err\endcsname{\def\PY@bc##1{\setlength{\fboxsep}{0pt}\fcolorbox[rgb]{1.00,0.00,0.00}{1,1,1}{\strut ##1}}}
\expandafter\def\csname PY@tok@kc\endcsname{\let\PY@bf=\textbf\def\PY@tc##1{\textcolor[rgb]{0.00,0.50,0.00}{##1}}}
\expandafter\def\csname PY@tok@kd\endcsname{\let\PY@bf=\textbf\def\PY@tc##1{\textcolor[rgb]{0.00,0.50,0.00}{##1}}}
\expandafter\def\csname PY@tok@kn\endcsname{\let\PY@bf=\textbf\def\PY@tc##1{\textcolor[rgb]{0.00,0.50,0.00}{##1}}}
\expandafter\def\csname PY@tok@kr\endcsname{\let\PY@bf=\textbf\def\PY@tc##1{\textcolor[rgb]{0.00,0.50,0.00}{##1}}}
\expandafter\def\csname PY@tok@bp\endcsname{\def\PY@tc##1{\textcolor[rgb]{0.00,0.50,0.00}{##1}}}
\expandafter\def\csname PY@tok@fm\endcsname{\def\PY@tc##1{\textcolor[rgb]{0.00,0.00,1.00}{##1}}}
\expandafter\def\csname PY@tok@vc\endcsname{\def\PY@tc##1{\textcolor[rgb]{0.10,0.09,0.49}{##1}}}
\expandafter\def\csname PY@tok@vg\endcsname{\def\PY@tc##1{\textcolor[rgb]{0.10,0.09,0.49}{##1}}}
\expandafter\def\csname PY@tok@vi\endcsname{\def\PY@tc##1{\textcolor[rgb]{0.10,0.09,0.49}{##1}}}
\expandafter\def\csname PY@tok@vm\endcsname{\def\PY@tc##1{\textcolor[rgb]{0.10,0.09,0.49}{##1}}}
\expandafter\def\csname PY@tok@sa\endcsname{\def\PY@tc##1{\textcolor[rgb]{0.73,0.13,0.13}{##1}}}
\expandafter\def\csname PY@tok@sb\endcsname{\def\PY@tc##1{\textcolor[rgb]{0.73,0.13,0.13}{##1}}}
\expandafter\def\csname PY@tok@sc\endcsname{\def\PY@tc##1{\textcolor[rgb]{0.73,0.13,0.13}{##1}}}
\expandafter\def\csname PY@tok@dl\endcsname{\def\PY@tc##1{\textcolor[rgb]{0.73,0.13,0.13}{##1}}}
\expandafter\def\csname PY@tok@s2\endcsname{\def\PY@tc##1{\textcolor[rgb]{0.73,0.13,0.13}{##1}}}
\expandafter\def\csname PY@tok@sh\endcsname{\def\PY@tc##1{\textcolor[rgb]{0.73,0.13,0.13}{##1}}}
\expandafter\def\csname PY@tok@s1\endcsname{\def\PY@tc##1{\textcolor[rgb]{0.73,0.13,0.13}{##1}}}
\expandafter\def\csname PY@tok@mb\endcsname{\def\PY@tc##1{\textcolor[rgb]{0.40,0.40,0.40}{##1}}}
\expandafter\def\csname PY@tok@mf\endcsname{\def\PY@tc##1{\textcolor[rgb]{0.40,0.40,0.40}{##1}}}
\expandafter\def\csname PY@tok@mh\endcsname{\def\PY@tc##1{\textcolor[rgb]{0.40,0.40,0.40}{##1}}}
\expandafter\def\csname PY@tok@mi\endcsname{\def\PY@tc##1{\textcolor[rgb]{0.40,0.40,0.40}{##1}}}
\expandafter\def\csname PY@tok@il\endcsname{\def\PY@tc##1{\textcolor[rgb]{0.40,0.40,0.40}{##1}}}
\expandafter\def\csname PY@tok@mo\endcsname{\def\PY@tc##1{\textcolor[rgb]{0.40,0.40,0.40}{##1}}}
\expandafter\def\csname PY@tok@ch\endcsname{\let\PY@it=\textit\def\PY@tc##1{\textcolor[rgb]{0.25,0.50,0.50}{##1}}}
\expandafter\def\csname PY@tok@cm\endcsname{\let\PY@it=\textit\def\PY@tc##1{\textcolor[rgb]{0.25,0.50,0.50}{##1}}}
\expandafter\def\csname PY@tok@cpf\endcsname{\let\PY@it=\textit\def\PY@tc##1{\textcolor[rgb]{0.25,0.50,0.50}{##1}}}
\expandafter\def\csname PY@tok@c1\endcsname{\let\PY@it=\textit\def\PY@tc##1{\textcolor[rgb]{0.25,0.50,0.50}{##1}}}
\expandafter\def\csname PY@tok@cs\endcsname{\let\PY@it=\textit\def\PY@tc##1{\textcolor[rgb]{0.25,0.50,0.50}{##1}}}

\def\PYZbs{\char`\\}
\def\PYZus{\char`\_}
\def\PYZob{\char`\{}
\def\PYZcb{\char`\}}
\def\PYZca{\char`\^}
\def\PYZam{\char`\&}
\def\PYZlt{\char`\<}
\def\PYZgt{\char`\>}
\def\PYZsh{\char`\#}
\def\PYZpc{\char`\%}
\def\PYZdl{\char`\$}
\def\PYZhy{\char`\-}
\def\PYZsq{\char`\'}
\def\PYZdq{\char`\"}
\def\PYZti{\char`\~}
% for compatibility with earlier versions
\def\PYZat{@}
\def\PYZlb{[}
\def\PYZrb{]}
\makeatother


    % For linebreaks inside Verbatim environment from package fancyvrb. 
    \makeatletter
        \newbox\Wrappedcontinuationbox 
        \newbox\Wrappedvisiblespacebox 
        \newcommand*\Wrappedvisiblespace {\textcolor{red}{\textvisiblespace}} 
        \newcommand*\Wrappedcontinuationsymbol {\textcolor{red}{\llap{\tiny$\m@th\hookrightarrow$}}} 
        \newcommand*\Wrappedcontinuationindent {3ex } 
        \newcommand*\Wrappedafterbreak {\kern\Wrappedcontinuationindent\copy\Wrappedcontinuationbox} 
        % Take advantage of the already applied Pygments mark-up to insert 
        % potential linebreaks for TeX processing. 
        %        {, <, #, %, $, ' and ": go to next line. 
        %        _, }, ^, &, >, - and ~: stay at end of broken line. 
        % Use of \textquotesingle for straight quote. 
        \newcommand*\Wrappedbreaksatspecials {% 
            \def\PYGZus{\discretionary{\char`\_}{\Wrappedafterbreak}{\char`\_}}% 
            \def\PYGZob{\discretionary{}{\Wrappedafterbreak\char`\{}{\char`\{}}% 
            \def\PYGZcb{\discretionary{\char`\}}{\Wrappedafterbreak}{\char`\}}}% 
            \def\PYGZca{\discretionary{\char`\^}{\Wrappedafterbreak}{\char`\^}}% 
            \def\PYGZam{\discretionary{\char`\&}{\Wrappedafterbreak}{\char`\&}}% 
            \def\PYGZlt{\discretionary{}{\Wrappedafterbreak\char`\<}{\char`\<}}% 
            \def\PYGZgt{\discretionary{\char`\>}{\Wrappedafterbreak}{\char`\>}}% 
            \def\PYGZsh{\discretionary{}{\Wrappedafterbreak\char`\#}{\char`\#}}% 
            \def\PYGZpc{\discretionary{}{\Wrappedafterbreak\char`\%}{\char`\%}}% 
            \def\PYGZdl{\discretionary{}{\Wrappedafterbreak\char`\$}{\char`\$}}% 
            \def\PYGZhy{\discretionary{\char`\-}{\Wrappedafterbreak}{\char`\-}}% 
            \def\PYGZsq{\discretionary{}{\Wrappedafterbreak\textquotesingle}{\textquotesingle}}% 
            \def\PYGZdq{\discretionary{}{\Wrappedafterbreak\char`\"}{\char`\"}}% 
            \def\PYGZti{\discretionary{\char`\~}{\Wrappedafterbreak}{\char`\~}}% 
        } 
        % Some characters . , ; ? ! / are not pygmentized. 
        % This macro makes them "active" and they will insert potential linebreaks 
        \newcommand*\Wrappedbreaksatpunct {% 
            \lccode`\~`\.\lowercase{\def~}{\discretionary{\hbox{\char`\.}}{\Wrappedafterbreak}{\hbox{\char`\.}}}% 
            \lccode`\~`\,\lowercase{\def~}{\discretionary{\hbox{\char`\,}}{\Wrappedafterbreak}{\hbox{\char`\,}}}% 
            \lccode`\~`\;\lowercase{\def~}{\discretionary{\hbox{\char`\;}}{\Wrappedafterbreak}{\hbox{\char`\;}}}% 
            \lccode`\~`\:\lowercase{\def~}{\discretionary{\hbox{\char`\:}}{\Wrappedafterbreak}{\hbox{\char`\:}}}% 
            \lccode`\~`\?\lowercase{\def~}{\discretionary{\hbox{\char`\?}}{\Wrappedafterbreak}{\hbox{\char`\?}}}% 
            \lccode`\~`\!\lowercase{\def~}{\discretionary{\hbox{\char`\!}}{\Wrappedafterbreak}{\hbox{\char`\!}}}% 
            \lccode`\~`\/\lowercase{\def~}{\discretionary{\hbox{\char`\/}}{\Wrappedafterbreak}{\hbox{\char`\/}}}% 
            \catcode`\.\active
            \catcode`\,\active 
            \catcode`\;\active
            \catcode`\:\active
            \catcode`\?\active
            \catcode`\!\active
            \catcode`\/\active 
            \lccode`\~`\~ 	
        }
    \makeatother

    \let\OriginalVerbatim=\Verbatim
    \makeatletter
    \renewcommand{\Verbatim}[1][1]{%
        %\parskip\z@skip
        \sbox\Wrappedcontinuationbox {\Wrappedcontinuationsymbol}%
        \sbox\Wrappedvisiblespacebox {\FV@SetupFont\Wrappedvisiblespace}%
        \def\FancyVerbFormatLine ##1{\hsize\linewidth
            \vtop{\raggedright\hyphenpenalty\z@\exhyphenpenalty\z@
                \doublehyphendemerits\z@\finalhyphendemerits\z@
                \strut ##1\strut}%
        }%
        % If the linebreak is at a space, the latter will be displayed as visible
        % space at end of first line, and a continuation symbol starts next line.
        % Stretch/shrink are however usually zero for typewriter font.
        \def\FV@Space {%
            \nobreak\hskip\z@ plus\fontdimen3\font minus\fontdimen4\font
            \discretionary{\copy\Wrappedvisiblespacebox}{\Wrappedafterbreak}
            {\kern\fontdimen2\font}%
        }%
        
        % Allow breaks at special characters using \PYG... macros.
        \Wrappedbreaksatspecials
        % Breaks at punctuation characters . , ; ? ! and / need catcode=\active 	
        \OriginalVerbatim[#1,codes*=\Wrappedbreaksatpunct]%
    }
    \makeatother

    % Exact colors from NB
    \definecolor{incolor}{HTML}{303F9F}
    \definecolor{outcolor}{HTML}{D84315}
    \definecolor{cellborder}{HTML}{CFCFCF}
    \definecolor{cellbackground}{HTML}{F7F7F7}
    
    % prompt
    \makeatletter
    \newcommand{\boxspacing}{\kern\kvtcb@left@rule\kern\kvtcb@boxsep}
    \makeatother
    \newcommand{\prompt}[4]{
        \ttfamily\llap{{\color{#2}[#3]:\hspace{3pt}#4}}\vspace{-\baselineskip}
    }
    

    
    % Prevent overflowing lines due to hard-to-break entities
    \sloppy 
    % Setup hyperref package
    \hypersetup{
      breaklinks=true,  % so long urls are correctly broken across lines
      colorlinks=true,
      urlcolor=urlcolor,
      linkcolor=linkcolor,
      citecolor=citecolor,
      }
    % Slightly bigger margins than the latex defaults
    
    \geometry{verbose,tmargin=1in,bmargin=1in,lmargin=1in,rmargin=1in}
    
    

\begin{document}
\maketitle
\thispagestyle{empty}
\tableofcontents
\pagebreak


\hypertarget{ux432ux432ux435ux434ux435ux43dux438ux435}{%
\section{Введение}\label{ux432ux432ux435ux434ux435ux43dux438ux435}}

\hypertarget{ux43eux441ux43dux43eux432ux43dux44bux435-ux437ux430ux434ux430ux447ux438}{%
\subsection{Основные
задачи}\label{ux43eux441ux43dux43eux432ux43dux44bux435-ux437ux430ux434ux430ux447ux438}}

Что мы хотим вспомнить из линейной алгебры? Мы хотим уметь решать пять
базовых задач:

\begin{enumerate}
\def\labelenumi{\arabic{enumi}.}
\item
  Найти \(\mathbf{x}\) в уравнении \(A\mathbf{x} = \mathbf{b}\),
\item
  Найти \(\mathbf{x}\) и \(\lambda\) в уравнении
  \(A\mathbf{x} = \lambda \mathbf{x}\),
\item
  Найти \(\mathbf{v}\), \(\mathbf{u}\) и \(\sigma\) в уравнении
  \(A\mathbf{v} = \sigma \mathbf{u}\).
\end{enumerate}

Подробнее о задачах:
\begin{enumerate}
\item
  Можно ли вектор \(\mathbf{b}\) представить в виде линейной комбинации
  векторов матрицы \(A\)?
\item
  Вектор \(A\mathbf{x}\) имеет то же направление, что и вектор
  \(\mathbf{x}\). Вдоль этого направления все сложные взаимодействия с
  матрицей \(A\) чрезвычайно упрощаются. Например, вектор
  \(A^2 \mathbf{x}\) становится просто \(\lambda^2 \mathbf{x}\).
  Упрощается вычисление матричной экспоненты:
  \(e^{A} = e^{X \Lambda X^{-1}} = X e^{\Lambda} X^{-1}\). Короче
  говоря, многие действия становятся линейными.
\item
  Уравнение \(A\mathbf{v} = \sigma \mathbf{u}\) похоже на предыдущее. Но
  матрица \(A\) больше не квадратная. Это сингулярное разложение.
\end{enumerate}

    \hypertarget{ux43eux431ux43eux437ux43dux430ux447ux435ux43dux438ux44f}{%
\subsection{Обозначения}\label{ux43eux431ux43eux437ux43dux430ux447ux435ux43dux438ux44f}}

Для начала введём обозначения. Вектор будем записывать в виде столбца и
обозначать стрелочкой \(\vec{a}\) или жирной буквой \(\mathbf{a}\).
Строку будем обозначать с помощью звёздочки \(\vec{a}^*\) или
\(\mathbf{a}^*\): \[
  \mathbf{a} =
  \begin{pmatrix}
     a_1    \\
     \cdots \\
     a_n    \\
  \end{pmatrix},
  \mathbf{a}^* = (a_1, \ldots, a_n).
\]

    \begin{center}\rule{0.5\linewidth}{\linethickness}\end{center}

    \hypertarget{ux443ux43cux43dux43eux436ux435ux43dux438ux435-ux43cux430ux442ux440ux438ux446}{%
\section{Умножение
матриц}\label{ux443ux43cux43dux43eux436ux435ux43dux438ux435-ux43cux430ux442ux440ux438ux446}}

Рассмотрим 4 способа умножения матриц \(A \cdot B = C\):

\begin{enumerate}
\def\labelenumi{\arabic{enumi}.}
\tightlist
\item
  Строка на столбец
\item
  Столбец на строку
\item
  Столбец на столбец
\item
  Строка на строку
\end{enumerate}

    \hypertarget{ux441ux43fux43eux441ux43eux431-1-ux441ux442ux440ux43eux43aux430-ux43dux430-ux441ux442ux43eux43bux431ux435ux446}{%
\subsection{Способ 1: «строка на
столбец»}\label{ux441ux43fux43eux441ux43eux431-1-ux441ux442ux440ux43eux43aux430-ux43dux430-ux441ux442ux43eux43bux431ux435ux446}}

Это стандартный способ умножения матриц:
\(\mathbf{c_{ij}} = \mathbf{a}_i^* \cdot \mathbf{b}_j\).

\textbf{Пример 1.} Скалярное произведение векторов \(\mathbf{a}\) и
\(\mathbf{b}\):
\[ (\mathbf{a}, \mathbf{b}) = \mathbf{a}^\top \cdot \mathbf{b}. \]

    \hypertarget{ux441ux43fux43eux441ux43eux431-2-ux441ux442ux43eux43bux431ux435ux446-ux43dux430-ux441ux442ux440ux43eux43aux443}{%
\subsection{Способ 2: «столбец на
строку»}\label{ux441ux43fux43eux441ux43eux431-2-ux441ux442ux43eux43bux431ux435ux446-ux43dux430-ux441ux442ux440ux43eux43aux443}}

Существует другой способ умножения матриц. Произведение \(AB\) равно
\emph{сумме произведений} всех столбцов матрицы \(A\) на соответствующие
строки матрицы \(B\).
\[ A \cdot B = \sum_i \mathbf{a}_i \cdot \mathbf{b}_i^*. \]

\textbf{Пример 2.} \[
  \begin{pmatrix}
     1 & 2 \\
     0 & 2 \\
  \end{pmatrix}
  \cdot
  \begin{pmatrix}
     0 & 5 \\
     1 & 1 \\
  \end{pmatrix}
  =
  \begin{pmatrix}
     1 \\
     0 \\
  \end{pmatrix}
  \cdot
  \begin{pmatrix}
     0 & 5 \\
  \end{pmatrix}
  +
  \begin{pmatrix}
     2 \\
     2 \\
  \end{pmatrix}
  \cdot
  \begin{pmatrix}
     1 & 1 \\
  \end{pmatrix}
  =
  \begin{pmatrix}
     0 & 5 \\
     0 & 0 \\
  \end{pmatrix}
  +
  \begin{pmatrix}
     2 & 2 \\
     2 & 2 \\
  \end{pmatrix}
  =
  \begin{pmatrix}
     2 & 7 \\
     2 & 2 \\
  \end{pmatrix}.
\]

    \textbf{Пример 3.} Разложение матрицы на сумму матриц ранга 1.

Представим матрицу в виде произведения двух матриц и применим умножение
«столбец на строку»: \[
  A = 
  \begin{pmatrix}
     1 & 2 & 3 \\
     2 & 1 & 3 \\
     3 & 1 & 4 \\
  \end{pmatrix}
  =
  \begin{pmatrix}
     1 & 2 \\
     2 & 1 \\
     3 & 1 \\
  \end{pmatrix}
  \cdot
  \begin{pmatrix}
     1 & 0 & 1 \\
     0 & 1 & 1 \\
  \end{pmatrix}
  =
  \begin{pmatrix}
     1 & 0 & 1 \\
     2 & 0 & 2 \\
     3 & 0 & 3 \\
  \end{pmatrix}
  +
  \begin{pmatrix}
     0 & 2 & 2 \\
     0 & 1 & 1 \\
     0 & 1 & 1 \\
  \end{pmatrix}.
\]

    \hypertarget{ux441ux43fux43eux441ux43eux431-3-ux441ux442ux43eux43bux431ux435ux446-ux43dux430-ux441ux442ux43eux43bux431ux435ux446}{%
\subsection{Способ 3: «столбец на
столбец»}\label{ux441ux43fux43eux441ux43eux431-3-ux441ux442ux43eux43bux431ux435ux446-ux43dux430-ux441ux442ux43eux43bux431ux435ux446}}

Посмотрим внимательно на то, как мы получили первый столбец
результирующей матрицы. Он является суммой столбцов матрицы \(A\) с
коэффициентами из первого столбца матрицы \(B\)! То же самое можно
сказать и про второй столбец. Таким образом, можно сформулировать
следующее правило:\\
\emph{\(\mathbf{i}\)-ый столбец результирующей матрицы есть линейная
комбинация столбцов левой матрицы с коэффициентами из \(\mathbf{i}\)-ого
столбца правой матрицы.}

    \hypertarget{ux441ux43fux43eux441ux43eux431-4-ux441ux442ux440ux43eux43aux430-ux43dux430-ux441ux442ux440ux43eux43aux443}{%
\subsection{Способ 4: «строка на
строку»}\label{ux441ux43fux43eux441ux43eux431-4-ux441ux442ux440ux43eux43aux430-ux43dux430-ux441ux442ux440ux43eux43aux443}}

Аналогичным образом можно вывести правило и для строк:\\
\emph{\(\mathbf{i}\)-ая строка результирующей матрицы есть линейная
комбинация строк правой матрицы с коэффициентами из \(\mathbf{i}\)-ой
строки левой матрицы.}

    \textbf{Пример 4. Умножение матрицы на вектор}

Рассмотрим умножение матрицы \(A\) размером \(m \times n\) на вектор:
\(A \mathbf{x}\).

По определению произведение \(A \mathbf{x}\) есть вектор, в котором на
\(i\)-ом месте находится скалярное произведение \(i\)-ой \emph{строки}
на столбец \(\mathbf{x}\): \[
  A \mathbf{x} = 
  \begin{pmatrix}
    -\, \mathbf{a}_1^* \,- \\
    \cdots \\
    -\, \mathbf{a}_i^* \,- \\
    \cdots \\
    -\, \mathbf{a}_m^* \,- \\
  \end{pmatrix}
  \cdot \mathbf{x} = 
  \begin{pmatrix}
    \mathbf{a}_1^* \cdot \mathbf{x} \\
    \cdots \\
    \mathbf{a}_i^* \cdot \mathbf{x} \\
    \cdots \\
    \mathbf{a}_m^* \cdot \mathbf{x} \\
  \end{pmatrix}.
\]

Но можно посмотреть на это иначе, как на произведение \emph{столбцов}
матрицы \(A\) на элементы вектора \(\mathbf{x}\): \[
  A \mathbf{x} = 
  \begin{pmatrix}
     | & {} & | & {} & | \\
     \mathbf{a}_1 & \cdots & \mathbf{a}_i & \cdots & \mathbf{a}_n \\
     | & {} & | & {} & | \\
  \end{pmatrix}
  \begin{pmatrix}
     x_1    \\
     \cdots \\
     x_i    \\
     \cdots \\
     x_m \\
  \end{pmatrix}
  = 
  x_1 \mathbf{a}_1 + \dots + x_i \mathbf{a}_i + \dots + x_m \mathbf{a}_m.
\] Таким образом, результирующий вектор есть \emph{линейная комбинация}
столбцов матрицы \(A\) с коэффициентами из вектора \(\mathbf{x}\).

    \textbf{Пример 5. Умножение столбцов (строк) матрицы на скаляры}

Чтобы каждый вектор матрицы \(A\) умножить на скаляр \(\lambda_i\),
нужно умножить \(A\) на матрицу \(\Lambda = \mathrm{diag}(\lambda_i)\)
\emph{справа}: \[
  \begin{pmatrix}
    | & {} & | \\
    \lambda_1 \mathbf{a}_1 & \cdots & \lambda_n \mathbf{a}_n \\
    | & {} & | \\
  \end{pmatrix}
  =
  \begin{pmatrix}
    | & {} & | \\
    \mathbf{a}_1 & \cdots & \mathbf{a}_n \\
    | & {} & | \\
  \end{pmatrix}
  \begin{pmatrix}
    \lambda_{1} & \ldots & 0         \\
    \vdots      & \ddots & \vdots    \\
    0           & \ldots & \lambda_n \\
  \end{pmatrix}
  = A \cdot \Lambda.
\]

Чтобы проделать то же самое со строками матрицы, её нужно умножить на
\(\Lambda\) \emph{слева}: \[
  \begin{pmatrix}
    -\, \lambda_1 \mathbf{a}_1^* \,- \\
    \cdots \\
    -\, \lambda_m \mathbf{a}_m^* \,- \\
  \end{pmatrix}
  =
  \begin{pmatrix}
    \lambda_{1} & \ldots & 0         \\
    \vdots      & \ddots & \vdots    \\
    0           & \ldots & \lambda_n \\
  \end{pmatrix}
  \begin{pmatrix}
    -\, \mathbf{a}_1^* \,- \\
    \cdots \\
    -\, \mathbf{a}_m^* \,- \\
  \end{pmatrix}
  = \Lambda \cdot A.
\]

    \textbf{Пример 6. Умножение блочных матриц}

Рассмотрим блочную матрицу следующего вида:
\[ A = \begin{pmatrix} A_1 \\ A_2 \\ \end{pmatrix}. \]

\begin{enumerate}
\def\labelenumi{\arabic{enumi}.}
\tightlist
\item
  Формулу для \(A A^\top\) получим по способу «строка на столбец»: \[
    A A^\top = 
    \begin{pmatrix}
   A_1 \\
   A_2 \\
    \end{pmatrix}
    \cdot
    \begin{pmatrix}
   A_1^\top & A_2^\top \\
    \end{pmatrix}
    =
    \begin{pmatrix}
   A_1 A_1^\top & A_1 A_2^\top \\
   A_2 A_1^\top & A_2 A_2^\top \\
    \end{pmatrix}
  \]
\item
  Для \(A^\top A\) удобно применить способ «столбец на строку»: \[
    A^\top A =
    \begin{pmatrix}
   A_1^\top & A_2^\top \\
    \end{pmatrix}
    \begin{pmatrix}
   A_1 \\
   A_2 \\
    \end{pmatrix}
    = A_1^\top A_1 + A_2^\top A_2
  \]
\end{enumerate}

    \begin{center}\rule{0.5\linewidth}{\linethickness}\end{center}

    \hypertarget{ux440ux430ux43dux433-ux43cux430ux442ux440ux438ux446ux44b}{%
\section{Ранг
матрицы}\label{ux440ux430ux43dux433-ux43cux430ux442ux440ux438ux446ux44b}}

Посмотрим на задачу \(A \mathbf{x} = \mathbf{b}\).\\
Её можно сформулировать в виде следующего вопроса: можно ли столбец
\(\mathbf{b}\) представить в виде линейной комбинации столбцов матрицы
\(A\)?

Поясним на примере. Пусть \[
  A = 
  \begin{pmatrix}
     1 & 2 & 3 \\
     2 & 1 & 3 \\
     3 & 1 & 4 \\
  \end{pmatrix}.
\] Что мы можем сказать о линейной оболочке её столбцов? Что это за
пространство? Какой размерности?

    \textbf{Определение 1.} Рангом матрицы \(A\) с \(m\) строк и \(n\)
столбцов называется максимальное число линейно независимых столбцов
(строк).

    \textbf{Свойства ранга:}

\begin{enumerate}
\def\labelenumi{\arabic{enumi}.}
\tightlist
\item
  \(r(AB) \le r(A), r(B)\),
\item
  \(r(A+B) \le r(A) + r(B)\),
\item
  \(r(A^\top A) = r(AA^\top) = r(A) = r(A^\top)\),
\item
  Пусть \(A: m \times r\), \(B: r \times n\) и \(r(A) = r(B) = r\),
  тогда \(r(AB) = r\).
\end{enumerate}

\textbf{Доказательства:}

\begin{enumerate}
\def\labelenumi{\arabic{enumi}.}
\tightlist
\item
  При умножении матриц ранг не может увеличиться. Каждый столбец матрицы
  \(AB\) является линейной комбинацией столбцов матрицы \(A\), а каждая
  строка матрицы \(AB\) является линейной комбинацией строк матрицы
  \(B\). Поэтому пространство столбцов матрицы \(AB\) содержится в
  пространстве столбцов матрицы \(A\), а пространство строк матрицы
  \(AB\) содержится в пространстве строк матрицы \(B\).
\item
  Базис пространства столбцов матрицы \(A+B\) (\(\mathbf{C}(A+B)\))
  является комбинацией (возможно, с пересечениями) базисов пространств
  \(\mathbf{C}(A)\) и \(\mathbf{C}(B)\).
\item
  Матрицы \(A\) и \(A^\top A\) имеют одно и то же нуль-пространство
  (доказать), поэтому их ранг одинаков.
\item
  Матрицы \(A^\top A\) и \(BB^\top\) невырождены, так как
  \(r(A^\top A) = r(BB^\top) = r\). Их произведение, матрица
  \(A^\top A BB^\top\), тоже невырождена и её ранг равен \(r\). Отсюда
  \(r = r(A^\top A BB^\top) \le r(AB) \le r(A) = r\).
\end{enumerate}

    \begin{quote}
Как доказать, что \(A\) и \(A^\top A\) имеют одно и то же
нуль-пространство?\\
Если \(Ax=0\), то \(A^\top Ax = 0\). Поэтому
\(\mathbf{N}(A) \subset \mathbf{N}(A^\top A)\).\\
Если \(A^\top Ax = 0\), то \(x^\top A^\top Ax = \|Ax\|^2 = 0\). Поэтому
\(\mathbf{N}(A^\top A) \subset \mathbf{N}(A)\).
\end{quote}

    \emph{Замечание.} Свойство 4 работает только в случае, когда \(A\) имеет
\emph{ровно} \(r\) столбцов, а \(B\) имеет \emph{ровно} \(r\) строк. В
частности, \(r(BA) \le r\) (в соответствии со свойством 1).

Для свойства 3 отметим важный частный случай. Это случай, когда столбцы
матрицы \(A\) линейно независимы, так что её ранг \(r\) равен \(n\).
Тогда матрица \(A^\top A\) является квадратной симметрической обратимой
матрицей.

    \begin{center}\rule{0.5\linewidth}{\linethickness}\end{center}

    \hypertarget{ux441ux43aux435ux43bux435ux442ux43dux43eux435-ux440ux430ux437ux43bux43eux436ux435ux43dux438ux435}{%
\section{Скелетное
разложение}\label{ux441ux43aux435ux43bux435ux442ux43dux43eux435-ux440ux430ux437ux43bux43eux436ux435ux43dux438ux435}}

В матрице \(A\) первые два вектора линейно независимы. Попробуем взять
их в качестве базиса и разложить по ним третий. Запишем это в матричном
виде, пользуясь правилом умножения «столбец на столбец».

\[
  A = 
  \begin{pmatrix}
     1 & 2 & 3 \\
     2 & 1 & 3 \\
     3 & 1 & 4 \\
  \end{pmatrix}
  =
  \begin{pmatrix}
     1 & 2 \\
     2 & 1 \\
     3 & 1 \\
  \end{pmatrix}
  \cdot
  \begin{pmatrix}
     1 & 0 & 1 \\
     0 & 1 & 1 \\
  \end{pmatrix}
  = C \cdot R
\]

Здесь \(C\) --- базисные столбцы, \(R\) --- базисные строки. Мы получили
скелетное разложение матрицы.

\textbf{Определение 2.} \emph{Скелетным разложением} матрицы \(A\)
размеров \(m \times n\) и ранга \(r>0\) называется разложение вида
\(A = CR\), где матрицы \(C\) и \(R\) имеют размеры соответственно
\(m \times r\) и \(r \times n\). Другое название скелетного разложения
--- \emph{ранговая факторизация}.

Это разложение иллюстрирует \textbf{теорему}: ранг матрицы по столбцам
(количество независимых столбцов) равен рангу матрицы по строкам
(количество независимых строк).

    \textbf{Дополнительно}

Существует другой вариант скелетного разложения: \(A = CMR\). В этом
случае матрица \(С\), как и ранее, состоит из \(r\) независимых столбцов
матрицы \(A\), но матрица \(R\) теперь состоит из \(r\) независимых
строк матрицы \(A\), а матрица \(M\) размером \(r \times r\) называется
смешанной матрицей (mixing matrix). Для \(M\) можно получить следующую
формулу: \[
  A = CMR \\
  C^\top A R^\top = C^\top C M R R^\top \\
  M = \left[ (C^\top C)^{-1}C^\top \right] A \left[ R^\top (R R^\top)^{-1} \right].
\]

    \begin{quote}
Матрицы \(C^+ = (C^\top C)^{-1}C^\top\) и
\(R^+ = R^\top (R R^\top)^{-1}\) являются \emph{псевдообратными} к
матрицам соответственно \(C\) и \(R\).\\
Можно показать, что если \emph{столбцы} матрицы линейно независимы (как
у матрицы \(C\)), то \(C^+\) является \emph{левой} обратной матрицей для
\(C\): \(C^+ C = I\). Если независимы строки (как у \(R\)), то \(R^+\)
--- \emph{правая} обратная матрица для \(R\): \(R R^+ = I\).\\
Более подробный материал о псевдообратных матрицах будет на несколько
занятий позже.
\end{quote}

    \begin{center}\rule{0.5\linewidth}{\linethickness}\end{center}

    \hypertarget{ux438ux441ux442ux43eux447ux43dux438ux43aux438}{%
\section{Источники}\label{ux438ux441ux442ux43eux447ux43dux438ux43aux438}}

\begin{enumerate}
\def\labelenumi{\arabic{enumi}.}
\tightlist
\item
  \emph{Strang G.} Linear algebra and learning from data. ---
  Wellesley-Cambridge Press, 2019. --- 432\,p.
\item
  \emph{Гантмахер Ф.Р.} Теория матриц. --- М.: Наука, 1967. --- 576\,с.
\item
  \emph{Беклемишев Д.В.} Дополнительные главы линейной алгебры. --- М.:
  Наука, 1983. --- 336\,с.
\end{enumerate}


    % Add a bibliography block to the postdoc
    
    
    
\end{document}
