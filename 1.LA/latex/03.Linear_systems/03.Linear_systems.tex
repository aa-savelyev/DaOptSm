\documentclass[11pt,a4paper]{article}

    \usepackage[breakable]{tcolorbox}
    \usepackage{parskip} % Stop auto-indenting (to mimic markdown behaviour)

    \usepackage{iftex}
    \ifPDFTeX
      \usepackage[T2A]{fontenc}
      \usepackage{mathpazo}
      \usepackage[russian,english]{babel}
    \else
      \usepackage{fontspec}
      \usepackage{polyglossia}
      \setmainlanguage[babelshorthands=true]{russian}    % Язык по-умолчанию русский с поддержкой приятных команд пакета babel
      \setotherlanguage{english}                         % Дополнительный язык = английский (в американской вариации по-умолчанию)

      \defaultfontfeatures{Ligatures=TeX}
      \setmainfont[BoldFont={STIX Two Text SemiBold}]%
      {STIX Two Text}                                    % Шрифт с засечками
      \newfontfamily\cyrillicfont[BoldFont={STIX Two Text SemiBold}]%
      {STIX Two Text}                                    % Шрифт с засечками для кириллицы
      \setsansfont{Fira Sans}                            % Шрифт без засечек
      \newfontfamily\cyrillicfontsf{Fira Sans}           % Шрифт без засечек для кириллицы
      \setmonofont[Scale=0.87,BoldFont={Fira Mono Medium},ItalicFont=[FiraMono-Oblique]]%
      {Fira Mono}%                                       % Моноширинный шрифт
      \newfontfamily\cyrillicfonttt[Scale=0.87,BoldFont={Fira Mono Medium},ItalicFont=[FiraMono-Oblique]]%
      {Fira Mono}                                        % Моноширинный шрифт для кириллицы

      %%% Математические пакеты %%%
      \usepackage{amsthm,amsmath,amscd}   % Математические дополнения от AMS
      \usepackage{amsfonts,amssymb}       % Математические дополнения от AMS
      \usepackage{mathtools}              % Добавляет окружение multlined
      \usepackage{unicode-math}           % Для шрифта STIX Two Math
      \setmathfont{STIX Two Math}         % Математический шрифт
    \fi

    % Basic figure setup, for now with no caption control since it's done
    % automatically by Pandoc (which extracts ![](path) syntax from Markdown).
    \usepackage{graphicx}
    % Maintain compatibility with old templates. Remove in nbconvert 6.0
    \let\Oldincludegraphics\includegraphics
    % Ensure that by default, figures have no caption (until we provide a
    % proper Figure object with a Caption API and a way to capture that
    % in the conversion process - todo).
    \usepackage{caption}
    \DeclareCaptionFormat{nocaption}{}
    \captionsetup{format=nocaption,aboveskip=0pt,belowskip=0pt}

    \usepackage{float}
    \floatplacement{figure}{H} % forces figures to be placed at the correct location
    \usepackage{xcolor} % Allow colors to be defined
    \usepackage{enumerate} % Needed for markdown enumerations to work
    \usepackage{geometry} % Used to adjust the document margins
    \usepackage{amsmath} % Equations
    \usepackage{amssymb} % Equations
    \usepackage{textcomp} % defines textquotesingle
    % Hack from http://tex.stackexchange.com/a/47451/13684:
    \AtBeginDocument{%
        \def\PYZsq{\textquotesingle}% Upright quotes in Pygmentized code
    }
    \usepackage{upquote} % Upright quotes for verbatim code
    \usepackage{eurosym} % defines \euro
    \usepackage[mathletters]{ucs} % Extended unicode (utf-8) support
    \usepackage{fancyvrb} % verbatim replacement that allows latex
    \usepackage{grffile} % extends the file name processing of package graphics
                         % to support a larger range
    \makeatletter % fix for old versions of grffile with XeLaTeX
    \@ifpackagelater{grffile}{2019/11/01}
    {
      % Do nothing on new versions
    }
    {
      \def\Gread@@xetex#1{%
        \IfFileExists{"\Gin@base".bb}%
        {\Gread@eps{\Gin@base.bb}}%
        {\Gread@@xetex@aux#1}%
      }
    }
    \makeatother
    \usepackage[Export]{adjustbox} % Used to constrain images to a maximum size
    \adjustboxset{max size={0.9\linewidth}{0.9\paperheight}}

    % The hyperref package gives us a pdf with properly built
    % internal navigation ('pdf bookmarks' for the table of contents,
    % internal cross-reference links, web links for URLs, etc.)
    \usepackage{hyperref}
    % The default LaTeX title has an obnoxious amount of whitespace. By default,
    % titling removes some of it. It also provides customization options.
    \usepackage{titling}
    \usepackage{longtable} % longtable support required by pandoc >1.10
    \usepackage{booktabs}  % table support for pandoc > 1.12.2
    \usepackage[inline]{enumitem} % IRkernel/repr support (it uses the enumerate* environment)
    \usepackage[normalem]{ulem} % ulem is needed to support strikethroughs (\sout)
                                % normalem makes italics be italics, not underlines
    \usepackage{mathrsfs}



    % Colors for the hyperref package
    \definecolor{urlcolor}{rgb}{0,.145,.698}
    \definecolor{linkcolor}{rgb}{.71,0.21,0.01}
    \definecolor{citecolor}{rgb}{.12,.54,.11}

    % ANSI colors
    \definecolor{ansi-black}{HTML}{3E424D}
    \definecolor{ansi-black-intense}{HTML}{282C36}
    \definecolor{ansi-red}{HTML}{E75C58}
    \definecolor{ansi-red-intense}{HTML}{B22B31}
    \definecolor{ansi-green}{HTML}{00A250}
    \definecolor{ansi-green-intense}{HTML}{007427}
    \definecolor{ansi-yellow}{HTML}{DDB62B}
    \definecolor{ansi-yellow-intense}{HTML}{B27D12}
    \definecolor{ansi-blue}{HTML}{208FFB}
    \definecolor{ansi-blue-intense}{HTML}{0065CA}
    \definecolor{ansi-magenta}{HTML}{D160C4}
    \definecolor{ansi-magenta-intense}{HTML}{A03196}
    \definecolor{ansi-cyan}{HTML}{60C6C8}
    \definecolor{ansi-cyan-intense}{HTML}{258F8F}
    \definecolor{ansi-white}{HTML}{C5C1B4}
    \definecolor{ansi-white-intense}{HTML}{A1A6B2}
    \definecolor{ansi-default-inverse-fg}{HTML}{FFFFFF}
    \definecolor{ansi-default-inverse-bg}{HTML}{000000}

    % common color for the border for error outputs.
    \definecolor{outerrorbackground}{HTML}{FFDFDF}

    % commands and environments needed by pandoc snippets
    % extracted from the output of `pandoc -s`
    \providecommand{\tightlist}{%
      \setlength{\itemsep}{0pt}\setlength{\parskip}{0pt}}
    \DefineVerbatimEnvironment{Highlighting}{Verbatim}{commandchars=\\\{\}}
    % Add ',fontsize=\small' for more characters per line
    \newenvironment{Shaded}{}{}
    \newcommand{\KeywordTok}[1]{\textcolor[rgb]{0.00,0.44,0.13}{\textbf{{#1}}}}
    \newcommand{\DataTypeTok}[1]{\textcolor[rgb]{0.56,0.13,0.00}{{#1}}}
    \newcommand{\DecValTok}[1]{\textcolor[rgb]{0.25,0.63,0.44}{{#1}}}
    \newcommand{\BaseNTok}[1]{\textcolor[rgb]{0.25,0.63,0.44}{{#1}}}
    \newcommand{\FloatTok}[1]{\textcolor[rgb]{0.25,0.63,0.44}{{#1}}}
    \newcommand{\CharTok}[1]{\textcolor[rgb]{0.25,0.44,0.63}{{#1}}}
    \newcommand{\StringTok}[1]{\textcolor[rgb]{0.25,0.44,0.63}{{#1}}}
    \newcommand{\CommentTok}[1]{\textcolor[rgb]{0.38,0.63,0.69}{\textit{{#1}}}}
    \newcommand{\OtherTok}[1]{\textcolor[rgb]{0.00,0.44,0.13}{{#1}}}
    \newcommand{\AlertTok}[1]{\textcolor[rgb]{1.00,0.00,0.00}{\textbf{{#1}}}}
    \newcommand{\FunctionTok}[1]{\textcolor[rgb]{0.02,0.16,0.49}{{#1}}}
    \newcommand{\RegionMarkerTok}[1]{{#1}}
    \newcommand{\ErrorTok}[1]{\textcolor[rgb]{1.00,0.00,0.00}{\textbf{{#1}}}}
    \newcommand{\NormalTok}[1]{{#1}}

    % Additional commands for more recent versions of Pandoc
    \newcommand{\ConstantTok}[1]{\textcolor[rgb]{0.53,0.00,0.00}{{#1}}}
    \newcommand{\SpecialCharTok}[1]{\textcolor[rgb]{0.25,0.44,0.63}{{#1}}}
    \newcommand{\VerbatimStringTok}[1]{\textcolor[rgb]{0.25,0.44,0.63}{{#1}}}
    \newcommand{\SpecialStringTok}[1]{\textcolor[rgb]{0.73,0.40,0.53}{{#1}}}
    \newcommand{\ImportTok}[1]{{#1}}
    \newcommand{\DocumentationTok}[1]{\textcolor[rgb]{0.73,0.13,0.13}{\textit{{#1}}}}
    \newcommand{\AnnotationTok}[1]{\textcolor[rgb]{0.38,0.63,0.69}{\textbf{\textit{{#1}}}}}
    \newcommand{\CommentVarTok}[1]{\textcolor[rgb]{0.38,0.63,0.69}{\textbf{\textit{{#1}}}}}
    \newcommand{\VariableTok}[1]{\textcolor[rgb]{0.10,0.09,0.49}{{#1}}}
    \newcommand{\ControlFlowTok}[1]{\textcolor[rgb]{0.00,0.44,0.13}{\textbf{{#1}}}}
    \newcommand{\OperatorTok}[1]{\textcolor[rgb]{0.40,0.40,0.40}{{#1}}}
    \newcommand{\BuiltInTok}[1]{{#1}}
    \newcommand{\ExtensionTok}[1]{{#1}}
    \newcommand{\PreprocessorTok}[1]{\textcolor[rgb]{0.74,0.48,0.00}{{#1}}}
    \newcommand{\AttributeTok}[1]{\textcolor[rgb]{0.49,0.56,0.16}{{#1}}}
    \newcommand{\InformationTok}[1]{\textcolor[rgb]{0.38,0.63,0.69}{\textbf{\textit{{#1}}}}}
    \newcommand{\WarningTok}[1]{\textcolor[rgb]{0.38,0.63,0.69}{\textbf{\textit{{#1}}}}}


    % Define a nice break command that doesn't care if a line doesn't already
    % exist.
    \def\br{\hspace*{\fill} \\* }
    % Math Jax compatibility definitions
    \def\gt{>}
    \def\lt{<}
    \let\Oldtex\TeX
    \let\Oldlatex\LaTeX
    \renewcommand{\TeX}{\textrm{\Oldtex}}
    \renewcommand{\LaTeX}{\textrm{\Oldlatex}}
    % Document parameters
    % Document title
    \title{
      {\Large Лекция 3} \\
      Системы линейных уравнений. \\
      Псевдообратные матрицы
    }
    % \date{21 сентября 2022\,г.}
    \date{}



% Pygments definitions
\makeatletter
\def\PY@reset{\let\PY@it=\relax \let\PY@bf=\relax%
    \let\PY@ul=\relax \let\PY@tc=\relax%
    \let\PY@bc=\relax \let\PY@ff=\relax}
\def\PY@tok#1{\csname PY@tok@#1\endcsname}
\def\PY@toks#1+{\ifx\relax#1\empty\else%
    \PY@tok{#1}\expandafter\PY@toks\fi}
\def\PY@do#1{\PY@bc{\PY@tc{\PY@ul{%
    \PY@it{\PY@bf{\PY@ff{#1}}}}}}}
\def\PY#1#2{\PY@reset\PY@toks#1+\relax+\PY@do{#2}}

\@namedef{PY@tok@w}{\def\PY@tc##1{\textcolor[rgb]{0.73,0.73,0.73}{##1}}}
\@namedef{PY@tok@c}{\let\PY@it=\textit\def\PY@tc##1{\textcolor[rgb]{0.24,0.48,0.48}{##1}}}
\@namedef{PY@tok@cp}{\def\PY@tc##1{\textcolor[rgb]{0.61,0.40,0.00}{##1}}}
\@namedef{PY@tok@k}{\let\PY@bf=\textbf\def\PY@tc##1{\textcolor[rgb]{0.00,0.50,0.00}{##1}}}
\@namedef{PY@tok@kp}{\def\PY@tc##1{\textcolor[rgb]{0.00,0.50,0.00}{##1}}}
\@namedef{PY@tok@kt}{\def\PY@tc##1{\textcolor[rgb]{0.69,0.00,0.25}{##1}}}
\@namedef{PY@tok@o}{\def\PY@tc##1{\textcolor[rgb]{0.40,0.40,0.40}{##1}}}
\@namedef{PY@tok@ow}{\let\PY@bf=\textbf\def\PY@tc##1{\textcolor[rgb]{0.67,0.13,1.00}{##1}}}
\@namedef{PY@tok@nb}{\def\PY@tc##1{\textcolor[rgb]{0.00,0.50,0.00}{##1}}}
\@namedef{PY@tok@nf}{\def\PY@tc##1{\textcolor[rgb]{0.00,0.00,1.00}{##1}}}
\@namedef{PY@tok@nc}{\let\PY@bf=\textbf\def\PY@tc##1{\textcolor[rgb]{0.00,0.00,1.00}{##1}}}
\@namedef{PY@tok@nn}{\let\PY@bf=\textbf\def\PY@tc##1{\textcolor[rgb]{0.00,0.00,1.00}{##1}}}
\@namedef{PY@tok@ne}{\let\PY@bf=\textbf\def\PY@tc##1{\textcolor[rgb]{0.80,0.25,0.22}{##1}}}
\@namedef{PY@tok@nv}{\def\PY@tc##1{\textcolor[rgb]{0.10,0.09,0.49}{##1}}}
\@namedef{PY@tok@no}{\def\PY@tc##1{\textcolor[rgb]{0.53,0.00,0.00}{##1}}}
\@namedef{PY@tok@nl}{\def\PY@tc##1{\textcolor[rgb]{0.46,0.46,0.00}{##1}}}
\@namedef{PY@tok@ni}{\let\PY@bf=\textbf\def\PY@tc##1{\textcolor[rgb]{0.44,0.44,0.44}{##1}}}
\@namedef{PY@tok@na}{\def\PY@tc##1{\textcolor[rgb]{0.41,0.47,0.13}{##1}}}
\@namedef{PY@tok@nt}{\let\PY@bf=\textbf\def\PY@tc##1{\textcolor[rgb]{0.00,0.50,0.00}{##1}}}
\@namedef{PY@tok@nd}{\def\PY@tc##1{\textcolor[rgb]{0.67,0.13,1.00}{##1}}}
\@namedef{PY@tok@s}{\def\PY@tc##1{\textcolor[rgb]{0.73,0.13,0.13}{##1}}}
\@namedef{PY@tok@sd}{\let\PY@it=\textit\def\PY@tc##1{\textcolor[rgb]{0.73,0.13,0.13}{##1}}}
\@namedef{PY@tok@si}{\let\PY@bf=\textbf\def\PY@tc##1{\textcolor[rgb]{0.64,0.35,0.47}{##1}}}
\@namedef{PY@tok@se}{\let\PY@bf=\textbf\def\PY@tc##1{\textcolor[rgb]{0.67,0.36,0.12}{##1}}}
\@namedef{PY@tok@sr}{\def\PY@tc##1{\textcolor[rgb]{0.64,0.35,0.47}{##1}}}
\@namedef{PY@tok@ss}{\def\PY@tc##1{\textcolor[rgb]{0.10,0.09,0.49}{##1}}}
\@namedef{PY@tok@sx}{\def\PY@tc##1{\textcolor[rgb]{0.00,0.50,0.00}{##1}}}
\@namedef{PY@tok@m}{\def\PY@tc##1{\textcolor[rgb]{0.40,0.40,0.40}{##1}}}
\@namedef{PY@tok@gh}{\let\PY@bf=\textbf\def\PY@tc##1{\textcolor[rgb]{0.00,0.00,0.50}{##1}}}
\@namedef{PY@tok@gu}{\let\PY@bf=\textbf\def\PY@tc##1{\textcolor[rgb]{0.50,0.00,0.50}{##1}}}
\@namedef{PY@tok@gd}{\def\PY@tc##1{\textcolor[rgb]{0.63,0.00,0.00}{##1}}}
\@namedef{PY@tok@gi}{\def\PY@tc##1{\textcolor[rgb]{0.00,0.52,0.00}{##1}}}
\@namedef{PY@tok@gr}{\def\PY@tc##1{\textcolor[rgb]{0.89,0.00,0.00}{##1}}}
\@namedef{PY@tok@ge}{\let\PY@it=\textit}
\@namedef{PY@tok@gs}{\let\PY@bf=\textbf}
\@namedef{PY@tok@gp}{\let\PY@bf=\textbf\def\PY@tc##1{\textcolor[rgb]{0.00,0.00,0.50}{##1}}}
\@namedef{PY@tok@go}{\def\PY@tc##1{\textcolor[rgb]{0.44,0.44,0.44}{##1}}}
\@namedef{PY@tok@gt}{\def\PY@tc##1{\textcolor[rgb]{0.00,0.27,0.87}{##1}}}
\@namedef{PY@tok@err}{\def\PY@bc##1{{\setlength{\fboxsep}{\string -\fboxrule}\fcolorbox[rgb]{1.00,0.00,0.00}{1,1,1}{\strut ##1}}}}
\@namedef{PY@tok@kc}{\let\PY@bf=\textbf\def\PY@tc##1{\textcolor[rgb]{0.00,0.50,0.00}{##1}}}
\@namedef{PY@tok@kd}{\let\PY@bf=\textbf\def\PY@tc##1{\textcolor[rgb]{0.00,0.50,0.00}{##1}}}
\@namedef{PY@tok@kn}{\let\PY@bf=\textbf\def\PY@tc##1{\textcolor[rgb]{0.00,0.50,0.00}{##1}}}
\@namedef{PY@tok@kr}{\let\PY@bf=\textbf\def\PY@tc##1{\textcolor[rgb]{0.00,0.50,0.00}{##1}}}
\@namedef{PY@tok@bp}{\def\PY@tc##1{\textcolor[rgb]{0.00,0.50,0.00}{##1}}}
\@namedef{PY@tok@fm}{\def\PY@tc##1{\textcolor[rgb]{0.00,0.00,1.00}{##1}}}
\@namedef{PY@tok@vc}{\def\PY@tc##1{\textcolor[rgb]{0.10,0.09,0.49}{##1}}}
\@namedef{PY@tok@vg}{\def\PY@tc##1{\textcolor[rgb]{0.10,0.09,0.49}{##1}}}
\@namedef{PY@tok@vi}{\def\PY@tc##1{\textcolor[rgb]{0.10,0.09,0.49}{##1}}}
\@namedef{PY@tok@vm}{\def\PY@tc##1{\textcolor[rgb]{0.10,0.09,0.49}{##1}}}
\@namedef{PY@tok@sa}{\def\PY@tc##1{\textcolor[rgb]{0.73,0.13,0.13}{##1}}}
\@namedef{PY@tok@sb}{\def\PY@tc##1{\textcolor[rgb]{0.73,0.13,0.13}{##1}}}
\@namedef{PY@tok@sc}{\def\PY@tc##1{\textcolor[rgb]{0.73,0.13,0.13}{##1}}}
\@namedef{PY@tok@dl}{\def\PY@tc##1{\textcolor[rgb]{0.73,0.13,0.13}{##1}}}
\@namedef{PY@tok@s2}{\def\PY@tc##1{\textcolor[rgb]{0.73,0.13,0.13}{##1}}}
\@namedef{PY@tok@sh}{\def\PY@tc##1{\textcolor[rgb]{0.73,0.13,0.13}{##1}}}
\@namedef{PY@tok@s1}{\def\PY@tc##1{\textcolor[rgb]{0.73,0.13,0.13}{##1}}}
\@namedef{PY@tok@mb}{\def\PY@tc##1{\textcolor[rgb]{0.40,0.40,0.40}{##1}}}
\@namedef{PY@tok@mf}{\def\PY@tc##1{\textcolor[rgb]{0.40,0.40,0.40}{##1}}}
\@namedef{PY@tok@mh}{\def\PY@tc##1{\textcolor[rgb]{0.40,0.40,0.40}{##1}}}
\@namedef{PY@tok@mi}{\def\PY@tc##1{\textcolor[rgb]{0.40,0.40,0.40}{##1}}}
\@namedef{PY@tok@il}{\def\PY@tc##1{\textcolor[rgb]{0.40,0.40,0.40}{##1}}}
\@namedef{PY@tok@mo}{\def\PY@tc##1{\textcolor[rgb]{0.40,0.40,0.40}{##1}}}
\@namedef{PY@tok@ch}{\let\PY@it=\textit\def\PY@tc##1{\textcolor[rgb]{0.24,0.48,0.48}{##1}}}
\@namedef{PY@tok@cm}{\let\PY@it=\textit\def\PY@tc##1{\textcolor[rgb]{0.24,0.48,0.48}{##1}}}
\@namedef{PY@tok@cpf}{\let\PY@it=\textit\def\PY@tc##1{\textcolor[rgb]{0.24,0.48,0.48}{##1}}}
\@namedef{PY@tok@c1}{\let\PY@it=\textit\def\PY@tc##1{\textcolor[rgb]{0.24,0.48,0.48}{##1}}}
\@namedef{PY@tok@cs}{\let\PY@it=\textit\def\PY@tc##1{\textcolor[rgb]{0.24,0.48,0.48}{##1}}}

\def\PYZbs{\char`\\}
\def\PYZus{\char`\_}
\def\PYZob{\char`\{}
\def\PYZcb{\char`\}}
\def\PYZca{\char`\^}
\def\PYZam{\char`\&}
\def\PYZlt{\char`\<}
\def\PYZgt{\char`\>}
\def\PYZsh{\char`\#}
\def\PYZpc{\char`\%}
\def\PYZdl{\char`\$}
\def\PYZhy{\char`\-}
\def\PYZsq{\char`\'}
\def\PYZdq{\char`\"}
\def\PYZti{\char`\~}
% for compatibility with earlier versions
\def\PYZat{@}
\def\PYZlb{[}
\def\PYZrb{]}
\makeatother


    % For linebreaks inside Verbatim environment from package fancyvrb.
    \makeatletter
        \newbox\Wrappedcontinuationbox
        \newbox\Wrappedvisiblespacebox
        \newcommand*\Wrappedvisiblespace {\textcolor{red}{\textvisiblespace}}
        \newcommand*\Wrappedcontinuationsymbol {\textcolor{red}{\llap{\tiny$\m@th\hookrightarrow$}}}
        \newcommand*\Wrappedcontinuationindent {3ex }
        \newcommand*\Wrappedafterbreak {\kern\Wrappedcontinuationindent\copy\Wrappedcontinuationbox}
        % Take advantage of the already applied Pygments mark-up to insert
        % potential linebreaks for TeX processing.
        %        {, <, #, %, $, ' and ": go to next line.
        %        _, }, ^, &, >, - and ~: stay at end of broken line.
        % Use of \textquotesingle for straight quote.
        \newcommand*\Wrappedbreaksatspecials {%
            \def\PYGZus{\discretionary{\char`\_}{\Wrappedafterbreak}{\char`\_}}%
            \def\PYGZob{\discretionary{}{\Wrappedafterbreak\char`\{}{\char`\{}}%
            \def\PYGZcb{\discretionary{\char`\}}{\Wrappedafterbreak}{\char`\}}}%
            \def\PYGZca{\discretionary{\char`\^}{\Wrappedafterbreak}{\char`\^}}%
            \def\PYGZam{\discretionary{\char`\&}{\Wrappedafterbreak}{\char`\&}}%
            \def\PYGZlt{\discretionary{}{\Wrappedafterbreak\char`\<}{\char`\<}}%
            \def\PYGZgt{\discretionary{\char`\>}{\Wrappedafterbreak}{\char`\>}}%
            \def\PYGZsh{\discretionary{}{\Wrappedafterbreak\char`\#}{\char`\#}}%
            \def\PYGZpc{\discretionary{}{\Wrappedafterbreak\char`\%}{\char`\%}}%
            \def\PYGZdl{\discretionary{}{\Wrappedafterbreak\char`\$}{\char`\$}}%
            \def\PYGZhy{\discretionary{\char`\-}{\Wrappedafterbreak}{\char`\-}}%
            \def\PYGZsq{\discretionary{}{\Wrappedafterbreak\textquotesingle}{\textquotesingle}}%
            \def\PYGZdq{\discretionary{}{\Wrappedafterbreak\char`\"}{\char`\"}}%
            \def\PYGZti{\discretionary{\char`\~}{\Wrappedafterbreak}{\char`\~}}%
        }
        % Some characters . , ; ? ! / are not pygmentized.
        % This macro makes them "active" and they will insert potential linebreaks
        \newcommand*\Wrappedbreaksatpunct {%
            \lccode`\~`\.\lowercase{\def~}{\discretionary{\hbox{\char`\.}}{\Wrappedafterbreak}{\hbox{\char`\.}}}%
            \lccode`\~`\,\lowercase{\def~}{\discretionary{\hbox{\char`\,}}{\Wrappedafterbreak}{\hbox{\char`\,}}}%
            \lccode`\~`\;\lowercase{\def~}{\discretionary{\hbox{\char`\;}}{\Wrappedafterbreak}{\hbox{\char`\;}}}%
            \lccode`\~`\:\lowercase{\def~}{\discretionary{\hbox{\char`\:}}{\Wrappedafterbreak}{\hbox{\char`\:}}}%
            \lccode`\~`\?\lowercase{\def~}{\discretionary{\hbox{\char`\?}}{\Wrappedafterbreak}{\hbox{\char`\?}}}%
            \lccode`\~`\!\lowercase{\def~}{\discretionary{\hbox{\char`\!}}{\Wrappedafterbreak}{\hbox{\char`\!}}}%
            \lccode`\~`\/\lowercase{\def~}{\discretionary{\hbox{\char`\/}}{\Wrappedafterbreak}{\hbox{\char`\/}}}%
            \catcode`\.\active
            \catcode`\,\active
            \catcode`\;\active
            \catcode`\:\active
            \catcode`\?\active
            \catcode`\!\active
            \catcode`\/\active
            \lccode`\~`\~
        }
    \makeatother

    \let\OriginalVerbatim=\Verbatim
    \makeatletter
    \renewcommand{\Verbatim}[1][1]{%
        %\parskip\z@skip
        \sbox\Wrappedcontinuationbox {\Wrappedcontinuationsymbol}%
        \sbox\Wrappedvisiblespacebox {\FV@SetupFont\Wrappedvisiblespace}%
        \def\FancyVerbFormatLine ##1{\hsize\linewidth
            \vtop{\raggedright\hyphenpenalty\z@\exhyphenpenalty\z@
                \doublehyphendemerits\z@\finalhyphendemerits\z@
                \strut ##1\strut}%
        }%
        % If the linebreak is at a space, the latter will be displayed as visible
        % space at end of first line, and a continuation symbol starts next line.
        % Stretch/shrink are however usually zero for typewriter font.
        \def\FV@Space {%
            \nobreak\hskip\z@ plus\fontdimen3\font minus\fontdimen4\font
            \discretionary{\copy\Wrappedvisiblespacebox}{\Wrappedafterbreak}
            {\kern\fontdimen2\font}%
        }%

        % Allow breaks at special characters using \PYG... macros.
        \Wrappedbreaksatspecials
        % Breaks at punctuation characters . , ; ? ! and / need catcode=\active
        \OriginalVerbatim[#1,codes*=\Wrappedbreaksatpunct]%
    }
    \makeatother

    % Exact colors from NB
    \definecolor{incolor}{HTML}{303F9F}
    \definecolor{outcolor}{HTML}{D84315}
    \definecolor{cellborder}{HTML}{CFCFCF}
    \definecolor{cellbackground}{HTML}{F7F7F7}

    % prompt
    \makeatletter
    \newcommand{\boxspacing}{\kern\kvtcb@left@rule\kern\kvtcb@boxsep}
    \makeatother
    \newcommand{\prompt}[4]{
        {\ttfamily\llap{{\color{#2}[#3]:\hspace{3pt}#4}}\vspace{-\baselineskip}}
    }



    % Prevent overflowing lines due to hard-to-break entities
    \sloppy
    % Setup hyperref package
    \hypersetup{
      breaklinks=true,  % so long urls are correctly broken across lines
      colorlinks=true,
      urlcolor=urlcolor,
      linkcolor=linkcolor,
      citecolor=citecolor,
      }
    % Slightly bigger margins than the latex defaults

    \geometry{verbose,tmargin=1in,bmargin=1in,lmargin=1in,rmargin=1in}



\begin{document}

  \maketitle
  \thispagestyle{empty}
  \tableofcontents

%  \let\thefootnote\relax\footnote{
%    \textit{День 21 сентября в истории:
%      \begin{itemize}[topsep=2pt,itemsep=1pt]
%        \item 862 г. --- новгородцы призвали на княжение братьев варягов Рюрика, Синеуса и Трувора.
%        \item 1721 г. --- указом Петра I основан первый военный порт в России --- Новая Голландия.
%        \item 1799 г. --- начался переход русских и австрийских войск под командованием А. В. Суворова через Швейцарские Альпы.
%        \item 1920 г. --- Совнарком Украинской ССР ввёл обязательное изучение украинского языка в школах.
%        \item 1937 г. --- вышел роман Дж.\,Р.\,Р.\,Толкина <<Хоббит, или Туда и обратно>>.
%        \item 1993 г. --- Президент России Б.\,Н.\,Ельцин подписал указ о роспуске Верховного Совета.
%        \item 1993 г. --- Конституционный суд Российской Федерации вынес заключение о неконституционности действий Президента.
%      \end{itemize}
%    }
%  }

  \newpage

\hypertarget{mathbflu-ux440ux430ux437ux43bux43eux436ux435ux43dux438ux435}{%
\section{\texorpdfstring{\(\mathbf{LU}\)-разложение}{LU-разложение}}\label{mathbflu-ux440ux430ux437ux43bux43eux436ux435ux43dux438ux435}}

\begin{quote}
Ограничения: матрица \(A\) --- квадратная и невырожденная.
\end{quote}

    \hypertarget{ux43eux43fux440ux435ux434ux435ux43bux435ux43dux438ux435-ux438-ux43aux440ux438ux442ux435ux440ux438ux439-ux441ux443ux449ux435ux441ux442ux432ux43eux432ux430ux43dux438ux44f}{%
\subsection{Определение и критерий
существования}\label{ux43eux43fux440ux435ux434ux435ux43bux435ux43dux438ux435-ux438-ux43aux440ux438ux442ux435ux440ux438ux439-ux441ux443ux449ux435ux441ux442ux432ux43eux432ux430ux43dux438ux44f}}

\textbf{Определение.} \(LU\)-разложением квадратной матрицы \(A\)
называется разложение матрицы \(A\) в произведение невырожденной нижней
треугольной матрицы \(L\) и верхней треугольной матрицы \(U\) с
единицами на главной диагонали.

\textbf{Теорема (критерий существования).} \(LU\)-разложение матрицы
\(A\) существует тогда и только тогда, когда все главные миноры матрицы
\(A\) отличны от нуля. Если \(LU\)-разложение существует, то оно
единственно.

    \hypertarget{ux43cux435ux442ux43eux434-ux433ux430ux443ux441ux441ux430}{%
\subsection{Метод
Гаусса}\label{ux43cux435ux442ux43eux434-ux433ux430ux443ux441ux441ux430}}

\textbf{Замечание.} \(LU\)-разложение можно рассматривать, как матричную
форму записи метода исключения Гаусса.

Пусть дана система линейных уравнений вида
\[ A \mathbf{x} = \mathbf{b}, \] где \(A\) --- невырожденная квадратная
матрица порядка \(n\).

Метод Гаусса состоит в том, что элементарными преобразованиями
\emph{строк} матрица \(A\) превращается в единичную матрицу. Если
преобразования производить над расширенной матрицей (включающей столбец
свободных членов), то последний столбец превратится в решение системы.

    \hypertarget{ux441ux43bux443ux447ux430ux439-1-ux432ux441ux435-ux433ux43bux430ux432ux43dux44bux435-ux43cux438ux43dux43eux440ux44b-ux43eux442ux43bux438ux447ux43dux44b-ux43eux442-ux43dux443ux43bux44f}{%
\subsection{Случай 1: все главные миноры отличны от
нуля}\label{ux441ux43bux443ux447ux430ux439-1-ux432ux441ux435-ux433ux43bux430ux432ux43dux44bux435-ux43cux438ux43dux43eux440ux44b-ux43eux442ux43bux438ux447ux43dux44b-ux43eux442-ux43dux443ux43bux44f}}

Отличие от нуля главных миноров позволяет не включать в число
производимых элементарных операций перестановки строк.

Метод Гаусса можно разделить на два этапа: прямое исключение и обратная
подстановка. Первым этапом решения СЛАУ методом Гаусса является процесс
превращения матрицы \(A\) элементарными преобразованиями в верхнюю
треугольную матрицу \(U\).

Известно, что выполнение какой-либо элементарной операции со строками
матрицы \(A\) равносильно умножению \(A\) \emph{слева} на некоторую
невырожденную матрицу, а последовательное выполнение ряда таких операций
--- умножению на матрицу \(S\), равную произведению соответствующих
матриц.

На этапе прямого исключения кроме умножения строк на числа употребляется
только прибавление строки к нижележащей строке. Следовательно матрица
\(S\) является нижней треугольной.

\textbf{Утверждение.} Для любой матрицы \(A\) с ненулевыми главными
минорами найдётся такая невырожденная нижняя треугольная матрица \(S\),
что \(SA\) есть верхняя треугольная матрица \(U\) с единицами на главной
диагонали: \[ SA = U. \]

Матрица \(L\), обратная к нижней треугольной матрице \(S\), сама
является нижней треугольной. Тогда получаем: \[ A = LU. \]

    \hypertarget{ux441ux43bux443ux447ux430ux439-2-ux441ux443ux449ux435ux441ux442ux432ux443ux44eux442-ux43dux435ux43dux443ux43bux435ux432ux44bux435-ux433ux43bux430ux432ux43dux44bux435-ux43cux438ux43dux43eux440ux44b-mathbflup-ux440ux430ux437ux43bux43eux436ux435ux43dux438ux435}{%
\subsection{\texorpdfstring{Случай 2: существуют нулевые главные
миноры
(\(\mathbf{LUP}\)-разложение)}{Случай 2: существуют нулевые главные миноры (LUP-разложение)}}\label{ux441ux43bux443ux447ux430ux439-2-ux441ux443ux449ux435ux441ux442ux432ux443ux44eux442-ux43dux435ux43dux443ux43bux435ux432ux44bux435-ux433ux43bux430ux432ux43dux44bux435-ux43cux438ux43dux43eux440ux44b-mathbflup-ux440ux430ux437ux43bux43eux436ux435ux43dux438ux435}}

Что делать, если не все главные миноры отличны от нуля? К используемым
элементарным операциям нужно добавить \emph{перестановки} строк или
столбцов.

\textbf{Утверждение.} Невырожденную матрицу \(A\) перестановкой строк
(или столбцов) можно перевести в матрицу, главные миноры которой отличны
от нуля.

Тогда справедливо \[ PA = LU,\] где \(P\) --- матрица, полученная из
единичной перестановками строк.

    \hypertarget{mathbfldu-ux440ux430ux437ux43bux43eux436ux435ux43dux438ux435}{%
\subsection{\texorpdfstring{\(\mathbf{LDU}\)-разложение}{LDU-разложение}}\label{mathbfldu-ux440ux430ux437ux43bux43eux436ux435ux43dux438ux435}}

\textbf{Замечание.} Единственным является разложение на такие
треугольные множители, что у~второго из них на главной диагонали стоят
единицы. Вообще же существует много треугольных разложений, в частности
такое, в котором единицы находятся на главной диагонали у~первого
сомножителя.

Матрицу \(L\) можно представить как произведение матрицы \(L_1\),
имеющей единицы на главной диагонали, и диагональной матрицы \(D\).
Тогда мы получим

\textbf{Утверждение.} Матрицу \(A\), главные миноры которой не равны
нулю, можно единственным образом разложить в произведение \(L_1 D U\), в
котором \(D\) --- диагональная, а \(L_1\) и \(U\) --- нижняя и верхняя
треугольные матрицы с единицами на главной диагонали.

    \hypertarget{ux440ux430ux437ux43bux43eux436ux435ux43dux438ux435-ux445ux43eux43bux435ux446ux43aux43eux433ux43e}{%
\subsection{Разложение
Холецкого}\label{ux440ux430ux437ux43bux43eux436ux435ux43dux438ux435-ux445ux43eux43bux435ux446ux43aux43eux433ux43e}}

Рассмотрим важный частный случай --- \(LDU\)-разложение симметричной
матрицы \(S = LDU\).\\
Тогда \(S^\top = U^\top D L^\top\), причём \(U^\top\) --- нижняя, а
\(L^\top\) --- верхняя треугольные матрицы.\\
В силу единственности разложения получаем \[ S = U^\top D U. \]

Если же матрица \(S\) является не только симметричной, но и положительно
определённой, то все диагональные элементы матрицы \(D\) положительны и
мы можем ввести матрицу
\(D^{1/2} = \mathrm{diag}\left(\sqrt{d_1}, \dots, \sqrt{d_n}\right)\) и
\(V = D^{1/2}U\).
Тогда мы получаем \emph{разложение Холецкого} \[
  S = V^\top V.
\]
Разложение Холецкого играет важную роль в численных методах, так
как существует эффективный алгоритм, позволяющий получить его для
положительно определённой симметричной матрицы \(S\).

    \begin{center}\rule{0.5\linewidth}{0.5pt}\end{center}

    \hypertarget{ux43eux441ux43dux43eux432ux43dux430ux44f-ux442ux435ux43eux440ux435ux43cux430-ux43bux438ux43dux435ux439ux43dux43eux439-ux430ux43bux433ux435ux431ux440ux44b}{%
\section{Основная теорема линейной
алгебры}\label{ux43eux441ux43dux43eux432ux43dux430ux44f-ux442ux435ux43eux440ux435ux43cux430-ux43bux438ux43dux435ux439ux43dux43eux439-ux430ux43bux433ux435ux431ux440ux44b}}

\hypertarget{ux447ux435ux442ux44bux440ux435-ux43eux441ux43dux43eux432ux43dux44bux445-ux43fux43eux434ux43fux440ux43eux441ux442ux440ux430ux43dux441ux442ux432ux430}{%
\subsection{Четыре основных
подпространства}\label{ux447ux435ux442ux44bux440ux435-ux43eux441ux43dux43eux432ux43dux44bux445-ux43fux43eux434ux43fux440ux43eux441ux442ux440ux430ux43dux441ux442ux432ux430}}

Обычно подпространства описывают одним из двух способов.
Первый способ --- задать множество векторов, порождающих данное подпространство.
Например, при определении пространства строк или пространства столбцов некоторой матрицы, когда указываются порождающие эти пространства строки или столбцы.

Второй способ --- задать перечень ограничений на подпространство.
В этом случае указывают не векторы, порождающие это подпространство, а ограничения, которым должны удовлетворять векторы этого подпространства.
Нуль-пространство, например, состоит из всех векторов, удовлетворяющих системе \(Ax=0\), и каждое уравнение этой системы представляет собой такое ограничение.

Пусть матрица A имеет размер \(m \times n\).\\
Рассмотрим 4 основных подпространства:

\begin{enumerate}
\def\labelenumi{\arabic{enumi}.}
\tightlist
\item
  Пространство строк матрицы \(A\), \(\mathrm{dim} = r\)
\item
  Нуль-пространство матрицы \(A\) (ядро), \(\mathrm{dim} = n-r\)
\item
  Пространство столбцов матрицы \(A\) (образ), \(\mathrm{dim} = r\)
\item
  Нуль-пространство матрицы \(A^\top\), \(\mathrm{dim} = m-r\)
\end{enumerate}

    \textbf{Определение.} Пусть \(V\) -- подпространство пространства
\(\mathbb{R}^n\). Тогда пространство всех \(n\)-мерных векторов,
ортогональных к подпространству \(V\), называется \emph{ортогональным
дополнением} к \(V\) и обозначается символом \(V^\perp\).

\textbf{Основная теорема линейной алгебры.} Пространство строк матрицы \(A\) и нуль-пространство матрицы \(A\), а также пространство столбцов \(A\) и нуль-пространство \(A^\top\) являются ортогональными дополнениями друг к другу.

\begin{center}
  \adjustimage{max size={0.8\linewidth}{0.9\paperheight}}{../../pix/03.Linear_systems/Fundamental_subspaces.png}
\end{center}


    \hypertarget{ux442ux435ux43eux440ux435ux43cux430-ux444ux440ux435ux434ux433ux43eux43bux44cux43cux430}{%
\subsection{Теорема
Фредгольма}\label{ux442ux435ux43eux440ux435ux43cux430-ux444ux440ux435ux434ux433ux43eux43bux44cux43cux430}}

\textbf{Формулировка 1}:\\
Для того чтобы система уравнений \(Ax=b\) была совместна, необходимо и
достаточно, чтобы каждое решение сопряжённой однородной системы
\(A^\top y = 0\) удовлетворяло уравнению \(y^\top b = 0\).

\textbf{Формулировка 2} (\emph{альтернатива Фредгольма}):\\
Для любых \(A\) и \(b\) одна и только одна из следующих задач имеет
решение:

\begin{enumerate}
\def\labelenumi{\arabic{enumi}.}
\tightlist
\item
  \(Ax = b\);
\item
  \(A^\top y = 0\), \(y^\top b \ne 0\).
\end{enumerate}

Иначе говоря, либо вектор \(b\) лежит в пространстве столбцов \(A\),
либо не лежит в нём. В первом случае согласно основной теореме вектор
\(b\) ортогонален любому вектору из \(\mathrm{Ker}(A^\top)\), во-втором
же случае в пространстве \(\mathrm{Ker}(A^\top)\) найдётся вектор \(y\),
неортогональный вектору \(b\): \(y^\top b \ne 0\).

\textbf{Следствие}: Для того чтобы уравнение \(Ax=b\) имело решение при
любой правой части, сопряжённое к нему однородное уравнение
\(A^\top y = 0\) должно иметь только тривиальное решение.

    \begin{center}\rule{0.5\linewidth}{0.5pt}\end{center}

    \hypertarget{ux43fux441ux435ux432ux434ux43eux440ux435ux448ux435ux43dux438ux44f-ux438-ux43fux441ux435ux432ux434ux43eux43eux431ux440ux430ux442ux43dux44bux435-ux43cux430ux442ux440ux438ux446ux44b}{%
\section{Псевдорешения и псевдообратные
матрицы}\label{ux43fux441ux435ux432ux434ux43eux440ux435ux448ux435ux43dux438ux44f-ux438-ux43fux441ux435ux432ux434ux43eux43eux431ux440ux430ux442ux43dux44bux435-ux43cux430ux442ux440ux438ux446ux44b}}

    \hypertarget{ux43fux43eux441ux442ux430ux43dux43eux432ux43aux430-ux437ux430ux434ux430ux447ux438}{%
\subsection{Постановка
задачи}\label{ux43fux43eux441ux442ux430ux43dux43eux432ux43aux430-ux437ux430ux434ux430ux447ux438}}

В практических задачах часто требуется найти решение, удовлетворяющее
большому числу возможно противоречивых требований. Если такая задача
сводится к системе линейных уравнений, то система оказывается, вообще
говоря, несовместной. В этом случае задача может быть решена только
путём выбора некоторого компромисса --- все требования могут быть
удовлетворены не полностью, а лишь до некоторой степени.

Рассмотрим систему линейных уравнений
\[
  A\mathbf{x} = \mathbf{b} \tag{1}\label{eq:system}
\]
с матрицей \(A\) размеров \(m \times n\) и ранга \(r\). Поскольку
\(\mathbf{x}\) --- столбец высоты \(n\), а \(\mathbf{b}\) --- столбец
высоты \(m\), для геометрической иллюстрации естественно будет
использовать пространства \(\mathbb{R}^n\) и \(\mathbb{R}^m\).

Под нормой столбца \(\mathbf{x}\) мы будем понимать его евклидову норму,
т.е. число \[
  \|\mathbf{x}\| = \sqrt{\mathbf{x^\top x}} = \sqrt{x_1^2 + \ldots + x_n^2}.
\]

Невязкой, которую даёт столбец \(\mathbf{x}\) при подстановке в систему
\(\eqref{eq:system}\), называется столбец \[
  \mathbf{u} = \mathbf{b} - A\mathbf{x}.
\] Решение системы --- это столбец, дающий нулевую невязку.

Если система \(\eqref{eq:system}\) несовместна, естественно постараться
найти столбец \(\mathbf{x}\), который даёт невязку с минимальной нормой,
и если такой столбец найдётся, считать его обобщённым решением.

    \hypertarget{ux43fux441ux435ux432ux434ux43eux440ux435ux448ux435ux43dux438ux435}{%
\subsection{Псевдорешение}\label{ux43fux441ux435ux432ux434ux43eux440ux435ux448ux435ux43dux438ux435}}

Для сравнения невязок воспользуемся евклидовой нормой и, следовательно,
будем искать столбец \(\mathbf{x}\), для которого минимальная величина
\[
  \|\mathbf{u}\|^2 = (\mathbf{b} - A\mathbf{x})^\top (\mathbf{b} - A \mathbf{x}).
\]

Найдём полный дифференциал \(\|\mathbf{u}\|^2\): \[
  d\|\mathbf{u}\|^2 = -d\mathbf{x}^\top A^\top (\mathbf{b}-A\mathbf{x}) - (\mathbf{b}-A\mathbf{x})^\top A d\mathbf{x} = \
  -2d\mathbf{x}^\top A^\top (\mathbf{b} - A\mathbf{x}).
\]

Дифференциал равен нулю тогда и только тогда, когда
\[
  A^\top A \mathbf{x} = A^\top \mathbf{b}. \tag{2}\label{eq:norm_system}
\]
Эта система линейных уравнений по отношению к системе
\(\eqref{eq:system}\) называется \emph{нормальной системой}.
Независимо от совместности системы \(\eqref{eq:system}\) справедливо

\textbf{Утверждение 1.} Нормальная система уравнений всегда совместна.\\
\emph{Доказательство.} Применим критерий Фредгольма: система
\(A\mathbf{x}=\mathbf{b}\) совместна тогда и только тогда, когда
\(\mathbf{b}\) ортогонален любому решению \(\mathbf{y}\) сопряжённой
однородной системы. Пусть \(\mathbf{y}\) --- решение сопряжённой
однородной системы \((A^\top A)^\top \mathbf{y} = 0\). Тогда \[
  \mathbf{y}^\top A^\top A \mathbf{y} = (A \mathbf{y})^\top (A \mathbf{y}) = 0 \quad \Rightarrow \quad
  A \mathbf{y} = 0 \quad \Rightarrow \quad
  \mathbf{y}^\top (A^\top \mathbf{b}) = (A\mathbf{y})^\top \mathbf{b} = 0.
\]

\textbf{Утверждение 2.} Точная нижняя грань квадрата нормы невязки
достигается для всех решений нормальной системы и только для них.

\textbf{Утверждение 3.} Нормальная система имеет единственное решение
тогда и только тогда, когда столбцы матрицы \(A\) линейно независимы.

%\textbf{Определение.} \emph{Нормальным псевдорешением} системы линейных
%уравнений называется столбец с минимальной нормой среди всех столбцов,
%дающих минимальную по норме невязку при подстановке в эту систему.

\textbf{Определение.} \emph{Нормальным псевдорешением} системы линейных уравнений \(\eqref{eq:system}\) называется минимальное по норме решение её нормальной системы \(\eqref{eq:norm_system}\).

\textbf{Теорема.} Каждая система линейных уравнений имеет одно и только одно нормальное псевдорешение.

%    \hypertarget{ux43fux440ux438ux43cux435ux440ux44b}{%
%\paragraph{Примеры}\label{ux43fux440ux438ux43cux435ux440ux44b}}

\textbf{Пример 1.} Система из двух уравнений с одной неизвестной:
\[ x=1, \; x=2. \]

Нормальная система уравнений для этой системы есть \[
  \begin{pmatrix}
    1 & 1 \\
  \end{pmatrix}
  \begin{pmatrix}
    1 \\
    1 \\
  \end{pmatrix}
  x =
  \begin{pmatrix}
    1 & 1 \\
  \end{pmatrix}
  \begin{pmatrix}
    1 \\
    2 \\
  \end{pmatrix}.
\] Отсюда получаем псевдорешене \(x = 3/2\).

    \textbf{Пример 2.} Система из одного уравнения с двумя неизвестными:
\[ x + y = 2. \]

Нормальной системой уравнений будет система \[
  \begin{pmatrix}
    1 \\
    1 \\
  \end{pmatrix}
  \begin{pmatrix}
    1 & 1 \\
  \end{pmatrix}
  \begin{pmatrix}
    x \\
    y \\
  \end{pmatrix}
  =
  \begin{pmatrix}
    1 \\
    1 \\
  \end{pmatrix}
  2,
\] содержащая то же уравнение, повторенное дважды. Её общее решение \[
  \begin{pmatrix}
    x \\
    y \\
  \end{pmatrix}
  =
  \begin{pmatrix}
    2 \\
    0 \\
  \end{pmatrix}
  + \alpha
  \begin{pmatrix}
    -1 \\
    1 \\
  \end{pmatrix}.
\]

Действуя согласно определению нормального псевдорешения, среди всех
псевдорешений выберем решение с минимальной нормой.\\
Квадрат нормы решения равен \[
  \|\mathbf{x}\|_2^2 = (2-\alpha)^2 + \alpha^2 = 2\alpha^2 - 4\alpha + 4.
\]

Тогда искомым решением будет столбец \[
  \begin{pmatrix}
    x \\
    y \\
  \end{pmatrix}
  =
  \begin{pmatrix}
    1 \\
    1 \\
  \end{pmatrix}.
\]

    \textbf{Пример 3.} Система из одного уравнения с одним неизвестным:
\(ax = b\).

Если \(a \ne 0\), то псевдорешение совпадает с решением \(x = b/a\).\\
Если \(a = 0\), то любое решение даёт одну и ту же невязку \(b\) с
нормой \(|b|\). Выбирая решение с~минимальной нормой, получаем
\(x = 0\).

    \textbf{Пример 4.} Система линейных уравнений с нулевой матрицей:
\(O \mathbf{x} = \mathbf{b}\).

Аналогично примеру 3 находим, что псевдорешением будет нулевой столбец.

    \hypertarget{ux43fux441ux435ux432ux434ux43eux43eux431ux440ux430ux442ux43dux430ux44f-ux43cux430ux442ux440ux438ux446ux430}{%
\subsection{Псевдообратная
матрица}\label{ux43fux441ux435ux432ux434ux43eux43eux431ux440ux430ux442ux43dux430ux44f-ux43cux430ux442ux440ux438ux446ux430}}

Для невырожденной квадратной матрицы \(A\) порядка \(n\) обратную
матрицу можно определить как такую, столбцы которой являются решениями
систем линейных уравнений вида \[
  A\mathbf{x} = \mathbf{e}_i, \tag{3} \label{eq:inv_definition}
\] где \(\mathbf{e}_i\) --- \(i\)-й столбец единичной матрицы порядка
\(n\).

По аналогии можно дать следующее\\
\textbf{Определение.} \emph{Псевдообратной матрицей} для матрицы \(A\)
размеров \(m \times n\) называется матрица \(A^+\), столбцы которой ---
псевдорешения систем линейных уравнений вида
\(\eqref{eq:inv_definition}\), где \(\mathbf{e}_i\) --- столбцы
единичной матрицы порядка \(m\).

Из теоремы 1 следует, что каждая матрица имеет одну и только одну
псевдообратную. Для невырожденной квадратной матрицы псевдообратная
матрица совпадает с обратной.

\textbf{Утверждение 4.} Если столбцы матрицы \(A\) линейно независимы,
то \[
  A^+ = (A^\top A)^{-1} A^\top.
\] Если строки матрицы \(A\) линейно независимы, то \[
  A^+ = A^\top (A A^\top)^{-1}.
\] В первом случае \(A^+\) является левой обратной матрицей для \(A\)
(\(A^+A=I\)), во втором --- правой (\(A A^+ = I\)).

\textbf{Утверждение 5.} Для любого столбца
\(\mathbf{y} \in \mathbb{R}^m\) столбец \(A A^+ \mathbf{y}\) есть
ортогональная проекция \(\mathbf{y}\) на линейную оболочку столбцов
матрицы \(A\).

\textbf{Утверждение 6.} Если \(A = CR\) --- скелетное разложение матрицы
\(A\), то её псевдообратная равна \[
  A^+ = R^+ C^+ = R^\top (R R^\top)^{-1} (C^\top C)^{-1} C^\top.
\]

\textbf{Утверждение 7.} Если \(A = U \Sigma V^\top\) --- сингулярное
разложение матрицы \(A\), то \(A^+ = V \Sigma^+ U^\top\).\\
\emph{Примечание.} Для диагональной матрицы псевдообратная получается
заменой каждого ненулевого элемента на диагонали на обратный к нему.

    \begin{center}\rule{0.5\linewidth}{0.5pt}\end{center}

    \hypertarget{ux438ux441ux442ux43eux447ux43dux438ux43aux438}{%
\section{Источники}\label{ux438ux441ux442ux43eux447ux43dux438ux43aux438}}

\begin{enumerate}
\def\labelenumi{\arabic{enumi}.}
\tightlist
\item
  \emph{Беклемишев Д.В.} Дополнительные главы линейной алгебры. --- М.:
  Наука, 1983. --- 336\,с.
\item
  \emph{Strang G.} Linear algebra and learning from data. ---
  Wellesley-Cambridge Press, 2019. --- 432\,p.

\end{enumerate}



    % Add a bibliography block to the postdoc



\end{document}
