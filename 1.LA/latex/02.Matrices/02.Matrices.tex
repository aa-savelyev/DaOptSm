\documentclass[11pt,a4paper]{article}

\usepackage[breakable]{tcolorbox}
\usepackage{parskip} % Stop auto-indenting (to mimic markdown behaviour)

\usepackage{iftex}
\ifPDFTeX
  \usepackage[T2A]{fontenc}
  \usepackage{mathpazo}
  \usepackage[russian,english]{babel}
\else
  \usepackage{fontspec}
  \usepackage{polyglossia}
  \setmainlanguage[babelshorthands=true]{russian}    % Язык по-умолчанию русский с поддержкой приятных команд пакета babel
  \setotherlanguage{english}                         % Дополнительный язык = английский (в американской вариации по-умолчанию)

  \defaultfontfeatures{Ligatures=TeX}
  \setmainfont[BoldFont={STIX Two Text SemiBold}]%
  {STIX Two Text}                                    % Шрифт с засечками
  \newfontfamily\cyrillicfont[BoldFont={STIX Two Text SemiBold}]%
  {STIX Two Text}                                    % Шрифт с засечками для кириллицы
  \setsansfont{Fira Sans}                            % Шрифт без засечек
  \newfontfamily\cyrillicfontsf{Fira Sans}           % Шрифт без засечек для кириллицы
  \setmonofont[Scale=0.87,BoldFont={Fira Mono Medium},ItalicFont=[FiraMono-Oblique]]%
  {Fira Mono}%                                       % Моноширинный шрифт
  \newfontfamily\cyrillicfonttt[Scale=0.87,BoldFont={Fira Mono Medium},ItalicFont=[FiraMono-Oblique]]%
  {Fira Mono}                                        % Моноширинный шрифт для кириллицы

  %%% Математические пакеты %%%
  \usepackage{amsthm,amsmath,amscd}   % Математические дополнения от AMS
  \usepackage{amsfonts,amssymb}       % Математические дополнения от AMS
  \usepackage{mathtools}              % Добавляет окружение multlined
  \usepackage{unicode-math}           % Для шрифта STIX Two Math
  \setmathfont{STIX Two Math}         % Математический шрифт
\fi

% Basic figure setup, for now with no caption control since it's done
% automatically by Pandoc (which extracts ![](path) syntax from Markdown).
\usepackage{graphicx}
% Maintain compatibility with old templates. Remove in nbconvert 6.0
\let\Oldincludegraphics\includegraphics
% Ensure that by default, figures have no caption (until we provide a
% proper Figure object with a Caption API and a way to capture that
% in the conversion process - todo).
\usepackage{caption}
\DeclareCaptionFormat{nocaption}{}
\captionsetup{format=nocaption,aboveskip=0pt,belowskip=0pt}

\usepackage{float}
\floatplacement{figure}{H} % forces figures to be placed at the correct location
\usepackage{xcolor} % Allow colors to be defined
\usepackage{enumerate} % Needed for markdown enumerations to work
\usepackage{geometry} % Used to adjust the document margins
\usepackage{amsmath} % Equations
\usepackage{amssymb} % Equations
\usepackage{textcomp} % defines textquotesingle
% Hack from http://tex.stackexchange.com/a/47451/13684:
\AtBeginDocument{%
    \def\PYZsq{\textquotesingle}% Upright quotes in Pygmentized code
}
\usepackage{upquote} % Upright quotes for verbatim code
\usepackage{eurosym} % defines \euro
\usepackage[mathletters]{ucs} % Extended unicode (utf-8) support
\usepackage{fancyvrb} % verbatim replacement that allows latex
\usepackage{grffile} % extends the file name processing of package graphics
                     % to support a larger range
\makeatletter % fix for old versions of grffile with XeLaTeX
\@ifpackagelater{grffile}{2019/11/01}
{
  % Do nothing on new versions
}
{
  \def\Gread@@xetex#1{%
    \IfFileExists{"\Gin@base".bb}%
    {\Gread@eps{\Gin@base.bb}}%
    {\Gread@@xetex@aux#1}%
  }
}
\makeatother
\usepackage[Export]{adjustbox} % Used to constrain images to a maximum size
\adjustboxset{max size={0.9\linewidth}{0.9\paperheight}}

% The hyperref package gives us a pdf with properly built
% internal navigation ('pdf bookmarks' for the table of contents,
% internal cross-reference links, web links for URLs, etc.)
\usepackage{hyperref}
% The default LaTeX title has an obnoxious amount of whitespace. By default,
% titling removes some of it. It also provides customization options.
\usepackage{titling}
\usepackage{longtable} % longtable support required by pandoc >1.10
\usepackage{booktabs}  % table support for pandoc > 1.12.2
\usepackage[inline]{enumitem} % IRkernel/repr support (it uses the enumerate* environment)
\usepackage[normalem]{ulem} % ulem is needed to support strikethroughs (\sout)
                            % normalem makes italics be italics, not underlines
\usepackage{mathrsfs}



% Colors for the hyperref package
\definecolor{urlcolor}{rgb}{0,.145,.698}
\definecolor{linkcolor}{rgb}{.71,0.21,0.01}
\definecolor{citecolor}{rgb}{.12,.54,.11}

% ANSI colors
\definecolor{ansi-black}{HTML}{3E424D}
\definecolor{ansi-black-intense}{HTML}{282C36}
\definecolor{ansi-red}{HTML}{E75C58}
\definecolor{ansi-red-intense}{HTML}{B22B31}
\definecolor{ansi-green}{HTML}{00A250}
\definecolor{ansi-green-intense}{HTML}{007427}
\definecolor{ansi-yellow}{HTML}{DDB62B}
\definecolor{ansi-yellow-intense}{HTML}{B27D12}
\definecolor{ansi-blue}{HTML}{208FFB}
\definecolor{ansi-blue-intense}{HTML}{0065CA}
\definecolor{ansi-magenta}{HTML}{D160C4}
\definecolor{ansi-magenta-intense}{HTML}{A03196}
\definecolor{ansi-cyan}{HTML}{60C6C8}
\definecolor{ansi-cyan-intense}{HTML}{258F8F}
\definecolor{ansi-white}{HTML}{C5C1B4}
\definecolor{ansi-white-intense}{HTML}{A1A6B2}
\definecolor{ansi-default-inverse-fg}{HTML}{FFFFFF}
\definecolor{ansi-default-inverse-bg}{HTML}{000000}

% common color for the border for error outputs.
\definecolor{outerrorbackground}{HTML}{FFDFDF}

% commands and environments needed by pandoc snippets
% extracted from the output of `pandoc -s`
\providecommand{\tightlist}{%
  \setlength{\itemsep}{0pt}\setlength{\parskip}{0pt}}
\DefineVerbatimEnvironment{Highlighting}{Verbatim}{commandchars=\\\{\}}
% Add ',fontsize=\small' for more characters per line
\newenvironment{Shaded}{}{}
\newcommand{\KeywordTok}[1]{\textcolor[rgb]{0.00,0.44,0.13}{\textbf{{#1}}}}
\newcommand{\DataTypeTok}[1]{\textcolor[rgb]{0.56,0.13,0.00}{{#1}}}
\newcommand{\DecValTok}[1]{\textcolor[rgb]{0.25,0.63,0.44}{{#1}}}
\newcommand{\BaseNTok}[1]{\textcolor[rgb]{0.25,0.63,0.44}{{#1}}}
\newcommand{\FloatTok}[1]{\textcolor[rgb]{0.25,0.63,0.44}{{#1}}}
\newcommand{\CharTok}[1]{\textcolor[rgb]{0.25,0.44,0.63}{{#1}}}
\newcommand{\StringTok}[1]{\textcolor[rgb]{0.25,0.44,0.63}{{#1}}}
\newcommand{\CommentTok}[1]{\textcolor[rgb]{0.38,0.63,0.69}{\textit{{#1}}}}
\newcommand{\OtherTok}[1]{\textcolor[rgb]{0.00,0.44,0.13}{{#1}}}
\newcommand{\AlertTok}[1]{\textcolor[rgb]{1.00,0.00,0.00}{\textbf{{#1}}}}
\newcommand{\FunctionTok}[1]{\textcolor[rgb]{0.02,0.16,0.49}{{#1}}}
\newcommand{\RegionMarkerTok}[1]{{#1}}
\newcommand{\ErrorTok}[1]{\textcolor[rgb]{1.00,0.00,0.00}{\textbf{{#1}}}}
\newcommand{\NormalTok}[1]{{#1}}

% Additional commands for more recent versions of Pandoc
\newcommand{\ConstantTok}[1]{\textcolor[rgb]{0.53,0.00,0.00}{{#1}}}
\newcommand{\SpecialCharTok}[1]{\textcolor[rgb]{0.25,0.44,0.63}{{#1}}}
\newcommand{\VerbatimStringTok}[1]{\textcolor[rgb]{0.25,0.44,0.63}{{#1}}}
\newcommand{\SpecialStringTok}[1]{\textcolor[rgb]{0.73,0.40,0.53}{{#1}}}
\newcommand{\ImportTok}[1]{{#1}}
\newcommand{\DocumentationTok}[1]{\textcolor[rgb]{0.73,0.13,0.13}{\textit{{#1}}}}
\newcommand{\AnnotationTok}[1]{\textcolor[rgb]{0.38,0.63,0.69}{\textbf{\textit{{#1}}}}}
\newcommand{\CommentVarTok}[1]{\textcolor[rgb]{0.38,0.63,0.69}{\textbf{\textit{{#1}}}}}
\newcommand{\VariableTok}[1]{\textcolor[rgb]{0.10,0.09,0.49}{{#1}}}
\newcommand{\ControlFlowTok}[1]{\textcolor[rgb]{0.00,0.44,0.13}{\textbf{{#1}}}}
\newcommand{\OperatorTok}[1]{\textcolor[rgb]{0.40,0.40,0.40}{{#1}}}
\newcommand{\BuiltInTok}[1]{{#1}}
\newcommand{\ExtensionTok}[1]{{#1}}
\newcommand{\PreprocessorTok}[1]{\textcolor[rgb]{0.74,0.48,0.00}{{#1}}}
\newcommand{\AttributeTok}[1]{\textcolor[rgb]{0.49,0.56,0.16}{{#1}}}
\newcommand{\InformationTok}[1]{\textcolor[rgb]{0.38,0.63,0.69}{\textbf{\textit{{#1}}}}}
\newcommand{\WarningTok}[1]{\textcolor[rgb]{0.38,0.63,0.69}{\textbf{\textit{{#1}}}}}


% Define a nice break command that doesn't care if a line doesn't already
% exist.
\def\br{\hspace*{\fill} \\* }
% Math Jax compatibility definitions
\def\gt{>}
\def\lt{<}
\let\Oldtex\TeX
\let\Oldlatex\LaTeX
\renewcommand{\TeX}{\textrm{\Oldtex}}
\renewcommand{\LaTeX}{\textrm{\Oldlatex}}
% Document parameters
% Document title
\title{
  {\Large Лекция 2} \\
  Матрицы и действия над ними. Ранг матрицы
}
% \date{14 сентября 2022\,г.}
\date{}



% Pygments definitions
\makeatletter
\def\PY@reset{\let\PY@it=\relax \let\PY@bf=\relax%
    \let\PY@ul=\relax \let\PY@tc=\relax%
    \let\PY@bc=\relax \let\PY@ff=\relax}
\def\PY@tok#1{\csname PY@tok@#1\endcsname}
\def\PY@toks#1+{\ifx\relax#1\empty\else%
    \PY@tok{#1}\expandafter\PY@toks\fi}
\def\PY@do#1{\PY@bc{\PY@tc{\PY@ul{%
    \PY@it{\PY@bf{\PY@ff{#1}}}}}}}
\def\PY#1#2{\PY@reset\PY@toks#1+\relax+\PY@do{#2}}

\@namedef{PY@tok@w}{\def\PY@tc##1{\textcolor[rgb]{0.73,0.73,0.73}{##1}}}
\@namedef{PY@tok@c}{\let\PY@it=\textit\def\PY@tc##1{\textcolor[rgb]{0.24,0.48,0.48}{##1}}}
\@namedef{PY@tok@cp}{\def\PY@tc##1{\textcolor[rgb]{0.61,0.40,0.00}{##1}}}
\@namedef{PY@tok@k}{\let\PY@bf=\textbf\def\PY@tc##1{\textcolor[rgb]{0.00,0.50,0.00}{##1}}}
\@namedef{PY@tok@kp}{\def\PY@tc##1{\textcolor[rgb]{0.00,0.50,0.00}{##1}}}
\@namedef{PY@tok@kt}{\def\PY@tc##1{\textcolor[rgb]{0.69,0.00,0.25}{##1}}}
\@namedef{PY@tok@o}{\def\PY@tc##1{\textcolor[rgb]{0.40,0.40,0.40}{##1}}}
\@namedef{PY@tok@ow}{\let\PY@bf=\textbf\def\PY@tc##1{\textcolor[rgb]{0.67,0.13,1.00}{##1}}}
\@namedef{PY@tok@nb}{\def\PY@tc##1{\textcolor[rgb]{0.00,0.50,0.00}{##1}}}
\@namedef{PY@tok@nf}{\def\PY@tc##1{\textcolor[rgb]{0.00,0.00,1.00}{##1}}}
\@namedef{PY@tok@nc}{\let\PY@bf=\textbf\def\PY@tc##1{\textcolor[rgb]{0.00,0.00,1.00}{##1}}}
\@namedef{PY@tok@nn}{\let\PY@bf=\textbf\def\PY@tc##1{\textcolor[rgb]{0.00,0.00,1.00}{##1}}}
\@namedef{PY@tok@ne}{\let\PY@bf=\textbf\def\PY@tc##1{\textcolor[rgb]{0.80,0.25,0.22}{##1}}}
\@namedef{PY@tok@nv}{\def\PY@tc##1{\textcolor[rgb]{0.10,0.09,0.49}{##1}}}
\@namedef{PY@tok@no}{\def\PY@tc##1{\textcolor[rgb]{0.53,0.00,0.00}{##1}}}
\@namedef{PY@tok@nl}{\def\PY@tc##1{\textcolor[rgb]{0.46,0.46,0.00}{##1}}}
\@namedef{PY@tok@ni}{\let\PY@bf=\textbf\def\PY@tc##1{\textcolor[rgb]{0.44,0.44,0.44}{##1}}}
\@namedef{PY@tok@na}{\def\PY@tc##1{\textcolor[rgb]{0.41,0.47,0.13}{##1}}}
\@namedef{PY@tok@nt}{\let\PY@bf=\textbf\def\PY@tc##1{\textcolor[rgb]{0.00,0.50,0.00}{##1}}}
\@namedef{PY@tok@nd}{\def\PY@tc##1{\textcolor[rgb]{0.67,0.13,1.00}{##1}}}
\@namedef{PY@tok@s}{\def\PY@tc##1{\textcolor[rgb]{0.73,0.13,0.13}{##1}}}
\@namedef{PY@tok@sd}{\let\PY@it=\textit\def\PY@tc##1{\textcolor[rgb]{0.73,0.13,0.13}{##1}}}
\@namedef{PY@tok@si}{\let\PY@bf=\textbf\def\PY@tc##1{\textcolor[rgb]{0.64,0.35,0.47}{##1}}}
\@namedef{PY@tok@se}{\let\PY@bf=\textbf\def\PY@tc##1{\textcolor[rgb]{0.67,0.36,0.12}{##1}}}
\@namedef{PY@tok@sr}{\def\PY@tc##1{\textcolor[rgb]{0.64,0.35,0.47}{##1}}}
\@namedef{PY@tok@ss}{\def\PY@tc##1{\textcolor[rgb]{0.10,0.09,0.49}{##1}}}
\@namedef{PY@tok@sx}{\def\PY@tc##1{\textcolor[rgb]{0.00,0.50,0.00}{##1}}}
\@namedef{PY@tok@m}{\def\PY@tc##1{\textcolor[rgb]{0.40,0.40,0.40}{##1}}}
\@namedef{PY@tok@gh}{\let\PY@bf=\textbf\def\PY@tc##1{\textcolor[rgb]{0.00,0.00,0.50}{##1}}}
\@namedef{PY@tok@gu}{\let\PY@bf=\textbf\def\PY@tc##1{\textcolor[rgb]{0.50,0.00,0.50}{##1}}}
\@namedef{PY@tok@gd}{\def\PY@tc##1{\textcolor[rgb]{0.63,0.00,0.00}{##1}}}
\@namedef{PY@tok@gi}{\def\PY@tc##1{\textcolor[rgb]{0.00,0.52,0.00}{##1}}}
\@namedef{PY@tok@gr}{\def\PY@tc##1{\textcolor[rgb]{0.89,0.00,0.00}{##1}}}
\@namedef{PY@tok@ge}{\let\PY@it=\textit}
\@namedef{PY@tok@gs}{\let\PY@bf=\textbf}
\@namedef{PY@tok@gp}{\let\PY@bf=\textbf\def\PY@tc##1{\textcolor[rgb]{0.00,0.00,0.50}{##1}}}
\@namedef{PY@tok@go}{\def\PY@tc##1{\textcolor[rgb]{0.44,0.44,0.44}{##1}}}
\@namedef{PY@tok@gt}{\def\PY@tc##1{\textcolor[rgb]{0.00,0.27,0.87}{##1}}}
\@namedef{PY@tok@err}{\def\PY@bc##1{{\setlength{\fboxsep}{\string -\fboxrule}\fcolorbox[rgb]{1.00,0.00,0.00}{1,1,1}{\strut ##1}}}}
\@namedef{PY@tok@kc}{\let\PY@bf=\textbf\def\PY@tc##1{\textcolor[rgb]{0.00,0.50,0.00}{##1}}}
\@namedef{PY@tok@kd}{\let\PY@bf=\textbf\def\PY@tc##1{\textcolor[rgb]{0.00,0.50,0.00}{##1}}}
\@namedef{PY@tok@kn}{\let\PY@bf=\textbf\def\PY@tc##1{\textcolor[rgb]{0.00,0.50,0.00}{##1}}}
\@namedef{PY@tok@kr}{\let\PY@bf=\textbf\def\PY@tc##1{\textcolor[rgb]{0.00,0.50,0.00}{##1}}}
\@namedef{PY@tok@bp}{\def\PY@tc##1{\textcolor[rgb]{0.00,0.50,0.00}{##1}}}
\@namedef{PY@tok@fm}{\def\PY@tc##1{\textcolor[rgb]{0.00,0.00,1.00}{##1}}}
\@namedef{PY@tok@vc}{\def\PY@tc##1{\textcolor[rgb]{0.10,0.09,0.49}{##1}}}
\@namedef{PY@tok@vg}{\def\PY@tc##1{\textcolor[rgb]{0.10,0.09,0.49}{##1}}}
\@namedef{PY@tok@vi}{\def\PY@tc##1{\textcolor[rgb]{0.10,0.09,0.49}{##1}}}
\@namedef{PY@tok@vm}{\def\PY@tc##1{\textcolor[rgb]{0.10,0.09,0.49}{##1}}}
\@namedef{PY@tok@sa}{\def\PY@tc##1{\textcolor[rgb]{0.73,0.13,0.13}{##1}}}
\@namedef{PY@tok@sb}{\def\PY@tc##1{\textcolor[rgb]{0.73,0.13,0.13}{##1}}}
\@namedef{PY@tok@sc}{\def\PY@tc##1{\textcolor[rgb]{0.73,0.13,0.13}{##1}}}
\@namedef{PY@tok@dl}{\def\PY@tc##1{\textcolor[rgb]{0.73,0.13,0.13}{##1}}}
\@namedef{PY@tok@s2}{\def\PY@tc##1{\textcolor[rgb]{0.73,0.13,0.13}{##1}}}
\@namedef{PY@tok@sh}{\def\PY@tc##1{\textcolor[rgb]{0.73,0.13,0.13}{##1}}}
\@namedef{PY@tok@s1}{\def\PY@tc##1{\textcolor[rgb]{0.73,0.13,0.13}{##1}}}
\@namedef{PY@tok@mb}{\def\PY@tc##1{\textcolor[rgb]{0.40,0.40,0.40}{##1}}}
\@namedef{PY@tok@mf}{\def\PY@tc##1{\textcolor[rgb]{0.40,0.40,0.40}{##1}}}
\@namedef{PY@tok@mh}{\def\PY@tc##1{\textcolor[rgb]{0.40,0.40,0.40}{##1}}}
\@namedef{PY@tok@mi}{\def\PY@tc##1{\textcolor[rgb]{0.40,0.40,0.40}{##1}}}
\@namedef{PY@tok@il}{\def\PY@tc##1{\textcolor[rgb]{0.40,0.40,0.40}{##1}}}
\@namedef{PY@tok@mo}{\def\PY@tc##1{\textcolor[rgb]{0.40,0.40,0.40}{##1}}}
\@namedef{PY@tok@ch}{\let\PY@it=\textit\def\PY@tc##1{\textcolor[rgb]{0.24,0.48,0.48}{##1}}}
\@namedef{PY@tok@cm}{\let\PY@it=\textit\def\PY@tc##1{\textcolor[rgb]{0.24,0.48,0.48}{##1}}}
\@namedef{PY@tok@cpf}{\let\PY@it=\textit\def\PY@tc##1{\textcolor[rgb]{0.24,0.48,0.48}{##1}}}
\@namedef{PY@tok@c1}{\let\PY@it=\textit\def\PY@tc##1{\textcolor[rgb]{0.24,0.48,0.48}{##1}}}
\@namedef{PY@tok@cs}{\let\PY@it=\textit\def\PY@tc##1{\textcolor[rgb]{0.24,0.48,0.48}{##1}}}

\def\PYZbs{\char`\\}
\def\PYZus{\char`\_}
\def\PYZob{\char`\{}
\def\PYZcb{\char`\}}
\def\PYZca{\char`\^}
\def\PYZam{\char`\&}
\def\PYZlt{\char`\<}
\def\PYZgt{\char`\>}
\def\PYZsh{\char`\#}
\def\PYZpc{\char`\%}
\def\PYZdl{\char`\$}
\def\PYZhy{\char`\-}
\def\PYZsq{\char`\'}
\def\PYZdq{\char`\"}
\def\PYZti{\char`\~}
% for compatibility with earlier versions
\def\PYZat{@}
\def\PYZlb{[}
\def\PYZrb{]}
\makeatother


% For linebreaks inside Verbatim environment from package fancyvrb.
\makeatletter
    \newbox\Wrappedcontinuationbox
    \newbox\Wrappedvisiblespacebox
    \newcommand*\Wrappedvisiblespace {\textcolor{red}{\textvisiblespace}}
    \newcommand*\Wrappedcontinuationsymbol {\textcolor{red}{\llap{\tiny$\m@th\hookrightarrow$}}}
    \newcommand*\Wrappedcontinuationindent {3ex }
    \newcommand*\Wrappedafterbreak {\kern\Wrappedcontinuationindent\copy\Wrappedcontinuationbox}
    % Take advantage of the already applied Pygments mark-up to insert
    % potential linebreaks for TeX processing.
    %        {, <, #, %, $, ' and ": go to next line.
    %        _, }, ^, &, >, - and ~: stay at end of broken line.
    % Use of \textquotesingle for straight quote.
    \newcommand*\Wrappedbreaksatspecials {%
        \def\PYGZus{\discretionary{\char`\_}{\Wrappedafterbreak}{\char`\_}}%
        \def\PYGZob{\discretionary{}{\Wrappedafterbreak\char`\{}{\char`\{}}%
        \def\PYGZcb{\discretionary{\char`\}}{\Wrappedafterbreak}{\char`\}}}%
        \def\PYGZca{\discretionary{\char`\^}{\Wrappedafterbreak}{\char`\^}}%
        \def\PYGZam{\discretionary{\char`\&}{\Wrappedafterbreak}{\char`\&}}%
        \def\PYGZlt{\discretionary{}{\Wrappedafterbreak\char`\<}{\char`\<}}%
        \def\PYGZgt{\discretionary{\char`\>}{\Wrappedafterbreak}{\char`\>}}%
        \def\PYGZsh{\discretionary{}{\Wrappedafterbreak\char`\#}{\char`\#}}%
        \def\PYGZpc{\discretionary{}{\Wrappedafterbreak\char`\%}{\char`\%}}%
        \def\PYGZdl{\discretionary{}{\Wrappedafterbreak\char`\$}{\char`\$}}%
        \def\PYGZhy{\discretionary{\char`\-}{\Wrappedafterbreak}{\char`\-}}%
        \def\PYGZsq{\discretionary{}{\Wrappedafterbreak\textquotesingle}{\textquotesingle}}%
        \def\PYGZdq{\discretionary{}{\Wrappedafterbreak\char`\"}{\char`\"}}%
        \def\PYGZti{\discretionary{\char`\~}{\Wrappedafterbreak}{\char`\~}}%
    }
    % Some characters . , ; ? ! / are not pygmentized.
    % This macro makes them "active" and they will insert potential linebreaks
    \newcommand*\Wrappedbreaksatpunct {%
        \lccode`\~`\.\lowercase{\def~}{\discretionary{\hbox{\char`\.}}{\Wrappedafterbreak}{\hbox{\char`\.}}}%
        \lccode`\~`\,\lowercase{\def~}{\discretionary{\hbox{\char`\,}}{\Wrappedafterbreak}{\hbox{\char`\,}}}%
        \lccode`\~`\;\lowercase{\def~}{\discretionary{\hbox{\char`\;}}{\Wrappedafterbreak}{\hbox{\char`\;}}}%
        \lccode`\~`\:\lowercase{\def~}{\discretionary{\hbox{\char`\:}}{\Wrappedafterbreak}{\hbox{\char`\:}}}%
        \lccode`\~`\?\lowercase{\def~}{\discretionary{\hbox{\char`\?}}{\Wrappedafterbreak}{\hbox{\char`\?}}}%
        \lccode`\~`\!\lowercase{\def~}{\discretionary{\hbox{\char`\!}}{\Wrappedafterbreak}{\hbox{\char`\!}}}%
        \lccode`\~`\/\lowercase{\def~}{\discretionary{\hbox{\char`\/}}{\Wrappedafterbreak}{\hbox{\char`\/}}}%
        \catcode`\.\active
        \catcode`\,\active
        \catcode`\;\active
        \catcode`\:\active
        \catcode`\?\active
        \catcode`\!\active
        \catcode`\/\active
        \lccode`\~`\~
    }
\makeatother

\let\OriginalVerbatim=\Verbatim
\makeatletter
\renewcommand{\Verbatim}[1][1]{%
    %\parskip\z@skip
    \sbox\Wrappedcontinuationbox {\Wrappedcontinuationsymbol}%
    \sbox\Wrappedvisiblespacebox {\FV@SetupFont\Wrappedvisiblespace}%
    \def\FancyVerbFormatLine ##1{\hsize\linewidth
        \vtop{\raggedright\hyphenpenalty\z@\exhyphenpenalty\z@
            \doublehyphendemerits\z@\finalhyphendemerits\z@
            \strut ##1\strut}%
    }%
    % If the linebreak is at a space, the latter will be displayed as visible
    % space at end of first line, and a continuation symbol starts next line.
    % Stretch/shrink are however usually zero for typewriter font.
    \def\FV@Space {%
        \nobreak\hskip\z@ plus\fontdimen3\font minus\fontdimen4\font
        \discretionary{\copy\Wrappedvisiblespacebox}{\Wrappedafterbreak}
        {\kern\fontdimen2\font}%
    }%

    % Allow breaks at special characters using \PYG... macros.
    \Wrappedbreaksatspecials
    % Breaks at punctuation characters . , ; ? ! and / need catcode=\active
    \OriginalVerbatim[#1,codes*=\Wrappedbreaksatpunct]%
}
\makeatother

% Exact colors from NB
\definecolor{incolor}{HTML}{303F9F}
\definecolor{outcolor}{HTML}{D84315}
\definecolor{cellborder}{HTML}{CFCFCF}
\definecolor{cellbackground}{HTML}{F7F7F7}

% prompt
\makeatletter
\newcommand{\boxspacing}{\kern\kvtcb@left@rule\kern\kvtcb@boxsep}
\makeatother
\newcommand{\prompt}[4]{
    {\ttfamily\llap{{\color{#2}[#3]:\hspace{3pt}#4}}\vspace{-\baselineskip}}
}



% Prevent overflowing lines due to hard-to-break entities
\sloppy
% Setup hyperref package
\hypersetup{
  breaklinks=true,  % so long urls are correctly broken across lines
  colorlinks=true,
  urlcolor=urlcolor,
  linkcolor=linkcolor,
  citecolor=citecolor,
  }
% Slightly bigger margins than the latex defaults

\geometry{verbose,tmargin=1in,bmargin=1in,lmargin=1in,rmargin=1in}



\begin{document}

  \maketitle
  \thispagestyle{empty}
  \tableofcontents
  \newpage


\hypertarget{ux432ux432ux435ux434ux435ux43dux438ux435}{%
\section{Введение}\label{ux432ux432ux435ux434ux435ux43dux438ux435}}

\hypertarget{ux43eux441ux43dux43eux432ux43dux44bux435-ux437ux430ux434ux430ux447ux438}{%
\subsection{Основные
задачи}\label{ux43eux441ux43dux43eux432ux43dux44bux435-ux437ux430ux434ux430ux447ux438}}

Что мы хотим вспомнить из линейной алгебры? Мы хотим уметь решать три базовые задачи:

\begin{enumerate}
\def\labelenumi{\arabic{enumi}.}
\tightlist
\item
  Найти \(\mathbf{x}\) в уравнении \(A\mathbf{x} = \mathbf{b}\),
\item
  Найти \(\mathbf{x}\) и \(\lambda\) в уравнении
  \(A\mathbf{x} = \lambda \mathbf{x}\),
\item
  Найти \(\mathbf{v}\), \(\mathbf{u}\) и \(\sigma\) в уравнении
  \(A\mathbf{v} = \sigma \mathbf{u}\).
\end{enumerate}

Подробнее о задачах:
\begin{enumerate}
\item
  Можно ли вектор \(\mathbf{b}\) представить в виде линейной комбинации
  векторов матрицы \(A\)?
\item
  Вектор \(A\mathbf{x}\) имеет то же направление, что и вектор
  \(\mathbf{x}\). Вдоль этого направления все сложные взаимодействия с
  матрицей \(A\) чрезвычайно упрощаются. Например, вектор
  \(A^2 \mathbf{x}\) становится просто \(\lambda^2 \mathbf{x}\).
  Упрощается вычисление матричной экспоненты:
  \(e^{A} = e^{X \Lambda X^{-1}} = X e^{\Lambda} X^{-1}\). Короче
  говоря, многие действия становятся линейными.
\item
  Уравнение \(A\mathbf{v} = \sigma \mathbf{u}\) похоже на предыдущее. Но
  матрица \(A\) больше не квадратная. Это сингулярное разложение.
\end{enumerate}

    \hypertarget{ux43eux431ux43eux437ux43dux430ux447ux435ux43dux438ux44f}{%
\subsection{Обозначения}\label{ux43eux431ux43eux437ux43dux430ux447ux435ux43dux438ux44f}}

    Для начала введём обозначения. Вектор будем записывать в виде столбца и
обозначать стрелочкой \(\vec{a}\) или жирной буквой \(\mathbf{a}\): \[
  \vec{a} = \mathbf{a} =
  \begin{pmatrix}
     a_1    \\
     \cdots \\
     a_m    \\
  \end{pmatrix}
\]

    Матрицу будем обозначать большой буквой, а записывать в виде набора
элементов, строк или столбцов: \[
  A = 
  \begin{pmatrix}
    a_{11} & a_{12} & \ldots & a_{1n} \\
    a_{21} & a_{22} & \ldots & a_{2n} \\
    \vdots & \vdots & \ddots & \vdots \\
    a_{m1} & a_{m2} & \ldots & a_{mn} \\
  \end{pmatrix}
  =
  \begin{pmatrix}
  -\, \mathbf{a}_{1*} \,- \\
  -\, \mathbf{a}_{2*} \,- \\
  \cdots \\
  -\, \mathbf{a}_{m*} \,- \\
  \end{pmatrix}
  =
  \begin{pmatrix}
     | & | & {} & | \\
     \mathbf{a}_{*1} & \mathbf{a}_{*2} & \cdots & \mathbf{a}_{*n} \\
     | & | & {} & | \\
  \end{pmatrix}
  .
\]

    \begin{center}\rule{0.5\linewidth}{0.5pt}\end{center}

    \hypertarget{ux443ux43cux43dux43eux436ux435ux43dux438ux435-ux43cux430ux442ux440ux438ux446}{%
\section{Умножение
матриц}\label{ux443ux43cux43dux43eux436ux435ux43dux438ux435-ux43cux430ux442ux440ux438ux446}}

Рассмотрим 4 способа умножения матриц \(A \cdot B = C\):

\begin{enumerate}
\def\labelenumi{\arabic{enumi}.}
\tightlist
\item
  Строка на столбец
\item
  Столбец на строку
\item
  Столбец на столбец
\item
  Строка на строку
\end{enumerate}

    \hypertarget{ux441ux43fux43eux441ux43eux431-1-ux441ux442ux440ux43eux43aux430-ux43dux430-ux441ux442ux43eux43bux431ux435ux446}{%
\subsection{Способ 1: «строка на
столбец»}\label{ux441ux43fux43eux441ux43eux431-1-ux441ux442ux440ux43eux43aux430-ux43dux430-ux441ux442ux43eux43bux431ux435ux446}}

Это стандартный способ умножения матриц:
\(\mathbf{c_{ij}} = \mathbf{a}_{i*} \cdot \mathbf{b}_{*j}\).

\textbf{Пример 1.} Скалярное произведение векторов \(\mathbf{a}\) и
\(\mathbf{b}\):
\[ (\mathbf{a}, \mathbf{b}) = \mathbf{a}^\top \cdot \mathbf{b}. \]

    \hypertarget{ux441ux43fux43eux441ux43eux431-2-ux441ux442ux43eux43bux431ux435ux446-ux43dux430-ux441ux442ux440ux43eux43aux443}{%
\subsection{Способ 2: «столбец на
строку»}\label{ux441ux43fux43eux441ux43eux431-2-ux441ux442ux43eux43bux431ux435ux446-ux43dux430-ux441ux442ux440ux43eux43aux443}}

Существует другой способ умножения матриц. Произведение \(AB\) равно
\emph{сумме произведений} всех столбцов матрицы \(A\) на соответствующие
строки матрицы \(B\).
\[ A \cdot B = \sum_i \mathbf{a}_{*i} \cdot \mathbf{b}_{i*}. \]

    \textbf{Пример 2.} \[
\begin{aligned}
  \begin{pmatrix}
     1 & 2 \\
     0 & 2 \\
  \end{pmatrix}
  \cdot
  \begin{pmatrix}
     0 & 5 \\
     1 & 1 \\
  \end{pmatrix}
  &=
  \begin{pmatrix}
     1 \\
     0 \\
  \end{pmatrix}
  \cdot
  \begin{pmatrix}
     0 & 5 \\
  \end{pmatrix}
  +
  \begin{pmatrix}
     2 \\
     2 \\
  \end{pmatrix}
  \cdot
  \begin{pmatrix}
     1 & 1 \\
  \end{pmatrix}
  \\
  &=
  \begin{pmatrix}
     0 & 5 \\
     0 & 0 \\
  \end{pmatrix}
  +
  \begin{pmatrix}
     2 & 2 \\
     2 & 2 \\
  \end{pmatrix}
  =
  \begin{pmatrix}
     2 & 7 \\
     2 & 2 \\
  \end{pmatrix}
  .
\end{aligned}
\]

    \hypertarget{ux441ux43fux43eux441ux43eux431-3-ux441ux442ux43eux43bux431ux435ux446-ux43dux430-ux441ux442ux43eux43bux431ux435ux446}{%
\subsection{Способ 3: «столбец на
столбец»}\label{ux441ux43fux43eux441ux43eux431-3-ux441ux442ux43eux43bux431ux435ux446-ux43dux430-ux441ux442ux43eux43bux431ux435ux446}}

Посмотрим внимательно на то, как мы получили первый столбец
результирующей матрицы. Он является суммой столбцов матрицы \(A\) с
коэффициентами из первого столбца матрицы \(B\). То же самое можно
сказать и про второй столбец. Таким образом, можно сформулировать
следующее правило:\\
\emph{\(\mathbf{i}\)-ый столбец результирующей матрицы есть линейная
комбинация столбцов левой матрицы с коэффициентами из \(\mathbf{i}\)-ого
столбца правой матрицы.}

    \hypertarget{ux441ux43fux43eux441ux43eux431-4-ux441ux442ux440ux43eux43aux430-ux43dux430-ux441ux442ux440ux43eux43aux443}{%
\subsection{Способ 4: «строка на
строку»}\label{ux441ux43fux43eux441ux43eux431-4-ux441ux442ux440ux43eux43aux430-ux43dux430-ux441ux442ux440ux43eux43aux443}}

Аналогичным образом можно вывести правило и для строк:\\
\emph{\(\mathbf{i}\)-ая строка результирующей матрицы есть линейная
комбинация строк правой матрицы с коэффициентами из \(\mathbf{i}\)-ой
строки левой матрицы.}

    \textbf{Пример 3. Умножение матрицы на вектор}

Рассмотрим умножение матрицы \(A\) размером \(m \times n\) на вектор:
\(A \mathbf{x}\).

По определению произведение \(A \mathbf{x}\) есть вектор, в котором на
\(i\)-ом месте находится скалярное произведение \(i\)-ой \emph{строки}
на столбец \(\mathbf{x}\): \[
  A \mathbf{x} = 
  \begin{pmatrix}
    -\, \mathbf{a}_{1*} \,- \\
    -\, \mathbf{a}_{2*} \,- \\
    \cdots \\
    -\, \mathbf{a}_{m*} \,- \\
  \end{pmatrix}
  \cdot \mathbf{x} = 
  \begin{pmatrix}
    \mathbf{a}_{1*} \cdot \mathbf{x} \\
    \mathbf{a}_{2*} \cdot \mathbf{x} \\
    \cdots \\
    \mathbf{a}_{m*} \cdot \mathbf{x} \\
  \end{pmatrix}.
\]

Но можно посмотреть на это иначе, как на произведение \emph{столбцов}
матрицы \(A\) на элементы вектора \(\mathbf{x}\): \[
  A \mathbf{x} = 
  \begin{pmatrix}
     | & | & {} & | \\
     \mathbf{a}_{*1} & \mathbf{a}_{*2} & \cdots & \mathbf{a}_{*n} \\
     | & | & {} & | \\
  \end{pmatrix}
  \begin{pmatrix}
     x_1    \\
     x_2    \\
     \cdots \\
     x_m \\
  \end{pmatrix}
  = 
  x_1 \mathbf{a}_{*1} + x_2 \mathbf{a}_{*2} + \dots + x_m \mathbf{a}_{*m}.
\] Таким образом, результирующий вектор есть \emph{линейная комбинация}
столбцов матрицы \(A\) с коэффициентами из вектора \(\mathbf{x}\).

    \textbf{Пример 4. Умножение столбцов (строк) матрицы на скаляры}

Чтобы каждый вектор матрицы \(A\) умножить на скаляр \(\lambda_i\),
нужно умножить \(A\) на матрицу \(\Lambda = \mathrm{diag}(\lambda_i)\)
\emph{справа}: \[
  \begin{pmatrix}
    | & {} & | \\
    \lambda_1 \mathbf{a}_{*1} & \cdots & \lambda_n \mathbf{a}_{*n} \\
    | & {} & | \\
  \end{pmatrix}
  =
  \begin{pmatrix}
    | & {} & | \\
    \mathbf{a}_{*1} & \cdots & \mathbf{a}_{*n} \\
    | & {} & | \\
  \end{pmatrix}
  \begin{pmatrix}
    \lambda_{1} & \ldots & 0         \\
    \vdots      & \ddots & \vdots    \\
    0           & \ldots & \lambda_n \\
  \end{pmatrix}
  = A \cdot \Lambda.
\]

Чтобы проделать то же самое со строками матрицы, её нужно умножить на
\(\Lambda\) \emph{слева}: \[
  \begin{pmatrix}
    -\, \lambda_1 \mathbf{a}_{1*} \,- \\
    \cdots \\
    -\, \lambda_m \mathbf{a}_{m*} \,- \\
  \end{pmatrix}
  =
  \begin{pmatrix}
    \lambda_{1} & \ldots & 0         \\
    \vdots      & \ddots & \vdots    \\
    0           & \ldots & \lambda_n \\
  \end{pmatrix}
  \begin{pmatrix}
    -\, \mathbf{a}_{1*} \,- \\
    \cdots \\
    -\, \mathbf{a}_{m*} \,- \\
  \end{pmatrix}
  = \Lambda \cdot A.
\]

    \textbf{Пример 5.} Разложение матрицы на сумму матриц ранга 1.

Представим матрицу в виде произведения двух матриц и применим умножение
«столбец на строку».

    Первый способ (раскладываем матрицу по двум её первым столбцам): \[
  A = 
  \begin{pmatrix}
     1 & 2 & 3 \\
     2 & 1 & 3 \\
     3 & 3 & 6 \\
  \end{pmatrix}
  =
  \begin{pmatrix}
     1 & 2 \\
     2 & 1 \\
     3 & 3 \\
  \end{pmatrix}
  \cdot
  \begin{pmatrix}
     1 & 0 & 1 \\
     0 & 1 & 1 \\
  \end{pmatrix}
  =
  \begin{pmatrix}
     1 & 0 & 1 \\
     2 & 0 & 2 \\
     3 & 0 & 3 \\
  \end{pmatrix}
  +
  \begin{pmatrix}
     0 & 2 & 2 \\
     0 & 1 & 1 \\
     0 & 3 & 3 \\
  \end{pmatrix}.
\]

    Второй способ (раскладываем матрицу по двум её первым строкам): \[
  A = 
  \begin{pmatrix}
     1 & 2 & 3 \\
     2 & 1 & 3 \\
     3 & 3 & 6 \\
  \end{pmatrix}
  =
  \begin{pmatrix}
     1 & 0 \\
     0 & 1 \\
     1 & 1 \\
  \end{pmatrix}
  \cdot
  \begin{pmatrix}
     1 & 2 & 3 \\
     2 & 1 & 3 \\
  \end{pmatrix}
  =
  \begin{pmatrix}
     1 & 2 & 3 \\
     0 & 0 & 0 \\
     1 & 2 & 3 \\
  \end{pmatrix}
  +
  \begin{pmatrix}
     0 & 0 & 0 \\
     2 & 1 & 3 \\
     2 & 1 & 3 \\
  \end{pmatrix}.
\]

    \textbf{Пример 6. Умножение блочных матриц}

Рассмотрим блочную матрицу следующего вида:
\[ A = \begin{pmatrix} A_1 \\ A_2 \\ \end{pmatrix}. \]

\begin{enumerate}
\def\labelenumi{\arabic{enumi}.}
\tightlist
\item
  Формулу для \(A A^\top\) получим по способу «строка на столбец»: \[
    A A^\top = 
    \begin{pmatrix}
   A_1 \\
   A_2 \\
    \end{pmatrix}
    \cdot
    \begin{pmatrix}
   A_1^\top & A_2^\top \\
    \end{pmatrix}
    =
    \begin{pmatrix}
   A_1 A_1^\top & A_1 A_2^\top \\
   A_2 A_1^\top & A_2 A_2^\top \\
    \end{pmatrix}
  \]
\item
  Для \(A^\top A\) удобно применить способ «столбец на строку»: \[
    A^\top A =
    \begin{pmatrix}
   A_1^\top & A_2^\top \\
    \end{pmatrix}
    \begin{pmatrix}
   A_1 \\
   A_2 \\
    \end{pmatrix}
    = A_1^\top A_1 + A_2^\top A_2
  \]
\end{enumerate}

    \begin{center}\rule{0.5\linewidth}{0.5pt}\end{center}

    \hypertarget{ux440ux430ux43dux433-ux43cux430ux442ux440ux438ux446ux44b}{%
\section{Ранг
матрицы}\label{ux440ux430ux43dux433-ux43cux430ux442ux440ux438ux446ux44b}}

Посмотрим на задачу \(A \mathbf{x} = \mathbf{b}\).\\
Её можно сформулировать в виде следующего вопроса: можно ли столбец \(\mathbf{b}\) представить в~виде линейной комбинации столбцов матрицы
\(A\)?

Поясним на примере. Пусть \[
  A = 
  \begin{pmatrix}
     1 & 2 & 3 \\
     2 & 1 & 3 \\
     3 & 3 & 6 \\
  \end{pmatrix}.
\] Что мы можем сказать о линейной оболочке её столбцов? Что это за
пространство? Какой размерности?

    \textbf{Определение.} Рангом матрицы \(A\) с \(m\) строк и \(n\)
столбцов называется максимальное число линейно независимых столбцов
(строк).

    \textbf{Свойства ранга:}

\begin{enumerate}
\def\labelenumi{\arabic{enumi}.}
\tightlist
\item
  \(r(AB) \le r(A), r(B)\),
\item
  \(r(A+B) \le r(A) + r(B)\),
\item
  \(r(A^\top A) = r(AA^\top) = r(A) = r(A^\top)\),
\item
  Пусть \(A: m \times r\), \(B: r \times n\) и \(r(A) = r(B) = r\),
  тогда \(r(AB) = r\).
\end{enumerate}

\textbf{Доказательства:}

\begin{enumerate}
\def\labelenumi{\arabic{enumi}.}
\tightlist
\item
  При умножении матриц ранг не может увеличиться. Каждый столбец матрицы
  \(AB\) является линейной комбинацией столбцов матрицы \(A\), а каждая
  строка матрицы \(AB\) является линейной комбинацией строк матрицы
  \(B\). Поэтому пространство столбцов матрицы \(AB\) содержится в
  пространстве столбцов матрицы \(A\), а пространство строк матрицы
  \(AB\) содержится в пространстве строк матрицы \(B\).
\item
  Базис пространства столбцов матрицы \(A+B\) (т.е. \(\mathbf{C}(A+B)\)
  является комбинацией (возможно, с пересечениями) базисов пространств
  \(\mathbf{C}(A)\)) и \(\mathbf{C}(B)\).
\item
  Матрицы \(A\) и \(A^\top A\) имеют одно и то же нуль-пространство
  (доказать), поэтому их ранг одинаков.
\item
  Матрицы \(A^\top A\) и \(BB^\top\) невырождены, так как
  \(r(A^\top A) = r(BB^\top) = r\). Их произведение, матрица
  \(A^\top A BB^\top\), тоже невырождена и её ранг равен \(r\). Отсюда
  \(r = r(A^\top A BB^\top) \le r(AB) \le r(A) = r\).
\end{enumerate}

    \begin{quote}
Как доказать, что \(A\) и \(A^\top A\) имеют одно и то же
нуль-пространство?\\
Если \(Ax=0\), то \(A^\top Ax = 0\). Поэтому
\(\mathbf{N}(A) \subset \mathbf{N}(A^\top A)\).\\
Если \(A^\top Ax = 0\), то \(x^\top A^\top Ax = \|Ax\|^2 = 0\). Поэтому
\(\mathbf{N}(A^\top A) \subset \mathbf{N}(A)\).
\end{quote}

    \emph{Замечание.} Свойство 4 работает только в случае, когда \(A\) имеет
\emph{ровно} \(r\) столбцов, а \(B\) имеет \emph{ровно} \(r\) строк. В
частности, \(r(BA) \le r\) (в соответствии со свойством 1).

Для свойства 3 отметим важный частный случай. Это случай, когда столбцы
матрицы \(A\) линейно независимы, так что её ранг \(r\) равен \(n\).
Тогда матрица \(A^\top A\) является симметричной невырожденной матрицей.

    \begin{center}\rule{0.5\linewidth}{0.5pt}\end{center}

    \hypertarget{ux441ux43aux435ux43bux435ux442ux43dux43eux435-ux440ux430ux437ux43bux43eux436ux435ux43dux438ux435}{%
\section{Скелетное
разложение}\label{ux441ux43aux435ux43bux435ux442ux43dux43eux435-ux440ux430ux437ux43bux43eux436ux435ux43dux438ux435}}

В матрице \(A\) первые два вектора линейно независимы. Попробуем взять
их в качестве базиса и разложить по ним третий. Запишем это в матричном
виде, пользуясь правилом умножения «столбец на столбец».

\[
  A = 
  \begin{pmatrix}
     1 & 2 & 3 \\
     2 & 1 & 3 \\
     3 & 3 & 6 \\
  \end{pmatrix}
  =
  \begin{pmatrix}
     1 & 2 \\
     2 & 1 \\
     3 & 3 \\
  \end{pmatrix}
  \cdot
  \begin{pmatrix}
     1 & 0 & 1 \\
     0 & 1 & 1 \\
  \end{pmatrix}
  = C \cdot R
\]

Здесь \(C\) --- базисные столбцы, \(R\) --- базисные строки. Мы получили
скелетное разложение матрицы.

\textbf{Определение.} \emph{Скелетным разложением} матрицы \(A\)
размеров \(m \times n\) и ранга \(r>0\) называется разложение вида
\(A = CR\), где матрицы \(C\) и \(R\) имеют размеры соответственно
\(m \times r\) и \(r \times n\). Другое название скелетного разложения
--- \emph{ранговая факторизация}.

Это разложение иллюстрирует \textbf{теорему}: ранг матрицы по столбцам
(количество независимых столбцов) равен рангу матрицы по строкам
(количество независимых строк).

    \textbf{Дополнительно}

Существует другой вариант скелетного разложения: \(A = CMR\). В этом
случае \(С\) состоит из \(r\) независимых \emph{столбцов} матрицы \(A\),
а \(R\) --- из \(r\) независимых \emph{строк} матрицы \(A\). Матрица
\(M\) размером \(r \times r\) называется смешанной матрицей (mixing
matrix). Для \(M\) можно получить следующую формулу:
\[
\begin{aligned}
  A               &= CMR \\
  C^\top A R^\top &= C^\top C M R R^\top \\
  M               &= \left[ (C^\top C)^{-1}C^\top \right] A \left[ R^\top (R R^\top)^{-1} \right].
\end{aligned}
\]

    \begin{quote}
Матрицы \(C^+ = (C^\top C)^{-1}C^\top\) и
\(R^+ = R^\top (R R^\top)^{-1}\) являются \emph{псевдообратными} к
матрицам \(C\) и \(R\) соответственно.\\
Можно показать, что если \emph{столбцы} матрицы линейно независимы (как
у матрицы \(C\)), то \(C^+\) является \emph{левой} обратной матрицей для
\(C\): \(C^+ C = I\). Если независимы строки (как у \(R\)), то \(R^+\)
--- \emph{правая} обратная матрица для \(R\): \(R R^+ = I\).\\
Более подробный материал о псевдообратных матрицах будет на несколько
занятий позже.
\end{quote}

    \begin{center}\rule{0.5\linewidth}{0.5pt}\end{center}

    \hypertarget{ux438ux441ux442ux43eux447ux43dux438ux43aux438}{%
\section{Источники}\label{ux438ux441ux442ux43eux447ux43dux438ux43aux438}}

\begin{enumerate}
\def\labelenumi{\arabic{enumi}.}
\tightlist
\item
  \emph{Strang G.} Linear algebra and learning from data. ---
  Wellesley-Cambridge Press, 2019. --- 432\,p.
\item
  \emph{Гантмахер Ф.Р.} Теория матриц. --- М.: Наука, 1967. --- 576\,с.
\item
  \emph{Беклемишев Д.В.} Дополнительные главы линейной алгебры. --- М.:
  Наука, 1983. --- 336\,с.
\end{enumerate}


    % Add a bibliography block to the postdoc



\end{document}
