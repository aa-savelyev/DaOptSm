%%%%%%%%%%%%%%%%%%%%%%%%%%%%%%%%%%%%%%%%%%%%%%%%%%%%%%%%%%%%%%%%%%%%%%%%%%%
%% Packages
\documentclass[12pt,oneside,openany]{article}
\usepackage{polyglossia}                       % загружает пакет многоязыковой вёрстки
\usepackage{geometry}                          % для последующего задания полей

\usepackage{indentfirst}                       % первый абзац с красной строки
\usepackage[dvipsnames]{xcolor}
%\usepackage{savetrees}
\usepackage{hyperref}                          % гиперссылки

%%%%%%%%%%%%%%%%%%%%%%%%%%%%%%%%%%%%%%%%%%%%%%%%%%%%%%%%%%%%%%%%%%%%%%%%%%%
%% Styles
\geometry{a4paper, top=1cm, bottom=1cm, left=2.5cm, right=1cm, nofoot, nomarginpar} %, heightrounded, showframe

%% Languages
\setmainlanguage[babelshorthands=true]{russian} % устанавливает главный язык документа русский с поддержкой приятных команд пакета babel
\setotherlanguage{english}                     % устанавливает второй язык документа

%% Fonts
\defaultfontfeatures{Ligatures={TeX},Renderer=Basic} %% задаёт свойства шрифтов по умолчанию
\setmainfont{STIX Two Text}                    % задаёт основной шрифт документа
\setsansfont{Tahoma}                           % задаёт шрифт без засечек
\setmonofont{Courier New}                      % задаёт моноширинный шрифт
\newfontfamily\cyrillicfonttt[Script=Cyrillic]{Courier New}
\usepackage{unicode-math}                      % использование Unicode-шрифтов для формул
\setmathfont{STIX Two Math}

%% Выравнивание и переносы
\tolerance 1414
\hbadness 1414
\emergencystretch 1.5em % В случае проблем регулировать в первую очередь
\hfuzz 0.3pt
\vfuzz \hfuzz
%\raggedbottom
%\sloppy                % Избавляемся от переполнений
\clubpenalty=10000      % Запрещаем разрыв страницы после первой строки абзаца
\widowpenalty=10000     % Запрещаем разрыв страницы после последней строки абзаца
\brokenpenalty=4991     % Ограничение на разрыв страницы, если строка заканчивается переносом

\hypersetup{pdfstartview=FitH, colorlinks=true, urlcolor=blue}
%%%%%%%%%%%%%%%%%%%%%%%%%%%%%%%%%%%%%%%%%%%%%%%%%%%%%%%%%%%%%%%%%%%%%%%%%%%
%% Commands
\newcommand*{\todo}[1]{\textcolor{magenta}{\textbf{#1}}}
\newcommand*{\Rey}{\mathrm{Re}}

%%%%%%%%%%%%%%%%%%%%%%%%%%%%%%%%%%%%%%%%%%%%%%%%%%%%%%%%%%%%%%%%%%%%%%%%%%%


\begin{document}

\title{
  \large
  \textbf{Программа курса <<Анализ данных, суррогатное моделирование и~оптимизация в~прикладных задачах>>} \\
  осенний семестр
}

\author{}
\date{}

\maketitle
\thispagestyle{empty}

\vspace{-10ex}
%\section{Отчётность}


\begin{enumerate}

  \item \textbf{Введение. Постановка задачи} \\
  Содержание курса, взаимосвязь с другими курсами. Постановка задачи аэродинамического проектирования. Примеры из практики.

  \item \textbf{Матрицы и действия над ними} \\
  Умножение матриц, умножение блочных матриц. Ранг, ранговая факторизация матриц. Ортогональные, симметричные и положительно определённые матрицы.

  \item \textbf{Теория систем линейных уравнений} \\
  Алгоритм Гаусса. LU-разложение матриц, разложение Холецкого. Псевдорешения и~псевдообратные матрицы. Собственные подпространства, ядро и образ отображения, теорема Фредгольма.

  \item \textbf{Ортогональность} \\
  Скалярное произведение. Ортогональные векторы и ортогональные матрицы. Ортогональное проектирование, проекционные матрицы. Алгоритм ортогонализации Грама --- Шмидта, QR-разложение матриц.

  \item \textbf{Линейная регрессия} \\
  Формулировка задачи. Метод наименьших квадратов. Алгоритмы решения задачи наименьших квадратов: разложение Холецкого, QR-разложение.

  \item \textbf{Спектральное разложение матриц} \\
  Собственные значения и собственные векторы. Диагонализируемые и недиагонализируемые матрицы. Спектральное разложение матриц, полярное разложение матриц. Методы поиска собственных чисел: прямые, итерационные. QR-алгоритм.

  \item \textbf{Сингулярное разложение матриц} \\
  Сингулярные базисы, сингулярные числа. Геометрический смысл сингулярного разложения. Нормы векторов, нормы матриц. Малоранговые аппроксимации матриц, теорема Эккарта --- Янга. Метод главных компонент, его связь с сингулярным разложением. Эффективная размерность. Аппроксимация изображений.

  \item \textbf{Проблема мультиколлинеарности} \\
  Число обусловленности матрицы, его геометрическая интерпретация. Проблема мультиколлинеарности, её интерпретация с помощью сингулярного разложения. Методы решения проблемы мультиколлинеарности: метод главных компонент, гребневая регрессия, Лассо Тибширани.

  \item \textbf{Методы оптимизации} \\
  Решение задач условной и безусловной оптимизации. Градиентные методы (алгоритм Бройдена --- Флетчера --- Гольдфарба --- Шанно, алгоритм SLSQP), безградиентные методы (симплекс-метод Нелдера --- Мида, алгоритм COBYLA). Метод штрафных функций.

\end{enumerate}





\end{document}
%%%%%%%%%%%%%%%%%%%%%%%%%%%%%%%%%%%%%%%%%%%%%%%%%%%%%%%%%%%%%%%%%%%%%%%%%%%