%%%%%%%%%%%%%%%%%%%%%%%%%%%%%%%%%%%%%%%%%%%%%%%%%%%%%%%%%%%%%%%%%%%%%%%%%%%
%% Packages
\documentclass[12pt,oneside,openany]{article}
\usepackage{polyglossia}                       % загружает пакет многоязыковой вёрстки
\usepackage{geometry}                          % для последующего задания полей

\usepackage{indentfirst}                       % первый абзац с красной строки
\usepackage[dvipsnames]{xcolor}
%\usepackage{savetrees}
\usepackage{hyperref}                          % гиперссылки

%%%%%%%%%%%%%%%%%%%%%%%%%%%%%%%%%%%%%%%%%%%%%%%%%%%%%%%%%%%%%%%%%%%%%%%%%%%
%% Styles
\geometry{a4paper, top=1cm, bottom=1cm, left=2.5cm, right=1cm, nofoot, nomarginpar} %, heightrounded, showframe

%% Languages
\setmainlanguage[babelshorthands=true]{russian} % устанавливает главный язык документа русский с поддержкой приятных команд пакета babel
\setotherlanguage{english}                     % устанавливает второй язык документа

%% Fonts
\defaultfontfeatures{Ligatures={TeX},Renderer=Basic} %% задаёт свойства шрифтов по умолчанию
\setmainfont{STIX Two Text}                    % задаёт основной шрифт документа
\setsansfont{Tahoma}                           % задаёт шрифт без засечек
\setmonofont{Courier New}                      % задаёт моноширинный шрифт
\newfontfamily\cyrillicfonttt[Script=Cyrillic]{Courier New}
\usepackage{unicode-math}                      % использование Unicode-шрифтов для формул
\setmathfont{STIX Two Math}

%% Выравнивание и переносы
\tolerance 1414
\hbadness 1414
\emergencystretch 1.5em % В случае проблем регулировать в первую очередь
\hfuzz 0.3pt
\vfuzz \hfuzz
%\raggedbottom
%\sloppy                % Избавляемся от переполнений
\clubpenalty=10000      % Запрещаем разрыв страницы после первой строки абзаца
\widowpenalty=10000     % Запрещаем разрыв страницы после последней строки абзаца
\brokenpenalty=4991     % Ограничение на разрыв страницы, если строка заканчивается переносом

\hypersetup{pdfstartview=FitH, colorlinks=true, urlcolor=blue}
%%%%%%%%%%%%%%%%%%%%%%%%%%%%%%%%%%%%%%%%%%%%%%%%%%%%%%%%%%%%%%%%%%%%%%%%%%%
%% Commands
\newcommand*{\todo}[1]{\textcolor{magenta}{\textbf{#1}}}

%%%%%%%%%%%%%%%%%%%%%%%%%%%%%%%%%%%%%%%%%%%%%%%%%%%%%%%%%%%%%%%%%%%%%%%%%%%


\begin{document}

\title{
  \large
  \textbf{Суррогатное моделирование и~оптимизация в~прикладных задачах} \\
  Минимальный набор знаний, осенний семестр
%  (оценка <<удовлетворительно>>)
}

\author{}
\date{}

\maketitle
\thispagestyle{empty}

\vspace{-10ex}


\begin{enumerate}

%  \item[] \textbf{Матрицы и действия над ними}
    \item Ортогональные, симметричные, положительно определённые матрицы
    \item Ранг матрицы, ранговая факторизация

%  \item[] \textbf{Теория систем линейных уравнений}
    \item Алгоритм Гаусса, LU-разложение
    \item LDU-разложение, разложение Холецкого
    \item Нормальная система, нормальное псевдорешение
    \item Псевдообратная матрица: определение, методы вычисления

%  \item \textbf{Ортогональность}
    \item Проекционная матрица, ортогональный проектор
    \item Алгоритм Грама --- Шмидта, QR-разложение

%  \item \textbf{Линейная регрессия}
    \item Метод наименьших квадратов (МНК): формулировка задачи, связь с ортогональным проектированием
    \item Алгоритмы решения задачи МНК: разложение Холецкого, QR-разложение
    \item Рекурсивный МНК, формула Шермана --- Моррисона --- Вудбери

%  \item \textbf{Спектральное разложение матриц}
    \item Собственные векторы и собственные значения
    \item Спектральное разложение
    \item Полярное разложение
    \item QR-алгоритм вычисления собственных чисел

%  \item \textbf{Сингулярное разложение матриц}
    \item Сингулярные базисы и сингулярные числа
    \item Сингулярное разложение
    \item Связь сингулярного и полярного разложений

%  \item \textbf{Главные компоненты}
    \item Нормы векторов и матриц, примеры
    \item Теорема Эккарта --- Янга
    \item Эффективная размерность матрицы
    
%  \item \textbf{Проблема мультиколлинеарности}
    \item Число обусловленности системы, число обусловленности матрицы
    \item Проблема мультиколлинеарности
    \item Решение проблемы мультиколлинеарности: метод главных компонент, гребневая регрессия, лассо Тибширани

\end{enumerate}



\end{document}
%%%%%%%%%%%%%%%%%%%%%%%%%%%%%%%%%%%%%%%%%%%%%%%%%%%%%%%%%%%%%%%%%%%%%%%%%%%