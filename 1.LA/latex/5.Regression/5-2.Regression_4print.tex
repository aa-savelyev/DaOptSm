\documentclass[11pt,a4paper]{article}

    \usepackage[breakable]{tcolorbox}
    \usepackage{parskip} % Stop auto-indenting (to mimic markdown behaviour)
    
    \usepackage{iftex}
    \ifPDFTeX
      \usepackage[T2A]{fontenc}
      \usepackage{mathpazo}
      \usepackage[russian,english]{babel}
    \else
      \usepackage{fontspec}
      \usepackage{polyglossia}
      \setmainlanguage[babelshorthands=true]{russian}    % Язык по-умолчанию русский с поддержкой приятных команд пакета babel
      \setotherlanguage{english}                         % Дополнительный язык = английский (в американской вариации по-умолчанию)
      \newfontfamily\cyrillicfonttt[Scale=0.87,BoldFont={Fira Mono Medium}] {Fira Mono}  % Моноширинный шрифт для кириллицы
      \defaultfontfeatures{Ligatures=TeX}
      \newfontfamily\cyrillicfont{STIX Two Text}         % Шрифт с засечками для кириллицы
    \fi
    \renewcommand{\linethickness}{0.1ex}

    % Basic figure setup, for now with no caption control since it's done
    % automatically by Pandoc (which extracts ![](path) syntax from Markdown).
    \usepackage{graphicx}
    % Maintain compatibility with old templates. Remove in nbconvert 6.0
    \let\Oldincludegraphics\includegraphics
    % Ensure that by default, figures have no caption (until we provide a
    % proper Figure object with a Caption API and a way to capture that
    % in the conversion process - todo).
    \usepackage{caption}
    \DeclareCaptionFormat{nocaption}{}
    \captionsetup{format=nocaption,aboveskip=0pt,belowskip=0pt}

    \usepackage[Export]{adjustbox} % Used to constrain images to a maximum size
    \adjustboxset{max size={0.9\linewidth}{0.9\paperheight}}
    \usepackage{float}
    \floatplacement{figure}{H} % forces figures to be placed at the correct location
    \usepackage{xcolor} % Allow colors to be defined
    \usepackage{enumerate} % Needed for markdown enumerations to work
    \usepackage{geometry} % Used to adjust the document margins
    \usepackage{amsmath} % Equations
    \usepackage{amssymb} % Equations
    \usepackage{textcomp} % defines textquotesingle
    % Hack from http://tex.stackexchange.com/a/47451/13684:
    \AtBeginDocument{%
        \def\PYZsq{\textquotesingle}% Upright quotes in Pygmentized code
    }
    \usepackage{upquote} % Upright quotes for verbatim code
    \usepackage{eurosym} % defines \euro
    \usepackage[mathletters]{ucs} % Extended unicode (utf-8) support
    \usepackage{fancyvrb} % verbatim replacement that allows latex
    \usepackage{grffile} % extends the file name processing of package graphics 
                         % to support a larger range
    \makeatletter % fix for grffile with XeLaTeX
    \def\Gread@@xetex#1{%
      \IfFileExists{"\Gin@base".bb}%
      {\Gread@eps{\Gin@base.bb}}%
      {\Gread@@xetex@aux#1}%
    }
    \makeatother

    % The hyperref package gives us a pdf with properly built
    % internal navigation ('pdf bookmarks' for the table of contents,
    % internal cross-reference links, web links for URLs, etc.)
    \usepackage{hyperref}
    % The default LaTeX title has an obnoxious amount of whitespace. By default,
    % titling removes some of it. It also provides customization options.
    \usepackage{titling}
    \usepackage{longtable} % longtable support required by pandoc >1.10
    \usepackage{booktabs}  % table support for pandoc > 1.12.2
    \usepackage[inline]{enumitem} % IRkernel/repr support (it uses the enumerate* environment)
    \usepackage[normalem]{ulem} % ulem is needed to support strikethroughs (\sout)
                                % normalem makes italics be italics, not underlines
    \usepackage{mathrsfs}
    

    
    % Colors for the hyperref package
    \definecolor{urlcolor}{rgb}{0,.145,.698}
    \definecolor{linkcolor}{rgb}{.71,0.21,0.01}
    \definecolor{citecolor}{rgb}{.12,.54,.11}

    % ANSI colors
    \definecolor{ansi-black}{HTML}{3E424D}
    \definecolor{ansi-black-intense}{HTML}{282C36}
    \definecolor{ansi-red}{HTML}{E75C58}
    \definecolor{ansi-red-intense}{HTML}{B22B31}
    \definecolor{ansi-green}{HTML}{00A250}
    \definecolor{ansi-green-intense}{HTML}{007427}
    \definecolor{ansi-yellow}{HTML}{DDB62B}
    \definecolor{ansi-yellow-intense}{HTML}{B27D12}
    \definecolor{ansi-blue}{HTML}{208FFB}
    \definecolor{ansi-blue-intense}{HTML}{0065CA}
    \definecolor{ansi-magenta}{HTML}{D160C4}
    \definecolor{ansi-magenta-intense}{HTML}{A03196}
    \definecolor{ansi-cyan}{HTML}{60C6C8}
    \definecolor{ansi-cyan-intense}{HTML}{258F8F}
    \definecolor{ansi-white}{HTML}{C5C1B4}
    \definecolor{ansi-white-intense}{HTML}{A1A6B2}
    \definecolor{ansi-default-inverse-fg}{HTML}{FFFFFF}
    \definecolor{ansi-default-inverse-bg}{HTML}{000000}

    % commands and environments needed by pandoc snippets
    % extracted from the output of `pandoc -s`
    \providecommand{\tightlist}{%
      \setlength{\itemsep}{0pt}\setlength{\parskip}{0pt}}
    \DefineVerbatimEnvironment{Highlighting}{Verbatim}{commandchars=\\\{\}}
    % Add ',fontsize=\small' for more characters per line
    \newenvironment{Shaded}{}{}
    \newcommand{\KeywordTok}[1]{\textcolor[rgb]{0.00,0.44,0.13}{\textbf{{#1}}}}
    \newcommand{\DataTypeTok}[1]{\textcolor[rgb]{0.56,0.13,0.00}{{#1}}}
    \newcommand{\DecValTok}[1]{\textcolor[rgb]{0.25,0.63,0.44}{{#1}}}
    \newcommand{\BaseNTok}[1]{\textcolor[rgb]{0.25,0.63,0.44}{{#1}}}
    \newcommand{\FloatTok}[1]{\textcolor[rgb]{0.25,0.63,0.44}{{#1}}}
    \newcommand{\CharTok}[1]{\textcolor[rgb]{0.25,0.44,0.63}{{#1}}}
    \newcommand{\StringTok}[1]{\textcolor[rgb]{0.25,0.44,0.63}{{#1}}}
    \newcommand{\CommentTok}[1]{\textcolor[rgb]{0.38,0.63,0.69}{\textit{{#1}}}}
    \newcommand{\OtherTok}[1]{\textcolor[rgb]{0.00,0.44,0.13}{{#1}}}
    \newcommand{\AlertTok}[1]{\textcolor[rgb]{1.00,0.00,0.00}{\textbf{{#1}}}}
    \newcommand{\FunctionTok}[1]{\textcolor[rgb]{0.02,0.16,0.49}{{#1}}}
    \newcommand{\RegionMarkerTok}[1]{{#1}}
    \newcommand{\ErrorTok}[1]{\textcolor[rgb]{1.00,0.00,0.00}{\textbf{{#1}}}}
    \newcommand{\NormalTok}[1]{{#1}}
    
    % Additional commands for more recent versions of Pandoc
    \newcommand{\ConstantTok}[1]{\textcolor[rgb]{0.53,0.00,0.00}{{#1}}}
    \newcommand{\SpecialCharTok}[1]{\textcolor[rgb]{0.25,0.44,0.63}{{#1}}}
    \newcommand{\VerbatimStringTok}[1]{\textcolor[rgb]{0.25,0.44,0.63}{{#1}}}
    \newcommand{\SpecialStringTok}[1]{\textcolor[rgb]{0.73,0.40,0.53}{{#1}}}
    \newcommand{\ImportTok}[1]{{#1}}
    \newcommand{\DocumentationTok}[1]{\textcolor[rgb]{0.73,0.13,0.13}{\textit{{#1}}}}
    \newcommand{\AnnotationTok}[1]{\textcolor[rgb]{0.38,0.63,0.69}{\textbf{\textit{{#1}}}}}
    \newcommand{\CommentVarTok}[1]{\textcolor[rgb]{0.38,0.63,0.69}{\textbf{\textit{{#1}}}}}
    \newcommand{\VariableTok}[1]{\textcolor[rgb]{0.10,0.09,0.49}{{#1}}}
    \newcommand{\ControlFlowTok}[1]{\textcolor[rgb]{0.00,0.44,0.13}{\textbf{{#1}}}}
    \newcommand{\OperatorTok}[1]{\textcolor[rgb]{0.40,0.40,0.40}{{#1}}}
    \newcommand{\BuiltInTok}[1]{{#1}}
    \newcommand{\ExtensionTok}[1]{{#1}}
    \newcommand{\PreprocessorTok}[1]{\textcolor[rgb]{0.74,0.48,0.00}{{#1}}}
    \newcommand{\AttributeTok}[1]{\textcolor[rgb]{0.49,0.56,0.16}{{#1}}}
    \newcommand{\InformationTok}[1]{\textcolor[rgb]{0.38,0.63,0.69}{\textbf{\textit{{#1}}}}}
    \newcommand{\WarningTok}[1]{\textcolor[rgb]{0.38,0.63,0.69}{\textbf{\textit{{#1}}}}}
    
    
    % Define a nice break command that doesn't care if a line doesn't already
    % exist.
    \def\br{\hspace*{\fill} \\* }
    % Math Jax compatibility definitions
    \def\gt{>}
    \def\lt{<}
    \let\Oldtex\TeX
    \let\Oldlatex\LaTeX
    \renewcommand{\TeX}{\textrm{\Oldtex}}
    \renewcommand{\LaTeX}{\textrm{\Oldlatex}}
    % Document parameters
    % Document title
    \title{Лекция 5-2 \\
    Линейная регрессия
    }
    
    
    
    
    
% Pygments definitions
\makeatletter
\def\PY@reset{\let\PY@it=\relax \let\PY@bf=\relax%
    \let\PY@ul=\relax \let\PY@tc=\relax%
    \let\PY@bc=\relax \let\PY@ff=\relax}
\def\PY@tok#1{\csname PY@tok@#1\endcsname}
\def\PY@toks#1+{\ifx\relax#1\empty\else%
    \PY@tok{#1}\expandafter\PY@toks\fi}
\def\PY@do#1{\PY@bc{\PY@tc{\PY@ul{%
    \PY@it{\PY@bf{\PY@ff{#1}}}}}}}
\def\PY#1#2{\PY@reset\PY@toks#1+\relax+\PY@do{#2}}

\expandafter\def\csname PY@tok@w\endcsname{\def\PY@tc##1{\textcolor[rgb]{0.73,0.73,0.73}{##1}}}
\expandafter\def\csname PY@tok@c\endcsname{\let\PY@it=\textit\def\PY@tc##1{\textcolor[rgb]{0.25,0.50,0.50}{##1}}}
\expandafter\def\csname PY@tok@cp\endcsname{\def\PY@tc##1{\textcolor[rgb]{0.74,0.48,0.00}{##1}}}
\expandafter\def\csname PY@tok@k\endcsname{\let\PY@bf=\textbf\def\PY@tc##1{\textcolor[rgb]{0.00,0.50,0.00}{##1}}}
\expandafter\def\csname PY@tok@kp\endcsname{\def\PY@tc##1{\textcolor[rgb]{0.00,0.50,0.00}{##1}}}
\expandafter\def\csname PY@tok@kt\endcsname{\def\PY@tc##1{\textcolor[rgb]{0.69,0.00,0.25}{##1}}}
\expandafter\def\csname PY@tok@o\endcsname{\def\PY@tc##1{\textcolor[rgb]{0.40,0.40,0.40}{##1}}}
\expandafter\def\csname PY@tok@ow\endcsname{\let\PY@bf=\textbf\def\PY@tc##1{\textcolor[rgb]{0.67,0.13,1.00}{##1}}}
\expandafter\def\csname PY@tok@nb\endcsname{\def\PY@tc##1{\textcolor[rgb]{0.00,0.50,0.00}{##1}}}
\expandafter\def\csname PY@tok@nf\endcsname{\def\PY@tc##1{\textcolor[rgb]{0.00,0.00,1.00}{##1}}}
\expandafter\def\csname PY@tok@nc\endcsname{\let\PY@bf=\textbf\def\PY@tc##1{\textcolor[rgb]{0.00,0.00,1.00}{##1}}}
\expandafter\def\csname PY@tok@nn\endcsname{\let\PY@bf=\textbf\def\PY@tc##1{\textcolor[rgb]{0.00,0.00,1.00}{##1}}}
\expandafter\def\csname PY@tok@ne\endcsname{\let\PY@bf=\textbf\def\PY@tc##1{\textcolor[rgb]{0.82,0.25,0.23}{##1}}}
\expandafter\def\csname PY@tok@nv\endcsname{\def\PY@tc##1{\textcolor[rgb]{0.10,0.09,0.49}{##1}}}
\expandafter\def\csname PY@tok@no\endcsname{\def\PY@tc##1{\textcolor[rgb]{0.53,0.00,0.00}{##1}}}
\expandafter\def\csname PY@tok@nl\endcsname{\def\PY@tc##1{\textcolor[rgb]{0.63,0.63,0.00}{##1}}}
\expandafter\def\csname PY@tok@ni\endcsname{\let\PY@bf=\textbf\def\PY@tc##1{\textcolor[rgb]{0.60,0.60,0.60}{##1}}}
\expandafter\def\csname PY@tok@na\endcsname{\def\PY@tc##1{\textcolor[rgb]{0.49,0.56,0.16}{##1}}}
\expandafter\def\csname PY@tok@nt\endcsname{\let\PY@bf=\textbf\def\PY@tc##1{\textcolor[rgb]{0.00,0.50,0.00}{##1}}}
\expandafter\def\csname PY@tok@nd\endcsname{\def\PY@tc##1{\textcolor[rgb]{0.67,0.13,1.00}{##1}}}
\expandafter\def\csname PY@tok@s\endcsname{\def\PY@tc##1{\textcolor[rgb]{0.73,0.13,0.13}{##1}}}
\expandafter\def\csname PY@tok@sd\endcsname{\let\PY@it=\textit\def\PY@tc##1{\textcolor[rgb]{0.73,0.13,0.13}{##1}}}
\expandafter\def\csname PY@tok@si\endcsname{\let\PY@bf=\textbf\def\PY@tc##1{\textcolor[rgb]{0.73,0.40,0.53}{##1}}}
\expandafter\def\csname PY@tok@se\endcsname{\let\PY@bf=\textbf\def\PY@tc##1{\textcolor[rgb]{0.73,0.40,0.13}{##1}}}
\expandafter\def\csname PY@tok@sr\endcsname{\def\PY@tc##1{\textcolor[rgb]{0.73,0.40,0.53}{##1}}}
\expandafter\def\csname PY@tok@ss\endcsname{\def\PY@tc##1{\textcolor[rgb]{0.10,0.09,0.49}{##1}}}
\expandafter\def\csname PY@tok@sx\endcsname{\def\PY@tc##1{\textcolor[rgb]{0.00,0.50,0.00}{##1}}}
\expandafter\def\csname PY@tok@m\endcsname{\def\PY@tc##1{\textcolor[rgb]{0.40,0.40,0.40}{##1}}}
\expandafter\def\csname PY@tok@gh\endcsname{\let\PY@bf=\textbf\def\PY@tc##1{\textcolor[rgb]{0.00,0.00,0.50}{##1}}}
\expandafter\def\csname PY@tok@gu\endcsname{\let\PY@bf=\textbf\def\PY@tc##1{\textcolor[rgb]{0.50,0.00,0.50}{##1}}}
\expandafter\def\csname PY@tok@gd\endcsname{\def\PY@tc##1{\textcolor[rgb]{0.63,0.00,0.00}{##1}}}
\expandafter\def\csname PY@tok@gi\endcsname{\def\PY@tc##1{\textcolor[rgb]{0.00,0.63,0.00}{##1}}}
\expandafter\def\csname PY@tok@gr\endcsname{\def\PY@tc##1{\textcolor[rgb]{1.00,0.00,0.00}{##1}}}
\expandafter\def\csname PY@tok@ge\endcsname{\let\PY@it=\textit}
\expandafter\def\csname PY@tok@gs\endcsname{\let\PY@bf=\textbf}
\expandafter\def\csname PY@tok@gp\endcsname{\let\PY@bf=\textbf\def\PY@tc##1{\textcolor[rgb]{0.00,0.00,0.50}{##1}}}
\expandafter\def\csname PY@tok@go\endcsname{\def\PY@tc##1{\textcolor[rgb]{0.53,0.53,0.53}{##1}}}
\expandafter\def\csname PY@tok@gt\endcsname{\def\PY@tc##1{\textcolor[rgb]{0.00,0.27,0.87}{##1}}}
\expandafter\def\csname PY@tok@err\endcsname{\def\PY@bc##1{\setlength{\fboxsep}{0pt}\fcolorbox[rgb]{1.00,0.00,0.00}{1,1,1}{\strut ##1}}}
\expandafter\def\csname PY@tok@kc\endcsname{\let\PY@bf=\textbf\def\PY@tc##1{\textcolor[rgb]{0.00,0.50,0.00}{##1}}}
\expandafter\def\csname PY@tok@kd\endcsname{\let\PY@bf=\textbf\def\PY@tc##1{\textcolor[rgb]{0.00,0.50,0.00}{##1}}}
\expandafter\def\csname PY@tok@kn\endcsname{\let\PY@bf=\textbf\def\PY@tc##1{\textcolor[rgb]{0.00,0.50,0.00}{##1}}}
\expandafter\def\csname PY@tok@kr\endcsname{\let\PY@bf=\textbf\def\PY@tc##1{\textcolor[rgb]{0.00,0.50,0.00}{##1}}}
\expandafter\def\csname PY@tok@bp\endcsname{\def\PY@tc##1{\textcolor[rgb]{0.00,0.50,0.00}{##1}}}
\expandafter\def\csname PY@tok@fm\endcsname{\def\PY@tc##1{\textcolor[rgb]{0.00,0.00,1.00}{##1}}}
\expandafter\def\csname PY@tok@vc\endcsname{\def\PY@tc##1{\textcolor[rgb]{0.10,0.09,0.49}{##1}}}
\expandafter\def\csname PY@tok@vg\endcsname{\def\PY@tc##1{\textcolor[rgb]{0.10,0.09,0.49}{##1}}}
\expandafter\def\csname PY@tok@vi\endcsname{\def\PY@tc##1{\textcolor[rgb]{0.10,0.09,0.49}{##1}}}
\expandafter\def\csname PY@tok@vm\endcsname{\def\PY@tc##1{\textcolor[rgb]{0.10,0.09,0.49}{##1}}}
\expandafter\def\csname PY@tok@sa\endcsname{\def\PY@tc##1{\textcolor[rgb]{0.73,0.13,0.13}{##1}}}
\expandafter\def\csname PY@tok@sb\endcsname{\def\PY@tc##1{\textcolor[rgb]{0.73,0.13,0.13}{##1}}}
\expandafter\def\csname PY@tok@sc\endcsname{\def\PY@tc##1{\textcolor[rgb]{0.73,0.13,0.13}{##1}}}
\expandafter\def\csname PY@tok@dl\endcsname{\def\PY@tc##1{\textcolor[rgb]{0.73,0.13,0.13}{##1}}}
\expandafter\def\csname PY@tok@s2\endcsname{\def\PY@tc##1{\textcolor[rgb]{0.73,0.13,0.13}{##1}}}
\expandafter\def\csname PY@tok@sh\endcsname{\def\PY@tc##1{\textcolor[rgb]{0.73,0.13,0.13}{##1}}}
\expandafter\def\csname PY@tok@s1\endcsname{\def\PY@tc##1{\textcolor[rgb]{0.73,0.13,0.13}{##1}}}
\expandafter\def\csname PY@tok@mb\endcsname{\def\PY@tc##1{\textcolor[rgb]{0.40,0.40,0.40}{##1}}}
\expandafter\def\csname PY@tok@mf\endcsname{\def\PY@tc##1{\textcolor[rgb]{0.40,0.40,0.40}{##1}}}
\expandafter\def\csname PY@tok@mh\endcsname{\def\PY@tc##1{\textcolor[rgb]{0.40,0.40,0.40}{##1}}}
\expandafter\def\csname PY@tok@mi\endcsname{\def\PY@tc##1{\textcolor[rgb]{0.40,0.40,0.40}{##1}}}
\expandafter\def\csname PY@tok@il\endcsname{\def\PY@tc##1{\textcolor[rgb]{0.40,0.40,0.40}{##1}}}
\expandafter\def\csname PY@tok@mo\endcsname{\def\PY@tc##1{\textcolor[rgb]{0.40,0.40,0.40}{##1}}}
\expandafter\def\csname PY@tok@ch\endcsname{\let\PY@it=\textit\def\PY@tc##1{\textcolor[rgb]{0.25,0.50,0.50}{##1}}}
\expandafter\def\csname PY@tok@cm\endcsname{\let\PY@it=\textit\def\PY@tc##1{\textcolor[rgb]{0.25,0.50,0.50}{##1}}}
\expandafter\def\csname PY@tok@cpf\endcsname{\let\PY@it=\textit\def\PY@tc##1{\textcolor[rgb]{0.25,0.50,0.50}{##1}}}
\expandafter\def\csname PY@tok@c1\endcsname{\let\PY@it=\textit\def\PY@tc##1{\textcolor[rgb]{0.25,0.50,0.50}{##1}}}
\expandafter\def\csname PY@tok@cs\endcsname{\let\PY@it=\textit\def\PY@tc##1{\textcolor[rgb]{0.25,0.50,0.50}{##1}}}

\def\PYZbs{\char`\\}
\def\PYZus{\char`\_}
\def\PYZob{\char`\{}
\def\PYZcb{\char`\}}
\def\PYZca{\char`\^}
\def\PYZam{\char`\&}
\def\PYZlt{\char`\<}
\def\PYZgt{\char`\>}
\def\PYZsh{\char`\#}
\def\PYZpc{\char`\%}
\def\PYZdl{\char`\$}
\def\PYZhy{\char`\-}
\def\PYZsq{\char`\'}
\def\PYZdq{\char`\"}
\def\PYZti{\char`\~}
% for compatibility with earlier versions
\def\PYZat{@}
\def\PYZlb{[}
\def\PYZrb{]}
\makeatother


    % For linebreaks inside Verbatim environment from package fancyvrb. 
    \makeatletter
        \newbox\Wrappedcontinuationbox 
        \newbox\Wrappedvisiblespacebox 
        \newcommand*\Wrappedvisiblespace {\textcolor{red}{\textvisiblespace}} 
        \newcommand*\Wrappedcontinuationsymbol {\textcolor{red}{\llap{\tiny$\m@th\hookrightarrow$}}} 
        \newcommand*\Wrappedcontinuationindent {3ex } 
        \newcommand*\Wrappedafterbreak {\kern\Wrappedcontinuationindent\copy\Wrappedcontinuationbox} 
        % Take advantage of the already applied Pygments mark-up to insert 
        % potential linebreaks for TeX processing. 
        %        {, <, #, %, $, ' and ": go to next line. 
        %        _, }, ^, &, >, - and ~: stay at end of broken line. 
        % Use of \textquotesingle for straight quote. 
        \newcommand*\Wrappedbreaksatspecials {% 
            \def\PYGZus{\discretionary{\char`\_}{\Wrappedafterbreak}{\char`\_}}% 
            \def\PYGZob{\discretionary{}{\Wrappedafterbreak\char`\{}{\char`\{}}% 
            \def\PYGZcb{\discretionary{\char`\}}{\Wrappedafterbreak}{\char`\}}}% 
            \def\PYGZca{\discretionary{\char`\^}{\Wrappedafterbreak}{\char`\^}}% 
            \def\PYGZam{\discretionary{\char`\&}{\Wrappedafterbreak}{\char`\&}}% 
            \def\PYGZlt{\discretionary{}{\Wrappedafterbreak\char`\<}{\char`\<}}% 
            \def\PYGZgt{\discretionary{\char`\>}{\Wrappedafterbreak}{\char`\>}}% 
            \def\PYGZsh{\discretionary{}{\Wrappedafterbreak\char`\#}{\char`\#}}% 
            \def\PYGZpc{\discretionary{}{\Wrappedafterbreak\char`\%}{\char`\%}}% 
            \def\PYGZdl{\discretionary{}{\Wrappedafterbreak\char`\$}{\char`\$}}% 
            \def\PYGZhy{\discretionary{\char`\-}{\Wrappedafterbreak}{\char`\-}}% 
            \def\PYGZsq{\discretionary{}{\Wrappedafterbreak\textquotesingle}{\textquotesingle}}% 
            \def\PYGZdq{\discretionary{}{\Wrappedafterbreak\char`\"}{\char`\"}}% 
            \def\PYGZti{\discretionary{\char`\~}{\Wrappedafterbreak}{\char`\~}}% 
        } 
        % Some characters . , ; ? ! / are not pygmentized. 
        % This macro makes them "active" and they will insert potential linebreaks 
        \newcommand*\Wrappedbreaksatpunct {% 
            \lccode`\~`\.\lowercase{\def~}{\discretionary{\hbox{\char`\.}}{\Wrappedafterbreak}{\hbox{\char`\.}}}% 
            \lccode`\~`\,\lowercase{\def~}{\discretionary{\hbox{\char`\,}}{\Wrappedafterbreak}{\hbox{\char`\,}}}% 
            \lccode`\~`\;\lowercase{\def~}{\discretionary{\hbox{\char`\;}}{\Wrappedafterbreak}{\hbox{\char`\;}}}% 
            \lccode`\~`\:\lowercase{\def~}{\discretionary{\hbox{\char`\:}}{\Wrappedafterbreak}{\hbox{\char`\:}}}% 
            \lccode`\~`\?\lowercase{\def~}{\discretionary{\hbox{\char`\?}}{\Wrappedafterbreak}{\hbox{\char`\?}}}% 
            \lccode`\~`\!\lowercase{\def~}{\discretionary{\hbox{\char`\!}}{\Wrappedafterbreak}{\hbox{\char`\!}}}% 
            \lccode`\~`\/\lowercase{\def~}{\discretionary{\hbox{\char`\/}}{\Wrappedafterbreak}{\hbox{\char`\/}}}% 
            \catcode`\.\active
            \catcode`\,\active 
            \catcode`\;\active
            \catcode`\:\active
            \catcode`\?\active
            \catcode`\!\active
            \catcode`\/\active 
            \lccode`\~`\~ 	
        }
    \makeatother

    \let\OriginalVerbatim=\Verbatim
    \makeatletter
    \renewcommand{\Verbatim}[1][1]{%
        %\parskip\z@skip
        \sbox\Wrappedcontinuationbox {\Wrappedcontinuationsymbol}%
        \sbox\Wrappedvisiblespacebox {\FV@SetupFont\Wrappedvisiblespace}%
        \def\FancyVerbFormatLine ##1{\hsize\linewidth
            \vtop{\raggedright\hyphenpenalty\z@\exhyphenpenalty\z@
                \doublehyphendemerits\z@\finalhyphendemerits\z@
                \strut ##1\strut}%
        }%
        % If the linebreak is at a space, the latter will be displayed as visible
        % space at end of first line, and a continuation symbol starts next line.
        % Stretch/shrink are however usually zero for typewriter font.
        \def\FV@Space {%
            \nobreak\hskip\z@ plus\fontdimen3\font minus\fontdimen4\font
            \discretionary{\copy\Wrappedvisiblespacebox}{\Wrappedafterbreak}
            {\kern\fontdimen2\font}%
        }%
        
        % Allow breaks at special characters using \PYG... macros.
        \Wrappedbreaksatspecials
        % Breaks at punctuation characters . , ; ? ! and / need catcode=\active 	
        \OriginalVerbatim[#1,codes*=\Wrappedbreaksatpunct]%
    }
    \makeatother

    % Exact colors from NB
    \definecolor{incolor}{HTML}{303F9F}
    \definecolor{outcolor}{HTML}{D84315}
    \definecolor{cellborder}{HTML}{CFCFCF}
    \definecolor{cellbackground}{HTML}{F7F7F7}
    
    % prompt
    \makeatletter
    \newcommand{\boxspacing}{\kern\kvtcb@left@rule\kern\kvtcb@boxsep}
    \makeatother
    \newcommand{\prompt}[4]{
        \ttfamily\llap{{\color{#2}[#3]:\hspace{3pt}#4}}\vspace{-\baselineskip}
    }
    

    
    % Prevent overflowing lines due to hard-to-break entities
    \sloppy 
    % Setup hyperref package
    \hypersetup{
      breaklinks=true,  % so long urls are correctly broken across lines
      colorlinks=true,
      urlcolor=urlcolor,
      linkcolor=linkcolor,
      citecolor=citecolor,
      }
    % Slightly bigger margins than the latex defaults
    
    \geometry{verbose,tmargin=1in,bmargin=1in,lmargin=1in,rmargin=1in}
    
    

\begin{document}
    
    \maketitle

    \hypertarget{ux43eux431ux43eux437ux43dux430ux447ux435ux43dux438ux44f}{%
\section{Обозначения}\label{ux43eux431ux43eux437ux43dux430ux447ux435ux43dux438ux44f}}

Задачу обучения по прецедентам при \(Y = \mathbb{R}\) принято называть
задачей \emph{восстановления регрессии}. Введём основные обозначения.

Задано пространство объектов \(X\) и множество ответов \(Y\). Мы
предполагаем существование зависимости \(y^*:X \rightarrow Y\), значения
которой известны только на объектах обучающей выборки
\(X^m = (x_i, y_i)_{i=1}^m\).

Требуется построить алгоритм («\emph{функцию регрессии}»)
\(a: X \rightarrow Y\), аппроксимирующий целевую зависимость \(y^*\).

\begin{itemize}
\tightlist
\item
  \(X\) --- объекты; \(Y\) --- ответы;
\item
  \(X^m = (x_i, y_i)_{i=1}^m\) --- обучающая выборка;
\item
  \(y_i = y^*(x_i), y^*:X \rightarrow Y\) --- неизвестная зависимость.
\end{itemize}

Общее количество \emph{объектов} \(m\), для их индексации используется
буква \(i\).\\
Общее количество \emph{признаков} \(n\), для их индексации используется
буква \(j\).\\
Матрица объекты--признаки имеет размерность \(m \times n\): \[
  \mathbf{F} = 
  \begin{pmatrix}
    f_1(x_1) & \ldots & f_n(x_1) \\
    \vdots   & \ddots & \vdots   \\
    f_1(x_m) & \ldots & f_n(x_m) \\
  \end{pmatrix}
\].

    \begin{center}\rule{0.5\linewidth}{\linethickness}\end{center}

    \hypertarget{ux433ux435ux43dux435ux440ux430ux446ux438ux44f-ux434ux430ux43dux43dux44bux445}{%
\section{Генерация
данных}\label{ux433ux435ux43dux435ux440ux430ux446ux438ux44f-ux434ux430ux43dux43dux44bux445}}

В качестве обучающих данных будем использовать зашумлённую линейную
зависимость между \(y\) и \(x\). Для выборки данных размером \(m\)
предполагаемая зависимость может быть смоделирована следующим образом:

\[ y_i = \theta_0 + \theta_1 x_i + \epsilon_i  \quad (i = 1, \ldots, m). \]

Здесь: - \(x_i\) --- независимая (входная) переменная выборки \(i\), с
\(x = \{x_i \ldots x_m \}\); - \(y_i\) --- зависимая (выходная)
переменная выборки \(i\), с \(y = \{y_i \ldots y_m \}\); -
\(\epsilon_i \sim \mathcal{N}(0, \sigma^2)\) --- нормальный шум,
влияющий на выходной сигнал \(y_i\); -
\(\theta = \{\theta_0, \theta_1 \}\) --- набор параметров: смещение
\(\theta_0\) и наклон \(\theta_1\).

    \begin{center}
    \adjustimage{max size={0.75\linewidth}{0.9\paperheight}}{output_7_0.png}
    \end{center}
    { \hspace*{\fill} \\}
    
    \begin{center}\rule{0.5\linewidth}{\linethickness}\end{center}

    \hypertarget{ux43cux435ux442ux43eux434-ux43dux430ux438ux43cux435ux43dux44cux448ux438ux445-ux43aux432ux430ux434ux440ux430ux442ux43eux432}{%
\section{Метод наименьших
квадратов}\label{ux43cux435ux442ux43eux434-ux43dux430ux438ux43cux435ux43dux44cux448ux438ux445-ux43aux432ux430ux434ux440ux430ux442ux43eux432}}

Пусть задана \emph{модель регрессии} --- параметрическое семейство
функций \(g(x,\alpha)\), где \(\alpha \in \mathbb{R}^p\) --- вектор
параметров модели.

Определим функционал качества аппроксимации целевой зависимости на
выборке \(X^m\) как сумму квадратов ошибок:
\[ Q(\alpha, X^m) = \sum_{i=1}^m \left( g(x_i, \alpha) - y_i \right)^2. \]

Обучение по \emph{методу наименьших квадратов} (МНК) состоит в том,
чтобы найти вектор параметров \(\alpha^*\), при котором достигается
минимум среднего квадрата ошибки на заданной обучающей выборке \(X^m\):

\[ \alpha^* = \underset{\alpha \in \mathbb{R}^p}{\mathrm{argmin}} \, {Q(\alpha, X^m)}. \]

Стандартный способ решения этой оптимизационной задачи ---
воспользоваться необходимым условием минимума. Если функция
\(g(x, \alpha)\) достаточное число раз дифференцируема по \(\alpha\), то
в точке минимума выполняется система \(p\) уравнений относительно \(p\)
неизвестных:

\[ \frac{\delta}{\delta \alpha} Q(\alpha, X^m) = 2 \sum_{i=1}^{m} \left( g(x_i, \alpha) -y_i \right) \frac{\delta}{\delta \alpha} g(x_i, \alpha) = 0. \]

    \begin{center}\rule{0.5\linewidth}{\linethickness}\end{center}

    \hypertarget{ux43bux438ux43dux435ux439ux43dux430ux44f-ux440ux435ux433ux440ux435ux441ux441ux438ux44f}{%
\section{Линейная
регрессия}\label{ux43bux438ux43dux435ux439ux43dux430ux44f-ux440ux435ux433ux440ux435ux441ux441ux438ux44f}}

\hypertarget{ux444ux43eux440ux43cux443ux43bux438ux440ux43eux432ux43aux430-ux437ux430ux434ux430ux447ux438}{%
\subsection{Формулировка
задачи}\label{ux444ux43eux440ux43cux443ux43bux438ux440ux43eux432ux43aux430-ux437ux430ux434ux430ux447ux438}}

\begin{quote}
Линейная регрессия является одной из самых простых моделей машинного
обучения. Есть мнение, что её даже не следует классифицировать как
«машинное обучение», потому что она слишком простая. Тем не менее,
простота делает её прекрасной отправной точкой для понимания более
сложных методов.
\end{quote}

Пусть каждому объекту соответствует его признаковое описание
\(\left( f_1(x), \ldots, f_n(x)\right)\), где
\(f_j: X \rightarrow \mathbb{R}\) --- числовые признаки,
\(j = 1, \ldots , n\). Линейной моделью регрессии называется линейная
комбинация признаков с коэффициентами \(\alpha \in \mathbb{R}^n\):
\[ g(x, \alpha) = \sum_{j=1}^n \alpha_j f_j(x). \]

Введём матричные обозначения:
\(F = \left( f_j(x_i) \right)_{m \times n}\) --- матрица
объекты--признаки; \(y = \left( y_i \right)_{m \times 1}\) --- целевой
вектор; \(\alpha = \left( \alpha_i \right)_{n \times 1}\) --- вектор
параметров.

Применим метод наименьших квадратов к нашей линейной модели.

В матричных обозначениях функционал качества \(Q\) принимает вид
\[ Q(\alpha) = \left\Vert F\alpha - y \right\Vert^2. \]

Тогда задача поиска параметров регрессии может быть сформулирована так:
\[ \alpha^* = \underset{\alpha \in \mathbb{R}^n}{\text{argmin}} \, {\left\Vert F\alpha - y \right\Vert^2}. \]

    \hypertarget{ux440ux435ux448ux435ux43dux438ux435}{%
\subsection{Решение}\label{ux440ux435ux448ux435ux43dux438ux435}}

Запишем функционал качества в матричном виде:
\[ Q(\alpha) = (F\alpha - y)^\top (F\alpha - y) \]

и выпишем необходимое условие минимума:
\[ \frac{\delta Q(\alpha)}{\delta \alpha} = 2F^\top (F\alpha - y) = 0. \]

Отсюда следует \(F^{\top} F \alpha = F^{\top}y\). Эта система линейных
уравнений относительно \(\alpha\) называется \emph{нормальной системой}
для задачи наименьших квадратов.

Если матрица \(F^{\top} F\) невырождена (для этого столбцы матрицы \(F\)
олжны быть линейно независимы), то решением нормальной системы является
вектор

\[ \alpha^* = (F^{\top} F)^{-1} F^{\top} y = F^{+} y. \]

Матрица \(F^{+} = (F^{\top} F)^{-1} F^{\top}\) является
\emph{псевдообратной} для прямоугольной матрицы \(F\).

Подставляя найденное решение в исходный функционал, получаем

\[ Q(\alpha^*) = \left\Vert P_Fy - y \right\Vert^2, \]

где \(P_F = FF^{+} = F(F^{\top}F)^{-1}F^{\top}\) --- проекционная
матрица.

Решение имеет простую геометрическую интерпретацию. Произведение
\(P_Fy\) есть проекция целевого вектора \(y\) на линейную оболочку
столбцов матрицы \(F\). Разность \((P_Fy-y)\) есть проекция целевого
вектора \(y\) на ортогональное дополнение этой линейной оболочки.
Значение функционала \(Q(\alpha^*) = \left\Vert P_Fy - y \right\Vert^2\)
есть квадрат длины перпендикуляра, опущенного из \(y\) на линейную
оболочку. Таким образом, МНК находит кратчайшее расстояние от \(y\) до
линейной оболочки столбцов F.

    \hypertarget{ux43fux440ux438ux43cux435ux440.-ux43fux43eux43bux438ux43dux43eux43cux438ux430ux43bux44cux43dux430ux44f-ux440ux435ux433ux440ux435ux441ux441ux438ux44f}{%
\subsection{Пример. Полиномиальная
регрессия}\label{ux43fux440ux438ux43cux435ux440.-ux43fux43eux43bux438ux43dux43eux43cux438ux430ux43bux44cux43dux430ux44f-ux440ux435ux433ux440ux435ux441ux441ux438ux44f}}

Рассмотрим частный случай линейной регрессии --- полиномиальную
регрессию.

Тогда линейная модель
\[ g(x, \alpha) = \sum_{j=0}^{n-1} \alpha_j p_j(x).\]

Здесь \(p(x) = \{1, x, \ldots, x^{n-1}\}\) --- набор базисных полиномов,
\(\alpha = \{\alpha_0, \ldots, \alpha_{n-1}\}\) --- набор искомых
параметров.

Матрица объекты--признаки в этом случае выглядит так: \[
  \mathbf{F} = 
  \begin{pmatrix}
    1      & x_1    & \ldots & x_1^{n-1} \\
    \vdots & \vdots & \ddots & \vdots    \\
    1      & x_m    & \ldots & x_m^{n-1} \\
  \end{pmatrix}.
\]

Пример построения полиномиальной регрессии для наших данных приведён
ниже.

    \begin{center}
    \adjustimage{max size={0.75\linewidth}{0.9\paperheight}}{output_15_0.png}
    \end{center}
    { \hspace*{\fill} \\}
    
    \hypertarget{ux430ux43bux433ux43eux440ux438ux442ux43cux44b-ux440ux435ux448ux435ux43dux438ux44f-ux437ux430ux434ux430ux447ux438-ux43cux43dux43a}{%
\section{Алгоритмы решения задачи
МНК}\label{ux430ux43bux433ux43eux440ux438ux442ux43cux44b-ux440ux435ux448ux435ux43dux438ux44f-ux437ux430ux434ux430ux447ux438-ux43cux43dux43a}}

Теперь попробуем в вычислительную математику.

\hypertarget{ux440ux430ux437ux43bux43eux436ux435ux43dux438ux435-ux445ux43eux43bux435ux446ux43aux43eux433ux43e}{%
\subsection{Разложение
Холецкого}\label{ux440ux430ux437ux43bux43eux436ux435ux43dux438ux435-ux445ux43eux43bux435ux446ux43aux43eux433ux43e}}

Матрица \(F\) должна быть полного ранга.\\
Так как матрица \(F^\top F\) симметрическая и положительно определённая,
её можно разложить по Холецкому.
Дальше нужно последовательно решить две системы линейных уравнений с~треугольными матрицами.

    \begin{tcolorbox}[breakable, size=fbox, boxrule=1pt, pad at break*=1mm,colback=cellbackground, colframe=cellborder]
\prompt{In}{incolor}{8}{\boxspacing}
\begin{Verbatim}[commandchars=\\\{\}]
\PY{c+c1}{\PYZsh{} 0. Set up the problem}
\PY{n}{FtF} \PY{o}{=} \PY{n}{F}\PY{o}{.}\PY{n}{T} \PY{o}{@} \PY{n}{F}
\PY{n}{Fty} \PY{o}{=} \PY{n}{F}\PY{o}{.}\PY{n}{T} \PY{o}{@} \PY{n}{Y\PYZus{}train}

\PY{c+c1}{\PYZsh{} 1. Compute Cholesky factorization of FtF}
\PY{n}{L} \PY{o}{=} \PY{n}{LA}\PY{o}{.}\PY{n}{cholesky}\PY{p}{(}\PY{n}{FtF}\PY{p}{)}

\PY{c+c1}{\PYZsh{} 2. Solve the lower triangular system L*w = Ft*y for w}
\PY{n}{w} \PY{o}{=} \PY{n}{LA}\PY{o}{.}\PY{n}{inv}\PY{p}{(}\PY{n}{L}\PY{p}{)} \PY{o}{@} \PY{n}{Fty}
\PY{c+c1}{\PYZsh{} w = LA.solve(L, Fty)}

\PY{c+c1}{\PYZsh{} 3. Solve the upper triangular system Lt*x = w for x}
\PY{n}{x} \PY{o}{=} \PY{n}{LA}\PY{o}{.}\PY{n}{inv}\PY{p}{(}\PY{n}{L}\PY{o}{.}\PY{n}{T}\PY{p}{)} \PY{o}{@} \PY{n}{w}
\PY{n}{display}\PY{p}{(}\PY{n}{x}\PY{p}{)}
\end{Verbatim}
\end{tcolorbox}

    
    \begin{verbatim}
array([1.86789431, 3.10047182])
    \end{verbatim}

    
    Алгоритм требует \(O(mn^2 + \frac{1}{3}n^3)\) операций.\\
Решение нормальной системы уравнений, возможно, будет неустойчивым,
поэтому метод рекомендуется только для небольших задач.\\
В целом метод \emph{не рекомендуется}.

    \hypertarget{qr-ux440ux430ux437ux43bux43eux436ux435ux43dux438ux435}{%
\subsection{QR-разложение}\label{qr-ux440ux430ux437ux43bux43eux436ux435ux43dux438ux435}}

Матрица \(F\) должна быть полного ранга.\\
Здесь применяется обобщение \(QR\)-разложения (reduced QR
factorization), в котором матрица \(Q\) размеров \(m \times n\)
составлена из \(n\) ортонормированных столбцов, а \(R\) --- квадратная
верхняя треугольная матрица порядка \(n\) (\(qR\)-разложение).

    \begin{tcolorbox}[breakable, size=fbox, boxrule=1pt, pad at break*=1mm,colback=cellbackground, colframe=cellborder]
\prompt{In}{incolor}{9}{\boxspacing}
\begin{Verbatim}[commandchars=\\\{\}]
\PY{c+c1}{\PYZsh{} 1. Compute reduced QR factorization of F}
\PY{n}{Q}\PY{p}{,} \PY{n}{R} \PY{o}{=} \PY{n}{LA}\PY{o}{.}\PY{n}{qr}\PY{p}{(}\PY{n}{F}\PY{p}{)}

\PY{c+c1}{\PYZsh{} 2. Solve the upper triangular system R*x = Qt*y for x}
\PY{n}{x} \PY{o}{=} \PY{n}{LA}\PY{o}{.}\PY{n}{inv}\PY{p}{(}\PY{n}{R}\PY{p}{)} \PY{o}{@} \PY{n}{Q}\PY{o}{.}\PY{n}{T} \PY{o}{@} \PY{n}{Y\PYZus{}train}
\PY{n}{display}\PY{p}{(}\PY{n}{x}\PY{p}{)}
\end{Verbatim}
\end{tcolorbox}

    
    \begin{verbatim}
array([1.86789431, 3.10047182])
    \end{verbatim}

    
    Алгоритм требует \(O(2mn^2 - \frac{2}{3}n^3)\) операций.\\
По сравнению с алгоритмом Холецкого \(QR\)-алгоритм более устойчив и может
рассматриваться как \emph{стандартный метод} решения задачи МНК.

    \begin{center}\rule{0.5\linewidth}{\linethickness}\end{center}


    % Add a bibliography block to the postdoc
    
    
    
\end{document}
