\documentclass[11pt,a4paper]{article}

    \usepackage[breakable]{tcolorbox}
    \usepackage{parskip} % Stop auto-indenting (to mimic markdown behaviour)

    \usepackage{iftex}
    \ifPDFTeX
      \usepackage[T2A]{fontenc}
      \usepackage{mathpazo}
      \usepackage[russian,english]{babel}
    \else
      \usepackage{fontspec}
      \usepackage{polyglossia}
      \setmainlanguage[babelshorthands=true]{russian}    % Язык по-умолчанию русский с поддержкой приятных команд пакета babel
      \setotherlanguage{english}                         % Дополнительный язык = английский (в американской вариации по-умолчанию)

      \defaultfontfeatures{Ligatures=TeX}
      \setmainfont[BoldFont={STIX Two Text SemiBold}]%
      {STIX Two Text}                                    % Шрифт с засечками
      \newfontfamily\cyrillicfont[BoldFont={STIX Two Text SemiBold}]%
      {STIX Two Text}                                    % Шрифт с засечками для кириллицы
      \setsansfont{Fira Sans}                            % Шрифт без засечек
      \newfontfamily\cyrillicfontsf{Fira Sans}           % Шрифт без засечек для кириллицы
      \setmonofont[Scale=0.87,BoldFont={Fira Mono Medium},ItalicFont=[FiraMono-Oblique]]%
      {Fira Mono}%                                       % Моноширинный шрифт
      \newfontfamily\cyrillicfonttt[Scale=0.87,BoldFont={Fira Mono Medium},ItalicFont=[FiraMono-Oblique]]%
      {Fira Mono}                                        % Моноширинный шрифт для кириллицы

      %%% Математические пакеты %%%
      \usepackage{amsthm,amsmath,amscd}   % Математические дополнения от AMS
      \usepackage{amsfonts,amssymb}       % Математические дополнения от AMS
      \usepackage{mathtools}              % Добавляет окружение multlined
      \usepackage{unicode-math}           % Для шрифта STIX Two Math
      \setmathfont{STIX Two Math}         % Математический шрифт
    \fi

    % Basic figure setup, for now with no caption control since it's done
    % automatically by Pandoc (which extracts ![](path) syntax from Markdown).
    \usepackage{graphicx}
    % Maintain compatibility with old templates. Remove in nbconvert 6.0
    \let\Oldincludegraphics\includegraphics
    % Ensure that by default, figures have no caption (until we provide a
    % proper Figure object with a Caption API and a way to capture that
    % in the conversion process - todo).
    \usepackage{caption}
    \DeclareCaptionFormat{nocaption}{}
    \captionsetup{format=nocaption,aboveskip=0pt,belowskip=0pt}

    \usepackage{float}
    \floatplacement{figure}{H} % forces figures to be placed at the correct location
    \usepackage{xcolor} % Allow colors to be defined
    \usepackage{enumerate} % Needed for markdown enumerations to work
    \usepackage{geometry} % Used to adjust the document margins
    \usepackage{amsmath} % Equations
    \usepackage{amssymb} % Equations
    \usepackage{textcomp} % defines textquotesingle
    % Hack from http://tex.stackexchange.com/a/47451/13684:
    \AtBeginDocument{%
        \def\PYZsq{\textquotesingle}% Upright quotes in Pygmentized code
    }
    \usepackage{upquote} % Upright quotes for verbatim code
    \usepackage{eurosym} % defines \euro
    \usepackage[mathletters]{ucs} % Extended unicode (utf-8) support
    \usepackage{fancyvrb} % verbatim replacement that allows latex
    \usepackage{grffile} % extends the file name processing of package graphics
                         % to support a larger range
    \makeatletter % fix for old versions of grffile with XeLaTeX
    \@ifpackagelater{grffile}{2019/11/01}
    {
      % Do nothing on new versions
    }
    {
      \def\Gread@@xetex#1{%
        \IfFileExists{"\Gin@base".bb}%
        {\Gread@eps{\Gin@base.bb}}%
        {\Gread@@xetex@aux#1}%
      }
    }
    \makeatother
    \usepackage[Export]{adjustbox} % Used to constrain images to a maximum size
    \adjustboxset{max size={0.9\linewidth}{0.9\paperheight}}

    % The hyperref package gives us a pdf with properly built
    % internal navigation ('pdf bookmarks' for the table of contents,
    % internal cross-reference links, web links for URLs, etc.)
    \usepackage{hyperref}
    % The default LaTeX title has an obnoxious amount of whitespace. By default,
    % titling removes some of it. It also provides customization options.
    \usepackage{titling}
    \usepackage{longtable} % longtable support required by pandoc >1.10
    \usepackage{booktabs}  % table support for pandoc > 1.12.2
    \usepackage[inline]{enumitem} % IRkernel/repr support (it uses the enumerate* environment)
    \usepackage[normalem]{ulem} % ulem is needed to support strikethroughs (\sout)
                                % normalem makes italics be italics, not underlines
    \usepackage{mathrsfs}



    % Colors for the hyperref package
    \definecolor{urlcolor}{rgb}{0,.145,.698}
    \definecolor{linkcolor}{rgb}{.71,0.21,0.01}
    \definecolor{citecolor}{rgb}{.12,.54,.11}

    % ANSI colors
    \definecolor{ansi-black}{HTML}{3E424D}
    \definecolor{ansi-black-intense}{HTML}{282C36}
    \definecolor{ansi-red}{HTML}{E75C58}
    \definecolor{ansi-red-intense}{HTML}{B22B31}
    \definecolor{ansi-green}{HTML}{00A250}
    \definecolor{ansi-green-intense}{HTML}{007427}
    \definecolor{ansi-yellow}{HTML}{DDB62B}
    \definecolor{ansi-yellow-intense}{HTML}{B27D12}
    \definecolor{ansi-blue}{HTML}{208FFB}
    \definecolor{ansi-blue-intense}{HTML}{0065CA}
    \definecolor{ansi-magenta}{HTML}{D160C4}
    \definecolor{ansi-magenta-intense}{HTML}{A03196}
    \definecolor{ansi-cyan}{HTML}{60C6C8}
    \definecolor{ansi-cyan-intense}{HTML}{258F8F}
    \definecolor{ansi-white}{HTML}{C5C1B4}
    \definecolor{ansi-white-intense}{HTML}{A1A6B2}
    \definecolor{ansi-default-inverse-fg}{HTML}{FFFFFF}
    \definecolor{ansi-default-inverse-bg}{HTML}{000000}

    % common color for the border for error outputs.
    \definecolor{outerrorbackground}{HTML}{FFDFDF}

    % commands and environments needed by pandoc snippets
    % extracted from the output of `pandoc -s`
    \providecommand{\tightlist}{%
      \setlength{\itemsep}{0pt}\setlength{\parskip}{0pt}}
    \DefineVerbatimEnvironment{Highlighting}{Verbatim}{commandchars=\\\{\}}
    % Add ',fontsize=\small' for more characters per line
    \newenvironment{Shaded}{}{}
    \newcommand{\KeywordTok}[1]{\textcolor[rgb]{0.00,0.44,0.13}{\textbf{{#1}}}}
    \newcommand{\DataTypeTok}[1]{\textcolor[rgb]{0.56,0.13,0.00}{{#1}}}
    \newcommand{\DecValTok}[1]{\textcolor[rgb]{0.25,0.63,0.44}{{#1}}}
    \newcommand{\BaseNTok}[1]{\textcolor[rgb]{0.25,0.63,0.44}{{#1}}}
    \newcommand{\FloatTok}[1]{\textcolor[rgb]{0.25,0.63,0.44}{{#1}}}
    \newcommand{\CharTok}[1]{\textcolor[rgb]{0.25,0.44,0.63}{{#1}}}
    \newcommand{\StringTok}[1]{\textcolor[rgb]{0.25,0.44,0.63}{{#1}}}
    \newcommand{\CommentTok}[1]{\textcolor[rgb]{0.38,0.63,0.69}{\textit{{#1}}}}
    \newcommand{\OtherTok}[1]{\textcolor[rgb]{0.00,0.44,0.13}{{#1}}}
    \newcommand{\AlertTok}[1]{\textcolor[rgb]{1.00,0.00,0.00}{\textbf{{#1}}}}
    \newcommand{\FunctionTok}[1]{\textcolor[rgb]{0.02,0.16,0.49}{{#1}}}
    \newcommand{\RegionMarkerTok}[1]{{#1}}
    \newcommand{\ErrorTok}[1]{\textcolor[rgb]{1.00,0.00,0.00}{\textbf{{#1}}}}
    \newcommand{\NormalTok}[1]{{#1}}

    % Additional commands for more recent versions of Pandoc
    \newcommand{\ConstantTok}[1]{\textcolor[rgb]{0.53,0.00,0.00}{{#1}}}
    \newcommand{\SpecialCharTok}[1]{\textcolor[rgb]{0.25,0.44,0.63}{{#1}}}
    \newcommand{\VerbatimStringTok}[1]{\textcolor[rgb]{0.25,0.44,0.63}{{#1}}}
    \newcommand{\SpecialStringTok}[1]{\textcolor[rgb]{0.73,0.40,0.53}{{#1}}}
    \newcommand{\ImportTok}[1]{{#1}}
    \newcommand{\DocumentationTok}[1]{\textcolor[rgb]{0.73,0.13,0.13}{\textit{{#1}}}}
    \newcommand{\AnnotationTok}[1]{\textcolor[rgb]{0.38,0.63,0.69}{\textbf{\textit{{#1}}}}}
    \newcommand{\CommentVarTok}[1]{\textcolor[rgb]{0.38,0.63,0.69}{\textbf{\textit{{#1}}}}}
    \newcommand{\VariableTok}[1]{\textcolor[rgb]{0.10,0.09,0.49}{{#1}}}
    \newcommand{\ControlFlowTok}[1]{\textcolor[rgb]{0.00,0.44,0.13}{\textbf{{#1}}}}
    \newcommand{\OperatorTok}[1]{\textcolor[rgb]{0.40,0.40,0.40}{{#1}}}
    \newcommand{\BuiltInTok}[1]{{#1}}
    \newcommand{\ExtensionTok}[1]{{#1}}
    \newcommand{\PreprocessorTok}[1]{\textcolor[rgb]{0.74,0.48,0.00}{{#1}}}
    \newcommand{\AttributeTok}[1]{\textcolor[rgb]{0.49,0.56,0.16}{{#1}}}
    \newcommand{\InformationTok}[1]{\textcolor[rgb]{0.38,0.63,0.69}{\textbf{\textit{{#1}}}}}
    \newcommand{\WarningTok}[1]{\textcolor[rgb]{0.38,0.63,0.69}{\textbf{\textit{{#1}}}}}


    % Define a nice break command that doesn't care if a line doesn't already
    % exist.
    \def\br{\hspace*{\fill} \\* }
    % Math Jax compatibility definitions
    \def\gt{>}
    \def\lt{<}
    \let\Oldtex\TeX
    \let\Oldlatex\LaTeX
    \renewcommand{\TeX}{\textrm{\Oldtex}}
    \renewcommand{\LaTeX}{\textrm{\Oldlatex}}
    % Document parameters
    % Document title
    \title{
      {\Large Лекция 4} \\
      Ортогональное проектирование
    }
    % \date{5 октября 2022\,г.}
    \date{}



% Pygments definitions
\makeatletter
\def\PY@reset{\let\PY@it=\relax \let\PY@bf=\relax%
    \let\PY@ul=\relax \let\PY@tc=\relax%
    \let\PY@bc=\relax \let\PY@ff=\relax}
\def\PY@tok#1{\csname PY@tok@#1\endcsname}
\def\PY@toks#1+{\ifx\relax#1\empty\else%
    \PY@tok{#1}\expandafter\PY@toks\fi}
\def\PY@do#1{\PY@bc{\PY@tc{\PY@ul{%
    \PY@it{\PY@bf{\PY@ff{#1}}}}}}}
\def\PY#1#2{\PY@reset\PY@toks#1+\relax+\PY@do{#2}}

\@namedef{PY@tok@w}{\def\PY@tc##1{\textcolor[rgb]{0.73,0.73,0.73}{##1}}}
\@namedef{PY@tok@c}{\let\PY@it=\textit\def\PY@tc##1{\textcolor[rgb]{0.24,0.48,0.48}{##1}}}
\@namedef{PY@tok@cp}{\def\PY@tc##1{\textcolor[rgb]{0.61,0.40,0.00}{##1}}}
\@namedef{PY@tok@k}{\let\PY@bf=\textbf\def\PY@tc##1{\textcolor[rgb]{0.00,0.50,0.00}{##1}}}
\@namedef{PY@tok@kp}{\def\PY@tc##1{\textcolor[rgb]{0.00,0.50,0.00}{##1}}}
\@namedef{PY@tok@kt}{\def\PY@tc##1{\textcolor[rgb]{0.69,0.00,0.25}{##1}}}
\@namedef{PY@tok@o}{\def\PY@tc##1{\textcolor[rgb]{0.40,0.40,0.40}{##1}}}
\@namedef{PY@tok@ow}{\let\PY@bf=\textbf\def\PY@tc##1{\textcolor[rgb]{0.67,0.13,1.00}{##1}}}
\@namedef{PY@tok@nb}{\def\PY@tc##1{\textcolor[rgb]{0.00,0.50,0.00}{##1}}}
\@namedef{PY@tok@nf}{\def\PY@tc##1{\textcolor[rgb]{0.00,0.00,1.00}{##1}}}
\@namedef{PY@tok@nc}{\let\PY@bf=\textbf\def\PY@tc##1{\textcolor[rgb]{0.00,0.00,1.00}{##1}}}
\@namedef{PY@tok@nn}{\let\PY@bf=\textbf\def\PY@tc##1{\textcolor[rgb]{0.00,0.00,1.00}{##1}}}
\@namedef{PY@tok@ne}{\let\PY@bf=\textbf\def\PY@tc##1{\textcolor[rgb]{0.80,0.25,0.22}{##1}}}
\@namedef{PY@tok@nv}{\def\PY@tc##1{\textcolor[rgb]{0.10,0.09,0.49}{##1}}}
\@namedef{PY@tok@no}{\def\PY@tc##1{\textcolor[rgb]{0.53,0.00,0.00}{##1}}}
\@namedef{PY@tok@nl}{\def\PY@tc##1{\textcolor[rgb]{0.46,0.46,0.00}{##1}}}
\@namedef{PY@tok@ni}{\let\PY@bf=\textbf\def\PY@tc##1{\textcolor[rgb]{0.44,0.44,0.44}{##1}}}
\@namedef{PY@tok@na}{\def\PY@tc##1{\textcolor[rgb]{0.41,0.47,0.13}{##1}}}
\@namedef{PY@tok@nt}{\let\PY@bf=\textbf\def\PY@tc##1{\textcolor[rgb]{0.00,0.50,0.00}{##1}}}
\@namedef{PY@tok@nd}{\def\PY@tc##1{\textcolor[rgb]{0.67,0.13,1.00}{##1}}}
\@namedef{PY@tok@s}{\def\PY@tc##1{\textcolor[rgb]{0.73,0.13,0.13}{##1}}}
\@namedef{PY@tok@sd}{\let\PY@it=\textit\def\PY@tc##1{\textcolor[rgb]{0.73,0.13,0.13}{##1}}}
\@namedef{PY@tok@si}{\let\PY@bf=\textbf\def\PY@tc##1{\textcolor[rgb]{0.64,0.35,0.47}{##1}}}
\@namedef{PY@tok@se}{\let\PY@bf=\textbf\def\PY@tc##1{\textcolor[rgb]{0.67,0.36,0.12}{##1}}}
\@namedef{PY@tok@sr}{\def\PY@tc##1{\textcolor[rgb]{0.64,0.35,0.47}{##1}}}
\@namedef{PY@tok@ss}{\def\PY@tc##1{\textcolor[rgb]{0.10,0.09,0.49}{##1}}}
\@namedef{PY@tok@sx}{\def\PY@tc##1{\textcolor[rgb]{0.00,0.50,0.00}{##1}}}
\@namedef{PY@tok@m}{\def\PY@tc##1{\textcolor[rgb]{0.40,0.40,0.40}{##1}}}
\@namedef{PY@tok@gh}{\let\PY@bf=\textbf\def\PY@tc##1{\textcolor[rgb]{0.00,0.00,0.50}{##1}}}
\@namedef{PY@tok@gu}{\let\PY@bf=\textbf\def\PY@tc##1{\textcolor[rgb]{0.50,0.00,0.50}{##1}}}
\@namedef{PY@tok@gd}{\def\PY@tc##1{\textcolor[rgb]{0.63,0.00,0.00}{##1}}}
\@namedef{PY@tok@gi}{\def\PY@tc##1{\textcolor[rgb]{0.00,0.52,0.00}{##1}}}
\@namedef{PY@tok@gr}{\def\PY@tc##1{\textcolor[rgb]{0.89,0.00,0.00}{##1}}}
\@namedef{PY@tok@ge}{\let\PY@it=\textit}
\@namedef{PY@tok@gs}{\let\PY@bf=\textbf}
\@namedef{PY@tok@gp}{\let\PY@bf=\textbf\def\PY@tc##1{\textcolor[rgb]{0.00,0.00,0.50}{##1}}}
\@namedef{PY@tok@go}{\def\PY@tc##1{\textcolor[rgb]{0.44,0.44,0.44}{##1}}}
\@namedef{PY@tok@gt}{\def\PY@tc##1{\textcolor[rgb]{0.00,0.27,0.87}{##1}}}
\@namedef{PY@tok@err}{\def\PY@bc##1{{\setlength{\fboxsep}{\string -\fboxrule}\fcolorbox[rgb]{1.00,0.00,0.00}{1,1,1}{\strut ##1}}}}
\@namedef{PY@tok@kc}{\let\PY@bf=\textbf\def\PY@tc##1{\textcolor[rgb]{0.00,0.50,0.00}{##1}}}
\@namedef{PY@tok@kd}{\let\PY@bf=\textbf\def\PY@tc##1{\textcolor[rgb]{0.00,0.50,0.00}{##1}}}
\@namedef{PY@tok@kn}{\let\PY@bf=\textbf\def\PY@tc##1{\textcolor[rgb]{0.00,0.50,0.00}{##1}}}
\@namedef{PY@tok@kr}{\let\PY@bf=\textbf\def\PY@tc##1{\textcolor[rgb]{0.00,0.50,0.00}{##1}}}
\@namedef{PY@tok@bp}{\def\PY@tc##1{\textcolor[rgb]{0.00,0.50,0.00}{##1}}}
\@namedef{PY@tok@fm}{\def\PY@tc##1{\textcolor[rgb]{0.00,0.00,1.00}{##1}}}
\@namedef{PY@tok@vc}{\def\PY@tc##1{\textcolor[rgb]{0.10,0.09,0.49}{##1}}}
\@namedef{PY@tok@vg}{\def\PY@tc##1{\textcolor[rgb]{0.10,0.09,0.49}{##1}}}
\@namedef{PY@tok@vi}{\def\PY@tc##1{\textcolor[rgb]{0.10,0.09,0.49}{##1}}}
\@namedef{PY@tok@vm}{\def\PY@tc##1{\textcolor[rgb]{0.10,0.09,0.49}{##1}}}
\@namedef{PY@tok@sa}{\def\PY@tc##1{\textcolor[rgb]{0.73,0.13,0.13}{##1}}}
\@namedef{PY@tok@sb}{\def\PY@tc##1{\textcolor[rgb]{0.73,0.13,0.13}{##1}}}
\@namedef{PY@tok@sc}{\def\PY@tc##1{\textcolor[rgb]{0.73,0.13,0.13}{##1}}}
\@namedef{PY@tok@dl}{\def\PY@tc##1{\textcolor[rgb]{0.73,0.13,0.13}{##1}}}
\@namedef{PY@tok@s2}{\def\PY@tc##1{\textcolor[rgb]{0.73,0.13,0.13}{##1}}}
\@namedef{PY@tok@sh}{\def\PY@tc##1{\textcolor[rgb]{0.73,0.13,0.13}{##1}}}
\@namedef{PY@tok@s1}{\def\PY@tc##1{\textcolor[rgb]{0.73,0.13,0.13}{##1}}}
\@namedef{PY@tok@mb}{\def\PY@tc##1{\textcolor[rgb]{0.40,0.40,0.40}{##1}}}
\@namedef{PY@tok@mf}{\def\PY@tc##1{\textcolor[rgb]{0.40,0.40,0.40}{##1}}}
\@namedef{PY@tok@mh}{\def\PY@tc##1{\textcolor[rgb]{0.40,0.40,0.40}{##1}}}
\@namedef{PY@tok@mi}{\def\PY@tc##1{\textcolor[rgb]{0.40,0.40,0.40}{##1}}}
\@namedef{PY@tok@il}{\def\PY@tc##1{\textcolor[rgb]{0.40,0.40,0.40}{##1}}}
\@namedef{PY@tok@mo}{\def\PY@tc##1{\textcolor[rgb]{0.40,0.40,0.40}{##1}}}
\@namedef{PY@tok@ch}{\let\PY@it=\textit\def\PY@tc##1{\textcolor[rgb]{0.24,0.48,0.48}{##1}}}
\@namedef{PY@tok@cm}{\let\PY@it=\textit\def\PY@tc##1{\textcolor[rgb]{0.24,0.48,0.48}{##1}}}
\@namedef{PY@tok@cpf}{\let\PY@it=\textit\def\PY@tc##1{\textcolor[rgb]{0.24,0.48,0.48}{##1}}}
\@namedef{PY@tok@c1}{\let\PY@it=\textit\def\PY@tc##1{\textcolor[rgb]{0.24,0.48,0.48}{##1}}}
\@namedef{PY@tok@cs}{\let\PY@it=\textit\def\PY@tc##1{\textcolor[rgb]{0.24,0.48,0.48}{##1}}}

\def\PYZbs{\char`\\}
\def\PYZus{\char`\_}
\def\PYZob{\char`\{}
\def\PYZcb{\char`\}}
\def\PYZca{\char`\^}
\def\PYZam{\char`\&}
\def\PYZlt{\char`\<}
\def\PYZgt{\char`\>}
\def\PYZsh{\char`\#}
\def\PYZpc{\char`\%}
\def\PYZdl{\char`\$}
\def\PYZhy{\char`\-}
\def\PYZsq{\char`\'}
\def\PYZdq{\char`\"}
\def\PYZti{\char`\~}
% for compatibility with earlier versions
\def\PYZat{@}
\def\PYZlb{[}
\def\PYZrb{]}
\makeatother


    % For linebreaks inside Verbatim environment from package fancyvrb.
    \makeatletter
        \newbox\Wrappedcontinuationbox
        \newbox\Wrappedvisiblespacebox
        \newcommand*\Wrappedvisiblespace {\textcolor{red}{\textvisiblespace}}
        \newcommand*\Wrappedcontinuationsymbol {\textcolor{red}{\llap{\tiny$\m@th\hookrightarrow$}}}
        \newcommand*\Wrappedcontinuationindent {3ex }
        \newcommand*\Wrappedafterbreak {\kern\Wrappedcontinuationindent\copy\Wrappedcontinuationbox}
        % Take advantage of the already applied Pygments mark-up to insert
        % potential linebreaks for TeX processing.
        %        {, <, #, %, $, ' and ": go to next line.
        %        _, }, ^, &, >, - and ~: stay at end of broken line.
        % Use of \textquotesingle for straight quote.
        \newcommand*\Wrappedbreaksatspecials {%
            \def\PYGZus{\discretionary{\char`\_}{\Wrappedafterbreak}{\char`\_}}%
            \def\PYGZob{\discretionary{}{\Wrappedafterbreak\char`\{}{\char`\{}}%
            \def\PYGZcb{\discretionary{\char`\}}{\Wrappedafterbreak}{\char`\}}}%
            \def\PYGZca{\discretionary{\char`\^}{\Wrappedafterbreak}{\char`\^}}%
            \def\PYGZam{\discretionary{\char`\&}{\Wrappedafterbreak}{\char`\&}}%
            \def\PYGZlt{\discretionary{}{\Wrappedafterbreak\char`\<}{\char`\<}}%
            \def\PYGZgt{\discretionary{\char`\>}{\Wrappedafterbreak}{\char`\>}}%
            \def\PYGZsh{\discretionary{}{\Wrappedafterbreak\char`\#}{\char`\#}}%
            \def\PYGZpc{\discretionary{}{\Wrappedafterbreak\char`\%}{\char`\%}}%
            \def\PYGZdl{\discretionary{}{\Wrappedafterbreak\char`\$}{\char`\$}}%
            \def\PYGZhy{\discretionary{\char`\-}{\Wrappedafterbreak}{\char`\-}}%
            \def\PYGZsq{\discretionary{}{\Wrappedafterbreak\textquotesingle}{\textquotesingle}}%
            \def\PYGZdq{\discretionary{}{\Wrappedafterbreak\char`\"}{\char`\"}}%
            \def\PYGZti{\discretionary{\char`\~}{\Wrappedafterbreak}{\char`\~}}%
        }
        % Some characters . , ; ? ! / are not pygmentized.
        % This macro makes them "active" and they will insert potential linebreaks
        \newcommand*\Wrappedbreaksatpunct {%
            \lccode`\~`\.\lowercase{\def~}{\discretionary{\hbox{\char`\.}}{\Wrappedafterbreak}{\hbox{\char`\.}}}%
            \lccode`\~`\,\lowercase{\def~}{\discretionary{\hbox{\char`\,}}{\Wrappedafterbreak}{\hbox{\char`\,}}}%
            \lccode`\~`\;\lowercase{\def~}{\discretionary{\hbox{\char`\;}}{\Wrappedafterbreak}{\hbox{\char`\;}}}%
            \lccode`\~`\:\lowercase{\def~}{\discretionary{\hbox{\char`\:}}{\Wrappedafterbreak}{\hbox{\char`\:}}}%
            \lccode`\~`\?\lowercase{\def~}{\discretionary{\hbox{\char`\?}}{\Wrappedafterbreak}{\hbox{\char`\?}}}%
            \lccode`\~`\!\lowercase{\def~}{\discretionary{\hbox{\char`\!}}{\Wrappedafterbreak}{\hbox{\char`\!}}}%
            \lccode`\~`\/\lowercase{\def~}{\discretionary{\hbox{\char`\/}}{\Wrappedafterbreak}{\hbox{\char`\/}}}%
            \catcode`\.\active
            \catcode`\,\active
            \catcode`\;\active
            \catcode`\:\active
            \catcode`\?\active
            \catcode`\!\active
            \catcode`\/\active
            \lccode`\~`\~
        }
    \makeatother

    \let\OriginalVerbatim=\Verbatim
    \makeatletter
    \renewcommand{\Verbatim}[1][1]{%
        %\parskip\z@skip
        \sbox\Wrappedcontinuationbox {\Wrappedcontinuationsymbol}%
        \sbox\Wrappedvisiblespacebox {\FV@SetupFont\Wrappedvisiblespace}%
        \def\FancyVerbFormatLine ##1{\hsize\linewidth
            \vtop{\raggedright\hyphenpenalty\z@\exhyphenpenalty\z@
                \doublehyphendemerits\z@\finalhyphendemerits\z@
                \strut ##1\strut}%
        }%
        % If the linebreak is at a space, the latter will be displayed as visible
        % space at end of first line, and a continuation symbol starts next line.
        % Stretch/shrink are however usually zero for typewriter font.
        \def\FV@Space {%
            \nobreak\hskip\z@ plus\fontdimen3\font minus\fontdimen4\font
            \discretionary{\copy\Wrappedvisiblespacebox}{\Wrappedafterbreak}
            {\kern\fontdimen2\font}%
        }%

        % Allow breaks at special characters using \PYG... macros.
        \Wrappedbreaksatspecials
        % Breaks at punctuation characters . , ; ? ! and / need catcode=\active
        \OriginalVerbatim[#1,codes*=\Wrappedbreaksatpunct]%
    }
    \makeatother

    % Exact colors from NB
    \definecolor{incolor}{HTML}{303F9F}
    \definecolor{outcolor}{HTML}{D84315}
    \definecolor{cellborder}{HTML}{CFCFCF}
    \definecolor{cellbackground}{HTML}{F7F7F7}

    % prompt
    \makeatletter
    \newcommand{\boxspacing}{\kern\kvtcb@left@rule\kern\kvtcb@boxsep}
    \makeatother
    \newcommand{\prompt}[4]{
        {\ttfamily\llap{{\color{#2}[#3]:\hspace{3pt}#4}}\vspace{-\baselineskip}}
    }



    % Prevent overflowing lines due to hard-to-break entities
    \sloppy
    % Setup hyperref package
    \hypersetup{
      breaklinks=true,  % so long urls are correctly broken across lines
      colorlinks=true,
      urlcolor=urlcolor,
      linkcolor=linkcolor,
      citecolor=citecolor,
      }
    % Slightly bigger margins than the latex defaults

    \geometry{verbose,tmargin=1in,bmargin=1in,lmargin=1in,rmargin=1in}



\begin{document}

  \maketitle
  \thispagestyle{empty}
  \tableofcontents

%  \let\thefootnote\relax\footnote{
%    \textit{День 5 октября в истории:
%      \begin{itemize}[topsep=2pt,itemsep=1pt]
%        \item 1502 г. --- Христофор Колумб открыл Коста-Рику.
%        \item 1823 г. --- в Англии начал издаваться первый медицинский журнал <<Ланцет>>.
%        \item 1969 г. --- на телеканале BBC прошло первое представление <<Летающего цирка Монти Пайтон>>.
%      \end{itemize}
%    }
%  }

  \newpage


    \hypertarget{ux43cux435ux442ux440ux438ux43aux430-ux43fux440ux43eux441ux442ux440ux430ux43dux441ux442ux432ux430}{%
\section{Метрика
пространства}\label{ux43cux435ux442ux440ux438ux43aux430-ux43fux440ux43eux441ux442ux440ux430ux43dux441ux442ux432ux430}}

\hypertarget{ux441ux43aux430ux43bux44fux440ux43dux43eux435-ux43fux440ux43eux438ux437ux432ux435ux434ux435ux43dux438ux435}{%
\subsection{Скалярное
произведение}\label{ux441ux43aux430ux43bux44fux440ux43dux43eux435-ux43fux440ux43eux438ux437ux432ux435ux434ux435ux43dux438ux435}}

Мы уже рассматривали линейный операторы в произвольном \(n\)-мерном
линейном пространстве. Все базисы такого пространства равноправны между
собой. Данному линейному оператору в каждом базисе соответствует
некоторая матрица. Матрицы, отвечающие одному и тому же оператору в
различных базисах, подобны между собой.

Теперь давайте в линейное \(n\)-мерное пространство введём метрику. Для
этого каждым двум векторам \(\mathbf{x}\) и \(\mathbf{y}\) мы
специальным образом поставим в соответствие некоторое число
\((\mathbf{x}, \mathbf{y})\) --- их \emph{скалярное произведение}.

Для любых векторов \(\mathbf{x}\), \(\mathbf{y}\), \(\mathbf{z}\) и
любого числа \(\alpha\) справедливы следующие \textbf{свойства
скалярного произведения}:

\begin{enumerate}
\def\labelenumi{\arabic{enumi}.}
\tightlist
\item
  \((\mathbf{x}, \mathbf{y}) = (\mathbf{y}, \mathbf{x})\)
  (коммутативность),
\item
  \((\alpha \mathbf{x}, \mathbf{y}) = \alpha (\mathbf{x}, \mathbf{y})\),
\item
  \((\mathbf{x} + \mathbf{y}, \mathbf{z}) = (\mathbf{x}, \mathbf{z}) + (\mathbf{y}, \mathbf{z})\)
  (дистрибутивность),
\item
  \((\mathbf{x}, \mathbf{x}) > 0\) при \(\mathbf{x} \ne 0\) (положительная определённость).
\end{enumerate}

\textbf{Определение.} Векторное пространство с положительно определённым скалярным произведением называется \textbf{евклидовым
пространством}.

\textbf{Длиной вектора} называется
\(|\mathbf{x}| = \sqrt{(\mathbf{x}, \mathbf{x})}\).

\textbf{Косинус угла} между двумя векторами равен
\[ cos(\theta) = \frac{(\mathbf{x}, \mathbf{y})}{|\mathbf{x}| |\mathbf{y}|}. \]

    \hypertarget{ux43eux440ux442ux43eux433ux43eux43dux430ux43bux44cux43dux44bux435-ux432ux435ux43aux442ux43eux440ux44b}{%
\subsection{Ортогональные
векторы}\label{ux43eux440ux442ux43eux433ux43eux43dux430ux43bux44cux43dux44bux435-ux432ux435ux43aux442ux43eux440ux44b}}

Два вектора \(\mathbf{x}\) и \(\mathbf{y}\) называются
\textbf{ортогональными} (\(\mathbf{x} \perp \mathbf{y}\)), если
\((\mathbf{x}, \mathbf{y}) = 0\).

Рассмотрим матрицу \(Q\), столбцами которой являются ортонормированные
векторы. Легко видеть, что \(Q^\top Q = I\).

Отсюда следует, что
\(\forall \mathbf{x}, \mathbf{y}: (Q\mathbf{x}, Q\mathbf{y}) = (Q\mathbf{x})^\top (Q\mathbf{y}) = \mathbf{x}^\top Q^\top Q \mathbf{y} = (\mathbf{x}, \mathbf{y})\).
В частности, при умножении на \(Q\) длина вектора не меняется:
\(|Q\mathbf{x}| = (Q\mathbf{x}, Q\mathbf{x}) = (\mathbf{x}, \mathbf{x}) = |\mathbf{x}|.\)

Что такое матрица \(P = Q Q^\top\)?

\[
  P = Q Q^\top = \mathbf{q}_1 \mathbf{q}_1^\top + \ldots + \mathbf{q}_n \mathbf{q}_n^\top
\]

Заметим, что $P^2 = (QQ^\top)(QQ^\top) = Q(Q^\top Q)Q^\top = P$.
А значит, $P$ --- проекционная матрица.

Если \(Q\) --- квадратная матрица, то \(Q Q^\top\) также является
единичной матрицей: \(Q Q^\top = I\).

    \hypertarget{ux43eux440ux442ux43eux433ux43eux43dux430ux43bux44cux43dux44bux435-ux432ux435ux43aux442ux43eux440ux44b-ux438-ux43cux430ux442ux440ux438ux446ux44b}{%
\subsection{Ортогональные
матрицы}\label{ux43eux440ux442ux43eux433ux43eux43dux430ux43bux44cux43dux44bux435-ux432ux435ux43aux442ux43eux440ux44b-ux438-ux43cux430ux442ux440ux438ux446ux44b}}

\textbf{Определение.} \textbf{Ортогональной матрицей} называется
\emph{квадратная} матрица, столбцами которой являются \emph{ортонормированные}
векторы.

Линейные преобразования, соответствующие ортогональным матрицам,
представляют собой некоторые «движения». В том смысле, что сохраняются
углы и длины.

\textbf{Пример 1.} Матрица поворота \[
  Q_\mathrm{rotate} =
  \begin{pmatrix}
     \cos\theta & -\sin\theta \\
     \sin\theta &  \cos\theta \\
  \end{pmatrix}
\]

\textbf{Пример 2.} Матрица отражения \[
  Q_\mathrm{reflect} =
  \begin{pmatrix}
     \cos\theta &  \sin\theta \\
     \sin\theta & -\cos\theta \\
  \end{pmatrix}
\]

\textbf{Пример 3.} Произведение двух ортогональных матриц \(Q_1 Q_2\)
--- ортогональная матрица.

\begin{quote}
Вращение на вращение = вращение, отражение на отражение = вращение,
вращение на отражение = отражение.
\end{quote}

    \begin{center}\rule{0.5\linewidth}{0.5pt}\end{center}

    \hypertarget{ux43eux440ux442ux43eux433ux43eux43dux430ux43bux44cux43dux43eux435-ux43fux440ux43eux435ux43aux442ux438ux440ux43eux432ux430ux43dux438ux435}{%
\section{Ортогональное
проектирование}\label{ux43eux440ux442ux43eux433ux43eux43dux430ux43bux44cux43dux43eux435-ux43fux440ux43eux435ux43aux442ux438ux440ux43eux432ux430ux43dux438ux435}}

\hypertarget{ux43eux431ux449ux438ux439-ux441ux43bux443ux447ux430ux439}{%
\subsection{Общий
случай}\label{ux43eux431ux449ux438ux439-ux441ux43bux443ux447ux430ux439}}

Пусть заданы вектор \(\mathbf{b}\) из \(\mathbb{R}^m\) и набор векторов
\(\mathbf{a}_i\), образующих базис в \(\mathbb{R}^n\) (\(n<m\)).

Требуется найти ортогональную проекцию вектора \(\mathbf{b}\) на
линейную оболочку векторов \(\mathbf{a}_i\).

Искомый вектор является линейной комбинацией базисных векторов
\(\mathbf{a}_i\) с неизвестными коэффициентами \(x_i\). Запишем это в
виде произведения матрицы \(A\), столбцы которой являются векторами
\(\mathbf{a}_i\), на неизвестный вектор \(\mathbf{x}\): \(A\mathbf{x}\).
Мы ищем ортогональную проекцию на пространство столбцов матрицы \(A\),
поэтому вектор \(\mathbf{e} = A\mathbf{x} - \mathbf{b}\) должен быть
ортогонален \emph{любому} вектору \(A\mathbf{y}\). Запишем это через
скалярное произведение: \[
  (A \mathbf{y})^\top(A\mathbf{x} - \mathbf{b}) = \mathbf{y}^\top (A^\top A \mathbf{x} - A^\top \mathbf{b}) = 0.
\]

Это справедливо для произвольного вектора \(\mathbf{y}\), откуда
следует, что \[
  A^\top A \mathbf{x} = A^\top \mathbf{b}.
\]

Ранг матрицы \(A^\top A\) равен рангу \(A\), а столбцы матрицы \(A\)
линейно независимы, следовательно матрица \(A^\top A\) обратима. Отсюда
находим выражение для вектора коэффициентов проекции и сам вектор
проекции \(\mathbf{p} = A\mathbf{x}\) \[
  \mathbf{x} = (A^\top A)^{-1} A^\top \mathbf{b}, \
  \mathbf{p} = A (A^\top A)^{-1} A^\top \mathbf{b}.
\]

Матрица \(P = A (A^\top A)^{-1} A^\top\), осуществляющая проекцию,
называется \emph{матрицей ортогонального проектирования}.
Матрица ортопроектирования обладает двумя основными свойствами:

\begin{enumerate}
\def\labelenumi{\arabic{enumi}.}
\tightlist
\item
  \(P^2 = P\) --- характеристическое свойство всех проекторов
  (\emph{идемпотентность}),
\item
  \(P^\top = P\) --- отличительное свойство ортогонального проектора
  (\emph{симметричность}).
\end{enumerate}

    \hypertarget{ux43eux434ux43dux43eux43cux435ux440ux43dux44bux439-ux441ux43bux443ux447ux430ux439}{%
\subsection{Одномерный
случай}\label{ux43eux434ux43dux43eux43cux435ux440ux43dux44bux439-ux441ux43bux443ux447ux430ux439}}

Матрица проектирования на прямую, определённую вектором \(\mathbf{a}\),
задаётся формулой
\[ P = A (A^\top A)^{-1} A^\top = \mathbf{a} (\mathbf{a^\top}\mathbf{a})^{-1}\mathbf{a^\top} = \frac{1}{\mathbf{a^\top}\mathbf{a}} \mathbf{a} \mathbf{a^\top}. \]

А проекция \(\mathbf{p}\) точки \(\mathbf{b}\) --- формулой
\[ \mathbf{p} = \frac{\mathbf{a^\top} \mathbf{b}}{\mathbf{a^\top}\mathbf{a}}\mathbf{a}. \]

    \hypertarget{ux43fux440ux43eux435ux43aux442ux438ux440ux43eux432ux430ux43dux438ux435-ux43dux430-ux43bux438ux43dux435ux439ux43dux443ux44e-ux43eux431ux43eux43bux43eux447ux43aux443-ux43eux440ux442ux43eux43dux43eux440ux43cux438ux440ux43eux432ux430ux43dux43dux44bux445-ux432ux435ux43aux442ux43eux440ux43eux432}{%
\subsection{Проектирование на линейную оболочку ортонормированных
векторов}\label{ux43fux440ux43eux435ux43aux442ux438ux440ux43eux432ux430ux43dux438ux435-ux43dux430-ux43bux438ux43dux435ux439ux43dux443ux44e-ux43eux431ux43eux43bux43eux447ux43aux443-ux43eux440ux442ux43eux43dux43eux440ux43cux438ux440ux43eux432ux430ux43dux43dux44bux445-ux432ux435ux43aux442ux43eux440ux43eux432}}

Если набор векторов образует ортонормированный базис \(\mathbf{q}_i\),
то формула для матрицы ортогонального проектирования существенно
упрощается: \[
  P = Q (Q^\top Q)^{-1} Q^\top = QQ^\top = \mathbf{q}_1 \mathbf{q}_1^\top + \ldots + \mathbf{q}_n \mathbf{q}_n^\top.
\]

Каждое слагаемое \(\mathbf{q}_i \mathbf{q}_i^\top\) является матрицей
ортогональной проекции на вектор \(\mathbf{q}_i\).

Таким образом, \emph{когда оси координат взаимно перпендикулярны,
проекция на пространство разлагается в сумму проекций на каждую из
осей}.

    \hypertarget{ux43eux440ux442ux43eux43dux43eux440ux43cux438ux440ux43eux432ux430ux43dux43dux44bux439-ux431ux430ux437ux438ux441}{%
\subsection{Ортонормированный
базис}\label{ux43eux440ux442ux43eux43dux43eux440ux43cux438ux440ux43eux432ux430ux43dux43dux44bux439-ux431ux430ux437ux438ux441}}

Разложение любого вектора по ортонормированному базису есть сумма
ортогональных проекций на каждый вектор:

\[
  \mathbf{v} = c_1 \mathbf{q}_1 + \ldots + c_n \mathbf{q}_n = (\mathbf{q}_1^\top \mathbf{v}) \cdot \mathbf{q}_1 + \ldots + (\mathbf{q}_n^\top \mathbf{v}) \cdot \mathbf{q}_n.
\]

    \begin{center}\rule{0.5\linewidth}{0.5pt}\end{center}

    \hypertarget{ux43eux440ux442ux43eux433ux43eux43dux430ux43bux438ux437ux430ux446ux438ux44f-ux432ux435ux43aux442ux43eux440ux43eux432}{%
\section{Ортогонализация
векторов}\label{ux43eux440ux442ux43eux433ux43eux43dux430ux43bux438ux437ux430ux446ux438ux44f-ux432ux435ux43aux442ux43eux440ux43eux432}}

\hypertarget{ux430ux43bux433ux43eux440ux438ux442ux43c-ux433ux440ux430ux43cux430-ux448ux43cux438ux434ux442ux430}{%
\subsection{Алгоритм Грама ---
Шмидта}\label{ux430ux43bux433ux43eux440ux438ux442ux43c-ux433ux440ux430ux43cux430-ux448ux43cux438ux434ux442ux430}}

Два ряда векторов называются \textbf{эквивалентными} если они содержат
одинаковое количество векторов и их линейные оболочки совпадают.

Под \textbf{ортогонализацией} ряда векторов будем понимать замену этого
ряда на \emph{эквивалентный} (порождающий ту же самую линейную оболочку)
ортогональный ряд.

\textbf{Теорема.} Всякий невырожденный ряд векторов можно
проортогонализировать.

Алгоритм Грама --- Шмидта рассмотрим на примере трёхмерного
пространства.

Пусть даны три линейно-независимых вектора \(\mathbf{a}_1\),
\(\mathbf{a}_2\), \(\mathbf{a}_3\).

\begin{enumerate}
\def\labelenumi{\arabic{enumi}.}
\tightlist
\item
  \(\mathbf{q}_1 = \mathbf{a}_1 / |\mathbf{a}_1|\)
\item
  \(\mathbf{\hat{a_2}} = \mathbf{a}_2 - (\mathbf{a}_2^\top \mathbf{q}_1) \cdot \mathbf{q}_1\);
  \(\quad \mathbf{q}_2 = \mathbf{\hat{a_2}} / |\mathbf{\hat{a_2}}|\)
\item
  \(\mathbf{\hat{a_3}} = \mathbf{a}_3 - (\mathbf{a}_3^\top \mathbf{q}_1) \cdot \mathbf{q}_1 - (\mathbf{a}_3^\top \mathbf{q}_2) \cdot \mathbf{q}_2\);
  \(\quad \mathbf{q}_3 = \mathbf{\hat{a_3}}/|\mathbf{\hat{a_3}}|\)
\end{enumerate}

    \hypertarget{mathbfqr-ux440ux430ux437ux43bux43eux436ux435ux43dux438ux435}{%
\subsection{\texorpdfstring{\(\mathbf{QR}\)-разложение}{QR-разложение}}\label{mathbfqr-ux440ux430ux437ux43bux43eux436ux435ux43dux438ux435}}

\hypertarget{ux43aux432ux430ux434ux440ux430ux442ux43dux44bux435-ux43cux430ux442ux440ux438ux446ux44b}{%
\paragraph{Квадратные матрицы.} \label{ux43aux432ux430ux434ux440ux430ux442ux43dux44bux435-ux43cux430ux442ux440ux438ux446ux44b}}

Важную роль в численных методах играет разложение \emph{квадратной
матрицы} \(A\) на ортогональную матрицу \(Q\) и верхнетреугольную
матрицу \(R\). \[
  A = QR.
\]

\textbf{Утверждение 1.} Любая квадратная матрица \(A\) может быть
разложена в произведение \(QR\).

\textbf{Утверждение 2.} Если матрица \(A\) невырождена, то её
\(QR\)-разложение, в котором диагональные элементы \(R\) положительны,
единственно.

\(QR\)-разложение тесно связано с процессом ортогонализации Грама ---
Шмидта. Действительно, \(AU = Q\) равносильно \(QR\)-разложению с
\(R = U^{-1}\). Причём, \(U\) и \(R\) --- верхнетреугольная матрица.

\begin{quote}
Все неособенные верхние (нижние) треугольные матрицы составляют группу
(некоммутативную) относительно умножения.
\end{quote}

\begin{quote}
\emph{Группой} называется всякая совокупность объектов, в которой
установлена операция, относящая любым двум элементам \(a\) и \(b\)
совокупности определённый третий элемент \(a \ast b\) той же
совокупности, если 1) операция обладает сочетательным свойством, 2) в
совокупности существует единичный элемент, 3) для любого элемента
совокупности существует обратный элемент. Группа называется
\emph{коммутативной}, если групповая операция обладает переместительным
свойством.
\end{quote}

\textbf{Пример.} Рассмотрим систему линейных уравнений
\(A\mathbf{x} = \mathbf{b}\). Используя \(QR\)-разложение матрицы \(A\),
получим

\[
\begin{aligned}
  QR \mathbf{x} &= \mathbf{b}, \\
  R \mathbf{x}  &= Q^\top \mathbf{b}.
\end{aligned}
\]

И в случае невырожденной \(R\) \[
  \mathbf{x} = R^{-1}Q^\top \mathbf{b}.
\]

    \hypertarget{ux43fux440ux44fux43cux43eux443ux433ux43eux43bux44cux43dux44bux435-ux43cux430ux442ux440ux438ux446ux44b}{%
\paragraph{Прямоугольные матрицы.} \label{ux43fux440ux44fux43cux43eux443ux433ux43eux43bux44cux43dux44bux435-ux43cux430ux442ux440ux438ux446ux44b}}

Произвольную матрицу размером \(m \times n\) можно представить в виде
разложения \[
  A = QR,
\] где \(Q\) --- ортогональная матрица порядка \(m\), а \(R\) ---
матрица размеров \(m \times n\), элементы \(r_{ij}\) которой
удовлетворяют условию \(r_{ij}=0\) при \(i>j\). Такое разложение назовём
\(Qr\)-разложением, чтобы подчеркнуть, что матрица \(R\), вообще говоря,
не является квадратной.

Второе обобщение \(QR\)-разложения можно получить, применяя метод
ортогонализации Грама --- Шмидта. Теперь матрица \(Q\) размеров
\(m \times n\) может рассматриваться как совокупность \(n\) столбцов из
некоторой ортогональной матрицы порядка \(m\), а \(R\) --- квадратная
верхняя треугольная матрица порядка \(n\). Такое разложение будем
называть \(qR\)-разложением.

    \begin{center}\rule{0.5\linewidth}{0.5pt}\end{center}

    \hypertarget{ux438ux441ux442ux43eux447ux43dux438ux43aux438}{%
\section{Источники}\label{ux438ux441ux442ux43eux447ux43dux438ux43aux438}}

\begin{enumerate}
\def\labelenumi{\arabic{enumi}.}
\tightlist
\item
  \emph{Strang G.} Linear algebra and learning from data. ---
  Wellesley-Cambridge Press, 2019. --- 432\,p.
\item
  \emph{Стренг Г.} Линейная алгебра и её применения. --- М.: Мир, 1980.
  --- 454\,с.
\item
  \emph{Гантмахер Ф.Р.} Теория матриц. --- М.: Наука, 1967. --- 576\,с.
\item
  \emph{Беклемишев Д.В.} Дополнительные главы линейной алгебры. --- М.:
  Наука, 1983. --- 336\,с.
\end{enumerate}


    % Add a bibliography block to the postdoc



\end{document}
