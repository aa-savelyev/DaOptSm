\documentclass[11pt,a4paper]{article}

    \usepackage[breakable]{tcolorbox}
    \usepackage{parskip} % Stop auto-indenting (to mimic markdown behaviour)
    
    \usepackage{iftex}
    \ifPDFTeX
      \usepackage[T2A]{fontenc}
      \usepackage{mathpazo}
      \usepackage[russian,english]{babel}
    \else
      \usepackage{fontspec}
      \usepackage{polyglossia}
      \setmainlanguage[babelshorthands=true]{russian}    % Язык по-умолчанию русский с поддержкой приятных команд пакета babel
      \setotherlanguage{english}                         % Дополнительный язык = английский (в американской вариации по-умолчанию)
      \newfontfamily\cyrillicfonttt[Scale=0.87,BoldFont={Fira Mono Medium}] {Fira Mono}  % Моноширинный шрифт для кириллицы
      \defaultfontfeatures{Ligatures=TeX}
      \newfontfamily\cyrillicfont{STIX Two Text}         % Шрифт с засечками для кириллицы
    \fi
    \renewcommand{\linethickness}{0.1ex}

    % Basic figure setup, for now with no caption control since it's done
    % automatically by Pandoc (which extracts ![](path) syntax from Markdown).
    \usepackage{graphicx}
    % Maintain compatibility with old templates. Remove in nbconvert 6.0
    \let\Oldincludegraphics\includegraphics
    % Ensure that by default, figures have no caption (until we provide a
    % proper Figure object with a Caption API and a way to capture that
    % in the conversion process - todo).
    \usepackage{caption}
    \DeclareCaptionFormat{nocaption}{}
    \captionsetup{format=nocaption,aboveskip=0pt,belowskip=0pt}

    \usepackage[Export]{adjustbox} % Used to constrain images to a maximum size
    \adjustboxset{max size={0.9\linewidth}{0.9\paperheight}}
    \usepackage{float}
    \floatplacement{figure}{H} % forces figures to be placed at the correct location
    \usepackage{xcolor} % Allow colors to be defined
    \usepackage{enumerate} % Needed for markdown enumerations to work
    \usepackage{geometry} % Used to adjust the document margins
    \usepackage{amsmath} % Equations
    \usepackage{amssymb} % Equations
    \usepackage{textcomp} % defines textquotesingle
    % Hack from http://tex.stackexchange.com/a/47451/13684:
    \AtBeginDocument{%
        \def\PYZsq{\textquotesingle}% Upright quotes in Pygmentized code
    }
    \usepackage{upquote} % Upright quotes for verbatim code
    \usepackage{eurosym} % defines \euro
    \usepackage[mathletters]{ucs} % Extended unicode (utf-8) support
    \usepackage{fancyvrb} % verbatim replacement that allows latex
    \usepackage{grffile} % extends the file name processing of package graphics 
                         % to support a larger range
    \makeatletter % fix for grffile with XeLaTeX
    \def\Gread@@xetex#1{%
      \IfFileExists{"\Gin@base".bb}%
      {\Gread@eps{\Gin@base.bb}}%
      {\Gread@@xetex@aux#1}%
    }
    \makeatother

    % The hyperref package gives us a pdf with properly built
    % internal navigation ('pdf bookmarks' for the table of contents,
    % internal cross-reference links, web links for URLs, etc.)
    \usepackage{hyperref}
    % The default LaTeX title has an obnoxious amount of whitespace. By default,
    % titling removes some of it. It also provides customization options.
    \usepackage{titling}
    \usepackage{longtable} % longtable support required by pandoc >1.10
    \usepackage{booktabs}  % table support for pandoc > 1.12.2
    \usepackage[inline]{enumitem} % IRkernel/repr support (it uses the enumerate* environment)
    \usepackage[normalem]{ulem} % ulem is needed to support strikethroughs (\sout)
                                % normalem makes italics be italics, not underlines
    \usepackage{mathrsfs}
    

    
    % Colors for the hyperref package
    \definecolor{urlcolor}{rgb}{0,.145,.698}
    \definecolor{linkcolor}{rgb}{.71,0.21,0.01}
    \definecolor{citecolor}{rgb}{.12,.54,.11}

    % ANSI colors
    \definecolor{ansi-black}{HTML}{3E424D}
    \definecolor{ansi-black-intense}{HTML}{282C36}
    \definecolor{ansi-red}{HTML}{E75C58}
    \definecolor{ansi-red-intense}{HTML}{B22B31}
    \definecolor{ansi-green}{HTML}{00A250}
    \definecolor{ansi-green-intense}{HTML}{007427}
    \definecolor{ansi-yellow}{HTML}{DDB62B}
    \definecolor{ansi-yellow-intense}{HTML}{B27D12}
    \definecolor{ansi-blue}{HTML}{208FFB}
    \definecolor{ansi-blue-intense}{HTML}{0065CA}
    \definecolor{ansi-magenta}{HTML}{D160C4}
    \definecolor{ansi-magenta-intense}{HTML}{A03196}
    \definecolor{ansi-cyan}{HTML}{60C6C8}
    \definecolor{ansi-cyan-intense}{HTML}{258F8F}
    \definecolor{ansi-white}{HTML}{C5C1B4}
    \definecolor{ansi-white-intense}{HTML}{A1A6B2}
    \definecolor{ansi-default-inverse-fg}{HTML}{FFFFFF}
    \definecolor{ansi-default-inverse-bg}{HTML}{000000}

    % commands and environments needed by pandoc snippets
    % extracted from the output of `pandoc -s`
    \providecommand{\tightlist}{%
      \setlength{\itemsep}{0pt}\setlength{\parskip}{0pt}}
    \DefineVerbatimEnvironment{Highlighting}{Verbatim}{commandchars=\\\{\}}
    % Add ',fontsize=\small' for more characters per line
    \newenvironment{Shaded}{}{}
    \newcommand{\KeywordTok}[1]{\textcolor[rgb]{0.00,0.44,0.13}{\textbf{{#1}}}}
    \newcommand{\DataTypeTok}[1]{\textcolor[rgb]{0.56,0.13,0.00}{{#1}}}
    \newcommand{\DecValTok}[1]{\textcolor[rgb]{0.25,0.63,0.44}{{#1}}}
    \newcommand{\BaseNTok}[1]{\textcolor[rgb]{0.25,0.63,0.44}{{#1}}}
    \newcommand{\FloatTok}[1]{\textcolor[rgb]{0.25,0.63,0.44}{{#1}}}
    \newcommand{\CharTok}[1]{\textcolor[rgb]{0.25,0.44,0.63}{{#1}}}
    \newcommand{\StringTok}[1]{\textcolor[rgb]{0.25,0.44,0.63}{{#1}}}
    \newcommand{\CommentTok}[1]{\textcolor[rgb]{0.38,0.63,0.69}{\textit{{#1}}}}
    \newcommand{\OtherTok}[1]{\textcolor[rgb]{0.00,0.44,0.13}{{#1}}}
    \newcommand{\AlertTok}[1]{\textcolor[rgb]{1.00,0.00,0.00}{\textbf{{#1}}}}
    \newcommand{\FunctionTok}[1]{\textcolor[rgb]{0.02,0.16,0.49}{{#1}}}
    \newcommand{\RegionMarkerTok}[1]{{#1}}
    \newcommand{\ErrorTok}[1]{\textcolor[rgb]{1.00,0.00,0.00}{\textbf{{#1}}}}
    \newcommand{\NormalTok}[1]{{#1}}
    
    % Additional commands for more recent versions of Pandoc
    \newcommand{\ConstantTok}[1]{\textcolor[rgb]{0.53,0.00,0.00}{{#1}}}
    \newcommand{\SpecialCharTok}[1]{\textcolor[rgb]{0.25,0.44,0.63}{{#1}}}
    \newcommand{\VerbatimStringTok}[1]{\textcolor[rgb]{0.25,0.44,0.63}{{#1}}}
    \newcommand{\SpecialStringTok}[1]{\textcolor[rgb]{0.73,0.40,0.53}{{#1}}}
    \newcommand{\ImportTok}[1]{{#1}}
    \newcommand{\DocumentationTok}[1]{\textcolor[rgb]{0.73,0.13,0.13}{\textit{{#1}}}}
    \newcommand{\AnnotationTok}[1]{\textcolor[rgb]{0.38,0.63,0.69}{\textbf{\textit{{#1}}}}}
    \newcommand{\CommentVarTok}[1]{\textcolor[rgb]{0.38,0.63,0.69}{\textbf{\textit{{#1}}}}}
    \newcommand{\VariableTok}[1]{\textcolor[rgb]{0.10,0.09,0.49}{{#1}}}
    \newcommand{\ControlFlowTok}[1]{\textcolor[rgb]{0.00,0.44,0.13}{\textbf{{#1}}}}
    \newcommand{\OperatorTok}[1]{\textcolor[rgb]{0.40,0.40,0.40}{{#1}}}
    \newcommand{\BuiltInTok}[1]{{#1}}
    \newcommand{\ExtensionTok}[1]{{#1}}
    \newcommand{\PreprocessorTok}[1]{\textcolor[rgb]{0.74,0.48,0.00}{{#1}}}
    \newcommand{\AttributeTok}[1]{\textcolor[rgb]{0.49,0.56,0.16}{{#1}}}
    \newcommand{\InformationTok}[1]{\textcolor[rgb]{0.38,0.63,0.69}{\textbf{\textit{{#1}}}}}
    \newcommand{\WarningTok}[1]{\textcolor[rgb]{0.38,0.63,0.69}{\textbf{\textit{{#1}}}}}
    
    
    % Define a nice break command that doesn't care if a line doesn't already
    % exist.
    \def\br{\hspace*{\fill} \\* }
    % Math Jax compatibility definitions
    \def\gt{>}
    \def\lt{<}
    \let\Oldtex\TeX
    \let\Oldlatex\LaTeX
    \renewcommand{\TeX}{\textrm{\Oldtex}}
    \renewcommand{\LaTeX}{\textrm{\Oldlatex}}
    % Document parameters
    % Document title
    \title{Лекция 9 \\
    Обусловленность СЛАУ
    }
    \date{3 ноября 2021\,г.}
    
    
    
    
    
% Pygments definitions
\makeatletter
\def\PY@reset{\let\PY@it=\relax \let\PY@bf=\relax%
    \let\PY@ul=\relax \let\PY@tc=\relax%
    \let\PY@bc=\relax \let\PY@ff=\relax}
\def\PY@tok#1{\csname PY@tok@#1\endcsname}
\def\PY@toks#1+{\ifx\relax#1\empty\else%
    \PY@tok{#1}\expandafter\PY@toks\fi}
\def\PY@do#1{\PY@bc{\PY@tc{\PY@ul{%
    \PY@it{\PY@bf{\PY@ff{#1}}}}}}}
\def\PY#1#2{\PY@reset\PY@toks#1+\relax+\PY@do{#2}}

\expandafter\def\csname PY@tok@w\endcsname{\def\PY@tc##1{\textcolor[rgb]{0.73,0.73,0.73}{##1}}}
\expandafter\def\csname PY@tok@c\endcsname{\let\PY@it=\textit\def\PY@tc##1{\textcolor[rgb]{0.25,0.50,0.50}{##1}}}
\expandafter\def\csname PY@tok@cp\endcsname{\def\PY@tc##1{\textcolor[rgb]{0.74,0.48,0.00}{##1}}}
\expandafter\def\csname PY@tok@k\endcsname{\let\PY@bf=\textbf\def\PY@tc##1{\textcolor[rgb]{0.00,0.50,0.00}{##1}}}
\expandafter\def\csname PY@tok@kp\endcsname{\def\PY@tc##1{\textcolor[rgb]{0.00,0.50,0.00}{##1}}}
\expandafter\def\csname PY@tok@kt\endcsname{\def\PY@tc##1{\textcolor[rgb]{0.69,0.00,0.25}{##1}}}
\expandafter\def\csname PY@tok@o\endcsname{\def\PY@tc##1{\textcolor[rgb]{0.40,0.40,0.40}{##1}}}
\expandafter\def\csname PY@tok@ow\endcsname{\let\PY@bf=\textbf\def\PY@tc##1{\textcolor[rgb]{0.67,0.13,1.00}{##1}}}
\expandafter\def\csname PY@tok@nb\endcsname{\def\PY@tc##1{\textcolor[rgb]{0.00,0.50,0.00}{##1}}}
\expandafter\def\csname PY@tok@nf\endcsname{\def\PY@tc##1{\textcolor[rgb]{0.00,0.00,1.00}{##1}}}
\expandafter\def\csname PY@tok@nc\endcsname{\let\PY@bf=\textbf\def\PY@tc##1{\textcolor[rgb]{0.00,0.00,1.00}{##1}}}
\expandafter\def\csname PY@tok@nn\endcsname{\let\PY@bf=\textbf\def\PY@tc##1{\textcolor[rgb]{0.00,0.00,1.00}{##1}}}
\expandafter\def\csname PY@tok@ne\endcsname{\let\PY@bf=\textbf\def\PY@tc##1{\textcolor[rgb]{0.82,0.25,0.23}{##1}}}
\expandafter\def\csname PY@tok@nv\endcsname{\def\PY@tc##1{\textcolor[rgb]{0.10,0.09,0.49}{##1}}}
\expandafter\def\csname PY@tok@no\endcsname{\def\PY@tc##1{\textcolor[rgb]{0.53,0.00,0.00}{##1}}}
\expandafter\def\csname PY@tok@nl\endcsname{\def\PY@tc##1{\textcolor[rgb]{0.63,0.63,0.00}{##1}}}
\expandafter\def\csname PY@tok@ni\endcsname{\let\PY@bf=\textbf\def\PY@tc##1{\textcolor[rgb]{0.60,0.60,0.60}{##1}}}
\expandafter\def\csname PY@tok@na\endcsname{\def\PY@tc##1{\textcolor[rgb]{0.49,0.56,0.16}{##1}}}
\expandafter\def\csname PY@tok@nt\endcsname{\let\PY@bf=\textbf\def\PY@tc##1{\textcolor[rgb]{0.00,0.50,0.00}{##1}}}
\expandafter\def\csname PY@tok@nd\endcsname{\def\PY@tc##1{\textcolor[rgb]{0.67,0.13,1.00}{##1}}}
\expandafter\def\csname PY@tok@s\endcsname{\def\PY@tc##1{\textcolor[rgb]{0.73,0.13,0.13}{##1}}}
\expandafter\def\csname PY@tok@sd\endcsname{\let\PY@it=\textit\def\PY@tc##1{\textcolor[rgb]{0.73,0.13,0.13}{##1}}}
\expandafter\def\csname PY@tok@si\endcsname{\let\PY@bf=\textbf\def\PY@tc##1{\textcolor[rgb]{0.73,0.40,0.53}{##1}}}
\expandafter\def\csname PY@tok@se\endcsname{\let\PY@bf=\textbf\def\PY@tc##1{\textcolor[rgb]{0.73,0.40,0.13}{##1}}}
\expandafter\def\csname PY@tok@sr\endcsname{\def\PY@tc##1{\textcolor[rgb]{0.73,0.40,0.53}{##1}}}
\expandafter\def\csname PY@tok@ss\endcsname{\def\PY@tc##1{\textcolor[rgb]{0.10,0.09,0.49}{##1}}}
\expandafter\def\csname PY@tok@sx\endcsname{\def\PY@tc##1{\textcolor[rgb]{0.00,0.50,0.00}{##1}}}
\expandafter\def\csname PY@tok@m\endcsname{\def\PY@tc##1{\textcolor[rgb]{0.40,0.40,0.40}{##1}}}
\expandafter\def\csname PY@tok@gh\endcsname{\let\PY@bf=\textbf\def\PY@tc##1{\textcolor[rgb]{0.00,0.00,0.50}{##1}}}
\expandafter\def\csname PY@tok@gu\endcsname{\let\PY@bf=\textbf\def\PY@tc##1{\textcolor[rgb]{0.50,0.00,0.50}{##1}}}
\expandafter\def\csname PY@tok@gd\endcsname{\def\PY@tc##1{\textcolor[rgb]{0.63,0.00,0.00}{##1}}}
\expandafter\def\csname PY@tok@gi\endcsname{\def\PY@tc##1{\textcolor[rgb]{0.00,0.63,0.00}{##1}}}
\expandafter\def\csname PY@tok@gr\endcsname{\def\PY@tc##1{\textcolor[rgb]{1.00,0.00,0.00}{##1}}}
\expandafter\def\csname PY@tok@ge\endcsname{\let\PY@it=\textit}
\expandafter\def\csname PY@tok@gs\endcsname{\let\PY@bf=\textbf}
\expandafter\def\csname PY@tok@gp\endcsname{\let\PY@bf=\textbf\def\PY@tc##1{\textcolor[rgb]{0.00,0.00,0.50}{##1}}}
\expandafter\def\csname PY@tok@go\endcsname{\def\PY@tc##1{\textcolor[rgb]{0.53,0.53,0.53}{##1}}}
\expandafter\def\csname PY@tok@gt\endcsname{\def\PY@tc##1{\textcolor[rgb]{0.00,0.27,0.87}{##1}}}
\expandafter\def\csname PY@tok@err\endcsname{\def\PY@bc##1{\setlength{\fboxsep}{0pt}\fcolorbox[rgb]{1.00,0.00,0.00}{1,1,1}{\strut ##1}}}
\expandafter\def\csname PY@tok@kc\endcsname{\let\PY@bf=\textbf\def\PY@tc##1{\textcolor[rgb]{0.00,0.50,0.00}{##1}}}
\expandafter\def\csname PY@tok@kd\endcsname{\let\PY@bf=\textbf\def\PY@tc##1{\textcolor[rgb]{0.00,0.50,0.00}{##1}}}
\expandafter\def\csname PY@tok@kn\endcsname{\let\PY@bf=\textbf\def\PY@tc##1{\textcolor[rgb]{0.00,0.50,0.00}{##1}}}
\expandafter\def\csname PY@tok@kr\endcsname{\let\PY@bf=\textbf\def\PY@tc##1{\textcolor[rgb]{0.00,0.50,0.00}{##1}}}
\expandafter\def\csname PY@tok@bp\endcsname{\def\PY@tc##1{\textcolor[rgb]{0.00,0.50,0.00}{##1}}}
\expandafter\def\csname PY@tok@fm\endcsname{\def\PY@tc##1{\textcolor[rgb]{0.00,0.00,1.00}{##1}}}
\expandafter\def\csname PY@tok@vc\endcsname{\def\PY@tc##1{\textcolor[rgb]{0.10,0.09,0.49}{##1}}}
\expandafter\def\csname PY@tok@vg\endcsname{\def\PY@tc##1{\textcolor[rgb]{0.10,0.09,0.49}{##1}}}
\expandafter\def\csname PY@tok@vi\endcsname{\def\PY@tc##1{\textcolor[rgb]{0.10,0.09,0.49}{##1}}}
\expandafter\def\csname PY@tok@vm\endcsname{\def\PY@tc##1{\textcolor[rgb]{0.10,0.09,0.49}{##1}}}
\expandafter\def\csname PY@tok@sa\endcsname{\def\PY@tc##1{\textcolor[rgb]{0.73,0.13,0.13}{##1}}}
\expandafter\def\csname PY@tok@sb\endcsname{\def\PY@tc##1{\textcolor[rgb]{0.73,0.13,0.13}{##1}}}
\expandafter\def\csname PY@tok@sc\endcsname{\def\PY@tc##1{\textcolor[rgb]{0.73,0.13,0.13}{##1}}}
\expandafter\def\csname PY@tok@dl\endcsname{\def\PY@tc##1{\textcolor[rgb]{0.73,0.13,0.13}{##1}}}
\expandafter\def\csname PY@tok@s2\endcsname{\def\PY@tc##1{\textcolor[rgb]{0.73,0.13,0.13}{##1}}}
\expandafter\def\csname PY@tok@sh\endcsname{\def\PY@tc##1{\textcolor[rgb]{0.73,0.13,0.13}{##1}}}
\expandafter\def\csname PY@tok@s1\endcsname{\def\PY@tc##1{\textcolor[rgb]{0.73,0.13,0.13}{##1}}}
\expandafter\def\csname PY@tok@mb\endcsname{\def\PY@tc##1{\textcolor[rgb]{0.40,0.40,0.40}{##1}}}
\expandafter\def\csname PY@tok@mf\endcsname{\def\PY@tc##1{\textcolor[rgb]{0.40,0.40,0.40}{##1}}}
\expandafter\def\csname PY@tok@mh\endcsname{\def\PY@tc##1{\textcolor[rgb]{0.40,0.40,0.40}{##1}}}
\expandafter\def\csname PY@tok@mi\endcsname{\def\PY@tc##1{\textcolor[rgb]{0.40,0.40,0.40}{##1}}}
\expandafter\def\csname PY@tok@il\endcsname{\def\PY@tc##1{\textcolor[rgb]{0.40,0.40,0.40}{##1}}}
\expandafter\def\csname PY@tok@mo\endcsname{\def\PY@tc##1{\textcolor[rgb]{0.40,0.40,0.40}{##1}}}
\expandafter\def\csname PY@tok@ch\endcsname{\let\PY@it=\textit\def\PY@tc##1{\textcolor[rgb]{0.25,0.50,0.50}{##1}}}
\expandafter\def\csname PY@tok@cm\endcsname{\let\PY@it=\textit\def\PY@tc##1{\textcolor[rgb]{0.25,0.50,0.50}{##1}}}
\expandafter\def\csname PY@tok@cpf\endcsname{\let\PY@it=\textit\def\PY@tc##1{\textcolor[rgb]{0.25,0.50,0.50}{##1}}}
\expandafter\def\csname PY@tok@c1\endcsname{\let\PY@it=\textit\def\PY@tc##1{\textcolor[rgb]{0.25,0.50,0.50}{##1}}}
\expandafter\def\csname PY@tok@cs\endcsname{\let\PY@it=\textit\def\PY@tc##1{\textcolor[rgb]{0.25,0.50,0.50}{##1}}}

\def\PYZbs{\char`\\}
\def\PYZus{\char`\_}
\def\PYZob{\char`\{}
\def\PYZcb{\char`\}}
\def\PYZca{\char`\^}
\def\PYZam{\char`\&}
\def\PYZlt{\char`\<}
\def\PYZgt{\char`\>}
\def\PYZsh{\char`\#}
\def\PYZpc{\char`\%}
\def\PYZdl{\char`\$}
\def\PYZhy{\char`\-}
\def\PYZsq{\char`\'}
\def\PYZdq{\char`\"}
\def\PYZti{\char`\~}
% for compatibility with earlier versions
\def\PYZat{@}
\def\PYZlb{[}
\def\PYZrb{]}
\makeatother


    % For linebreaks inside Verbatim environment from package fancyvrb. 
    \makeatletter
        \newbox\Wrappedcontinuationbox 
        \newbox\Wrappedvisiblespacebox 
        \newcommand*\Wrappedvisiblespace {\textcolor{red}{\textvisiblespace}} 
        \newcommand*\Wrappedcontinuationsymbol {\textcolor{red}{\llap{\tiny$\m@th\hookrightarrow$}}} 
        \newcommand*\Wrappedcontinuationindent {3ex } 
        \newcommand*\Wrappedafterbreak {\kern\Wrappedcontinuationindent\copy\Wrappedcontinuationbox} 
        % Take advantage of the already applied Pygments mark-up to insert 
        % potential linebreaks for TeX processing. 
        %        {, <, #, %, $, ' and ": go to next line. 
        %        _, }, ^, &, >, - and ~: stay at end of broken line. 
        % Use of \textquotesingle for straight quote. 
        \newcommand*\Wrappedbreaksatspecials {% 
            \def\PYGZus{\discretionary{\char`\_}{\Wrappedafterbreak}{\char`\_}}% 
            \def\PYGZob{\discretionary{}{\Wrappedafterbreak\char`\{}{\char`\{}}% 
            \def\PYGZcb{\discretionary{\char`\}}{\Wrappedafterbreak}{\char`\}}}% 
            \def\PYGZca{\discretionary{\char`\^}{\Wrappedafterbreak}{\char`\^}}% 
            \def\PYGZam{\discretionary{\char`\&}{\Wrappedafterbreak}{\char`\&}}% 
            \def\PYGZlt{\discretionary{}{\Wrappedafterbreak\char`\<}{\char`\<}}% 
            \def\PYGZgt{\discretionary{\char`\>}{\Wrappedafterbreak}{\char`\>}}% 
            \def\PYGZsh{\discretionary{}{\Wrappedafterbreak\char`\#}{\char`\#}}% 
            \def\PYGZpc{\discretionary{}{\Wrappedafterbreak\char`\%}{\char`\%}}% 
            \def\PYGZdl{\discretionary{}{\Wrappedafterbreak\char`\$}{\char`\$}}% 
            \def\PYGZhy{\discretionary{\char`\-}{\Wrappedafterbreak}{\char`\-}}% 
            \def\PYGZsq{\discretionary{}{\Wrappedafterbreak\textquotesingle}{\textquotesingle}}% 
            \def\PYGZdq{\discretionary{}{\Wrappedafterbreak\char`\"}{\char`\"}}% 
            \def\PYGZti{\discretionary{\char`\~}{\Wrappedafterbreak}{\char`\~}}% 
        } 
        % Some characters . , ; ? ! / are not pygmentized. 
        % This macro makes them "active" and they will insert potential linebreaks 
        \newcommand*\Wrappedbreaksatpunct {% 
            \lccode`\~`\.\lowercase{\def~}{\discretionary{\hbox{\char`\.}}{\Wrappedafterbreak}{\hbox{\char`\.}}}% 
            \lccode`\~`\,\lowercase{\def~}{\discretionary{\hbox{\char`\,}}{\Wrappedafterbreak}{\hbox{\char`\,}}}% 
            \lccode`\~`\;\lowercase{\def~}{\discretionary{\hbox{\char`\;}}{\Wrappedafterbreak}{\hbox{\char`\;}}}% 
            \lccode`\~`\:\lowercase{\def~}{\discretionary{\hbox{\char`\:}}{\Wrappedafterbreak}{\hbox{\char`\:}}}% 
            \lccode`\~`\?\lowercase{\def~}{\discretionary{\hbox{\char`\?}}{\Wrappedafterbreak}{\hbox{\char`\?}}}% 
            \lccode`\~`\!\lowercase{\def~}{\discretionary{\hbox{\char`\!}}{\Wrappedafterbreak}{\hbox{\char`\!}}}% 
            \lccode`\~`\/\lowercase{\def~}{\discretionary{\hbox{\char`\/}}{\Wrappedafterbreak}{\hbox{\char`\/}}}% 
            \catcode`\.\active
            \catcode`\,\active 
            \catcode`\;\active
            \catcode`\:\active
            \catcode`\?\active
            \catcode`\!\active
            \catcode`\/\active 
            \lccode`\~`\~ 	
        }
    \makeatother

    \let\OriginalVerbatim=\Verbatim
    \makeatletter
    \renewcommand{\Verbatim}[1][1]{%
        %\parskip\z@skip
        \sbox\Wrappedcontinuationbox {\Wrappedcontinuationsymbol}%
        \sbox\Wrappedvisiblespacebox {\FV@SetupFont\Wrappedvisiblespace}%
        \def\FancyVerbFormatLine ##1{\hsize\linewidth
            \vtop{\raggedright\hyphenpenalty\z@\exhyphenpenalty\z@
                \doublehyphendemerits\z@\finalhyphendemerits\z@
                \strut ##1\strut}%
        }%
        % If the linebreak is at a space, the latter will be displayed as visible
        % space at end of first line, and a continuation symbol starts next line.
        % Stretch/shrink are however usually zero for typewriter font.
        \def\FV@Space {%
            \nobreak\hskip\z@ plus\fontdimen3\font minus\fontdimen4\font
            \discretionary{\copy\Wrappedvisiblespacebox}{\Wrappedafterbreak}
            {\kern\fontdimen2\font}%
        }%
        
        % Allow breaks at special characters using \PYG... macros.
        \Wrappedbreaksatspecials
        % Breaks at punctuation characters . , ; ? ! and / need catcode=\active 	
        \OriginalVerbatim[#1,codes*=\Wrappedbreaksatpunct]%
    }
    \makeatother

    % Exact colors from NB
    \definecolor{incolor}{HTML}{303F9F}
    \definecolor{outcolor}{HTML}{D84315}
    \definecolor{cellborder}{HTML}{CFCFCF}
    \definecolor{cellbackground}{HTML}{F7F7F7}
    
    % prompt
    \makeatletter
    \newcommand{\boxspacing}{\kern\kvtcb@left@rule\kern\kvtcb@boxsep}
    \makeatother
    \newcommand{\prompt}[4]{
        \ttfamily\llap{{\color{#2}[#3]:\hspace{3pt}#4}}\vspace{-\baselineskip}
    }
    

    
    % Prevent overflowing lines due to hard-to-break entities
    \sloppy 
    % Setup hyperref package
    \hypersetup{
      breaklinks=true,  % so long urls are correctly broken across lines
      colorlinks=true,
      urlcolor=urlcolor,
      linkcolor=linkcolor,
      citecolor=citecolor,
      }
    % Slightly bigger margins than the latex defaults
    
    \geometry{verbose,tmargin=1in,bmargin=1in,lmargin=1in,rmargin=1in}
    
    

\begin{document}
    
\maketitle
\thispagestyle{empty}
\tableofcontents
\pagebreak



    \hypertarget{ux441ux443ux442ux44c-ux43fux440ux43eux431ux43bux435ux43cux44b}{%
\section{Суть
проблемы}\label{ux441ux443ux442ux44c-ux43fux440ux43eux431ux43bux435ux43cux44b}}

Пусть дана исходная система линейных уравнений
\[ A \mathbf{x} = \mathbf{b}, \] где \(A\) --- квадратная невырожденная
матрица порядка \(n\).

В этом случае система имеет единственное решение
\(\mathbf{x} = A^{-1} \mathbf{b}\).

    Решение системы линейных уравнений можно интерпретировать, как
разложение вектора \(\mathbf{b}\) по вектор-столбцам матрицы \(A\).

Рассмотрим матрицу \[
  A = 
  \begin{pmatrix}
    1.0 & 1.0 \\
    1.0 & 1.2
  \end{pmatrix}.
\]

Её столбцы, практически, коллинеарны. Из общих соображений понятно, что
выбирать эти столбцы в качестве базисных --- плохая идея. Разложение
любого вектора по таким столбцам может привести к ошибкам.


    \begin{center}
    \adjustimage{max size={0.4\linewidth}{0.4\paperheight}}{output_8_0.png}
    \end{center}
    { \hspace*{\fill} \\}
    
    Посмотрим, как появляются эти ошибки.

Пусть вектор \(\mathbf{b}\) известен не точно, а с некоторой
погрешностью. Посмотрим, какова будет погрешность решения системы ---
вектора \(\mathbf{x}\).

    \begin{quote}
Здесь и далее на рисунках вектор правых частей изображается на
\emph{правом} рисунке («После преобразования»), а вектор решения --- на
\emph{левом} («До преобразования»).
\end{quote}



    \begin{center}
    \adjustimage{max size={0.9\linewidth}{0.9\paperheight}}{output_12_0.png}
    \end{center}
    { \hspace*{\fill} \\}
    

    \begin{Verbatim}[commandchars=\\\{\}]
Максимальное относительное увеличение возмущения max(dx/x : db/b) =  14.88
    \end{Verbatim}

    Мы видим, что небольшое возмущение вектора правой части \(\mathbf{b}\)
привела к гораздо большим (почти в 15 раз) возмущениям вектора решений
\(\mathbf{x}\). В таком случае говорят, что система уравнений является
\emph{плохо обусловленной}. Разберёмся подробнее, что это значит.

    \begin{quote}
\textbf{Самостоятельно.} Проблема в том, что мы пытаемся разложить
вектор \(\mathbf{b}\) по базису, векторы которого почти коллинеарны.
Тогда почему бы нам не попытаться ортогонализовать наш базис, применив,
например, \(QR\)-разложение?
\end{quote}

    \begin{center}\rule{0.5\linewidth}{\linethickness}\end{center}

    \hypertarget{ux43eux431ux443ux441ux43bux43eux432ux43bux435ux43dux43dux43eux441ux442ux44c}{%
\section{Обусловленность}\label{ux43eux431ux443ux441ux43bux43eux432ux43bux435ux43dux43dux43eux441ux442ux44c}}

    \hypertarget{ux447ux438ux441ux43bux43e-ux43eux431ux443ux441ux43bux43eux432ux43bux435ux43dux43dux43eux441ux442ux438}{%
\subsection{Число
обусловленности}\label{ux447ux438ux441ux43bux43e-ux43eux431ux443ux441ux43bux43eux432ux43bux435ux43dux43dux43eux441ux442ux438}}

Рассмотрим возмущённую систему
\[ A \tilde{\mathbf{x}} = \mathbf{b} + \delta\mathbf{b}. \]

Введём вектор возмущения решения
\(\delta \mathbf{x} = \tilde{\mathbf{x}} - \mathbf{x}\).

Выразим \(\mathbf{x}\) и \(\delta\mathbf{x}\) через обратную матрицу
\(A^{-1}\):
\[ \delta \mathbf{x} = A^{-1}(\mathbf{b} + \delta\mathbf{b}) - A^{-1}\mathbf{b} = A^{-1}\delta\mathbf{b}. \]

Перейдём к оценке нормы возмущения:
\[ \|\delta \mathbf{x}\| = \|A^{-1} \delta \mathbf{b}\| \le \|A^{-1}\| \cdot \|\delta \mathbf{b}\|. \]

Учитывая, что \(\|\mathbf{b}\| \le \|A\| \cdot \|\mathbf{x}\|\), усилим
неравенство, умножим правую часть на
\(\|A\| \cdot \dfrac{\|\mathbf{x}\|}{\|\mathbf{b}\|}\):
\[ \|\delta \mathbf{x}\| \le \|A^{-1}\| \|A\| \|\mathbf{x}\| \frac{\|\delta \mathbf{b}\|}{\|\mathbf{b}\|}. \]

Наконец, разделим на \(\|\mathbf{x}\|\):
\[ \frac{\|\delta \mathbf{x}\|}{\|\mathbf{x}\|} \le \|A^{-1}\| \|A\| \frac{\|\delta \mathbf{b}\|}{\|\mathbf{b}\|}. \]

Мы получили связь между относительной погрешностью решения и
относительной погрешностью правой части системы уравнений.

    \textbf{Определение.} Величина \(\mu(A) = \|A^{-1}\| \|A\|\) называется
\emph{числом обусловленности} матрицы \(A\) в рассматриваемой норме. Она
показывает, во сколько раз может возрасти относительная погрешность
решения по сравнению с относительной погрешностью правой части.

Число обусловленности определяется не только матрицей, но и выбором
нормы. Рассмотрим один из наиболее употребительных вариантов ---
\emph{спектральное число обусловленоности}. Согласно формуле для
спектральной нормы матрицы \[ \mu(A) = \dfrac{\sigma_1}{\sigma_n}, \]
где \(\sigma_1\) и \(\sigma_n\) --- максимальное и минимальное
сингулярные числа матрицы \(A\).


    \begin{Verbatim}[commandchars=\\\{\}]
sigma =  [2.105 0.095]
mu(A) =  22.15
    \end{Verbatim}

    \begin{quote}
\textbf{Самостоятельно.} В нашем примере число обусловленности
\(\mu(A)=22.15\). Но выше мы нашли, что относительная погрешность
увеличилась в \(14.88\) раз. Почему так произошло? При каком условии
оценка, сделанная по числу обусловленности, будет достигаться?\\
Как, выбрав вектор \(\mathbf{b}\), сделать для него более точную оценку
возрастания относительной погрешности?
\end{quote}

    \hypertarget{ux433ux435ux43eux43cux435ux442ux440ux438ux447ux435ux441ux43aux430ux44f-ux438ux43dux442ux435ux440ux43fux440ux435ux442ux430ux446ux438ux44f}{%
\subsection{Геометрическая
интерпретация}\label{ux433ux435ux43eux43cux435ux442ux440ux438ux447ux435ux441ux43aux430ux44f-ux438ux43dux442ux435ux440ux43fux440ux435ux442ux430ux446ux438ux44f}}

Дадим геометрическую интерпретацию числа обусловленности.


    \begin{center}
    \adjustimage{max size={0.9\linewidth}{0.9\paperheight}}{output_24_0.png}
    \end{center}
    { \hspace*{\fill} \\}
    
    Для двумерного случая мы видим, что если векторы правой части
возмущённой системы лежат внутри окружности, то решения возмущённой
системы лежат внутри эллипса, являющегося прообразом этой окружности.
Причём отношение полуосей этого эллипса равно \emph{спектральному числу
обусловленности}.

    \begin{center}\rule{0.5\linewidth}{\linethickness}\end{center}

    \hypertarget{ux43fux440ux438ux43cux435ux440-ux43fux43eux43bux438ux43dux43eux43cux438ux430ux43bux44cux43dux430ux44f-ux440ux435ux433ux440ux435ux441ux441ux438ux44f}{%
\section{Линейная
регрессия}\label{ux43fux440ux438ux43cux435ux440-ux43fux43eux43bux438ux43dux43eux43cux438ux430ux43bux44cux43dux430ux44f-ux440ux435ux433ux440ux435ux441ux441ux438ux44f}}

\hypertarget{ux433ux435ux43dux435ux440ux430ux446ux438ux44f-ux434ux430ux43dux43dux44bux445}{%
\subsection{Генерация
данных}\label{ux433ux435ux43dux435ux440ux430ux446ux438ux44f-ux434ux430ux43dux43dux44bux445}}



    \begin{center}
    \adjustimage{max size={0.75\linewidth}{0.9\paperheight}}{output_29_0.png}
    \end{center}
    { \hspace*{\fill} \\}
    
    \hypertarget{ux43fux43eux438ux441ux43a-ux43aux43eux44dux444ux444ux438ux446ux438ux435ux43dux442ux43eux432-ux440ux435ux433ux440ux435ux441ux441ux438ux438}{%
\subsection{Поиск коэффициентов
регрессии}\label{ux43fux43eux438ux441ux43a-ux43aux43eux44dux444ux444ux438ux446ux438ux435ux43dux442ux43eux432-ux440ux435ux433ux440ux435ux441ux441ux438ux438}}


    \begin{center}
    \adjustimage{max size={0.75\linewidth}{0.9\paperheight}}{output_32_0.png}
    \end{center}
    { \hspace*{\fill} \\}
    
    \hypertarget{ux441ux438ux43dux433ux443ux43bux44fux440ux43dux43eux435-ux440ux430ux437ux43bux43eux436ux435ux43dux438ux435}{%
\subsection{Сингулярное
разложение}\label{ux441ux438ux43dux433ux443ux43bux44fux440ux43dux43eux435-ux440ux430ux437ux43bux43eux436ux435ux43dux438ux435}}

Рассмотрим матрицу \(F\) размерностью \(m \times n\).
Для определённости будем считать, что строк не меньше чем столбцов и столбцы линейно
независимы, т.е. \(n \le m\) и \(\mathrm{Rg}F = n\).

Представим матрицу \(F\) в виде сингулярного разложения
\[ F = U \Sigma V^\top. \]

Имея сингулярное разложение, легко записать

\begin{itemize}
\item
  псевдообратную матрицу:
  \[ F^{+} = (F^\top F)^{-1}F^\top = (V \Sigma U^\top U \Sigma V^\top)^{-1} \cdot V \Sigma U^\top = V \Sigma^{-1}U^\top = \sum_{j=1}^n \frac{1}{{\sigma_j} }v_j u_j^\top;  \label{eq:psevdo}\tag{1} \]
\item
  вектор МНК-решения:
  \[ \alpha^* = F^{+} y  = V \Sigma^{-1}U^\top y = \sum_{j=1}^n \frac{1}{{\sigma_j}}v_j (u_j^\top y);  \label{eq:alpha-res}\tag{2} \]
\item
  вектор \(F\alpha^*\) --- МНК-аппроксимацию целевого вектора \(y\):
  \[ F\alpha^* = P_F y = FF^{+}y = U \Sigma V^\top \cdot V \Sigma^{-1}U^\top y = UU^\top y = \sum_{j=1}^n u_j (u_j^\top y);  \label{eq:F-alpha-res}\tag{3} \]
\item
  норму вектора коэффициентов:
  \[ \Vert \alpha^* \Vert^2 = y^\top U \Sigma^{-1}V^\top \cdot V \Sigma^{-1}U^\top y = y^\top U \Sigma^{-2}U^\top y = \sum_{j=1}^n \frac{1}{\sigma_j^2} (u_j^\top y)^2.  \label{eq:alpha-res-norm}\tag{4} \]
\end{itemize}

    \begin{center}\rule{0.5\linewidth}{\linethickness}\end{center}

    \hypertarget{ux43fux440ux43eux431ux43bux435ux43cux430-ux43cux443ux43bux44cux442ux438ux43aux43eux43bux43bux438ux43dux435ux430ux440ux43dux43eux441ux442ux438}{%
\section{Проблема
мультиколлинеарности}\label{ux43fux440ux43eux431ux43bux435ux43cux430-ux43cux443ux43bux44cux442ux438ux43aux43eux43bux43bux438ux43dux435ux430ux440ux43dux43eux441ux442ux438}}

Если ковариационная матрица \(K = F^\top F\) имеет неполный ранг, то её
обращение невозможно. Тогда приходится отбрасывать линейно зависимые
признаки или применять описанные ниже методы --- регуляризацию или метод
главных компонент. На практике чаще встречается проблема
\emph{мультиколлинеарности} --- когда матрица \(K\) имеет полный ранг,
но близка к некоторой матрице неполного ранга. Тогда говорят, что \(K\)
--- матрица неполного псевдоранга или что она плохо обусловлена.
Геометрически это означает, что объекты выборки сосредоточены вблизи
линейного подпространства меньшей размерности \(m < n\). Признаком
мультиколлинеарности является наличие у матрицы \(K\) собственных
значений, близких к нулю.

Итак, матрица \(K\) плохо обусловлена (матрица считается плохо обусловленной, если \(\mu(K) \gtrsim 10^2 \div 10^4\)).
Обращение такой матрицы численно неустойчиво.
При умножении обратной матрицы на вектор относительная
погрешность усиливается в \(\mu(K)\) раз.

Именно это и происходит с МНК-решением в случае плохой обусловленности.
В формуле~\eqref{eq:alpha-res-norm} близкие к нулю
собственные значения оказываются в знаменателе, в результате
увеличивается разброс коэффициентов \(\alpha^*\), появляются большие по
абсолютной величине положительные и отрицательные коэффициенты.
МНК-решение становится неустойчивым --- малые погрешности измерения
признаков или ответов у обучающих объектов могут существенно повлиять на
вектор решения \(\alpha^*\), а погрешности измерения признаков у
тестового объекта \(x\) --- на значения функции регрессии
\(g(x, \alpha^*)\). Мультиколлинеарность влечёт не только неустойчивость
и переобучение, но и неинтерпретируемость коэффициентов, так как по
абсолютной величине коэффициента \(\alpha_j\) становится невозможно
судить о степени важности признака \(f_j\).

    Отметим, что проблема мультиколленеарности никак не проявляется на
обучающих данных: вектор \(F\alpha^*\) не зависит от собственных
значений \(\sigma\) (см. формулу~\eqref{eq:F-alpha-res}).

Убедимся в этом для нашего примера.


    Мы рассмотрим три метода решения проблемы мультиколлинеарности:

\begin{enumerate}
\def\labelenumi{\arabic{enumi}.}
\tightlist
\item
  метод главных компонент,
\item
  гребневая регрессия,
\item
  метод LASSO.
\end{enumerate}

    \hypertarget{ux43cux435ux442ux43eux434-ux433ux43bux430ux432ux43dux44bux445-ux43aux43eux43cux43fux43eux43dux435ux43dux442}{%
\subsection{Метод главных
компонент}\label{ux43cux435ux442ux43eux434-ux433ux43bux430ux432ux43dux44bux445-ux43aux43eux43cux43fux43eux43dux435ux43dux442}}

Сделаем сингулярное разложение матрицы \(F\). Посмотрим на набор её
сингулярных чисел и число обусловленности.


    \begin{Verbatim}[commandchars=\\\{\}]
Сингулярные числа:  [3.35347844 1.83764121 0.84631111 0.34807406 0.13585743
0.02843847 0.00497056]
Число обусловленности mu =  674.667900393541
    \end{Verbatim}

    Из всего спектра оставим \(k\) главных компонент. Для этого в формуле
\eqref{eq:alpha-res} ограничим сумму \(k\) слагаемыми.

Получим новый вектор коэффициентов, его норма значительно уменьшилась.



    \begin{Verbatim}[commandchars=\\\{\}]
Вектор коэффициентов:  [1.66759296 2.06378776 0.61322275 0.80793846 0.26252613
0.3592163 0.12147734]
Норма вектора коэффициентов:  2.8777779581638403
    \end{Verbatim}


    \begin{center}
    \adjustimage{max size={0.75\linewidth}{0.9\paperheight}}{output_44_0.png}
    \end{center}
    { \hspace*{\fill} \\}
    
    Мы видим, что больших коэффициентов регрессии больше нет, норма вектора
коэффициентов уменьшилась почти в 100 раз.
При этом отметим, что порядок
регрессионного полинома не уменьшился, уменьшились коэффициенты при
старших членах.
Это означает, что применяя метод главных компонент, мы не избавляемся от менее значимых признаков, а~просто уменьшаем их коэффициент в уравнении регрессии.

    \begin{center}\rule{0.5\linewidth}{\linethickness}\end{center}

    \hypertarget{ux438ux441ux442ux43eux447ux43dux438ux43aux438}{%
\section{Источники}\label{ux438ux441ux442ux43eux447ux43dux438ux43aux438}}

\begin{enumerate}
\def\labelenumi{\arabic{enumi}.}
\tightlist
\item
  \emph{Беклемишев Д.В.} Дополнительные главы линейной алгебры. --- М.:
  Наука, 1983. --- 336~с.
\item
  \emph{Воронцов К.В.}
  \href{http://www.machinelearning.ru/wiki/images/6/6d/Voron-ML-1.pdf}{Математические
  методы обучения по прецедентам (теория обучения машин)}. --- 141~c.
\end{enumerate}




    % Add a bibliography block to the postdoc
    
    
    
\end{document}
