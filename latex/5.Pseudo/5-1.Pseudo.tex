\documentclass[11pt,a4paper]{article}

    \usepackage[breakable]{tcolorbox}
    \usepackage{parskip} % Stop auto-indenting (to mimic markdown behaviour)
    
    \usepackage{iftex}
    \ifPDFTeX
      \usepackage[T2A]{fontenc}
      \usepackage{mathpazo}
      \usepackage[russian,english]{babel}
    \else
      \usepackage{fontspec}
      \usepackage{polyglossia}
      \setmainlanguage[babelshorthands=true]{russian}    % Язык по-умолчанию русский с поддержкой приятных команд пакета babel
      \setotherlanguage{english}                         % Дополнительный язык = английский (в американской вариации по-умолчанию)
      \newfontfamily\cyrillicfonttt[Scale=0.87,BoldFont={Fira Mono Medium}] {Fira Mono}  % Моноширинный шрифт для кириллицы
      \defaultfontfeatures{Ligatures=TeX}
      \newfontfamily\cyrillicfont{STIX Two Text}         % Шрифт с засечками для кириллицы
    \fi
    \renewcommand{\linethickness}{0.1ex}

    % Basic figure setup, for now with no caption control since it's done
    % automatically by Pandoc (which extracts ![](path) syntax from Markdown).
    \usepackage{graphicx}
    % Maintain compatibility with old templates. Remove in nbconvert 6.0
    \let\Oldincludegraphics\includegraphics
    % Ensure that by default, figures have no caption (until we provide a
    % proper Figure object with a Caption API and a way to capture that
    % in the conversion process - todo).
    \usepackage{caption}
    \DeclareCaptionFormat{nocaption}{}
    \captionsetup{format=nocaption,aboveskip=0pt,belowskip=0pt}

    \usepackage[Export]{adjustbox} % Used to constrain images to a maximum size
    \adjustboxset{max size={0.9\linewidth}{0.9\paperheight}}
    \usepackage{float}
    \floatplacement{figure}{H} % forces figures to be placed at the correct location
    \usepackage{xcolor} % Allow colors to be defined
    \usepackage{enumerate} % Needed for markdown enumerations to work
    \usepackage{geometry} % Used to adjust the document margins
    \usepackage{amsmath} % Equations
    \usepackage{amssymb} % Equations
    \usepackage{textcomp} % defines textquotesingle
    % Hack from http://tex.stackexchange.com/a/47451/13684:
    \AtBeginDocument{%
        \def\PYZsq{\textquotesingle}% Upright quotes in Pygmentized code
    }
    \usepackage{upquote} % Upright quotes for verbatim code
    \usepackage{eurosym} % defines \euro
    \usepackage[mathletters]{ucs} % Extended unicode (utf-8) support
    \usepackage{fancyvrb} % verbatim replacement that allows latex
    \usepackage{grffile} % extends the file name processing of package graphics 
                         % to support a larger range
    \makeatletter % fix for grffile with XeLaTeX
    \def\Gread@@xetex#1{%
      \IfFileExists{"\Gin@base".bb}%
      {\Gread@eps{\Gin@base.bb}}%
      {\Gread@@xetex@aux#1}%
    }
    \makeatother

    % The hyperref package gives us a pdf with properly built
    % internal navigation ('pdf bookmarks' for the table of contents,
    % internal cross-reference links, web links for URLs, etc.)
    \usepackage{hyperref}
    % The default LaTeX title has an obnoxious amount of whitespace. By default,
    % titling removes some of it. It also provides customization options.
    \usepackage{titling}
    \usepackage{longtable} % longtable support required by pandoc >1.10
    \usepackage{booktabs}  % table support for pandoc > 1.12.2
    \usepackage[inline]{enumitem} % IRkernel/repr support (it uses the enumerate* environment)
    \usepackage[normalem]{ulem} % ulem is needed to support strikethroughs (\sout)
                                % normalem makes italics be italics, not underlines
    \usepackage{mathrsfs}
    

    
    % Colors for the hyperref package
    \definecolor{urlcolor}{rgb}{0,.145,.698}
    \definecolor{linkcolor}{rgb}{.71,0.21,0.01}
    \definecolor{citecolor}{rgb}{.12,.54,.11}

    % ANSI colors
    \definecolor{ansi-black}{HTML}{3E424D}
    \definecolor{ansi-black-intense}{HTML}{282C36}
    \definecolor{ansi-red}{HTML}{E75C58}
    \definecolor{ansi-red-intense}{HTML}{B22B31}
    \definecolor{ansi-green}{HTML}{00A250}
    \definecolor{ansi-green-intense}{HTML}{007427}
    \definecolor{ansi-yellow}{HTML}{DDB62B}
    \definecolor{ansi-yellow-intense}{HTML}{B27D12}
    \definecolor{ansi-blue}{HTML}{208FFB}
    \definecolor{ansi-blue-intense}{HTML}{0065CA}
    \definecolor{ansi-magenta}{HTML}{D160C4}
    \definecolor{ansi-magenta-intense}{HTML}{A03196}
    \definecolor{ansi-cyan}{HTML}{60C6C8}
    \definecolor{ansi-cyan-intense}{HTML}{258F8F}
    \definecolor{ansi-white}{HTML}{C5C1B4}
    \definecolor{ansi-white-intense}{HTML}{A1A6B2}
    \definecolor{ansi-default-inverse-fg}{HTML}{FFFFFF}
    \definecolor{ansi-default-inverse-bg}{HTML}{000000}

    % commands and environments needed by pandoc snippets
    % extracted from the output of `pandoc -s`
    \providecommand{\tightlist}{%
      \setlength{\itemsep}{0pt}\setlength{\parskip}{0pt}}
    \DefineVerbatimEnvironment{Highlighting}{Verbatim}{commandchars=\\\{\}}
    % Add ',fontsize=\small' for more characters per line
    \newenvironment{Shaded}{}{}
    \newcommand{\KeywordTok}[1]{\textcolor[rgb]{0.00,0.44,0.13}{\textbf{{#1}}}}
    \newcommand{\DataTypeTok}[1]{\textcolor[rgb]{0.56,0.13,0.00}{{#1}}}
    \newcommand{\DecValTok}[1]{\textcolor[rgb]{0.25,0.63,0.44}{{#1}}}
    \newcommand{\BaseNTok}[1]{\textcolor[rgb]{0.25,0.63,0.44}{{#1}}}
    \newcommand{\FloatTok}[1]{\textcolor[rgb]{0.25,0.63,0.44}{{#1}}}
    \newcommand{\CharTok}[1]{\textcolor[rgb]{0.25,0.44,0.63}{{#1}}}
    \newcommand{\StringTok}[1]{\textcolor[rgb]{0.25,0.44,0.63}{{#1}}}
    \newcommand{\CommentTok}[1]{\textcolor[rgb]{0.38,0.63,0.69}{\textit{{#1}}}}
    \newcommand{\OtherTok}[1]{\textcolor[rgb]{0.00,0.44,0.13}{{#1}}}
    \newcommand{\AlertTok}[1]{\textcolor[rgb]{1.00,0.00,0.00}{\textbf{{#1}}}}
    \newcommand{\FunctionTok}[1]{\textcolor[rgb]{0.02,0.16,0.49}{{#1}}}
    \newcommand{\RegionMarkerTok}[1]{{#1}}
    \newcommand{\ErrorTok}[1]{\textcolor[rgb]{1.00,0.00,0.00}{\textbf{{#1}}}}
    \newcommand{\NormalTok}[1]{{#1}}
    
    % Additional commands for more recent versions of Pandoc
    \newcommand{\ConstantTok}[1]{\textcolor[rgb]{0.53,0.00,0.00}{{#1}}}
    \newcommand{\SpecialCharTok}[1]{\textcolor[rgb]{0.25,0.44,0.63}{{#1}}}
    \newcommand{\VerbatimStringTok}[1]{\textcolor[rgb]{0.25,0.44,0.63}{{#1}}}
    \newcommand{\SpecialStringTok}[1]{\textcolor[rgb]{0.73,0.40,0.53}{{#1}}}
    \newcommand{\ImportTok}[1]{{#1}}
    \newcommand{\DocumentationTok}[1]{\textcolor[rgb]{0.73,0.13,0.13}{\textit{{#1}}}}
    \newcommand{\AnnotationTok}[1]{\textcolor[rgb]{0.38,0.63,0.69}{\textbf{\textit{{#1}}}}}
    \newcommand{\CommentVarTok}[1]{\textcolor[rgb]{0.38,0.63,0.69}{\textbf{\textit{{#1}}}}}
    \newcommand{\VariableTok}[1]{\textcolor[rgb]{0.10,0.09,0.49}{{#1}}}
    \newcommand{\ControlFlowTok}[1]{\textcolor[rgb]{0.00,0.44,0.13}{\textbf{{#1}}}}
    \newcommand{\OperatorTok}[1]{\textcolor[rgb]{0.40,0.40,0.40}{{#1}}}
    \newcommand{\BuiltInTok}[1]{{#1}}
    \newcommand{\ExtensionTok}[1]{{#1}}
    \newcommand{\PreprocessorTok}[1]{\textcolor[rgb]{0.74,0.48,0.00}{{#1}}}
    \newcommand{\AttributeTok}[1]{\textcolor[rgb]{0.49,0.56,0.16}{{#1}}}
    \newcommand{\InformationTok}[1]{\textcolor[rgb]{0.38,0.63,0.69}{\textbf{\textit{{#1}}}}}
    \newcommand{\WarningTok}[1]{\textcolor[rgb]{0.38,0.63,0.69}{\textbf{\textit{{#1}}}}}
    
    
    % Define a nice break command that doesn't care if a line doesn't already
    % exist.
    \def\br{\hspace*{\fill} \\* }
    % Math Jax compatibility definitions
    \def\gt{>}
    \def\lt{<}
    \let\Oldtex\TeX
    \let\Oldlatex\LaTeX
    \renewcommand{\TeX}{\textrm{\Oldtex}}
    \renewcommand{\LaTeX}{\textrm{\Oldlatex}}
    % Document parameters
    % Document title
    \title{Лекция 4 \\
    Псевдорешения и псевдообратные матрицы
    }
    
    
    
    
    
% Pygments definitions
\makeatletter
\def\PY@reset{\let\PY@it=\relax \let\PY@bf=\relax%
    \let\PY@ul=\relax \let\PY@tc=\relax%
    \let\PY@bc=\relax \let\PY@ff=\relax}
\def\PY@tok#1{\csname PY@tok@#1\endcsname}
\def\PY@toks#1+{\ifx\relax#1\empty\else%
    \PY@tok{#1}\expandafter\PY@toks\fi}
\def\PY@do#1{\PY@bc{\PY@tc{\PY@ul{%
    \PY@it{\PY@bf{\PY@ff{#1}}}}}}}
\def\PY#1#2{\PY@reset\PY@toks#1+\relax+\PY@do{#2}}

\expandafter\def\csname PY@tok@w\endcsname{\def\PY@tc##1{\textcolor[rgb]{0.73,0.73,0.73}{##1}}}
\expandafter\def\csname PY@tok@c\endcsname{\let\PY@it=\textit\def\PY@tc##1{\textcolor[rgb]{0.25,0.50,0.50}{##1}}}
\expandafter\def\csname PY@tok@cp\endcsname{\def\PY@tc##1{\textcolor[rgb]{0.74,0.48,0.00}{##1}}}
\expandafter\def\csname PY@tok@k\endcsname{\let\PY@bf=\textbf\def\PY@tc##1{\textcolor[rgb]{0.00,0.50,0.00}{##1}}}
\expandafter\def\csname PY@tok@kp\endcsname{\def\PY@tc##1{\textcolor[rgb]{0.00,0.50,0.00}{##1}}}
\expandafter\def\csname PY@tok@kt\endcsname{\def\PY@tc##1{\textcolor[rgb]{0.69,0.00,0.25}{##1}}}
\expandafter\def\csname PY@tok@o\endcsname{\def\PY@tc##1{\textcolor[rgb]{0.40,0.40,0.40}{##1}}}
\expandafter\def\csname PY@tok@ow\endcsname{\let\PY@bf=\textbf\def\PY@tc##1{\textcolor[rgb]{0.67,0.13,1.00}{##1}}}
\expandafter\def\csname PY@tok@nb\endcsname{\def\PY@tc##1{\textcolor[rgb]{0.00,0.50,0.00}{##1}}}
\expandafter\def\csname PY@tok@nf\endcsname{\def\PY@tc##1{\textcolor[rgb]{0.00,0.00,1.00}{##1}}}
\expandafter\def\csname PY@tok@nc\endcsname{\let\PY@bf=\textbf\def\PY@tc##1{\textcolor[rgb]{0.00,0.00,1.00}{##1}}}
\expandafter\def\csname PY@tok@nn\endcsname{\let\PY@bf=\textbf\def\PY@tc##1{\textcolor[rgb]{0.00,0.00,1.00}{##1}}}
\expandafter\def\csname PY@tok@ne\endcsname{\let\PY@bf=\textbf\def\PY@tc##1{\textcolor[rgb]{0.82,0.25,0.23}{##1}}}
\expandafter\def\csname PY@tok@nv\endcsname{\def\PY@tc##1{\textcolor[rgb]{0.10,0.09,0.49}{##1}}}
\expandafter\def\csname PY@tok@no\endcsname{\def\PY@tc##1{\textcolor[rgb]{0.53,0.00,0.00}{##1}}}
\expandafter\def\csname PY@tok@nl\endcsname{\def\PY@tc##1{\textcolor[rgb]{0.63,0.63,0.00}{##1}}}
\expandafter\def\csname PY@tok@ni\endcsname{\let\PY@bf=\textbf\def\PY@tc##1{\textcolor[rgb]{0.60,0.60,0.60}{##1}}}
\expandafter\def\csname PY@tok@na\endcsname{\def\PY@tc##1{\textcolor[rgb]{0.49,0.56,0.16}{##1}}}
\expandafter\def\csname PY@tok@nt\endcsname{\let\PY@bf=\textbf\def\PY@tc##1{\textcolor[rgb]{0.00,0.50,0.00}{##1}}}
\expandafter\def\csname PY@tok@nd\endcsname{\def\PY@tc##1{\textcolor[rgb]{0.67,0.13,1.00}{##1}}}
\expandafter\def\csname PY@tok@s\endcsname{\def\PY@tc##1{\textcolor[rgb]{0.73,0.13,0.13}{##1}}}
\expandafter\def\csname PY@tok@sd\endcsname{\let\PY@it=\textit\def\PY@tc##1{\textcolor[rgb]{0.73,0.13,0.13}{##1}}}
\expandafter\def\csname PY@tok@si\endcsname{\let\PY@bf=\textbf\def\PY@tc##1{\textcolor[rgb]{0.73,0.40,0.53}{##1}}}
\expandafter\def\csname PY@tok@se\endcsname{\let\PY@bf=\textbf\def\PY@tc##1{\textcolor[rgb]{0.73,0.40,0.13}{##1}}}
\expandafter\def\csname PY@tok@sr\endcsname{\def\PY@tc##1{\textcolor[rgb]{0.73,0.40,0.53}{##1}}}
\expandafter\def\csname PY@tok@ss\endcsname{\def\PY@tc##1{\textcolor[rgb]{0.10,0.09,0.49}{##1}}}
\expandafter\def\csname PY@tok@sx\endcsname{\def\PY@tc##1{\textcolor[rgb]{0.00,0.50,0.00}{##1}}}
\expandafter\def\csname PY@tok@m\endcsname{\def\PY@tc##1{\textcolor[rgb]{0.40,0.40,0.40}{##1}}}
\expandafter\def\csname PY@tok@gh\endcsname{\let\PY@bf=\textbf\def\PY@tc##1{\textcolor[rgb]{0.00,0.00,0.50}{##1}}}
\expandafter\def\csname PY@tok@gu\endcsname{\let\PY@bf=\textbf\def\PY@tc##1{\textcolor[rgb]{0.50,0.00,0.50}{##1}}}
\expandafter\def\csname PY@tok@gd\endcsname{\def\PY@tc##1{\textcolor[rgb]{0.63,0.00,0.00}{##1}}}
\expandafter\def\csname PY@tok@gi\endcsname{\def\PY@tc##1{\textcolor[rgb]{0.00,0.63,0.00}{##1}}}
\expandafter\def\csname PY@tok@gr\endcsname{\def\PY@tc##1{\textcolor[rgb]{1.00,0.00,0.00}{##1}}}
\expandafter\def\csname PY@tok@ge\endcsname{\let\PY@it=\textit}
\expandafter\def\csname PY@tok@gs\endcsname{\let\PY@bf=\textbf}
\expandafter\def\csname PY@tok@gp\endcsname{\let\PY@bf=\textbf\def\PY@tc##1{\textcolor[rgb]{0.00,0.00,0.50}{##1}}}
\expandafter\def\csname PY@tok@go\endcsname{\def\PY@tc##1{\textcolor[rgb]{0.53,0.53,0.53}{##1}}}
\expandafter\def\csname PY@tok@gt\endcsname{\def\PY@tc##1{\textcolor[rgb]{0.00,0.27,0.87}{##1}}}
\expandafter\def\csname PY@tok@err\endcsname{\def\PY@bc##1{\setlength{\fboxsep}{0pt}\fcolorbox[rgb]{1.00,0.00,0.00}{1,1,1}{\strut ##1}}}
\expandafter\def\csname PY@tok@kc\endcsname{\let\PY@bf=\textbf\def\PY@tc##1{\textcolor[rgb]{0.00,0.50,0.00}{##1}}}
\expandafter\def\csname PY@tok@kd\endcsname{\let\PY@bf=\textbf\def\PY@tc##1{\textcolor[rgb]{0.00,0.50,0.00}{##1}}}
\expandafter\def\csname PY@tok@kn\endcsname{\let\PY@bf=\textbf\def\PY@tc##1{\textcolor[rgb]{0.00,0.50,0.00}{##1}}}
\expandafter\def\csname PY@tok@kr\endcsname{\let\PY@bf=\textbf\def\PY@tc##1{\textcolor[rgb]{0.00,0.50,0.00}{##1}}}
\expandafter\def\csname PY@tok@bp\endcsname{\def\PY@tc##1{\textcolor[rgb]{0.00,0.50,0.00}{##1}}}
\expandafter\def\csname PY@tok@fm\endcsname{\def\PY@tc##1{\textcolor[rgb]{0.00,0.00,1.00}{##1}}}
\expandafter\def\csname PY@tok@vc\endcsname{\def\PY@tc##1{\textcolor[rgb]{0.10,0.09,0.49}{##1}}}
\expandafter\def\csname PY@tok@vg\endcsname{\def\PY@tc##1{\textcolor[rgb]{0.10,0.09,0.49}{##1}}}
\expandafter\def\csname PY@tok@vi\endcsname{\def\PY@tc##1{\textcolor[rgb]{0.10,0.09,0.49}{##1}}}
\expandafter\def\csname PY@tok@vm\endcsname{\def\PY@tc##1{\textcolor[rgb]{0.10,0.09,0.49}{##1}}}
\expandafter\def\csname PY@tok@sa\endcsname{\def\PY@tc##1{\textcolor[rgb]{0.73,0.13,0.13}{##1}}}
\expandafter\def\csname PY@tok@sb\endcsname{\def\PY@tc##1{\textcolor[rgb]{0.73,0.13,0.13}{##1}}}
\expandafter\def\csname PY@tok@sc\endcsname{\def\PY@tc##1{\textcolor[rgb]{0.73,0.13,0.13}{##1}}}
\expandafter\def\csname PY@tok@dl\endcsname{\def\PY@tc##1{\textcolor[rgb]{0.73,0.13,0.13}{##1}}}
\expandafter\def\csname PY@tok@s2\endcsname{\def\PY@tc##1{\textcolor[rgb]{0.73,0.13,0.13}{##1}}}
\expandafter\def\csname PY@tok@sh\endcsname{\def\PY@tc##1{\textcolor[rgb]{0.73,0.13,0.13}{##1}}}
\expandafter\def\csname PY@tok@s1\endcsname{\def\PY@tc##1{\textcolor[rgb]{0.73,0.13,0.13}{##1}}}
\expandafter\def\csname PY@tok@mb\endcsname{\def\PY@tc##1{\textcolor[rgb]{0.40,0.40,0.40}{##1}}}
\expandafter\def\csname PY@tok@mf\endcsname{\def\PY@tc##1{\textcolor[rgb]{0.40,0.40,0.40}{##1}}}
\expandafter\def\csname PY@tok@mh\endcsname{\def\PY@tc##1{\textcolor[rgb]{0.40,0.40,0.40}{##1}}}
\expandafter\def\csname PY@tok@mi\endcsname{\def\PY@tc##1{\textcolor[rgb]{0.40,0.40,0.40}{##1}}}
\expandafter\def\csname PY@tok@il\endcsname{\def\PY@tc##1{\textcolor[rgb]{0.40,0.40,0.40}{##1}}}
\expandafter\def\csname PY@tok@mo\endcsname{\def\PY@tc##1{\textcolor[rgb]{0.40,0.40,0.40}{##1}}}
\expandafter\def\csname PY@tok@ch\endcsname{\let\PY@it=\textit\def\PY@tc##1{\textcolor[rgb]{0.25,0.50,0.50}{##1}}}
\expandafter\def\csname PY@tok@cm\endcsname{\let\PY@it=\textit\def\PY@tc##1{\textcolor[rgb]{0.25,0.50,0.50}{##1}}}
\expandafter\def\csname PY@tok@cpf\endcsname{\let\PY@it=\textit\def\PY@tc##1{\textcolor[rgb]{0.25,0.50,0.50}{##1}}}
\expandafter\def\csname PY@tok@c1\endcsname{\let\PY@it=\textit\def\PY@tc##1{\textcolor[rgb]{0.25,0.50,0.50}{##1}}}
\expandafter\def\csname PY@tok@cs\endcsname{\let\PY@it=\textit\def\PY@tc##1{\textcolor[rgb]{0.25,0.50,0.50}{##1}}}

\def\PYZbs{\char`\\}
\def\PYZus{\char`\_}
\def\PYZob{\char`\{}
\def\PYZcb{\char`\}}
\def\PYZca{\char`\^}
\def\PYZam{\char`\&}
\def\PYZlt{\char`\<}
\def\PYZgt{\char`\>}
\def\PYZsh{\char`\#}
\def\PYZpc{\char`\%}
\def\PYZdl{\char`\$}
\def\PYZhy{\char`\-}
\def\PYZsq{\char`\'}
\def\PYZdq{\char`\"}
\def\PYZti{\char`\~}
% for compatibility with earlier versions
\def\PYZat{@}
\def\PYZlb{[}
\def\PYZrb{]}
\makeatother


    % For linebreaks inside Verbatim environment from package fancyvrb. 
    \makeatletter
        \newbox\Wrappedcontinuationbox 
        \newbox\Wrappedvisiblespacebox 
        \newcommand*\Wrappedvisiblespace {\textcolor{red}{\textvisiblespace}} 
        \newcommand*\Wrappedcontinuationsymbol {\textcolor{red}{\llap{\tiny$\m@th\hookrightarrow$}}} 
        \newcommand*\Wrappedcontinuationindent {3ex } 
        \newcommand*\Wrappedafterbreak {\kern\Wrappedcontinuationindent\copy\Wrappedcontinuationbox} 
        % Take advantage of the already applied Pygments mark-up to insert 
        % potential linebreaks for TeX processing. 
        %        {, <, #, %, $, ' and ": go to next line. 
        %        _, }, ^, &, >, - and ~: stay at end of broken line. 
        % Use of \textquotesingle for straight quote. 
        \newcommand*\Wrappedbreaksatspecials {% 
            \def\PYGZus{\discretionary{\char`\_}{\Wrappedafterbreak}{\char`\_}}% 
            \def\PYGZob{\discretionary{}{\Wrappedafterbreak\char`\{}{\char`\{}}% 
            \def\PYGZcb{\discretionary{\char`\}}{\Wrappedafterbreak}{\char`\}}}% 
            \def\PYGZca{\discretionary{\char`\^}{\Wrappedafterbreak}{\char`\^}}% 
            \def\PYGZam{\discretionary{\char`\&}{\Wrappedafterbreak}{\char`\&}}% 
            \def\PYGZlt{\discretionary{}{\Wrappedafterbreak\char`\<}{\char`\<}}% 
            \def\PYGZgt{\discretionary{\char`\>}{\Wrappedafterbreak}{\char`\>}}% 
            \def\PYGZsh{\discretionary{}{\Wrappedafterbreak\char`\#}{\char`\#}}% 
            \def\PYGZpc{\discretionary{}{\Wrappedafterbreak\char`\%}{\char`\%}}% 
            \def\PYGZdl{\discretionary{}{\Wrappedafterbreak\char`\$}{\char`\$}}% 
            \def\PYGZhy{\discretionary{\char`\-}{\Wrappedafterbreak}{\char`\-}}% 
            \def\PYGZsq{\discretionary{}{\Wrappedafterbreak\textquotesingle}{\textquotesingle}}% 
            \def\PYGZdq{\discretionary{}{\Wrappedafterbreak\char`\"}{\char`\"}}% 
            \def\PYGZti{\discretionary{\char`\~}{\Wrappedafterbreak}{\char`\~}}% 
        } 
        % Some characters . , ; ? ! / are not pygmentized. 
        % This macro makes them "active" and they will insert potential linebreaks 
        \newcommand*\Wrappedbreaksatpunct {% 
            \lccode`\~`\.\lowercase{\def~}{\discretionary{\hbox{\char`\.}}{\Wrappedafterbreak}{\hbox{\char`\.}}}% 
            \lccode`\~`\,\lowercase{\def~}{\discretionary{\hbox{\char`\,}}{\Wrappedafterbreak}{\hbox{\char`\,}}}% 
            \lccode`\~`\;\lowercase{\def~}{\discretionary{\hbox{\char`\;}}{\Wrappedafterbreak}{\hbox{\char`\;}}}% 
            \lccode`\~`\:\lowercase{\def~}{\discretionary{\hbox{\char`\:}}{\Wrappedafterbreak}{\hbox{\char`\:}}}% 
            \lccode`\~`\?\lowercase{\def~}{\discretionary{\hbox{\char`\?}}{\Wrappedafterbreak}{\hbox{\char`\?}}}% 
            \lccode`\~`\!\lowercase{\def~}{\discretionary{\hbox{\char`\!}}{\Wrappedafterbreak}{\hbox{\char`\!}}}% 
            \lccode`\~`\/\lowercase{\def~}{\discretionary{\hbox{\char`\/}}{\Wrappedafterbreak}{\hbox{\char`\/}}}% 
            \catcode`\.\active
            \catcode`\,\active 
            \catcode`\;\active
            \catcode`\:\active
            \catcode`\?\active
            \catcode`\!\active
            \catcode`\/\active 
            \lccode`\~`\~ 	
        }
    \makeatother

    \let\OriginalVerbatim=\Verbatim
    \makeatletter
    \renewcommand{\Verbatim}[1][1]{%
        %\parskip\z@skip
        \sbox\Wrappedcontinuationbox {\Wrappedcontinuationsymbol}%
        \sbox\Wrappedvisiblespacebox {\FV@SetupFont\Wrappedvisiblespace}%
        \def\FancyVerbFormatLine ##1{\hsize\linewidth
            \vtop{\raggedright\hyphenpenalty\z@\exhyphenpenalty\z@
                \doublehyphendemerits\z@\finalhyphendemerits\z@
                \strut ##1\strut}%
        }%
        % If the linebreak is at a space, the latter will be displayed as visible
        % space at end of first line, and a continuation symbol starts next line.
        % Stretch/shrink are however usually zero for typewriter font.
        \def\FV@Space {%
            \nobreak\hskip\z@ plus\fontdimen3\font minus\fontdimen4\font
            \discretionary{\copy\Wrappedvisiblespacebox}{\Wrappedafterbreak}
            {\kern\fontdimen2\font}%
        }%
        
        % Allow breaks at special characters using \PYG... macros.
        \Wrappedbreaksatspecials
        % Breaks at punctuation characters . , ; ? ! and / need catcode=\active 	
        \OriginalVerbatim[#1,codes*=\Wrappedbreaksatpunct]%
    }
    \makeatother

    % Exact colors from NB
    \definecolor{incolor}{HTML}{303F9F}
    \definecolor{outcolor}{HTML}{D84315}
    \definecolor{cellborder}{HTML}{CFCFCF}
    \definecolor{cellbackground}{HTML}{F7F7F7}
    
    % prompt
    \makeatletter
    \newcommand{\boxspacing}{\kern\kvtcb@left@rule\kern\kvtcb@boxsep}
    \makeatother
    \newcommand{\prompt}[4]{
        \ttfamily\llap{{\color{#2}[#3]:\hspace{3pt}#4}}\vspace{-\baselineskip}
    }
    

    
    % Prevent overflowing lines due to hard-to-break entities
    \sloppy 
    % Setup hyperref package
    \hypersetup{
      breaklinks=true,  % so long urls are correctly broken across lines
      colorlinks=true,
      urlcolor=urlcolor,
      linkcolor=linkcolor,
      citecolor=citecolor,
      }
    % Slightly bigger margins than the latex defaults
    
    \geometry{verbose,tmargin=1in,bmargin=1in,lmargin=1in,rmargin=1in}
    
    

\begin{document}
    
    \maketitle
    
    
    \hypertarget{ux432ux432ux435ux434ux435ux43dux438ux435}{%
\section{Введение}\label{ux432ux432ux435ux434ux435ux43dux438ux435}}

В практических задачах часто требуется найти решение, удовлетворяющее
большому числу возможно противоречивых требований. Если такая задача
сводится к системе линейных уравнений, то система оказывается, вообще
говоря, несовместной. В этом случае задача может быть решена только
путём выбора некоторого компромисса --- все требования могут быть
удовлетворены не полностью, а лишь до некоторой степени.

    \hypertarget{ux43fux43eux441ux442ux430ux43dux43eux432ux43aux430-ux437ux430ux434ux430ux447ux438}{%
\section{Постановка
задачи}\label{ux43fux43eux441ux442ux430ux43dux43eux432ux43aux430-ux437ux430ux434ux430ux447ux438}}

Рассмотрим систему линейных уравнений \[
  A\mathbf{x} = \mathbf{b} \tag{1}\label{eq:system}
\] с матрицей \(A\) размеров \(m \times n\) и ранга \(r\). Поскольку
\(\mathbf{x}\) --- столбец высоты \(n\), а \(\mathbf{b}\) --- столбец
высоты \(m\), для геометрической иллюстрации естественно будет
спользовать пространства \(\mathbb{R}_n\) и \(\mathbb{R}_m\).

Под нормой столбца \(\mathbf{x}\) мы будем понимать его евклидову норму,
т.е. число \[
  \|\mathbf{x}\| = \sqrt{\mathbf{x^\top x}} = \sqrt{x_1^2 + \ldots + x_n^2}.
\]

Невязкой, которую даёт столбец \(\mathbf{x}\) при подстановке в систему
\(\eqref{eq:system}\), называется столбец \[
  \mathbf{u} = \mathbf{b} - A\mathbf{x}.
\] Решение системы --- это столбец, дающий нулевую невзяку.

Если система \(\eqref{eq:system}\) несовместна, естественно постараться
найти стобец \(\mathbf{x}\), который даёт невязку с минимальной нормой,
и если такой столбец найдётся, считать его обобщённым решением.

    \hypertarget{ux43fux441ux435ux432ux434ux43eux440ux435ux448ux435ux43dux438ux435}{%
\section{Псевдорешение}\label{ux43fux441ux435ux432ux434ux43eux440ux435ux448ux435ux43dux438ux435}}

Для сравнения невязок воспользуемся евклидовой нормой и, следовательно,
будем искать столбец \(\mathbf{x}\), для которого минимальная величина
\[
  \|\mathbf{u}\|^2 = (\mathbf{b} - A\mathbf{x})^\top (\mathbf{b} - A \mathbf{x}).
\]

Найдём полный дифференциал \(\|\mathbf{u}\|^2\): \[
  d\|\mathbf{u}\|^2 = -d\mathbf{x}^\top A^\top (\mathbf{b}-A\mathbf{x}) - (\mathbf{b}-A\mathbf{x})^\top A d\mathbf{x} = \
  -2d\mathbf{x}^\top A^\top (\mathbf{b} - A\mathbf{x}).
\]

Дифференциал равен нулю тогда и только тогда, когда \[
  A^\top A \mathbf{x} = A^\top \mathbf{b}.
\] Эта система линейных уравнений по отношению к системе
(Section \ref{mjx-eqn-eqsystem}) называется \emph{нормальной системой}.
Независимо от совместности системы (Section \ref{mjx-eqn-eqsystem})
справедливо

\textbf{Предложение 1.} Нормальная система уравнений всегда совместна.\\
\emph{Докозательство.} Применим критерий Фредгольма: система
\(A\mathbf{x}=\mathbf{b}\) совместна тогда и только тогда, когда
\(\mathbf{b}\) ортогонален любому решению \(\mathbf{y}\) сопряжённой
однородной системы. Пусть \(\mathbf{y}\) --- решение сопряжённой
однородной системы \((A^\top A)^\top \mathbf{y} = 0\). Тогда \[
  \mathbf{y}^\top A^\top A \mathbf{y} = (A \mathbf{y})^\top (A \mathbf{y}) = 0 \quad \Rightarrow \quad
  A \mathbf{y} = 0 \quad \Rightarrow \quad
  \mathbf{y}^\top (A^\top \mathbf{b}) = (A\mathbf{y})^\top \mathbf{b} = 0.
\]

\textbf{Предложение 2.} Точная нижняя грань квадрата нормы невязки
достигается для всех решений нормальной системы и только для них.

\textbf{Предложение 3.} Нормальная система имеет единственное решение
тогда и только тогда, когда столбцы матрицы \(A\) линейно независимы.

\textbf{Определение.} \emph{Нормальным псевдорешением} системы линейных
уравнений называется столбец с минимальной нормой среди всех столбцов,
дающих минимальную по норме невязку при подстановке в эту систему.

\textbf{Теорема 1.} Каждая система линейных уравнений имеет одно и
только одно нормальное псевдорешение.

    \hypertarget{ux43fux440ux438ux43cux435ux440ux44b}{%
\subsection{Примеры}\label{ux43fux440ux438ux43cux435ux440ux44b}}

\begin{enumerate}
\def\labelenumi{\arabic{enumi}.}
\tightlist
\item
  Система из двух уравнений с одной неизвестной: \(x=1, \; x=2\);
\item
  Система из одного уравнений с двумя неизвестными: \(x + y = 2\);
\item
  Система из одного уравнения с одним неизвестным: \(ax=b\);
\item
  Система линейных уравнений с нулевой матрицей:
  \(O \mathbf{x} = \mathbf{b}\).
\end{enumerate}

    \hypertarget{ux43fux441ux435ux432ux434ux43eux43eux431ux440ux430ux442ux43dux430ux44f-ux43cux430ux442ux440ux438ux446ux430}{%
\section{Псевдообратная
матрица}\label{ux43fux441ux435ux432ux434ux43eux43eux431ux440ux430ux442ux43dux430ux44f-ux43cux430ux442ux440ux438ux446ux430}}

Для невырожденной квадратной матрицы \(A\) порядка \(n\) обратную
матрицу можно определить как такую, столбы которой являются решениями
систем линейных уравнений вида \[
  A\mathbf{x} = \mathbf{e}_i, \tag{2} \label{eq:inv_definition}
\] где \(\mathbf{e}_i\) --- \(i\)-й столбец единичной матрицы порядка
\(n\).

По аналогии можно дать следующее\\
\textbf{Определение.} \emph{Псевдообратной матрицей} для матрицы \(A\)
размеров \(m \times n\) называется матрица \(A^+\), столбцы которой ---
псевдорешения систем линейных уравнений вида
\(\eqref{eq:inv_definition}\), где \(\mathbf{e}_i\) --- столбцы
единичной матрицы порядка \(m\).

Из теоремы 1 следует, что каждая матрица имеет одну и только одну
псевдообратную. Для невырожденной квадратной матрицы псевдообратная
матрица совпадает с обратной.

\textbf{Предложение 4.} Если столбцы мтрицы \(A\) линейно независимы, то
\[
  A^+ = (A^\top A)^{-1} A^\top.
\] Если строки матрицы \(A\) линейно независимы, то \[
  A^+ = A^\top (A A^\top)^{-1}.
\] В первом случае \(A^+\) является левой обратной матрицей для \(A\)
(\(A^+A=I\)), во втором --- правой (\(A A^+ = I\)).

\textbf{Предложение 5.} Для любого стобца
\(\mathbf{y} \in \mathbb{R}_m\) столбец \(A A^+ \mathbf{y}\) есть
ортогональная проекция \(\mathbf{y}\) на линейную оболочку столбцов
матрицы \(A\).

\textbf{Предложение 6.} Если \(A = CR\) --- секлетное разложение матрицы
\(A\), то её псевдообратная равна \[
  A^+ = R^+ C^+ = R^\top (R R^\top)^{-1} (C^\top C)^{-1} C^\top.
\]

\textbf{Предложение 7.} Если \(A = U \Sigma V^\top\) --- сингулярное
разложение матрицы \(A\), то \(A^+ = V \Sigma^+ U^\top\).\\
\emph{Примечание.} Для диагональной матрицы псевдообратная получается
заменой каждого ненулевого элемента на диагонали на обратный к нему.

    \begin{center}\rule{0.5\linewidth}{\linethickness}\end{center}

    \hypertarget{ux43cux435ux442ux43eux434-ux43dux430ux438ux43cux435ux43dux44cux448ux438ux445-ux43aux432ux430ux434ux440ux430ux442ux43eux432}{%
\section{Метод наименьших
квадратов}\label{ux43cux435ux442ux43eux434-ux43dux430ux438ux43cux435ux43dux44cux448ux438ux445-ux43aux432ux430ux434ux440ux430ux442ux43eux432}}

\textbf{Определение.} \emph{Функция потерь} (loss function) --- это
неотрицательная функция \(\mathcal{L}(a, x)\), характеризующая величину
ошибки алгоритма \(a\) на объекте \(x\). Если \(\mathcal{L}(a, x) = 0\),
то ответ \(a(x)\) называется \emph{корректным}.

\textbf{Определение.} \emph{Функционал качества} алгоритма \(a\) на
выборке \(X^l\):

\[ Q(a,X^l) = \frac{1}{l} \sum_{i=1}^{l} \mathcal{L}(a, x_i). \]

Функционал Q называют также функционалом \emph{средних потерь} или
\emph{эмпирическим риском}, так как он вычисляется по эмпирическим
данным \((x_i, y_i)_{i=1}^l\).

В задачах регрессии люычно применяется квадратичная функция потерь
\(\mathcal{L}(a, x) = (a(x) − y^*(x))^2\) \&mdash, функционал \(Q\) в
этом случае называется средней квадратичной ошибкой алгоритма \(a\) на
выборке \(X^l\); .

Классический метод обучения, называемый \emph{минимизацией эмпирического
риска} (empirical risk minimization, ERM), заключается в том, чтобы
найти в заданной модели \(A\) алгоритм \(a\), доставляющий минимальное
значение функционалу качества \(Q\) на заданной обучающей выборке
\(X^l\):

\[ \mu(X^l) = \underset{a \in A}{\mathrm{argmin}} Q(a, X^l). \]

В задаче восстановления регрессии с числовыми признаками и квадратичной
функцией потерь метод минимизации эмпирического риска есть ничто иное,
как метод наименьших квадратов.

Пусть задана \emph{модель регрессии} --- параметрическое семейство
функций \(g(x,\alpha)\), где \(\alpha \in \mathbb{R}^p\) --- вектор
параметров модели. Определим функционал качества аппроксимации целевой
зависимости на выборке \(X^l\) как сумму квадратов ошибок:

\[ Q(\alpha, X^l) = \sum_{i=1}^l \left( g(x_i, \alpha) - y_i \right)^2. \]

Обучение по \emph{методу наименьших квадратов} (МНК) состоит в том,
чтобы найти вектор параметров \(\alpha^*\), при котором достигается
минимум среднего квадрата ошибки на заданной обучающей выборке \(X^l\):

\[ \alpha^* = \underset{\alpha \in \mathbb{R}^n}{\mathrm{argmin}} \, {Q(\alpha, X^l)}. \]

    \begin{center}\rule{0.5\linewidth}{\linethickness}\end{center}

    \hypertarget{ux43bux438ux43dux435ux439ux43dux430ux44f-ux440ux435ux433ux440ux435ux441ux441ux438ux44f}{%
\section{Линейная
регрессия}\label{ux43bux438ux43dux435ux439ux43dux430ux44f-ux440ux435ux433ux440ux435ux441ux441ux438ux44f}}

Линейная регрессия является одной из самых простых моделей машинного
обучения. Есть мнение, что её даже не следует классифицировать как
«машинное обучение», потому что она слишком простая. Тем не менее,
простота делает её прекрасной отправной точкой для понимания более
сложных методов. На этом занятии мы исследуем основы линейной регрессии
и рассмотрим её вероятностную интерпретацию.

Пусть каждому объекту соответствует его признаковое описание
\(\left( f_1(x), \ldots, f_n(x)\right)\), где
\(f_j: X \rightarrow \mathbb{R}\) --- числовые признаки,
\(j = 1, \ldots , n\). Линейной моделью регрессии называется линейная
комбинация признаков с коэффициентами \(\alpha \in \mathbb{R}^n\):

\[ g(x, \alpha) = \sum_{j=1}^n \alpha_j f_j(x). \]

Введём матричные обозначения:
\(F = \left( f_j(x_i) \right)_{l \times n}\) --- матрица
объекты--признаки; \(y = \left( y_i \right)_{l \times 1}\) --- целевой
вектор; \(\alpha = \left( \alpha_i \right)_{l \times 1}\) --- вектор
параметров.

В матричных обозначениях функционал \(Q\) принимает вид

\[ Q(\alpha) = \left\Vert F\alpha - y \right\Vert^2. \]

Тогда задача поиска параметров регрессии может быть сформулирована так:

\[ \alpha^* = \underset{\alpha \in \mathbb{R}^n}{\text{argmin}} \, {\left\Vert F\alpha - y \right\Vert^2}. \]

Запишем необходимое условие минимума в матричном виде:

\[ \frac{\delta Q(\alpha)}{\delta \alpha} = 2F^\top (F\alpha - y) = 0. \]

Отсюда следует \(F^{\top} F \alpha = F^{\top}y\). Эта система линейных
уравнений относительно \(\alpha\) называется нормальной системой для
задачи наименьших квадратов. Если матрица \(F^{\top} F\) размера
\(n \times n\) невырождена, то решением нормальной системы является
вектор

\[ \alpha^* = (F^{\top} F)^{-1} F^{\top} y = F^{+} y. \]

Матрица \(F^{+} = (F^{\top} F)^{-1} F^{\top}\) называется
\emph{псевдообратной} для прямоугольной матрицы \(F\).

%    \begin{center}\rule{0.5\linewidth}{\linethickness}\end{center}
%
%    \hypertarget{ux43bux438ux442ux435ux440ux430ux442ux443ux440ux430}{%
%\section{Литература}\label{ux43bux438ux442ux435ux440ux430ux442ux443ux440ux430}}
%
%\begin{enumerate}
%\def\labelenumi{\arabic{enumi}.}
%\tightlist
%\item
%  \emph{Беклемишев Д.В.} Дополнительные главы линейной алгебры. --- М.:
%  Наука, 1983
%\item
%  \emph{Воронцов К.В.}
%  \href{http://www.machinelearning.ru/wiki/images/6/6d/Voron-ML-1.pdf}{Математические
%  методы обучения по прецендентам (теория обучения машин)}
%\end{enumerate}


    % Add a bibliography block to the postdoc
    
    
    
\end{document}
