\documentclass[11pt]{article}

    \usepackage[breakable]{tcolorbox}
    \usepackage{parskip} % Stop auto-indenting (to mimic markdown behaviour)
    
    \usepackage{iftex}
    \ifPDFTeX
      \usepackage[T2A]{fontenc}
      \usepackage{mathpazo}
      \usepackage[russian,english]{babel}
    \else
      \usepackage{fontspec}
      \usepackage{polyglossia}
      \setmainlanguage[babelshorthands=true]{russian}    % Язык по-умолчанию русский с поддержкой приятных команд пакета babel
      \setotherlanguage{english}                         % Дополнительный язык = английский (в американской вариации по-умолчанию)
      \newfontfamily\cyrillicfonttt[Scale=0.87,BoldFont={Fira Mono Medium}] {Fira Mono}  % Моноширинный шрифт для кириллицы
      \defaultfontfeatures{Ligatures=TeX}
      \newfontfamily\cyrillicfont{STIX Two Text}         % Шрифт с засечками для кириллицы
    \fi

    % Basic figure setup, for now with no caption control since it's done
    % automatically by Pandoc (which extracts ![](path) syntax from Markdown).
    \usepackage{graphicx}
    % Maintain compatibility with old templates. Remove in nbconvert 6.0
    \let\Oldincludegraphics\includegraphics
    % Ensure that by default, figures have no caption (until we provide a
    % proper Figure object with a Caption API and a way to capture that
    % in the conversion process - todo).
    \usepackage{caption}
    \DeclareCaptionFormat{nocaption}{}
    \captionsetup{format=nocaption,aboveskip=0pt,belowskip=0pt}

    \usepackage[Export]{adjustbox} % Used to constrain images to a maximum size
    \adjustboxset{max size={0.9\linewidth}{0.9\paperheight}}
    \usepackage{float}
    \floatplacement{figure}{H} % forces figures to be placed at the correct location
    \usepackage{xcolor} % Allow colors to be defined
    \usepackage{enumerate} % Needed for markdown enumerations to work
    \usepackage{geometry} % Used to adjust the document margins
    \usepackage{amsmath} % Equations
    \usepackage{amssymb} % Equations
    \usepackage{textcomp} % defines textquotesingle
    % Hack from http://tex.stackexchange.com/a/47451/13684:
    \AtBeginDocument{%
        \def\PYZsq{\textquotesingle}% Upright quotes in Pygmentized code
    }
    \usepackage{upquote} % Upright quotes for verbatim code
    \usepackage{eurosym} % defines \euro
    \usepackage[mathletters]{ucs} % Extended unicode (utf-8) support
    \usepackage{fancyvrb} % verbatim replacement that allows latex
    \usepackage{grffile} % extends the file name processing of package graphics 
                         % to support a larger range
    \makeatletter % fix for grffile with XeLaTeX
    \def\Gread@@xetex#1{%
      \IfFileExists{"\Gin@base".bb}%
      {\Gread@eps{\Gin@base.bb}}%
      {\Gread@@xetex@aux#1}%
    }
    \makeatother

    % The hyperref package gives us a pdf with properly built
    % internal navigation ('pdf bookmarks' for the table of contents,
    % internal cross-reference links, web links for URLs, etc.)
    \usepackage{hyperref}
    % The default LaTeX title has an obnoxious amount of whitespace. By default,
    % titling removes some of it. It also provides customization options.
    \usepackage{titling}
    \usepackage{longtable} % longtable support required by pandoc >1.10
    \usepackage{booktabs}  % table support for pandoc > 1.12.2
    \usepackage[inline]{enumitem} % IRkernel/repr support (it uses the enumerate* environment)
    \usepackage[normalem]{ulem} % ulem is needed to support strikethroughs (\sout)
                                % normalem makes italics be italics, not underlines
    \usepackage{mathrsfs}
    

    
    % Colors for the hyperref package
    \definecolor{urlcolor}{rgb}{0,.145,.698}
    \definecolor{linkcolor}{rgb}{.71,0.21,0.01}
    \definecolor{citecolor}{rgb}{.12,.54,.11}

    % ANSI colors
    \definecolor{ansi-black}{HTML}{3E424D}
    \definecolor{ansi-black-intense}{HTML}{282C36}
    \definecolor{ansi-red}{HTML}{E75C58}
    \definecolor{ansi-red-intense}{HTML}{B22B31}
    \definecolor{ansi-green}{HTML}{00A250}
    \definecolor{ansi-green-intense}{HTML}{007427}
    \definecolor{ansi-yellow}{HTML}{DDB62B}
    \definecolor{ansi-yellow-intense}{HTML}{B27D12}
    \definecolor{ansi-blue}{HTML}{208FFB}
    \definecolor{ansi-blue-intense}{HTML}{0065CA}
    \definecolor{ansi-magenta}{HTML}{D160C4}
    \definecolor{ansi-magenta-intense}{HTML}{A03196}
    \definecolor{ansi-cyan}{HTML}{60C6C8}
    \definecolor{ansi-cyan-intense}{HTML}{258F8F}
    \definecolor{ansi-white}{HTML}{C5C1B4}
    \definecolor{ansi-white-intense}{HTML}{A1A6B2}
    \definecolor{ansi-default-inverse-fg}{HTML}{FFFFFF}
    \definecolor{ansi-default-inverse-bg}{HTML}{000000}

    % commands and environments needed by pandoc snippets
    % extracted from the output of `pandoc -s`
    \providecommand{\tightlist}{%
      \setlength{\itemsep}{0pt}\setlength{\parskip}{0pt}}
    \DefineVerbatimEnvironment{Highlighting}{Verbatim}{commandchars=\\\{\}}
    % Add ',fontsize=\small' for more characters per line
    \newenvironment{Shaded}{}{}
    \newcommand{\KeywordTok}[1]{\textcolor[rgb]{0.00,0.44,0.13}{\textbf{{#1}}}}
    \newcommand{\DataTypeTok}[1]{\textcolor[rgb]{0.56,0.13,0.00}{{#1}}}
    \newcommand{\DecValTok}[1]{\textcolor[rgb]{0.25,0.63,0.44}{{#1}}}
    \newcommand{\BaseNTok}[1]{\textcolor[rgb]{0.25,0.63,0.44}{{#1}}}
    \newcommand{\FloatTok}[1]{\textcolor[rgb]{0.25,0.63,0.44}{{#1}}}
    \newcommand{\CharTok}[1]{\textcolor[rgb]{0.25,0.44,0.63}{{#1}}}
    \newcommand{\StringTok}[1]{\textcolor[rgb]{0.25,0.44,0.63}{{#1}}}
    \newcommand{\CommentTok}[1]{\textcolor[rgb]{0.38,0.63,0.69}{\textit{{#1}}}}
    \newcommand{\OtherTok}[1]{\textcolor[rgb]{0.00,0.44,0.13}{{#1}}}
    \newcommand{\AlertTok}[1]{\textcolor[rgb]{1.00,0.00,0.00}{\textbf{{#1}}}}
    \newcommand{\FunctionTok}[1]{\textcolor[rgb]{0.02,0.16,0.49}{{#1}}}
    \newcommand{\RegionMarkerTok}[1]{{#1}}
    \newcommand{\ErrorTok}[1]{\textcolor[rgb]{1.00,0.00,0.00}{\textbf{{#1}}}}
    \newcommand{\NormalTok}[1]{{#1}}
    
    % Additional commands for more recent versions of Pandoc
    \newcommand{\ConstantTok}[1]{\textcolor[rgb]{0.53,0.00,0.00}{{#1}}}
    \newcommand{\SpecialCharTok}[1]{\textcolor[rgb]{0.25,0.44,0.63}{{#1}}}
    \newcommand{\VerbatimStringTok}[1]{\textcolor[rgb]{0.25,0.44,0.63}{{#1}}}
    \newcommand{\SpecialStringTok}[1]{\textcolor[rgb]{0.73,0.40,0.53}{{#1}}}
    \newcommand{\ImportTok}[1]{{#1}}
    \newcommand{\DocumentationTok}[1]{\textcolor[rgb]{0.73,0.13,0.13}{\textit{{#1}}}}
    \newcommand{\AnnotationTok}[1]{\textcolor[rgb]{0.38,0.63,0.69}{\textbf{\textit{{#1}}}}}
    \newcommand{\CommentVarTok}[1]{\textcolor[rgb]{0.38,0.63,0.69}{\textbf{\textit{{#1}}}}}
    \newcommand{\VariableTok}[1]{\textcolor[rgb]{0.10,0.09,0.49}{{#1}}}
    \newcommand{\ControlFlowTok}[1]{\textcolor[rgb]{0.00,0.44,0.13}{\textbf{{#1}}}}
    \newcommand{\OperatorTok}[1]{\textcolor[rgb]{0.40,0.40,0.40}{{#1}}}
    \newcommand{\BuiltInTok}[1]{{#1}}
    \newcommand{\ExtensionTok}[1]{{#1}}
    \newcommand{\PreprocessorTok}[1]{\textcolor[rgb]{0.74,0.48,0.00}{{#1}}}
    \newcommand{\AttributeTok}[1]{\textcolor[rgb]{0.49,0.56,0.16}{{#1}}}
    \newcommand{\InformationTok}[1]{\textcolor[rgb]{0.38,0.63,0.69}{\textbf{\textit{{#1}}}}}
    \newcommand{\WarningTok}[1]{\textcolor[rgb]{0.38,0.63,0.69}{\textbf{\textit{{#1}}}}}
    
    
    % Define a nice break command that doesn't care if a line doesn't already
    % exist.
    \def\br{\hspace*{\fill} \\* }
    % Math Jax compatibility definitions
    \def\gt{>}
    \def\lt{<}
    \let\Oldtex\TeX
    \let\Oldlatex\LaTeX
    \renewcommand{\TeX}{\textrm{\Oldtex}}
    \renewcommand{\LaTeX}{\textrm{\Oldlatex}}
    % Document parameters
    % Document title
    \title{1.Probability\_elements}
    
    
    
    
    
% Pygments definitions
\makeatletter
\def\PY@reset{\let\PY@it=\relax \let\PY@bf=\relax%
    \let\PY@ul=\relax \let\PY@tc=\relax%
    \let\PY@bc=\relax \let\PY@ff=\relax}
\def\PY@tok#1{\csname PY@tok@#1\endcsname}
\def\PY@toks#1+{\ifx\relax#1\empty\else%
    \PY@tok{#1}\expandafter\PY@toks\fi}
\def\PY@do#1{\PY@bc{\PY@tc{\PY@ul{%
    \PY@it{\PY@bf{\PY@ff{#1}}}}}}}
\def\PY#1#2{\PY@reset\PY@toks#1+\relax+\PY@do{#2}}

\expandafter\def\csname PY@tok@w\endcsname{\def\PY@tc##1{\textcolor[rgb]{0.73,0.73,0.73}{##1}}}
\expandafter\def\csname PY@tok@c\endcsname{\let\PY@it=\textit\def\PY@tc##1{\textcolor[rgb]{0.25,0.50,0.50}{##1}}}
\expandafter\def\csname PY@tok@cp\endcsname{\def\PY@tc##1{\textcolor[rgb]{0.74,0.48,0.00}{##1}}}
\expandafter\def\csname PY@tok@k\endcsname{\let\PY@bf=\textbf\def\PY@tc##1{\textcolor[rgb]{0.00,0.50,0.00}{##1}}}
\expandafter\def\csname PY@tok@kp\endcsname{\def\PY@tc##1{\textcolor[rgb]{0.00,0.50,0.00}{##1}}}
\expandafter\def\csname PY@tok@kt\endcsname{\def\PY@tc##1{\textcolor[rgb]{0.69,0.00,0.25}{##1}}}
\expandafter\def\csname PY@tok@o\endcsname{\def\PY@tc##1{\textcolor[rgb]{0.40,0.40,0.40}{##1}}}
\expandafter\def\csname PY@tok@ow\endcsname{\let\PY@bf=\textbf\def\PY@tc##1{\textcolor[rgb]{0.67,0.13,1.00}{##1}}}
\expandafter\def\csname PY@tok@nb\endcsname{\def\PY@tc##1{\textcolor[rgb]{0.00,0.50,0.00}{##1}}}
\expandafter\def\csname PY@tok@nf\endcsname{\def\PY@tc##1{\textcolor[rgb]{0.00,0.00,1.00}{##1}}}
\expandafter\def\csname PY@tok@nc\endcsname{\let\PY@bf=\textbf\def\PY@tc##1{\textcolor[rgb]{0.00,0.00,1.00}{##1}}}
\expandafter\def\csname PY@tok@nn\endcsname{\let\PY@bf=\textbf\def\PY@tc##1{\textcolor[rgb]{0.00,0.00,1.00}{##1}}}
\expandafter\def\csname PY@tok@ne\endcsname{\let\PY@bf=\textbf\def\PY@tc##1{\textcolor[rgb]{0.82,0.25,0.23}{##1}}}
\expandafter\def\csname PY@tok@nv\endcsname{\def\PY@tc##1{\textcolor[rgb]{0.10,0.09,0.49}{##1}}}
\expandafter\def\csname PY@tok@no\endcsname{\def\PY@tc##1{\textcolor[rgb]{0.53,0.00,0.00}{##1}}}
\expandafter\def\csname PY@tok@nl\endcsname{\def\PY@tc##1{\textcolor[rgb]{0.63,0.63,0.00}{##1}}}
\expandafter\def\csname PY@tok@ni\endcsname{\let\PY@bf=\textbf\def\PY@tc##1{\textcolor[rgb]{0.60,0.60,0.60}{##1}}}
\expandafter\def\csname PY@tok@na\endcsname{\def\PY@tc##1{\textcolor[rgb]{0.49,0.56,0.16}{##1}}}
\expandafter\def\csname PY@tok@nt\endcsname{\let\PY@bf=\textbf\def\PY@tc##1{\textcolor[rgb]{0.00,0.50,0.00}{##1}}}
\expandafter\def\csname PY@tok@nd\endcsname{\def\PY@tc##1{\textcolor[rgb]{0.67,0.13,1.00}{##1}}}
\expandafter\def\csname PY@tok@s\endcsname{\def\PY@tc##1{\textcolor[rgb]{0.73,0.13,0.13}{##1}}}
\expandafter\def\csname PY@tok@sd\endcsname{\let\PY@it=\textit\def\PY@tc##1{\textcolor[rgb]{0.73,0.13,0.13}{##1}}}
\expandafter\def\csname PY@tok@si\endcsname{\let\PY@bf=\textbf\def\PY@tc##1{\textcolor[rgb]{0.73,0.40,0.53}{##1}}}
\expandafter\def\csname PY@tok@se\endcsname{\let\PY@bf=\textbf\def\PY@tc##1{\textcolor[rgb]{0.73,0.40,0.13}{##1}}}
\expandafter\def\csname PY@tok@sr\endcsname{\def\PY@tc##1{\textcolor[rgb]{0.73,0.40,0.53}{##1}}}
\expandafter\def\csname PY@tok@ss\endcsname{\def\PY@tc##1{\textcolor[rgb]{0.10,0.09,0.49}{##1}}}
\expandafter\def\csname PY@tok@sx\endcsname{\def\PY@tc##1{\textcolor[rgb]{0.00,0.50,0.00}{##1}}}
\expandafter\def\csname PY@tok@m\endcsname{\def\PY@tc##1{\textcolor[rgb]{0.40,0.40,0.40}{##1}}}
\expandafter\def\csname PY@tok@gh\endcsname{\let\PY@bf=\textbf\def\PY@tc##1{\textcolor[rgb]{0.00,0.00,0.50}{##1}}}
\expandafter\def\csname PY@tok@gu\endcsname{\let\PY@bf=\textbf\def\PY@tc##1{\textcolor[rgb]{0.50,0.00,0.50}{##1}}}
\expandafter\def\csname PY@tok@gd\endcsname{\def\PY@tc##1{\textcolor[rgb]{0.63,0.00,0.00}{##1}}}
\expandafter\def\csname PY@tok@gi\endcsname{\def\PY@tc##1{\textcolor[rgb]{0.00,0.63,0.00}{##1}}}
\expandafter\def\csname PY@tok@gr\endcsname{\def\PY@tc##1{\textcolor[rgb]{1.00,0.00,0.00}{##1}}}
\expandafter\def\csname PY@tok@ge\endcsname{\let\PY@it=\textit}
\expandafter\def\csname PY@tok@gs\endcsname{\let\PY@bf=\textbf}
\expandafter\def\csname PY@tok@gp\endcsname{\let\PY@bf=\textbf\def\PY@tc##1{\textcolor[rgb]{0.00,0.00,0.50}{##1}}}
\expandafter\def\csname PY@tok@go\endcsname{\def\PY@tc##1{\textcolor[rgb]{0.53,0.53,0.53}{##1}}}
\expandafter\def\csname PY@tok@gt\endcsname{\def\PY@tc##1{\textcolor[rgb]{0.00,0.27,0.87}{##1}}}
\expandafter\def\csname PY@tok@err\endcsname{\def\PY@bc##1{\setlength{\fboxsep}{0pt}\fcolorbox[rgb]{1.00,0.00,0.00}{1,1,1}{\strut ##1}}}
\expandafter\def\csname PY@tok@kc\endcsname{\let\PY@bf=\textbf\def\PY@tc##1{\textcolor[rgb]{0.00,0.50,0.00}{##1}}}
\expandafter\def\csname PY@tok@kd\endcsname{\let\PY@bf=\textbf\def\PY@tc##1{\textcolor[rgb]{0.00,0.50,0.00}{##1}}}
\expandafter\def\csname PY@tok@kn\endcsname{\let\PY@bf=\textbf\def\PY@tc##1{\textcolor[rgb]{0.00,0.50,0.00}{##1}}}
\expandafter\def\csname PY@tok@kr\endcsname{\let\PY@bf=\textbf\def\PY@tc##1{\textcolor[rgb]{0.00,0.50,0.00}{##1}}}
\expandafter\def\csname PY@tok@bp\endcsname{\def\PY@tc##1{\textcolor[rgb]{0.00,0.50,0.00}{##1}}}
\expandafter\def\csname PY@tok@fm\endcsname{\def\PY@tc##1{\textcolor[rgb]{0.00,0.00,1.00}{##1}}}
\expandafter\def\csname PY@tok@vc\endcsname{\def\PY@tc##1{\textcolor[rgb]{0.10,0.09,0.49}{##1}}}
\expandafter\def\csname PY@tok@vg\endcsname{\def\PY@tc##1{\textcolor[rgb]{0.10,0.09,0.49}{##1}}}
\expandafter\def\csname PY@tok@vi\endcsname{\def\PY@tc##1{\textcolor[rgb]{0.10,0.09,0.49}{##1}}}
\expandafter\def\csname PY@tok@vm\endcsname{\def\PY@tc##1{\textcolor[rgb]{0.10,0.09,0.49}{##1}}}
\expandafter\def\csname PY@tok@sa\endcsname{\def\PY@tc##1{\textcolor[rgb]{0.73,0.13,0.13}{##1}}}
\expandafter\def\csname PY@tok@sb\endcsname{\def\PY@tc##1{\textcolor[rgb]{0.73,0.13,0.13}{##1}}}
\expandafter\def\csname PY@tok@sc\endcsname{\def\PY@tc##1{\textcolor[rgb]{0.73,0.13,0.13}{##1}}}
\expandafter\def\csname PY@tok@dl\endcsname{\def\PY@tc##1{\textcolor[rgb]{0.73,0.13,0.13}{##1}}}
\expandafter\def\csname PY@tok@s2\endcsname{\def\PY@tc##1{\textcolor[rgb]{0.73,0.13,0.13}{##1}}}
\expandafter\def\csname PY@tok@sh\endcsname{\def\PY@tc##1{\textcolor[rgb]{0.73,0.13,0.13}{##1}}}
\expandafter\def\csname PY@tok@s1\endcsname{\def\PY@tc##1{\textcolor[rgb]{0.73,0.13,0.13}{##1}}}
\expandafter\def\csname PY@tok@mb\endcsname{\def\PY@tc##1{\textcolor[rgb]{0.40,0.40,0.40}{##1}}}
\expandafter\def\csname PY@tok@mf\endcsname{\def\PY@tc##1{\textcolor[rgb]{0.40,0.40,0.40}{##1}}}
\expandafter\def\csname PY@tok@mh\endcsname{\def\PY@tc##1{\textcolor[rgb]{0.40,0.40,0.40}{##1}}}
\expandafter\def\csname PY@tok@mi\endcsname{\def\PY@tc##1{\textcolor[rgb]{0.40,0.40,0.40}{##1}}}
\expandafter\def\csname PY@tok@il\endcsname{\def\PY@tc##1{\textcolor[rgb]{0.40,0.40,0.40}{##1}}}
\expandafter\def\csname PY@tok@mo\endcsname{\def\PY@tc##1{\textcolor[rgb]{0.40,0.40,0.40}{##1}}}
\expandafter\def\csname PY@tok@ch\endcsname{\let\PY@it=\textit\def\PY@tc##1{\textcolor[rgb]{0.25,0.50,0.50}{##1}}}
\expandafter\def\csname PY@tok@cm\endcsname{\let\PY@it=\textit\def\PY@tc##1{\textcolor[rgb]{0.25,0.50,0.50}{##1}}}
\expandafter\def\csname PY@tok@cpf\endcsname{\let\PY@it=\textit\def\PY@tc##1{\textcolor[rgb]{0.25,0.50,0.50}{##1}}}
\expandafter\def\csname PY@tok@c1\endcsname{\let\PY@it=\textit\def\PY@tc##1{\textcolor[rgb]{0.25,0.50,0.50}{##1}}}
\expandafter\def\csname PY@tok@cs\endcsname{\let\PY@it=\textit\def\PY@tc##1{\textcolor[rgb]{0.25,0.50,0.50}{##1}}}

\def\PYZbs{\char`\\}
\def\PYZus{\char`\_}
\def\PYZob{\char`\{}
\def\PYZcb{\char`\}}
\def\PYZca{\char`\^}
\def\PYZam{\char`\&}
\def\PYZlt{\char`\<}
\def\PYZgt{\char`\>}
\def\PYZsh{\char`\#}
\def\PYZpc{\char`\%}
\def\PYZdl{\char`\$}
\def\PYZhy{\char`\-}
\def\PYZsq{\char`\'}
\def\PYZdq{\char`\"}
\def\PYZti{\char`\~}
% for compatibility with earlier versions
\def\PYZat{@}
\def\PYZlb{[}
\def\PYZrb{]}
\makeatother


    % For linebreaks inside Verbatim environment from package fancyvrb. 
    \makeatletter
        \newbox\Wrappedcontinuationbox 
        \newbox\Wrappedvisiblespacebox 
        \newcommand*\Wrappedvisiblespace {\textcolor{red}{\textvisiblespace}} 
        \newcommand*\Wrappedcontinuationsymbol {\textcolor{red}{\llap{\tiny$\m@th\hookrightarrow$}}} 
        \newcommand*\Wrappedcontinuationindent {3ex } 
        \newcommand*\Wrappedafterbreak {\kern\Wrappedcontinuationindent\copy\Wrappedcontinuationbox} 
        % Take advantage of the already applied Pygments mark-up to insert 
        % potential linebreaks for TeX processing. 
        %        {, <, #, %, $, ' and ": go to next line. 
        %        _, }, ^, &, >, - and ~: stay at end of broken line. 
        % Use of \textquotesingle for straight quote. 
        \newcommand*\Wrappedbreaksatspecials {% 
            \def\PYGZus{\discretionary{\char`\_}{\Wrappedafterbreak}{\char`\_}}% 
            \def\PYGZob{\discretionary{}{\Wrappedafterbreak\char`\{}{\char`\{}}% 
            \def\PYGZcb{\discretionary{\char`\}}{\Wrappedafterbreak}{\char`\}}}% 
            \def\PYGZca{\discretionary{\char`\^}{\Wrappedafterbreak}{\char`\^}}% 
            \def\PYGZam{\discretionary{\char`\&}{\Wrappedafterbreak}{\char`\&}}% 
            \def\PYGZlt{\discretionary{}{\Wrappedafterbreak\char`\<}{\char`\<}}% 
            \def\PYGZgt{\discretionary{\char`\>}{\Wrappedafterbreak}{\char`\>}}% 
            \def\PYGZsh{\discretionary{}{\Wrappedafterbreak\char`\#}{\char`\#}}% 
            \def\PYGZpc{\discretionary{}{\Wrappedafterbreak\char`\%}{\char`\%}}% 
            \def\PYGZdl{\discretionary{}{\Wrappedafterbreak\char`\$}{\char`\$}}% 
            \def\PYGZhy{\discretionary{\char`\-}{\Wrappedafterbreak}{\char`\-}}% 
            \def\PYGZsq{\discretionary{}{\Wrappedafterbreak\textquotesingle}{\textquotesingle}}% 
            \def\PYGZdq{\discretionary{}{\Wrappedafterbreak\char`\"}{\char`\"}}% 
            \def\PYGZti{\discretionary{\char`\~}{\Wrappedafterbreak}{\char`\~}}% 
        } 
        % Some characters . , ; ? ! / are not pygmentized. 
        % This macro makes them "active" and they will insert potential linebreaks 
        \newcommand*\Wrappedbreaksatpunct {% 
            \lccode`\~`\.\lowercase{\def~}{\discretionary{\hbox{\char`\.}}{\Wrappedafterbreak}{\hbox{\char`\.}}}% 
            \lccode`\~`\,\lowercase{\def~}{\discretionary{\hbox{\char`\,}}{\Wrappedafterbreak}{\hbox{\char`\,}}}% 
            \lccode`\~`\;\lowercase{\def~}{\discretionary{\hbox{\char`\;}}{\Wrappedafterbreak}{\hbox{\char`\;}}}% 
            \lccode`\~`\:\lowercase{\def~}{\discretionary{\hbox{\char`\:}}{\Wrappedafterbreak}{\hbox{\char`\:}}}% 
            \lccode`\~`\?\lowercase{\def~}{\discretionary{\hbox{\char`\?}}{\Wrappedafterbreak}{\hbox{\char`\?}}}% 
            \lccode`\~`\!\lowercase{\def~}{\discretionary{\hbox{\char`\!}}{\Wrappedafterbreak}{\hbox{\char`\!}}}% 
            \lccode`\~`\/\lowercase{\def~}{\discretionary{\hbox{\char`\/}}{\Wrappedafterbreak}{\hbox{\char`\/}}}% 
            \catcode`\.\active
            \catcode`\,\active 
            \catcode`\;\active
            \catcode`\:\active
            \catcode`\?\active
            \catcode`\!\active
            \catcode`\/\active 
            \lccode`\~`\~ 	
        }
    \makeatother

    \let\OriginalVerbatim=\Verbatim
    \makeatletter
    \renewcommand{\Verbatim}[1][1]{%
        %\parskip\z@skip
        \sbox\Wrappedcontinuationbox {\Wrappedcontinuationsymbol}%
        \sbox\Wrappedvisiblespacebox {\FV@SetupFont\Wrappedvisiblespace}%
        \def\FancyVerbFormatLine ##1{\hsize\linewidth
            \vtop{\raggedright\hyphenpenalty\z@\exhyphenpenalty\z@
                \doublehyphendemerits\z@\finalhyphendemerits\z@
                \strut ##1\strut}%
        }%
        % If the linebreak is at a space, the latter will be displayed as visible
        % space at end of first line, and a continuation symbol starts next line.
        % Stretch/shrink are however usually zero for typewriter font.
        \def\FV@Space {%
            \nobreak\hskip\z@ plus\fontdimen3\font minus\fontdimen4\font
            \discretionary{\copy\Wrappedvisiblespacebox}{\Wrappedafterbreak}
            {\kern\fontdimen2\font}%
        }%
        
        % Allow breaks at special characters using \PYG... macros.
        \Wrappedbreaksatspecials
        % Breaks at punctuation characters . , ; ? ! and / need catcode=\active 	
        \OriginalVerbatim[#1,codes*=\Wrappedbreaksatpunct]%
    }
    \makeatother

    % Exact colors from NB
    \definecolor{incolor}{HTML}{303F9F}
    \definecolor{outcolor}{HTML}{D84315}
    \definecolor{cellborder}{HTML}{CFCFCF}
    \definecolor{cellbackground}{HTML}{F7F7F7}
    
    % prompt
    \makeatletter
    \newcommand{\boxspacing}{\kern\kvtcb@left@rule\kern\kvtcb@boxsep}
    \makeatother
    \newcommand{\prompt}[4]{
        \ttfamily\llap{{\color{#2}[#3]:\hspace{3pt}#4}}\vspace{-\baselineskip}
    }
    

    
    % Prevent overflowing lines due to hard-to-break entities
    \sloppy 
    % Setup hyperref package
    \hypersetup{
      breaklinks=true,  % so long urls are correctly broken across lines
      colorlinks=true,
      urlcolor=urlcolor,
      linkcolor=linkcolor,
      citecolor=citecolor,
      }
    % Slightly bigger margins than the latex defaults
    
    \geometry{verbose,tmargin=1in,bmargin=1in,lmargin=1in,rmargin=1in}
    
    

\begin{document}
    
    \maketitle
    
    

    
    \hypertarget{ux43eux441ux43dux43eux432ux43dux44bux435-ux441ux432ux435ux434ux435ux43dux438ux44f-ux438ux437-ux442ux435ux43eux440ux438ux438-ux432ux435ux440ux43eux44fux442ux43dux43eux441ux442ux435ux439}{%
\section{Основные сведения из теории
вероятностей}\label{ux43eux441ux43dux43eux432ux43dux44bux435-ux441ux432ux435ux434ux435ux43dux438ux44f-ux438ux437-ux442ux435ux43eux440ux438ux438-ux432ux435ux440ux43eux44fux442ux43dux43eux441ux442ux435ux439}}

    Предметом теории вероятностей является математический анализ случайных
явлений --- эмпирических феноменов, которые (при заданном «комплексе
условий») могут быть охарактеризованы тем, что

\begin{itemize}
\tightlist
\item
  для них отсутствует \emph{детерминистическая регулярность} (наблюдения
  над ними не всегда приводят к одним и тем же исходам)
\end{itemize}

и в то же самое время

\begin{itemize}
\tightlist
\item
  они обладают некоторой \emph{статистической регулярностью}
  (проявляющейся в статистической устойчивости частот).
\end{itemize}

    \begin{center}\rule{0.5\linewidth}{\linethickness}\end{center}

    \hypertarget{ux432ux435ux440ux43eux44fux442ux43dux43eux441ux442ux44c}{%
\subsection{Вероятность}\label{ux432ux435ux440ux43eux44fux442ux43dux43eux441ux442ux44c}}

\hypertarget{ux43cux43dux43eux436ux435ux441ux442ux432ux43e-ux438ux441ux445ux43eux434ux43eux432}{%
\subsubsection{Множество
исходов}\label{ux43cux43dux43eux436ux435ux441ux442ux432ux43e-ux438ux441ux445ux43eux434ux43eux432}}

Рассмотрим некоторый эксперимент, результаты которого (при данном
«\emph{комплексе условий}») описываются конечным числом различных
\emph{исходов} \(\omega_1, \dots , \omega_N\). Для нас несущественна
реальная природа этих исходов, важно лишь то, что их число \(N\)
конечно.

Исходы \(\omega_1, \dots , \omega_N\) будем также называть
\emph{элементарными событиями}, а их совокупность

\[ \Omega = \{ \omega_1, \dots , \omega_N \} \]

(конечным) \emph{пространством элементарных событий} или
\emph{пространством исходов}.

Выделение пространства элементарных событий представляет собой первый
шаг в формулировании понятия \emph{вероятностной модели} (вероятностной
«теории») того или иного эксперимента.

    \hypertarget{ux430ux43bux433ux435ux431ux440ux430}{%
\subsubsection{Алгебра}\label{ux430ux43bux433ux435ux431ux440ux430}}

Если рассматривается некоторая система \(\mathcal{A}_0\) множеств
\(A \subseteq \Omega\), то с помощью теоретико-множественных операций
\(\cup\), \(\cap\) и \(\setminus\) можно из элементов \(\mathcal{A}_0\)
построить новую систему множеств, которые также являются событиями.
Присоединяя к этим событиям достоверное и невозможное события \(\Omega\)
и \(\varnothing\), получаем систему множеств \(\mathcal{A}\), которая
является \emph{алгеброй}, т. е. такой системой подмножеств множества
\(\Omega\), что

\begin{enumerate}
\def\labelenumi{\arabic{enumi}.}
\tightlist
\item
  \(\Omega \in \mathcal{A}\),
\item
  если \(A \in \mathcal{A}\), \(B \in \mathcal{A}\), то множества
  \(A \cup B\), \(A \cap B\), \(A \setminus B\) также принадлежат
  \(\mathcal{A}\).
\end{enumerate}

Из сказанного следует, что в качестве систем событий целесообразно
рассматривать такие системы множеств, которые являются алгебрами.

    \hypertarget{ux432ux435ux440ux43eux44fux442ux43dux43eux441ux442ux43dux430ux44f-ux43cux435ux440ux430}{%
\subsubsection{Вероятностная
мера}\label{ux432ux435ux440ux43eux44fux442ux43dux43eux441ux442ux43dux430ux44f-ux43cux435ux440ux430}}

Пока мы сделали два первых шага к построению вероятностной модели
(«теории») эксперимента с конечным числом исходов: выделили пространство
исходов \(\Omega\) и некоторую систему \(\mathcal{A}\) его подмножеств,
образующих алгебру и называемых событиями. Сделаем теперь следующий шаг,
а именно припишем каждому элементарному событию (исходу, явлению)
\(\omega_i \in \Omega\), \(i=1, \ldots, N\), некоторый «вес»,
обозначаемый \(p(\omega_i)\) (или \(p_i\)) и называемый
\emph{вероятностью} исхода \(\omega_i\), который будем считать
удовлетворяющим следующим условиям:

\begin{enumerate}
\def\labelenumi{\arabic{enumi}.}
\tightlist
\item
  \(0 \le p(\omega_i) \le 1\) (\emph{неотрицательность}),
\item
  \(p(\omega_1) + \ldots + p(\omega_N) = 1\) (\emph{нормированность}).
\end{enumerate}

Отправляясь от заданных вероятностей \(p(\omega_i)\) исходов
\(\omega_i\) определим \emph{вероятность} \(P(A)\) любого события
\(A \in \mathcal{A}\) по формуле
\[ P(A) = \sum_{\{i:\omega_i \in A\}} p(\omega_i). \]

    \hypertarget{ux432ux435ux440ux43eux44fux442ux43dux43eux441ux442ux43dux43eux435-ux43fux440ux43eux441ux442ux440ux430ux43dux441ux442ux432ux43e}{%
\subsubsection{Вероятностное
пространство}\label{ux432ux435ux440ux43eux44fux442ux43dux43eux441ux442ux43dux43eux435-ux43fux440ux43eux441ux442ux440ux430ux43dux441ux442ux432ux43e}}

\textbf{Определение 1.} Принято говорить, что «вероятностное
пространство»

\[ \left( \Omega, \mathcal{A}, \mathbb{P} \right), \]

где \(\Omega = {\omega_1, \ldots, \omega_N}\), \(\mathcal{A}\) ---
некоторая алгебра подмножеств \(\Omega\) и
\(\mathbb{P} = \{ P(A); A \in \mathcal{A} \}\), определяет (задает)
вероятностную модель («теорию») эксперимента с (конечным) пространством
исходов (элементарных событий) \(\Omega\) и алгеброй событий
\(\mathcal{A}\).

    \hypertarget{ux441ux432ux43eux439ux441ux442ux432ux430-ux432ux435ux440ux43eux44fux442ux43dux43eux441ux442ux435ux439}{%
\subsubsection{Свойства
вероятностей}\label{ux441ux432ux43eux439ux441ux442ux432ux430-ux432ux435ux440ux43eux44fux442ux43dux43eux441ux442ux435ux439}}

\begin{enumerate}
\def\labelenumi{\arabic{enumi}.}
\tightlist
\item
  \(P(\varnothing) = 0\),
\item
  \(P(\Omega) = 1\),
\item
  \(P(A \cup B) = P(A) + P(B) - P(A \cap B)\),
\item
  \(P(\bar{A}) = 1 - P(A)\).
\end{enumerate}

    \hypertarget{ux437ux430ux43cux435ux447ux430ux43dux438ux44f}{%
\subsubsection{Замечания}\label{ux437ux430ux43cux435ux447ux430ux43dux438ux44f}}

При построении вероятностных моделей в конкретных ситуациях выделение
пространства элементарных событий \(\Omega\) и алгебры событий
\(\mathcal{A}\), как правило, не является сложной задачей. При этом в
элементарной теории вероятностей в качестве алгебры \(\mathcal{A}\)
обычно берется алгебра \emph{всех} подмножеств \(\Omega\). Труднее
обстоит дело с вопросом о том, как задавать вероятности элементарных
событий. В сущности, ответ на этот вопрос лежит вне рамок теории
вероятностей, и мы его подробно не рассматриваем, считая, что основной
нашей задачей является не вопрос о том, как приписывать исходам те или
иные вероятности, а \emph{вычисление} вероятностей сложных событий
(событий из \(\mathcal{A}\)) по вероятностям элементарных событий.

С математической точки зрения ясно, что в случае конечного пространства
элементарных событий с помощью приписывания исходам
\(\omega_1, \ldots , \omega_N\) неотрицательных чисел
\(p_1, \ldots , p_N\), удовлетворяющих условию
\(p_1 + \ldots + p_N = 1\), мы получаем все мыслимые (конечные)
вероятностные пространства.

\emph{Правильность} же назначенных для конкретной ситуации значений
\(p_1, \ldots , p_N\) может быть до известной степени проверена с
помощью рассматриваемого далее \emph{закона больших чисел}, согласно
которому в длинных сериях «независимых» экспериментов, происходящих при
одинаковых условиях, частоты появления элементарных событий «близки» к
их вероятностям.

    \begin{center}\rule{0.5\linewidth}{\linethickness}\end{center}

    \hypertarget{ux443ux441ux43bux43eux432ux43dux430ux44f-ux432ux435ux440ux43eux44fux442ux43dux43eux441ux442ux44c}{%
\subsection{Условная
вероятность}\label{ux443ux441ux43bux43eux432ux43dux430ux44f-ux432ux435ux440ux43eux44fux442ux43dux43eux441ux442ux44c}}

\hypertarget{ux43eux43fux440ux435ux434ux435ux43bux435ux43dux438ux435}{%
\subsubsection{Определение}\label{ux43eux43fux440ux435ux434ux435ux43bux435ux43dux438ux435}}

Понятие \emph{вероятности} события дает нам возможность ответить на
вопрос такого типа: если урна содержит M шаров, из которых \(M_1\) шаров
белого цвета и \(M_2\) --- чёрного, то какова вероятность \(P(A)\)
события \(A\), состоящего в том, что вытащенный шар имеет белый цвет? В
случае классического подхода \(P(A) =M_1/M\).

Вводимое ниже понятие условной вероятности позволяет отвечать на вопрос
следующего типа: какова вероятность того, что второй извлеченный шар
белого цвета (событие \(B\)), при условии, что первый шар также имеет
белый цвет (событие \(A\))? (Рассматривается выбор без возвращения.)

Естественно здесь рассуждать так: если первый извлеченный шар имел белый
цвет, то перед вторым извлечением мы имеем урну с \(M-1\) шаром, из
которых \(M_1 - 1\) шаров имеют белый цвет, а \(M_2\) --- чёрный;
поэтому интуитивно представляется целесообразным считать, что
интересующая нас (условная) вероятность равна \(\dfrac{M_1-1}{M-1}\).

Дадим теперь определение условной вероятности, согласующееся с
интуитивными представлениями о ней.

\textbf{Определение 2.} Условной вероятностью события \(B\) при условии
события \(A\) с \(P(A)>0\) (обозначение: \(P(B|A)\) ) называется
величина \[ \dfrac{P(AB)}{P(A)}. \]

    \hypertarget{ux441ux432ux43eux439ux441ux442ux432ux430-ux443ux441ux43bux43eux432ux43dux44bux445-ux432ux435ux440ux43eux44fux442ux43dux43eux441ux442ux435ux439}{%
\subsubsection{Свойства условных
вероятностей}\label{ux441ux432ux43eux439ux441ux442ux432ux430-ux443ux441ux43bux43eux432ux43dux44bux445-ux432ux435ux440ux43eux44fux442ux43dux43eux441ux442ux435ux439}}

\begin{enumerate}
\def\labelenumi{\arabic{enumi}.}
\tightlist
\item
  \(P(A|A) = 1\),
\item
  \(P(\varnothing|A) = 0\),
\item
  \(P(B|A) = 1\), \(B \supseteq A\),
\item
  \(P(B_1 + B_2|A) = P(B_1|A) + P(B_2|A)\).
\end{enumerate}

\textbf{Пример 1.} Рассмотрим семьи, имеющие двух детей. Спрашивается,
какова вероятность того, что в семье оба ребенка мальчики, в
предположении, что: 1. старший ребенок --- мальчик; 2. по крайней мере
один из детей --- мальчик?

    \hypertarget{ux444ux43eux440ux43cux443ux43bux430-ux43fux43eux43bux43dux43eux439-ux432ux435ux440ux43eux44fux442ux43dux43eux441ux442ux438}{%
\subsubsection{Формула полной
вероятности}\label{ux444ux43eux440ux43cux443ux43bux430-ux43fux43eux43bux43dux43eux439-ux432ux435ux440ux43eux44fux442ux43dux43eux441ux442ux438}}

Рассмотрим \emph{полную группу несовмстимых событий}
\(\mathcal{D} = \{A_1, \dots, A_n\}\). Имеет место \textbf{формула
полной вероятности} \[ P(B) = \sum_{i=1}^n P(B|A_i) P(A_i). \]

\textbf{Пример 2.} В урне имеется \(M\) шаров, среди которых \(m\)
«счастливых». Спрашивается, какова вероятность извлечь на втором шаге
«счастливый» шар (предполагается, что качество первого извлеченного шара
неизвестно).

Справедлива \textbf{формула умножения вероятностей}:
\[ P(AB) = P(B|A)P(A). \]

По индукции:
\[ P(A_1, \dots, A_n) = P(A_1)P(A_2|A_1) \dots P(A_n|A_1 \dots A_{n-1}). \]

    \hypertarget{ux43dux435ux437ux430ux432ux438ux441ux438ux43cux43eux441ux442ux44c}{%
\subsubsection{Независимость}\label{ux43dux435ux437ux430ux432ux438ux441ux438ux43cux43eux441ux442ux44c}}

\textbf{Определение 3.} События \(A\) и \(B\) называются
\emph{независимыми} или \emph{статистически независимыми} (относительно
вероятности \(P\)), если \[ P(AB) = P(A) \cdot P(B). \]

    \hypertarget{ux442ux435ux43eux440ux435ux43cux430-ux431ux430ux439ux435ux441ux430}{%
\subsubsection{Теорема
Байеса}\label{ux442ux435ux43eux440ux435ux43cux430-ux431ux430ux439ux435ux441ux430}}

Из формул \(P(B|A) = \dfrac{P(AB)}{P(A)}\) и
\(P(A|B) = \dfrac{P(AB)}{P(B)}\) получаем \textbf{формулу Байеса}:
\[ P(A|B) = \dfrac{P(B|A)P(A)}{P(B)}. \]

Если события \(A_1, \dots, A_n\) образубт разбиение \(\Omega\), то из
формул полной вероятности и Байеса следует \textbf{теорема Байеса}:
\[ P(A_i|B) = \frac{P(B|A_i) P(A_i)}{\sum_{j=1}^{n} P(A_j) P(B|A_j)}. \]

В статистических применениях события \(A_1, \dots, A_n\) образующие
«полную группу событий» (\(A_1 + \dots + A_n = \Omega\)), часто называют
«гипотезами», а \(P(A_i)\) -- \emph{априорной} вероятностью гипотезы
\(A_i\). Условная вероятность \(P(A_i|B)\) трактуется как
\emph{апостериорная} вероятность гипотезы \(A_i\) после наступления
события \(B\).

\textbf{Пример 3.} Пусть в урне находятся две монеты: \(A_1\) --
симметричная монета с вероятностью «герба» Г, равной 1/2, и \(A_2\) --
несимметричная монета с вероятностью «герба» Г, равной 1/3. Наудачу
вынимается и подбрасывается одна из монет. Предположим, что выпал герб.
Спрашивается, какова вероятность того, что выбранная монета симметрична.

    \begin{center}\rule{0.5\linewidth}{\linethickness}\end{center}

    \hypertarget{ux441ux43bux443ux447ux430ux439ux43dux44bux435-ux432ux435ux43bux438ux447ux438ux43dux44b-ux438-ux438ux445-ux445ux430ux440ux430ux43aux442ux435ux440ux438ux441ux442ux438ux43aux438}{%
\subsection{Случайные величины и их
характеристики}\label{ux441ux43bux443ux447ux430ux439ux43dux44bux435-ux432ux435ux43bux438ux447ux438ux43dux44b-ux438-ux438ux445-ux445ux430ux440ux430ux43aux442ux435ux440ux438ux441ux442ux438ux43aux438}}

\textbf{Определение 4.} Всякая числовая фнукция \(\xi = \xi(\omega)\),
определённая на (конечном) пространстве элементарных событий \(\Omega\),
будет называться \emph{случайной величиной}.

\textbf{Определение 5.} Пусть \(x \in R^1\). Функция \[
    F_\xi(x)  = \left\{ \omega: \xi(\omega) \le x \right\}
\] называется \emph{функцией распределения} случайной величины \(\xi\).

\textbf{Определение 6.} \emph{Математическим ожиданием} или
\emph{средним значением} случайной величины $ \xi =
\sum\limits\_\{i=1\}\^{}k x\_i I \left( A\_i \right) $ называется число
\[
    \mathrm{E}\xi = \sum\limits_{i=1}^k x_i P \left( A_i \right).
\]

\textbf{Определение 7.} \emph{Дисперсией} случайной величины \(\xi\)
называется величина \[
    \mathrm{D} \xi = \mathrm{E} \left( \xi - \mathrm{E} \xi \right)^2.
\] Величина \(\sigma = +\sqrt{D \xi}\) называется \emph{стандартным
отклонением} значений случайной величины \(\xi\) от её среднего значения
\(\mathrm{E} \xi\). \[
    \mathrm{D} \xi = \mathrm{E} \xi^2 - \left( \mathrm{E} \xi \right)^2
\]

\textbf{Определение 8.} Пусть \(( \xi, \eta )\) --- пара случайных
величин. Их \emph{ковариацией} называется величина \[
    \mathrm{cov}( \xi, \eta ) = \mathrm{E}(\xi - \mathrm{E} \xi)(\eta - \mathrm{E} \eta).
\] Если \(\mathrm{cov}\left( \xi, \eta \right) = 0\), то говорят, что
случайные величины \(\xi\) и \(\eta\) \emph{не коррелированы}.

Если \(0<\mathrm{D}\xi < \infty\), \(0<\mathrm{D}\eta < \infty\), то
величина \[
    \rho(\xi, \eta) \equiv \dfrac{\mathrm{cov}(\xi, \eta)}{\sqrt{\mathrm{D}\xi \cdot \mathrm{D}\eta}}
\] называется \emph{коэффициентом корреляции} случайных величин \(\xi\)
и \(\eta\).

\textbf{Замечание.} Из некоррелированности \(\xi\) и \(\eta\), вообще
говоря, не следует их независимость. Вот простой пример. Пусть случайная
величина \(\alpha\) принимает значения 0, \(\pi/2\) и \(\pi\) с
вероятностями 1/3. Тогда \(\xi = \sin \alpha\) и \(\eta = \cos \alpha\)
некоррелированы; в то же время они не только зависимы относительно
вероятности, но и \emph{функционально зависимы}.

    \begin{center}\rule{0.5\linewidth}{\linethickness}\end{center}

    \hypertarget{ux43bux438ux442ux435ux440ux430ux442ux443ux440ux430}{%
\subsection{Литература}\label{ux43bux438ux442ux435ux440ux430ux442ux443ux440ux430}}

\begin{enumerate}
\def\labelenumi{\arabic{enumi}.}
\tightlist
\item
  Ширяев А.Н. Вероятность --- 1. М.: МЦНМО, 2007
\end{enumerate}

    \hypertarget{ux447ux442ux43e-ux441ux434ux435ux43bux430ux442ux44c}{%
\subsection{Что
сделать:}\label{ux447ux442ux43e-ux441ux434ux435ux43bux430ux442ux44c}}

\begin{itemize}
\tightlist
\item
  добавить свойства мат. ожидания и дисперсии: сумма, афинные
  преобразования
\end{itemize}


    % Add a bibliography block to the postdoc
    
    
    
\end{document}
