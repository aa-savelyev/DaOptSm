\documentclass[11pt,a4paper]{article}

    \usepackage[breakable]{tcolorbox}
    \usepackage{parskip} % Stop auto-indenting (to mimic markdown behaviour)
    
    \usepackage{iftex}
    \ifPDFTeX
      \usepackage[T2A]{fontenc}
      \usepackage{mathpazo}
      \usepackage[russian,english]{babel}
    \else
      \usepackage{fontspec}
      \usepackage{polyglossia}
      \setmainlanguage[babelshorthands=true]{russian}    % Язык по-умолчанию русский с поддержкой приятных команд пакета babel
      \setotherlanguage{english}                         % Дополнительный язык = английский (в американской вариации по-умолчанию)
      \newfontfamily\cyrillicfonttt[Scale=0.87,BoldFont={Fira Mono Medium}] {Fira Mono}  % Моноширинный шрифт для кириллицы
      \defaultfontfeatures{Ligatures=TeX}
      \newfontfamily\cyrillicfont{STIX Two Text}         % Шрифт с засечками для кириллицы
    \fi
    \renewcommand{\linethickness}{0.1ex}

    % Basic figure setup, for now with no caption control since it's done
    % automatically by Pandoc (which extracts ![](path) syntax from Markdown).
    \usepackage{graphicx}
    % Maintain compatibility with old templates. Remove in nbconvert 6.0
    \let\Oldincludegraphics\includegraphics
    % Ensure that by default, figures have no caption (until we provide a
    % proper Figure object with a Caption API and a way to capture that
    % in the conversion process - todo).
    \usepackage{caption}
    \DeclareCaptionFormat{nocaption}{}
    \captionsetup{format=nocaption,aboveskip=0pt,belowskip=0pt}

    \usepackage[Export]{adjustbox} % Used to constrain images to a maximum size
    \adjustboxset{max size={0.9\linewidth}{0.9\paperheight}}
    \usepackage{float}
    \floatplacement{figure}{H} % forces figures to be placed at the correct location
    \usepackage{xcolor} % Allow colors to be defined
    \usepackage{enumerate} % Needed for markdown enumerations to work
    \usepackage{geometry} % Used to adjust the document margins
    \usepackage{amsmath} % Equations
    \usepackage{amssymb} % Equations
    \usepackage{textcomp} % defines textquotesingle
    % Hack from http://tex.stackexchange.com/a/47451/13684:
    \AtBeginDocument{%
        \def\PYZsq{\textquotesingle}% Upright quotes in Pygmentized code
    }
    \usepackage{upquote} % Upright quotes for verbatim code
    \usepackage{eurosym} % defines \euro
    \usepackage[mathletters]{ucs} % Extended unicode (utf-8) support
    \usepackage{fancyvrb} % verbatim replacement that allows latex
    \usepackage{grffile} % extends the file name processing of package graphics 
                         % to support a larger range
    \makeatletter % fix for grffile with XeLaTeX
    \def\Gread@@xetex#1{%
      \IfFileExists{"\Gin@base".bb}%
      {\Gread@eps{\Gin@base.bb}}%
      {\Gread@@xetex@aux#1}%
    }
    \makeatother

    % The hyperref package gives us a pdf with properly built
    % internal navigation ('pdf bookmarks' for the table of contents,
    % internal cross-reference links, web links for URLs, etc.)
    \usepackage{hyperref}
    % The default LaTeX title has an obnoxious amount of whitespace. By default,
    % titling removes some of it. It also provides customization options.
    \usepackage{titling}
    \usepackage{longtable} % longtable support required by pandoc >1.10
    \usepackage{booktabs}  % table support for pandoc > 1.12.2
    \usepackage[inline]{enumitem} % IRkernel/repr support (it uses the enumerate* environment)
    \usepackage[normalem]{ulem} % ulem is needed to support strikethroughs (\sout)
                                % normalem makes italics be italics, not underlines
    \usepackage{mathrsfs}
    

    
    % Colors for the hyperref package
    \definecolor{urlcolor}{rgb}{0,.145,.698}
    \definecolor{linkcolor}{rgb}{.71,0.21,0.01}
    \definecolor{citecolor}{rgb}{.12,.54,.11}

    % ANSI colors
    \definecolor{ansi-black}{HTML}{3E424D}
    \definecolor{ansi-black-intense}{HTML}{282C36}
    \definecolor{ansi-red}{HTML}{E75C58}
    \definecolor{ansi-red-intense}{HTML}{B22B31}
    \definecolor{ansi-green}{HTML}{00A250}
    \definecolor{ansi-green-intense}{HTML}{007427}
    \definecolor{ansi-yellow}{HTML}{DDB62B}
    \definecolor{ansi-yellow-intense}{HTML}{B27D12}
    \definecolor{ansi-blue}{HTML}{208FFB}
    \definecolor{ansi-blue-intense}{HTML}{0065CA}
    \definecolor{ansi-magenta}{HTML}{D160C4}
    \definecolor{ansi-magenta-intense}{HTML}{A03196}
    \definecolor{ansi-cyan}{HTML}{60C6C8}
    \definecolor{ansi-cyan-intense}{HTML}{258F8F}
    \definecolor{ansi-white}{HTML}{C5C1B4}
    \definecolor{ansi-white-intense}{HTML}{A1A6B2}
    \definecolor{ansi-default-inverse-fg}{HTML}{FFFFFF}
    \definecolor{ansi-default-inverse-bg}{HTML}{000000}

    % commands and environments needed by pandoc snippets
    % extracted from the output of `pandoc -s`
    \providecommand{\tightlist}{%
      \setlength{\itemsep}{0pt}\setlength{\parskip}{0pt}}
    \DefineVerbatimEnvironment{Highlighting}{Verbatim}{commandchars=\\\{\}}
    % Add ',fontsize=\small' for more characters per line
    \newenvironment{Shaded}{}{}
    \newcommand{\KeywordTok}[1]{\textcolor[rgb]{0.00,0.44,0.13}{\textbf{{#1}}}}
    \newcommand{\DataTypeTok}[1]{\textcolor[rgb]{0.56,0.13,0.00}{{#1}}}
    \newcommand{\DecValTok}[1]{\textcolor[rgb]{0.25,0.63,0.44}{{#1}}}
    \newcommand{\BaseNTok}[1]{\textcolor[rgb]{0.25,0.63,0.44}{{#1}}}
    \newcommand{\FloatTok}[1]{\textcolor[rgb]{0.25,0.63,0.44}{{#1}}}
    \newcommand{\CharTok}[1]{\textcolor[rgb]{0.25,0.44,0.63}{{#1}}}
    \newcommand{\StringTok}[1]{\textcolor[rgb]{0.25,0.44,0.63}{{#1}}}
    \newcommand{\CommentTok}[1]{\textcolor[rgb]{0.38,0.63,0.69}{\textit{{#1}}}}
    \newcommand{\OtherTok}[1]{\textcolor[rgb]{0.00,0.44,0.13}{{#1}}}
    \newcommand{\AlertTok}[1]{\textcolor[rgb]{1.00,0.00,0.00}{\textbf{{#1}}}}
    \newcommand{\FunctionTok}[1]{\textcolor[rgb]{0.02,0.16,0.49}{{#1}}}
    \newcommand{\RegionMarkerTok}[1]{{#1}}
    \newcommand{\ErrorTok}[1]{\textcolor[rgb]{1.00,0.00,0.00}{\textbf{{#1}}}}
    \newcommand{\NormalTok}[1]{{#1}}
    
    % Additional commands for more recent versions of Pandoc
    \newcommand{\ConstantTok}[1]{\textcolor[rgb]{0.53,0.00,0.00}{{#1}}}
    \newcommand{\SpecialCharTok}[1]{\textcolor[rgb]{0.25,0.44,0.63}{{#1}}}
    \newcommand{\VerbatimStringTok}[1]{\textcolor[rgb]{0.25,0.44,0.63}{{#1}}}
    \newcommand{\SpecialStringTok}[1]{\textcolor[rgb]{0.73,0.40,0.53}{{#1}}}
    \newcommand{\ImportTok}[1]{{#1}}
    \newcommand{\DocumentationTok}[1]{\textcolor[rgb]{0.73,0.13,0.13}{\textit{{#1}}}}
    \newcommand{\AnnotationTok}[1]{\textcolor[rgb]{0.38,0.63,0.69}{\textbf{\textit{{#1}}}}}
    \newcommand{\CommentVarTok}[1]{\textcolor[rgb]{0.38,0.63,0.69}{\textbf{\textit{{#1}}}}}
    \newcommand{\VariableTok}[1]{\textcolor[rgb]{0.10,0.09,0.49}{{#1}}}
    \newcommand{\ControlFlowTok}[1]{\textcolor[rgb]{0.00,0.44,0.13}{\textbf{{#1}}}}
    \newcommand{\OperatorTok}[1]{\textcolor[rgb]{0.40,0.40,0.40}{{#1}}}
    \newcommand{\BuiltInTok}[1]{{#1}}
    \newcommand{\ExtensionTok}[1]{{#1}}
    \newcommand{\PreprocessorTok}[1]{\textcolor[rgb]{0.74,0.48,0.00}{{#1}}}
    \newcommand{\AttributeTok}[1]{\textcolor[rgb]{0.49,0.56,0.16}{{#1}}}
    \newcommand{\InformationTok}[1]{\textcolor[rgb]{0.38,0.63,0.69}{\textbf{\textit{{#1}}}}}
    \newcommand{\WarningTok}[1]{\textcolor[rgb]{0.38,0.63,0.69}{\textbf{\textit{{#1}}}}}
    
    
    % Define a nice break command that doesn't care if a line doesn't already
    % exist.
    \def\br{\hspace*{\fill} \\* }
    % Math Jax compatibility definitions
    \def\gt{>}
    \def\lt{<}
    \let\Oldtex\TeX
    \let\Oldlatex\LaTeX
    \renewcommand{\TeX}{\textrm{\Oldtex}}
    \renewcommand{\LaTeX}{\textrm{\Oldlatex}}
    % Document parameters
    % Document title
    \title{Лекция 6 \\
    Собственные значения и собственные векторы
    }
    
    
    
    
    
% Pygments definitions
\makeatletter
\def\PY@reset{\let\PY@it=\relax \let\PY@bf=\relax%
    \let\PY@ul=\relax \let\PY@tc=\relax%
    \let\PY@bc=\relax \let\PY@ff=\relax}
\def\PY@tok#1{\csname PY@tok@#1\endcsname}
\def\PY@toks#1+{\ifx\relax#1\empty\else%
    \PY@tok{#1}\expandafter\PY@toks\fi}
\def\PY@do#1{\PY@bc{\PY@tc{\PY@ul{%
    \PY@it{\PY@bf{\PY@ff{#1}}}}}}}
\def\PY#1#2{\PY@reset\PY@toks#1+\relax+\PY@do{#2}}

\expandafter\def\csname PY@tok@w\endcsname{\def\PY@tc##1{\textcolor[rgb]{0.73,0.73,0.73}{##1}}}
\expandafter\def\csname PY@tok@c\endcsname{\let\PY@it=\textit\def\PY@tc##1{\textcolor[rgb]{0.25,0.50,0.50}{##1}}}
\expandafter\def\csname PY@tok@cp\endcsname{\def\PY@tc##1{\textcolor[rgb]{0.74,0.48,0.00}{##1}}}
\expandafter\def\csname PY@tok@k\endcsname{\let\PY@bf=\textbf\def\PY@tc##1{\textcolor[rgb]{0.00,0.50,0.00}{##1}}}
\expandafter\def\csname PY@tok@kp\endcsname{\def\PY@tc##1{\textcolor[rgb]{0.00,0.50,0.00}{##1}}}
\expandafter\def\csname PY@tok@kt\endcsname{\def\PY@tc##1{\textcolor[rgb]{0.69,0.00,0.25}{##1}}}
\expandafter\def\csname PY@tok@o\endcsname{\def\PY@tc##1{\textcolor[rgb]{0.40,0.40,0.40}{##1}}}
\expandafter\def\csname PY@tok@ow\endcsname{\let\PY@bf=\textbf\def\PY@tc##1{\textcolor[rgb]{0.67,0.13,1.00}{##1}}}
\expandafter\def\csname PY@tok@nb\endcsname{\def\PY@tc##1{\textcolor[rgb]{0.00,0.50,0.00}{##1}}}
\expandafter\def\csname PY@tok@nf\endcsname{\def\PY@tc##1{\textcolor[rgb]{0.00,0.00,1.00}{##1}}}
\expandafter\def\csname PY@tok@nc\endcsname{\let\PY@bf=\textbf\def\PY@tc##1{\textcolor[rgb]{0.00,0.00,1.00}{##1}}}
\expandafter\def\csname PY@tok@nn\endcsname{\let\PY@bf=\textbf\def\PY@tc##1{\textcolor[rgb]{0.00,0.00,1.00}{##1}}}
\expandafter\def\csname PY@tok@ne\endcsname{\let\PY@bf=\textbf\def\PY@tc##1{\textcolor[rgb]{0.82,0.25,0.23}{##1}}}
\expandafter\def\csname PY@tok@nv\endcsname{\def\PY@tc##1{\textcolor[rgb]{0.10,0.09,0.49}{##1}}}
\expandafter\def\csname PY@tok@no\endcsname{\def\PY@tc##1{\textcolor[rgb]{0.53,0.00,0.00}{##1}}}
\expandafter\def\csname PY@tok@nl\endcsname{\def\PY@tc##1{\textcolor[rgb]{0.63,0.63,0.00}{##1}}}
\expandafter\def\csname PY@tok@ni\endcsname{\let\PY@bf=\textbf\def\PY@tc##1{\textcolor[rgb]{0.60,0.60,0.60}{##1}}}
\expandafter\def\csname PY@tok@na\endcsname{\def\PY@tc##1{\textcolor[rgb]{0.49,0.56,0.16}{##1}}}
\expandafter\def\csname PY@tok@nt\endcsname{\let\PY@bf=\textbf\def\PY@tc##1{\textcolor[rgb]{0.00,0.50,0.00}{##1}}}
\expandafter\def\csname PY@tok@nd\endcsname{\def\PY@tc##1{\textcolor[rgb]{0.67,0.13,1.00}{##1}}}
\expandafter\def\csname PY@tok@s\endcsname{\def\PY@tc##1{\textcolor[rgb]{0.73,0.13,0.13}{##1}}}
\expandafter\def\csname PY@tok@sd\endcsname{\let\PY@it=\textit\def\PY@tc##1{\textcolor[rgb]{0.73,0.13,0.13}{##1}}}
\expandafter\def\csname PY@tok@si\endcsname{\let\PY@bf=\textbf\def\PY@tc##1{\textcolor[rgb]{0.73,0.40,0.53}{##1}}}
\expandafter\def\csname PY@tok@se\endcsname{\let\PY@bf=\textbf\def\PY@tc##1{\textcolor[rgb]{0.73,0.40,0.13}{##1}}}
\expandafter\def\csname PY@tok@sr\endcsname{\def\PY@tc##1{\textcolor[rgb]{0.73,0.40,0.53}{##1}}}
\expandafter\def\csname PY@tok@ss\endcsname{\def\PY@tc##1{\textcolor[rgb]{0.10,0.09,0.49}{##1}}}
\expandafter\def\csname PY@tok@sx\endcsname{\def\PY@tc##1{\textcolor[rgb]{0.00,0.50,0.00}{##1}}}
\expandafter\def\csname PY@tok@m\endcsname{\def\PY@tc##1{\textcolor[rgb]{0.40,0.40,0.40}{##1}}}
\expandafter\def\csname PY@tok@gh\endcsname{\let\PY@bf=\textbf\def\PY@tc##1{\textcolor[rgb]{0.00,0.00,0.50}{##1}}}
\expandafter\def\csname PY@tok@gu\endcsname{\let\PY@bf=\textbf\def\PY@tc##1{\textcolor[rgb]{0.50,0.00,0.50}{##1}}}
\expandafter\def\csname PY@tok@gd\endcsname{\def\PY@tc##1{\textcolor[rgb]{0.63,0.00,0.00}{##1}}}
\expandafter\def\csname PY@tok@gi\endcsname{\def\PY@tc##1{\textcolor[rgb]{0.00,0.63,0.00}{##1}}}
\expandafter\def\csname PY@tok@gr\endcsname{\def\PY@tc##1{\textcolor[rgb]{1.00,0.00,0.00}{##1}}}
\expandafter\def\csname PY@tok@ge\endcsname{\let\PY@it=\textit}
\expandafter\def\csname PY@tok@gs\endcsname{\let\PY@bf=\textbf}
\expandafter\def\csname PY@tok@gp\endcsname{\let\PY@bf=\textbf\def\PY@tc##1{\textcolor[rgb]{0.00,0.00,0.50}{##1}}}
\expandafter\def\csname PY@tok@go\endcsname{\def\PY@tc##1{\textcolor[rgb]{0.53,0.53,0.53}{##1}}}
\expandafter\def\csname PY@tok@gt\endcsname{\def\PY@tc##1{\textcolor[rgb]{0.00,0.27,0.87}{##1}}}
\expandafter\def\csname PY@tok@err\endcsname{\def\PY@bc##1{\setlength{\fboxsep}{0pt}\fcolorbox[rgb]{1.00,0.00,0.00}{1,1,1}{\strut ##1}}}
\expandafter\def\csname PY@tok@kc\endcsname{\let\PY@bf=\textbf\def\PY@tc##1{\textcolor[rgb]{0.00,0.50,0.00}{##1}}}
\expandafter\def\csname PY@tok@kd\endcsname{\let\PY@bf=\textbf\def\PY@tc##1{\textcolor[rgb]{0.00,0.50,0.00}{##1}}}
\expandafter\def\csname PY@tok@kn\endcsname{\let\PY@bf=\textbf\def\PY@tc##1{\textcolor[rgb]{0.00,0.50,0.00}{##1}}}
\expandafter\def\csname PY@tok@kr\endcsname{\let\PY@bf=\textbf\def\PY@tc##1{\textcolor[rgb]{0.00,0.50,0.00}{##1}}}
\expandafter\def\csname PY@tok@bp\endcsname{\def\PY@tc##1{\textcolor[rgb]{0.00,0.50,0.00}{##1}}}
\expandafter\def\csname PY@tok@fm\endcsname{\def\PY@tc##1{\textcolor[rgb]{0.00,0.00,1.00}{##1}}}
\expandafter\def\csname PY@tok@vc\endcsname{\def\PY@tc##1{\textcolor[rgb]{0.10,0.09,0.49}{##1}}}
\expandafter\def\csname PY@tok@vg\endcsname{\def\PY@tc##1{\textcolor[rgb]{0.10,0.09,0.49}{##1}}}
\expandafter\def\csname PY@tok@vi\endcsname{\def\PY@tc##1{\textcolor[rgb]{0.10,0.09,0.49}{##1}}}
\expandafter\def\csname PY@tok@vm\endcsname{\def\PY@tc##1{\textcolor[rgb]{0.10,0.09,0.49}{##1}}}
\expandafter\def\csname PY@tok@sa\endcsname{\def\PY@tc##1{\textcolor[rgb]{0.73,0.13,0.13}{##1}}}
\expandafter\def\csname PY@tok@sb\endcsname{\def\PY@tc##1{\textcolor[rgb]{0.73,0.13,0.13}{##1}}}
\expandafter\def\csname PY@tok@sc\endcsname{\def\PY@tc##1{\textcolor[rgb]{0.73,0.13,0.13}{##1}}}
\expandafter\def\csname PY@tok@dl\endcsname{\def\PY@tc##1{\textcolor[rgb]{0.73,0.13,0.13}{##1}}}
\expandafter\def\csname PY@tok@s2\endcsname{\def\PY@tc##1{\textcolor[rgb]{0.73,0.13,0.13}{##1}}}
\expandafter\def\csname PY@tok@sh\endcsname{\def\PY@tc##1{\textcolor[rgb]{0.73,0.13,0.13}{##1}}}
\expandafter\def\csname PY@tok@s1\endcsname{\def\PY@tc##1{\textcolor[rgb]{0.73,0.13,0.13}{##1}}}
\expandafter\def\csname PY@tok@mb\endcsname{\def\PY@tc##1{\textcolor[rgb]{0.40,0.40,0.40}{##1}}}
\expandafter\def\csname PY@tok@mf\endcsname{\def\PY@tc##1{\textcolor[rgb]{0.40,0.40,0.40}{##1}}}
\expandafter\def\csname PY@tok@mh\endcsname{\def\PY@tc##1{\textcolor[rgb]{0.40,0.40,0.40}{##1}}}
\expandafter\def\csname PY@tok@mi\endcsname{\def\PY@tc##1{\textcolor[rgb]{0.40,0.40,0.40}{##1}}}
\expandafter\def\csname PY@tok@il\endcsname{\def\PY@tc##1{\textcolor[rgb]{0.40,0.40,0.40}{##1}}}
\expandafter\def\csname PY@tok@mo\endcsname{\def\PY@tc##1{\textcolor[rgb]{0.40,0.40,0.40}{##1}}}
\expandafter\def\csname PY@tok@ch\endcsname{\let\PY@it=\textit\def\PY@tc##1{\textcolor[rgb]{0.25,0.50,0.50}{##1}}}
\expandafter\def\csname PY@tok@cm\endcsname{\let\PY@it=\textit\def\PY@tc##1{\textcolor[rgb]{0.25,0.50,0.50}{##1}}}
\expandafter\def\csname PY@tok@cpf\endcsname{\let\PY@it=\textit\def\PY@tc##1{\textcolor[rgb]{0.25,0.50,0.50}{##1}}}
\expandafter\def\csname PY@tok@c1\endcsname{\let\PY@it=\textit\def\PY@tc##1{\textcolor[rgb]{0.25,0.50,0.50}{##1}}}
\expandafter\def\csname PY@tok@cs\endcsname{\let\PY@it=\textit\def\PY@tc##1{\textcolor[rgb]{0.25,0.50,0.50}{##1}}}

\def\PYZbs{\char`\\}
\def\PYZus{\char`\_}
\def\PYZob{\char`\{}
\def\PYZcb{\char`\}}
\def\PYZca{\char`\^}
\def\PYZam{\char`\&}
\def\PYZlt{\char`\<}
\def\PYZgt{\char`\>}
\def\PYZsh{\char`\#}
\def\PYZpc{\char`\%}
\def\PYZdl{\char`\$}
\def\PYZhy{\char`\-}
\def\PYZsq{\char`\'}
\def\PYZdq{\char`\"}
\def\PYZti{\char`\~}
% for compatibility with earlier versions
\def\PYZat{@}
\def\PYZlb{[}
\def\PYZrb{]}
\makeatother


    % For linebreaks inside Verbatim environment from package fancyvrb. 
    \makeatletter
        \newbox\Wrappedcontinuationbox 
        \newbox\Wrappedvisiblespacebox 
        \newcommand*\Wrappedvisiblespace {\textcolor{red}{\textvisiblespace}} 
        \newcommand*\Wrappedcontinuationsymbol {\textcolor{red}{\llap{\tiny$\m@th\hookrightarrow$}}} 
        \newcommand*\Wrappedcontinuationindent {3ex } 
        \newcommand*\Wrappedafterbreak {\kern\Wrappedcontinuationindent\copy\Wrappedcontinuationbox} 
        % Take advantage of the already applied Pygments mark-up to insert 
        % potential linebreaks for TeX processing. 
        %        {, <, #, %, $, ' and ": go to next line. 
        %        _, }, ^, &, >, - and ~: stay at end of broken line. 
        % Use of \textquotesingle for straight quote. 
        \newcommand*\Wrappedbreaksatspecials {% 
            \def\PYGZus{\discretionary{\char`\_}{\Wrappedafterbreak}{\char`\_}}% 
            \def\PYGZob{\discretionary{}{\Wrappedafterbreak\char`\{}{\char`\{}}% 
            \def\PYGZcb{\discretionary{\char`\}}{\Wrappedafterbreak}{\char`\}}}% 
            \def\PYGZca{\discretionary{\char`\^}{\Wrappedafterbreak}{\char`\^}}% 
            \def\PYGZam{\discretionary{\char`\&}{\Wrappedafterbreak}{\char`\&}}% 
            \def\PYGZlt{\discretionary{}{\Wrappedafterbreak\char`\<}{\char`\<}}% 
            \def\PYGZgt{\discretionary{\char`\>}{\Wrappedafterbreak}{\char`\>}}% 
            \def\PYGZsh{\discretionary{}{\Wrappedafterbreak\char`\#}{\char`\#}}% 
            \def\PYGZpc{\discretionary{}{\Wrappedafterbreak\char`\%}{\char`\%}}% 
            \def\PYGZdl{\discretionary{}{\Wrappedafterbreak\char`\$}{\char`\$}}% 
            \def\PYGZhy{\discretionary{\char`\-}{\Wrappedafterbreak}{\char`\-}}% 
            \def\PYGZsq{\discretionary{}{\Wrappedafterbreak\textquotesingle}{\textquotesingle}}% 
            \def\PYGZdq{\discretionary{}{\Wrappedafterbreak\char`\"}{\char`\"}}% 
            \def\PYGZti{\discretionary{\char`\~}{\Wrappedafterbreak}{\char`\~}}% 
        } 
        % Some characters . , ; ? ! / are not pygmentized. 
        % This macro makes them "active" and they will insert potential linebreaks 
        \newcommand*\Wrappedbreaksatpunct {% 
            \lccode`\~`\.\lowercase{\def~}{\discretionary{\hbox{\char`\.}}{\Wrappedafterbreak}{\hbox{\char`\.}}}% 
            \lccode`\~`\,\lowercase{\def~}{\discretionary{\hbox{\char`\,}}{\Wrappedafterbreak}{\hbox{\char`\,}}}% 
            \lccode`\~`\;\lowercase{\def~}{\discretionary{\hbox{\char`\;}}{\Wrappedafterbreak}{\hbox{\char`\;}}}% 
            \lccode`\~`\:\lowercase{\def~}{\discretionary{\hbox{\char`\:}}{\Wrappedafterbreak}{\hbox{\char`\:}}}% 
            \lccode`\~`\?\lowercase{\def~}{\discretionary{\hbox{\char`\?}}{\Wrappedafterbreak}{\hbox{\char`\?}}}% 
            \lccode`\~`\!\lowercase{\def~}{\discretionary{\hbox{\char`\!}}{\Wrappedafterbreak}{\hbox{\char`\!}}}% 
            \lccode`\~`\/\lowercase{\def~}{\discretionary{\hbox{\char`\/}}{\Wrappedafterbreak}{\hbox{\char`\/}}}% 
            \catcode`\.\active
            \catcode`\,\active 
            \catcode`\;\active
            \catcode`\:\active
            \catcode`\?\active
            \catcode`\!\active
            \catcode`\/\active 
            \lccode`\~`\~ 	
        }
    \makeatother

    \let\OriginalVerbatim=\Verbatim
    \makeatletter
    \renewcommand{\Verbatim}[1][1]{%
        %\parskip\z@skip
        \sbox\Wrappedcontinuationbox {\Wrappedcontinuationsymbol}%
        \sbox\Wrappedvisiblespacebox {\FV@SetupFont\Wrappedvisiblespace}%
        \def\FancyVerbFormatLine ##1{\hsize\linewidth
            \vtop{\raggedright\hyphenpenalty\z@\exhyphenpenalty\z@
                \doublehyphendemerits\z@\finalhyphendemerits\z@
                \strut ##1\strut}%
        }%
        % If the linebreak is at a space, the latter will be displayed as visible
        % space at end of first line, and a continuation symbol starts next line.
        % Stretch/shrink are however usually zero for typewriter font.
        \def\FV@Space {%
            \nobreak\hskip\z@ plus\fontdimen3\font minus\fontdimen4\font
            \discretionary{\copy\Wrappedvisiblespacebox}{\Wrappedafterbreak}
            {\kern\fontdimen2\font}%
        }%
        
        % Allow breaks at special characters using \PYG... macros.
        \Wrappedbreaksatspecials
        % Breaks at punctuation characters . , ; ? ! and / need catcode=\active 	
        \OriginalVerbatim[#1,codes*=\Wrappedbreaksatpunct]%
    }
    \makeatother

    % Exact colors from NB
    \definecolor{incolor}{HTML}{303F9F}
    \definecolor{outcolor}{HTML}{D84315}
    \definecolor{cellborder}{HTML}{CFCFCF}
    \definecolor{cellbackground}{HTML}{F7F7F7}
    
    % prompt
    \makeatletter
    \newcommand{\boxspacing}{\kern\kvtcb@left@rule\kern\kvtcb@boxsep}
    \makeatother
    \newcommand{\prompt}[4]{
        \ttfamily\llap{{\color{#2}[#3]:\hspace{3pt}#4}}\vspace{-\baselineskip}
    }
    

    
    % Prevent overflowing lines due to hard-to-break entities
    \sloppy 
    % Setup hyperref package
    \hypersetup{
      breaklinks=true,  % so long urls are correctly broken across lines
      colorlinks=true,
      urlcolor=urlcolor,
      linkcolor=linkcolor,
      citecolor=citecolor,
      }
    % Slightly bigger margins than the latex defaults
    
    \geometry{verbose,tmargin=1in,bmargin=1in,lmargin=1in,rmargin=1in}
    
    

\begin{document}
    
    \maketitle


    \hypertarget{ux432ux432ux435ux434ux435ux43dux438ux435}{%
\section{Введение}\label{ux432ux432ux435ux434ux435ux43dux438ux435}}

Квадратную матрицу \(A\) размером \(n \times n\) мы интерпретируем как
преобразование, действующее на вектор \(\mathbf{x} \in \mathbb{R}^n\),
преобразуя его в новый вектор \(\mathbf{y} = A\mathbf{x}\), также
лежащий в \(\mathbb{R}^n\).

\textbf{Примеры:}

\begin{enumerate}
\def\labelenumi{\arabic{enumi}.}
\tightlist
\item
  Матрица поворота \[
  A = 
  \begin{pmatrix}
    \cos{\theta} & -\sin{\theta} \\
    \sin{\theta} &  \cos{\theta}
  \end{pmatrix}.
\]

\item
  Матрица растяжения \[
    B = 
    \begin{pmatrix}
   k & 0 \\
   0 & 1
    \end{pmatrix}.
  \]
\end{enumerate}

    \begin{center}
    \adjustimage{max size={0.9\linewidth}{0.9\paperheight}}{output_5_0.png}
    \end{center}
    { \hspace*{\fill} \\}
    
    Рассмотрим другую матрицу: \[
  C = 
  \begin{pmatrix}
    3 & 2 \\
    0 & 2
  \end{pmatrix}.
\]

На рисунке показано преобразование множества точек \(X\) (окружность) и,
в частности, двух векторов \(\mathbf{x_1}\) и \(\mathbf{x_2}\).
Начальные векторы \(X\) с левой стороны образуют окружность, матрица
преобразования изменяет эту окружность и превращает её в эллипс. Векторы
выборки \(\mathbf{x_1}\) и \(\mathbf{x_2}\) в окружности преобразуются в
\(\mathbf{y_1}\) и \(\mathbf{y_2}\) соответственно.

    \begin{center}
    \adjustimage{max size={0.9\linewidth}{0.9\paperheight}}{output_8_0.png}
    \end{center}
    { \hspace*{\fill} \\}
    
    Общее влияние преобразования \(C\) на вектор \(\mathbf{x}\) --- это
сочетание вращения и растяжения. Например, оно изменяет длину и
направление вектора \(\mathbf{x_1}\). Однако вектор \(\mathbf{x_2}\)
после преобразования изменяет только длину --- матрица \(C\) растягивает
вектор \(\mathbf{x_2}\). Таким образом, для вектора \(\mathbf{x_2}\)
эффект умножения на \(C\) подобен умножению на скалярное число
\(\lambda\): \[ C\mathbf{x_2} = \lambda \mathbf{x_2}. \]

    \begin{center}\rule{0.5\linewidth}{\linethickness}\end{center}

    \hypertarget{ux43eux441ux43dux43eux432ux43dux44bux435-ux43fux43eux43dux44fux442ux438ux44f}{%
\section{Основные
понятия}\label{ux43eux441ux43dux43eux432ux43dux44bux435-ux43fux43eux43dux44fux442ux438ux44f}}

При исследовании структуры линейного опреатора с матрицей \(A\) большую
роль играют векторы \(\mathbf{x}\), для которых
\[ A\mathbf{x} = \lambda\mathbf{x}. \]

Такие векторы называются \emph{собственными векторами}, а
соответствующие им числа \(\lambda\) --- \emph{собственными числами}
матрицы \(A\).

Для нахождения собственных чисел матрицы \(A\) используется
\emph{характеристическое уравнение} \[ \mathrm{det}(A - \lambda E). \]

\textbf{Некоторые свойства собственных векторов и собственных значений:}

\begin{itemize}
\tightlist
\item
  Сумма собственных значений равна следу матрицы:
  \(\sum\limits_{i+1}^{n} \lambda_i = \mathrm{tr}(A)\)
\item
  Произведение собственных значений равна определителю матрицы:
  \(\prod\limits_{i=1}^n \lambda_i = \mathrm{det}(A)\)
\item
  Если матрица \(A\) треугольная, то собственные значения совпадают с её
  диагональными элементами
\item
  Собственные значений матрицы \(A^k\) равны \(\lambda_i^k\), а
  собственные вектора матриц \(A^k\) и \(A\) совпадают
\item
  Собственные векторы матрицы \(A\) ортогональны тогда и только тогда,
  когда \(A^\top A = A A^\top\)
\end{itemize}

    Но почему для нас важны собственные векторы?\\
Как уже упоминалось ранее, собственный вектор превращает умножение на
матрицу в умножение на скаляр.

Собственные векторы, их количество и вид, зависят от матрицы \(A\).
Пусть матрица \(A\) размером \(n \times n\) имеет \(n\) независимых
собственных векторов (такая ситуация встречается довольно часто). В этом
случае собственные векторы образуют базис, по которому можно разложить
любой интересующий нас вектор:
\[ \mathbf{v} = c_1\mathbf{u_1} + \ldots + c_n\mathbf{u_n}. \]

И умножение любого вектора на матрицу \(A^k\) можно предстваить в виде
суммы
\[ A^k\mathbf{v} = c_1\lambda_1^k\mathbf{u_1} + \ldots + c_n\lambda_n^k\mathbf{u_n}. \]

    \begin{center}\rule{0.5\linewidth}{\linethickness}\end{center}

    \hypertarget{ux434ux438ux430ux433ux43eux43dux430ux43bux438ux437ux438ux440ux443ux435ux43cux43eux441ux442ux44c}{%
\section{Диагонализируемость}\label{ux434ux438ux430ux433ux43eux43dux430ux43bux438ux437ux438ux440ux443ux435ux43cux43eux441ux442ux44c}}

\hypertarget{ux43fux43eux434ux43eux431ux43dux44bux435-ux43cux430ux442ux440ux438ux446ux44b}{%
\subsection{Подобные
матрицы}\label{ux43fux43eux434ux43eux431ux43dux44bux435-ux43cux430ux442ux440ux438ux446ux44b}}

\textbf{Определение.} Матрицы \(A\) и \(B\) называются подобными, если
существует невырожденная матрица \(U\) такая, что \(B = U^{-1}AU\).
Матрица \(U\) в этом случае называется матрицей перехода к другому
базису.

\textbf{Предложение 1.} Подобные матрицы имеют одни и те же собственные
числа.

    \hypertarget{ux434ux438ux430ux433ux43eux43dux430ux43bux438ux437ux438ux440ux443ux435ux43cux44bux435-ux43cux430ux442ux440ux438ux446ux44b}{%
\subsection{Диагонализируемые
матрицы}\label{ux434ux438ux430ux433ux43eux43dux430ux43bux438ux437ux438ux440ux443ux435ux43cux44bux435-ux43cux430ux442ux440ux438ux446ux44b}}

Пусть матрица \(A\) размера \(n \times n\) имеет \(n\) линейно
независимых собственных векторов. Если взять эти векторы в качестве
столбцов матрицы \(U\), то \(U^{-1}AU\) будет диагональной матрицей
\(\Lambda\), у которой на диагонали стоят собственные значения матрица
\(A\).

\textbf{Теорема 1.} Матрица \(A\) диагонализируема тогда и только тогда,
когда существует базис из собственных векторов.

\textbf{Теорема 2.} Матрица \(A\) диагонализируема тогда и только тогда,
когда кратность корня характеристического уравнения совпадает с
размерностью собственного подпространства (алгебраическая кратность
собственного значения совпадает с его геометрической кратностью).

\textbf{Замечания:} 1. Если все собственные значения матрицы различны,
то матрица может быть приведена к диагональному виду. 1.
Диагонализирующая матрица \(U\) не единственна (особенно в случае кратных
собственных значений) 1. Равенство \(AU = U\Lambda\) выполняется тогда и
только тогда, когда столбцы матрицы \(U\) являются собственными
векторами 1. Не все матрицы диагонализируемы

    \hypertarget{ux43dux435ux434ux438ux430ux433ux43eux43dux430ux43bux438ux437ux438ux440ux443ux435ux43cux44bux435-ux43cux430ux442ux440ux438ux446ux44b}{%
\subsection{Недиагонализируемые
матрицы}\label{ux43dux435ux434ux438ux430ux433ux43eux43dux430ux43bux438ux437ux438ux440ux443ux435ux43cux44bux435-ux43cux430ux442ux440ux438ux446ux44b}}

В общем случае матрица поворота не является диагонализируемой над
вещественными числами, но все матрицы поворота диагонализируемы над
полем комплексных чисел.

Некоторые матрицы нельзя диагонализовать ни в \(\mathbb{R}\), ни в
\(\mathbb{C}\). Среди них можно указать \emph{ненулевые нильпотентные
матрицы}.

\textbf{Определение.} Нильпотентная матрица --- матрица \(N\), для
которой существует целое число \(n\) такое, что выполняется условие
\(N^n=0\).

\textbf{Примеры:}

\begin{enumerate}
\def\labelenumi{\arabic{enumi}.}
\item
  \[
  N_1 = 
  \begin{pmatrix}
    0 & 1 \\
    0 & 0
  \end{pmatrix},
  \quad
  N^2 = 0;
  \]
\item
  \[
  N_2 = 
  \begin{pmatrix}
    5  & -3 & 2 \\
    15 & -9 & 6 \\
    10 & -6 & 4 \\
  \end{pmatrix},
  \quad
  N^2 = 0;
  \]
\item
  \[
  A_1 = 
  \begin{pmatrix}
    5 & 1 \\
    0 & 5
  \end{pmatrix};
  \]
\item
  \[
  A_2 = 
  \begin{pmatrix}
    6 & -1 \\
    1 & 4
  \end{pmatrix}.
  \]
\end{enumerate}

    \begin{center}\rule{0.5\linewidth}{\linethickness}\end{center}

    \hypertarget{ux438ux43bux43bux44eux441ux442ux440ux430ux446ux438ux438-ux43dux430-ux43fux438ux442ux43eux43dux435}{%
\section{Иллюстрации на
Питоне}\label{ux438ux43bux43bux44eux441ux442ux440ux430ux446ux438ux438-ux43dux430-ux43fux438ux442ux43eux43dux435}}

Воспользуемся модулем \texttt{numpy.linalg} и найдём собственные числа и
собственные векторы матрицы \(C\). Далее немного порисуем.

Поработаем с матрицей \(C\): \[
  C = 
  \begin{pmatrix}
    3 & 2 \\
    0 & 2
  \end{pmatrix}.
\]

    \begin{center}
    \adjustimage{max size={0.9\linewidth}{0.9\paperheight}}{output_26_0.png}
    \end{center}
    { \hspace*{\fill} \\}
    
    Теперь рассмотрим действие матрицы \(S \Lambda S^{-1}\) пошагово.

    \begin{center}
    \adjustimage{max size={0.9\linewidth}{0.9\paperheight}}{output_29_0.png}
    \end{center}
    { \hspace*{\fill} \\}
    
    \hypertarget{ux441ux438ux43cux43cux435ux442ux440ux438ux447ux43dux44bux435-ux43cux430ux442ux440ux438ux446ux44b}{%
\section{Симметричные
матрицы}\label{ux441ux438ux43cux43cux435ux442ux440ux438ux447ux43dux44bux435-ux43cux430ux442ux440ux438ux446ux44b}}

Рассмотрим симметричную матрицу \(S = S^\top\).

\textbf{Свойства:}

\begin{enumerate}
\def\labelenumi{\arabic{enumi}.}
\tightlist
\item
  все собственные числа вещественны.
\item
  из собственных векторов всегда можно составить ортонормированный базис
\end{enumerate}

Симметричную матрицу можно привести к диагональному виду:
\[ S = Q \Lambda Q^\top, \] где \(Q\) --- диагональная матрица.

Следовательно, любая симметричная матрица может быть представлена в виде

\[ S = \sum\limits_{i=1}^n \lambda_i \mathbf{u}_i \mathbf{u}_i^\top. \]

Это разложение известно под названием \textbf{спектральное
разложение}.\\
Оно выражает матрицу \(S\) в виде комбинации одномерных проекций. Они
разбивают любой вектор \(\mathbf{v}\) на его компоненты
\(\mathbf{p} = \mathbf{u}_i \mathbf{u}_i^\top \mathbf{v}\) по
направлениям единичных собственных векторов. Действие оператора с
матрицей \(S\) сводится к растяжению этих проекций в \(\lambda_i\) раз:

\[ S\mathbf{v} = \sum\limits_{i=1}^n \lambda_i \mathbf{u}_i \mathbf{u}_i^\top \mathbf{v}. \]

    Рассмотрим симметричную матрицу: \[
  S = 
  \begin{pmatrix}
    3 & 1 \\
    1 & 2
  \end{pmatrix}.
\]

Найдём собственные значения и нарисуем собственные векторы.

Мы видим, что собственные векторы находятся вдоль главных осей эллипса.
Таким образом, матрица \(U\) преобразует начальную окружность,
растягивая её вдоль собственных векторов \(\mathbf{u_1}\) и
\(\mathbf{u_2}\) в \(\lambda_1\) и \(\lambda_2\) раз соотвественно.

    \begin{center}
    \adjustimage{max size={0.9\linewidth}{0.9\paperheight}}{output_33_0.png}
    \end{center}
    { \hspace*{\fill} \\}
    
    \begin{center}\rule{0.5\linewidth}{\linethickness}\end{center}

    \hypertarget{ux43fux43eux43bux44fux440ux43dux43eux435-ux440ux430ux437ux43bux43eux436ux435ux43dux438ux435}{%
\section{Полярное
разложение}\label{ux43fux43eux43bux44fux440ux43dux43eux435-ux440ux430ux437ux43bux43eux436ux435ux43dux438ux435}}

Каждое комплексное число \(z = x + iy\) можно представить в виде
\(z = r e^{i\theta}\). Вектор \(e^{i\theta}\), лежащий на единичной
окуржности, умножается на число \(r \ge 0\) («растягивается» в \(r\)
раз).

Между комплексными числами и матрицами можно провести аналогию: \(r\) и
\(e^{i\theta}\) --- это симметричная матрица \(S\) и ортогональная
\(Q\).

\textbf{Предложение.} Любая квадратная матрица может быть представлена в
виде

\[ A = QS, \]

где \(Q\) --- ортогональная, а \(S\) --- положительно полуопределённая
матрицы. Причём если \(A\) невырождена, то \(S\) --- строго положительно
определённая матрица. Такое разложение называется \emph{полярным
разложением} матрицы \(A\).

Таким образом, любое линейное преобразование \(A\) можно представить в
виде комбинации \emph{вращения} и \emph{растяжения к взаимно
перпендикулярным осям}.

    \begin{center}\rule{0.5\linewidth}{\linethickness}\end{center}

    \hypertarget{ux430ux43bux433ux43eux440ux438ux442ux43cux44b-ux43fux43eux438ux441ux43aux430-ux441ux43eux431ux441ux442ux432ux435ux43dux43dux44bux445-ux447ux438ux441ux435ux43b}{%
\section{Алгоритмы поиска собственных
чисел}\label{ux430ux43bux433ux43eux440ux438ux442ux43cux44b-ux43fux43eux438ux441ux43aux430-ux441ux43eux431ux441ux442ux432ux435ux43dux43dux44bux445-ux447ux438ux441ux435ux43b}}

Проблема собственных значений намного сложнее, чем рассматриваемая нами
ранее задача решения системы линейных уравнений. Все имеющиеся методы её
решения могут быть разделены на две большие группы: прямые методы,
основанные на решении характеристического уравнения, и итерационные
методы.

В прямых методах важным этапом является нахождение коэффициентов
характеристического многочлена, так как их вычисление требует
осуществления очень большого числа арифметических операций. Результат,
получаемый прямыми методами, является в принципе приближённым, так как
корни характеристического многочлена могут быть найдены только
приближённо.

К итерационным методам относятся метод вращений, степенной метод и
\(QR-алгоритм\). Остановимся на последнем.

    \hypertarget{qr-ux430ux43bux433ux43eux440ux438ux442ux43c}{%
\subsection{\texorpdfstring{\(QR\)-алгоритм}{QR-алгоритм}}\label{qr-ux430ux43bux433ux43eux440ux438ux442ux43c}}

Основой этого алгоритма является следующий процесс.

Найдём \(QR\)-разложение исходной матрицы. Пусть \(A = Q_1 R_1\).\\
Положим \(A_1 = R_1 Q_1\) и найдём для матрицы \(A_1\) её
\(QR\)-разложение \(A_1 = Q_2 R_2\). Матрицу \(A_2\) получим, переставив
сомножители \(Q_2\) и \(R_2\) и т.д.
\[ A_{k-1} - Q_k R_k, \quad A_k = R_k Q_k. \]

При этом \(A_k = Q_k^{-1} A_{k-1} Q_k\). Поэтому все
характеристические числа матриц \(A_k\) совпадают.

Можно доказать, что те элементы \(A_k\), которые лежат ниже диагональных
клеток, стремятся к нулю, а элементы этих клеток и вышележащие элеенты
ограничены.

    \begin{center}\rule{0.5\linewidth}{\linethickness}\end{center}

    \hypertarget{ux43bux438ux442ux435ux440ux430ux442ux443ux440ux430}{%
\section{Литература}\label{ux43bux438ux442ux435ux440ux430ux442ux443ux440ux430}}

\begin{enumerate}
\def\labelenumi{\arabic{enumi}.}
\tightlist
\item
  \emph{Гантмахер Ф.Р.} Теория матриц. --- М.: Наука, 1967. --- 576 с.
\item
  \emph{Стренг Г.} Линейная алгебра и её применения. --- М.: Мир, 1980.
  --- 454 с.
\item
  \emph{Strang G.} Linear algebra and learning from data. ---
  Wellesley-Cambridge Press, 2019. --- 432~p.
\item
  \href{https://towardsdatascience.com/understanding-singular-value-decomposition-and-its-application-in-data-science-388a54be95d}{Материалы}
  автора \href{https://medium.com/@reza.bagheri79}{Reza Bagheri}.
\item
  \emph{Беклемишев Д.В.} Дополнительные главы линейной алгебры. --- М.:
  Наука, 1983. --- 336~с.
\end{enumerate}


    % Add a bibliography block to the postdoc
    
    
    
\end{document}
