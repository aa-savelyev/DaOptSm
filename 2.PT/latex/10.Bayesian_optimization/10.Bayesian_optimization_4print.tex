\documentclass[11pt,a4paper]{article}

    \usepackage[breakable]{tcolorbox}
    \usepackage{parskip} % Stop auto-indenting (to mimic markdown behaviour)
    
    \usepackage{iftex}
    \ifPDFTeX
      \usepackage[T2A]{fontenc}
      \usepackage{mathpazo}
      \usepackage[russian,english]{babel}
    \else
      \usepackage{fontspec}
      \usepackage{polyglossia}
      \setmainlanguage[babelshorthands=true]{russian}    % Язык по-умолчанию русский с поддержкой приятных команд пакета babel
      \setotherlanguage{english}                         % Дополнительный язык = английский (в американской вариации по-умолчанию)

      \defaultfontfeatures{Ligatures=TeX}
      \setmainfont[BoldFont={STIX Two Text SemiBold}]%
      {STIX Two Text}                                    % Шрифт с засечками
      \newfontfamily\cyrillicfont[BoldFont={STIX Two Text SemiBold}]%
      {STIX Two Text}                                    % Шрифт с засечками для кириллицы
      \setsansfont{Fira Sans}                            % Шрифт без засечек
      \newfontfamily\cyrillicfontsf{Fira Sans}           % Шрифт без засечек для кириллицы
      \setmonofont[Scale=0.87,BoldFont={Fira Mono Medium},ItalicFont=[FiraMono-Oblique]]%
      {Fira Mono}%                                       % Моноширинный шрифт
      \newfontfamily\cyrillicfonttt[Scale=0.87,BoldFont={Fira Mono Medium},ItalicFont=[FiraMono-Oblique]]%
      {Fira Mono}                                        % Моноширинный шрифт для кириллицы

      %%% Математические пакеты %%%
      \usepackage{amsthm,amsmath,amscd}   % Математические дополнения от AMS
      \usepackage{amsfonts,amssymb}       % Математические дополнения от AMS
      \usepackage{mathtools}              % Добавляет окружение multlined
      \usepackage{unicode-math}           % Для шрифта STIX Two Math
      \setmathfont{STIX Two Math}         % Математический шрифт
    \fi

    % Basic figure setup, for now with no caption control since it's done
    % automatically by Pandoc (which extracts ![](path) syntax from Markdown).
    \usepackage{graphicx}
    % Maintain compatibility with old templates. Remove in nbconvert 6.0
    \let\Oldincludegraphics\includegraphics
    % Ensure that by default, figures have no caption (until we provide a
    % proper Figure object with a Caption API and a way to capture that
    % in the conversion process - todo).
    \usepackage{caption}
    \DeclareCaptionFormat{nocaption}{}
    \captionsetup{format=nocaption,aboveskip=0pt,belowskip=0pt}

    \usepackage{float}
    \floatplacement{figure}{H} % forces figures to be placed at the correct location
    \usepackage{xcolor} % Allow colors to be defined
    \usepackage{enumerate} % Needed for markdown enumerations to work
    \usepackage{geometry} % Used to adjust the document margins
    \usepackage{amsmath} % Equations
    \usepackage{amssymb} % Equations
    \usepackage{textcomp} % defines textquotesingle
    % Hack from http://tex.stackexchange.com/a/47451/13684:
    \AtBeginDocument{%
        \def\PYZsq{\textquotesingle}% Upright quotes in Pygmentized code
    }
    \usepackage{upquote} % Upright quotes for verbatim code
    \usepackage{eurosym} % defines \euro
    \usepackage[mathletters]{ucs} % Extended unicode (utf-8) support
    \usepackage{fancyvrb} % verbatim replacement that allows latex
    \usepackage{grffile} % extends the file name processing of package graphics 
                         % to support a larger range
    \makeatletter % fix for old versions of grffile with XeLaTeX
    \@ifpackagelater{grffile}{2019/11/01}
    {
      % Do nothing on new versions
    }
    {
      \def\Gread@@xetex#1{%
        \IfFileExists{"\Gin@base".bb}%
        {\Gread@eps{\Gin@base.bb}}%
        {\Gread@@xetex@aux#1}%
      }
    }
    \makeatother
    \usepackage[Export]{adjustbox} % Used to constrain images to a maximum size
    \adjustboxset{max size={0.9\linewidth}{0.9\paperheight}}

    % The hyperref package gives us a pdf with properly built
    % internal navigation ('pdf bookmarks' for the table of contents,
    % internal cross-reference links, web links for URLs, etc.)
    \usepackage{hyperref}
    % The default LaTeX title has an obnoxious amount of whitespace. By default,
    % titling removes some of it. It also provides customization options.
    \usepackage{titling}
    \usepackage{longtable} % longtable support required by pandoc >1.10
    \usepackage{booktabs}  % table support for pandoc > 1.12.2
    \usepackage[inline]{enumitem} % IRkernel/repr support (it uses the enumerate* environment)
    \usepackage[normalem]{ulem} % ulem is needed to support strikethroughs (\sout)
                                % normalem makes italics be italics, not underlines
    \usepackage{mathrsfs}
    

    
    % Colors for the hyperref package
    \definecolor{urlcolor}{rgb}{0,.145,.698}
    \definecolor{linkcolor}{rgb}{.71,0.21,0.01}
    \definecolor{citecolor}{rgb}{.12,.54,.11}

    % ANSI colors
    \definecolor{ansi-black}{HTML}{3E424D}
    \definecolor{ansi-black-intense}{HTML}{282C36}
    \definecolor{ansi-red}{HTML}{E75C58}
    \definecolor{ansi-red-intense}{HTML}{B22B31}
    \definecolor{ansi-green}{HTML}{00A250}
    \definecolor{ansi-green-intense}{HTML}{007427}
    \definecolor{ansi-yellow}{HTML}{DDB62B}
    \definecolor{ansi-yellow-intense}{HTML}{B27D12}
    \definecolor{ansi-blue}{HTML}{208FFB}
    \definecolor{ansi-blue-intense}{HTML}{0065CA}
    \definecolor{ansi-magenta}{HTML}{D160C4}
    \definecolor{ansi-magenta-intense}{HTML}{A03196}
    \definecolor{ansi-cyan}{HTML}{60C6C8}
    \definecolor{ansi-cyan-intense}{HTML}{258F8F}
    \definecolor{ansi-white}{HTML}{C5C1B4}
    \definecolor{ansi-white-intense}{HTML}{A1A6B2}
    \definecolor{ansi-default-inverse-fg}{HTML}{FFFFFF}
    \definecolor{ansi-default-inverse-bg}{HTML}{000000}

    % common color for the border for error outputs.
    \definecolor{outerrorbackground}{HTML}{FFDFDF}

    % commands and environments needed by pandoc snippets
    % extracted from the output of `pandoc -s`
    \providecommand{\tightlist}{%
      \setlength{\itemsep}{0pt}\setlength{\parskip}{0pt}}
    \DefineVerbatimEnvironment{Highlighting}{Verbatim}{commandchars=\\\{\}}
    % Add ',fontsize=\small' for more characters per line
    \newenvironment{Shaded}{}{}
    \newcommand{\KeywordTok}[1]{\textcolor[rgb]{0.00,0.44,0.13}{\textbf{{#1}}}}
    \newcommand{\DataTypeTok}[1]{\textcolor[rgb]{0.56,0.13,0.00}{{#1}}}
    \newcommand{\DecValTok}[1]{\textcolor[rgb]{0.25,0.63,0.44}{{#1}}}
    \newcommand{\BaseNTok}[1]{\textcolor[rgb]{0.25,0.63,0.44}{{#1}}}
    \newcommand{\FloatTok}[1]{\textcolor[rgb]{0.25,0.63,0.44}{{#1}}}
    \newcommand{\CharTok}[1]{\textcolor[rgb]{0.25,0.44,0.63}{{#1}}}
    \newcommand{\StringTok}[1]{\textcolor[rgb]{0.25,0.44,0.63}{{#1}}}
    \newcommand{\CommentTok}[1]{\textcolor[rgb]{0.38,0.63,0.69}{\textit{{#1}}}}
    \newcommand{\OtherTok}[1]{\textcolor[rgb]{0.00,0.44,0.13}{{#1}}}
    \newcommand{\AlertTok}[1]{\textcolor[rgb]{1.00,0.00,0.00}{\textbf{{#1}}}}
    \newcommand{\FunctionTok}[1]{\textcolor[rgb]{0.02,0.16,0.49}{{#1}}}
    \newcommand{\RegionMarkerTok}[1]{{#1}}
    \newcommand{\ErrorTok}[1]{\textcolor[rgb]{1.00,0.00,0.00}{\textbf{{#1}}}}
    \newcommand{\NormalTok}[1]{{#1}}
    
    % Additional commands for more recent versions of Pandoc
    \newcommand{\ConstantTok}[1]{\textcolor[rgb]{0.53,0.00,0.00}{{#1}}}
    \newcommand{\SpecialCharTok}[1]{\textcolor[rgb]{0.25,0.44,0.63}{{#1}}}
    \newcommand{\VerbatimStringTok}[1]{\textcolor[rgb]{0.25,0.44,0.63}{{#1}}}
    \newcommand{\SpecialStringTok}[1]{\textcolor[rgb]{0.73,0.40,0.53}{{#1}}}
    \newcommand{\ImportTok}[1]{{#1}}
    \newcommand{\DocumentationTok}[1]{\textcolor[rgb]{0.73,0.13,0.13}{\textit{{#1}}}}
    \newcommand{\AnnotationTok}[1]{\textcolor[rgb]{0.38,0.63,0.69}{\textbf{\textit{{#1}}}}}
    \newcommand{\CommentVarTok}[1]{\textcolor[rgb]{0.38,0.63,0.69}{\textbf{\textit{{#1}}}}}
    \newcommand{\VariableTok}[1]{\textcolor[rgb]{0.10,0.09,0.49}{{#1}}}
    \newcommand{\ControlFlowTok}[1]{\textcolor[rgb]{0.00,0.44,0.13}{\textbf{{#1}}}}
    \newcommand{\OperatorTok}[1]{\textcolor[rgb]{0.40,0.40,0.40}{{#1}}}
    \newcommand{\BuiltInTok}[1]{{#1}}
    \newcommand{\ExtensionTok}[1]{{#1}}
    \newcommand{\PreprocessorTok}[1]{\textcolor[rgb]{0.74,0.48,0.00}{{#1}}}
    \newcommand{\AttributeTok}[1]{\textcolor[rgb]{0.49,0.56,0.16}{{#1}}}
    \newcommand{\InformationTok}[1]{\textcolor[rgb]{0.38,0.63,0.69}{\textbf{\textit{{#1}}}}}
    \newcommand{\WarningTok}[1]{\textcolor[rgb]{0.38,0.63,0.69}{\textbf{\textit{{#1}}}}}
    
    
    % Define a nice break command that doesn't care if a line doesn't already
    % exist.
    \def\br{\hspace*{\fill} \\* }
    % Math Jax compatibility definitions
    \def\gt{>}
    \def\lt{<}
    \let\Oldtex\TeX
    \let\Oldlatex\LaTeX
    \renewcommand{\TeX}{\textrm{\Oldtex}}
    \renewcommand{\LaTeX}{\textrm{\Oldlatex}}
    % Document parameters
    % Document title
    \title{
      {\Large Лекция 10} \\
      Байесовская оптимизация
    }
    \date{19 апреля 2023\,г.}
    \date{}
    
    
    
% Pygments definitions
\makeatletter
\def\PY@reset{\let\PY@it=\relax \let\PY@bf=\relax%
    \let\PY@ul=\relax \let\PY@tc=\relax%
    \let\PY@bc=\relax \let\PY@ff=\relax}
\def\PY@tok#1{\csname PY@tok@#1\endcsname}
\def\PY@toks#1+{\ifx\relax#1\empty\else%
    \PY@tok{#1}\expandafter\PY@toks\fi}
\def\PY@do#1{\PY@bc{\PY@tc{\PY@ul{%
    \PY@it{\PY@bf{\PY@ff{#1}}}}}}}
\def\PY#1#2{\PY@reset\PY@toks#1+\relax+\PY@do{#2}}

\@namedef{PY@tok@w}{\def\PY@tc##1{\textcolor[rgb]{0.73,0.73,0.73}{##1}}}
\@namedef{PY@tok@c}{\let\PY@it=\textit\def\PY@tc##1{\textcolor[rgb]{0.24,0.48,0.48}{##1}}}
\@namedef{PY@tok@cp}{\def\PY@tc##1{\textcolor[rgb]{0.61,0.40,0.00}{##1}}}
\@namedef{PY@tok@k}{\let\PY@bf=\textbf\def\PY@tc##1{\textcolor[rgb]{0.00,0.50,0.00}{##1}}}
\@namedef{PY@tok@kp}{\def\PY@tc##1{\textcolor[rgb]{0.00,0.50,0.00}{##1}}}
\@namedef{PY@tok@kt}{\def\PY@tc##1{\textcolor[rgb]{0.69,0.00,0.25}{##1}}}
\@namedef{PY@tok@o}{\def\PY@tc##1{\textcolor[rgb]{0.40,0.40,0.40}{##1}}}
\@namedef{PY@tok@ow}{\let\PY@bf=\textbf\def\PY@tc##1{\textcolor[rgb]{0.67,0.13,1.00}{##1}}}
\@namedef{PY@tok@nb}{\def\PY@tc##1{\textcolor[rgb]{0.00,0.50,0.00}{##1}}}
\@namedef{PY@tok@nf}{\def\PY@tc##1{\textcolor[rgb]{0.00,0.00,1.00}{##1}}}
\@namedef{PY@tok@nc}{\let\PY@bf=\textbf\def\PY@tc##1{\textcolor[rgb]{0.00,0.00,1.00}{##1}}}
\@namedef{PY@tok@nn}{\let\PY@bf=\textbf\def\PY@tc##1{\textcolor[rgb]{0.00,0.00,1.00}{##1}}}
\@namedef{PY@tok@ne}{\let\PY@bf=\textbf\def\PY@tc##1{\textcolor[rgb]{0.80,0.25,0.22}{##1}}}
\@namedef{PY@tok@nv}{\def\PY@tc##1{\textcolor[rgb]{0.10,0.09,0.49}{##1}}}
\@namedef{PY@tok@no}{\def\PY@tc##1{\textcolor[rgb]{0.53,0.00,0.00}{##1}}}
\@namedef{PY@tok@nl}{\def\PY@tc##1{\textcolor[rgb]{0.46,0.46,0.00}{##1}}}
\@namedef{PY@tok@ni}{\let\PY@bf=\textbf\def\PY@tc##1{\textcolor[rgb]{0.44,0.44,0.44}{##1}}}
\@namedef{PY@tok@na}{\def\PY@tc##1{\textcolor[rgb]{0.41,0.47,0.13}{##1}}}
\@namedef{PY@tok@nt}{\let\PY@bf=\textbf\def\PY@tc##1{\textcolor[rgb]{0.00,0.50,0.00}{##1}}}
\@namedef{PY@tok@nd}{\def\PY@tc##1{\textcolor[rgb]{0.67,0.13,1.00}{##1}}}
\@namedef{PY@tok@s}{\def\PY@tc##1{\textcolor[rgb]{0.73,0.13,0.13}{##1}}}
\@namedef{PY@tok@sd}{\let\PY@it=\textit\def\PY@tc##1{\textcolor[rgb]{0.73,0.13,0.13}{##1}}}
\@namedef{PY@tok@si}{\let\PY@bf=\textbf\def\PY@tc##1{\textcolor[rgb]{0.64,0.35,0.47}{##1}}}
\@namedef{PY@tok@se}{\let\PY@bf=\textbf\def\PY@tc##1{\textcolor[rgb]{0.67,0.36,0.12}{##1}}}
\@namedef{PY@tok@sr}{\def\PY@tc##1{\textcolor[rgb]{0.64,0.35,0.47}{##1}}}
\@namedef{PY@tok@ss}{\def\PY@tc##1{\textcolor[rgb]{0.10,0.09,0.49}{##1}}}
\@namedef{PY@tok@sx}{\def\PY@tc##1{\textcolor[rgb]{0.00,0.50,0.00}{##1}}}
\@namedef{PY@tok@m}{\def\PY@tc##1{\textcolor[rgb]{0.40,0.40,0.40}{##1}}}
\@namedef{PY@tok@gh}{\let\PY@bf=\textbf\def\PY@tc##1{\textcolor[rgb]{0.00,0.00,0.50}{##1}}}
\@namedef{PY@tok@gu}{\let\PY@bf=\textbf\def\PY@tc##1{\textcolor[rgb]{0.50,0.00,0.50}{##1}}}
\@namedef{PY@tok@gd}{\def\PY@tc##1{\textcolor[rgb]{0.63,0.00,0.00}{##1}}}
\@namedef{PY@tok@gi}{\def\PY@tc##1{\textcolor[rgb]{0.00,0.52,0.00}{##1}}}
\@namedef{PY@tok@gr}{\def\PY@tc##1{\textcolor[rgb]{0.89,0.00,0.00}{##1}}}
\@namedef{PY@tok@ge}{\let\PY@it=\textit}
\@namedef{PY@tok@gs}{\let\PY@bf=\textbf}
\@namedef{PY@tok@gp}{\let\PY@bf=\textbf\def\PY@tc##1{\textcolor[rgb]{0.00,0.00,0.50}{##1}}}
\@namedef{PY@tok@go}{\def\PY@tc##1{\textcolor[rgb]{0.44,0.44,0.44}{##1}}}
\@namedef{PY@tok@gt}{\def\PY@tc##1{\textcolor[rgb]{0.00,0.27,0.87}{##1}}}
\@namedef{PY@tok@err}{\def\PY@bc##1{{\setlength{\fboxsep}{\string -\fboxrule}\fcolorbox[rgb]{1.00,0.00,0.00}{1,1,1}{\strut ##1}}}}
\@namedef{PY@tok@kc}{\let\PY@bf=\textbf\def\PY@tc##1{\textcolor[rgb]{0.00,0.50,0.00}{##1}}}
\@namedef{PY@tok@kd}{\let\PY@bf=\textbf\def\PY@tc##1{\textcolor[rgb]{0.00,0.50,0.00}{##1}}}
\@namedef{PY@tok@kn}{\let\PY@bf=\textbf\def\PY@tc##1{\textcolor[rgb]{0.00,0.50,0.00}{##1}}}
\@namedef{PY@tok@kr}{\let\PY@bf=\textbf\def\PY@tc##1{\textcolor[rgb]{0.00,0.50,0.00}{##1}}}
\@namedef{PY@tok@bp}{\def\PY@tc##1{\textcolor[rgb]{0.00,0.50,0.00}{##1}}}
\@namedef{PY@tok@fm}{\def\PY@tc##1{\textcolor[rgb]{0.00,0.00,1.00}{##1}}}
\@namedef{PY@tok@vc}{\def\PY@tc##1{\textcolor[rgb]{0.10,0.09,0.49}{##1}}}
\@namedef{PY@tok@vg}{\def\PY@tc##1{\textcolor[rgb]{0.10,0.09,0.49}{##1}}}
\@namedef{PY@tok@vi}{\def\PY@tc##1{\textcolor[rgb]{0.10,0.09,0.49}{##1}}}
\@namedef{PY@tok@vm}{\def\PY@tc##1{\textcolor[rgb]{0.10,0.09,0.49}{##1}}}
\@namedef{PY@tok@sa}{\def\PY@tc##1{\textcolor[rgb]{0.73,0.13,0.13}{##1}}}
\@namedef{PY@tok@sb}{\def\PY@tc##1{\textcolor[rgb]{0.73,0.13,0.13}{##1}}}
\@namedef{PY@tok@sc}{\def\PY@tc##1{\textcolor[rgb]{0.73,0.13,0.13}{##1}}}
\@namedef{PY@tok@dl}{\def\PY@tc##1{\textcolor[rgb]{0.73,0.13,0.13}{##1}}}
\@namedef{PY@tok@s2}{\def\PY@tc##1{\textcolor[rgb]{0.73,0.13,0.13}{##1}}}
\@namedef{PY@tok@sh}{\def\PY@tc##1{\textcolor[rgb]{0.73,0.13,0.13}{##1}}}
\@namedef{PY@tok@s1}{\def\PY@tc##1{\textcolor[rgb]{0.73,0.13,0.13}{##1}}}
\@namedef{PY@tok@mb}{\def\PY@tc##1{\textcolor[rgb]{0.40,0.40,0.40}{##1}}}
\@namedef{PY@tok@mf}{\def\PY@tc##1{\textcolor[rgb]{0.40,0.40,0.40}{##1}}}
\@namedef{PY@tok@mh}{\def\PY@tc##1{\textcolor[rgb]{0.40,0.40,0.40}{##1}}}
\@namedef{PY@tok@mi}{\def\PY@tc##1{\textcolor[rgb]{0.40,0.40,0.40}{##1}}}
\@namedef{PY@tok@il}{\def\PY@tc##1{\textcolor[rgb]{0.40,0.40,0.40}{##1}}}
\@namedef{PY@tok@mo}{\def\PY@tc##1{\textcolor[rgb]{0.40,0.40,0.40}{##1}}}
\@namedef{PY@tok@ch}{\let\PY@it=\textit\def\PY@tc##1{\textcolor[rgb]{0.24,0.48,0.48}{##1}}}
\@namedef{PY@tok@cm}{\let\PY@it=\textit\def\PY@tc##1{\textcolor[rgb]{0.24,0.48,0.48}{##1}}}
\@namedef{PY@tok@cpf}{\let\PY@it=\textit\def\PY@tc##1{\textcolor[rgb]{0.24,0.48,0.48}{##1}}}
\@namedef{PY@tok@c1}{\let\PY@it=\textit\def\PY@tc##1{\textcolor[rgb]{0.24,0.48,0.48}{##1}}}
\@namedef{PY@tok@cs}{\let\PY@it=\textit\def\PY@tc##1{\textcolor[rgb]{0.24,0.48,0.48}{##1}}}

\def\PYZbs{\char`\\}
\def\PYZus{\char`\_}
\def\PYZob{\char`\{}
\def\PYZcb{\char`\}}
\def\PYZca{\char`\^}
\def\PYZam{\char`\&}
\def\PYZlt{\char`\<}
\def\PYZgt{\char`\>}
\def\PYZsh{\char`\#}
\def\PYZpc{\char`\%}
\def\PYZdl{\char`\$}
\def\PYZhy{\char`\-}
\def\PYZsq{\char`\'}
\def\PYZdq{\char`\"}
\def\PYZti{\char`\~}
% for compatibility with earlier versions
\def\PYZat{@}
\def\PYZlb{[}
\def\PYZrb{]}
\makeatother


    % For linebreaks inside Verbatim environment from package fancyvrb. 
    \makeatletter
        \newbox\Wrappedcontinuationbox 
        \newbox\Wrappedvisiblespacebox 
        \newcommand*\Wrappedvisiblespace {\textcolor{red}{\textvisiblespace}} 
        \newcommand*\Wrappedcontinuationsymbol {\textcolor{red}{\llap{\tiny$\m@th\hookrightarrow$}}} 
        \newcommand*\Wrappedcontinuationindent {3ex } 
        \newcommand*\Wrappedafterbreak {\kern\Wrappedcontinuationindent\copy\Wrappedcontinuationbox} 
        % Take advantage of the already applied Pygments mark-up to insert 
        % potential linebreaks for TeX processing. 
        %        {, <, #, %, $, ' and ": go to next line. 
        %        _, }, ^, &, >, - and ~: stay at end of broken line. 
        % Use of \textquotesingle for straight quote. 
        \newcommand*\Wrappedbreaksatspecials {% 
            \def\PYGZus{\discretionary{\char`\_}{\Wrappedafterbreak}{\char`\_}}% 
            \def\PYGZob{\discretionary{}{\Wrappedafterbreak\char`\{}{\char`\{}}% 
            \def\PYGZcb{\discretionary{\char`\}}{\Wrappedafterbreak}{\char`\}}}% 
            \def\PYGZca{\discretionary{\char`\^}{\Wrappedafterbreak}{\char`\^}}% 
            \def\PYGZam{\discretionary{\char`\&}{\Wrappedafterbreak}{\char`\&}}% 
            \def\PYGZlt{\discretionary{}{\Wrappedafterbreak\char`\<}{\char`\<}}% 
            \def\PYGZgt{\discretionary{\char`\>}{\Wrappedafterbreak}{\char`\>}}% 
            \def\PYGZsh{\discretionary{}{\Wrappedafterbreak\char`\#}{\char`\#}}% 
            \def\PYGZpc{\discretionary{}{\Wrappedafterbreak\char`\%}{\char`\%}}% 
            \def\PYGZdl{\discretionary{}{\Wrappedafterbreak\char`\$}{\char`\$}}% 
            \def\PYGZhy{\discretionary{\char`\-}{\Wrappedafterbreak}{\char`\-}}% 
            \def\PYGZsq{\discretionary{}{\Wrappedafterbreak\textquotesingle}{\textquotesingle}}% 
            \def\PYGZdq{\discretionary{}{\Wrappedafterbreak\char`\"}{\char`\"}}% 
            \def\PYGZti{\discretionary{\char`\~}{\Wrappedafterbreak}{\char`\~}}% 
        } 
        % Some characters . , ; ? ! / are not pygmentized. 
        % This macro makes them "active" and they will insert potential linebreaks 
        \newcommand*\Wrappedbreaksatpunct {% 
            \lccode`\~`\.\lowercase{\def~}{\discretionary{\hbox{\char`\.}}{\Wrappedafterbreak}{\hbox{\char`\.}}}% 
            \lccode`\~`\,\lowercase{\def~}{\discretionary{\hbox{\char`\,}}{\Wrappedafterbreak}{\hbox{\char`\,}}}% 
            \lccode`\~`\;\lowercase{\def~}{\discretionary{\hbox{\char`\;}}{\Wrappedafterbreak}{\hbox{\char`\;}}}% 
            \lccode`\~`\:\lowercase{\def~}{\discretionary{\hbox{\char`\:}}{\Wrappedafterbreak}{\hbox{\char`\:}}}% 
            \lccode`\~`\?\lowercase{\def~}{\discretionary{\hbox{\char`\?}}{\Wrappedafterbreak}{\hbox{\char`\?}}}% 
            \lccode`\~`\!\lowercase{\def~}{\discretionary{\hbox{\char`\!}}{\Wrappedafterbreak}{\hbox{\char`\!}}}% 
            \lccode`\~`\/\lowercase{\def~}{\discretionary{\hbox{\char`\/}}{\Wrappedafterbreak}{\hbox{\char`\/}}}% 
            \catcode`\.\active
            \catcode`\,\active 
            \catcode`\;\active
            \catcode`\:\active
            \catcode`\?\active
            \catcode`\!\active
            \catcode`\/\active 
            \lccode`\~`\~ 	
        }
    \makeatother

    \let\OriginalVerbatim=\Verbatim
    \makeatletter
    \renewcommand{\Verbatim}[1][1]{%
        %\parskip\z@skip
        \sbox\Wrappedcontinuationbox {\Wrappedcontinuationsymbol}%
        \sbox\Wrappedvisiblespacebox {\FV@SetupFont\Wrappedvisiblespace}%
        \def\FancyVerbFormatLine ##1{\hsize\linewidth
            \vtop{\raggedright\hyphenpenalty\z@\exhyphenpenalty\z@
                \doublehyphendemerits\z@\finalhyphendemerits\z@
                \strut ##1\strut}%
        }%
        % If the linebreak is at a space, the latter will be displayed as visible
        % space at end of first line, and a continuation symbol starts next line.
        % Stretch/shrink are however usually zero for typewriter font.
        \def\FV@Space {%
            \nobreak\hskip\z@ plus\fontdimen3\font minus\fontdimen4\font
            \discretionary{\copy\Wrappedvisiblespacebox}{\Wrappedafterbreak}
            {\kern\fontdimen2\font}%
        }%
        
        % Allow breaks at special characters using \PYG... macros.
        \Wrappedbreaksatspecials
        % Breaks at punctuation characters . , ; ? ! and / need catcode=\active 	
        \OriginalVerbatim[#1,codes*=\Wrappedbreaksatpunct]%
    }
    \makeatother

    % Exact colors from NB
    \definecolor{incolor}{HTML}{303F9F}
    \definecolor{outcolor}{HTML}{D84315}
    \definecolor{cellborder}{HTML}{CFCFCF}
    \definecolor{cellbackground}{HTML}{F7F7F7}
    
    % prompt
    \makeatletter
    \newcommand{\boxspacing}{\kern\kvtcb@left@rule\kern\kvtcb@boxsep}
    \makeatother
    \newcommand{\prompt}[4]{
        {\ttfamily\llap{{\color{#2}[#3]:\hspace{3pt}#4}}\vspace{-\baselineskip}}
    }
    

    
    % Prevent overflowing lines due to hard-to-break entities
    \sloppy 
    % Setup hyperref package
    \hypersetup{
      breaklinks=true,  % so long urls are correctly broken across lines
      colorlinks=true,
      urlcolor=urlcolor,
      linkcolor=linkcolor,
      citecolor=citecolor,
      }
    % Slightly bigger margins than the latex defaults
    
    \geometry{verbose,tmargin=1in,bmargin=1in,lmargin=1in,rmargin=1in}
    
    

\begin{document}

  \maketitle
  \thispagestyle{empty}
  \tableofcontents

%\let\thefootnote\relax\footnote{
%  \textit{День 27 апреля в истории:
%    \begin{itemize}[topsep=2pt,itemsep=1pt]
%      \item 1521 г. --- на Филиппинах аборигены убили португальского мореплавателя Фернана Магеллана;
%      \item 1773 г. --- английский парламент принял <<чайный акт>>, разрешавший находившейся на грани банкротства Ост-Индской компании ввезти в североамериканские колонии фактически беспошлинно полмиллиона фунтов чая. Это событие стало толчком к началу Американской революции;
%      \item 1959 г. --- СССР и Египет заключили договор о строительстве Асуанской плотины;
%      \item 1978 г. --- государственный переворот (Апрельская революция) в Афганистане. Начало гражданской войны, продолжающейся до сих пор;
%      \item 1986 г. --- эвакуация населения города Припять из-за Чернобыльской аварии. За 3 часа из города были эвакуированы все 40 тысяч человек;
%      \item 2005 г. --- первый полёт совершил авиалайнер Airbus А380.
%    \end{itemize}
%  }
%}

  \newpage


%    \begin{tcolorbox}[breakable, size=fbox, boxrule=1pt, pad at break*=1mm,colback=cellbackground, colframe=cellborder]
%\prompt{In}{incolor}{1}{\boxspacing}
%\begin{Verbatim}[commandchars=\\\{\}]
%\PY{c+c1}{\PYZsh{} Imports}
%\PY{k+kn}{import} \PY{n+nn}{numpy} \PY{k}{as} \PY{n+nn}{np}
%\PY{k+kn}{from} \PY{n+nn}{scipy}\PY{n+nn}{.}\PY{n+nn}{stats} \PY{k+kn}{import} \PY{n}{norm}
%\PY{k+kn}{from} \PY{n+nn}{scipy}\PY{n+nn}{.}\PY{n+nn}{optimize} \PY{k+kn}{import} \PY{n}{minimize}
%
%\PY{k+kn}{import} \PY{n+nn}{sys}
%\PY{n}{sys}\PY{o}{.}\PY{n}{path}\PY{o}{.}\PY{n}{append}\PY{p}{(}\PY{l+s+s1}{\PYZsq{}}\PY{l+s+s1}{./scripts}\PY{l+s+s1}{\PYZsq{}}\PY{p}{)}
%\PY{k+kn}{import} \PY{n+nn}{GP\PYZus{}kernels}
%\PY{k+kn}{from} \PY{n+nn}{GP\PYZus{}utils} \PY{k+kn}{import} \PY{n}{plot\PYZus{}GP}\PY{p}{,} \PY{n}{GP\PYZus{}predictor}
%\end{Verbatim}
%\end{tcolorbox}
%
%    \begin{tcolorbox}[breakable, size=fbox, boxrule=1pt, pad at break*=1mm,colback=cellbackground, colframe=cellborder]
%\prompt{In}{incolor}{2}{\boxspacing}
%\begin{Verbatim}[commandchars=\\\{\}]
%\PY{c+c1}{\PYZsh{} Styles, fonts}
%\PY{k+kn}{import} \PY{n+nn}{matplotlib}
%\PY{n}{matplotlib}\PY{o}{.}\PY{n}{rcParams}\PY{p}{[}\PY{l+s+s1}{\PYZsq{}}\PY{l+s+s1}{font.size}\PY{l+s+s1}{\PYZsq{}}\PY{p}{]} \PY{o}{=} \PY{l+m+mi}{12}
%\PY{n}{matplotlib}\PY{o}{.}\PY{n}{rcParams}\PY{p}{[}\PY{l+s+s1}{\PYZsq{}}\PY{l+s+s1}{lines.markeredgewidth}\PY{l+s+s1}{\PYZsq{}}\PY{p}{]} \PY{o}{=} \PY{l+m+mf}{1.5}
%\PY{k+kn}{import} \PY{n+nn}{matplotlib}\PY{n+nn}{.}\PY{n+nn}{pyplot} \PY{k}{as} \PY{n+nn}{plt}
%\PY{k+kn}{from} \PY{n+nn}{matplotlib} \PY{k+kn}{import} \PY{n}{cm} \PY{c+c1}{\PYZsh{} Colormaps}
%
%\PY{k+kn}{import} \PY{n+nn}{seaborn}
%\PY{n}{seaborn}\PY{o}{.}\PY{n}{set\PYZus{}style}\PY{p}{(}\PY{l+s+s1}{\PYZsq{}}\PY{l+s+s1}{whitegrid}\PY{l+s+s1}{\PYZsq{}}\PY{p}{)}
%
%\PY{k+kn}{from} \PY{n+nn}{IPython}\PY{n+nn}{.}\PY{n+nn}{display} \PY{k+kn}{import} \PY{n}{Image}
%\PY{n}{im\PYZus{}width} \PY{o}{=} \PY{l+m+mi}{1000}
%\end{Verbatim}
%\end{tcolorbox}
%
%    \begin{tcolorbox}[breakable, size=fbox, boxrule=1pt, pad at break*=1mm,colback=cellbackground, colframe=cellborder]
%\prompt{In}{incolor}{3}{\boxspacing}
%\begin{Verbatim}[commandchars=\\\{\}]
%\PY{c+c1}{\PYZsh{} \PYZpc{}config InlineBackend.figure\PYZus{}formats = [\PYZsq{}pdf\PYZsq{}]}
%\PY{c+c1}{\PYZsh{} \PYZpc{}config Completer.use\PYZus{}jedi = False}
%\end{Verbatim}
%\end{tcolorbox}
%
%    \begin{center}\rule{0.5\linewidth}{0.5pt}\end{center}

    \hypertarget{ux430ux43bux433ux43eux440ux438ux442ux43c-ux431ux430ux439ux435ux441ux43eux432ux441ux43aux43eux439-ux43eux43fux442ux438ux43cux438ux437ux430ux446ux438ux438}{%
\section{Алгоритм байесовской
оптимизации}\label{ux430ux43bux433ux43eux440ux438ux442ux43c-ux431ux430ux439ux435ux441ux43eux432ux441ux43aux43eux439-ux43eux43fux442ux438ux43cux438ux437ux430ux446ux438ux438}}

    При решении практических задач часто приходится иметь дело с
оптимизацией «чёрного ящика». В таком случае мы не имеем почти никакой
информации о целевой функции \(f\): мы не знаем её аналитического
выражения, значений производных и т. д. Всё что мы можем --- это
работать с чёрным ящиком по системе «запрос -- ответ», т. е. получать
значения функции (отклики) в нужных нам точках. Причём эти отклики могут
быть шумными, т. е. могут быть подвержены влиянию некоторой случайной
ошибки.

Если чёрный ящик работает быстро, можно воспользоваться методом «грубой
силы»: вычислить отклики на большом массиве точек и выбрать оптимум. По
такому принципу работают методы поиска по сетке (grid search) или
случайного поиска (random search). Другой вариант --- воспользоваться
градиентным методом, а значения частных производных найти численно.
Однако в этом случае нужно действовать аккуратно, особенно в случае
шумных откликов.

Если чёрный ящик работает долго и оценка целевой функции является
вычислительно дорогой, как, например, проведение аэродинамического
расчёта, то количество обращений к чёрному ящику лучше свести к
минимуму. Именно в этом случае наиболее полезны байесовские методы
оптимизации, так как они призваны найти глобальный экстремум целевой
функции за минимальное количество итераций. Байесовская оптимизация
начинается с априорной оценки целевой функции \(f\) и обновляет её на
каждой итерации, используя вновь полученные данные.

Модель, используемая для аппроксимации целевой функции, называется
\emph{суррогатной моделью}. За поиск новой точки для следующей итерации
отвечает \emph{функция продвижения} (acquisition function), направляющая
процесс поиска в те области, где наиболее вероятно получить улучшение
результата.

    Алгоритм байесовской оптимизации можно формализовать следующим образом:

\begin{enumerate}
\def\labelenumi{\arabic{enumi}.}
\tightlist
\item
  по точкам обучающей выборки \(\{X_{train}, Y_{train}\}\) построить
  суррогатную модель;
\item
  используя функцию продвижения, найти следующую точку алгоритма
  \(x_{next}\);
\item
  если выполнен критерий останова, закончить;
\item
  получить значение целевой функции \(y_{next} = f_{obj}(x_{next})\);
\item
  добавить точку \((x_{next}, y_{next})\) в обучающую выборку и перейти
  к пункту 1.
\end{enumerate}

В литературе данный алгоритм встречается также под названием
\emph{Efficient Global Optimization (EGO)}.

    Реализуем алгоритм оптимизации с помощью функции \texttt{next\_point},
возвращающей следующую исследуемую точку.

%    \begin{tcolorbox}[breakable, size=fbox, boxrule=1pt, pad at break*=1mm,colback=cellbackground, colframe=cellborder]
%\prompt{In}{incolor}{4}{\boxspacing}
%\begin{Verbatim}[commandchars=\\\{\}]
%\PY{k}{def} \PY{n+nf}{next\PYZus{}point}\PY{p}{(}\PY{n}{acquisition}\PY{p}{,} \PY{n}{X\PYZus{}train}\PY{p}{,} \PY{n}{Y\PYZus{}train}\PY{p}{,} \PY{n}{bounds}\PY{p}{)}\PY{p}{:}
%    \PY{l+s+sd}{\PYZsq{}\PYZsq{}\PYZsq{}}
%\PY{l+s+sd}{    Proposes the next sampling point by optimizing the acquisition function.}
%\PY{l+s+sd}{    }
%\PY{l+s+sd}{    Args:}
%\PY{l+s+sd}{        acquisition: Acquisition function}
%\PY{l+s+sd}{        X\PYZus{}train: Sample locations (n x d)}
%\PY{l+s+sd}{        Y\PYZus{}train: Sample values (n x 1)}
%
%\PY{l+s+sd}{    Returns:}
%\PY{l+s+sd}{        Location of the acquisition function minimum}
%\PY{l+s+sd}{    \PYZsq{}\PYZsq{}\PYZsq{}}
%    \PY{n}{dim} \PY{o}{=} \PY{n}{X\PYZus{}train}\PY{o}{.}\PY{n}{shape}\PY{p}{[}\PY{l+m+mi}{1}\PY{p}{]}
%    \PY{k}{def} \PY{n+nf}{min\PYZus{}obj}\PY{p}{(}\PY{n}{X}\PY{p}{)}\PY{p}{:}
%        \PY{c+c1}{\PYZsh{} Minimization objective is the negative acquisition function}
%        \PY{k}{return} \PY{o}{\PYZhy{}}\PY{n}{acquisition}\PY{p}{(}\PY{n}{X}\PY{o}{.}\PY{n}{reshape}\PY{p}{(}\PY{o}{\PYZhy{}}\PY{l+m+mi}{1}\PY{p}{,} \PY{n}{dim}\PY{p}{)}\PY{p}{,} \PY{n}{X\PYZus{}train}\PY{p}{,} \PY{n}{Y\PYZus{}train}\PY{p}{)}
%    
%    \PY{n}{X} \PY{o}{=} \PY{n}{np}\PY{o}{.}\PY{n}{linspace}\PY{p}{(}\PY{n}{bounds}\PY{p}{[}\PY{l+m+mi}{0}\PY{p}{]}\PY{p}{,} \PY{n}{bounds}\PY{p}{[}\PY{l+m+mi}{1}\PY{p}{]}\PY{p}{,} \PY{n}{N\PYZus{}test}\PY{p}{)}\PY{o}{.}\PY{n}{reshape}\PY{p}{(}\PY{o}{\PYZhy{}}\PY{l+m+mi}{1}\PY{p}{,} \PY{l+m+mi}{1}\PY{p}{)}
%    \PY{n}{Y} \PY{o}{=} \PY{n}{min\PYZus{}obj}\PY{p}{(}\PY{n}{X}\PY{p}{)}
%    \PY{n}{i\PYZus{}min} \PY{o}{=} \PY{n}{np}\PY{o}{.}\PY{n}{argmin}\PY{p}{(}\PY{n}{Y}\PY{p}{)}
%    
%    \PY{k}{return} \PY{n}{X}\PY{p}{[}\PY{n}{i\PYZus{}min}\PY{p}{]}\PY{o}{.}\PY{n}{reshape}\PY{p}{(}\PY{o}{\PYZhy{}}\PY{l+m+mi}{1}\PY{p}{,} \PY{l+m+mi}{1}\PY{p}{)}\PY{p}{,} \PY{n}{Y}\PY{p}{[}\PY{n}{i\PYZus{}min}\PY{p}{]}
%\end{Verbatim}
%\end{tcolorbox}

%    \begin{center}\rule{0.5\linewidth}{0.5pt}\end{center}

    \hypertarget{ux444ux443ux43dux43aux446ux438ux44f-ux43fux440ux43eux434ux432ux438ux436ux435ux43dux438ux44f}{%
\section{Функция
продвижения}\label{ux444ux443ux43dux43aux446ux438ux44f-ux43fux440ux43eux434ux432ux438ux436ux435ux43dux438ux44f}}

    Поиск следующей исследуемой точки осуществляет функция продвижения
(acquisition function). Функция продвижения реализует подход, известный
как «эксплуатация и эксплорация» (exploitation and exploration). Данный
подход призван найти компромисс между локальным и глобальным поиском.
Эксплуатация отвечает за поиск минимума там, где суррогатная модель
предсказывает хороший результат, а эксплорация отвечает за поиск в зонах
с большой дисперсией прогноза суррогатной модели.

    Далее мы рассмотрим три варианта функции продвижения:

\begin{itemize}
\tightlist
\item
  нижняя граница доверительного интервала (LB),
\item
  вероятность улучшения (PI),
\item
  ожидаемое улучшение (EI).
\end{itemize}

    \hypertarget{ux43dux438ux436ux43dux44fux44f-ux433ux440ux430ux43dux438ux446ux430-ux434ux43eux432ux435ux440ux438ux442ux435ux43bux44cux43dux43eux433ux43e-ux438ux43dux442ux435ux440ux432ux430ux43bux430}{%
\subsection{Нижняя граница доверительного
интервала}\label{ux43dux438ux436ux43dux44fux44f-ux433ux440ux430ux43dux438ux446ux430-ux434ux43eux432ux435ux440ux438ux442ux435ux43bux44cux43dux43eux433ux43e-ux438ux43dux442ux435ux440ux432ux430ux43bux430}}

Первый вариант функции продвижения основан на значении нижней границы
доверительного интервала \(\mu(x) - k \sigma(x)\). Функция продвижения
возвращает разницу между \emph{располагаемым} минимальным значением
\(f_\mathrm{min}\) (минимум по обучающей выборке) и
\emph{предсказываемым} моделью значением нижней границы доверительного
интервала:
\[
  \mathrm{LB}(x) = f_\mathrm{min} - \left(\mu(x) - k \sigma(x) \right).
\]
В качестве величины доверительного интервала используется несколько
среднеквадратичных отклонений: \(k\sigma\). Параметр \(k\) влияет на
баланс между эксплуатацией и эксплорацией: при \(k \rightarrow 0\),
\(\mathrm{LB}(x) \rightarrow f_\mathrm{min} - \mu(x)\) (чистая
эксплуатация), при \(k \rightarrow \infty\),
\(\mathrm{LB}(x) \rightarrow -k \sigma(x)\) (чистая эксплорация).

%    \begin{tcolorbox}[breakable, size=fbox, boxrule=1pt, pad at break*=1mm,colback=cellbackground, colframe=cellborder]
%\prompt{In}{incolor}{5}{\boxspacing}
%\begin{Verbatim}[commandchars=\\\{\}]
%\PY{k}{def} \PY{n+nf}{lower\PYZus{}bound}\PY{p}{(}\PY{n}{X}\PY{p}{,} \PY{n}{X\PYZus{}train}\PY{p}{,} \PY{n}{Y\PYZus{}train}\PY{p}{,} \PY{n}{k}\PY{o}{=}\PY{l+m+mf}{2.}\PY{p}{)}\PY{p}{:}
%    \PY{l+s+sd}{\PYZsq{}\PYZsq{}\PYZsq{}}
%\PY{l+s+sd}{    Predicted Minimum = mu \PYZhy{} k*std}
%\PY{l+s+sd}{    Computes the predicted minimum at points X based on existing samples}
%\PY{l+s+sd}{    X\PYZus{}train and Y\PYZus{}train using a Gaussian process surrogate model.}
%\PY{l+s+sd}{    }
%\PY{l+s+sd}{    Args:}
%\PY{l+s+sd}{        X: Points at which function shall be computed (m x d)}
%\PY{l+s+sd}{        X\PYZus{}train: Sample locations (n x d)}
%\PY{l+s+sd}{        Y\PYZus{}train: Sample values (n x 1)}
%\PY{l+s+sd}{        k: Exploitation\PYZhy{}exploration trade\PYZhy{}off parameter}
%\PY{l+s+sd}{    }
%\PY{l+s+sd}{    Returns:}
%\PY{l+s+sd}{        Predicted minimum at points X}
%\PY{l+s+sd}{    \PYZsq{}\PYZsq{}\PYZsq{}}
%    \PY{n}{mu}\PY{p}{,} \PY{n}{cov} \PY{o}{=} \PY{n}{GP\PYZus{}predictor}\PY{p}{(}\PY{n}{X}\PY{p}{,} \PY{n}{X\PYZus{}train}\PY{p}{,} \PY{n}{Y\PYZus{}train}\PY{p}{,}
%                           \PY{n}{kernel\PYZus{}fun}\PY{p}{,} \PY{n}{kernel\PYZus{}args}\PY{p}{,} \PY{n}{sigma\PYZus{}n}\PY{p}{)}
%    \PY{n}{mu} \PY{o}{=} \PY{n}{mu}\PY{o}{.}\PY{n}{flatten}\PY{p}{(}\PY{p}{)}
%    \PY{n}{std} \PY{o}{=} \PY{n}{np}\PY{o}{.}\PY{n}{sqrt}\PY{p}{(}\PY{n}{np}\PY{o}{.}\PY{n}{diag}\PY{p}{(}\PY{n}{cov}\PY{p}{)}\PY{p}{)}
%    \PY{n}{y\PYZus{}min} \PY{o}{=} \PY{n}{np}\PY{o}{.}\PY{n}{min}\PY{p}{(}\PY{n}{Y\PYZus{}train}\PY{p}{)}
%    \PY{n}{res} \PY{o}{=} \PY{n}{y\PYZus{}min} \PY{o}{\PYZhy{}} \PY{p}{(}\PY{n}{mu} \PY{o}{\PYZhy{}} \PY{n}{k}\PY{o}{*}\PY{n}{std}\PY{p}{)}
%    \PY{n}{res}\PY{p}{[}\PY{n}{res}\PY{o}{\PYZlt{}}\PY{l+m+mi}{0}\PY{p}{]} \PY{o}{=} \PY{l+m+mi}{0}
%    \PY{k}{return} \PY{n}{res}
%\end{Verbatim}
%\end{tcolorbox}

    \hypertarget{ux432ux435ux440ux43eux44fux442ux43dux43eux441ux442ux44c-ux443ux43bux443ux447ux448ux435ux43dux438ux44f}{%
\subsection{Вероятность
улучшения}\label{ux432ux435ux440ux43eux44fux442ux43dux43eux441ux442ux44c-ux443ux43bux443ux447ux448ux435ux43dux438ux44f}}

Результат работы предыдущей функции сильно зависит от значения параметра
\(k\). Попробуем избавиться от него. Для этого посчитаем
\emph{вероятность улучшения}.

    Функция улучшения определяется как \[
  I(x) = \max \left( f_\mathrm{min} - y(x), 0 \right),
\] где \(f_\mathrm{min}\) --- значение текущего минимума, а \(y(x)\) ---
предсказываемое суррогатной моделью значение в точке \(x\). Отметим, что
\(y(x)\) является сечением гауссовского процесса в точке \(x\) и,
следовательно, гауссовской случайной величиной с плотностью
распределения
\(p(y) \sim \mathcal{N}\left( \mu(x), \sigma^2(x) \right)\).

%    \begin{tcolorbox}[breakable, size=fbox, boxrule=1pt, pad at break*=1mm,colback=cellbackground, colframe=cellborder]
%\prompt{In}{incolor}{6}{\boxspacing}
%\begin{Verbatim}[commandchars=\\\{\}]
%\PY{n}{display}\PY{p}{(}\PY{n}{Image}\PY{p}{(}\PY{l+s+s1}{\PYZsq{}}\PY{l+s+s1}{./pix/10.Bayesian\PYZus{}optimization/PI.png}\PY{l+s+s1}{\PYZsq{}}\PY{p}{,} \PY{n}{width}\PY{o}{=}\PY{l+m+mf}{0.65}\PY{o}{*}\PY{n}{im\PYZus{}width}\PY{p}{)}\PY{p}{)}
%\end{Verbatim}
%\end{tcolorbox}

    \begin{center}
    \adjustimage{max size={0.65\linewidth}{0.65\paperheight}}{PI.png}
    \end{center}
%    { \hspace*{\fill} \\}
    
    Если в качестве суррогатной модели используется регрессия на гауссовых
процессах, вероятность улучшения можно посчитать аналитически: \[
  PI(x) = \mathrm{P}[I(x)>0] = \int \limits_{-\infty}^{f_\mathrm{min}} p(y) dy =
  \int \limits_{-\infty}^{f_\mathrm{min}} \frac{1}{\sqrt{2\pi}\sigma} \exp{ \left( -\frac{(y - \mu)^2}{2\sigma^2}\right)} dy =
  \int \limits_{-\infty}^{\frac{f_\mathrm{min}-\mu}{\sigma}}\phi(z) dz = \Phi\left(\frac{f_\mathrm{min} - \mu(x)}{\sigma(x)}\right).
\] Здесь \(\Phi(z)\) --- функция стандартного нормального распределения,
а \(\phi(z)\) --- его плотность.

%    \begin{tcolorbox}[breakable, size=fbox, boxrule=1pt, pad at break*=1mm,colback=cellbackground, colframe=cellborder]
%\prompt{In}{incolor}{7}{\boxspacing}
%\begin{Verbatim}[commandchars=\\\{\}]
%\PY{k}{def} \PY{n+nf}{probability\PYZus{}of\PYZus{}improvement}\PY{p}{(}\PY{n}{X}\PY{p}{,} \PY{n}{X\PYZus{}train}\PY{p}{,} \PY{n}{Y\PYZus{}train}\PY{p}{,} \PY{n}{xi}\PY{o}{=}\PY{l+m+mf}{0.01}\PY{p}{)}\PY{p}{:}
%    \PY{l+s+sd}{\PYZsq{}\PYZsq{}\PYZsq{}\PYZsq{}\PYZsq{}\PYZsq{}}
%    \PY{n}{mu}\PY{p}{,} \PY{n}{cov} \PY{o}{=} \PY{n}{GP\PYZus{}predictor}\PY{p}{(}\PY{n}{X}\PY{p}{,} \PY{n}{X\PYZus{}train}\PY{p}{,} \PY{n}{Y\PYZus{}train}\PY{p}{,}
%                           \PY{n}{kernel\PYZus{}fun}\PY{p}{,} \PY{n}{kernel\PYZus{}args}\PY{p}{,} \PY{n}{sigma\PYZus{}n}\PY{p}{)}
%    \PY{n}{mu} \PY{o}{=} \PY{n}{mu}\PY{o}{.}\PY{n}{flatten}\PY{p}{(}\PY{p}{)}
%    \PY{n}{std} \PY{o}{=} \PY{n}{np}\PY{o}{.}\PY{n}{sqrt}\PY{p}{(}\PY{n}{np}\PY{o}{.}\PY{n}{diag}\PY{p}{(}\PY{n}{cov}\PY{p}{)}\PY{p}{)}
%    \PY{n}{y\PYZus{}min} \PY{o}{=} \PY{n}{np}\PY{o}{.}\PY{n}{min}\PY{p}{(}\PY{n}{Y\PYZus{}train}\PY{p}{)}
%    \PY{n}{std\PYZus{}max} \PY{o}{=} \PY{n}{np}\PY{o}{.}\PY{n}{max}\PY{p}{(}\PY{n}{std}\PY{p}{)}
%
%    \PY{k}{with} \PY{n}{np}\PY{o}{.}\PY{n}{errstate}\PY{p}{(}\PY{n}{divide}\PY{o}{=}\PY{l+s+s1}{\PYZsq{}}\PY{l+s+s1}{warn}\PY{l+s+s1}{\PYZsq{}}\PY{p}{)}\PY{p}{:}
%        \PY{n}{imp} \PY{o}{=} \PY{n}{y\PYZus{}min} \PY{o}{\PYZhy{}} \PY{n}{mu} \PY{o}{\PYZhy{}} \PY{n}{xi}\PY{o}{*}\PY{n}{std\PYZus{}max}
%        \PY{n}{z} \PY{o}{=} \PY{n}{imp} \PY{o}{/} \PY{n}{std}
%    \PY{n}{res} \PY{o}{=} \PY{n}{norm}\PY{o}{.}\PY{n}{cdf}\PY{p}{(}\PY{n}{z}\PY{p}{)}
%
%    \PY{k}{return} \PY{n}{res}
%\end{Verbatim}
%\end{tcolorbox}

    \hypertarget{ux43eux436ux438ux434ux430ux435ux43cux43eux435-ux443ux43bux443ux447ux448ux435ux43dux438ux435}{%
\subsection{Ожидаемое
улучшение}\label{ux43eux436ux438ux434ux430ux435ux43cux43eux435-ux443ux43bux443ux447ux448ux435ux43dux438ux435}}

Следующий шаг --- использовать не вероятность улучшения, а его ожидаемую
величину. Для этого вычислим (тоже аналитически) \emph{математическое
ожидание} улучшения. \[
\begin{split}
  \mathrm{E}[I]
  &= \int \limits_{-\infty}^{\infty} I p(y) dy
   = \int \limits_{-\infty}^{f_\mathrm{min}} (f_\mathrm{min} - y)\,p(y) dy
   = \int \limits_{-\infty}^{\frac{f_\mathrm{min}-\mu}{\sigma}} \left( f_\mathrm{min} - (\mu+\sigma z) \right) \phi(z) dz \\
  &= \left( f_\mathrm{min} - \mu(x) \right) \, \Phi\left(\frac{f_\mathrm{min} - \mu(x)}{\sigma(x)}\right) + \sigma(x) \, \phi\left(\frac{f_\mathrm{min} - \mu(x)}{\sigma(x)}\right).
\end{split}
\]

Получившаяся функция продвижения \(EI(x)\), называемая \emph{ожидаемым
улучшением}, удачно сочетает эксплуатацию и эксплорацию и поэтому
используется чаще других. Первое слагаемое, с точностью до множителя
совпадающее с вероятностью улучшения \(PI(x)\), отвечает за
эксплуатацию, второе --- за эксплорацию.

%    \begin{tcolorbox}[breakable, size=fbox, boxrule=1pt, pad at break*=1mm,colback=cellbackground, colframe=cellborder]
%\prompt{In}{incolor}{8}{\boxspacing}
%\begin{Verbatim}[commandchars=\\\{\}]
%\PY{k}{def} \PY{n+nf}{expected\PYZus{}improvement}\PY{p}{(}\PY{n}{X}\PY{p}{,} \PY{n}{X\PYZus{}train}\PY{p}{,} \PY{n}{Y\PYZus{}train}\PY{p}{,} \PY{n}{xi}\PY{o}{=}\PY{l+m+mf}{0.01}\PY{p}{)}\PY{p}{:}
%    \PY{l+s+sd}{\PYZsq{}\PYZsq{}\PYZsq{}}
%\PY{l+s+sd}{    Expected Improvement}
%\PY{l+s+sd}{    Computes the expected improvement at points X based on existing samples}
%\PY{l+s+sd}{    X\PYZus{}train and Y\PYZus{}train using a Gaussian process surrogate model.}
%\PY{l+s+sd}{    }
%\PY{l+s+sd}{    Args:}
%\PY{l+s+sd}{        X: Points at which EI shall be computed (m x d)}
%\PY{l+s+sd}{        X\PYZus{}train: Sample locations (n x d)}
%\PY{l+s+sd}{        Y\PYZus{}train: Sample values (n x 1)}
%\PY{l+s+sd}{        xi: Exploitation\PYZhy{}exploration trade\PYZhy{}off parameter}
%\PY{l+s+sd}{    }
%\PY{l+s+sd}{    Returns:}
%\PY{l+s+sd}{        Expected improvement at points X}
%\PY{l+s+sd}{    \PYZsq{}\PYZsq{}\PYZsq{}}
%    \PY{n}{mu}\PY{p}{,} \PY{n}{cov} \PY{o}{=} \PY{n}{GP\PYZus{}predictor}\PY{p}{(}\PY{n}{X}\PY{p}{,} \PY{n}{X\PYZus{}train}\PY{p}{,} \PY{n}{Y\PYZus{}train}\PY{p}{,}
%                           \PY{n}{kernel\PYZus{}fun}\PY{p}{,} \PY{n}{kernel\PYZus{}args}\PY{p}{,} \PY{n}{sigma\PYZus{}n}\PY{p}{)}
%    \PY{n}{mu} \PY{o}{=} \PY{n}{mu}\PY{o}{.}\PY{n}{flatten}\PY{p}{(}\PY{p}{)}
%    \PY{n}{std} \PY{o}{=} \PY{n}{np}\PY{o}{.}\PY{n}{sqrt}\PY{p}{(}\PY{n}{np}\PY{o}{.}\PY{n}{diag}\PY{p}{(}\PY{n}{cov}\PY{p}{)}\PY{p}{)}
%    \PY{n}{y\PYZus{}min} \PY{o}{=} \PY{n}{np}\PY{o}{.}\PY{n}{min}\PY{p}{(}\PY{n}{Y\PYZus{}train}\PY{p}{)}
%    \PY{n}{std\PYZus{}max} \PY{o}{=} \PY{n}{np}\PY{o}{.}\PY{n}{max}\PY{p}{(}\PY{n}{std}\PY{p}{)}
%
%    \PY{k}{with} \PY{n}{np}\PY{o}{.}\PY{n}{errstate}\PY{p}{(}\PY{n}{divide}\PY{o}{=}\PY{l+s+s1}{\PYZsq{}}\PY{l+s+s1}{warn}\PY{l+s+s1}{\PYZsq{}}\PY{p}{)}\PY{p}{:}
%        \PY{n}{imp} \PY{o}{=} \PY{n}{y\PYZus{}min} \PY{o}{\PYZhy{}} \PY{n}{mu} \PY{o}{\PYZhy{}} \PY{n}{xi}\PY{o}{*}\PY{n}{std\PYZus{}max}
%        \PY{n}{z} \PY{o}{=} \PY{n}{imp} \PY{o}{/} \PY{n}{std}
%    \PY{n}{res} \PY{o}{=} \PY{n}{imp} \PY{o}{*} \PY{n}{norm}\PY{o}{.}\PY{n}{cdf}\PY{p}{(}\PY{n}{z}\PY{p}{)} \PY{o}{+} \PY{n}{std} \PY{o}{*} \PY{n}{norm}\PY{o}{.}\PY{n}{pdf}\PY{p}{(}\PY{n}{z}\PY{p}{)}
%
%    \PY{k}{return} \PY{n}{res}
%\end{Verbatim}
%\end{tcolorbox}

%    \begin{center}\rule{0.5\linewidth}{0.5pt}\end{center}

    \hypertarget{ux442ux435ux441ux442-1.-ux431ux435ux441ux448ux443ux43cux43dux44bux435-ux434ux430ux43dux43dux44bux435}{%
\section{Тест 1. Бесшумные
данные}\label{ux442ux435ux441ux442-1.-ux431ux435ux441ux448ux443ux43cux43dux44bux435-ux434ux430ux43dux43dux44bux435}}

    Проведём тестирование байесовского алгоритма оптимизации.

Будем использовать следующую целевую функцию \texttt{f\_obj}: \[
  f_{obj} = (6x-2)^2 \cdot \sin\left(12x-3\right).
\]
На первом этапе рассмотрим бесшумные отклики. Исходная обучающая выборка
задаётся переменными \texttt{X\_init} и \texttt{Y\_init}.

    \hypertarget{ux434ux430ux43dux43dux44bux435}{%
\subsection{Данные}\label{ux434ux430ux43dux43dux44bux435}}

    Для начала нарисуем исследуемую бесшумную целевую функцию.

%    \begin{tcolorbox}[breakable, size=fbox, boxrule=1pt, pad at break*=1mm,colback=cellbackground, colframe=cellborder]
%\prompt{In}{incolor}{9}{\boxspacing}
%\begin{Verbatim}[commandchars=\\\{\}]
%\PY{c+c1}{\PYZsh{} Objective function}
%\PY{k}{def} \PY{n+nf}{f\PYZus{}obj}\PY{p}{(}\PY{n}{X}\PY{p}{)}\PY{p}{:}
%    \PY{k}{return} \PY{p}{(}\PY{l+m+mi}{6}\PY{o}{*}\PY{n}{X}\PY{o}{\PYZhy{}}\PY{l+m+mi}{2}\PY{p}{)}\PY{o}{*}\PY{o}{*}\PY{l+m+mi}{2} \PY{o}{*} \PY{n}{np}\PY{o}{.}\PY{n}{sin}\PY{p}{(}\PY{l+m+mi}{12}\PY{o}{*}\PY{n}{X}\PY{o}{\PYZhy{}}\PY{l+m+mi}{4}\PY{p}{)}
%\end{Verbatim}
%\end{tcolorbox}
%
%    \begin{tcolorbox}[breakable, size=fbox, boxrule=1pt, pad at break*=1mm,colback=cellbackground, colframe=cellborder]
%\prompt{In}{incolor}{10}{\boxspacing}
%\begin{Verbatim}[commandchars=\\\{\}]
%\PY{c+c1}{\PYZsh{} Dense grid of points within bounds}
%\PY{n}{x\PYZus{}lims} \PY{o}{=} \PY{p}{[}\PY{l+m+mf}{0.}\PY{p}{,} \PY{l+m+mf}{1.}\PY{p}{]}
%\PY{n}{N\PYZus{}test} \PY{o}{=} \PY{l+m+mi}{501}
%\PY{n}{X\PYZus{}test} \PY{o}{=} \PY{n}{np}\PY{o}{.}\PY{n}{linspace}\PY{p}{(}\PY{o}{*}\PY{n}{x\PYZus{}lims}\PY{p}{,} \PY{n}{N\PYZus{}test}\PY{p}{)}\PY{o}{.}\PY{n}{reshape}\PY{p}{(}\PY{o}{\PYZhy{}}\PY{l+m+mi}{1}\PY{p}{,} \PY{l+m+mi}{1}\PY{p}{)}
%\PY{c+c1}{\PYZsh{} Objective function values at X\PYZus{}test }
%\PY{n}{Y\PYZus{}true} \PY{o}{=} \PY{n}{f\PYZus{}obj}\PY{p}{(}\PY{n}{X\PYZus{}test}\PY{p}{)}
%\end{Verbatim}
%\end{tcolorbox}
%
%    \begin{tcolorbox}[breakable, size=fbox, boxrule=1pt, pad at break*=1mm,colback=cellbackground, colframe=cellborder]
%\prompt{In}{incolor}{11}{\boxspacing}
%\begin{Verbatim}[commandchars=\\\{\}]
%\PY{c+c1}{\PYZsh{} Plot optimization objective}
%\PY{n}{plt}\PY{o}{.}\PY{n}{figure}\PY{p}{(}\PY{n}{figsize}\PY{o}{=}\PY{p}{(}\PY{l+m+mi}{8}\PY{p}{,} \PY{l+m+mi}{5}\PY{p}{)}\PY{p}{)}
%\PY{n}{plt}\PY{o}{.}\PY{n}{plot}\PY{p}{(}\PY{n}{X\PYZus{}test}\PY{p}{,} \PY{n}{Y\PYZus{}true}\PY{p}{,} \PY{l+s+s1}{\PYZsq{}}\PY{l+s+s1}{k\PYZhy{}}\PY{l+s+s1}{\PYZsq{}}\PY{p}{,} \PY{n}{label}\PY{o}{=}\PY{l+s+s1}{\PYZsq{}}\PY{l+s+s1}{Objective function}\PY{l+s+s1}{\PYZsq{}}\PY{p}{)}
%\PY{n}{plt}\PY{o}{.}\PY{n}{legend}\PY{p}{(}\PY{p}{)}
%\PY{n}{plt}\PY{o}{.}\PY{n}{tight\PYZus{}layout}\PY{p}{(}\PY{p}{)}
%\PY{n}{plt}\PY{o}{.}\PY{n}{show}\PY{p}{(}\PY{p}{)}
%\end{Verbatim}
%\end{tcolorbox}

    \begin{center}
    \adjustimage{max size={0.65\linewidth}{0.65\paperheight}}{Test1_objective.pdf}
    \end{center}
%    { \hspace*{\fill} \\}
    
    \hypertarget{ux441ux442ux430ux43dux434ux430ux440ux442ux43dux430ux44f-ux43eux43fux442ux438ux43cux438ux437ux430ux446ux438ux44f}{%
\subsection{Стандартная
оптимизация}\label{ux441ux442ux430ux43dux434ux430ux440ux442ux43dux430ux44f-ux43eux43fux442ux438ux43cux438ux437ux430ux446ux438ux44f}}

    Далее попробуем найти оптимум стандартными алгоритмами.

    \textbf{Метод Бройдена --- Флетчера --- Гольдфарба --- Шанно (BFGS)}

%    \begin{tcolorbox}[breakable, size=fbox, boxrule=1pt, pad at break*=1mm,colback=cellbackground, colframe=cellborder]
%\prompt{In}{incolor}{12}{\boxspacing}
%\begin{Verbatim}[commandchars=\\\{\}]
%\PY{c+c1}{\PYZsh{} BFGS}
%\PY{n}{x0} \PY{o}{=} \PY{l+m+mf}{0.5}
%\PY{n}{res} \PY{o}{=} \PY{n}{minimize}\PY{p}{(}\PY{n}{f\PYZus{}obj}\PY{p}{,} \PY{n}{x0}\PY{p}{,} \PY{n}{bounds}\PY{o}{=}\PY{p}{[}\PY{p}{(}\PY{l+m+mi}{0}\PY{p}{,} \PY{l+m+mi}{1}\PY{p}{)}\PY{p}{]}\PY{p}{,} \PY{n}{tol}\PY{o}{=}\PY{l+m+mf}{1e\PYZhy{}2}\PY{p}{,}
%               \PY{n}{method}\PY{o}{=}\PY{l+s+s1}{\PYZsq{}}\PY{l+s+s1}{L\PYZhy{}BFGS\PYZhy{}B}\PY{l+s+s1}{\PYZsq{}}\PY{p}{,} \PY{n}{options}\PY{o}{=}\PY{p}{\PYZob{}}\PY{l+s+s1}{\PYZsq{}}\PY{l+s+s1}{disp}\PY{l+s+s1}{\PYZsq{}}\PY{p}{:}\PY{k+kc}{True}\PY{p}{\PYZcb{}}\PY{p}{)}
%\PY{n+nb}{print}\PY{p}{(}\PY{l+s+sa}{f}\PY{l+s+s1}{\PYZsq{}}\PY{l+s+s1}{x = }\PY{l+s+si}{\PYZob{}}\PY{n}{res}\PY{o}{.}\PY{n}{x}\PY{l+s+si}{\PYZcb{}}\PY{l+s+se}{\PYZbs{}n}\PY{l+s+s1}{\PYZsq{}}\PY{p}{)}
%\end{Verbatim}
%\end{tcolorbox}
%
%    \begin{Verbatim}[commandchars=\\\{\}]
%x = [0.14273509]
%
%    \end{Verbatim}
%
%    \begin{tcolorbox}[breakable, size=fbox, boxrule=1pt, pad at break*=1mm,colback=cellbackground, colframe=cellborder]
%\prompt{In}{incolor}{13}{\boxspacing}
%\begin{Verbatim}[commandchars=\\\{\}]
%\PY{c+c1}{\PYZsh{} Plot optimization objective}
%\PY{n}{plt}\PY{o}{.}\PY{n}{figure}\PY{p}{(}\PY{n}{figsize}\PY{o}{=}\PY{p}{(}\PY{l+m+mi}{8}\PY{p}{,} \PY{l+m+mi}{5}\PY{p}{)}\PY{p}{)}
%\PY{n}{plt}\PY{o}{.}\PY{n}{plot}\PY{p}{(}\PY{n}{X\PYZus{}test}\PY{p}{,} \PY{n}{Y\PYZus{}true}\PY{p}{,} \PY{l+s+s1}{\PYZsq{}}\PY{l+s+s1}{k\PYZhy{}}\PY{l+s+s1}{\PYZsq{}}\PY{p}{,} \PY{n}{label}\PY{o}{=}\PY{l+s+s1}{\PYZsq{}}\PY{l+s+s1}{Objective function}\PY{l+s+s1}{\PYZsq{}}\PY{p}{)}
%\PY{n}{plt}\PY{o}{.}\PY{n}{plot}\PY{p}{(}\PY{n}{res}\PY{o}{.}\PY{n}{x}\PY{p}{,} \PY{n}{f\PYZus{}obj}\PY{p}{(}\PY{n}{res}\PY{o}{.}\PY{n}{x}\PY{p}{)}\PY{p}{,} \PY{l+s+s1}{\PYZsq{}}\PY{l+s+s1}{*}\PY{l+s+s1}{\PYZsq{}}\PY{p}{,} \PY{n}{ms}\PY{o}{=}\PY{l+m+mi}{20}\PY{p}{,} \PY{n}{c}\PY{o}{=}\PY{n}{cm}\PY{o}{.}\PY{n}{tab10}\PY{p}{(}\PY{l+m+mi}{3}\PY{p}{)}\PY{p}{,} \PY{n}{label}\PY{o}{=}\PY{l+s+s1}{\PYZsq{}}\PY{l+s+s1}{Minimum}\PY{l+s+s1}{\PYZsq{}}\PY{p}{)}
%\PY{n}{plt}\PY{o}{.}\PY{n}{legend}\PY{p}{(}\PY{p}{)}
%\PY{n}{plt}\PY{o}{.}\PY{n}{tight\PYZus{}layout}\PY{p}{(}\PY{p}{)}
%\PY{n}{plt}\PY{o}{.}\PY{n}{show}\PY{p}{(}\PY{p}{)}
%\end{Verbatim}
%\end{tcolorbox}

%    \begin{center}
%    \adjustimage{max size={0.65\linewidth}{0.65\paperheight}}{Test1_BFGS.pdf}
%    \end{center}
%%    { \hspace*{\fill} \\}
    
    \textbf{Метод Нелдера --- Мида (симплекс)}

%    \begin{tcolorbox}[breakable, size=fbox, boxrule=1pt, pad at break*=1mm,colback=cellbackground, colframe=cellborder]
%\prompt{In}{incolor}{14}{\boxspacing}
%\begin{Verbatim}[commandchars=\\\{\}]
%\PY{c+c1}{\PYZsh{} Nelder\PYZhy{}Mead}
%\PY{n}{x0} \PY{o}{=} \PY{l+m+mf}{0.5}
%\PY{n}{res} \PY{o}{=} \PY{n}{minimize}\PY{p}{(}\PY{n}{f\PYZus{}obj}\PY{p}{,} \PY{n}{x0}\PY{p}{,} \PY{n}{bounds}\PY{o}{=}\PY{p}{[}\PY{p}{(}\PY{l+m+mi}{0}\PY{p}{,} \PY{l+m+mi}{1}\PY{p}{)}\PY{p}{]}\PY{p}{,} \PY{n}{tol}\PY{o}{=}\PY{l+m+mf}{1e\PYZhy{}2}\PY{p}{,}
%               \PY{n}{method}\PY{o}{=}\PY{l+s+s1}{\PYZsq{}}\PY{l+s+s1}{Nelder\PYZhy{}Mead}\PY{l+s+s1}{\PYZsq{}}\PY{p}{,} \PY{n}{options}\PY{o}{=}\PY{p}{\PYZob{}}\PY{l+s+s1}{\PYZsq{}}\PY{l+s+s1}{disp}\PY{l+s+s1}{\PYZsq{}}\PY{p}{:}\PY{k+kc}{True}\PY{p}{\PYZcb{}}\PY{p}{)}
%\PY{n+nb}{print}\PY{p}{(}\PY{l+s+sa}{f}\PY{l+s+s1}{\PYZsq{}}\PY{l+s+s1}{x = }\PY{l+s+si}{\PYZob{}}\PY{n}{res}\PY{o}{.}\PY{n}{x}\PY{l+s+si}{\PYZcb{}}\PY{l+s+se}{\PYZbs{}n}\PY{l+s+s1}{\PYZsq{}}\PY{p}{)}
%\end{Verbatim}
%\end{tcolorbox}
%
%    \begin{Verbatim}[commandchars=\\\{\}]
%Optimization terminated successfully.
%         Current function value: -0.986121
%         Iterations: 9
%         Function evaluations: 18
%x = [0.14375]
%
%    \end{Verbatim}
%
%    \begin{tcolorbox}[breakable, size=fbox, boxrule=1pt, pad at break*=1mm,colback=cellbackground, colframe=cellborder]
%\prompt{In}{incolor}{15}{\boxspacing}
%\begin{Verbatim}[commandchars=\\\{\}]
%\PY{c+c1}{\PYZsh{} Plot optimization objective}
%\PY{n}{plt}\PY{o}{.}\PY{n}{figure}\PY{p}{(}\PY{n}{figsize}\PY{o}{=}\PY{p}{(}\PY{l+m+mi}{8}\PY{p}{,} \PY{l+m+mi}{5}\PY{p}{)}\PY{p}{)}
%\PY{n}{plt}\PY{o}{.}\PY{n}{plot}\PY{p}{(}\PY{n}{X\PYZus{}test}\PY{p}{,} \PY{n}{Y\PYZus{}true}\PY{p}{,} \PY{l+s+s1}{\PYZsq{}}\PY{l+s+s1}{k\PYZhy{}}\PY{l+s+s1}{\PYZsq{}}\PY{p}{,} \PY{n}{label}\PY{o}{=}\PY{l+s+s1}{\PYZsq{}}\PY{l+s+s1}{Objective function}\PY{l+s+s1}{\PYZsq{}}\PY{p}{)}
%\PY{n}{plt}\PY{o}{.}\PY{n}{plot}\PY{p}{(}\PY{n}{res}\PY{o}{.}\PY{n}{x}\PY{p}{,} \PY{n}{f\PYZus{}obj}\PY{p}{(}\PY{n}{res}\PY{o}{.}\PY{n}{x}\PY{p}{)}\PY{p}{,} \PY{l+s+s1}{\PYZsq{}}\PY{l+s+s1}{*}\PY{l+s+s1}{\PYZsq{}}\PY{p}{,} \PY{n}{ms}\PY{o}{=}\PY{l+m+mi}{20}\PY{p}{,} \PY{n}{c}\PY{o}{=}\PY{n}{cm}\PY{o}{.}\PY{n}{tab10}\PY{p}{(}\PY{l+m+mi}{3}\PY{p}{)}\PY{p}{,} \PY{n}{label}\PY{o}{=}\PY{l+s+s1}{\PYZsq{}}\PY{l+s+s1}{Minimum}\PY{l+s+s1}{\PYZsq{}}\PY{p}{)}
%\PY{n}{plt}\PY{o}{.}\PY{n}{legend}\PY{p}{(}\PY{p}{)}
%\PY{n}{plt}\PY{o}{.}\PY{n}{tight\PYZus{}layout}\PY{p}{(}\PY{p}{)}
%\PY{n}{plt}\PY{o}{.}\PY{n}{show}\PY{p}{(}\PY{p}{)}
%\end{Verbatim}
%\end{tcolorbox}

%    \begin{center}
%    \adjustimage{max size={0.65\linewidth}{0.65\paperheight}}{Test1_Nelder-Mead.pdf}
%    \end{center}
%%    { \hspace*{\fill} \\}
    
    \hypertarget{ux431ux430ux439ux435ux441ux43eux432ux441ux43aux430ux44f-ux43eux43fux442ux438ux43cux438ux437ux430ux446ux438ux44f}{%
\subsection{Байесовская
оптимизация}\label{ux431ux430ux439ux435ux441ux43eux432ux441ux43aux430ux44f-ux43eux43fux442ux438ux43cux438ux437ux430ux446ux438ux44f}}

    Инициализируем начальную обучающую выборку.

%    \begin{tcolorbox}[breakable, size=fbox, boxrule=1pt, pad at break*=1mm,colback=cellbackground, colframe=cellborder]
%\prompt{In}{incolor}{16}{\boxspacing}
%\begin{Verbatim}[commandchars=\\\{\}]
%\PY{n}{x\PYZus{}range} \PY{o}{=} \PY{n}{x\PYZus{}lims}\PY{p}{[}\PY{l+m+mi}{1}\PY{p}{]} \PY{o}{\PYZhy{}} \PY{n}{x\PYZus{}lims}\PY{p}{[}\PY{l+m+mi}{0}\PY{p}{]}
%\PY{n}{X\PYZus{}init} \PY{o}{=} \PY{p}{[}\PY{n}{x\PYZus{}lims}\PY{p}{[}\PY{l+m+mi}{0}\PY{p}{]}\PY{p}{,} \PY{l+m+mf}{0.5}\PY{o}{*}\PY{n+nb}{sum}\PY{p}{(}\PY{n}{x\PYZus{}lims}\PY{p}{)}\PY{p}{,} \PY{n}{x\PYZus{}lims}\PY{p}{[}\PY{l+m+mi}{1}\PY{p}{]}\PY{p}{]}
%\PY{n}{X\PYZus{}init} \PY{o}{=} \PY{n}{np}\PY{o}{.}\PY{n}{array}\PY{p}{(}\PY{n}{X\PYZus{}init}\PY{p}{)}\PY{o}{.}\PY{n}{reshape}\PY{p}{(}\PY{o}{\PYZhy{}}\PY{l+m+mi}{1}\PY{p}{,} \PY{l+m+mi}{1}\PY{p}{)}
%\PY{n}{Y\PYZus{}init} \PY{o}{=} \PY{n}{f\PYZus{}obj}\PY{p}{(}\PY{n}{X\PYZus{}init}\PY{p}{)}
%\PY{n}{N\PYZus{}init} \PY{o}{=} \PY{n+nb}{len}\PY{p}{(}\PY{n}{X\PYZus{}init}\PY{p}{)}
%\end{Verbatim}
%\end{tcolorbox}
%
%    \begin{tcolorbox}[breakable, size=fbox, boxrule=1pt, pad at break*=1mm,colback=cellbackground, colframe=cellborder]
%\prompt{In}{incolor}{17}{\boxspacing}
%\begin{Verbatim}[commandchars=\\\{\}]
%\PY{c+c1}{\PYZsh{} Plot objective function}
%\PY{n}{plt}\PY{o}{.}\PY{n}{figure}\PY{p}{(}\PY{n}{figsize}\PY{o}{=}\PY{p}{(}\PY{l+m+mi}{8}\PY{p}{,} \PY{l+m+mi}{5}\PY{p}{)}\PY{p}{)}
%\PY{n}{plt}\PY{o}{.}\PY{n}{plot}\PY{p}{(}\PY{n}{X\PYZus{}test}\PY{p}{,} \PY{n}{Y\PYZus{}true}\PY{p}{,} \PY{l+s+s1}{\PYZsq{}}\PY{l+s+s1}{k\PYZhy{}}\PY{l+s+s1}{\PYZsq{}}\PY{p}{,} \PY{n}{label}\PY{o}{=}\PY{l+s+s1}{\PYZsq{}}\PY{l+s+s1}{Objective function}\PY{l+s+s1}{\PYZsq{}}\PY{p}{)}
%\PY{n}{plt}\PY{o}{.}\PY{n}{plot}\PY{p}{(}\PY{n}{X\PYZus{}init}\PY{p}{,} \PY{n}{Y\PYZus{}init}\PY{p}{,} \PY{l+s+s1}{\PYZsq{}}\PY{l+s+s1}{o}\PY{l+s+s1}{\PYZsq{}}\PY{p}{,} \PY{n}{c}\PY{o}{=}\PY{n}{cm}\PY{o}{.}\PY{n}{tab10}\PY{p}{(}\PY{l+m+mi}{3}\PY{p}{)}\PY{p}{,} \PY{n}{ms}\PY{o}{=}\PY{l+m+mi}{8}\PY{p}{,} \PY{n}{label}\PY{o}{=}\PY{l+s+s1}{\PYZsq{}}\PY{l+s+s1}{Initial samples}\PY{l+s+s1}{\PYZsq{}}\PY{p}{)}
%\PY{n}{plt}\PY{o}{.}\PY{n}{legend}\PY{p}{(}\PY{p}{)}
%\PY{n}{plt}\PY{o}{.}\PY{n}{tight\PYZus{}layout}\PY{p}{(}\PY{p}{)}
%\PY{n}{plt}\PY{o}{.}\PY{n}{show}\PY{p}{(}\PY{p}{)}
%\end{Verbatim}
%\end{tcolorbox}

%    \begin{center}
%    \adjustimage{max size={0.65\linewidth}{0.65\paperheight}}{Test1_DOE_points.pdf}
%    \end{center}
%%    { \hspace*{\fill} \\}
    
    Построим начальную суррогатную модель.

%    \begin{tcolorbox}[breakable, size=fbox, boxrule=1pt, pad at break*=1mm,colback=cellbackground, colframe=cellborder]
%\prompt{In}{incolor}{18}{\boxspacing}
%\begin{Verbatim}[commandchars=\\\{\}]
%\PY{n}{kernel\PYZus{}fun} \PY{o}{=} \PY{n}{GP\PYZus{}kernels}\PY{o}{.}\PY{n}{gauss}
%\PY{n}{kernel\PYZus{}args} \PY{o}{=} \PY{p}{\PYZob{}}\PY{l+s+s1}{\PYZsq{}}\PY{l+s+s1}{l}\PY{l+s+s1}{\PYZsq{}}\PY{p}{:}\PY{l+m+mf}{.2}\PY{p}{,} \PY{l+s+s1}{\PYZsq{}}\PY{l+s+s1}{sigma\PYZus{}k}\PY{l+s+s1}{\PYZsq{}}\PY{p}{:}\PY{l+m+mf}{2.}\PY{p}{\PYZcb{}}
%\PY{n}{sigma\PYZus{}n} \PY{o}{=} \PY{l+m+mf}{1e\PYZhy{}3}
%
%\PY{n}{mu}\PY{p}{,} \PY{n}{cov} \PY{o}{=} \PY{n}{GP\PYZus{}predictor}\PY{p}{(}\PY{n}{X\PYZus{}test}\PY{p}{,} \PY{n}{X\PYZus{}init}\PY{p}{,} \PY{n}{Y\PYZus{}init}\PY{p}{,}
%                       \PY{n}{kernel\PYZus{}fun}\PY{p}{,} \PY{n}{kernel\PYZus{}args}\PY{p}{,} \PY{n}{sigma\PYZus{}n}\PY{p}{)}
%\end{Verbatim}
%\end{tcolorbox}
%
%    \begin{tcolorbox}[breakable, size=fbox, boxrule=1pt, pad at break*=1mm,colback=cellbackground, colframe=cellborder]
%\prompt{In}{incolor}{19}{\boxspacing}
%\begin{Verbatim}[commandchars=\\\{\}]
%\PY{n}{plt}\PY{o}{.}\PY{n}{figure}\PY{p}{(}\PY{n}{figsize}\PY{o}{=}\PY{p}{(}\PY{l+m+mi}{8}\PY{p}{,} \PY{l+m+mi}{5}\PY{p}{)}\PY{p}{)}
%\PY{n}{plot\PYZus{}GP}\PY{p}{(}\PY{n}{X\PYZus{}test}\PY{p}{,} \PY{n}{mu}\PY{p}{,} \PY{n}{cov}\PY{p}{,} \PY{n}{X\PYZus{}init}\PY{p}{,} \PY{n}{Y\PYZus{}init}\PY{p}{,} \PY{n}{draw\PYZus{}ci}\PY{o}{=}\PY{k+kc}{True}\PY{p}{)}
%\PY{n}{plt}\PY{o}{.}\PY{n}{plot}\PY{p}{(}\PY{n}{X\PYZus{}test}\PY{p}{,} \PY{n}{Y\PYZus{}true}\PY{p}{,} \PY{l+s+s1}{\PYZsq{}}\PY{l+s+s1}{k\PYZhy{}\PYZhy{}}\PY{l+s+s1}{\PYZsq{}}\PY{p}{,} \PY{n}{label}\PY{o}{=}\PY{l+s+s1}{\PYZsq{}}\PY{l+s+s1}{Objective function}\PY{l+s+s1}{\PYZsq{}}\PY{p}{)}
%\PY{n}{plt}\PY{o}{.}\PY{n}{ylim}\PY{p}{(}\PY{l+m+mf}{1.1}\PY{o}{*}\PY{n}{Y\PYZus{}true}\PY{o}{.}\PY{n}{min}\PY{p}{(}\PY{p}{)}\PY{p}{,} \PY{l+m+mf}{1.1}\PY{o}{*}\PY{n}{Y\PYZus{}true}\PY{o}{.}\PY{n}{max}\PY{p}{(}\PY{p}{)}\PY{p}{)}
%\PY{n}{plt}\PY{o}{.}\PY{n}{legend}\PY{p}{(}\PY{n}{loc}\PY{o}{=}\PY{l+m+mi}{2}\PY{p}{)}
%\PY{n}{plt}\PY{o}{.}\PY{n}{tight\PYZus{}layout}\PY{p}{(}\PY{p}{)}
%\PY{n}{plt}\PY{o}{.}\PY{n}{show}\PY{p}{(}\PY{p}{)}
%\end{Verbatim}
%\end{tcolorbox}

    \begin{center}
    \adjustimage{max size={0.65\linewidth}{0.65\paperheight}}{Test1_GP_model.pdf}
    \end{center}
%    { \hspace*{\fill} \\}
    
    Функции \texttt{plot\_approximation} и \texttt{plot\_acquisition}
выводят графики суррогатной модели и функции продвижения.

%    \begin{tcolorbox}[breakable, size=fbox, boxrule=1pt, pad at break*=1mm,colback=cellbackground, colframe=cellborder]
%\prompt{In}{incolor}{20}{\boxspacing}
%\begin{Verbatim}[commandchars=\\\{\}]
%\PY{k}{def} \PY{n+nf}{plot\PYZus{}approximation}\PY{p}{(}\PY{n}{X}\PY{p}{,} \PY{n}{Y}\PY{p}{,} \PY{n}{X\PYZus{}train}\PY{p}{,} \PY{n}{Y\PYZus{}train}\PY{p}{,} \PY{n}{X\PYZus{}next}\PY{o}{=}\PY{k+kc}{None}\PY{p}{,} \PY{n}{show\PYZus{}legend}\PY{o}{=}\PY{k+kc}{False}\PY{p}{)}\PY{p}{:}
%    \PY{l+s+sd}{\PYZsq{}\PYZsq{}\PYZsq{}\PYZsq{}\PYZsq{}\PYZsq{}}
%    \PY{n}{mu}\PY{p}{,} \PY{n}{cov} \PY{o}{=} \PY{n}{GP\PYZus{}predictor}\PY{p}{(}\PY{n}{X}\PY{p}{,} \PY{n}{X\PYZus{}train}\PY{p}{,} \PY{n}{Y\PYZus{}train}\PY{p}{,}
%                           \PY{n}{kernel\PYZus{}fun}\PY{p}{,} \PY{n}{kernel\PYZus{}args}\PY{p}{,} \PY{n}{sigma\PYZus{}n}\PY{p}{)}
%    \PY{n}{std} \PY{o}{=} \PY{n}{np}\PY{o}{.}\PY{n}{sqrt}\PY{p}{(}\PY{n}{np}\PY{o}{.}\PY{n}{diag}\PY{p}{(}\PY{n}{cov}\PY{p}{)}\PY{p}{)}\PY{o}{.}\PY{n}{flatten}\PY{p}{(}\PY{p}{)}
%    \PY{n}{plt}\PY{o}{.}\PY{n}{fill\PYZus{}between}\PY{p}{(}\PY{n}{X}\PY{o}{.}\PY{n}{flatten}\PY{p}{(}\PY{p}{)}\PY{p}{,} \PY{n}{mu}\PY{o}{.}\PY{n}{flatten}\PY{p}{(}\PY{p}{)}\PY{o}{+}\PY{l+m+mi}{2}\PY{o}{*}\PY{n}{std}\PY{p}{,} \PY{n}{mu}\PY{o}{.}\PY{n}{flatten}\PY{p}{(}\PY{p}{)}\PY{o}{\PYZhy{}}\PY{l+m+mi}{2}\PY{o}{*}\PY{n}{std}\PY{p}{,}
%                     \PY{n}{color}\PY{o}{=}\PY{n}{cm}\PY{o}{.}\PY{n}{tab10}\PY{p}{(}\PY{l+m+mi}{4}\PY{p}{)}\PY{p}{,} \PY{n}{alpha}\PY{o}{=}\PY{l+m+mf}{0.1}\PY{p}{)} 
%    \PY{n}{plt}\PY{o}{.}\PY{n}{plot}\PY{p}{(}\PY{n}{X}\PY{p}{,} \PY{n}{Y}\PY{p}{,} \PY{l+s+s1}{\PYZsq{}}\PY{l+s+s1}{k\PYZhy{}\PYZhy{}}\PY{l+s+s1}{\PYZsq{}}\PY{p}{,} \PY{n}{label}\PY{o}{=}\PY{l+s+s1}{\PYZsq{}}\PY{l+s+s1}{Objective function}\PY{l+s+s1}{\PYZsq{}}\PY{p}{)}
%    \PY{n}{plt}\PY{o}{.}\PY{n}{plot}\PY{p}{(}\PY{n}{X}\PY{p}{,} \PY{n}{mu}\PY{p}{,} \PY{l+s+s1}{\PYZsq{}}\PY{l+s+s1}{\PYZhy{}}\PY{l+s+s1}{\PYZsq{}}\PY{p}{,} \PY{n}{c}\PY{o}{=}\PY{n}{cm}\PY{o}{.}\PY{n}{tab10}\PY{p}{(}\PY{l+m+mi}{0}\PY{p}{)}\PY{p}{,} \PY{n}{label}\PY{o}{=}\PY{l+s+s1}{\PYZsq{}}\PY{l+s+s1}{Surrogate function}\PY{l+s+s1}{\PYZsq{}}\PY{p}{)}
%    \PY{n}{plt}\PY{o}{.}\PY{n}{plot}\PY{p}{(}\PY{n}{X\PYZus{}train}\PY{p}{,} \PY{n}{Y\PYZus{}train}\PY{p}{,} \PY{l+s+s1}{\PYZsq{}}\PY{l+s+s1}{o}\PY{l+s+s1}{\PYZsq{}}\PY{p}{,} \PY{n}{ms}\PY{o}{=}\PY{l+m+mi}{7}\PY{p}{,} \PY{n}{c}\PY{o}{=}\PY{n}{cm}\PY{o}{.}\PY{n}{tab10}\PY{p}{(}\PY{l+m+mi}{3}\PY{p}{)}\PY{p}{,} \PY{n}{label}\PY{o}{=}\PY{l+s+s1}{\PYZsq{}}\PY{l+s+s1}{Train samples}\PY{l+s+s1}{\PYZsq{}}\PY{p}{)}
%    \PY{n}{i\PYZus{}min} \PY{o}{=} \PY{n}{np}\PY{o}{.}\PY{n}{argmin}\PY{p}{(}\PY{n}{Y\PYZus{}train}\PY{p}{)}
%    \PY{n}{plt}\PY{o}{.}\PY{n}{plot}\PY{p}{(}\PY{n}{X\PYZus{}train}\PY{p}{[}\PY{n}{i\PYZus{}min}\PY{p}{]}\PY{p}{,}\PY{n}{Y\PYZus{}train}\PY{p}{[}\PY{n}{i\PYZus{}min}\PY{p}{]}\PY{p}{,}\PY{l+s+s1}{\PYZsq{}}\PY{l+s+s1}{*}\PY{l+s+s1}{\PYZsq{}}\PY{p}{,}\PY{n}{ms}\PY{o}{=}\PY{l+m+mi}{15}\PY{p}{,}\PY{n}{c}\PY{o}{=}\PY{n}{cm}\PY{o}{.}\PY{n}{tab10}\PY{p}{(}\PY{l+m+mi}{3}\PY{p}{)}\PY{p}{)}
%    \PY{k}{if} \PY{n}{X\PYZus{}next}\PY{p}{:}
%        \PY{n}{plt}\PY{o}{.}\PY{n}{axvline}\PY{p}{(}\PY{n}{x}\PY{o}{=}\PY{n}{X\PYZus{}next}\PY{p}{,} \PY{n}{ls}\PY{o}{=}\PY{l+s+s1}{\PYZsq{}}\PY{l+s+s1}{:}\PY{l+s+s1}{\PYZsq{}}\PY{p}{,} \PY{n}{c}\PY{o}{=}\PY{l+s+s1}{\PYZsq{}}\PY{l+s+s1}{k}\PY{l+s+s1}{\PYZsq{}}\PY{p}{)}
%    \PY{k}{if} \PY{n}{show\PYZus{}legend}\PY{p}{:} \PY{n}{plt}\PY{o}{.}\PY{n}{legend}\PY{p}{(}\PY{n}{loc}\PY{o}{=}\PY{l+m+mi}{2}\PY{p}{)}
%
%\PY{k}{def} \PY{n+nf}{plot\PYZus{}acquisition}\PY{p}{(}\PY{n}{X}\PY{p}{,} \PY{n}{Y}\PY{p}{,} \PY{n}{X\PYZus{}next}\PY{p}{,} \PY{n}{show\PYZus{}legend}\PY{o}{=}\PY{k+kc}{False}\PY{p}{)}\PY{p}{:}
%    \PY{l+s+sd}{\PYZsq{}\PYZsq{}\PYZsq{}\PYZsq{}\PYZsq{}\PYZsq{}}
%    \PY{n}{plt}\PY{o}{.}\PY{n}{plot}\PY{p}{(}\PY{n}{X}\PY{p}{,} \PY{n}{Y}\PY{p}{,} \PY{l+s+s1}{\PYZsq{}}\PY{l+s+s1}{\PYZhy{}}\PY{l+s+s1}{\PYZsq{}}\PY{p}{,} \PY{n}{c}\PY{o}{=}\PY{n}{cm}\PY{o}{.}\PY{n}{tab10}\PY{p}{(}\PY{l+m+mi}{3}\PY{p}{)}\PY{p}{,} \PY{n}{label}\PY{o}{=}\PY{l+s+s1}{\PYZsq{}}\PY{l+s+s1}{Acquisition function}\PY{l+s+s1}{\PYZsq{}}\PY{p}{)}
%    \PY{n}{plt}\PY{o}{.}\PY{n}{axvline}\PY{p}{(}\PY{n}{X\PYZus{}next}\PY{p}{,} \PY{n}{ls}\PY{o}{=}\PY{l+s+s1}{\PYZsq{}}\PY{l+s+s1}{:}\PY{l+s+s1}{\PYZsq{}}\PY{p}{,} \PY{n}{c}\PY{o}{=}\PY{l+s+s1}{\PYZsq{}}\PY{l+s+s1}{k}\PY{l+s+s1}{\PYZsq{}}\PY{p}{,} \PY{n}{label}\PY{o}{=}\PY{l+s+s1}{\PYZsq{}}\PY{l+s+s1}{Next point}\PY{l+s+s1}{\PYZsq{}}\PY{p}{)}
%    \PY{k}{if} \PY{n}{show\PYZus{}legend}\PY{p}{:} \PY{n}{plt}\PY{o}{.}\PY{n}{legend}\PY{p}{(}\PY{n}{loc}\PY{o}{=}\PY{l+m+mi}{1}\PY{p}{)}    
%\end{Verbatim}
%\end{tcolorbox}

    Запустим алгоритм байесовской оптимизации.

Обучающая выборка для суррогатной модели содержится в переменных
\texttt{X\_train} и \texttt{Y\_train} и обновляется на каждой итерации.
Переменная \texttt{N\_budget} задаёт максимальное количество итераций
(бюджет оптимизации).

%    \begin{tcolorbox}[breakable, size=fbox, boxrule=1pt, pad at break*=1mm,colback=cellbackground, colframe=cellborder]
%\prompt{In}{incolor}{21}{\boxspacing}
%\begin{Verbatim}[commandchars=\\\{\}]
%\PY{c+c1}{\PYZsh{} Choose acquisition function}
%\PY{n}{acqusition\PYZus{}id} \PY{o}{=} \PY{l+s+s1}{\PYZsq{}}\PY{l+s+s1}{EI}\PY{l+s+s1}{\PYZsq{}} \PY{c+c1}{\PYZsh{} \PYZsq{}LB\PYZsq{}, \PYZsq{}PI\PYZsq{}, \PYZsq{}EI\PYZsq{}}
%
%\PY{c+c1}{\PYZsh{} Hyperparameters}
%\PY{n}{kernel\PYZus{}args} \PY{o}{=} \PY{p}{\PYZob{}}\PY{l+s+s1}{\PYZsq{}}\PY{l+s+s1}{l}\PY{l+s+s1}{\PYZsq{}}\PY{p}{:}\PY{l+m+mf}{0.2}\PY{p}{,} \PY{l+s+s1}{\PYZsq{}}\PY{l+s+s1}{sigma\PYZus{}k}\PY{l+s+s1}{\PYZsq{}}\PY{p}{:}\PY{l+m+mf}{2.0}\PY{p}{\PYZcb{}}
%\PY{n}{sigma\PYZus{}n} \PY{o}{=} \PY{l+m+mf}{1e\PYZhy{}3}
%
%\PY{c+c1}{\PYZsh{} Number of iterations}
%\PY{n}{N\PYZus{}budget} \PY{o}{=} \PY{l+m+mi}{21} \PY{o}{\PYZhy{}} \PY{n}{N\PYZus{}init}
%
%\PY{c+c1}{\PYZsh{} acquisition function}
%\PY{k}{def} \PY{n+nf}{acq\PYZus{}function}\PY{p}{(}\PY{n}{X}\PY{p}{,} \PY{n}{X\PYZus{}train}\PY{p}{,} \PY{n}{Y\PYZus{}train}\PY{p}{)}\PY{p}{:}
%    \PY{k}{if}   \PY{n}{acqusition\PYZus{}id} \PY{o}{==} \PY{l+s+s1}{\PYZsq{}}\PY{l+s+s1}{LB}\PY{l+s+s1}{\PYZsq{}}\PY{p}{:}
%        \PY{k}{return} \PY{n}{lower\PYZus{}bound}\PY{p}{(}\PY{n}{X}\PY{p}{,} \PY{n}{X\PYZus{}train}\PY{p}{,} \PY{n}{Y\PYZus{}train}\PY{p}{,} \PY{n}{k}\PY{o}{=}\PY{l+m+mf}{2.}\PY{p}{)}
%    \PY{k}{elif} \PY{n}{acqusition\PYZus{}id} \PY{o}{==} \PY{l+s+s1}{\PYZsq{}}\PY{l+s+s1}{PI}\PY{l+s+s1}{\PYZsq{}}\PY{p}{:}
%        \PY{k}{return} \PY{n}{probability\PYZus{}of\PYZus{}improvement}\PY{p}{(}\PY{n}{X}\PY{p}{,} \PY{n}{X\PYZus{}train}\PY{p}{,} \PY{n}{Y\PYZus{}train}\PY{p}{,} \PY{n}{xi}\PY{o}{=}\PY{l+m+mf}{0.1}\PY{p}{)}
%    \PY{k}{elif} \PY{n}{acqusition\PYZus{}id} \PY{o}{==} \PY{l+s+s1}{\PYZsq{}}\PY{l+s+s1}{EI}\PY{l+s+s1}{\PYZsq{}}\PY{p}{:}
%        \PY{k}{return} \PY{n}{expected\PYZus{}improvement}\PY{p}{(}\PY{n}{X}\PY{p}{,} \PY{n}{X\PYZus{}train}\PY{p}{,} \PY{n}{Y\PYZus{}train}\PY{p}{,} \PY{n}{xi}\PY{o}{=}\PY{l+m+mf}{0.1}\PY{p}{)}
%\end{Verbatim}
%\end{tcolorbox}

%    \begin{tcolorbox}[breakable, size=fbox, boxrule=1pt, pad at break*=1mm,colback=cellbackground, colframe=cellborder]
%\prompt{In}{incolor}{22}{\boxspacing}
%\begin{Verbatim}[commandchars=\\\{\}]
%\PY{c+c1}{\PYZsh{} Initialize samples}
%\PY{n}{X\PYZus{}train} \PY{o}{=} \PY{n}{X\PYZus{}init}
%\PY{n}{Y\PYZus{}train} \PY{o}{=} \PY{n}{f\PYZus{}obj}\PY{p}{(}\PY{n}{X\PYZus{}init}\PY{p}{)}
%\end{Verbatim}
%\end{tcolorbox}

%    \begin{tcolorbox}[breakable, size=fbox, boxrule=1pt, pad at break*=1mm,colback=cellbackground, colframe=cellborder]
%\prompt{In}{incolor}{23}{\boxspacing}
%\begin{Verbatim}[commandchars=\\\{\}]
%\PY{n}{plt}\PY{o}{.}\PY{n}{figure}\PY{p}{(}\PY{n}{figsize}\PY{o}{=}\PY{p}{(}\PY{l+m+mi}{16}\PY{p}{,} \PY{n}{N\PYZus{}budget}\PY{o}{*}\PY{l+m+mi}{5}\PY{p}{)}\PY{p}{)}
%\PY{n}{plt}\PY{o}{.}\PY{n}{subplots\PYZus{}adjust}\PY{p}{(}\PY{n}{hspace}\PY{o}{=}\PY{l+m+mf}{0.4}\PY{p}{)}
%
%\PY{k}{for} \PY{n}{i} \PY{o+ow}{in} \PY{n+nb}{range}\PY{p}{(}\PY{n}{N\PYZus{}budget}\PY{p}{)}\PY{p}{:}   
%    \PY{c+c1}{\PYZsh{} Obtain next sampling point from the acquisition function}
%    \PY{n}{X\PYZus{}next}\PY{p}{,} \PY{n}{acq} \PY{o}{=} \PY{n}{next\PYZus{}point}\PY{p}{(}\PY{n}{acq\PYZus{}function}\PY{p}{,} \PY{n}{X\PYZus{}train}\PY{p}{,} \PY{n}{Y\PYZus{}train}\PY{p}{,} \PY{n}{x\PYZus{}lims}\PY{p}{)}
%    
%    \PY{c+c1}{\PYZsh{} Obtain next sample from the objective function}
%    \PY{n}{Y\PYZus{}next} \PY{o}{=} \PY{n}{f\PYZus{}obj}\PY{p}{(}\PY{n}{X\PYZus{}next}\PY{p}{)}
%    
%    \PY{c+c1}{\PYZsh{} Plot samples, surrogate function, noise\PYZhy{}free objective and next sampling location}
%    \PY{n}{ax} \PY{o}{=} \PY{n}{plt}\PY{o}{.}\PY{n}{subplot}\PY{p}{(}\PY{n}{N\PYZus{}budget}\PY{p}{,} \PY{l+m+mi}{2}\PY{p}{,} \PY{l+m+mi}{2}\PY{o}{*}\PY{n}{i} \PY{o}{+} \PY{l+m+mi}{1}\PY{p}{)}
%    \PY{n}{plot\PYZus{}approximation}\PY{p}{(}\PY{n}{X\PYZus{}test}\PY{p}{,} \PY{n}{Y\PYZus{}true}\PY{p}{,} \PY{n}{X\PYZus{}train}\PY{p}{,} \PY{n}{Y\PYZus{}train}\PY{p}{,} \PY{n}{X\PYZus{}next}\PY{p}{,} \PY{n}{show\PYZus{}legend}\PY{o}{=}\PY{p}{(}\PY{n}{i}\PY{o}{==}\PY{l+m+mi}{0}\PY{p}{)}\PY{p}{)}
%    \PY{n}{plt}\PY{o}{.}\PY{n}{title}\PY{p}{(}\PY{l+s+sa}{f}\PY{l+s+s1}{\PYZsq{}}\PY{l+s+s1}{Iteration }\PY{l+s+si}{\PYZob{}}\PY{n}{i}\PY{o}{+}\PY{n}{N\PYZus{}init}\PY{l+s+si}{\PYZcb{}}\PY{l+s+s1}{, X\PYZus{}next = }\PY{l+s+si}{\PYZob{}}\PY{n}{X\PYZus{}next}\PY{p}{[}\PY{l+m+mi}{0}\PY{p}{]}\PY{p}{[}\PY{l+m+mi}{0}\PY{p}{]}\PY{l+s+si}{:}\PY{l+s+s1}{.3f}\PY{l+s+si}{\PYZcb{}}\PY{l+s+s1}{\PYZsq{}}\PY{p}{)}
%
%    \PY{n}{plt}\PY{o}{.}\PY{n}{subplot}\PY{p}{(}\PY{n}{N\PYZus{}budget}\PY{p}{,} \PY{l+m+mi}{2}\PY{p}{,} \PY{l+m+mi}{2}\PY{o}{*}\PY{n}{i} \PY{o}{+} \PY{l+m+mi}{2}\PY{p}{)}
%    \PY{n}{Y\PYZus{}acq} \PY{o}{=} \PY{n}{acq\PYZus{}function}\PY{p}{(}\PY{n}{X\PYZus{}test}\PY{p}{,} \PY{n}{X\PYZus{}train}\PY{p}{,} \PY{n}{Y\PYZus{}train}\PY{p}{)}
%    \PY{n}{plot\PYZus{}acquisition}\PY{p}{(}\PY{n}{X\PYZus{}test}\PY{p}{,} \PY{n}{Y\PYZus{}acq}\PY{p}{,} \PY{n}{X\PYZus{}next}\PY{p}{,} \PY{n}{show\PYZus{}legend}\PY{o}{=}\PY{p}{(}\PY{n}{i}\PY{o}{==}\PY{l+m+mi}{0}\PY{p}{)}\PY{p}{)}
%    \PY{n}{plt}\PY{o}{.}\PY{n}{title}\PY{p}{(}\PY{l+s+sa}{f}\PY{l+s+s1}{\PYZsq{}}\PY{l+s+s1}{AF\PYZus{}max = }\PY{l+s+si}{\PYZob{}}\PY{o}{\PYZhy{}}\PY{n}{acq}\PY{l+s+si}{:}\PY{l+s+s1}{.3g}\PY{l+s+si}{\PYZcb{}}\PY{l+s+s1}{\PYZsq{}}\PY{p}{)}
%    
%    \PY{c+c1}{\PYZsh{} Add a new sample to train samples}
%    \PY{n}{X\PYZus{}train} \PY{o}{=} \PY{n}{np}\PY{o}{.}\PY{n}{vstack}\PY{p}{(}\PY{p}{(}\PY{n}{X\PYZus{}train}\PY{p}{,} \PY{n}{X\PYZus{}next}\PY{p}{)}\PY{p}{)}
%    \PY{n}{Y\PYZus{}train} \PY{o}{=} \PY{n}{np}\PY{o}{.}\PY{n}{vstack}\PY{p}{(}\PY{p}{(}\PY{n}{Y\PYZus{}train}\PY{p}{,} \PY{n}{Y\PYZus{}next}\PY{p}{)}\PY{p}{)}
%    
%    \PY{n+nb}{print}\PY{p}{(}\PY{n}{i}\PY{o}{+}\PY{n}{N\PYZus{}init}\PY{p}{,} \PY{o}{*}\PY{n}{X\PYZus{}next}\PY{p}{,} \PY{o}{\PYZhy{}}\PY{n}{acq}\PY{p}{)}
%    \PY{k}{if} \PY{p}{(}\PY{o}{\PYZhy{}}\PY{n}{acq} \PY{o}{\PYZlt{}} \PY{l+m+mf}{1e\PYZhy{}100}\PY{p}{)} \PY{o+ow}{or} \PY{p}{(}\PY{n+nb}{abs}\PY{p}{(}\PY{n}{X\PYZus{}train}\PY{p}{[}\PY{o}{\PYZhy{}}\PY{l+m+mi}{2}\PY{p}{]}\PY{o}{\PYZhy{}}\PY{n}{X\PYZus{}next}\PY{p}{)}\PY{p}{[}\PY{l+m+mi}{0}\PY{p}{,}\PY{l+m+mi}{0}\PY{p}{]} \PY{o}{\PYZlt{}} \PY{l+m+mf}{1e\PYZhy{}3}\PY{o}{*}\PY{n}{x\PYZus{}range}\PY{p}{)}\PY{p}{:}
%        \PY{k}{break}
%\end{Verbatim}
%\end{tcolorbox}

%    \begin{Verbatim}[commandchars=\\\{\}]
%3 [0.344] 0.48228021979931546
%4 [0.254] 0.02438780852034747
%5 [0.652] 3.1369267032338493e-06
%6 [0.682] 0.15095921320242822
%7 [0.77] 2.6940422398150687
%8 [0.758] 0.04952640807326861
%9 [0.756] 1.5331698309352136e-116
%    \end{Verbatim}

    \begin{center}
    \adjustimage{max size={0.99\linewidth}{0.99\paperheight}}{Test1_EGO.pdf}
    \end{center}
%    { \hspace*{\fill} \\}
    
%    \begin{center}\rule{0.5\linewidth}{0.5pt}\end{center}

    \hypertarget{ux442ux435ux441ux442-2.-ux448ux443ux43cux43dux44bux435-ux434ux430ux43dux43dux44bux435}{%
\section{Тест 2. Шумные
данные}\label{ux442ux435ux441ux442-2.-ux448ux443ux43cux43dux44bux435-ux434ux430ux43dux43dux44bux435}}

    Теперь добавим к целевой функции шум: \[
  f_{obj} = (6x-2)^2 \cdot \sin\left(12x-3\right) + \sigma_{in} \xi.
\]

Здесь \(\xi\) --- нормальная случайная величина, переменная
\(\sigma_{in}\) задаёт амплитуду шума.

    \hypertarget{ux434ux430ux43dux43dux44bux435}{%
\subsection{Данные}\label{ux434ux430ux43dux43dux44bux435}}

%    \begin{tcolorbox}[breakable, size=fbox, boxrule=1pt, pad at break*=1mm,colback=cellbackground, colframe=cellborder]
%\prompt{In}{incolor}{24}{\boxspacing}
%\begin{Verbatim}[commandchars=\\\{\}]
%\PY{c+c1}{\PYZsh{} Objective function}
%\PY{k}{def} \PY{n+nf}{f\PYZus{}obj\PYZus{}noisy}\PY{p}{(}\PY{n}{X}\PY{p}{,} \PY{n}{sigma\PYZus{}in}\PY{p}{)}\PY{p}{:}
%    \PY{n}{xi} \PY{o}{=} \PY{n}{np}\PY{o}{.}\PY{n}{random}\PY{o}{.}\PY{n}{randn}\PY{p}{(}\PY{o}{*}\PY{n}{X}\PY{o}{.}\PY{n}{shape}\PY{p}{)}
%    \PY{k}{return} \PY{p}{(}\PY{l+m+mi}{6}\PY{o}{*}\PY{n}{X}\PY{o}{\PYZhy{}}\PY{l+m+mi}{2}\PY{p}{)}\PY{o}{*}\PY{o}{*}\PY{l+m+mi}{2} \PY{o}{*} \PY{n}{np}\PY{o}{.}\PY{n}{sin}\PY{p}{(}\PY{l+m+mi}{12}\PY{o}{*}\PY{n}{X}\PY{o}{\PYZhy{}}\PY{l+m+mi}{4}\PY{p}{)} \PY{o}{+} \PY{n}{sigma\PYZus{}in} \PY{o}{*} \PY{n}{xi}
%\end{Verbatim}
%\end{tcolorbox}
%
%    \begin{tcolorbox}[breakable, size=fbox, boxrule=1pt, pad at break*=1mm,colback=cellbackground, colframe=cellborder]
%\prompt{In}{incolor}{25}{\boxspacing}
%\begin{Verbatim}[commandchars=\\\{\}]
%\PY{c+c1}{\PYZsh{} Objective function values at X\PYZus{}test}
%\PY{n}{sigma\PYZus{}in} \PY{o}{=} \PY{l+m+mf}{0.5}
%\PY{n}{Y\PYZus{}noisy} \PY{o}{=} \PY{n}{f\PYZus{}obj\PYZus{}noisy}\PY{p}{(}\PY{n}{X\PYZus{}test}\PY{p}{,} \PY{n}{sigma\PYZus{}in}\PY{p}{)}
%\end{Verbatim}
%\end{tcolorbox}
%
%    \begin{tcolorbox}[breakable, size=fbox, boxrule=1pt, pad at break*=1mm,colback=cellbackground, colframe=cellborder]
%\prompt{In}{incolor}{26}{\boxspacing}
%\begin{Verbatim}[commandchars=\\\{\}]
%\PY{c+c1}{\PYZsh{} Plot optimization objective}
%\PY{n}{plt}\PY{o}{.}\PY{n}{figure}\PY{p}{(}\PY{n}{figsize}\PY{o}{=}\PY{p}{(}\PY{l+m+mi}{8}\PY{p}{,} \PY{l+m+mi}{5}\PY{p}{)}\PY{p}{)}
%\PY{n}{plt}\PY{o}{.}\PY{n}{plot}\PY{p}{(}\PY{n}{X\PYZus{}test}\PY{p}{,} \PY{n}{Y\PYZus{}true}\PY{p}{,}  \PY{l+s+s1}{\PYZsq{}}\PY{l+s+s1}{k\PYZhy{}\PYZhy{}}\PY{l+s+s1}{\PYZsq{}}\PY{p}{,} \PY{n}{label}\PY{o}{=}\PY{l+s+s1}{\PYZsq{}}\PY{l+s+s1}{True objective}\PY{l+s+s1}{\PYZsq{}}\PY{p}{)}
%\PY{n}{plt}\PY{o}{.}\PY{n}{plot}\PY{p}{(}\PY{n}{X\PYZus{}test}\PY{p}{,} \PY{n}{Y\PYZus{}noisy}\PY{p}{,} \PY{l+s+s1}{\PYZsq{}}\PY{l+s+s1}{kx}\PY{l+s+s1}{\PYZsq{}}\PY{p}{,} \PY{n}{alpha}\PY{o}{=}\PY{l+m+mf}{.3}\PY{p}{,} \PY{n}{label}\PY{o}{=}\PY{l+s+s1}{\PYZsq{}}\PY{l+s+s1}{Noisy objective}\PY{l+s+s1}{\PYZsq{}}\PY{p}{)}
%\PY{n}{plt}\PY{o}{.}\PY{n}{legend}\PY{p}{(}\PY{p}{)}
%\PY{n}{plt}\PY{o}{.}\PY{n}{tight\PYZus{}layout}\PY{p}{(}\PY{p}{)}
%\PY{n}{plt}\PY{o}{.}\PY{n}{show}\PY{p}{(}\PY{p}{)}
%\end{Verbatim}
%\end{tcolorbox}

    \begin{center}
    \adjustimage{max size={0.65\linewidth}{0.65\paperheight}}{Test2_objective.pdf}
    \end{center}
%    { \hspace*{\fill} \\}
    
    \hypertarget{ux431ux430ux439ux435ux441ux43eux432ux441ux43aux430ux44f-ux43eux43fux442ux438ux43cux438ux437ux430ux446ux438ux44f}{%
\subsection{Байесовская
оптимизация}\label{ux431ux430ux439ux435ux441ux43eux432ux441ux43aux430ux44f-ux43eux43fux442ux438ux43cux438ux437ux430ux446ux438ux44f}}

    Требуется немного изменить функцию \texttt{plot\_approximation}.

%    \begin{tcolorbox}[breakable, size=fbox, boxrule=1pt, pad at break*=1mm,colback=cellbackground, colframe=cellborder]
%\prompt{In}{incolor}{27}{\boxspacing}
%\begin{Verbatim}[commandchars=\\\{\}]
%\PY{k}{def} \PY{n+nf}{plot\PYZus{}approximation}\PY{p}{(}\PY{n}{X}\PY{p}{,} \PY{n}{Y}\PY{p}{,} \PY{n}{X\PYZus{}train}\PY{p}{,} \PY{n}{Y\PYZus{}train}\PY{p}{,} \PY{n}{X\PYZus{}next}\PY{o}{=}\PY{k+kc}{None}\PY{p}{,} \PY{n}{show\PYZus{}legend}\PY{o}{=}\PY{k+kc}{False}\PY{p}{)}\PY{p}{:}
%    \PY{l+s+sd}{\PYZsq{}\PYZsq{}\PYZsq{}\PYZsq{}\PYZsq{}\PYZsq{}}
%    \PY{n}{mu}\PY{p}{,} \PY{n}{cov} \PY{o}{=} \PY{n}{GP\PYZus{}predictor}\PY{p}{(}\PY{n}{X}\PY{p}{,} \PY{n}{X\PYZus{}train}\PY{p}{,} \PY{n}{Y\PYZus{}train}\PY{p}{,}
%                           \PY{n}{kernel\PYZus{}fun}\PY{p}{,} \PY{n}{kernel\PYZus{}args}\PY{p}{,} \PY{n}{sigma\PYZus{}n}\PY{p}{)}
%    \PY{n}{std} \PY{o}{=} \PY{n}{np}\PY{o}{.}\PY{n}{sqrt}\PY{p}{(}\PY{n}{np}\PY{o}{.}\PY{n}{diag}\PY{p}{(}\PY{n}{cov}\PY{p}{)}\PY{p}{)}\PY{o}{.}\PY{n}{flatten}\PY{p}{(}\PY{p}{)}
%    \PY{n}{plt}\PY{o}{.}\PY{n}{fill\PYZus{}between}\PY{p}{(}\PY{n}{X}\PY{o}{.}\PY{n}{flatten}\PY{p}{(}\PY{p}{)}\PY{p}{,} \PY{n}{mu}\PY{o}{.}\PY{n}{flatten}\PY{p}{(}\PY{p}{)}\PY{o}{+}\PY{l+m+mi}{2}\PY{o}{*}\PY{n}{std}\PY{p}{,} \PY{n}{mu}\PY{o}{.}\PY{n}{flatten}\PY{p}{(}\PY{p}{)}\PY{o}{\PYZhy{}}\PY{l+m+mi}{2}\PY{o}{*}\PY{n}{std}\PY{p}{,}
%                     \PY{n}{color}\PY{o}{=}\PY{n}{cm}\PY{o}{.}\PY{n}{tab10}\PY{p}{(}\PY{l+m+mi}{4}\PY{p}{)}\PY{p}{,} \PY{n}{alpha}\PY{o}{=}\PY{l+m+mf}{0.1}\PY{p}{)}
%    \PY{n}{plt}\PY{o}{.}\PY{n}{plot}\PY{p}{(}\PY{n}{X}\PY{p}{,} \PY{n}{Y\PYZus{}true}\PY{p}{,} \PY{l+s+s1}{\PYZsq{}}\PY{l+s+s1}{k\PYZhy{}\PYZhy{}}\PY{l+s+s1}{\PYZsq{}}\PY{p}{,} \PY{n}{label}\PY{o}{=}\PY{l+s+s1}{\PYZsq{}}\PY{l+s+s1}{True objective}\PY{l+s+s1}{\PYZsq{}}\PY{p}{)}
%    \PY{n}{plt}\PY{o}{.}\PY{n}{plot}\PY{p}{(}\PY{n}{X}\PY{p}{,} \PY{n}{Y}\PY{p}{,} \PY{l+s+s1}{\PYZsq{}}\PY{l+s+s1}{kx}\PY{l+s+s1}{\PYZsq{}}\PY{p}{,} \PY{n}{alpha}\PY{o}{=}\PY{l+m+mf}{.2}\PY{p}{,} \PY{n}{label}\PY{o}{=}\PY{l+s+s1}{\PYZsq{}}\PY{l+s+s1}{Noisy objective}\PY{l+s+s1}{\PYZsq{}}\PY{p}{)}
%    \PY{n}{plt}\PY{o}{.}\PY{n}{plot}\PY{p}{(}\PY{n}{X}\PY{p}{,} \PY{n}{mu}\PY{p}{,} \PY{l+s+s1}{\PYZsq{}}\PY{l+s+s1}{\PYZhy{}}\PY{l+s+s1}{\PYZsq{}}\PY{p}{,} \PY{n}{c}\PY{o}{=}\PY{n}{cm}\PY{o}{.}\PY{n}{tab10}\PY{p}{(}\PY{l+m+mi}{0}\PY{p}{)}\PY{p}{,} \PY{n}{label}\PY{o}{=}\PY{l+s+s1}{\PYZsq{}}\PY{l+s+s1}{Surrogate model}\PY{l+s+s1}{\PYZsq{}}\PY{p}{)}
%    \PY{n}{plt}\PY{o}{.}\PY{n}{plot}\PY{p}{(}\PY{n}{X\PYZus{}train}\PY{p}{,} \PY{n}{Y\PYZus{}train}\PY{p}{,} \PY{l+s+s1}{\PYZsq{}}\PY{l+s+s1}{o}\PY{l+s+s1}{\PYZsq{}}\PY{p}{,} \PY{n}{ms}\PY{o}{=}\PY{l+m+mi}{7}\PY{p}{,} \PY{n}{c}\PY{o}{=}\PY{n}{cm}\PY{o}{.}\PY{n}{tab10}\PY{p}{(}\PY{l+m+mi}{3}\PY{p}{)}\PY{p}{,} \PY{n}{label}\PY{o}{=}\PY{l+s+s1}{\PYZsq{}}\PY{l+s+s1}{Train samples}\PY{l+s+s1}{\PYZsq{}}\PY{p}{)}
%    \PY{n}{i\PYZus{}min} \PY{o}{=} \PY{n}{np}\PY{o}{.}\PY{n}{argmin}\PY{p}{(}\PY{n}{Y\PYZus{}train}\PY{p}{)}
%    \PY{n}{plt}\PY{o}{.}\PY{n}{plot}\PY{p}{(}\PY{n}{X\PYZus{}train}\PY{p}{[}\PY{n}{i\PYZus{}min}\PY{p}{]}\PY{p}{,}\PY{n}{Y\PYZus{}train}\PY{p}{[}\PY{n}{i\PYZus{}min}\PY{p}{]}\PY{p}{,}\PY{l+s+s1}{\PYZsq{}}\PY{l+s+s1}{*}\PY{l+s+s1}{\PYZsq{}}\PY{p}{,}\PY{n}{ms}\PY{o}{=}\PY{l+m+mi}{15}\PY{p}{,}\PY{n}{c}\PY{o}{=}\PY{n}{cm}\PY{o}{.}\PY{n}{tab10}\PY{p}{(}\PY{l+m+mi}{3}\PY{p}{)}\PY{p}{)}
%    \PY{k}{if} \PY{n}{X\PYZus{}next}\PY{p}{:}
%        \PY{n}{plt}\PY{o}{.}\PY{n}{axvline}\PY{p}{(}\PY{n}{x}\PY{o}{=}\PY{n}{X\PYZus{}next}\PY{p}{,} \PY{n}{ls}\PY{o}{=}\PY{l+s+s1}{\PYZsq{}}\PY{l+s+s1}{:}\PY{l+s+s1}{\PYZsq{}}\PY{p}{,} \PY{n}{c}\PY{o}{=}\PY{l+s+s1}{\PYZsq{}}\PY{l+s+s1}{k}\PY{l+s+s1}{\PYZsq{}}\PY{p}{)}
%    \PY{k}{if} \PY{n}{show\PYZus{}legend}\PY{p}{:} \PY{n}{plt}\PY{o}{.}\PY{n}{legend}\PY{p}{(}\PY{n}{loc}\PY{o}{=}\PY{l+m+mi}{2}\PY{p}{)}
%\end{Verbatim}
%\end{tcolorbox}

    Запуск алгоритма.

%    \begin{tcolorbox}[breakable, size=fbox, boxrule=1pt, pad at break*=1mm,colback=cellbackground, colframe=cellborder]
%\prompt{In}{incolor}{28}{\boxspacing}
%\begin{Verbatim}[commandchars=\\\{\}]
%\PY{c+c1}{\PYZsh{} Choose acquisition function}
%\PY{n}{acqusition\PYZus{}id} \PY{o}{=} \PY{l+s+s1}{\PYZsq{}}\PY{l+s+s1}{EI}\PY{l+s+s1}{\PYZsq{}} \PY{c+c1}{\PYZsh{} \PYZsq{}LB\PYZsq{}, \PYZsq{}PI\PYZsq{}, \PYZsq{}EI\PYZsq{}}
%
%\PY{c+c1}{\PYZsh{} Hyperparameters}
%\PY{n}{kernel\PYZus{}args} \PY{o}{=} \PY{p}{\PYZob{}}\PY{l+s+s1}{\PYZsq{}}\PY{l+s+s1}{l}\PY{l+s+s1}{\PYZsq{}}\PY{p}{:}\PY{l+m+mf}{0.2}\PY{p}{,} \PY{l+s+s1}{\PYZsq{}}\PY{l+s+s1}{sigma\PYZus{}k}\PY{l+s+s1}{\PYZsq{}}\PY{p}{:}\PY{l+m+mf}{2.0}\PY{p}{\PYZcb{}}
%\PY{n}{sigma\PYZus{}n} \PY{o}{=} \PY{l+m+mf}{1e\PYZhy{}1}
%
%\PY{c+c1}{\PYZsh{} Number of iterations}
%\PY{n}{N\PYZus{}budget} \PY{o}{=} \PY{l+m+mi}{21} \PY{o}{\PYZhy{}} \PY{n}{N\PYZus{}init}
%
%\PY{c+c1}{\PYZsh{} acquisition function}
%\PY{k}{def} \PY{n+nf}{acq\PYZus{}function}\PY{p}{(}\PY{n}{X}\PY{p}{,} \PY{n}{X\PYZus{}train}\PY{p}{,} \PY{n}{Y\PYZus{}train}\PY{p}{)}\PY{p}{:}
%    \PY{k}{if}   \PY{n}{acqusition\PYZus{}id} \PY{o}{==} \PY{l+s+s1}{\PYZsq{}}\PY{l+s+s1}{LB}\PY{l+s+s1}{\PYZsq{}}\PY{p}{:}
%        \PY{k}{return} \PY{n}{lower\PYZus{}bound}\PY{p}{(}\PY{n}{X}\PY{p}{,} \PY{n}{X\PYZus{}train}\PY{p}{,} \PY{n}{Y\PYZus{}train}\PY{p}{,} \PY{n}{k}\PY{o}{=}\PY{l+m+mf}{5.}\PY{p}{)}
%    \PY{k}{elif} \PY{n}{acqusition\PYZus{}id} \PY{o}{==} \PY{l+s+s1}{\PYZsq{}}\PY{l+s+s1}{PI}\PY{l+s+s1}{\PYZsq{}}\PY{p}{:}
%        \PY{k}{return} \PY{n}{probability\PYZus{}of\PYZus{}improvement}\PY{p}{(}\PY{n}{X}\PY{p}{,} \PY{n}{X\PYZus{}train}\PY{p}{,} \PY{n}{Y\PYZus{}train}\PY{p}{,} \PY{n}{xi}\PY{o}{=}\PY{l+m+mf}{0.1}\PY{p}{)}
%    \PY{k}{elif} \PY{n}{acqusition\PYZus{}id} \PY{o}{==} \PY{l+s+s1}{\PYZsq{}}\PY{l+s+s1}{EI}\PY{l+s+s1}{\PYZsq{}}\PY{p}{:}
%        \PY{k}{return} \PY{n}{expected\PYZus{}improvement}\PY{p}{(}\PY{n}{X}\PY{p}{,} \PY{n}{X\PYZus{}train}\PY{p}{,} \PY{n}{Y\PYZus{}train}\PY{p}{,} \PY{n}{xi}\PY{o}{=}\PY{l+m+mf}{0.5}\PY{p}{)}
%\end{Verbatim}
%\end{tcolorbox}
%
%    \begin{tcolorbox}[breakable, size=fbox, boxrule=1pt, pad at break*=1mm,colback=cellbackground, colframe=cellborder]
%\prompt{In}{incolor}{29}{\boxspacing}
%\begin{Verbatim}[commandchars=\\\{\}]
%\PY{c+c1}{\PYZsh{} Initialize samples}
%\PY{n}{np}\PY{o}{.}\PY{n}{random}\PY{o}{.}\PY{n}{seed}\PY{p}{(}\PY{l+m+mi}{42}\PY{p}{)}
%\PY{n}{X\PYZus{}train} \PY{o}{=} \PY{n}{X\PYZus{}init}
%\PY{n}{Y\PYZus{}train} \PY{o}{=} \PY{n}{f\PYZus{}obj\PYZus{}noisy}\PY{p}{(}\PY{n}{X\PYZus{}init}\PY{p}{,} \PY{n}{sigma\PYZus{}in}\PY{p}{)}
%\end{Verbatim}
%\end{tcolorbox}

%    \begin{tcolorbox}[breakable, size=fbox, boxrule=1pt, pad at break*=1mm,colback=cellbackground, colframe=cellborder]
%\prompt{In}{incolor}{30}{\boxspacing}
%\begin{Verbatim}[commandchars=\\\{\}]
%\PY{n}{plt}\PY{o}{.}\PY{n}{figure}\PY{p}{(}\PY{n}{figsize}\PY{o}{=}\PY{p}{(}\PY{l+m+mi}{14}\PY{p}{,} \PY{n}{N\PYZus{}budget} \PY{o}{*} \PY{l+m+mi}{5}\PY{p}{)}\PY{p}{)}
%\PY{n}{plt}\PY{o}{.}\PY{n}{subplots\PYZus{}adjust}\PY{p}{(}\PY{n}{hspace}\PY{o}{=}\PY{l+m+mf}{0.4}\PY{p}{)}
%
%\PY{k}{for} \PY{n}{i} \PY{o+ow}{in} \PY{n+nb}{range}\PY{p}{(}\PY{n}{N\PYZus{}budget}\PY{p}{)}\PY{p}{:}
%    \PY{c+c1}{\PYZsh{} Obtain next sampling point from the acquisition function}
%    \PY{n}{X\PYZus{}next}\PY{p}{,} \PY{n}{acq} \PY{o}{=} \PY{n}{next\PYZus{}point}\PY{p}{(}\PY{n}{acq\PYZus{}function}\PY{p}{,} \PY{n}{X\PYZus{}train}\PY{p}{,} \PY{n}{Y\PYZus{}train}\PY{p}{,} \PY{n}{x\PYZus{}lims}\PY{p}{)}
%    
%    \PY{c+c1}{\PYZsh{} Obtain next sample from the objective function}
%    \PY{n}{Y\PYZus{}next} \PY{o}{=} \PY{n}{f\PYZus{}obj\PYZus{}noisy}\PY{p}{(}\PY{n}{X\PYZus{}next}\PY{p}{,} \PY{n}{sigma\PYZus{}in}\PY{p}{)}
%    
%    \PY{c+c1}{\PYZsh{} Plot samples, surrogate function, noise\PYZhy{}free objective and next sampling location}
%    \PY{n}{ax} \PY{o}{=} \PY{n}{plt}\PY{o}{.}\PY{n}{subplot}\PY{p}{(}\PY{n}{N\PYZus{}budget}\PY{p}{,} \PY{l+m+mi}{2}\PY{p}{,} \PY{l+m+mi}{2}\PY{o}{*}\PY{n}{i} \PY{o}{+} \PY{l+m+mi}{1}\PY{p}{)}
%    \PY{n}{plot\PYZus{}approximation}\PY{p}{(}\PY{n}{X\PYZus{}test}\PY{p}{,} \PY{n}{Y\PYZus{}noisy}\PY{p}{,} \PY{n}{X\PYZus{}train}\PY{p}{,} \PY{n}{Y\PYZus{}train}\PY{p}{,} \PY{n}{X\PYZus{}next}\PY{p}{,} \PY{n}{show\PYZus{}legend}\PY{o}{=}\PY{p}{(}\PY{n}{i}\PY{o}{==}\PY{l+m+mi}{0}\PY{p}{)}\PY{p}{)}
%    \PY{n}{plt}\PY{o}{.}\PY{n}{title}\PY{p}{(}\PY{l+s+sa}{f}\PY{l+s+s1}{\PYZsq{}}\PY{l+s+s1}{Iteration }\PY{l+s+si}{\PYZob{}}\PY{n}{i}\PY{o}{+}\PY{n}{N\PYZus{}init}\PY{l+s+si}{\PYZcb{}}\PY{l+s+s1}{, X\PYZus{}next = }\PY{l+s+si}{\PYZob{}}\PY{n}{X\PYZus{}next}\PY{p}{[}\PY{l+m+mi}{0}\PY{p}{]}\PY{p}{[}\PY{l+m+mi}{0}\PY{p}{]}\PY{l+s+si}{:}\PY{l+s+s1}{.3f}\PY{l+s+si}{\PYZcb{}}\PY{l+s+s1}{\PYZsq{}}\PY{p}{)}
%
%    \PY{n}{plt}\PY{o}{.}\PY{n}{subplot}\PY{p}{(}\PY{n}{N\PYZus{}budget}\PY{p}{,} \PY{l+m+mi}{2}\PY{p}{,} \PY{l+m+mi}{2}\PY{o}{*}\PY{n}{i} \PY{o}{+} \PY{l+m+mi}{2}\PY{p}{)}
%    \PY{n}{Y\PYZus{}acq} \PY{o}{=} \PY{n}{acq\PYZus{}function}\PY{p}{(}\PY{n}{X\PYZus{}test}\PY{p}{,} \PY{n}{X\PYZus{}train}\PY{p}{,} \PY{n}{Y\PYZus{}train}\PY{p}{)}
%    \PY{n}{plot\PYZus{}acquisition}\PY{p}{(}\PY{n}{X\PYZus{}test}\PY{p}{,} \PY{n}{Y\PYZus{}acq}\PY{p}{,} \PY{n}{X\PYZus{}next}\PY{p}{,} \PY{n}{show\PYZus{}legend}\PY{o}{=}\PY{p}{(}\PY{n}{i}\PY{o}{==}\PY{l+m+mi}{0}\PY{p}{)}\PY{p}{)}
%    \PY{n}{plt}\PY{o}{.}\PY{n}{title}\PY{p}{(}\PY{l+s+sa}{f}\PY{l+s+s1}{\PYZsq{}}\PY{l+s+s1}{AF\PYZus{}max = }\PY{l+s+si}{\PYZob{}}\PY{o}{\PYZhy{}}\PY{n}{acq}\PY{l+s+si}{:}\PY{l+s+s1}{.3g}\PY{l+s+si}{\PYZcb{}}\PY{l+s+s1}{\PYZsq{}}\PY{p}{)}
%    
%    \PY{c+c1}{\PYZsh{} Add a new sample to train samples}
%    \PY{n}{X\PYZus{}train} \PY{o}{=} \PY{n}{np}\PY{o}{.}\PY{n}{vstack}\PY{p}{(}\PY{p}{(}\PY{n}{X\PYZus{}train}\PY{p}{,} \PY{n}{X\PYZus{}next}\PY{p}{)}\PY{p}{)}
%    \PY{n}{Y\PYZus{}train} \PY{o}{=} \PY{n}{np}\PY{o}{.}\PY{n}{vstack}\PY{p}{(}\PY{p}{(}\PY{n}{Y\PYZus{}train}\PY{p}{,} \PY{n}{Y\PYZus{}next}\PY{p}{)}\PY{p}{)}
%    
%    \PY{n+nb}{print}\PY{p}{(}\PY{n}{i}\PY{o}{+}\PY{n}{N\PYZus{}init}\PY{p}{,} \PY{o}{*}\PY{n}{X\PYZus{}next}\PY{p}{,} \PY{n}{acq}\PY{p}{)}
%    \PY{k}{if} \PY{p}{(}\PY{o}{\PYZhy{}}\PY{n}{acq} \PY{o}{\PYZlt{}} \PY{l+m+mf}{1e\PYZhy{}200}\PY{p}{)} \PY{o+ow}{or} \PY{p}{(}\PY{n+nb}{abs}\PY{p}{(}\PY{n}{X\PYZus{}train}\PY{p}{[}\PY{o}{\PYZhy{}}\PY{l+m+mi}{2}\PY{p}{]}\PY{o}{\PYZhy{}}\PY{n}{X\PYZus{}train}\PY{p}{[}\PY{o}{\PYZhy{}}\PY{l+m+mi}{1}\PY{p}{]}\PY{p}{)}\PY{p}{[}\PY{l+m+mi}{0}\PY{p}{]} \PY{o}{\PYZlt{}} \PY{l+m+mf}{1e\PYZhy{}3}\PY{o}{*}\PY{n}{x\PYZus{}range}\PY{p}{)}\PY{p}{:}
%        \PY{k}{break}
%\end{Verbatim}
%\end{tcolorbox}

%    \begin{Verbatim}[commandchars=\\\{\}]
%3 [0.334] -0.22861509622233311
%4 [0.418] -0.0015340924538004789
%5 [0.168] -2.815236454093269e-12
%6 [0.146] -1.5030481541109453e-15
%7 [0.678] -1.5751651745938518e-18
%8 [0.712] -0.0003871825958827219
%9 [0.758] -0.002569332828286898
%10 [0.762] -0.00010878059581965365
%11 [0.772] -1.798728993909064e-11
%12 [0.772] -1.6041269276979263e-11
%    \end{Verbatim}

    \begin{center}
    \adjustimage{max size={0.99\linewidth}{0.99\paperheight}}{Test2_EGO.pdf}
    \end{center}
%    { \hspace*{\fill} \\}
    
%    \begin{center}\rule{0.5\linewidth}{0.5pt}\end{center}

%    \hypertarget{ux438ux441ux442ux43eux447ux43dux438ux43aux438}{%
%\section{Источники}\label{ux438ux441ux442ux43eux447ux43dux438ux43aux438}}
%
%\begin{enumerate}
%\def\labelenumi{\arabic{enumi}.}
%\tightlist
%\item
%  \emph{Forrester A. I. J., Sobester A., Keane A. J.} Engineering design
%  via surrogate modelling. --- John Wiley \& Sons Ltd., 2008. --- 210
%  p.~(University of Southampton, UK)
%\item
%  \emph{Krasser M.}
%  \href{http://krasserm.github.io/2018/03/19/gaussian-processes/}{Gaussian
%  processes}.
%\end{enumerate}

%    \begin{tcolorbox}[breakable, size=fbox, boxrule=1pt, pad at break*=1mm,colback=cellbackground, colframe=cellborder]
%\prompt{In}{incolor}{31}{\boxspacing}
%\begin{Verbatim}[commandchars=\\\{\}]
%\PY{c+c1}{\PYZsh{} Versions used}
%\PY{n+nb}{print}\PY{p}{(}\PY{l+s+s1}{\PYZsq{}}\PY{l+s+s1}{Python: }\PY{l+s+si}{\PYZob{}\PYZcb{}}\PY{l+s+s1}{.}\PY{l+s+si}{\PYZob{}\PYZcb{}}\PY{l+s+s1}{.}\PY{l+s+si}{\PYZob{}\PYZcb{}}\PY{l+s+s1}{\PYZsq{}}\PY{o}{.}\PY{n}{format}\PY{p}{(}\PY{o}{*}\PY{n}{sys}\PY{o}{.}\PY{n}{version\PYZus{}info}\PY{p}{[}\PY{p}{:}\PY{l+m+mi}{3}\PY{p}{]}\PY{p}{)}\PY{p}{)}
%\PY{n+nb}{print}\PY{p}{(}\PY{l+s+s1}{\PYZsq{}}\PY{l+s+s1}{numpy: }\PY{l+s+si}{\PYZob{}\PYZcb{}}\PY{l+s+s1}{\PYZsq{}}\PY{o}{.}\PY{n}{format}\PY{p}{(}\PY{n}{np}\PY{o}{.}\PY{n}{\PYZus{}\PYZus{}version\PYZus{}\PYZus{}}\PY{p}{)}\PY{p}{)}
%\PY{n+nb}{print}\PY{p}{(}\PY{l+s+s1}{\PYZsq{}}\PY{l+s+s1}{matplotlib: }\PY{l+s+si}{\PYZob{}\PYZcb{}}\PY{l+s+s1}{\PYZsq{}}\PY{o}{.}\PY{n}{format}\PY{p}{(}\PY{n}{matplotlib}\PY{o}{.}\PY{n}{\PYZus{}\PYZus{}version\PYZus{}\PYZus{}}\PY{p}{)}\PY{p}{)}
%\PY{n+nb}{print}\PY{p}{(}\PY{l+s+s1}{\PYZsq{}}\PY{l+s+s1}{seaborn: }\PY{l+s+si}{\PYZob{}\PYZcb{}}\PY{l+s+s1}{\PYZsq{}}\PY{o}{.}\PY{n}{format}\PY{p}{(}\PY{n}{seaborn}\PY{o}{.}\PY{n}{\PYZus{}\PYZus{}version\PYZus{}\PYZus{}}\PY{p}{)}\PY{p}{)}
%\end{Verbatim}
%\end{tcolorbox}
%
%    \begin{Verbatim}[commandchars=\\\{\}]
%Python: 3.7.16
%numpy: 1.20.3
%matplotlib: 3.5.1
%seaborn: 0.12.2
%    \end{Verbatim}


    % Add a bibliography block to the postdoc
    
    
    
\end{document}
