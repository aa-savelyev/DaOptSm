\documentclass[11pt,a4paper]{article}

    \usepackage[breakable]{tcolorbox}
    \usepackage{parskip} % Stop auto-indenting (to mimic markdown behaviour)
    
    \usepackage{iftex}
    \ifPDFTeX
      \usepackage[T2A]{fontenc}
      \usepackage{mathpazo}
      \usepackage[russian,english]{babel}
    \else
      \usepackage{fontspec}
      \usepackage{polyglossia}
      \setmainlanguage[babelshorthands=true]{russian}    % Язык по-умолчанию русский с поддержкой приятных команд пакета babel
      \setotherlanguage{english}                         % Дополнительный язык = английский (в американской вариации по-умолчанию)
      \newfontfamily\cyrillicfonttt[Scale=0.87,BoldFont={Fira Mono Medium}] {Fira Mono}  % Моноширинный шрифт для кириллицы
      \defaultfontfeatures{Ligatures=TeX}
      \newfontfamily\cyrillicfont{STIX Two Text}         % Шрифт с засечками для кириллицы
    \fi
    \renewcommand{\linethickness}{0.1ex}

    % Basic figure setup, for now with no caption control since it's done
    % automatically by Pandoc (which extracts ![](path) syntax from Markdown).
    \usepackage{graphicx}
    % Maintain compatibility with old templates. Remove in nbconvert 6.0
    \let\Oldincludegraphics\includegraphics
    % Ensure that by default, figures have no caption (until we provide a
    % proper Figure object with a Caption API and a way to capture that
    % in the conversion process - todo).
    \usepackage{caption}
    \DeclareCaptionFormat{nocaption}{}
    \captionsetup{format=nocaption,aboveskip=0pt,belowskip=0pt}

    \usepackage[Export]{adjustbox} % Used to constrain images to a maximum size
    \adjustboxset{max size={0.9\linewidth}{0.9\paperheight}}
    \usepackage{float}
    \floatplacement{figure}{H} % forces figures to be placed at the correct location
    \usepackage{xcolor} % Allow colors to be defined
    \usepackage{enumerate} % Needed for markdown enumerations to work
    \usepackage{geometry} % Used to adjust the document margins
    \usepackage{amsmath} % Equations
    \usepackage{amssymb} % Equations
    \usepackage{textcomp} % defines textquotesingle
    % Hack from http://tex.stackexchange.com/a/47451/13684:
    \AtBeginDocument{%
        \def\PYZsq{\textquotesingle}% Upright quotes in Pygmentized code
    }
    \usepackage{upquote} % Upright quotes for verbatim code
    \usepackage{eurosym} % defines \euro
    \usepackage[mathletters]{ucs} % Extended unicode (utf-8) support
    \usepackage{fancyvrb} % verbatim replacement that allows latex
    \usepackage{grffile} % extends the file name processing of package graphics 
                         % to support a larger range
    \makeatletter % fix for grffile with XeLaTeX
    \def\Gread@@xetex#1{%
      \IfFileExists{"\Gin@base".bb}%
      {\Gread@eps{\Gin@base.bb}}%
      {\Gread@@xetex@aux#1}%
    }
    \makeatother

    % The hyperref package gives us a pdf with properly built
    % internal navigation ('pdf bookmarks' for the table of contents,
    % internal cross-reference links, web links for URLs, etc.)
    \usepackage{hyperref}
    % The default LaTeX title has an obnoxious amount of whitespace. By default,
    % titling removes some of it. It also provides customization options.
    \usepackage{titling}
    \usepackage{longtable} % longtable support required by pandoc >1.10
    \usepackage{booktabs}  % table support for pandoc > 1.12.2
    \usepackage[inline]{enumitem} % IRkernel/repr support (it uses the enumerate* environment)
    \usepackage[normalem]{ulem} % ulem is needed to support strikethroughs (\sout)
                                % normalem makes italics be italics, not underlines
    \usepackage{mathrsfs}
    

    
    % Colors for the hyperref package
    \definecolor{urlcolor}{rgb}{0,.145,.698}
    \definecolor{linkcolor}{rgb}{.71,0.21,0.01}
    \definecolor{citecolor}{rgb}{.12,.54,.11}

    % ANSI colors
    \definecolor{ansi-black}{HTML}{3E424D}
    \definecolor{ansi-black-intense}{HTML}{282C36}
    \definecolor{ansi-red}{HTML}{E75C58}
    \definecolor{ansi-red-intense}{HTML}{B22B31}
    \definecolor{ansi-green}{HTML}{00A250}
    \definecolor{ansi-green-intense}{HTML}{007427}
    \definecolor{ansi-yellow}{HTML}{DDB62B}
    \definecolor{ansi-yellow-intense}{HTML}{B27D12}
    \definecolor{ansi-blue}{HTML}{208FFB}
    \definecolor{ansi-blue-intense}{HTML}{0065CA}
    \definecolor{ansi-magenta}{HTML}{D160C4}
    \definecolor{ansi-magenta-intense}{HTML}{A03196}
    \definecolor{ansi-cyan}{HTML}{60C6C8}
    \definecolor{ansi-cyan-intense}{HTML}{258F8F}
    \definecolor{ansi-white}{HTML}{C5C1B4}
    \definecolor{ansi-white-intense}{HTML}{A1A6B2}
    \definecolor{ansi-default-inverse-fg}{HTML}{FFFFFF}
    \definecolor{ansi-default-inverse-bg}{HTML}{000000}

    % commands and environments needed by pandoc snippets
    % extracted from the output of `pandoc -s`
    \providecommand{\tightlist}{%
      \setlength{\itemsep}{0pt}\setlength{\parskip}{0pt}}
    \DefineVerbatimEnvironment{Highlighting}{Verbatim}{commandchars=\\\{\}}
    % Add ',fontsize=\small' for more characters per line
    \newenvironment{Shaded}{}{}
    \newcommand{\KeywordTok}[1]{\textcolor[rgb]{0.00,0.44,0.13}{\textbf{{#1}}}}
    \newcommand{\DataTypeTok}[1]{\textcolor[rgb]{0.56,0.13,0.00}{{#1}}}
    \newcommand{\DecValTok}[1]{\textcolor[rgb]{0.25,0.63,0.44}{{#1}}}
    \newcommand{\BaseNTok}[1]{\textcolor[rgb]{0.25,0.63,0.44}{{#1}}}
    \newcommand{\FloatTok}[1]{\textcolor[rgb]{0.25,0.63,0.44}{{#1}}}
    \newcommand{\CharTok}[1]{\textcolor[rgb]{0.25,0.44,0.63}{{#1}}}
    \newcommand{\StringTok}[1]{\textcolor[rgb]{0.25,0.44,0.63}{{#1}}}
    \newcommand{\CommentTok}[1]{\textcolor[rgb]{0.38,0.63,0.69}{\textit{{#1}}}}
    \newcommand{\OtherTok}[1]{\textcolor[rgb]{0.00,0.44,0.13}{{#1}}}
    \newcommand{\AlertTok}[1]{\textcolor[rgb]{1.00,0.00,0.00}{\textbf{{#1}}}}
    \newcommand{\FunctionTok}[1]{\textcolor[rgb]{0.02,0.16,0.49}{{#1}}}
    \newcommand{\RegionMarkerTok}[1]{{#1}}
    \newcommand{\ErrorTok}[1]{\textcolor[rgb]{1.00,0.00,0.00}{\textbf{{#1}}}}
    \newcommand{\NormalTok}[1]{{#1}}
    
    % Additional commands for more recent versions of Pandoc
    \newcommand{\ConstantTok}[1]{\textcolor[rgb]{0.53,0.00,0.00}{{#1}}}
    \newcommand{\SpecialCharTok}[1]{\textcolor[rgb]{0.25,0.44,0.63}{{#1}}}
    \newcommand{\VerbatimStringTok}[1]{\textcolor[rgb]{0.25,0.44,0.63}{{#1}}}
    \newcommand{\SpecialStringTok}[1]{\textcolor[rgb]{0.73,0.40,0.53}{{#1}}}
    \newcommand{\ImportTok}[1]{{#1}}
    \newcommand{\DocumentationTok}[1]{\textcolor[rgb]{0.73,0.13,0.13}{\textit{{#1}}}}
    \newcommand{\AnnotationTok}[1]{\textcolor[rgb]{0.38,0.63,0.69}{\textbf{\textit{{#1}}}}}
    \newcommand{\CommentVarTok}[1]{\textcolor[rgb]{0.38,0.63,0.69}{\textbf{\textit{{#1}}}}}
    \newcommand{\VariableTok}[1]{\textcolor[rgb]{0.10,0.09,0.49}{{#1}}}
    \newcommand{\ControlFlowTok}[1]{\textcolor[rgb]{0.00,0.44,0.13}{\textbf{{#1}}}}
    \newcommand{\OperatorTok}[1]{\textcolor[rgb]{0.40,0.40,0.40}{{#1}}}
    \newcommand{\BuiltInTok}[1]{{#1}}
    \newcommand{\ExtensionTok}[1]{{#1}}
    \newcommand{\PreprocessorTok}[1]{\textcolor[rgb]{0.74,0.48,0.00}{{#1}}}
    \newcommand{\AttributeTok}[1]{\textcolor[rgb]{0.49,0.56,0.16}{{#1}}}
    \newcommand{\InformationTok}[1]{\textcolor[rgb]{0.38,0.63,0.69}{\textbf{\textit{{#1}}}}}
    \newcommand{\WarningTok}[1]{\textcolor[rgb]{0.38,0.63,0.69}{\textbf{\textit{{#1}}}}}
    
    
    % Define a nice break command that doesn't care if a line doesn't already
    % exist.
    \def\br{\hspace*{\fill} \\* }
    % Math Jax compatibility definitions
    \def\gt{>}
    \def\lt{<}
    \let\Oldtex\TeX
    \let\Oldlatex\LaTeX
    \renewcommand{\TeX}{\textrm{\Oldtex}}
    \renewcommand{\LaTeX}{\textrm{\Oldlatex}}
    % Document parameters
    % Document title
    \title{
    {\Large Лекция 4} \\
    Распределение Гаусса
  }
    
    
    
    
    
% Pygments definitions
\makeatletter
\def\PY@reset{\let\PY@it=\relax \let\PY@bf=\relax%
    \let\PY@ul=\relax \let\PY@tc=\relax%
    \let\PY@bc=\relax \let\PY@ff=\relax}
\def\PY@tok#1{\csname PY@tok@#1\endcsname}
\def\PY@toks#1+{\ifx\relax#1\empty\else%
    \PY@tok{#1}\expandafter\PY@toks\fi}
\def\PY@do#1{\PY@bc{\PY@tc{\PY@ul{%
    \PY@it{\PY@bf{\PY@ff{#1}}}}}}}
\def\PY#1#2{\PY@reset\PY@toks#1+\relax+\PY@do{#2}}

\expandafter\def\csname PY@tok@w\endcsname{\def\PY@tc##1{\textcolor[rgb]{0.73,0.73,0.73}{##1}}}
\expandafter\def\csname PY@tok@c\endcsname{\let\PY@it=\textit\def\PY@tc##1{\textcolor[rgb]{0.25,0.50,0.50}{##1}}}
\expandafter\def\csname PY@tok@cp\endcsname{\def\PY@tc##1{\textcolor[rgb]{0.74,0.48,0.00}{##1}}}
\expandafter\def\csname PY@tok@k\endcsname{\let\PY@bf=\textbf\def\PY@tc##1{\textcolor[rgb]{0.00,0.50,0.00}{##1}}}
\expandafter\def\csname PY@tok@kp\endcsname{\def\PY@tc##1{\textcolor[rgb]{0.00,0.50,0.00}{##1}}}
\expandafter\def\csname PY@tok@kt\endcsname{\def\PY@tc##1{\textcolor[rgb]{0.69,0.00,0.25}{##1}}}
\expandafter\def\csname PY@tok@o\endcsname{\def\PY@tc##1{\textcolor[rgb]{0.40,0.40,0.40}{##1}}}
\expandafter\def\csname PY@tok@ow\endcsname{\let\PY@bf=\textbf\def\PY@tc##1{\textcolor[rgb]{0.67,0.13,1.00}{##1}}}
\expandafter\def\csname PY@tok@nb\endcsname{\def\PY@tc##1{\textcolor[rgb]{0.00,0.50,0.00}{##1}}}
\expandafter\def\csname PY@tok@nf\endcsname{\def\PY@tc##1{\textcolor[rgb]{0.00,0.00,1.00}{##1}}}
\expandafter\def\csname PY@tok@nc\endcsname{\let\PY@bf=\textbf\def\PY@tc##1{\textcolor[rgb]{0.00,0.00,1.00}{##1}}}
\expandafter\def\csname PY@tok@nn\endcsname{\let\PY@bf=\textbf\def\PY@tc##1{\textcolor[rgb]{0.00,0.00,1.00}{##1}}}
\expandafter\def\csname PY@tok@ne\endcsname{\let\PY@bf=\textbf\def\PY@tc##1{\textcolor[rgb]{0.82,0.25,0.23}{##1}}}
\expandafter\def\csname PY@tok@nv\endcsname{\def\PY@tc##1{\textcolor[rgb]{0.10,0.09,0.49}{##1}}}
\expandafter\def\csname PY@tok@no\endcsname{\def\PY@tc##1{\textcolor[rgb]{0.53,0.00,0.00}{##1}}}
\expandafter\def\csname PY@tok@nl\endcsname{\def\PY@tc##1{\textcolor[rgb]{0.63,0.63,0.00}{##1}}}
\expandafter\def\csname PY@tok@ni\endcsname{\let\PY@bf=\textbf\def\PY@tc##1{\textcolor[rgb]{0.60,0.60,0.60}{##1}}}
\expandafter\def\csname PY@tok@na\endcsname{\def\PY@tc##1{\textcolor[rgb]{0.49,0.56,0.16}{##1}}}
\expandafter\def\csname PY@tok@nt\endcsname{\let\PY@bf=\textbf\def\PY@tc##1{\textcolor[rgb]{0.00,0.50,0.00}{##1}}}
\expandafter\def\csname PY@tok@nd\endcsname{\def\PY@tc##1{\textcolor[rgb]{0.67,0.13,1.00}{##1}}}
\expandafter\def\csname PY@tok@s\endcsname{\def\PY@tc##1{\textcolor[rgb]{0.73,0.13,0.13}{##1}}}
\expandafter\def\csname PY@tok@sd\endcsname{\let\PY@it=\textit\def\PY@tc##1{\textcolor[rgb]{0.73,0.13,0.13}{##1}}}
\expandafter\def\csname PY@tok@si\endcsname{\let\PY@bf=\textbf\def\PY@tc##1{\textcolor[rgb]{0.73,0.40,0.53}{##1}}}
\expandafter\def\csname PY@tok@se\endcsname{\let\PY@bf=\textbf\def\PY@tc##1{\textcolor[rgb]{0.73,0.40,0.13}{##1}}}
\expandafter\def\csname PY@tok@sr\endcsname{\def\PY@tc##1{\textcolor[rgb]{0.73,0.40,0.53}{##1}}}
\expandafter\def\csname PY@tok@ss\endcsname{\def\PY@tc##1{\textcolor[rgb]{0.10,0.09,0.49}{##1}}}
\expandafter\def\csname PY@tok@sx\endcsname{\def\PY@tc##1{\textcolor[rgb]{0.00,0.50,0.00}{##1}}}
\expandafter\def\csname PY@tok@m\endcsname{\def\PY@tc##1{\textcolor[rgb]{0.40,0.40,0.40}{##1}}}
\expandafter\def\csname PY@tok@gh\endcsname{\let\PY@bf=\textbf\def\PY@tc##1{\textcolor[rgb]{0.00,0.00,0.50}{##1}}}
\expandafter\def\csname PY@tok@gu\endcsname{\let\PY@bf=\textbf\def\PY@tc##1{\textcolor[rgb]{0.50,0.00,0.50}{##1}}}
\expandafter\def\csname PY@tok@gd\endcsname{\def\PY@tc##1{\textcolor[rgb]{0.63,0.00,0.00}{##1}}}
\expandafter\def\csname PY@tok@gi\endcsname{\def\PY@tc##1{\textcolor[rgb]{0.00,0.63,0.00}{##1}}}
\expandafter\def\csname PY@tok@gr\endcsname{\def\PY@tc##1{\textcolor[rgb]{1.00,0.00,0.00}{##1}}}
\expandafter\def\csname PY@tok@ge\endcsname{\let\PY@it=\textit}
\expandafter\def\csname PY@tok@gs\endcsname{\let\PY@bf=\textbf}
\expandafter\def\csname PY@tok@gp\endcsname{\let\PY@bf=\textbf\def\PY@tc##1{\textcolor[rgb]{0.00,0.00,0.50}{##1}}}
\expandafter\def\csname PY@tok@go\endcsname{\def\PY@tc##1{\textcolor[rgb]{0.53,0.53,0.53}{##1}}}
\expandafter\def\csname PY@tok@gt\endcsname{\def\PY@tc##1{\textcolor[rgb]{0.00,0.27,0.87}{##1}}}
\expandafter\def\csname PY@tok@err\endcsname{\def\PY@bc##1{\setlength{\fboxsep}{0pt}\fcolorbox[rgb]{1.00,0.00,0.00}{1,1,1}{\strut ##1}}}
\expandafter\def\csname PY@tok@kc\endcsname{\let\PY@bf=\textbf\def\PY@tc##1{\textcolor[rgb]{0.00,0.50,0.00}{##1}}}
\expandafter\def\csname PY@tok@kd\endcsname{\let\PY@bf=\textbf\def\PY@tc##1{\textcolor[rgb]{0.00,0.50,0.00}{##1}}}
\expandafter\def\csname PY@tok@kn\endcsname{\let\PY@bf=\textbf\def\PY@tc##1{\textcolor[rgb]{0.00,0.50,0.00}{##1}}}
\expandafter\def\csname PY@tok@kr\endcsname{\let\PY@bf=\textbf\def\PY@tc##1{\textcolor[rgb]{0.00,0.50,0.00}{##1}}}
\expandafter\def\csname PY@tok@bp\endcsname{\def\PY@tc##1{\textcolor[rgb]{0.00,0.50,0.00}{##1}}}
\expandafter\def\csname PY@tok@fm\endcsname{\def\PY@tc##1{\textcolor[rgb]{0.00,0.00,1.00}{##1}}}
\expandafter\def\csname PY@tok@vc\endcsname{\def\PY@tc##1{\textcolor[rgb]{0.10,0.09,0.49}{##1}}}
\expandafter\def\csname PY@tok@vg\endcsname{\def\PY@tc##1{\textcolor[rgb]{0.10,0.09,0.49}{##1}}}
\expandafter\def\csname PY@tok@vi\endcsname{\def\PY@tc##1{\textcolor[rgb]{0.10,0.09,0.49}{##1}}}
\expandafter\def\csname PY@tok@vm\endcsname{\def\PY@tc##1{\textcolor[rgb]{0.10,0.09,0.49}{##1}}}
\expandafter\def\csname PY@tok@sa\endcsname{\def\PY@tc##1{\textcolor[rgb]{0.73,0.13,0.13}{##1}}}
\expandafter\def\csname PY@tok@sb\endcsname{\def\PY@tc##1{\textcolor[rgb]{0.73,0.13,0.13}{##1}}}
\expandafter\def\csname PY@tok@sc\endcsname{\def\PY@tc##1{\textcolor[rgb]{0.73,0.13,0.13}{##1}}}
\expandafter\def\csname PY@tok@dl\endcsname{\def\PY@tc##1{\textcolor[rgb]{0.73,0.13,0.13}{##1}}}
\expandafter\def\csname PY@tok@s2\endcsname{\def\PY@tc##1{\textcolor[rgb]{0.73,0.13,0.13}{##1}}}
\expandafter\def\csname PY@tok@sh\endcsname{\def\PY@tc##1{\textcolor[rgb]{0.73,0.13,0.13}{##1}}}
\expandafter\def\csname PY@tok@s1\endcsname{\def\PY@tc##1{\textcolor[rgb]{0.73,0.13,0.13}{##1}}}
\expandafter\def\csname PY@tok@mb\endcsname{\def\PY@tc##1{\textcolor[rgb]{0.40,0.40,0.40}{##1}}}
\expandafter\def\csname PY@tok@mf\endcsname{\def\PY@tc##1{\textcolor[rgb]{0.40,0.40,0.40}{##1}}}
\expandafter\def\csname PY@tok@mh\endcsname{\def\PY@tc##1{\textcolor[rgb]{0.40,0.40,0.40}{##1}}}
\expandafter\def\csname PY@tok@mi\endcsname{\def\PY@tc##1{\textcolor[rgb]{0.40,0.40,0.40}{##1}}}
\expandafter\def\csname PY@tok@il\endcsname{\def\PY@tc##1{\textcolor[rgb]{0.40,0.40,0.40}{##1}}}
\expandafter\def\csname PY@tok@mo\endcsname{\def\PY@tc##1{\textcolor[rgb]{0.40,0.40,0.40}{##1}}}
\expandafter\def\csname PY@tok@ch\endcsname{\let\PY@it=\textit\def\PY@tc##1{\textcolor[rgb]{0.25,0.50,0.50}{##1}}}
\expandafter\def\csname PY@tok@cm\endcsname{\let\PY@it=\textit\def\PY@tc##1{\textcolor[rgb]{0.25,0.50,0.50}{##1}}}
\expandafter\def\csname PY@tok@cpf\endcsname{\let\PY@it=\textit\def\PY@tc##1{\textcolor[rgb]{0.25,0.50,0.50}{##1}}}
\expandafter\def\csname PY@tok@c1\endcsname{\let\PY@it=\textit\def\PY@tc##1{\textcolor[rgb]{0.25,0.50,0.50}{##1}}}
\expandafter\def\csname PY@tok@cs\endcsname{\let\PY@it=\textit\def\PY@tc##1{\textcolor[rgb]{0.25,0.50,0.50}{##1}}}

\def\PYZbs{\char`\\}
\def\PYZus{\char`\_}
\def\PYZob{\char`\{}
\def\PYZcb{\char`\}}
\def\PYZca{\char`\^}
\def\PYZam{\char`\&}
\def\PYZlt{\char`\<}
\def\PYZgt{\char`\>}
\def\PYZsh{\char`\#}
\def\PYZpc{\char`\%}
\def\PYZdl{\char`\$}
\def\PYZhy{\char`\-}
\def\PYZsq{\char`\'}
\def\PYZdq{\char`\"}
\def\PYZti{\char`\~}
% for compatibility with earlier versions
\def\PYZat{@}
\def\PYZlb{[}
\def\PYZrb{]}
\makeatother


    % For linebreaks inside Verbatim environment from package fancyvrb. 
    \makeatletter
        \newbox\Wrappedcontinuationbox 
        \newbox\Wrappedvisiblespacebox 
        \newcommand*\Wrappedvisiblespace {\textcolor{red}{\textvisiblespace}} 
        \newcommand*\Wrappedcontinuationsymbol {\textcolor{red}{\llap{\tiny$\m@th\hookrightarrow$}}} 
        \newcommand*\Wrappedcontinuationindent {3ex } 
        \newcommand*\Wrappedafterbreak {\kern\Wrappedcontinuationindent\copy\Wrappedcontinuationbox} 
        % Take advantage of the already applied Pygments mark-up to insert 
        % potential linebreaks for TeX processing. 
        %        {, <, #, %, $, ' and ": go to next line. 
        %        _, }, ^, &, >, - and ~: stay at end of broken line. 
        % Use of \textquotesingle for straight quote. 
        \newcommand*\Wrappedbreaksatspecials {% 
            \def\PYGZus{\discretionary{\char`\_}{\Wrappedafterbreak}{\char`\_}}% 
            \def\PYGZob{\discretionary{}{\Wrappedafterbreak\char`\{}{\char`\{}}% 
            \def\PYGZcb{\discretionary{\char`\}}{\Wrappedafterbreak}{\char`\}}}% 
            \def\PYGZca{\discretionary{\char`\^}{\Wrappedafterbreak}{\char`\^}}% 
            \def\PYGZam{\discretionary{\char`\&}{\Wrappedafterbreak}{\char`\&}}% 
            \def\PYGZlt{\discretionary{}{\Wrappedafterbreak\char`\<}{\char`\<}}% 
            \def\PYGZgt{\discretionary{\char`\>}{\Wrappedafterbreak}{\char`\>}}% 
            \def\PYGZsh{\discretionary{}{\Wrappedafterbreak\char`\#}{\char`\#}}% 
            \def\PYGZpc{\discretionary{}{\Wrappedafterbreak\char`\%}{\char`\%}}% 
            \def\PYGZdl{\discretionary{}{\Wrappedafterbreak\char`\$}{\char`\$}}% 
            \def\PYGZhy{\discretionary{\char`\-}{\Wrappedafterbreak}{\char`\-}}% 
            \def\PYGZsq{\discretionary{}{\Wrappedafterbreak\textquotesingle}{\textquotesingle}}% 
            \def\PYGZdq{\discretionary{}{\Wrappedafterbreak\char`\"}{\char`\"}}% 
            \def\PYGZti{\discretionary{\char`\~}{\Wrappedafterbreak}{\char`\~}}% 
        } 
        % Some characters . , ; ? ! / are not pygmentized. 
        % This macro makes them "active" and they will insert potential linebreaks 
        \newcommand*\Wrappedbreaksatpunct {% 
            \lccode`\~`\.\lowercase{\def~}{\discretionary{\hbox{\char`\.}}{\Wrappedafterbreak}{\hbox{\char`\.}}}% 
            \lccode`\~`\,\lowercase{\def~}{\discretionary{\hbox{\char`\,}}{\Wrappedafterbreak}{\hbox{\char`\,}}}% 
            \lccode`\~`\;\lowercase{\def~}{\discretionary{\hbox{\char`\;}}{\Wrappedafterbreak}{\hbox{\char`\;}}}% 
            \lccode`\~`\:\lowercase{\def~}{\discretionary{\hbox{\char`\:}}{\Wrappedafterbreak}{\hbox{\char`\:}}}% 
            \lccode`\~`\?\lowercase{\def~}{\discretionary{\hbox{\char`\?}}{\Wrappedafterbreak}{\hbox{\char`\?}}}% 
            \lccode`\~`\!\lowercase{\def~}{\discretionary{\hbox{\char`\!}}{\Wrappedafterbreak}{\hbox{\char`\!}}}% 
            \lccode`\~`\/\lowercase{\def~}{\discretionary{\hbox{\char`\/}}{\Wrappedafterbreak}{\hbox{\char`\/}}}% 
            \catcode`\.\active
            \catcode`\,\active 
            \catcode`\;\active
            \catcode`\:\active
            \catcode`\?\active
            \catcode`\!\active
            \catcode`\/\active 
            \lccode`\~`\~ 	
        }
    \makeatother

    \let\OriginalVerbatim=\Verbatim
    \makeatletter
    \renewcommand{\Verbatim}[1][1]{%
        %\parskip\z@skip
        \sbox\Wrappedcontinuationbox {\Wrappedcontinuationsymbol}%
        \sbox\Wrappedvisiblespacebox {\FV@SetupFont\Wrappedvisiblespace}%
        \def\FancyVerbFormatLine ##1{\hsize\linewidth
            \vtop{\raggedright\hyphenpenalty\z@\exhyphenpenalty\z@
                \doublehyphendemerits\z@\finalhyphendemerits\z@
                \strut ##1\strut}%
        }%
        % If the linebreak is at a space, the latter will be displayed as visible
        % space at end of first line, and a continuation symbol starts next line.
        % Stretch/shrink are however usually zero for typewriter font.
        \def\FV@Space {%
            \nobreak\hskip\z@ plus\fontdimen3\font minus\fontdimen4\font
            \discretionary{\copy\Wrappedvisiblespacebox}{\Wrappedafterbreak}
            {\kern\fontdimen2\font}%
        }%
        
        % Allow breaks at special characters using \PYG... macros.
        \Wrappedbreaksatspecials
        % Breaks at punctuation characters . , ; ? ! and / need catcode=\active 	
        \OriginalVerbatim[#1,codes*=\Wrappedbreaksatpunct]%
    }
    \makeatother

    % Exact colors from NB
    \definecolor{incolor}{HTML}{303F9F}
    \definecolor{outcolor}{HTML}{D84315}
    \definecolor{cellborder}{HTML}{CFCFCF}
    \definecolor{cellbackground}{HTML}{F7F7F7}
    
    % prompt
    \makeatletter
    \newcommand{\boxspacing}{\kern\kvtcb@left@rule\kern\kvtcb@boxsep}
    \makeatother
    \newcommand{\prompt}[4]{
        \ttfamily\llap{{\color{#2}[#3]:\hspace{3pt}#4}}\vspace{-\baselineskip}
    }
    

    
    % Prevent overflowing lines due to hard-to-break entities
    \sloppy 
    % Setup hyperref package
    \hypersetup{
      breaklinks=true,  % so long urls are correctly broken across lines
      colorlinks=true,
      urlcolor=urlcolor,
      linkcolor=linkcolor,
      citecolor=citecolor,
      }
    % Slightly bigger margins than the latex defaults
    
    \geometry{verbose,tmargin=1in,bmargin=1in,lmargin=1in,rmargin=1in}
    
    

\begin{document}

\maketitle
\tableofcontents
\pagebreak


%    \begin{tcolorbox}[breakable, size=fbox, boxrule=1pt, pad at break*=1mm,colback=cellbackground, colframe=cellborder]
%\prompt{In}{incolor}{3}{\boxspacing}
%\begin{Verbatim}[commandchars=\\\{\}]
%\PY{c+c1}{\PYZsh{} Imports}
%\PY{k+kn}{import} \PY{n+nn}{sys}
%\PY{k+kn}{import} \PY{n+nn}{numpy} \PY{k}{as} \PY{n+nn}{np}
%\PY{k+kn}{import} \PY{n+nn}{scipy}\PY{n+nn}{.}\PY{n+nn}{stats} \PY{k}{as} \PY{n+nn}{stats}
%\PY{n}{np}\PY{o}{.}\PY{n}{random}\PY{o}{.}\PY{n}{seed}\PY{p}{(}\PY{l+m+mi}{42}\PY{p}{)}
%
%\PY{k+kn}{import} \PY{n+nn}{matplotlib}
%\PY{k+kn}{import} \PY{n+nn}{matplotlib}\PY{n+nn}{.}\PY{n+nn}{pyplot} \PY{k}{as} \PY{n+nn}{plt}
%\PY{k+kn}{from} \PY{n+nn}{matplotlib} \PY{k+kn}{import} \PY{n}{cm} \PY{c+c1}{\PYZsh{} Colormaps}
%\PY{k+kn}{from} \PY{n+nn}{matplotlib}\PY{n+nn}{.}\PY{n+nn}{patches} \PY{k+kn}{import} \PY{n}{Ellipse}
%\PY{k+kn}{import} \PY{n+nn}{matplotlib}\PY{n+nn}{.}\PY{n+nn}{gridspec} \PY{k}{as} \PY{n+nn}{gridspec}
%\PY{k+kn}{from} \PY{n+nn}{mpl\PYZus{}toolkits}\PY{n+nn}{.}\PY{n+nn}{axes\PYZus{}grid1} \PY{k+kn}{import} \PY{n}{make\PYZus{}axes\PYZus{}locatable}
%\PY{k+kn}{import} \PY{n+nn}{seaborn} \PY{k}{as} \PY{n+nn}{sns}
%\end{Verbatim}
%\end{tcolorbox}
%
%    \begin{tcolorbox}[breakable, size=fbox, boxrule=1pt, pad at break*=1mm,colback=cellbackground, colframe=cellborder]
%\prompt{In}{incolor}{4}{\boxspacing}
%\begin{Verbatim}[commandchars=\\\{\}]
%\PY{c+c1}{\PYZsh{} Styles, fonts}
%\PY{n}{sns}\PY{o}{.}\PY{n}{set\PYZus{}style}\PY{p}{(}\PY{l+s+s1}{\PYZsq{}}\PY{l+s+s1}{whitegrid}\PY{l+s+s1}{\PYZsq{}}\PY{p}{)}
%\PY{n}{matplotlib}\PY{o}{.}\PY{n}{rcParams}\PY{p}{[}\PY{l+s+s1}{\PYZsq{}}\PY{l+s+s1}{font.size}\PY{l+s+s1}{\PYZsq{}}\PY{p}{]} \PY{o}{=} \PY{l+m+mi}{12}
%\end{Verbatim}
%\end{tcolorbox}

    \hypertarget{ux433ux430ux443ux441ux441ux43eux432ux441ux43aux438ux435-ux441ux43bux443ux447ux430ux439ux43dux44bux435-ux432ux435ux43bux438ux447ux438ux43dux44b}{%
\section{Гауссовские случайные
величины}\label{ux433ux430ux443ux441ux441ux43eux432ux441ux43aux438ux435-ux441ux43bux443ux447ux430ux439ux43dux44bux435-ux432ux435ux43bux438ux447ux438ux43dux44b}}

\hypertarget{ux43eux43fux440ux435ux434ux435ux43bux435ux43dux438ux435}{%
\subsection{Определение}\label{ux43eux43fux440ux435ux434ux435ux43bux435ux43dux438ux435}}

Если \(\xi\) --- случайная величина с гауссовской (нормальной)
плотностью (probability density function, pdf)

\[ f_\xi(x|\mu,\sigma) = \frac{1}{\sqrt{2\pi}\sigma} \exp{ \left( -\frac{(x - \mu)^2}{2\sigma^2}\right)}, \quad \sigma>0, \quad -\infty < \mu < \infty, \]

то смысл параметров \(\mu\) и \(\sigma\) оказывается очень простым:
\[ \mu = \mathrm{E} \xi, \quad \sigma^2 = \mathrm{D} \xi . \]

Таким образом, рапределение вероятностей этой случайной величины
\(\xi\), называемой \emph{гауссовской} или \emph{нормально
распределённой}, полностью определяется её средним значением \(\mu\) и
дисперсией \(\sigma^2\). В этой связи часто используется запись
\[ \xi \sim \mathcal{N}\left( \mu, \sigma^2 \right). \]

%    \begin{tcolorbox}[breakable, size=fbox, boxrule=1pt, pad at break*=1mm,colback=cellbackground, colframe=cellborder]
%\prompt{In}{incolor}{5}{\boxspacing}
%\begin{Verbatim}[commandchars=\\\{\}]
%\PY{k}{def} \PY{n+nf}{univariate\PYZus{}normal}\PY{p}{(}\PY{n}{x}\PY{p}{,} \PY{n}{mean}\PY{p}{,} \PY{n}{variance}\PY{p}{)}\PY{p}{:}
%    \PY{l+s+sd}{\PYZdq{}\PYZdq{}\PYZdq{}pdf of the univariate normal distribution\PYZdq{}\PYZdq{}\PYZdq{}}
%    \PY{k}{return} \PY{p}{(}\PY{p}{(}\PY{l+m+mf}{1.} \PY{o}{/} \PY{n}{np}\PY{o}{.}\PY{n}{sqrt}\PY{p}{(}\PY{l+m+mi}{2} \PY{o}{*} \PY{n}{np}\PY{o}{.}\PY{n}{pi} \PY{o}{*} \PY{n}{variance}\PY{p}{)}\PY{p}{)} \PY{o}{*} 
%            \PY{n}{np}\PY{o}{.}\PY{n}{exp}\PY{p}{(}\PY{o}{\PYZhy{}}\PY{p}{(}\PY{n}{x} \PY{o}{\PYZhy{}} \PY{n}{mean}\PY{p}{)}\PY{o}{*}\PY{o}{*}\PY{l+m+mi}{2} \PY{o}{/} \PY{p}{(}\PY{l+m+mi}{2} \PY{o}{*} \PY{n}{variance}\PY{p}{)}\PY{p}{)}\PY{p}{)}
%\end{Verbatim}
%\end{tcolorbox}

    \hypertarget{ux441ux432ux43eux439ux441ux442ux432ux430}{%
\subsection{Свойства}\label{ux441ux432ux43eux439ux441ux442ux432ux430}}

\begin{enumerate}
\def\labelenumi{\arabic{enumi}.}
\item
  Если \(\xi\) и \(\eta\) --- гауссовские случайные величины, то из их
  \emph{некоррелированности} следует их \emph{независимость}.
\item
  Сумма двух независимых гауссовских случайных величин снова есть
  гауссовская случайная величина со средним \(\mu_1 + \mu_2\) и
  дисперсией \(\sigma_1^2 + \sigma_2^2\).
\item
  \textbf{Центральная предельная теорема:} распределение суммы большого
  числа независимых случайных величин или случайных векторов,
  подчиняющихся не слишком стеснительным условиям, хорошо
  аппроксимируется нормальным распределением.
\end{enumerate}

    \hypertarget{ux43fux440ux438ux43cux435ux440ux44b}{%
\subsection{Примеры}\label{ux43fux440ux438ux43cux435ux440ux44b}}

Ниже приведены примеры трёх одномерных нормальных распределений:

\begin{enumerate}
\def\labelenumi{\arabic{enumi}.}
\tightlist
\item
  \(\mathcal{N}(0, 1)\),
\item
  \(\mathcal{N}(2, 3)\),
\item
  \(\mathcal{N}(0, 0.2)\).
\end{enumerate}

%    \begin{tcolorbox}[breakable, size=fbox, boxrule=1pt, pad at break*=1mm,colback=cellbackground, colframe=cellborder]
%\prompt{In}{incolor}{24}{\boxspacing}
%\begin{Verbatim}[commandchars=\\\{\}]
%\PY{c+c1}{\PYZsh{} Plot different Univariate Normals}
%\PY{n}{x} \PY{o}{=} \PY{n}{np}\PY{o}{.}\PY{n}{linspace}\PY{p}{(}\PY{o}{\PYZhy{}}\PY{l+m+mi}{3}\PY{p}{,} \PY{l+m+mi}{5}\PY{p}{,} \PY{n}{num}\PY{o}{=}\PY{l+m+mi}{150}\PY{p}{)}
%\PY{n}{fig} \PY{o}{=} \PY{n}{plt}\PY{o}{.}\PY{n}{figure}\PY{p}{(}\PY{n}{figsize}\PY{o}{=}\PY{p}{(}\PY{l+m+mi}{8}\PY{p}{,} \PY{l+m+mi}{5}\PY{p}{)}\PY{p}{)}
%\PY{n}{plt}\PY{o}{.}\PY{n}{plot}\PY{p}{(}
%    \PY{n}{x}\PY{p}{,} \PY{n}{univariate\PYZus{}normal}\PY{p}{(}\PY{n}{x}\PY{p}{,} \PY{n}{mean}\PY{o}{=}\PY{l+m+mi}{0}\PY{p}{,} \PY{n}{variance}\PY{o}{=}\PY{l+m+mi}{1}\PY{p}{)}\PY{p}{,} 
%    \PY{n}{label}\PY{o}{=}\PY{l+s+s2}{\PYZdq{}}\PY{l+s+s2}{\PYZdl{}}\PY{l+s+s2}{\PYZbs{}}\PY{l+s+s2}{mathcal}\PY{l+s+si}{\PYZob{}N\PYZcb{}}\PY{l+s+s2}{(0, 1)\PYZdl{}}\PY{l+s+s2}{\PYZdq{}}\PY{p}{)}
%\PY{n}{plt}\PY{o}{.}\PY{n}{plot}\PY{p}{(}
%    \PY{n}{x}\PY{p}{,} \PY{n}{univariate\PYZus{}normal}\PY{p}{(}\PY{n}{x}\PY{p}{,} \PY{n}{mean}\PY{o}{=}\PY{l+m+mi}{2}\PY{p}{,} \PY{n}{variance}\PY{o}{=}\PY{l+m+mi}{3}\PY{p}{)}\PY{p}{,} 
%    \PY{n}{label}\PY{o}{=}\PY{l+s+s2}{\PYZdq{}}\PY{l+s+s2}{\PYZdl{}}\PY{l+s+s2}{\PYZbs{}}\PY{l+s+s2}{mathcal}\PY{l+s+si}{\PYZob{}N\PYZcb{}}\PY{l+s+s2}{(2, 3)\PYZdl{}}\PY{l+s+s2}{\PYZdq{}}\PY{p}{)}
%\PY{n}{plt}\PY{o}{.}\PY{n}{plot}\PY{p}{(}
%    \PY{n}{x}\PY{p}{,} \PY{n}{univariate\PYZus{}normal}\PY{p}{(}\PY{n}{x}\PY{p}{,} \PY{n}{mean}\PY{o}{=}\PY{l+m+mi}{0}\PY{p}{,} \PY{n}{variance}\PY{o}{=}\PY{l+m+mf}{0.2}\PY{p}{)}\PY{p}{,} 
%    \PY{n}{label}\PY{o}{=}\PY{l+s+s2}{\PYZdq{}}\PY{l+s+s2}{\PYZdl{}}\PY{l+s+s2}{\PYZbs{}}\PY{l+s+s2}{mathcal}\PY{l+s+si}{\PYZob{}N\PYZcb{}}\PY{l+s+s2}{(0, 0.2)\PYZdl{}}\PY{l+s+s2}{\PYZdq{}}\PY{p}{)}
%\PY{n}{plt}\PY{o}{.}\PY{n}{xlabel}\PY{p}{(}\PY{l+s+s1}{\PYZsq{}}\PY{l+s+s1}{\PYZdl{}x\PYZdl{}}\PY{l+s+s1}{\PYZsq{}}\PY{p}{)}
%\PY{n}{plt}\PY{o}{.}\PY{n}{ylabel}\PY{p}{(}\PY{l+s+s1}{\PYZsq{}}\PY{l+s+s1}{density: \PYZdl{}p(x)\PYZdl{}}\PY{l+s+s1}{\PYZsq{}}\PY{p}{)}
%\PY{n}{plt}\PY{o}{.}\PY{n}{title}\PY{p}{(}\PY{l+s+s1}{\PYZsq{}}\PY{l+s+s1}{Одномерные нормальные распределения}\PY{l+s+s1}{\PYZsq{}}\PY{p}{)}
%\PY{n}{plt}\PY{o}{.}\PY{n}{ylim}\PY{p}{(}\PY{p}{[}\PY{l+m+mi}{0}\PY{p}{,} \PY{l+m+mi}{1}\PY{p}{]}\PY{p}{)}
%\PY{n}{plt}\PY{o}{.}\PY{n}{xlim}\PY{p}{(}\PY{p}{[}\PY{o}{\PYZhy{}}\PY{l+m+mi}{3}\PY{p}{,} \PY{l+m+mi}{5}\PY{p}{]}\PY{p}{)}
%\PY{n}{plt}\PY{o}{.}\PY{n}{legend}\PY{p}{(}\PY{n}{loc}\PY{o}{=}\PY{l+m+mi}{1}\PY{p}{)}
%\PY{c+c1}{\PYZsh{} fig.subplots\PYZus{}adjust(bottom=0.15)}
%\PY{n}{plt}\PY{o}{.}\PY{n}{show}\PY{p}{(}\PY{p}{)}
%\end{Verbatim}
%\end{tcolorbox}

    \begin{center}
    \adjustimage{max size={0.6\linewidth}{0.6\paperheight}}{output_8_0.png}
    \end{center}
    { \hspace*{\fill} \\}
    
%    \begin{tcolorbox}[breakable, size=fbox, boxrule=1pt, pad at break*=1mm,colback=cellbackground, colframe=cellborder]
%\prompt{In}{incolor}{17}{\boxspacing}
%\begin{Verbatim}[commandchars=\\\{\}]
%\PY{n}{N} \PY{o}{=} \PY{n+nb}{int}\PY{p}{(}\PY{l+m+mf}{1e5}\PY{p}{)}
%\PY{n}{m1}\PY{p}{,} \PY{n}{s1} \PY{o}{=} \PY{o}{\PYZhy{}}\PY{l+m+mf}{1.}\PY{p}{,} \PY{l+m+mf}{1.}
%\PY{n}{X1} \PY{o}{=} \PY{n}{np}\PY{o}{.}\PY{n}{random}\PY{o}{.}\PY{n}{normal}\PY{p}{(}\PY{n}{loc}\PY{o}{=}\PY{n}{m1}\PY{p}{,} \PY{n}{scale}\PY{o}{=}\PY{n}{s1}\PY{p}{,} \PY{n}{size}\PY{o}{=}\PY{n}{N}\PY{p}{)}
%\PY{n}{m2}\PY{p}{,} \PY{n}{s2} \PY{o}{=} \PY{l+m+mf}{2.}\PY{p}{,} \PY{l+m+mf}{0.8}
%\PY{n}{X2} \PY{o}{=} \PY{n}{np}\PY{o}{.}\PY{n}{random}\PY{o}{.}\PY{n}{normal}\PY{p}{(}\PY{n}{loc}\PY{o}{=}\PY{n}{m2}\PY{p}{,} \PY{n}{scale}\PY{o}{=}\PY{n}{s2}\PY{p}{,} \PY{n}{size}\PY{o}{=}\PY{n}{N}\PY{p}{)}
%\end{Verbatim}
%\end{tcolorbox}
%
%    \begin{tcolorbox}[breakable, size=fbox, boxrule=1pt, pad at break*=1mm,colback=cellbackground, colframe=cellborder]
%\prompt{In}{incolor}{25}{\boxspacing}
%\begin{Verbatim}[commandchars=\\\{\}]
%\PY{n}{bins} \PY{o}{=} \PY{l+m+mi}{100}
%\PY{n}{fig} \PY{o}{=} \PY{n}{plt}\PY{o}{.}\PY{n}{figure}\PY{p}{(}\PY{n}{figsize}\PY{o}{=}\PY{p}{(}\PY{l+m+mi}{8}\PY{p}{,} \PY{l+m+mi}{5}\PY{p}{)}\PY{p}{)}
%\PY{n}{ax} \PY{o}{=} \PY{n}{plt}\PY{o}{.}\PY{n}{subplot}\PY{p}{(}\PY{l+m+mi}{1}\PY{p}{,}\PY{l+m+mi}{1}\PY{p}{,}\PY{l+m+mi}{1}\PY{p}{)}
%\PY{n}{plt}\PY{o}{.}\PY{n}{hist}\PY{p}{(}\PY{n}{X1}\PY{p}{,} \PY{n}{bins}\PY{o}{=}\PY{n}{bins}\PY{p}{,} \PY{n}{density}\PY{o}{=}\PY{k+kc}{True}\PY{p}{,} \PY{n}{label}\PY{o}{=}\PY{l+s+sa}{f}\PY{l+s+s2}{\PYZdq{}}\PY{l+s+s2}{\PYZdl{}X\PYZus{}1=}\PY{l+s+s2}{\PYZbs{}}\PY{l+s+s2}{mathcal}\PY{l+s+se}{\PYZob{}\PYZob{}}\PY{l+s+s2}{N}\PY{l+s+se}{\PYZcb{}\PYZcb{}}\PY{l+s+s2}{(}\PY{l+s+si}{\PYZob{}}\PY{n}{m1}\PY{l+s+si}{\PYZcb{}}\PY{l+s+s2}{, }\PY{l+s+si}{\PYZob{}}\PY{n}{s1}\PY{o}{*}\PY{o}{*}\PY{l+m+mi}{2}\PY{l+s+si}{:}\PY{l+s+s2}{.2}\PY{l+s+si}{\PYZcb{}}\PY{l+s+s2}{)\PYZdl{}}\PY{l+s+s2}{\PYZdq{}}\PY{p}{)}
%\PY{n}{plt}\PY{o}{.}\PY{n}{hist}\PY{p}{(}\PY{n}{X2}\PY{p}{,} \PY{n}{bins}\PY{o}{=}\PY{n}{bins}\PY{p}{,} \PY{n}{density}\PY{o}{=}\PY{k+kc}{True}\PY{p}{,} \PY{n}{label}\PY{o}{=}\PY{l+s+sa}{f}\PY{l+s+s2}{\PYZdq{}}\PY{l+s+s2}{\PYZdl{}X\PYZus{}2=}\PY{l+s+s2}{\PYZbs{}}\PY{l+s+s2}{mathcal}\PY{l+s+se}{\PYZob{}\PYZob{}}\PY{l+s+s2}{N}\PY{l+s+se}{\PYZcb{}\PYZcb{}}\PY{l+s+s2}{(}\PY{l+s+si}{\PYZob{}}\PY{n}{m2}\PY{l+s+si}{\PYZcb{}}\PY{l+s+s2}{, }\PY{l+s+si}{\PYZob{}}\PY{n}{s2}\PY{o}{*}\PY{o}{*}\PY{l+m+mi}{2}\PY{l+s+si}{:}\PY{l+s+s2}{.2}\PY{l+s+si}{\PYZcb{}}\PY{l+s+s2}{)\PYZdl{}}\PY{l+s+s2}{\PYZdq{}}\PY{p}{)}
%\PY{n}{plt}\PY{o}{.}\PY{n}{hist}\PY{p}{(}\PY{n}{X1}\PY{o}{+}\PY{n}{X2}\PY{p}{,} \PY{n}{bins}\PY{o}{=}\PY{n}{bins}\PY{p}{,} \PY{n}{density}\PY{o}{=}\PY{k+kc}{True}\PY{p}{,} \PY{n}{alpha}\PY{o}{=}\PY{l+m+mf}{0.5}\PY{p}{,} \PY{n}{label}\PY{o}{=}\PY{l+s+s2}{\PYZdq{}}\PY{l+s+s2}{\PYZdl{}X\PYZus{}1+X\PYZus{}2\PYZdl{}}\PY{l+s+s2}{\PYZdq{}}\PY{p}{)}
%
%\PY{n}{ms}\PY{p}{,} \PY{n}{ss} \PY{o}{=} \PY{n}{m1}\PY{o}{+}\PY{n}{m2}\PY{p}{,} \PY{n}{s1}\PY{o}{*}\PY{o}{*}\PY{l+m+mi}{2}\PY{o}{+}\PY{n}{s2}\PY{o}{*}\PY{o}{*}\PY{l+m+mi}{2}
%\PY{n}{plt}\PY{o}{.}\PY{n}{plot}\PY{p}{(}\PY{n}{x}\PY{p}{,} \PY{n}{univariate\PYZus{}normal}\PY{p}{(}\PY{n}{x}\PY{p}{,} \PY{n}{mean}\PY{o}{=}\PY{n}{ms}\PY{p}{,} \PY{n}{variance}\PY{o}{=}\PY{n}{ss}\PY{p}{)}\PY{p}{,} \PY{n}{c}\PY{o}{=}\PY{l+s+s1}{\PYZsq{}}\PY{l+s+s1}{k}\PY{l+s+s1}{\PYZsq{}}\PY{p}{,}
%         \PY{n}{label}\PY{o}{=}\PY{l+s+sa}{f}\PY{l+s+s2}{\PYZdq{}}\PY{l+s+s2}{\PYZdl{}}\PY{l+s+s2}{\PYZbs{}}\PY{l+s+s2}{mathcal}\PY{l+s+se}{\PYZob{}\PYZob{}}\PY{l+s+s2}{N}\PY{l+s+se}{\PYZcb{}\PYZcb{}}\PY{l+s+s2}{(}\PY{l+s+si}{\PYZob{}}\PY{n}{ms}\PY{l+s+si}{\PYZcb{}}\PY{l+s+s2}{, }\PY{l+s+si}{\PYZob{}}\PY{n}{ss}\PY{l+s+si}{:}\PY{l+s+s2}{.3}\PY{l+s+si}{\PYZcb{}}\PY{l+s+s2}{)\PYZdl{}}\PY{l+s+s2}{\PYZdq{}}\PY{p}{)}
%
%\PY{n}{plt}\PY{o}{.}\PY{n}{xlabel}\PY{p}{(}\PY{l+s+s1}{\PYZsq{}}\PY{l+s+s1}{\PYZdl{}x\PYZdl{}}\PY{l+s+s1}{\PYZsq{}}\PY{p}{)}
%\PY{n}{plt}\PY{o}{.}\PY{n}{ylabel}\PY{p}{(}\PY{l+s+s1}{\PYZsq{}}\PY{l+s+s1}{density: \PYZdl{}p(x)\PYZdl{}}\PY{l+s+s1}{\PYZsq{}}\PY{p}{)}
%\PY{n}{plt}\PY{o}{.}\PY{n}{title}\PY{p}{(}\PY{l+s+s1}{\PYZsq{}}\PY{l+s+s1}{Одномерные нормальные распределения}\PY{l+s+s1}{\PYZsq{}}\PY{p}{)}
%\PY{n}{plt}\PY{o}{.}\PY{n}{xlim}\PY{p}{(}\PY{p}{[}\PY{o}{\PYZhy{}}\PY{l+m+mi}{3}\PY{p}{,} \PY{l+m+mi}{6}\PY{p}{]}\PY{p}{)}
%\PY{c+c1}{\PYZsh{} plt.ylim([0, 1])}
%
%\PY{n}{handles}\PY{p}{,} \PY{n}{labels} \PY{o}{=} \PY{n}{ax}\PY{o}{.}\PY{n}{get\PYZus{}legend\PYZus{}handles\PYZus{}labels}\PY{p}{(}\PY{p}{)}
%\PY{n}{plt}\PY{o}{.}\PY{n}{legend}\PY{p}{(}\PY{n}{handles}\PY{p}{[}\PY{l+m+mi}{1}\PY{p}{:}\PY{p}{]}\PY{o}{+}\PY{p}{[}\PY{n}{handles}\PY{p}{[}\PY{l+m+mi}{0}\PY{p}{]}\PY{p}{]}\PY{p}{,} \PY{n}{labels}\PY{p}{[}\PY{l+m+mi}{1}\PY{p}{:}\PY{p}{]}\PY{o}{+}\PY{p}{[}\PY{n}{labels}\PY{p}{[}\PY{l+m+mi}{0}\PY{p}{]}\PY{p}{]}\PY{p}{)}
%\PY{n}{plt}\PY{o}{.}\PY{n}{show}\PY{p}{(}\PY{p}{)}
%\end{Verbatim}
%\end{tcolorbox}

    \begin{center}
    \adjustimage{max size={0.6\linewidth}{0.6\paperheight}}{output_10_0.png}
    \end{center}
    { \hspace*{\fill} \\}
    
    \begin{center}\rule{0.5\linewidth}{\linethickness}\end{center}

    \hypertarget{ux43cux43dux43eux433ux43eux43cux435ux440ux43dux43eux435-ux43dux43eux440ux43cux430ux43bux44cux43dux43eux435-ux440ux430ux441ux43fux440ux435ux434ux435ux43bux435ux43dux438ux435}{%
\section{Многомерное нормальное
распределение}\label{ux43cux43dux43eux433ux43eux43cux435ux440ux43dux43eux435-ux43dux43eux440ux43cux430ux43bux44cux43dux43eux435-ux440ux430ux441ux43fux440ux435ux434ux435ux43bux435ux43dux438ux435}}

\hypertarget{ux441ux43bux443ux447ux430ux439ux43dux44bux439-ux432ux435ux43aux442ux43eux440}{%
\subsection{Случайный
вектор}\label{ux441ux43bux443ux447ux430ux439ux43dux44bux439-ux432ux435ux43aux442ux43eux440}}

\textbf{Определение.} Всякий упорядоченный набор случайных величин
\(\vec{\xi} = (\xi_1, \ldots, \xi_n)\) будем называть \emph{\(n\)-мерным
случайным вектором}.

\textbf{Определение.} Математическим ожиданием случайного вектора будем
называть вектор математических ожиданий его каждой компоненты:
\(\mathrm{E}\vec{\xi} = (\mathrm{E}\xi_1, \ldots, \mathrm{E}\xi_n)\).

Для математического ожидания случайного вектора справедливы все свойства
математического ожидания случайной величины. В том числе
\emph{линейность:}
\(\mathrm{E}(A\vec{\xi} + B\vec{\eta}) = A \cdot \mathrm{E}\vec{\xi} + B \cdot \mathrm{E}\vec{\eta}\).

    \hypertarget{ux43aux43eux432ux430ux440ux438ux430ux446ux438ux43eux43dux43dux430ux44f-ux43cux430ux442ux440ux438ux446ux430}{%
\subsection{Ковариационная
матрица}\label{ux43aux43eux432ux430ux440ux438ux430ux446ux438ux43eux43dux43dux430ux44f-ux43cux430ux442ux440ux438ux446ux430}}

Пусть \(\vec\xi = \left( \xi_1, \dots, \xi_n \right)\) --- случайный
вектор, компоненты которого имеют конечный второй момент. Назовём
\emph{матрицей ковариаций} (ковариационной матрицей) вектора \(\xi\)
матрицу (порядка \(n \times n\)) \(\Sigma = ||\Sigma_{ij}||\), где
\(\Sigma_{ij} = \text{cov}\left( \xi_i, \xi_j \right)\).

Ковариационная матрица случайного вектора является многомерным аналогом
дисперсии случайной величины для случайных векторов. На диагонали
\(\Sigma\) располагаются дисперсии компонент вектора, а внедиагональные
элементы --- ковариации между компонентами.

    \textbf{Свойства ковариационной матрицы:}

\begin{enumerate}
\def\labelenumi{\arabic{enumi}.}
\tightlist
\item
  \(\mathrm{cov}(\vec\xi) = \mathrm{E}\left[(\vec\xi -\mathrm{E}\vec\xi) \cdot (\vec\xi -\mathrm{E}\vec\xi)^\top \right]\)
\item
  \(\mathrm{cov}(\vec\xi) = \mathrm{E}[\vec\xi \cdot \vec\xi^\top] - \mathrm{E}[\vec\xi] \cdot \mathrm{E}[\vec\xi^\top]\)
\item
  Положительная полуопределённость: \(\mathrm{cov}(\vec\xi) \ge 0\)
\item
  Аффинное преобразование:
  \(\mathrm{cov}(A\vec\xi + \vec{b}) = A \cdot \mathrm{cov}(\vec\xi) \cdot A^\top\)
\end{enumerate}

    \textbf{Предложение}. Ковариационная матрица случайного вектора является
\emph{симметричной} и \emph{неотрицательно определённой}.

\begin{quote}
\emph{Упражнение.} Доказать, что ковариационная матрица случайного
вектора является \emph{неотрицательно определённой}.
\end{quote}

Справедлив и обратный результат.

\textbf{Предложение.} Для того, чтобы матрица \(\Sigma\) порядка
\(n \times n\) была ковариационной матрицей некоторого случайного
вектора \(\vec\xi = \left( \xi_1, \dots, \xi_n \right)\), необходимо и
достаточно, чтобы эта матрица была симметричной и положительно
определённой.

\emph{Доказательство}. Тот факт, что всякая ковариационная матрица
является симетричной и положительно определённой будем считать
доказанным. Покажем теперь обратное, что \(\Sigma\) является
ковариационной матрицей некоторого случайного вектора.

Пусть \(\vec\eta\) --- вектор нормально распределённых случайных величин
\(\vec\eta \sim \mathcal{N}(0, 1)\). Покажем, что вектор
\(\vec\xi = L\vec\eta\) имеет ковариационную матрицу \(\Sigma\).

Воспользуемся \emph{разложением Холецкого} --- представлением
симметричной положительно определённой матрицы в виде произведения
нижнетреугольной матрицы \(L\) и верхнетреугольной матрицы \(L^\top\).

\[
\mathrm{cov}(\vec\xi) = \mathrm{E} \vec\xi \cdot \vec\xi^\top = \mathrm{E}(L\vec\eta)(L\vec\eta)^\top = L \cdot \mathrm{E}\vec\eta \cdot \vec\eta^\top \cdot L^\top = L I_d L^\top = LL^\top = \Sigma. \mathrm{\square}
\]

    \hypertarget{ux43cux43dux43eux433ux43eux43cux435ux440ux43dux43eux435-ux43dux43eux440ux43cux430ux43bux44cux43dux43eux435-ux440ux430ux441ux43fux440ux435ux434ux435ux43bux435ux43dux438ux435}{%
\subsection{Многомерное нормальное
распределение}\label{ux43cux43dux43eux433ux43eux43cux435ux440ux43dux43eux435-ux43dux43eux440ux43cux430ux43bux44cux43dux43eux435-ux440ux430ux441ux43fux440ux435ux434ux435ux43bux435ux43dux438ux435}}

Многомерное нормальное распределение представляет собой многомерное
обобщение одномерного нормального распределения. Оно представляет собой
распределение многомерной случайной величины, состоящей из нескольких
случайных величин, которые могут быть скоррелированы друг с другом.

Как и одномерное, многомерное нормальное распределение определяется
набором параметров: вектором средних значений \(\mathbf{\mu}\), который
является вектором математических ожиданий распределения, и
ковариоционной матрицей \(\Sigma\), которая измеряет степень зависимости
двух случайных величин и их совместного изменения.

Многомерное нормальное распределение случайного вектора
\(\overline{\xi}\) размерностью \(n\) имеет следующую функцию плотности
совместной вероятности:

\[
f_n(\vec{x}|\vec{\mu}, \Sigma) = \frac{1}{\sqrt{(2\pi)^n |\Sigma|}} \exp{ \left( -\frac{1}{2}(\vec{x} - \vec{\mu})^\top \Sigma^{-1} (\vec{x} - \vec{\mu}) \right)}.
\]

Здесь \(\vec{x}\) --- случайный вектор размерностью \(n\), \(\vec{\mu}\)
--- вектор математического ожидания, \(\Sigma\) --- ковариационная
матрица (симметричная, положительно определённая матрица с размерностью
\(n \times n\), \(\Sigma_{ij} = \text{cov}(\xi_i, \xi_j)\)), а
\(\lvert\Sigma\rvert\) --- её определитель. Многомерное нормальное
распределение принято обозначать следующим образом:

\[
    \vec{\xi} \sim \mathcal{N}(\vec{\mu}, \Sigma)
\]

\begin{quote}
\emph{Замечание.} Далее для простоты записи стрелка над вектором будет
опускаться, т. е. вместо \(\vec{\xi}\) будем писать просто \(\xi\).
\end{quote}

%    \begin{tcolorbox}[breakable, size=fbox, boxrule=1pt, pad at break*=1mm,colback=cellbackground, colframe=cellborder]
%\prompt{In}{incolor}{7}{\boxspacing}
%\begin{Verbatim}[commandchars=\\\{\}]
%\PY{k}{def} \PY{n+nf}{multivariate\PYZus{}normal}\PY{p}{(}\PY{n}{x}\PY{p}{,} \PY{n}{d}\PY{p}{,} \PY{n}{mean}\PY{p}{,} \PY{n}{covariance}\PY{p}{)}\PY{p}{:}
%    \PY{l+s+sd}{\PYZdq{}\PYZdq{}\PYZdq{}pdf of the multivariate normal distribution.\PYZdq{}\PYZdq{}\PYZdq{}}
%    \PY{n}{x\PYZus{}m} \PY{o}{=} \PY{n}{x} \PY{o}{\PYZhy{}} \PY{n}{mean}
%    \PY{k}{return} \PY{p}{(}\PY{l+m+mf}{1.} \PY{o}{/} \PY{p}{(}\PY{n}{np}\PY{o}{.}\PY{n}{sqrt}\PY{p}{(}\PY{p}{(}\PY{l+m+mi}{2} \PY{o}{*} \PY{n}{np}\PY{o}{.}\PY{n}{pi}\PY{p}{)}\PY{o}{*}\PY{o}{*}\PY{n}{d} \PY{o}{*} \PY{n}{np}\PY{o}{.}\PY{n}{linalg}\PY{o}{.}\PY{n}{det}\PY{p}{(}\PY{n}{covariance}\PY{p}{)}\PY{p}{)}\PY{p}{)} \PY{o}{*} 
%            \PY{n}{np}\PY{o}{.}\PY{n}{exp}\PY{p}{(}\PY{o}{\PYZhy{}}\PY{l+m+mf}{0.5}\PY{o}{*}\PY{p}{(}\PY{n}{np}\PY{o}{.}\PY{n}{linalg}\PY{o}{.}\PY{n}{solve}\PY{p}{(}\PY{n}{covariance}\PY{p}{,} \PY{n}{x\PYZus{}m}\PY{p}{)}\PY{o}{.}\PY{n}{T}\PY{o}{.}\PY{n}{dot}\PY{p}{(}\PY{n}{x\PYZus{}m}\PY{p}{)}\PY{p}{)}\PY{p}{)}\PY{p}{)}
%\end{Verbatim}
%\end{tcolorbox}

    \hypertarget{ux434ux432ux443ux43cux435ux440ux43dux43eux435-ux43dux43eux440ux43cux430ux43bux44cux43dux43eux435-ux440ux430ux441ux43fux440ux435ux434ux435ux43bux435ux43dux438ux435}{%
\subsection{Двумерное нормальное
распределение}\label{ux434ux432ux443ux43cux435ux440ux43dux43eux435-ux43dux43eux440ux43cux430ux43bux44cux43dux43eux435-ux440ux430ux441ux43fux440ux435ux434ux435ux43bux435ux43dux438ux435}}

В качестве примера рассмотрим двумерный случайный вектор. В этом случае
плотность \(p(x_1, x_2)\) может быть приведена к виду \[
    f_{\xi,\eta}(x_1, x_2) = \frac{1}{2\pi\sigma_1\sigma_2\sqrt{1-\rho^2}} \, \exp\left\{-\frac{1}{2(1-\rho^2)} \left[ \frac{(x_1-m_1)^2}{\sigma_1^2} - 2\rho\frac{(x_1-m_1)(x_2-m_2)}{\sigma_1\sigma_2} + \frac{(x_2-m_2)^2}{\sigma_2^2} \right]\right\},
\]

где\\
\(m_1 = \mathrm{E} \xi\), \(m_2 = \mathrm{E} \eta\) --- математические
ожидания,\\
\(\sigma_1^2 = \mathrm{D} \xi\), \(\sigma_2^2 = \mathrm{D} \eta\) ---
стандартное отклонение \(x_i\),\\
\(\rho = \dfrac{\mathrm{cov}(\xi, \eta)}{\sigma_1 \cdot \sigma_2} \)---
коэффициент корреляции.

\textbf{Замечание.} Можно убедиться, что если пара (\(\xi\), \(\eta\))
--- гауссовская, то из некоррелированности \(\xi\) и \(\eta\) следует их
независимость.\\
Действительно, если \(\rho=0\), то \[
    f_{\xi,\eta}(x_1, x_2) = \frac{1}{2\pi\sigma_1\sigma_2} \, \exp\left\{-\frac{(x_1-m_1)^2}{2\sigma_1^2}\right\} \, \exp\left\{-\frac{(x_2-m_2)^2}{2\sigma_2^2}\right\}
    = f_\xi(x_1) \cdot f_{\eta}(x_2).
\]

    \hypertarget{ux43fux440ux438ux43cux435ux440ux44b}{%
\paragraph{Примеры}\label{ux43fux440ux438ux43cux435ux440ux44b}}

Примеры двумерных нормальных распределений приведены ниже.

\begin{enumerate}
\def\labelenumi{\arabic{enumi}.}
\item
  Двумерное распределение с ковариацией между \(x_1\) и \(x_2\) равной
  \(0\) (независимые переменные): \[
  \mathcal{N}\left(
  \begin{bmatrix}
   0 \\
   0
  \end{bmatrix}, 
  \begin{bmatrix}
   1 & 0 \\
   0 & 1 
  \end{bmatrix}\right)
  \]
\item
  Двумерное распределение параметров \(x_1\) и \(x_2\) с отличной от
  \(0\) ковариацией (скоррелированые переменные): \[
  \mathcal{N}\left(
  \begin{bmatrix}
   0 \\
   0.5
  \end{bmatrix}, 
  \begin{bmatrix}
   1 & 0.8 \\
   0.8 & 1
  \end{bmatrix}\right)
  \] Увеличение \(x_1\) увеличивает вероятность того, что \(x_2\) также
  увеличится.
\end{enumerate}

%    \begin{tcolorbox}[breakable, size=fbox, boxrule=1pt, pad at break*=1mm,colback=cellbackground, colframe=cellborder]
%\prompt{In}{incolor}{8}{\boxspacing}
%\begin{Verbatim}[commandchars=\\\{\}]
%\PY{c+c1}{\PYZsh{} Plot bivariate distribution}
%\PY{k}{def} \PY{n+nf}{generate\PYZus{}surface}\PY{p}{(}\PY{n}{mean}\PY{p}{,} \PY{n}{covariance}\PY{p}{,} \PY{n}{d}\PY{p}{)}\PY{p}{:}
%    \PY{l+s+sd}{\PYZdq{}\PYZdq{}\PYZdq{}Helper function to generate density surface.\PYZdq{}\PYZdq{}\PYZdq{}}
%    \PY{n}{nb\PYZus{}of\PYZus{}x} \PY{o}{=} \PY{l+m+mi}{100} \PY{c+c1}{\PYZsh{} grid size}
%    \PY{n}{x1s} \PY{o}{=} \PY{n}{np}\PY{o}{.}\PY{n}{linspace}\PY{p}{(}\PY{o}{\PYZhy{}}\PY{l+m+mi}{5}\PY{p}{,} \PY{l+m+mi}{5}\PY{p}{,} \PY{n}{num}\PY{o}{=}\PY{n}{nb\PYZus{}of\PYZus{}x}\PY{p}{)}
%    \PY{n}{x2s} \PY{o}{=} \PY{n}{np}\PY{o}{.}\PY{n}{linspace}\PY{p}{(}\PY{o}{\PYZhy{}}\PY{l+m+mi}{5}\PY{p}{,} \PY{l+m+mi}{5}\PY{p}{,} \PY{n}{num}\PY{o}{=}\PY{n}{nb\PYZus{}of\PYZus{}x}\PY{p}{)}
%    \PY{n}{x1}\PY{p}{,} \PY{n}{x2} \PY{o}{=} \PY{n}{np}\PY{o}{.}\PY{n}{meshgrid}\PY{p}{(}\PY{n}{x1s}\PY{p}{,} \PY{n}{x2s}\PY{p}{)} \PY{c+c1}{\PYZsh{} Generate grid}
%    \PY{n}{pdf} \PY{o}{=} \PY{n}{np}\PY{o}{.}\PY{n}{zeros}\PY{p}{(}\PY{p}{(}\PY{n}{nb\PYZus{}of\PYZus{}x}\PY{p}{,} \PY{n}{nb\PYZus{}of\PYZus{}x}\PY{p}{)}\PY{p}{)}
%    \PY{c+c1}{\PYZsh{} Fill the cost matrix for each combination of weights}
%    \PY{k}{for} \PY{n}{i} \PY{o+ow}{in} \PY{n+nb}{range}\PY{p}{(}\PY{n}{nb\PYZus{}of\PYZus{}x}\PY{p}{)}\PY{p}{:}
%        \PY{k}{for} \PY{n}{j} \PY{o+ow}{in} \PY{n+nb}{range}\PY{p}{(}\PY{n}{nb\PYZus{}of\PYZus{}x}\PY{p}{)}\PY{p}{:}
%            \PY{n}{pdf}\PY{p}{[}\PY{n}{i}\PY{p}{,}\PY{n}{j}\PY{p}{]} \PY{o}{=} \PY{n}{multivariate\PYZus{}normal}\PY{p}{(}
%                \PY{n}{np}\PY{o}{.}\PY{n}{matrix}\PY{p}{(}\PY{p}{[}\PY{p}{[}\PY{n}{x1}\PY{p}{[}\PY{n}{i}\PY{p}{,}\PY{n}{j}\PY{p}{]}\PY{p}{]}\PY{p}{,} \PY{p}{[}\PY{n}{x2}\PY{p}{[}\PY{n}{i}\PY{p}{,}\PY{n}{j}\PY{p}{]}\PY{p}{]}\PY{p}{]}\PY{p}{)}\PY{p}{,} 
%                \PY{n}{d}\PY{p}{,} \PY{n}{mean}\PY{p}{,} \PY{n}{covariance}\PY{p}{)}
%    \PY{k}{return} \PY{n}{x1}\PY{p}{,} \PY{n}{x2}\PY{p}{,} \PY{n}{pdf}  \PY{c+c1}{\PYZsh{} x1, x2, pdf(x1,x2)}
%\end{Verbatim}
%\end{tcolorbox}

%    \begin{tcolorbox}[breakable, size=fbox, boxrule=1pt, pad at break*=1mm,colback=cellbackground, colframe=cellborder]
%\prompt{In}{incolor}{13}{\boxspacing}
%\begin{Verbatim}[commandchars=\\\{\}]
%\PY{c+c1}{\PYZsh{} subplot}
%\PY{n}{fig}\PY{p}{,} \PY{p}{(}\PY{n}{ax1}\PY{p}{,} \PY{n}{ax2}\PY{p}{)} \PY{o}{=} \PY{n}{plt}\PY{o}{.}\PY{n}{subplots}\PY{p}{(}\PY{n}{nrows}\PY{o}{=}\PY{l+m+mi}{1}\PY{p}{,} \PY{n}{ncols}\PY{o}{=}\PY{l+m+mi}{2}\PY{p}{,} \PY{n}{figsize}\PY{o}{=}\PY{p}{(}\PY{l+m+mi}{8}\PY{p}{,}\PY{l+m+mi}{4}\PY{p}{)}\PY{p}{)}
%\PY{n}{d} \PY{o}{=} \PY{l+m+mi}{2}  \PY{c+c1}{\PYZsh{} number of dimensions}
%\PY{n}{cmap} \PY{o}{=} \PY{n}{cm}\PY{o}{.}\PY{n}{magma\PYZus{}r}
%
%\PY{c+c1}{\PYZsh{} Generate independent Normals}
%\PY{n}{bivariate\PYZus{}mean} \PY{o}{=} \PY{n}{np}\PY{o}{.}\PY{n}{matrix}\PY{p}{(}\PY{p}{[}\PY{p}{[}\PY{l+m+mf}{0.}\PY{p}{]}\PY{p}{,} \PY{p}{[}\PY{l+m+mf}{0.}\PY{p}{]}\PY{p}{]}\PY{p}{)}  \PY{c+c1}{\PYZsh{} Mean}
%\PY{n}{bivariate\PYZus{}covariance} \PY{o}{=} \PY{n}{np}\PY{o}{.}\PY{n}{matrix}\PY{p}{(}\PY{p}{[}
%    \PY{p}{[}\PY{l+m+mf}{1.}\PY{p}{,} \PY{l+m+mf}{0.}\PY{p}{]}\PY{p}{,} 
%    \PY{p}{[}\PY{l+m+mf}{0.}\PY{p}{,} \PY{l+m+mf}{1.}\PY{p}{]}\PY{p}{]}\PY{p}{)}  \PY{c+c1}{\PYZsh{} Covariance}
%\PY{n}{x1}\PY{p}{,} \PY{n}{x2}\PY{p}{,} \PY{n}{p} \PY{o}{=} \PY{n}{generate\PYZus{}surface}\PY{p}{(}
%    \PY{n}{bivariate\PYZus{}mean}\PY{p}{,} \PY{n}{bivariate\PYZus{}covariance}\PY{p}{,} \PY{n}{d}\PY{p}{)}
%
%\PY{c+c1}{\PYZsh{} Plot bivariate distribution 1}
%\PY{n}{con} \PY{o}{=} \PY{n}{ax1}\PY{o}{.}\PY{n}{contourf}\PY{p}{(}\PY{n}{x1}\PY{p}{,} \PY{n}{x2}\PY{p}{,} \PY{n}{p}\PY{p}{,} \PY{l+m+mi}{100}\PY{p}{,} \PY{n}{cmap}\PY{o}{=}\PY{n}{cmap}\PY{p}{)}
%\PY{c+c1}{\PYZsh{} Plot 95\PYZpc{} Interval}
%\PY{c+c1}{\PYZsh{} e = make\PYZus{}ellipse(bivariate\PYZus{}mean, bivariate\PYZus{}covariance)}
%\PY{c+c1}{\PYZsh{} ax1.add\PYZus{}artist(e)}
%\PY{n}{ax1}\PY{o}{.}\PY{n}{set\PYZus{}xlabel}\PY{p}{(}\PY{l+s+s1}{\PYZsq{}}\PY{l+s+s1}{\PYZdl{}x\PYZus{}1\PYZdl{}}\PY{l+s+s1}{\PYZsq{}}\PY{p}{)}
%\PY{n}{ax1}\PY{o}{.}\PY{n}{set\PYZus{}ylabel}\PY{p}{(}\PY{l+s+s1}{\PYZsq{}}\PY{l+s+s1}{\PYZdl{}x\PYZus{}2\PYZdl{}}\PY{l+s+s1}{\PYZsq{}}\PY{p}{,} \PY{n}{va}\PY{o}{=}\PY{l+s+s1}{\PYZsq{}}\PY{l+s+s1}{center}\PY{l+s+s1}{\PYZsq{}}\PY{p}{)}
%\PY{n}{ax1}\PY{o}{.}\PY{n}{axis}\PY{p}{(}\PY{p}{[}\PY{o}{\PYZhy{}}\PY{l+m+mf}{3.}\PY{p}{,} \PY{l+m+mf}{3.}\PY{p}{,} \PY{o}{\PYZhy{}}\PY{l+m+mf}{3.}\PY{p}{,} \PY{l+m+mf}{3.}\PY{p}{]}\PY{p}{)}
%\PY{n}{ax1}\PY{o}{.}\PY{n}{set\PYZus{}aspect}\PY{p}{(}\PY{l+s+s1}{\PYZsq{}}\PY{l+s+s1}{equal}\PY{l+s+s1}{\PYZsq{}}\PY{p}{)}
%\PY{n}{ax1}\PY{o}{.}\PY{n}{set\PYZus{}title}\PY{p}{(}\PY{l+s+s1}{\PYZsq{}}\PY{l+s+s1}{Независимые величины}\PY{l+s+s1}{\PYZsq{}}\PY{p}{)}
%
%
%\PY{c+c1}{\PYZsh{} Generate correlated Normals}
%\PY{n}{bivariate\PYZus{}mean} \PY{o}{=} \PY{n}{np}\PY{o}{.}\PY{n}{matrix}\PY{p}{(}\PY{p}{[}\PY{p}{[}\PY{l+m+mf}{0.}\PY{p}{]}\PY{p}{,} \PY{p}{[}\PY{l+m+mf}{0.5}\PY{p}{]}\PY{p}{]}\PY{p}{)}  \PY{c+c1}{\PYZsh{} Mean}
%\PY{n}{bivariate\PYZus{}covariance} \PY{o}{=} \PY{n}{np}\PY{o}{.}\PY{n}{matrix}\PY{p}{(}\PY{p}{[}
%    \PY{p}{[}\PY{l+m+mf}{1.}\PY{p}{,} \PY{l+m+mf}{0.8}\PY{p}{]}\PY{p}{,} 
%    \PY{p}{[}\PY{l+m+mf}{0.8}\PY{p}{,} \PY{l+m+mf}{1.}\PY{p}{]}\PY{p}{]}\PY{p}{)}  \PY{c+c1}{\PYZsh{} Covariance}
%\PY{n}{x1}\PY{p}{,} \PY{n}{x2}\PY{p}{,} \PY{n}{p} \PY{o}{=} \PY{n}{generate\PYZus{}surface}\PY{p}{(}
%    \PY{n}{bivariate\PYZus{}mean}\PY{p}{,} \PY{n}{bivariate\PYZus{}covariance}\PY{p}{,} \PY{n}{d}\PY{p}{)}
%
%\PY{c+c1}{\PYZsh{} Plot bivariate distribution 2}
%\PY{n}{con} \PY{o}{=} \PY{n}{ax2}\PY{o}{.}\PY{n}{contourf}\PY{p}{(}\PY{n}{x1}\PY{p}{,} \PY{n}{x2}\PY{p}{,} \PY{n}{p}\PY{p}{,} \PY{l+m+mi}{100}\PY{p}{,} \PY{n}{cmap}\PY{o}{=}\PY{n}{cmap}\PY{p}{)}
%\PY{c+c1}{\PYZsh{} Plot 95\PYZpc{} Interval}
%\PY{c+c1}{\PYZsh{} e = make\PYZus{}ellipse(bivariate\PYZus{}mean, bivariate\PYZus{}covariance)}
%\PY{c+c1}{\PYZsh{} ax2.add\PYZus{}artist(e)}
%\PY{n}{ax2}\PY{o}{.}\PY{n}{set\PYZus{}xlabel}\PY{p}{(}\PY{l+s+s1}{\PYZsq{}}\PY{l+s+s1}{\PYZdl{}x\PYZus{}1\PYZdl{}}\PY{l+s+s1}{\PYZsq{}}\PY{p}{)}
%\PY{n}{ax2}\PY{o}{.}\PY{n}{set\PYZus{}ylabel}\PY{p}{(}\PY{l+s+s1}{\PYZsq{}}\PY{l+s+s1}{\PYZdl{}x\PYZus{}2\PYZdl{}}\PY{l+s+s1}{\PYZsq{}}\PY{p}{,} \PY{n}{va}\PY{o}{=}\PY{l+s+s1}{\PYZsq{}}\PY{l+s+s1}{center}\PY{l+s+s1}{\PYZsq{}}\PY{p}{)}
%\PY{n}{ax2}\PY{o}{.}\PY{n}{axis}\PY{p}{(}\PY{p}{[}\PY{o}{\PYZhy{}}\PY{l+m+mf}{3.}\PY{p}{,} \PY{l+m+mf}{3.}\PY{p}{,} \PY{o}{\PYZhy{}}\PY{l+m+mf}{3.}\PY{p}{,} \PY{l+m+mf}{3.}\PY{p}{]}\PY{p}{)}
%\PY{n}{ax2}\PY{o}{.}\PY{n}{set\PYZus{}aspect}\PY{p}{(}\PY{l+s+s1}{\PYZsq{}}\PY{l+s+s1}{equal}\PY{l+s+s1}{\PYZsq{}}\PY{p}{)}
%\PY{n}{ax2}\PY{o}{.}\PY{n}{set\PYZus{}title}\PY{p}{(}\PY{l+s+s1}{\PYZsq{}}\PY{l+s+s1}{Коррелированные величины}\PY{l+s+s1}{\PYZsq{}}\PY{p}{)}
%
%\PY{c+c1}{\PYZsh{} Add colorbar and title}
%\PY{n}{fig}\PY{o}{.}\PY{n}{subplots\PYZus{}adjust}\PY{p}{(}\PY{n}{right}\PY{o}{=}\PY{l+m+mf}{0.8}\PY{p}{)}
%\PY{n}{cbar\PYZus{}ax} \PY{o}{=} \PY{n}{fig}\PY{o}{.}\PY{n}{add\PYZus{}axes}\PY{p}{(}\PY{p}{[}\PY{l+m+mf}{0.85}\PY{p}{,} \PY{l+m+mf}{0.15}\PY{p}{,} \PY{l+m+mf}{0.02}\PY{p}{,} \PY{l+m+mf}{0.7}\PY{p}{]}\PY{p}{)}
%\PY{n}{cbar} \PY{o}{=} \PY{n}{fig}\PY{o}{.}\PY{n}{colorbar}\PY{p}{(}\PY{n}{con}\PY{p}{,} \PY{n}{cax}\PY{o}{=}\PY{n}{cbar\PYZus{}ax}\PY{p}{)}
%\PY{n}{cbar}\PY{o}{.}\PY{n}{ax}\PY{o}{.}\PY{n}{set\PYZus{}ylabel}\PY{p}{(}\PY{l+s+s1}{\PYZsq{}}\PY{l+s+s1}{\PYZdl{}p(x\PYZus{}1, x\PYZus{}2)\PYZdl{}}\PY{l+s+s1}{\PYZsq{}}\PY{p}{)}
%\PY{n}{plt}\PY{o}{.}\PY{n}{suptitle}\PY{p}{(}\PY{l+s+s1}{\PYZsq{}}\PY{l+s+s1}{Двумерное нормальное распределение}\PY{l+s+s1}{\PYZsq{}}\PY{p}{,} \PY{n}{y}\PY{o}{=}\PY{l+m+mf}{0.95}\PY{p}{)}
%\PY{n}{plt}\PY{o}{.}\PY{n}{show}\PY{p}{(}\PY{p}{)}
%\end{Verbatim}
%\end{tcolorbox}

    \begin{center}
    \adjustimage{max size={0.9\linewidth}{0.9\paperheight}}{output_21_0.png}
    \end{center}
    { \hspace*{\fill} \\}
    
    \hypertarget{ux430ux444ux444ux438ux43dux43dux43eux435-ux43fux440ux435ux43eux431ux440ux430ux437ux43eux432ux430ux43dux438ux435-ux43cux43dux43eux433ux43eux43cux435ux440ux43dux43eux433ux43e-ux43dux43eux440ux43cux430ux43bux44cux43dux43eux433ux43e-ux440ux430ux441ux43fux440ux435ux434ux435ux43bux435ux43dux438ux44f}{%
\subsection{Аффинное преобразование многомерного нормального
распределения}\label{ux430ux444ux444ux438ux43dux43dux43eux435-ux43fux440ux435ux43eux431ux440ux430ux437ux43eux432ux430ux43dux438ux435-ux43cux43dux43eux433ux43eux43cux435ux440ux43dux43eux433ux43e-ux43dux43eux440ux43cux430ux43bux44cux43dux43eux433ux43e-ux440ux430ux441ux43fux440ux435ux434ux435ux43bux435ux43dux438ux44f}}

Многомерное нормальное распределение можно преобразовать с помощью
аффинного преобразования. Так, если \(X\) --- нормально распределённый
случайный вектор, а \(Y = u + LX\) --- аффинное преобразованием \(X\) с
матрицей \(L\) и вектором \(u\), то \(Y\) также нормально распределён со
средним значением \(\mu_{Y} = u + L\mu_{X}\) и ковариационной матрицей
\(\Sigma_{Y} = L\Sigma_{X}L^\top\).

\[X \sim \mathcal{N}(\mu_{X}, \Sigma_{X}) \quad\quad Y \sim \mathcal{N}(\mu_{Y}, \Sigma_{Y}) \\
\mathcal{N}(\mu_{Y}, \Sigma_{Y}) = \mathcal{N}(u + L\mu_{X}, L\Sigma_{X}L^\top) = u + L\mathcal{N}(\mu_{X}, \Sigma_{X})\]

Это можно доказать следующим образом:

\[\mu_{Y} = \mathrm{E}[Y] = \mathrm{E}[u + LX] = u + \mathrm{E}[LX] = u + L\mu_{X}\]

\[\begin{split}
\Sigma_{Y} & = \mathrm{E}[(Y-\mu_{Y})(Y-\mu_{Y})^\top] \\
           & = \mathrm{E}[(u+LX - u-L\mu_{X})(u+LX - u-L\mu_{X})^\top] \\
           & = \mathrm{E}[(L(X-\mu_{X})) (L(X-\mu_{X}))^\top] \\
           & = \mathrm{E}[L(X-\mu_{X}) (X-\mu_{X})^\top L^\top] \\
           & = L\mathrm{E}[(X-\mu_{X})(X-\mu_{X})^\top]L^\top \\
           & = L\Sigma_{X}L^\top
\end{split}\]

    \hypertarget{ux433ux435ux43dux435ux440ux430ux446ux438ux44f-ux432ux44bux431ux43eux440ux43aux438-ux433ux430ux443ux441ux441ux43eux432ux441ux43aux438ux445-ux432ux435ux43aux442ux43eux440ux43eux432}{%
\subsection{Генерация выборки гауссовских
векторов}\label{ux433ux435ux43dux435ux440ux430ux446ux438ux44f-ux432ux44bux431ux43eux440ux43aux438-ux433ux430ux443ux441ux441ux43eux432ux441ux43aux438ux445-ux432ux435ux43aux442ux43eux440ux43eux432}}

Предыдущая формула поможет нам сгенерировать гауссовский вектор с
заданными вектором средних знгачений и ковариационной матрицей.\\
Для этого сгенерируем вектор \(X\), подчиняющийся стандартному
нормальному распределению \(X \sim \mathcal{N}(0, I_d)\) со средним
значением \(\mu_{X} = 0\) и единичной ковариационной матрицей
\(\Sigma_{X} = I_d\). Генерация такого вектора не представляет труда,
так как каждая переменная в \(X\) независима от всех других переменных,
и мы можем просто генерировать каждую переменную отдельно, пользуясь
одномерным распределением Гаусса.

Для генерации \(Y \sim \mathcal{N}(\mu_{Y}, \Sigma_{Y})\) возьмём \(X\)
и применим к нему аффинное преобразование \(Y = u + LX\). Из предыдущего
раздела мы знаем, что ковариация \(Y\) будет
\(\Sigma_{Y} = L\Sigma_{X}L^\top\). Поскольку \(\Sigma_{X}=I_d\), а
\(\mu_{X} = 0\), то \(\Sigma_{Y} = L L^\top\) и \(\mu_{Y} = u\). В итоге
получаем, что искомое преобразование \(Y = \mu_{Y} + L_{Y}X\), где
матрица \(L_{Y}\) --- нижнетреугольная матрица, которую можно найти с
помощью разложения Холецкого матрицы \(\Sigma_{Y}\).

В качестве иллюстрации сгенерируем 100 двумерных векторов для следующего
распределения:

\[
Y
\sim
\mathcal{N}\left(
\begin{bmatrix}
    0 \\ 
    1 
\end{bmatrix},
\begin{bmatrix}
    1 & 0.8 \\
    0.8 & 1
\end{bmatrix}\right).
\]

%    \begin{tcolorbox}[breakable, size=fbox, boxrule=1pt, pad at break*=1mm,colback=cellbackground, colframe=cellborder]
%\prompt{In}{incolor}{38}{\boxspacing}
%\begin{Verbatim}[commandchars=\\\{\}]
%\PY{k}{def} \PY{n+nf}{make\PYZus{}ellipse}\PY{p}{(}\PY{n}{mu}\PY{p}{,} \PY{n}{cov}\PY{p}{,} \PY{n}{ci}\PY{o}{=}\PY{l+m+mf}{0.95}\PY{p}{,} \PY{n}{color}\PY{o}{=}\PY{l+s+s1}{\PYZsq{}}\PY{l+s+s1}{gray}\PY{l+s+s1}{\PYZsq{}}\PY{p}{,} \PY{n}{label}\PY{o}{=}\PY{l+s+s1}{\PYZsq{}}\PY{l+s+s1}{\PYZdl{}}\PY{l+s+s1}{\PYZbs{}}\PY{l+s+s1}{pm 2}\PY{l+s+s1}{\PYZbs{}}\PY{l+s+s1}{,}\PY{l+s+s1}{\PYZbs{}}\PY{l+s+s1}{sigma\PYZdl{}}\PY{l+s+s1}{\PYZsq{}}\PY{p}{)}\PY{p}{:}
%    \PY{l+s+sd}{\PYZdq{}\PYZdq{}\PYZdq{}Make covariance isoline\PYZdq{}\PYZdq{}\PYZdq{}}
%    \PY{n}{e}\PY{p}{,} \PY{n}{v} \PY{o}{=} \PY{n}{np}\PY{o}{.}\PY{n}{linalg}\PY{o}{.}\PY{n}{eig}\PY{p}{(}\PY{n}{cov}\PY{p}{)}
%    \PY{n}{angle} \PY{o}{=} \PY{n}{np}\PY{o}{.}\PY{n}{sign}\PY{p}{(}\PY{n}{v}\PY{p}{[}\PY{l+m+mi}{1}\PY{p}{,} \PY{l+m+mi}{0}\PY{p}{]}\PY{p}{)} \PY{o}{*} \PY{l+m+mi}{180}\PY{o}{/}\PY{n}{np}\PY{o}{.}\PY{n}{pi} \PY{o}{*} \PY{n}{np}\PY{o}{.}\PY{n}{arccos}\PY{p}{(}\PY{n}{v}\PY{p}{[}\PY{l+m+mi}{0}\PY{p}{,} \PY{l+m+mi}{0}\PY{p}{]}\PY{p}{)}
%    \PY{n}{q} \PY{o}{=} \PY{n}{stats}\PY{o}{.}\PY{n}{chi2}\PY{p}{(}\PY{l+m+mi}{2}\PY{p}{)}\PY{o}{.}\PY{n}{ppf}\PY{p}{(}\PY{n}{ci}\PY{p}{)}
%    \PY{n}{e} \PY{o}{=} \PY{n}{Ellipse}\PY{p}{(}\PY{n}{mu}\PY{p}{,} \PY{l+m+mi}{2}\PY{o}{*}\PY{n}{np}\PY{o}{.}\PY{n}{sqrt}\PY{p}{(}\PY{n}{q}\PY{o}{*}\PY{n}{e}\PY{p}{[}\PY{l+m+mi}{0}\PY{p}{]}\PY{p}{)}\PY{p}{,} \PY{l+m+mi}{2}\PY{o}{*}\PY{n}{np}\PY{o}{.}\PY{n}{sqrt}\PY{p}{(}\PY{n}{q}\PY{o}{*}\PY{n}{e}\PY{p}{[}\PY{l+m+mi}{1}\PY{p}{]}\PY{p}{)}\PY{p}{,} \PY{n}{angle}\PY{o}{=}\PY{n}{angle}\PY{p}{,}
%                \PY{n}{fill}\PY{o}{=}\PY{k+kc}{False}\PY{p}{,} \PY{n}{color}\PY{o}{=}\PY{n}{color}\PY{p}{,} \PY{n}{label}\PY{o}{=}\PY{n}{label}\PY{p}{)}
%    \PY{k}{return} \PY{n}{e}
%\end{Verbatim}
%\end{tcolorbox}

%    \begin{tcolorbox}[breakable, size=fbox, boxrule=1pt, pad at break*=1mm,colback=cellbackground, colframe=cellborder]
%\prompt{In}{incolor}{39}{\boxspacing}
%\begin{Verbatim}[commandchars=\\\{\}]
%\PY{c+c1}{\PYZsh{} Sample from:}
%\PY{n}{d} \PY{o}{=} \PY{l+m+mi}{2} \PY{c+c1}{\PYZsh{} Number of dimensions}
%\PY{n}{mean} \PY{o}{=} \PY{n}{np}\PY{o}{.}\PY{n}{matrix}\PY{p}{(}\PY{p}{[}\PY{p}{[}\PY{l+m+mf}{0.}\PY{p}{]}\PY{p}{,} \PY{p}{[}\PY{l+m+mf}{0.}\PY{p}{]}\PY{p}{]}\PY{p}{)}
%\PY{n}{covariance} \PY{o}{=} \PY{n}{np}\PY{o}{.}\PY{n}{matrix}\PY{p}{(}\PY{p}{[}
%    \PY{p}{[}\PY{l+m+mi}{1}\PY{p}{,} \PY{l+m+mf}{0.8}\PY{p}{]}\PY{p}{,} 
%    \PY{p}{[}\PY{l+m+mf}{0.8}\PY{p}{,} \PY{l+m+mi}{1}\PY{p}{]}
%\PY{p}{]}\PY{p}{)}
%
%\PY{c+c1}{\PYZsh{} Create L}
%\PY{n}{L} \PY{o}{=} \PY{n}{np}\PY{o}{.}\PY{n}{linalg}\PY{o}{.}\PY{n}{cholesky}\PY{p}{(}\PY{n}{covariance}\PY{p}{)}
%
%\PY{c+c1}{\PYZsh{} Sample X from standard normal}
%\PY{n}{n} \PY{o}{=} \PY{l+m+mi}{100}  \PY{c+c1}{\PYZsh{} Samples to draw}
%\PY{n}{X} \PY{o}{=} \PY{n}{np}\PY{o}{.}\PY{n}{random}\PY{o}{.}\PY{n}{normal}\PY{p}{(}\PY{n}{size}\PY{o}{=}\PY{p}{(}\PY{n}{d}\PY{p}{,} \PY{n}{n}\PY{p}{)}\PY{p}{)}
%\PY{c+c1}{\PYZsh{} Apply the transformation}
%\PY{n}{Y} \PY{o}{=} \PY{p}{(}\PY{n}{mean} \PY{o}{+} \PY{n}{L}\PY{o}{.}\PY{n}{dot}\PY{p}{(}\PY{n}{X}\PY{p}{)}\PY{p}{)}\PY{o}{.}\PY{n}{T}
%\end{Verbatim}
%\end{tcolorbox}

%    \begin{tcolorbox}[breakable, size=fbox, boxrule=1pt, pad at break*=1mm,colback=cellbackground, colframe=cellborder]
%\prompt{In}{incolor}{40}{\boxspacing}
%\begin{Verbatim}[commandchars=\\\{\}]
%\PY{c+c1}{\PYZsh{} Plot the samples and the distribution}
%\PY{n}{fig}\PY{p}{,} \PY{n}{ax} \PY{o}{=} \PY{n}{plt}\PY{o}{.}\PY{n}{subplots}\PY{p}{(}\PY{n}{figsize}\PY{o}{=}\PY{p}{(}\PY{l+m+mi}{6}\PY{p}{,} \PY{l+m+mf}{4.5}\PY{p}{)}\PY{p}{)}
%\PY{c+c1}{\PYZsh{} Plot bivariate distribution}
%\PY{n}{x1}\PY{p}{,} \PY{n}{x2}\PY{p}{,} \PY{n}{p} \PY{o}{=} \PY{n}{generate\PYZus{}surface}\PY{p}{(}\PY{n}{mean}\PY{p}{,} \PY{n}{covariance}\PY{p}{,} \PY{n}{d}\PY{p}{)}
%\PY{n}{con} \PY{o}{=} \PY{n}{ax}\PY{o}{.}\PY{n}{contourf}\PY{p}{(}\PY{n}{x1}\PY{p}{,} \PY{n}{x2}\PY{p}{,} \PY{n}{p}\PY{p}{,} \PY{l+m+mi}{100}\PY{p}{,} \PY{n}{cmap}\PY{o}{=}\PY{n}{cm}\PY{o}{.}\PY{n}{magma\PYZus{}r}\PY{p}{)}
%\PY{c+c1}{\PYZsh{} Plot 95\PYZpc{} Interval}
%\PY{n}{e} \PY{o}{=} \PY{n}{make\PYZus{}ellipse}\PY{p}{(}\PY{n}{mean}\PY{p}{,} \PY{n}{covariance}\PY{p}{)}
%\PY{n}{ax}\PY{o}{.}\PY{n}{add\PYZus{}artist}\PY{p}{(}\PY{n}{e}\PY{p}{)}
%\PY{c+c1}{\PYZsh{} Plot samples}
%\PY{n}{s} \PY{o}{=} \PY{n}{ax}\PY{o}{.}\PY{n}{plot}\PY{p}{(}\PY{n}{Y}\PY{p}{[}\PY{p}{:}\PY{p}{,}\PY{l+m+mi}{0}\PY{p}{]}\PY{p}{,} \PY{n}{Y}\PY{p}{[}\PY{p}{:}\PY{p}{,}\PY{l+m+mi}{1}\PY{p}{]}\PY{p}{,} \PY{l+s+s1}{\PYZsq{}}\PY{l+s+s1}{o}\PY{l+s+s1}{\PYZsq{}}\PY{p}{,} \PY{n}{c}\PY{o}{=}\PY{n}{cm}\PY{o}{.}\PY{n}{tab10}\PY{p}{(}\PY{l+m+mi}{0}\PY{p}{)}\PY{p}{,} \PY{n}{ms}\PY{o}{=}\PY{l+m+mi}{2}\PY{p}{,} \PY{n}{label}\PY{o}{=}\PY{l+s+s1}{\PYZsq{}}\PY{l+s+s1}{точки}\PY{l+s+s1}{\PYZsq{}}\PY{p}{)}
%\PY{n}{ax}\PY{o}{.}\PY{n}{set\PYZus{}xlabel}\PY{p}{(}\PY{l+s+s1}{\PYZsq{}}\PY{l+s+s1}{\PYZdl{}y\PYZus{}1\PYZdl{}}\PY{l+s+s1}{\PYZsq{}}\PY{p}{)}
%\PY{n}{ax}\PY{o}{.}\PY{n}{set\PYZus{}ylabel}\PY{p}{(}\PY{l+s+s1}{\PYZsq{}}\PY{l+s+s1}{\PYZdl{}y\PYZus{}2\PYZdl{}}\PY{l+s+s1}{\PYZsq{}}\PY{p}{)}
%\PY{n}{ax}\PY{o}{.}\PY{n}{axis}\PY{p}{(}\PY{p}{[}\PY{o}{\PYZhy{}}\PY{l+m+mf}{3.}\PY{p}{,} \PY{l+m+mf}{3.}\PY{p}{,} \PY{o}{\PYZhy{}}\PY{l+m+mf}{3.}\PY{p}{,} \PY{l+m+mf}{3.}\PY{p}{]}\PY{p}{)}
%\PY{n}{ax}\PY{o}{.}\PY{n}{set\PYZus{}aspect}\PY{p}{(}\PY{l+s+s1}{\PYZsq{}}\PY{l+s+s1}{equal}\PY{l+s+s1}{\PYZsq{}}\PY{p}{)}
%\PY{n}{ax}\PY{o}{.}\PY{n}{set\PYZus{}title}\PY{p}{(}\PY{l+s+s1}{\PYZsq{}}\PY{l+s+s1}{Выборка из двумерного нормального распределения}\PY{l+s+s1}{\PYZsq{}}\PY{p}{)}
%\PY{n}{cbar} \PY{o}{=} \PY{n}{plt}\PY{o}{.}\PY{n}{colorbar}\PY{p}{(}\PY{n}{con}\PY{p}{)}
%\PY{n}{cbar}\PY{o}{.}\PY{n}{ax}\PY{o}{.}\PY{n}{set\PYZus{}ylabel}\PY{p}{(}\PY{l+s+s1}{\PYZsq{}}\PY{l+s+s1}{\PYZdl{}p(y\PYZus{}1, y\PYZus{}2)\PYZdl{}}\PY{l+s+s1}{\PYZsq{}}\PY{p}{)}
%
%\PY{n}{plt}\PY{o}{.}\PY{n}{legend}\PY{p}{(}\PY{n}{handles}\PY{o}{=}\PY{p}{[}\PY{n}{e}\PY{p}{,} \PY{p}{]}\PY{p}{,} \PY{n}{loc}\PY{o}{=}\PY{l+m+mi}{2}\PY{p}{)}
%\PY{n}{plt}\PY{o}{.}\PY{n}{show}\PY{p}{(}\PY{p}{)}
%\end{Verbatim}
%\end{tcolorbox}

    \begin{center}
    \adjustimage{max size={0.6\linewidth}{0.6\paperheight}}{output_26_0.png}
    \end{center}
    { \hspace*{\fill} \\}
    
    \begin{center}\rule{0.5\linewidth}{\linethickness}\end{center}

    \hypertarget{ux43cux430ux440ux433ux438ux43dux430ux43bux44cux43dux44bux435-ux438-ux443ux441ux43bux43eux432ux43dux44bux435-ux440ux430ux441ux43fux440ux435ux434ux435ux43bux435ux43dux438ux44f}{%
\section{Маргинальные и условные
распределения}\label{ux43cux430ux440ux433ux438ux43dux430ux43bux44cux43dux44bux435-ux438-ux443ux441ux43bux43eux432ux43dux44bux435-ux440ux430ux441ux43fux440ux435ux434ux435ux43bux435ux43dux438ux44f}}

Пусть дан нормальный случайный вектор \(\mathbf{z}\) с \(n\)
компонентами. Пусть \(\mathbf{z} = (\mathbf{x}, \mathbf{y})\), где
\(\mathbf{x}\) и \(\mathbf{y}\) --- два подвектора вектора
\(\mathbf{z}\) с \(n_1\) и \(n_2\) компонентами, соответственно;
\(n = n_1 + n_2\). В этом случае говорят, что случайные векторы
\(\mathbf{x}\) и \(\mathbf{y}\) имеют \emph{совместное нормальное
распределение}, определяемое следующим образом:

\[
\begin{bmatrix}
    \mathbf{x} \\
    \mathbf{y} 
\end{bmatrix}
\sim
\mathcal{N}\left(
\begin{bmatrix}
    \mu_{\mathbf{x}} \\
    \mu_{\mathbf{y}}
\end{bmatrix},
\begin{bmatrix}
    \Sigma_{11} & \Sigma_{12} \\
    \Sigma_{21} & \Sigma_{22}
\end{bmatrix}
\right)
= \mathcal{N}(\mu, \Sigma).
\]

Здесь \(\Sigma_{11}\) --- корреляционная матрица вектора \(\mathbf{x}\),
\(\Sigma_{22}\) --- корреляционная матрица вектора \(\mathbf{y}\), а
матрицы \(\Sigma_{12}\) и \(\Sigma_{21} = \Sigma_{12}^T\) состоят из
корреляций компонент вектора \(\mathbf{x}\) и \(\mathbf{y}\) (взаимные
корреляционные матрицы). Вектор математического ожидания
\(\mathrm{E}\mathbf{z} = \mathbf{\mu}\) также разбивается на два
подвектора \(\mathrm{E}\mathbf{x} = \mathbf{\mu_x}\) и
\(\mathrm{E}\mathbf{y} = \mathbf{\mu_y}\).

%    \begin{tcolorbox}[breakable, size=fbox, boxrule=1pt, pad at break*=1mm,colback=cellbackground, colframe=cellborder]
%\prompt{In}{incolor}{42}{\boxspacing}
%\begin{Verbatim}[commandchars=\\\{\}]
%\PY{n}{d} \PY{o}{=} \PY{l+m+mi}{2}  \PY{c+c1}{\PYZsh{} dimensions}
%\PY{n}{mean} \PY{o}{=} \PY{n}{np}\PY{o}{.}\PY{n}{matrix}\PY{p}{(}\PY{p}{[}\PY{p}{[}\PY{l+m+mf}{0.}\PY{p}{]}\PY{p}{,} \PY{p}{[}\PY{l+m+mf}{0.}\PY{p}{]}\PY{p}{]}\PY{p}{)}
%\PY{n}{cov} \PY{o}{=} \PY{n}{np}\PY{o}{.}\PY{n}{matrix}\PY{p}{(}\PY{p}{[}
%    \PY{p}{[}\PY{l+m+mi}{1}\PY{p}{,} \PY{l+m+mf}{0.8}\PY{p}{]}\PY{p}{,} 
%    \PY{p}{[}\PY{l+m+mf}{0.8}\PY{p}{,} \PY{l+m+mi}{1}\PY{p}{]}
%\PY{p}{]}\PY{p}{)}
%
%\PY{c+c1}{\PYZsh{} Get the mean values from the vector}
%\PY{n}{mean\PYZus{}x} \PY{o}{=} \PY{n}{mean}\PY{p}{[}\PY{l+m+mi}{0}\PY{p}{,}\PY{l+m+mi}{0}\PY{p}{]}
%\PY{n}{mean\PYZus{}y} \PY{o}{=} \PY{n}{mean}\PY{p}{[}\PY{l+m+mi}{1}\PY{p}{,}\PY{l+m+mi}{0}\PY{p}{]}
%\PY{c+c1}{\PYZsh{} Get the blocks (single values in this case) from }
%\PY{c+c1}{\PYZsh{}  the covariance matrix}
%\PY{n}{Sigma\PYZus{}11} \PY{o}{=} \PY{n}{cov}\PY{p}{[}\PY{l+m+mi}{0}\PY{p}{,} \PY{l+m+mi}{0}\PY{p}{]}
%\PY{n}{Sigma\PYZus{}22} \PY{o}{=} \PY{n}{cov}\PY{p}{[}\PY{l+m+mi}{1}\PY{p}{,} \PY{l+m+mi}{1}\PY{p}{]}
%\PY{n}{Sigma\PYZus{}12} \PY{o}{=} \PY{n}{cov}\PY{p}{[}\PY{l+m+mi}{0}\PY{p}{,} \PY{l+m+mi}{1}\PY{p}{]}  \PY{c+c1}{\PYZsh{} = Sigma\PYZus{}21 transpose in this case}
%\end{Verbatim}
%\end{tcolorbox}

    \hypertarget{ux43cux430ux440ux433ux438ux43dux430ux43bux44cux43dux44bux435-ux447ux430ux441ux442ux43dux44bux435-ux440ux430ux441ux43fux440ux435ux434ux435ux43bux435ux43dux438ux44f}{%
\subsection{Маргинальные (частные)
распределения}\label{ux43cux430ux440ux433ux438ux43dux430ux43bux44cux43dux44bux435-ux447ux430ux441ux442ux43dux44bux435-ux440ux430ux441ux43fux440ux435ux434ux435ux43bux435ux43dux438ux44f}}

\begin{quote}
Название «частное распределение» используется в переводах под редакцией
Колмогорова, «маргинальное распределение» --- в более современной
литературе путём заимствования из английского языка (англ. marginal
distribution). Название в английском языке в свою очередь является
переводом с немецкого (нем. Randverteilungen) из публикации Колмогорова:
A. Kolmogoroff «Grundbegriffe der Wahrscheinlichkeitsrechnung»,
Springer-Verlag, 1933.
\href{https://ru.wikipedia.org/wiki/\%D0\%A7\%D0\%B0\%D1\%81\%D1\%82\%D0\%BD\%D0\%BE\%D0\%B5_\%D1\%80\%D0\%B0\%D1\%81\%D0\%BF\%D1\%80\%D0\%B5\%D0\%B4\%D0\%B5\%D0\%BB\%D0\%B5\%D0\%BD\%D0\%B8\%D0\%B5}{{[}1{]}}
\end{quote}

Маргинальное распределение --- это вероятностное распределение
подмножества случайных величин, рассматриваемых в качестве компоненты
или множества компонент некоторого известного многомерного
распределения. Оно представляет собой распределение вероятностей
переменных в подмножестве вне зависимости от значений других переменных
в исходном распределении.

В случае двумерного нормального распределения частными распределениями
являются одномерные распределения каждой компоненты \(\mathbf{x}\) и
\(\mathbf{y}\) по отдельности. Они определяются так: \[
\begin{aligned}
    f_\xi(\mathbf{x}) & = \mathcal{N}(\mu_{\mathbf{x}}, \Sigma_{11}) \\
    f_\eta(\mathbf{y}) & = \mathcal{N}(\mu_{\mathbf{y}}, \Sigma_{22}).
\end{aligned}
\]

%    \begin{tcolorbox}[breakable, size=fbox, boxrule=1pt, pad at break*=1mm,colback=cellbackground, colframe=cellborder]
%\prompt{In}{incolor}{43}{\boxspacing}
%\begin{Verbatim}[commandchars=\\\{\}]
%\PY{c+c1}{\PYZsh{} Plot the conditional distributions}
%\PY{n}{fig} \PY{o}{=} \PY{n}{plt}\PY{o}{.}\PY{n}{figure}\PY{p}{(}\PY{n}{figsize}\PY{o}{=}\PY{p}{(}\PY{l+m+mi}{7}\PY{p}{,} \PY{l+m+mi}{7}\PY{p}{)}\PY{p}{)}
%\PY{n}{gs} \PY{o}{=} \PY{n}{gridspec}\PY{o}{.}\PY{n}{GridSpec}\PY{p}{(}\PY{l+m+mi}{2}\PY{p}{,} \PY{l+m+mi}{2}\PY{p}{,} \PY{n}{width\PYZus{}ratios}\PY{o}{=}\PY{p}{[}\PY{l+m+mi}{2}\PY{p}{,} \PY{l+m+mi}{1}\PY{p}{]}\PY{p}{,} \PY{n}{height\PYZus{}ratios}\PY{o}{=}\PY{p}{[}\PY{l+m+mi}{2}\PY{p}{,} \PY{l+m+mi}{1}\PY{p}{]}\PY{p}{)}
%\PY{c+c1}{\PYZsh{} gs.update(wspace=0., hspace=0.)}
%\PY{n}{plt}\PY{o}{.}\PY{n}{suptitle}\PY{p}{(}\PY{l+s+s1}{\PYZsq{}}\PY{l+s+s1}{Маргинальные распределения}\PY{l+s+s1}{\PYZsq{}}\PY{p}{,} \PY{n}{y}\PY{o}{=}\PY{l+m+mf}{0.92}\PY{p}{)}
%
%\PY{c+c1}{\PYZsh{} Plot surface on top left}
%\PY{n}{ax1} \PY{o}{=} \PY{n}{plt}\PY{o}{.}\PY{n}{subplot}\PY{p}{(}\PY{n}{gs}\PY{p}{[}\PY{l+m+mi}{0}\PY{p}{]}\PY{p}{)}
%\PY{n}{x}\PY{p}{,} \PY{n}{y}\PY{p}{,} \PY{n}{p} \PY{o}{=} \PY{n}{generate\PYZus{}surface}\PY{p}{(}\PY{n}{mean}\PY{p}{,} \PY{n}{cov}\PY{p}{,} \PY{n}{d}\PY{p}{)}
%\PY{c+c1}{\PYZsh{} Plot bivariate distribution}
%\PY{n}{con} \PY{o}{=} \PY{n}{ax1}\PY{o}{.}\PY{n}{contourf}\PY{p}{(}\PY{n}{x}\PY{p}{,} \PY{n}{y}\PY{p}{,} \PY{n}{p}\PY{p}{,} \PY{l+m+mi}{100}\PY{p}{,} \PY{n}{cmap}\PY{o}{=}\PY{n}{cm}\PY{o}{.}\PY{n}{magma\PYZus{}r}\PY{p}{)}
%\PY{n}{ax1}\PY{o}{.}\PY{n}{set\PYZus{}xlabel}\PY{p}{(}\PY{l+s+s1}{\PYZsq{}}\PY{l+s+s1}{\PYZdl{}x\PYZdl{}}\PY{l+s+s1}{\PYZsq{}}\PY{p}{)}
%\PY{n}{ax1}\PY{o}{.}\PY{n}{set\PYZus{}ylabel}\PY{p}{(}\PY{l+s+s1}{\PYZsq{}}\PY{l+s+s1}{\PYZdl{}y\PYZdl{}}\PY{l+s+s1}{\PYZsq{}}\PY{p}{)}
%\PY{n}{ax1}\PY{o}{.}\PY{n}{yaxis}\PY{o}{.}\PY{n}{set\PYZus{}label\PYZus{}position}\PY{p}{(}\PY{l+s+s1}{\PYZsq{}}\PY{l+s+s1}{right}\PY{l+s+s1}{\PYZsq{}}\PY{p}{)}
%\PY{n}{ax1}\PY{o}{.}\PY{n}{axis}\PY{p}{(}\PY{p}{[}\PY{o}{\PYZhy{}}\PY{l+m+mf}{2.5}\PY{p}{,} \PY{l+m+mf}{2.5}\PY{p}{,} \PY{o}{\PYZhy{}}\PY{l+m+mf}{2.5}\PY{p}{,} \PY{l+m+mf}{2.5}\PY{p}{]}\PY{p}{)}
%
%\PY{c+c1}{\PYZsh{} Plot y}
%\PY{n}{ax2} \PY{o}{=} \PY{n}{plt}\PY{o}{.}\PY{n}{subplot}\PY{p}{(}\PY{n}{gs}\PY{p}{[}\PY{l+m+mi}{1}\PY{p}{]}\PY{p}{)}
%\PY{n}{y} \PY{o}{=} \PY{n}{np}\PY{o}{.}\PY{n}{linspace}\PY{p}{(}\PY{o}{\PYZhy{}}\PY{l+m+mi}{5}\PY{p}{,} \PY{l+m+mi}{5}\PY{p}{,} \PY{n}{num}\PY{o}{=}\PY{l+m+mi}{100}\PY{p}{)}
%\PY{n}{py} \PY{o}{=} \PY{n}{univariate\PYZus{}normal}\PY{p}{(}\PY{n}{y}\PY{p}{,} \PY{n}{mean\PYZus{}y}\PY{p}{,} \PY{n}{Sigma\PYZus{}22}\PY{p}{)}
%\PY{c+c1}{\PYZsh{} Plot univariate distribution}
%\PY{n}{ax2}\PY{o}{.}\PY{n}{plot}\PY{p}{(}\PY{n}{py}\PY{p}{,} \PY{n}{y}\PY{p}{,} \PY{l+s+s1}{\PYZsq{}}\PY{l+s+s1}{\PYZhy{}}\PY{l+s+s1}{\PYZsq{}}\PY{p}{,} \PY{n}{c}\PY{o}{=}\PY{n}{cm}\PY{o}{.}\PY{n}{tab10}\PY{p}{(}\PY{l+m+mi}{0}\PY{p}{)}\PY{p}{,} \PY{n}{label}\PY{o}{=}\PY{l+s+sa}{f}\PY{l+s+s1}{\PYZsq{}}\PY{l+s+s1}{\PYZdl{}p(y)\PYZdl{}}\PY{l+s+s1}{\PYZsq{}}\PY{p}{)}
%\PY{n}{ax2}\PY{o}{.}\PY{n}{legend}\PY{p}{(}\PY{n}{loc}\PY{o}{=}\PY{l+m+mi}{0}\PY{p}{)}
%\PY{c+c1}{\PYZsh{} ax2.set\PYZus{}xlabel(\PYZsq{}density\PYZsq{})}
%\PY{n}{ax2}\PY{o}{.}\PY{n}{set\PYZus{}ylim}\PY{p}{(}\PY{o}{\PYZhy{}}\PY{l+m+mf}{2.5}\PY{p}{,} \PY{l+m+mf}{2.5}\PY{p}{)}
%
%\PY{c+c1}{\PYZsh{} Plot x}
%\PY{n}{ax3} \PY{o}{=} \PY{n}{plt}\PY{o}{.}\PY{n}{subplot}\PY{p}{(}\PY{n}{gs}\PY{p}{[}\PY{l+m+mi}{2}\PY{p}{]}\PY{p}{)}
%\PY{n}{x} \PY{o}{=} \PY{n}{np}\PY{o}{.}\PY{n}{linspace}\PY{p}{(}\PY{o}{\PYZhy{}}\PY{l+m+mi}{5}\PY{p}{,} \PY{l+m+mi}{5}\PY{p}{,} \PY{n}{num}\PY{o}{=}\PY{l+m+mi}{100}\PY{p}{)}
%\PY{n}{px} \PY{o}{=} \PY{n}{univariate\PYZus{}normal}\PY{p}{(}\PY{n}{x}\PY{p}{,} \PY{n}{mean\PYZus{}x}\PY{p}{,} \PY{n}{Sigma\PYZus{}11}\PY{p}{)}
%\PY{c+c1}{\PYZsh{} Plot univariate distribution}
%\PY{n}{ax3}\PY{o}{.}\PY{n}{plot}\PY{p}{(}\PY{n}{x}\PY{p}{,} \PY{n}{px}\PY{p}{,} \PY{l+s+s1}{\PYZsq{}}\PY{l+s+s1}{\PYZhy{}}\PY{l+s+s1}{\PYZsq{}}\PY{p}{,} \PY{n}{c}\PY{o}{=}\PY{n}{cm}\PY{o}{.}\PY{n}{tab10}\PY{p}{(}\PY{l+m+mi}{3}\PY{p}{)}\PY{p}{,} \PY{n}{label}\PY{o}{=}\PY{l+s+sa}{f}\PY{l+s+s1}{\PYZsq{}}\PY{l+s+s1}{\PYZdl{}p(x\PYZus{}1)\PYZdl{}}\PY{l+s+s1}{\PYZsq{}}\PY{p}{)}
%\PY{n}{ax3}\PY{o}{.}\PY{n}{legend}\PY{p}{(}\PY{n}{loc}\PY{o}{=}\PY{l+m+mi}{0}\PY{p}{)}
%\PY{c+c1}{\PYZsh{} ax3.set\PYZus{}ylabel(\PYZsq{}density\PYZsq{})}
%\PY{n}{ax3}\PY{o}{.}\PY{n}{yaxis}\PY{o}{.}\PY{n}{set\PYZus{}label\PYZus{}position}\PY{p}{(}\PY{l+s+s1}{\PYZsq{}}\PY{l+s+s1}{right}\PY{l+s+s1}{\PYZsq{}}\PY{p}{)}
%\PY{n}{ax3}\PY{o}{.}\PY{n}{set\PYZus{}xlim}\PY{p}{(}\PY{o}{\PYZhy{}}\PY{l+m+mf}{2.5}\PY{p}{,} \PY{l+m+mf}{2.5}\PY{p}{)}
%
%\PY{c+c1}{\PYZsh{} Clear axis 4 and plot colarbar in its place}
%\PY{n}{ax4} \PY{o}{=} \PY{n}{plt}\PY{o}{.}\PY{n}{subplot}\PY{p}{(}\PY{n}{gs}\PY{p}{[}\PY{l+m+mi}{3}\PY{p}{]}\PY{p}{)}
%\PY{n}{ax4}\PY{o}{.}\PY{n}{set\PYZus{}visible}\PY{p}{(}\PY{k+kc}{False}\PY{p}{)}
%\PY{n}{divider} \PY{o}{=} \PY{n}{make\PYZus{}axes\PYZus{}locatable}\PY{p}{(}\PY{n}{ax4}\PY{p}{)}
%\PY{n}{cax} \PY{o}{=} \PY{n}{divider}\PY{o}{.}\PY{n}{append\PYZus{}axes}\PY{p}{(}\PY{l+s+s1}{\PYZsq{}}\PY{l+s+s1}{left}\PY{l+s+s1}{\PYZsq{}}\PY{p}{,} \PY{n}{size}\PY{o}{=}\PY{l+s+s1}{\PYZsq{}}\PY{l+s+s1}{20}\PY{l+s+s1}{\PYZpc{}}\PY{l+s+s1}{\PYZsq{}}\PY{p}{,} \PY{n}{pad}\PY{o}{=}\PY{l+m+mf}{0.05}\PY{p}{)}
%\PY{n}{cbar} \PY{o}{=} \PY{n}{fig}\PY{o}{.}\PY{n}{colorbar}\PY{p}{(}\PY{n}{con}\PY{p}{,} \PY{n}{cax}\PY{o}{=}\PY{n}{cax}\PY{p}{)}
%\PY{n}{cbar}\PY{o}{.}\PY{n}{ax}\PY{o}{.}\PY{n}{set\PYZus{}ylabel}\PY{p}{(}\PY{l+s+s1}{\PYZsq{}}\PY{l+s+s1}{\PYZdl{}p(x, xy)\PYZdl{}}\PY{l+s+s1}{\PYZsq{}}\PY{p}{)}
%\PY{n}{cbar}\PY{o}{.}\PY{n}{ax}\PY{o}{.}\PY{n}{tick\PYZus{}params}\PY{p}{(}\PY{n}{labelsize}\PY{o}{=}\PY{l+m+mi}{10}\PY{p}{)}
%\PY{n}{plt}\PY{o}{.}\PY{n}{show}\PY{p}{(}\PY{p}{)}
%\end{Verbatim}
%\end{tcolorbox}

    \begin{center}
    \adjustimage{max size={0.7\linewidth}{0.7\paperheight}}{output_31_0.png}
    \end{center}
    { \hspace*{\fill} \\}
    
    \hypertarget{ux443ux441ux43bux43eux432ux43dux44bux435-ux440ux430ux441ux43fux440ux435ux434ux435ux43bux435ux43dux438ux44f}{%
\subsection{Условные
распределения}\label{ux443ux441ux43bux43eux432ux43dux44bux435-ux440ux430ux441ux43fux440ux435ux434ux435ux43bux435ux43dux438ux44f}}

Условное распределение \(\mathbf{x}\) при фиксированном \(\mathbf{y}\)
можно получить с помощью формулы Байеса \[
    p(\mathbf{x}|\mathbf{y}) = \frac{p(\mathbf{x}, \mathbf{y})}{p(\mathbf{y})} \propto 
    \frac{\exp\left\{(\mathbf{x, y})^\top \Sigma^{-1} (\mathbf{x, y})\right\}}{\exp\left\{\mathbf{y}^\top \Sigma_{22}^{-1} \mathbf{y}\right\}} \label{eq:GP_bayes}\tag{1}.
\]

Оно также подчиняется нормальному закону:
\[ p(\mathbf{x}|\mathbf{y}) = \mathcal{N}(\mu_{x|y}, \Sigma_{x|y}) \]

с \emph{условным математическим ожиданием}
\[ \mu_{x|y} = \mu_x + \Sigma_{12}\Sigma_{22}^{-1}(\mathbf{y}-\mu_y) \label{eq:GP_mean}\tag{2} \]

и \emph{условной ковариационной матрицей}
\[ \Sigma_{x|y} = \Sigma_{11} - \Sigma_{12} \Sigma_{22}^{-1} \Sigma_{12}^\top. \label{eq:GP_cov}\tag{3} \]

Отметим, что условная ковариационная матрица не зависит от
\(\mathbf{y}\), а условное математическое ожидание является линейной
функцией от \(\mathbf{y}\). Другими словами, выражение
\(\eqref{eq:GP_mean}\) определяет функцию \emph{линейной регрессии}
(зависимости условного математического ожидания вектора \(\mathbf{x}\)
от заданного значения случайного вектора \(\mathbf{y}\)), где
\(\Sigma_{12}\Sigma_{22}^{-1}\) --- матрица коэффициентов регрессии.

Сдвиг математического ожидания можно рассматривать как невязку условной
переменной \((\mathbf{y}-\mu_y)\), нормализованную с ковариационной
матрицей условной переменной \(\Sigma_{22}\) и преобразованную в
пространство переменной \(\mathbf{x}\). Последнее делается с помощью
матрицы ковариаций между \(\mathbf{x}\) и \(\mathbf{y}\) ---
\(\Sigma_{12}\).

\textbf{Теорема.} Условное математическое ожидание является проекцией на
подпространство функций от случайных величин, стоящих в условии
условного математического ожидания.

Построим условные распределения \(p(x|y= 2)\) и \(p(y|x=-1)\).

%    \begin{tcolorbox}[breakable, size=fbox, boxrule=1pt, pad at break*=1mm,colback=cellbackground, colframe=cellborder]
%\prompt{In}{incolor}{50}{\boxspacing}
%\begin{Verbatim}[commandchars=\\\{\}]
%\PY{c+c1}{\PYZsh{} Calculate x|y}
%\PY{n}{y\PYZus{}condition} \PY{o}{=} \PY{l+m+mf}{1.5}  \PY{c+c1}{\PYZsh{} To condition on y}
%\PY{n}{mean\PYZus{}xgiveny} \PY{o}{=} \PY{n}{mean\PYZus{}x} \PY{o}{+} \PY{p}{(}\PY{n}{Sigma\PYZus{}12} \PY{o}{*} \PY{p}{(}\PY{l+m+mi}{1}\PY{o}{/}\PY{n}{Sigma\PYZus{}22}\PY{p}{)} \PY{o}{*} \PY{p}{(}\PY{n}{y\PYZus{}condition} \PY{o}{\PYZhy{}} \PY{n}{mean\PYZus{}y}\PY{p}{)}\PY{p}{)}
%\PY{n}{cov\PYZus{}xgiveny} \PY{o}{=} \PY{n}{Sigma\PYZus{}11} \PY{o}{\PYZhy{}} \PY{n}{Sigma\PYZus{}12} \PY{o}{*} \PY{p}{(}\PY{l+m+mi}{1}\PY{o}{/}\PY{n}{Sigma\PYZus{}22}\PY{p}{)} \PY{o}{*} \PY{n}{Sigma\PYZus{}12}
%\PY{n+nb}{print}\PY{p}{(}\PY{l+s+sa}{f}\PY{l+s+s1}{\PYZsq{}}\PY{l+s+s1}{mean\PYZus{}x|y=}\PY{l+s+si}{\PYZob{}}\PY{n}{mean\PYZus{}xgiveny}\PY{l+s+si}{:}\PY{l+s+s1}{0.2}\PY{l+s+si}{\PYZcb{}}\PY{l+s+s1}{, cov\PYZus{}x|y=}\PY{l+s+si}{\PYZob{}}\PY{n}{cov\PYZus{}xgiveny}\PY{l+s+si}{:}\PY{l+s+s1}{0.4}\PY{l+s+si}{\PYZcb{}}\PY{l+s+s1}{\PYZsq{}}\PY{p}{)}
%
%\PY{c+c1}{\PYZsh{} Calculate y|x}
%\PY{n}{x\PYZus{}condition} \PY{o}{=} \PY{o}{\PYZhy{}}\PY{l+m+mf}{1.}  \PY{c+c1}{\PYZsh{} To condition on x}
%\PY{n}{mean\PYZus{}ygivenx} \PY{o}{=} \PY{n}{mean\PYZus{}y} \PY{o}{+} \PY{p}{(}\PY{n}{Sigma\PYZus{}12} \PY{o}{*} \PY{p}{(}\PY{l+m+mi}{1}\PY{o}{/}\PY{n}{Sigma\PYZus{}11}\PY{p}{)} \PY{o}{*} \PY{p}{(}\PY{n}{x\PYZus{}condition} \PY{o}{\PYZhy{}} \PY{n}{mean\PYZus{}x}\PY{p}{)}\PY{p}{)}
%\PY{n}{cov\PYZus{}ygivenx} \PY{o}{=} \PY{n}{Sigma\PYZus{}22} \PY{o}{\PYZhy{}} \PY{p}{(}\PY{n}{Sigma\PYZus{}12} \PY{o}{*} \PY{p}{(}\PY{l+m+mi}{1}\PY{o}{/}\PY{n}{Sigma\PYZus{}11}\PY{p}{)} \PY{o}{*} \PY{n}{Sigma\PYZus{}12}\PY{p}{)}
%\PY{n+nb}{print}\PY{p}{(}\PY{l+s+sa}{f}\PY{l+s+s1}{\PYZsq{}}\PY{l+s+s1}{mean\PYZus{}y|x=}\PY{l+s+si}{\PYZob{}}\PY{n}{mean\PYZus{}ygivenx}\PY{l+s+si}{:}\PY{l+s+s1}{0.2}\PY{l+s+si}{\PYZcb{}}\PY{l+s+s1}{, cov\PYZus{}y|x=}\PY{l+s+si}{\PYZob{}}\PY{n}{cov\PYZus{}ygivenx}\PY{l+s+si}{:}\PY{l+s+s1}{0.4}\PY{l+s+si}{\PYZcb{}}\PY{l+s+s1}{\PYZsq{}}\PY{p}{)}
%\end{Verbatim}
%\end{tcolorbox}

    \begin{Verbatim}[commandchars=\\\{\}]
mean\_x|y=1.2, cov\_x|y=0.36
mean\_y|x=-0.8, cov\_y|x=0.36
    \end{Verbatim}

%    \begin{tcolorbox}[breakable, size=fbox, boxrule=1pt, pad at break*=1mm,colback=cellbackground, colframe=cellborder]
%\prompt{In}{incolor}{94}{\boxspacing}
%\begin{Verbatim}[commandchars=\\\{\}]
%\PY{c+c1}{\PYZsh{} Plot the conditional distributions}
%\PY{n}{fig} \PY{o}{=} \PY{n}{plt}\PY{o}{.}\PY{n}{figure}\PY{p}{(}\PY{n}{figsize}\PY{o}{=}\PY{p}{(}\PY{l+m+mi}{7}\PY{p}{,} \PY{l+m+mi}{7}\PY{p}{)}\PY{p}{)}
%\PY{n}{gs} \PY{o}{=} \PY{n}{gridspec}\PY{o}{.}\PY{n}{GridSpec}\PY{p}{(}
%    \PY{l+m+mi}{2}\PY{p}{,} \PY{l+m+mi}{2}\PY{p}{,} \PY{n}{width\PYZus{}ratios}\PY{o}{=}\PY{p}{[}\PY{l+m+mi}{2}\PY{p}{,} \PY{l+m+mi}{1}\PY{p}{]}\PY{p}{,} \PY{n}{height\PYZus{}ratios}\PY{o}{=}\PY{p}{[}\PY{l+m+mi}{2}\PY{p}{,} \PY{l+m+mi}{1}\PY{p}{]}\PY{p}{)}
%\PY{c+c1}{\PYZsh{} gs.update(wspace=0., hspace=0.)}
%\PY{n}{plt}\PY{o}{.}\PY{n}{suptitle}\PY{p}{(}\PY{l+s+s1}{\PYZsq{}}\PY{l+s+s1}{Условные распределения}\PY{l+s+s1}{\PYZsq{}}\PY{p}{,} \PY{n}{y}\PY{o}{=}\PY{l+m+mf}{0.93}\PY{p}{)}
%
%\PY{c+c1}{\PYZsh{} Plot surface on top left}
%\PY{n}{ax1} \PY{o}{=} \PY{n}{plt}\PY{o}{.}\PY{n}{subplot}\PY{p}{(}\PY{n}{gs}\PY{p}{[}\PY{l+m+mi}{0}\PY{p}{]}\PY{p}{)}
%\PY{n}{x}\PY{p}{,} \PY{n}{y}\PY{p}{,} \PY{n}{p} \PY{o}{=} \PY{n}{generate\PYZus{}surface}\PY{p}{(}\PY{n}{mean}\PY{p}{,} \PY{n}{cov}\PY{p}{,} \PY{n}{d}\PY{p}{)}
%\PY{c+c1}{\PYZsh{} Plot bivariate distribution}
%\PY{n}{con} \PY{o}{=} \PY{n}{ax1}\PY{o}{.}\PY{n}{contourf}\PY{p}{(}\PY{n}{x}\PY{p}{,} \PY{n}{y}\PY{p}{,} \PY{n}{p}\PY{p}{,} \PY{l+m+mi}{100}\PY{p}{,} \PY{n}{cmap}\PY{o}{=}\PY{n}{cm}\PY{o}{.}\PY{n}{magma\PYZus{}r}\PY{p}{)}
%\PY{c+c1}{\PYZsh{} y=1 that is conditioned upon}
%\PY{n}{ax1}\PY{o}{.}\PY{n}{plot}\PY{p}{(}\PY{p}{[}\PY{o}{\PYZhy{}}\PY{l+m+mf}{2.5}\PY{p}{,} \PY{l+m+mf}{2.5}\PY{p}{]}\PY{p}{,} \PY{p}{[}\PY{n}{y\PYZus{}condition}\PY{p}{,} \PY{n}{y\PYZus{}condition}\PY{p}{]}\PY{p}{,} \PY{l+s+s1}{\PYZsq{}}\PY{l+s+s1}{\PYZhy{}\PYZhy{}}\PY{l+s+s1}{\PYZsq{}}\PY{p}{,} \PY{n}{c}\PY{o}{=}\PY{n}{cm}\PY{o}{.}\PY{n}{tab10}\PY{p}{(}\PY{l+m+mi}{3}\PY{p}{)}\PY{p}{)}
%\PY{c+c1}{\PYZsh{} x=\PYZhy{}1. that is conditioned upon}
%\PY{n}{ax1}\PY{o}{.}\PY{n}{plot}\PY{p}{(}\PY{p}{[}\PY{n}{x\PYZus{}condition}\PY{p}{,} \PY{n}{x\PYZus{}condition}\PY{p}{]}\PY{p}{,} \PY{p}{[}\PY{o}{\PYZhy{}}\PY{l+m+mf}{2.5}\PY{p}{,} \PY{l+m+mf}{2.5}\PY{p}{]}\PY{p}{,} \PY{l+s+s1}{\PYZsq{}}\PY{l+s+s1}{\PYZhy{}\PYZhy{}}\PY{l+s+s1}{\PYZsq{}}\PY{p}{,} \PY{n}{c}\PY{o}{=}\PY{n}{cm}\PY{o}{.}\PY{n}{tab10}\PY{p}{(}\PY{l+m+mi}{0}\PY{p}{)}\PY{p}{)}
%\PY{n}{ax1}\PY{o}{.}\PY{n}{set\PYZus{}xlabel}\PY{p}{(}\PY{l+s+s1}{\PYZsq{}}\PY{l+s+s1}{\PYZdl{}x\PYZdl{}}\PY{l+s+s1}{\PYZsq{}}\PY{p}{)}
%\PY{n}{ax1}\PY{o}{.}\PY{n}{set\PYZus{}ylabel}\PY{p}{(}\PY{l+s+s1}{\PYZsq{}}\PY{l+s+s1}{\PYZdl{}y\PYZdl{}}\PY{l+s+s1}{\PYZsq{}}\PY{p}{)}
%\PY{n}{ax1}\PY{o}{.}\PY{n}{yaxis}\PY{o}{.}\PY{n}{set\PYZus{}label\PYZus{}position}\PY{p}{(}\PY{l+s+s1}{\PYZsq{}}\PY{l+s+s1}{right}\PY{l+s+s1}{\PYZsq{}}\PY{p}{)}
%\PY{n}{ax1}\PY{o}{.}\PY{n}{axis}\PY{p}{(}\PY{p}{[}\PY{o}{\PYZhy{}}\PY{l+m+mf}{2.5}\PY{p}{,} \PY{l+m+mf}{2.5}\PY{p}{,} \PY{o}{\PYZhy{}}\PY{l+m+mf}{2.5}\PY{p}{,} \PY{l+m+mf}{2.5}\PY{p}{]}\PY{p}{)}
%
%\PY{c+c1}{\PYZsh{} Plot y|x}
%\PY{n}{ax2} \PY{o}{=} \PY{n}{plt}\PY{o}{.}\PY{n}{subplot}\PY{p}{(}\PY{n}{gs}\PY{p}{[}\PY{l+m+mi}{1}\PY{p}{]}\PY{p}{)}
%\PY{n}{yx} \PY{o}{=} \PY{n}{np}\PY{o}{.}\PY{n}{linspace}\PY{p}{(}\PY{o}{\PYZhy{}}\PY{l+m+mi}{5}\PY{p}{,} \PY{l+m+mi}{5}\PY{p}{,} \PY{n}{num}\PY{o}{=}\PY{l+m+mi}{100}\PY{p}{)}
%\PY{n}{pyx} \PY{o}{=} \PY{n}{univariate\PYZus{}normal}\PY{p}{(}\PY{n}{yx}\PY{p}{,} \PY{n}{mean\PYZus{}ygivenx}\PY{p}{,} \PY{n}{cov\PYZus{}ygivenx}\PY{p}{)}
%\PY{c+c1}{\PYZsh{} Plot univariate distribution}
%\PY{n}{ax2}\PY{o}{.}\PY{n}{plot}\PY{p}{(}\PY{n}{pyx}\PY{p}{,} \PY{n}{yx}\PY{p}{,} \PY{l+s+s1}{\PYZsq{}}\PY{l+s+s1}{\PYZhy{}}\PY{l+s+s1}{\PYZsq{}}\PY{p}{,} \PY{n}{c}\PY{o}{=}\PY{n}{cm}\PY{o}{.}\PY{n}{tab10}\PY{p}{(}\PY{l+m+mi}{0}\PY{p}{)}\PY{p}{,} \PY{n}{label}\PY{o}{=}\PY{l+s+sa}{f}\PY{l+s+s1}{\PYZsq{}}\PY{l+s+s1}{\PYZdl{}p(y|x=}\PY{l+s+si}{\PYZob{}}\PY{n}{x\PYZus{}condition}\PY{l+s+si}{:}\PY{l+s+s1}{.1f}\PY{l+s+si}{\PYZcb{}}\PY{l+s+s1}{)\PYZdl{}}\PY{l+s+s1}{\PYZsq{}}\PY{p}{)}
%\PY{n}{ax2}\PY{o}{.}\PY{n}{legend}\PY{p}{(}\PY{n}{loc}\PY{o}{=}\PY{l+m+mi}{0}\PY{p}{,} \PY{n}{fontsize}\PY{o}{=}\PY{l+m+mi}{10}\PY{p}{)}
%\PY{c+c1}{\PYZsh{} ax2.set\PYZus{}xlabel(\PYZsq{}density\PYZsq{})}
%\PY{n}{ax2}\PY{o}{.}\PY{n}{set\PYZus{}ylim}\PY{p}{(}\PY{o}{\PYZhy{}}\PY{l+m+mf}{2.5}\PY{p}{,} \PY{l+m+mf}{2.5}\PY{p}{)}
%
%\PY{c+c1}{\PYZsh{} Plot x|y}
%\PY{n}{ax3} \PY{o}{=} \PY{n}{plt}\PY{o}{.}\PY{n}{subplot}\PY{p}{(}\PY{n}{gs}\PY{p}{[}\PY{l+m+mi}{2}\PY{p}{]}\PY{p}{)}
%\PY{n}{xy} \PY{o}{=} \PY{n}{np}\PY{o}{.}\PY{n}{linspace}\PY{p}{(}\PY{o}{\PYZhy{}}\PY{l+m+mi}{5}\PY{p}{,} \PY{l+m+mi}{5}\PY{p}{,} \PY{n}{num}\PY{o}{=}\PY{l+m+mi}{100}\PY{p}{)}
%\PY{n}{pxy} \PY{o}{=} \PY{n}{univariate\PYZus{}normal}\PY{p}{(}\PY{n}{xy}\PY{p}{,} \PY{n}{mean\PYZus{}xgiveny}\PY{p}{,} \PY{n}{cov\PYZus{}xgiveny}\PY{p}{)}
%\PY{c+c1}{\PYZsh{} Plot univariate distribution}
%\PY{n}{ax3}\PY{o}{.}\PY{n}{plot}\PY{p}{(}\PY{n}{xy}\PY{p}{,} \PY{n}{pxy}\PY{p}{,} \PY{l+s+s1}{\PYZsq{}}\PY{l+s+s1}{\PYZhy{}}\PY{l+s+s1}{\PYZsq{}}\PY{p}{,} \PY{n}{c}\PY{o}{=}\PY{n}{cm}\PY{o}{.}\PY{n}{tab10}\PY{p}{(}\PY{l+m+mi}{3}\PY{p}{)}\PY{p}{,} \PY{n}{label}\PY{o}{=}\PY{l+s+sa}{f}\PY{l+s+s1}{\PYZsq{}}\PY{l+s+s1}{\PYZdl{}p(x|y=}\PY{l+s+si}{\PYZob{}}\PY{n}{y\PYZus{}condition}\PY{l+s+si}{:}\PY{l+s+s1}{.1f}\PY{l+s+si}{\PYZcb{}}\PY{l+s+s1}{)\PYZdl{}}\PY{l+s+s1}{\PYZsq{}}\PY{p}{)}
%\PY{n}{ax3}\PY{o}{.}\PY{n}{legend}\PY{p}{(}\PY{n}{loc}\PY{o}{=}\PY{l+m+mi}{0}\PY{p}{,} \PY{n}{fontsize}\PY{o}{=}\PY{l+m+mi}{10}\PY{p}{)}
%\PY{c+c1}{\PYZsh{} ax3.set\PYZus{}ylabel(\PYZsq{}density\PYZsq{})}
%\PY{n}{ax3}\PY{o}{.}\PY{n}{yaxis}\PY{o}{.}\PY{n}{set\PYZus{}label\PYZus{}position}\PY{p}{(}\PY{l+s+s1}{\PYZsq{}}\PY{l+s+s1}{right}\PY{l+s+s1}{\PYZsq{}}\PY{p}{)}
%\PY{n}{ax3}\PY{o}{.}\PY{n}{set\PYZus{}xlim}\PY{p}{(}\PY{o}{\PYZhy{}}\PY{l+m+mf}{2.5}\PY{p}{,} \PY{l+m+mf}{2.5}\PY{p}{)}
%
%\PY{c+c1}{\PYZsh{} Clear axis 4 and plot colarbar in its place}
%\PY{n}{ax4} \PY{o}{=} \PY{n}{plt}\PY{o}{.}\PY{n}{subplot}\PY{p}{(}\PY{n}{gs}\PY{p}{[}\PY{l+m+mi}{3}\PY{p}{]}\PY{p}{)}
%\PY{n}{ax4}\PY{o}{.}\PY{n}{set\PYZus{}visible}\PY{p}{(}\PY{k+kc}{False}\PY{p}{)}
%\PY{n}{divider} \PY{o}{=} \PY{n}{make\PYZus{}axes\PYZus{}locatable}\PY{p}{(}\PY{n}{ax4}\PY{p}{)}
%\PY{n}{cax} \PY{o}{=} \PY{n}{divider}\PY{o}{.}\PY{n}{append\PYZus{}axes}\PY{p}{(}\PY{l+s+s1}{\PYZsq{}}\PY{l+s+s1}{left}\PY{l+s+s1}{\PYZsq{}}\PY{p}{,} \PY{n}{size}\PY{o}{=}\PY{l+s+s1}{\PYZsq{}}\PY{l+s+s1}{20}\PY{l+s+s1}{\PYZpc{}}\PY{l+s+s1}{\PYZsq{}}\PY{p}{,} \PY{n}{pad}\PY{o}{=}\PY{l+m+mf}{0.05}\PY{p}{)}
%\PY{n}{cbar} \PY{o}{=} \PY{n}{fig}\PY{o}{.}\PY{n}{colorbar}\PY{p}{(}\PY{n}{con}\PY{p}{,} \PY{n}{cax}\PY{o}{=}\PY{n}{cax}\PY{p}{)}
%\PY{n}{cbar}\PY{o}{.}\PY{n}{ax}\PY{o}{.}\PY{n}{set\PYZus{}ylabel}\PY{p}{(}\PY{l+s+s1}{\PYZsq{}}\PY{l+s+s1}{\PYZdl{}p(x, y)\PYZdl{}}\PY{l+s+s1}{\PYZsq{}}\PY{p}{)}
%\PY{n}{cbar}\PY{o}{.}\PY{n}{ax}\PY{o}{.}\PY{n}{tick\PYZus{}params}\PY{p}{(}\PY{n}{labelsize}\PY{o}{=}\PY{l+m+mi}{10}\PY{p}{)}
%\PY{n}{plt}\PY{o}{.}\PY{n}{show}\PY{p}{(}\PY{p}{)}
%\end{Verbatim}
%\end{tcolorbox}

    \begin{center}
    \adjustimage{max size={0.7\linewidth}{0.7\paperheight}}{output_34_0.png}
    \end{center}
    { \hspace*{\fill} \\}
    
    \begin{center}\rule{0.5\linewidth}{\linethickness}\end{center}

    \hypertarget{ux437ux430ux434ux430ux43dux438ux44f}{%
\section{Задания}\label{ux437ux430ux434ux430ux43dux438ux44f}}

\begin{enumerate}
\def\labelenumi{\arabic{enumi}.}
\tightlist
\item
  Напишите функцию, осуществляющую разложение Холецкого симметричной
  положительно определённой матрицы \(A = L L^\top\).
\item
  Применяя формулу Байеса \eqref{eq:GP_bayes}, получите
  формулы для условного математического ожидания
  \eqref{eq:GP_mean} и условной ковариационной матрицы
  \eqref{eq:GP_cov}. \emph{Подсказка}: используйте
  следующий результат для обращения блочной матрицы:
  \[
  \Sigma =
  \begin{bmatrix}
   P & Q \\
   R & S
  \end{bmatrix},
  \quad
  \Sigma^{-1} =
  \begin{bmatrix}
   \tilde{P} & \tilde{Q} \\
   \tilde{R} & \tilde{S}
  \end{bmatrix},
  \]
  где
  \[
  \begin{aligned}
   \tilde{P} &\;= \left(P-QS^{-1}R\right)^{-1}     & =\;& P^{-1} + P^{-1}Q\tilde{S}RP^{-1} \\
   \tilde{Q} &\;= -\tilde{P}QS^{-1}                & =\;& -P^{-1}Q\tilde{S} \\
   \tilde{R} &\;= -S^{-1}R\tilde{P}                & =\;& -\tilde{S}RP^{-1}  \\
   \tilde{S} &\;= S^{-1} + S^{-1}R\tilde{P}QS^{-1} & =\;& \left(S-RP^{-1}Q\right)^{-1}.
  \end{aligned}
  \]
\end{enumerate}

    \hypertarget{ux43bux438ux442ux435ux440ux430ux442ux443ux440ux430}{%
\section{Литература}\label{ux43bux438ux442ux435ux440ux430ux442ux443ux440ux430}}

\begin{enumerate}
\def\labelenumi{\arabic{enumi}.}
\tightlist
\item
  \emph{Roelants P.}
  \href{https://peterroelants.github.io/posts/multivariate-normal-primer/}{Multivariate
  normal distribution}.
\item
  \emph{Ширяев А.Н.} Вероятность --- 1. --- М.: МЦНМО, 2007. --- 517 с.
\item
  \emph{Rasmussen C.E., Williams C.K.I.}
  \href{http://www.gaussianprocess.org/gpml/}{Gaussian Processes for
  Machine Learning}. --- The MIT Press, 2006. --- 248 p.
\end{enumerate}

%    \begin{tcolorbox}[breakable, size=fbox, boxrule=1pt, pad at break*=1mm,colback=cellbackground, colframe=cellborder]
%\prompt{In}{incolor}{ }{\boxspacing}
%\begin{Verbatim}[commandchars=\\\{\}]
%\PY{c+c1}{\PYZsh{} Versions used}
%\PY{n+nb}{print}\PY{p}{(}\PY{l+s+s1}{\PYZsq{}}\PY{l+s+s1}{Python: }\PY{l+s+si}{\PYZob{}\PYZcb{}}\PY{l+s+s1}{.}\PY{l+s+si}{\PYZob{}\PYZcb{}}\PY{l+s+s1}{.}\PY{l+s+si}{\PYZob{}\PYZcb{}}\PY{l+s+s1}{\PYZsq{}}\PY{o}{.}\PY{n}{format}\PY{p}{(}\PY{o}{*}\PY{n}{sys}\PY{o}{.}\PY{n}{version\PYZus{}info}\PY{p}{[}\PY{p}{:}\PY{l+m+mi}{3}\PY{p}{]}\PY{p}{)}\PY{p}{)}
%\PY{n+nb}{print}\PY{p}{(}\PY{l+s+s1}{\PYZsq{}}\PY{l+s+s1}{numpy: }\PY{l+s+si}{\PYZob{}\PYZcb{}}\PY{l+s+s1}{\PYZsq{}}\PY{o}{.}\PY{n}{format}\PY{p}{(}\PY{n}{np}\PY{o}{.}\PY{n}{\PYZus{}\PYZus{}version\PYZus{}\PYZus{}}\PY{p}{)}\PY{p}{)}
%\PY{n+nb}{print}\PY{p}{(}\PY{l+s+s1}{\PYZsq{}}\PY{l+s+s1}{matplotlib: }\PY{l+s+si}{\PYZob{}\PYZcb{}}\PY{l+s+s1}{\PYZsq{}}\PY{o}{.}\PY{n}{format}\PY{p}{(}\PY{n}{matplotlib}\PY{o}{.}\PY{n}{\PYZus{}\PYZus{}version\PYZus{}\PYZus{}}\PY{p}{)}\PY{p}{)}
%\PY{n+nb}{print}\PY{p}{(}\PY{l+s+s1}{\PYZsq{}}\PY{l+s+s1}{seaborn: }\PY{l+s+si}{\PYZob{}\PYZcb{}}\PY{l+s+s1}{\PYZsq{}}\PY{o}{.}\PY{n}{format}\PY{p}{(}\PY{n}{sns}\PY{o}{.}\PY{n}{\PYZus{}\PYZus{}version\PYZus{}\PYZus{}}\PY{p}{)}\PY{p}{)}
%\end{Verbatim}
%\end{tcolorbox}


    % Add a bibliography block to the postdoc
    
    
    
\end{document}
