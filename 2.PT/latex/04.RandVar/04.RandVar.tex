\documentclass[11pt,a4paper]{article}

    \usepackage[breakable]{tcolorbox}
    \usepackage{parskip} % Stop auto-indenting (to mimic markdown behaviour)
    
    \usepackage{iftex}
    \ifPDFTeX
      \usepackage[T2A]{fontenc}
      \usepackage{mathpazo}
      \usepackage[russian,english]{babel}
    \else
      \usepackage{fontspec}
      \usepackage{polyglossia}
      \setmainlanguage[babelshorthands=true]{russian}    % Язык по-умолчанию русский с поддержкой приятных команд пакета babel
      \setotherlanguage{english}                         % Дополнительный язык = английский (в американской вариации по-умолчанию)
      \newfontfamily\cyrillicfonttt[Scale=0.87,BoldFont={Fira Mono Medium}] {Fira Mono}  % Моноширинный шрифт для кириллицы
      \defaultfontfeatures{Ligatures=TeX}
      \newfontfamily\cyrillicfont{STIX Two Text}         % Шрифт с засечками для кириллицы
    \fi
    \renewcommand{\linethickness}{0.1ex}

    % Basic figure setup, for now with no caption control since it's done
    % automatically by Pandoc (which extracts ![](path) syntax from Markdown).
    \usepackage{graphicx}
    % Maintain compatibility with old templates. Remove in nbconvert 6.0
    \let\Oldincludegraphics\includegraphics
    % Ensure that by default, figures have no caption (until we provide a
    % proper Figure object with a Caption API and a way to capture that
    % in the conversion process - todo).
    \usepackage{caption}
    \DeclareCaptionFormat{nocaption}{}
    \captionsetup{format=nocaption,aboveskip=0pt,belowskip=0pt}

    \usepackage{float}
    \floatplacement{figure}{H} % forces figures to be placed at the correct location
    \usepackage{xcolor} % Allow colors to be defined
    \usepackage{enumerate} % Needed for markdown enumerations to work
    \usepackage{geometry} % Used to adjust the document margins
    \usepackage{amsmath} % Equations
    \usepackage{amssymb} % Equations
    \usepackage{textcomp} % defines textquotesingle
    % Hack from http://tex.stackexchange.com/a/47451/13684:
    \AtBeginDocument{%
        \def\PYZsq{\textquotesingle}% Upright quotes in Pygmentized code
    }
    \usepackage{upquote} % Upright quotes for verbatim code
    \usepackage{eurosym} % defines \euro
    \usepackage[mathletters]{ucs} % Extended unicode (utf-8) support
    \usepackage{fancyvrb} % verbatim replacement that allows latex
    \usepackage{grffile} % extends the file name processing of package graphics 
                         % to support a larger range
    \makeatletter % fix for old versions of grffile with XeLaTeX
    \@ifpackagelater{grffile}{2019/11/01}
    {
      % Do nothing on new versions
    }
    {
      \def\Gread@@xetex#1{%
        \IfFileExists{"\Gin@base".bb}%
        {\Gread@eps{\Gin@base.bb}}%
        {\Gread@@xetex@aux#1}%
      }
    }
    \makeatother
    \usepackage[Export]{adjustbox} % Used to constrain images to a maximum size
    \adjustboxset{max size={0.9\linewidth}{0.9\paperheight}}

    % The hyperref package gives us a pdf with properly built
    % internal navigation ('pdf bookmarks' for the table of contents,
    % internal cross-reference links, web links for URLs, etc.)
    \usepackage{hyperref}
    % The default LaTeX title has an obnoxious amount of whitespace. By default,
    % titling removes some of it. It also provides customization options.
    \usepackage{titling}
    \usepackage{longtable} % longtable support required by pandoc >1.10
    \usepackage{booktabs}  % table support for pandoc > 1.12.2
    \usepackage[inline]{enumitem} % IRkernel/repr support (it uses the enumerate* environment)
    \usepackage[normalem]{ulem} % ulem is needed to support strikethroughs (\sout)
                                % normalem makes italics be italics, not underlines
    \usepackage{mathrsfs}
    

    
    % Colors for the hyperref package
    \definecolor{urlcolor}{rgb}{0,.145,.698}
    \definecolor{linkcolor}{rgb}{.71,0.21,0.01}
    \definecolor{citecolor}{rgb}{.12,.54,.11}

    % ANSI colors
    \definecolor{ansi-black}{HTML}{3E424D}
    \definecolor{ansi-black-intense}{HTML}{282C36}
    \definecolor{ansi-red}{HTML}{E75C58}
    \definecolor{ansi-red-intense}{HTML}{B22B31}
    \definecolor{ansi-green}{HTML}{00A250}
    \definecolor{ansi-green-intense}{HTML}{007427}
    \definecolor{ansi-yellow}{HTML}{DDB62B}
    \definecolor{ansi-yellow-intense}{HTML}{B27D12}
    \definecolor{ansi-blue}{HTML}{208FFB}
    \definecolor{ansi-blue-intense}{HTML}{0065CA}
    \definecolor{ansi-magenta}{HTML}{D160C4}
    \definecolor{ansi-magenta-intense}{HTML}{A03196}
    \definecolor{ansi-cyan}{HTML}{60C6C8}
    \definecolor{ansi-cyan-intense}{HTML}{258F8F}
    \definecolor{ansi-white}{HTML}{C5C1B4}
    \definecolor{ansi-white-intense}{HTML}{A1A6B2}
    \definecolor{ansi-default-inverse-fg}{HTML}{FFFFFF}
    \definecolor{ansi-default-inverse-bg}{HTML}{000000}

    % common color for the border for error outputs.
    \definecolor{outerrorbackground}{HTML}{FFDFDF}

    % commands and environments needed by pandoc snippets
    % extracted from the output of `pandoc -s`
    \providecommand{\tightlist}{%
      \setlength{\itemsep}{0pt}\setlength{\parskip}{0pt}}
    \DefineVerbatimEnvironment{Highlighting}{Verbatim}{commandchars=\\\{\}}
    % Add ',fontsize=\small' for more characters per line
    \newenvironment{Shaded}{}{}
    \newcommand{\KeywordTok}[1]{\textcolor[rgb]{0.00,0.44,0.13}{\textbf{{#1}}}}
    \newcommand{\DataTypeTok}[1]{\textcolor[rgb]{0.56,0.13,0.00}{{#1}}}
    \newcommand{\DecValTok}[1]{\textcolor[rgb]{0.25,0.63,0.44}{{#1}}}
    \newcommand{\BaseNTok}[1]{\textcolor[rgb]{0.25,0.63,0.44}{{#1}}}
    \newcommand{\FloatTok}[1]{\textcolor[rgb]{0.25,0.63,0.44}{{#1}}}
    \newcommand{\CharTok}[1]{\textcolor[rgb]{0.25,0.44,0.63}{{#1}}}
    \newcommand{\StringTok}[1]{\textcolor[rgb]{0.25,0.44,0.63}{{#1}}}
    \newcommand{\CommentTok}[1]{\textcolor[rgb]{0.38,0.63,0.69}{\textit{{#1}}}}
    \newcommand{\OtherTok}[1]{\textcolor[rgb]{0.00,0.44,0.13}{{#1}}}
    \newcommand{\AlertTok}[1]{\textcolor[rgb]{1.00,0.00,0.00}{\textbf{{#1}}}}
    \newcommand{\FunctionTok}[1]{\textcolor[rgb]{0.02,0.16,0.49}{{#1}}}
    \newcommand{\RegionMarkerTok}[1]{{#1}}
    \newcommand{\ErrorTok}[1]{\textcolor[rgb]{1.00,0.00,0.00}{\textbf{{#1}}}}
    \newcommand{\NormalTok}[1]{{#1}}
    
    % Additional commands for more recent versions of Pandoc
    \newcommand{\ConstantTok}[1]{\textcolor[rgb]{0.53,0.00,0.00}{{#1}}}
    \newcommand{\SpecialCharTok}[1]{\textcolor[rgb]{0.25,0.44,0.63}{{#1}}}
    \newcommand{\VerbatimStringTok}[1]{\textcolor[rgb]{0.25,0.44,0.63}{{#1}}}
    \newcommand{\SpecialStringTok}[1]{\textcolor[rgb]{0.73,0.40,0.53}{{#1}}}
    \newcommand{\ImportTok}[1]{{#1}}
    \newcommand{\DocumentationTok}[1]{\textcolor[rgb]{0.73,0.13,0.13}{\textit{{#1}}}}
    \newcommand{\AnnotationTok}[1]{\textcolor[rgb]{0.38,0.63,0.69}{\textbf{\textit{{#1}}}}}
    \newcommand{\CommentVarTok}[1]{\textcolor[rgb]{0.38,0.63,0.69}{\textbf{\textit{{#1}}}}}
    \newcommand{\VariableTok}[1]{\textcolor[rgb]{0.10,0.09,0.49}{{#1}}}
    \newcommand{\ControlFlowTok}[1]{\textcolor[rgb]{0.00,0.44,0.13}{\textbf{{#1}}}}
    \newcommand{\OperatorTok}[1]{\textcolor[rgb]{0.40,0.40,0.40}{{#1}}}
    \newcommand{\BuiltInTok}[1]{{#1}}
    \newcommand{\ExtensionTok}[1]{{#1}}
    \newcommand{\PreprocessorTok}[1]{\textcolor[rgb]{0.74,0.48,0.00}{{#1}}}
    \newcommand{\AttributeTok}[1]{\textcolor[rgb]{0.49,0.56,0.16}{{#1}}}
    \newcommand{\InformationTok}[1]{\textcolor[rgb]{0.38,0.63,0.69}{\textbf{\textit{{#1}}}}}
    \newcommand{\WarningTok}[1]{\textcolor[rgb]{0.38,0.63,0.69}{\textbf{\textit{{#1}}}}}
    
    
    % Define a nice break command that doesn't care if a line doesn't already
    % exist.
    \def\br{\hspace*{\fill} \\* }
    % Math Jax compatibility definitions
    \def\gt{>}
    \def\lt{<}
    \let\Oldtex\TeX
    \let\Oldlatex\LaTeX
    \renewcommand{\TeX}{\textrm{\Oldtex}}
    \renewcommand{\LaTeX}{\textrm{\Oldlatex}}
    % Document parameters
    % Document title
    \title{
      {\Large Лекция 4} \\
      Случайные величины и их распределения
    }
    \date{09 марта 2022\,г.}
    
    
    
% Pygments definitions
\makeatletter
\def\PY@reset{\let\PY@it=\relax \let\PY@bf=\relax%
    \let\PY@ul=\relax \let\PY@tc=\relax%
    \let\PY@bc=\relax \let\PY@ff=\relax}
\def\PY@tok#1{\csname PY@tok@#1\endcsname}
\def\PY@toks#1+{\ifx\relax#1\empty\else%
    \PY@tok{#1}\expandafter\PY@toks\fi}
\def\PY@do#1{\PY@bc{\PY@tc{\PY@ul{%
    \PY@it{\PY@bf{\PY@ff{#1}}}}}}}
\def\PY#1#2{\PY@reset\PY@toks#1+\relax+\PY@do{#2}}

\@namedef{PY@tok@w}{\def\PY@tc##1{\textcolor[rgb]{0.73,0.73,0.73}{##1}}}
\@namedef{PY@tok@c}{\let\PY@it=\textit\def\PY@tc##1{\textcolor[rgb]{0.24,0.48,0.48}{##1}}}
\@namedef{PY@tok@cp}{\def\PY@tc##1{\textcolor[rgb]{0.61,0.40,0.00}{##1}}}
\@namedef{PY@tok@k}{\let\PY@bf=\textbf\def\PY@tc##1{\textcolor[rgb]{0.00,0.50,0.00}{##1}}}
\@namedef{PY@tok@kp}{\def\PY@tc##1{\textcolor[rgb]{0.00,0.50,0.00}{##1}}}
\@namedef{PY@tok@kt}{\def\PY@tc##1{\textcolor[rgb]{0.69,0.00,0.25}{##1}}}
\@namedef{PY@tok@o}{\def\PY@tc##1{\textcolor[rgb]{0.40,0.40,0.40}{##1}}}
\@namedef{PY@tok@ow}{\let\PY@bf=\textbf\def\PY@tc##1{\textcolor[rgb]{0.67,0.13,1.00}{##1}}}
\@namedef{PY@tok@nb}{\def\PY@tc##1{\textcolor[rgb]{0.00,0.50,0.00}{##1}}}
\@namedef{PY@tok@nf}{\def\PY@tc##1{\textcolor[rgb]{0.00,0.00,1.00}{##1}}}
\@namedef{PY@tok@nc}{\let\PY@bf=\textbf\def\PY@tc##1{\textcolor[rgb]{0.00,0.00,1.00}{##1}}}
\@namedef{PY@tok@nn}{\let\PY@bf=\textbf\def\PY@tc##1{\textcolor[rgb]{0.00,0.00,1.00}{##1}}}
\@namedef{PY@tok@ne}{\let\PY@bf=\textbf\def\PY@tc##1{\textcolor[rgb]{0.80,0.25,0.22}{##1}}}
\@namedef{PY@tok@nv}{\def\PY@tc##1{\textcolor[rgb]{0.10,0.09,0.49}{##1}}}
\@namedef{PY@tok@no}{\def\PY@tc##1{\textcolor[rgb]{0.53,0.00,0.00}{##1}}}
\@namedef{PY@tok@nl}{\def\PY@tc##1{\textcolor[rgb]{0.46,0.46,0.00}{##1}}}
\@namedef{PY@tok@ni}{\let\PY@bf=\textbf\def\PY@tc##1{\textcolor[rgb]{0.44,0.44,0.44}{##1}}}
\@namedef{PY@tok@na}{\def\PY@tc##1{\textcolor[rgb]{0.41,0.47,0.13}{##1}}}
\@namedef{PY@tok@nt}{\let\PY@bf=\textbf\def\PY@tc##1{\textcolor[rgb]{0.00,0.50,0.00}{##1}}}
\@namedef{PY@tok@nd}{\def\PY@tc##1{\textcolor[rgb]{0.67,0.13,1.00}{##1}}}
\@namedef{PY@tok@s}{\def\PY@tc##1{\textcolor[rgb]{0.73,0.13,0.13}{##1}}}
\@namedef{PY@tok@sd}{\let\PY@it=\textit\def\PY@tc##1{\textcolor[rgb]{0.73,0.13,0.13}{##1}}}
\@namedef{PY@tok@si}{\let\PY@bf=\textbf\def\PY@tc##1{\textcolor[rgb]{0.64,0.35,0.47}{##1}}}
\@namedef{PY@tok@se}{\let\PY@bf=\textbf\def\PY@tc##1{\textcolor[rgb]{0.67,0.36,0.12}{##1}}}
\@namedef{PY@tok@sr}{\def\PY@tc##1{\textcolor[rgb]{0.64,0.35,0.47}{##1}}}
\@namedef{PY@tok@ss}{\def\PY@tc##1{\textcolor[rgb]{0.10,0.09,0.49}{##1}}}
\@namedef{PY@tok@sx}{\def\PY@tc##1{\textcolor[rgb]{0.00,0.50,0.00}{##1}}}
\@namedef{PY@tok@m}{\def\PY@tc##1{\textcolor[rgb]{0.40,0.40,0.40}{##1}}}
\@namedef{PY@tok@gh}{\let\PY@bf=\textbf\def\PY@tc##1{\textcolor[rgb]{0.00,0.00,0.50}{##1}}}
\@namedef{PY@tok@gu}{\let\PY@bf=\textbf\def\PY@tc##1{\textcolor[rgb]{0.50,0.00,0.50}{##1}}}
\@namedef{PY@tok@gd}{\def\PY@tc##1{\textcolor[rgb]{0.63,0.00,0.00}{##1}}}
\@namedef{PY@tok@gi}{\def\PY@tc##1{\textcolor[rgb]{0.00,0.52,0.00}{##1}}}
\@namedef{PY@tok@gr}{\def\PY@tc##1{\textcolor[rgb]{0.89,0.00,0.00}{##1}}}
\@namedef{PY@tok@ge}{\let\PY@it=\textit}
\@namedef{PY@tok@gs}{\let\PY@bf=\textbf}
\@namedef{PY@tok@gp}{\let\PY@bf=\textbf\def\PY@tc##1{\textcolor[rgb]{0.00,0.00,0.50}{##1}}}
\@namedef{PY@tok@go}{\def\PY@tc##1{\textcolor[rgb]{0.44,0.44,0.44}{##1}}}
\@namedef{PY@tok@gt}{\def\PY@tc##1{\textcolor[rgb]{0.00,0.27,0.87}{##1}}}
\@namedef{PY@tok@err}{\def\PY@bc##1{{\setlength{\fboxsep}{\string -\fboxrule}\fcolorbox[rgb]{1.00,0.00,0.00}{1,1,1}{\strut ##1}}}}
\@namedef{PY@tok@kc}{\let\PY@bf=\textbf\def\PY@tc##1{\textcolor[rgb]{0.00,0.50,0.00}{##1}}}
\@namedef{PY@tok@kd}{\let\PY@bf=\textbf\def\PY@tc##1{\textcolor[rgb]{0.00,0.50,0.00}{##1}}}
\@namedef{PY@tok@kn}{\let\PY@bf=\textbf\def\PY@tc##1{\textcolor[rgb]{0.00,0.50,0.00}{##1}}}
\@namedef{PY@tok@kr}{\let\PY@bf=\textbf\def\PY@tc##1{\textcolor[rgb]{0.00,0.50,0.00}{##1}}}
\@namedef{PY@tok@bp}{\def\PY@tc##1{\textcolor[rgb]{0.00,0.50,0.00}{##1}}}
\@namedef{PY@tok@fm}{\def\PY@tc##1{\textcolor[rgb]{0.00,0.00,1.00}{##1}}}
\@namedef{PY@tok@vc}{\def\PY@tc##1{\textcolor[rgb]{0.10,0.09,0.49}{##1}}}
\@namedef{PY@tok@vg}{\def\PY@tc##1{\textcolor[rgb]{0.10,0.09,0.49}{##1}}}
\@namedef{PY@tok@vi}{\def\PY@tc##1{\textcolor[rgb]{0.10,0.09,0.49}{##1}}}
\@namedef{PY@tok@vm}{\def\PY@tc##1{\textcolor[rgb]{0.10,0.09,0.49}{##1}}}
\@namedef{PY@tok@sa}{\def\PY@tc##1{\textcolor[rgb]{0.73,0.13,0.13}{##1}}}
\@namedef{PY@tok@sb}{\def\PY@tc##1{\textcolor[rgb]{0.73,0.13,0.13}{##1}}}
\@namedef{PY@tok@sc}{\def\PY@tc##1{\textcolor[rgb]{0.73,0.13,0.13}{##1}}}
\@namedef{PY@tok@dl}{\def\PY@tc##1{\textcolor[rgb]{0.73,0.13,0.13}{##1}}}
\@namedef{PY@tok@s2}{\def\PY@tc##1{\textcolor[rgb]{0.73,0.13,0.13}{##1}}}
\@namedef{PY@tok@sh}{\def\PY@tc##1{\textcolor[rgb]{0.73,0.13,0.13}{##1}}}
\@namedef{PY@tok@s1}{\def\PY@tc##1{\textcolor[rgb]{0.73,0.13,0.13}{##1}}}
\@namedef{PY@tok@mb}{\def\PY@tc##1{\textcolor[rgb]{0.40,0.40,0.40}{##1}}}
\@namedef{PY@tok@mf}{\def\PY@tc##1{\textcolor[rgb]{0.40,0.40,0.40}{##1}}}
\@namedef{PY@tok@mh}{\def\PY@tc##1{\textcolor[rgb]{0.40,0.40,0.40}{##1}}}
\@namedef{PY@tok@mi}{\def\PY@tc##1{\textcolor[rgb]{0.40,0.40,0.40}{##1}}}
\@namedef{PY@tok@il}{\def\PY@tc##1{\textcolor[rgb]{0.40,0.40,0.40}{##1}}}
\@namedef{PY@tok@mo}{\def\PY@tc##1{\textcolor[rgb]{0.40,0.40,0.40}{##1}}}
\@namedef{PY@tok@ch}{\let\PY@it=\textit\def\PY@tc##1{\textcolor[rgb]{0.24,0.48,0.48}{##1}}}
\@namedef{PY@tok@cm}{\let\PY@it=\textit\def\PY@tc##1{\textcolor[rgb]{0.24,0.48,0.48}{##1}}}
\@namedef{PY@tok@cpf}{\let\PY@it=\textit\def\PY@tc##1{\textcolor[rgb]{0.24,0.48,0.48}{##1}}}
\@namedef{PY@tok@c1}{\let\PY@it=\textit\def\PY@tc##1{\textcolor[rgb]{0.24,0.48,0.48}{##1}}}
\@namedef{PY@tok@cs}{\let\PY@it=\textit\def\PY@tc##1{\textcolor[rgb]{0.24,0.48,0.48}{##1}}}

\def\PYZbs{\char`\\}
\def\PYZus{\char`\_}
\def\PYZob{\char`\{}
\def\PYZcb{\char`\}}
\def\PYZca{\char`\^}
\def\PYZam{\char`\&}
\def\PYZlt{\char`\<}
\def\PYZgt{\char`\>}
\def\PYZsh{\char`\#}
\def\PYZpc{\char`\%}
\def\PYZdl{\char`\$}
\def\PYZhy{\char`\-}
\def\PYZsq{\char`\'}
\def\PYZdq{\char`\"}
\def\PYZti{\char`\~}
% for compatibility with earlier versions
\def\PYZat{@}
\def\PYZlb{[}
\def\PYZrb{]}
\makeatother


    % For linebreaks inside Verbatim environment from package fancyvrb. 
    \makeatletter
        \newbox\Wrappedcontinuationbox 
        \newbox\Wrappedvisiblespacebox 
        \newcommand*\Wrappedvisiblespace {\textcolor{red}{\textvisiblespace}} 
        \newcommand*\Wrappedcontinuationsymbol {\textcolor{red}{\llap{\tiny$\m@th\hookrightarrow$}}} 
        \newcommand*\Wrappedcontinuationindent {3ex } 
        \newcommand*\Wrappedafterbreak {\kern\Wrappedcontinuationindent\copy\Wrappedcontinuationbox} 
        % Take advantage of the already applied Pygments mark-up to insert 
        % potential linebreaks for TeX processing. 
        %        {, <, #, %, $, ' and ": go to next line. 
        %        _, }, ^, &, >, - and ~: stay at end of broken line. 
        % Use of \textquotesingle for straight quote. 
        \newcommand*\Wrappedbreaksatspecials {% 
            \def\PYGZus{\discretionary{\char`\_}{\Wrappedafterbreak}{\char`\_}}% 
            \def\PYGZob{\discretionary{}{\Wrappedafterbreak\char`\{}{\char`\{}}% 
            \def\PYGZcb{\discretionary{\char`\}}{\Wrappedafterbreak}{\char`\}}}% 
            \def\PYGZca{\discretionary{\char`\^}{\Wrappedafterbreak}{\char`\^}}% 
            \def\PYGZam{\discretionary{\char`\&}{\Wrappedafterbreak}{\char`\&}}% 
            \def\PYGZlt{\discretionary{}{\Wrappedafterbreak\char`\<}{\char`\<}}% 
            \def\PYGZgt{\discretionary{\char`\>}{\Wrappedafterbreak}{\char`\>}}% 
            \def\PYGZsh{\discretionary{}{\Wrappedafterbreak\char`\#}{\char`\#}}% 
            \def\PYGZpc{\discretionary{}{\Wrappedafterbreak\char`\%}{\char`\%}}% 
            \def\PYGZdl{\discretionary{}{\Wrappedafterbreak\char`\$}{\char`\$}}% 
            \def\PYGZhy{\discretionary{\char`\-}{\Wrappedafterbreak}{\char`\-}}% 
            \def\PYGZsq{\discretionary{}{\Wrappedafterbreak\textquotesingle}{\textquotesingle}}% 
            \def\PYGZdq{\discretionary{}{\Wrappedafterbreak\char`\"}{\char`\"}}% 
            \def\PYGZti{\discretionary{\char`\~}{\Wrappedafterbreak}{\char`\~}}% 
        } 
        % Some characters . , ; ? ! / are not pygmentized. 
        % This macro makes them "active" and they will insert potential linebreaks 
        \newcommand*\Wrappedbreaksatpunct {% 
            \lccode`\~`\.\lowercase{\def~}{\discretionary{\hbox{\char`\.}}{\Wrappedafterbreak}{\hbox{\char`\.}}}% 
            \lccode`\~`\,\lowercase{\def~}{\discretionary{\hbox{\char`\,}}{\Wrappedafterbreak}{\hbox{\char`\,}}}% 
            \lccode`\~`\;\lowercase{\def~}{\discretionary{\hbox{\char`\;}}{\Wrappedafterbreak}{\hbox{\char`\;}}}% 
            \lccode`\~`\:\lowercase{\def~}{\discretionary{\hbox{\char`\:}}{\Wrappedafterbreak}{\hbox{\char`\:}}}% 
            \lccode`\~`\?\lowercase{\def~}{\discretionary{\hbox{\char`\?}}{\Wrappedafterbreak}{\hbox{\char`\?}}}% 
            \lccode`\~`\!\lowercase{\def~}{\discretionary{\hbox{\char`\!}}{\Wrappedafterbreak}{\hbox{\char`\!}}}% 
            \lccode`\~`\/\lowercase{\def~}{\discretionary{\hbox{\char`\/}}{\Wrappedafterbreak}{\hbox{\char`\/}}}% 
            \catcode`\.\active
            \catcode`\,\active 
            \catcode`\;\active
            \catcode`\:\active
            \catcode`\?\active
            \catcode`\!\active
            \catcode`\/\active 
            \lccode`\~`\~ 	
        }
    \makeatother

    \let\OriginalVerbatim=\Verbatim
    \makeatletter
    \renewcommand{\Verbatim}[1][1]{%
        %\parskip\z@skip
        \sbox\Wrappedcontinuationbox {\Wrappedcontinuationsymbol}%
        \sbox\Wrappedvisiblespacebox {\FV@SetupFont\Wrappedvisiblespace}%
        \def\FancyVerbFormatLine ##1{\hsize\linewidth
            \vtop{\raggedright\hyphenpenalty\z@\exhyphenpenalty\z@
                \doublehyphendemerits\z@\finalhyphendemerits\z@
                \strut ##1\strut}%
        }%
        % If the linebreak is at a space, the latter will be displayed as visible
        % space at end of first line, and a continuation symbol starts next line.
        % Stretch/shrink are however usually zero for typewriter font.
        \def\FV@Space {%
            \nobreak\hskip\z@ plus\fontdimen3\font minus\fontdimen4\font
            \discretionary{\copy\Wrappedvisiblespacebox}{\Wrappedafterbreak}
            {\kern\fontdimen2\font}%
        }%
        
        % Allow breaks at special characters using \PYG... macros.
        \Wrappedbreaksatspecials
        % Breaks at punctuation characters . , ; ? ! and / need catcode=\active 	
        \OriginalVerbatim[#1,codes*=\Wrappedbreaksatpunct]%
    }
    \makeatother

    % Exact colors from NB
    \definecolor{incolor}{HTML}{303F9F}
    \definecolor{outcolor}{HTML}{D84315}
    \definecolor{cellborder}{HTML}{CFCFCF}
    \definecolor{cellbackground}{HTML}{F7F7F7}
    
    % prompt
    \makeatletter
    \newcommand{\boxspacing}{\kern\kvtcb@left@rule\kern\kvtcb@boxsep}
    \makeatother
    \newcommand{\prompt}[4]{
        {\ttfamily\llap{{\color{#2}[#3]:\hspace{3pt}#4}}\vspace{-\baselineskip}}
    }
    

    
    % Prevent overflowing lines due to hard-to-break entities
    \sloppy 
    % Setup hyperref package
    \hypersetup{
      breaklinks=true,  % so long urls are correctly broken across lines
      colorlinks=true,
      urlcolor=urlcolor,
      linkcolor=linkcolor,
      citecolor=citecolor,
      }
    % Slightly bigger margins than the latex defaults
    
    \geometry{verbose,tmargin=1in,bmargin=1in,lmargin=1in,rmargin=1in}
    
    

\begin{document}
    
  \maketitle
  \thispagestyle{empty}
  \tableofcontents
  \pagebreak



    \hypertarget{ux43fux43eux43dux44fux442ux438ux435-ux441ux43bux443ux447ux430ux439ux43dux43eux439-ux432ux435ux43bux438ux447ux438ux43dux44b}{%
\section{Понятие случайной
величины}\label{ux43fux43eux43dux44fux442ux438ux435-ux441ux43bux443ux447ux430ux439ux43dux43eux439-ux432ux435ux43bux438ux447ux438ux43dux44b}}

\textbf{Определение.} Функция \(\xi: \Omega \rightarrow \mathbb{R}\)
называется \emph{случайной величиной}, если для любого
\(x \in \mathbb{R}\) множество \(\{\omega : \xi(\omega) \le x\}\)
является измеримым (или, что тоже самое, принадлежит \(\sigma\)-алгебре
\(\mathcal{F}\)).

\textbf{Замечание.} Читатель, не желающий забивать себе голову абстракциями, связанными с \(\sigma\)-алгебрами событий и с~измеримостью, может смело считать, что любое множество элементарных исходов есть событие, и, следовательно, случайная величина есть произвольная функция из \(\Omega\) в \(\mathbb{R}\).


\begin{center}\rule{0.5\linewidth}{0.5pt}\end{center}

\hypertarget{ux440ux430ux441ux43fux440ux435ux434ux435ux43bux435ux43dux438ux44f-ux441ux43bux443ux447ux430ux439ux43dux44bux445-ux432ux435ux43bux438ux447ux438ux43d}{%
\section{Распределения случайных
величин}\label{ux440ux430ux441ux43fux440ux435ux434ux435ux43bux435ux43dux438ux44f-ux441ux43bux443ux447ux430ux439ux43dux44bux445-ux432ux435ux43bux438ux447ux438ux43d}}

\hypertarget{ux444ux443ux43dux43aux446ux438ux44f-ux440ux430ux441ux43fux440ux435ux434ux435ux43bux435ux43dux438ux44f}{%
\subsection{Функция
распределения}\label{ux444ux443ux43dux43aux446ux438ux44f-ux440ux430ux441ux43fux440ux435ux434ux435ux43bux435ux43dux438ux44f}}

Существуют различные типы распределений случайных величин. Вся
вероятностная мера может быть сосредоточена в нескольких точках прямой,
а может быть «размазана» по некоторому интервалу или по всей прямой. В
зависимости от типа множества, на котором сосредоточена вероятностная
мера, распределения делят на дискретные, абсолютно непрерывные,
сингулярные и их смеси. Нас будут интересовать \emph{дискретные} и
\emph{абсолютно непрерывные} случайные величины.

\textbf{Определение.} \emph{Распределением} случайной величины \(\xi\)
называется вероятностная мера
\(P_\xi(B) = \mathrm{P}\{\omega: \xi(\omega) \in B\}\) на множестве
\(B\) (борелевских) подмножеств \(\mathbb{R}\).

Можно представлять себе распределение случайной величины \(\xi\) как
соответствие между множествами \(B\) и вероятностями
\(\mathrm{P}{\xi \in B}\).

\textbf{Определение.} Функция \[
  F_\xi(x) = \mathrm{P}\left\{ \omega: \xi(\omega) \le x \right\}, \quad x \in \mathbb{R}
\] называется \emph{функцией распределения} случайной величины \(\xi\).

    \textbf{Свойства функции распределения:}

\begin{enumerate}
\def\labelenumi{\arabic{enumi}.}
\tightlist
\item
  \(F_\xi(x) \in [0,1]\);
\item
  \(F_\xi(x)\) монотонно не убывает;
\item
  \(F_\xi(x)\) непрерывна справа;
\item
  существуют пределы \(\lim\limits_{x\rightarrow +\infty}F_\xi(x)=1\),
  \(\lim\limits_{x\rightarrow -\infty}F_\xi(x)=0\).
\end{enumerate}

    \hypertarget{ux434ux438ux441ux43aux440ux435ux442ux43dux44bux435-ux441ux43bux443ux447ux430ux439ux43dux44bux435-ux432ux435ux43bux438ux447ux438ux43dux44b}{%
\subsection{Дискретные случайные
величины}\label{ux434ux438ux441ux43aux440ux435ux442ux43dux44bux435-ux441ux43bux443ux447ux430ux439ux43dux44bux435-ux432ux435ux43bux438ux447ux438ux43dux44b}}

\textbf{Определение.} Случайная величина \(\xi\) называется
\emph{дискретной} (имеет \emph{дискретное распределение}), если она
принимает не более чем счётное число значений, т. е. если существует
конечный или счётный набор чисел \(a_1, a_2, \ldots\) такой, что \[
  \mathrm{P}\{\xi=a_i\} > 0 \quad \forall i.
\] Если число значений \emph{конечно}, то такая случайная величина
называется \emph{простой}.

Для дискретной случайной величины \(\xi\) мера \(P_\xi\) сосредоточена
не более чем в счётном числе точек и может быть представлена в виде \[
  \mathrm{P}\{\xi \in B\} \equiv P_\xi(B) = \sum\limits_{k:x_k \in B} \mathrm{P}\{\xi=x_k\}.
\]

    \hypertarget{ux430ux431ux441ux43eux43bux44eux442ux43dux43e-ux43dux435ux43fux440ux435ux440ux44bux432ux43dux44bux435-ux441ux43bux443ux447ux430ux439ux43dux44bux435-ux432ux435ux43bux438ux447ux438ux43dux44b}{%
\subsection{Абсолютно непрерывные случайные
величины}\label{ux430ux431ux441ux43eux43bux44eux442ux43dux43e-ux43dux435ux43fux440ux435ux440ux44bux432ux43dux44bux435-ux441ux43bux443ux447ux430ux439ux43dux44bux435-ux432ux435ux43bux438ux447ux438ux43dux44b}}

\textbf{Определение.} Случайная величина \(\xi\) называется
\emph{абсолютно непрерывной}, если существует неотрицательная функция
\(f = f_\xi(x)\), называемая \emph{плотностью распределения}, такая, что

\[ F_\xi(x) = \int\limits_{-\infty}^x f_\xi(t) dt, \quad \forall x \in \mathbb{R}. \]

\textbf{Свойства:}

\begin{enumerate}
\def\labelenumi{\arabic{enumi}.}
\tightlist
\item
  \(\forall x: f_\xi (x) \ge 0\);
\item
  \(\int\limits_{-\infty}^{\infty} f_\xi(t) dt = 1\).
\end{enumerate}

Эти два свойства полностью характеризуют класс плотностей.

\textbf{Теорема.} Если функция \(f\) обладает свойствами (1) и (2), то
существует вероятностное пространство и случайная величина \(\xi\) на
нём, для которой \(f\) является плотностью распределения.

    \begin{center}\rule{0.5\linewidth}{0.5pt}\end{center}

    \hypertarget{ux43cux43dux43eux433ux43eux43cux435ux440ux43dux44bux435-ux440ux430ux441ux43fux440ux435ux434ux435ux43bux435ux43dux438ux44f}{%
\section{Многомерные
распределения}\label{ux43cux43dux43eux433ux43eux43cux435ux440ux43dux44bux435-ux440ux430ux441ux43fux440ux435ux434ux435ux43bux435ux43dux438ux44f}}

    \hypertarget{ux43eux43fux440ux435ux434ux435ux43bux435ux43dux438ux435}{%
\subsection{Определение}\label{ux43eux43fux440ux435ux434ux435ux43bux435ux43dux438ux435}}

Пусть случайные величины \(\xi_1, \ldots, \xi_n\) заданы на одном
вероятностном пространстве \((\Omega, \mathcal{F}, \mathrm{P})\).

\textbf{Определение.} Функция \[
  F_{\xi_1, \ldots, \xi_n}(x_1, \ldots, x_n) = \mathrm{P}(\xi_1 \le x_1, \ldots, \xi_n \le x_n)
\] называется \emph{функцией совместного распределения} случайных
величин \(\xi_1, \ldots , \xi_n\).

Далее для простоты обозначений ограничимся вектором из двух величин
\((\xi, \eta)\).

\textbf{Определение.} Говорят, что случайные величины \(\xi, \eta\)
имеют \emph{абсолютно непрерывное совместное распределение}, если
существует неотрицательная функция \(f_{\xi, \eta}(x, y)\), называемая
\emph{плотностью}, такая, что

\[
  F_{\xi, \eta}(x, y) = \int\limits_{-\infty}^{x} \int\limits_{-\infty}^{y} f_{\xi, \eta}(t_1, t_2) dt_1 dt_2
\]

\textbf{Теорема.} Если случайные величины \(\xi\) и \(\eta\) имеют
абсолютно непрерывное совместное распределение с плотностью \(f(x, y)\),
то \(\xi\) и \(\eta\) в отдельности также имеют абсолютно непрерывное
распределение с плотностями:

\[
  f_{\xi}(x) = \int\limits_{-\infty}^{\infty} f_{\xi, \eta}(x, y)dy \quad \mathrm{и} \quad f_{\eta}(y) = \int\limits_{-\infty}^{\infty} f_{\xi, \eta}(x, y)dx.
\]

Соответствующие распределения называются \emph{частными} или
\emph{маргинальными}.

    \hypertarget{ux444ux443ux43dux43aux446ux438ux438-ux43eux442-ux434ux432ux443ux445-ux441ux43bux443ux447ux430ux439ux43dux44bux445-ux432ux435ux43bux438ux447ux438ux43d}{%
\subsection{Функции от двух случайных
величин}\label{ux444ux443ux43dux43aux446ux438ux438-ux43eux442-ux434ux432ux443ux445-ux441ux43bux443ux447ux430ux439ux43dux44bux445-ux432ux435ux43bux438ux447ux438ux43d}}

Если нам известно совместное распределение двух или нескольких случайных
величин, становится возможным отыскать распределение суммы, разности,
произведения, частного, иных функций от этих случайных величин.

Заметим, что знания только частных распределений двух случайных величин
недостаточно для отыскания распределения, например, суммы этих величин.
Для этого необходимо знать их \emph{совместное распределение}.
Распределение суммы (и любой иной функции) не определяется, вообще
говоря, распределениями слагаемых: при одних и тех же распределениях
слагаемых распределение суммы может быть разным в зависимости от
совместного распределения слагаемых.

\textbf{Пример.} Пусть случайная величина \(\xi\) имеет стандартное
нормальное распределение. Возьмём \(\eta = -\xi\). Тогда \(\eta\) тоже
имеет стандартное нормальное распределение, а сумма \(\xi + \eta = 0\)
имеет вырожденное распределение.

    Пусть \(\xi_1\) и \(\xi2\) --- случайные величины с плотностью
совместного распределения \(f_{\xi_1,\xi_2}(x_1, x_2)\). Пусть задана
функция \(g: \mathbb{R}^2 \rightarrow \mathbb{R}\). Требуется найти
функцию (а если существует, то и плотность) распределения случайной
величины \(\eta = g (\xi_1,\xi_2)\).

Пользуясь тем, что вероятность случайному вектору попасть в некоторую
область можно вычислить как объем под графиком плотности распределения
вектора над этой областью, сформулируем следующую теорему.

\textbf{Теорема.} Пусть \(x \in \mathbb{R}\), и область
\(D_x \subseteq \mathbb{R}^2\) состоит из точек \((u, v)\) таких, что
\(g(u, v) < x\). Тогда случайная величина \(\eta = g(\xi_1,\xi_2)\)
имеет функцию распределения \[
F_\eta(x) = \mathrm{P}\{g(\xi_1,\xi_2) < x\} = \mathrm{P}\{(\xi_1,\xi_2) \in D_x\} = \iint\limits_{D_x} f_{\xi_1,\xi_2}(u,v)\,dudv.
\]

\textbf{Формула свёртки}

Если случайные величины \(\xi_1\) и \(\xi_2\) независимы и имеют
абсолютно непрерывные распределения с плотностями \(f_{\xi_1}(x_1)\) и
\(f_{\xi_2}(x_2)\), то плотность распределения суммы \(\xi_1 + \xi_2\)
равна «свёртке» плотностей \(f_{\xi_1}\) и \(f_{\xi_2}\): \[
  f_{\xi_1 +\xi_2}(t) = \int\limits_{-\infty}^{\infty} f_{\xi_1}(u) f_{\xi_2}(t-u) du
  = \int\limits_{-\infty}^{\infty} f_{\xi_2}(u) f_{\xi_1}(t-u) du.
\]

    \emph{Доказательство.} Функция распределения суммы равна \[
  F_{\xi_1 +\xi_2}(x) = \iint\limits_{D_x} f_{\xi_1}(x_1) f_{\xi_2}(x_2) dx_1 dx_2,
\]

где \(D_x = \{(x_1,x_2) | x_1+x_2<x\}\).

Интегрирование по области \(D_x\) можно заменить последовательным
вычислением двух интегралов: наружного --- по переменной \(x_1\),
меняющейся в пределах от \(-\infty\) до \(+\infty\), и внутреннего ---
по переменной \(x_2\), которая при каждом \(x_1\) должна быть меньше,
чем \(x - x_1\): \[
  F_{\xi_1 +\xi_2}(x) = \int\limits_{-\infty}^{\infty} \left( \int\limits_{-\infty}^{x-x_1} f_{\xi_1}(x_1) f_{\xi_2}(x_2) dx_2 \right) dx_1.
\]

Сделав замену переменной \(x_2 = t - x_1\) и поменяв затем порядок
интегрирования, получим \[
  F_{\xi_1 +\xi_2}(x) = \int\limits_{-\infty}^{\infty} \left( \int\limits_{-\infty}^{x} f_{\xi_1}(x_1) f_{\xi_2}(t-x_1) dt \right) dx_1
  = \int\limits_{-\infty}^{x} \left( \int\limits_{-\infty}^{\infty} f_{\xi_1}(x_1) f_{\xi_2}(t-x_1) dx_1 \right) dt.
\]

Итак, мы представили функцию распределения \(F_{\xi_1 +\xi_2}(x)\) в
виде \(\int\limits_{-\infty}^{x} f_{\xi_1+\xi_2}(t)dt\), где \[
  f_{\xi_1+\xi_2}(t) = \int\limits_{-\infty}^{\infty} f_{\xi_1}(u) f_{\xi_2}(t-u) du.
\]

    \hypertarget{ux43dux435ux437ux430ux432ux438ux441ux438ux43cux43eux441ux442ux44c-ux441ux43bux443ux447ux430ux439ux43dux44bux445-ux432ux435ux43bux438ux447ux438ux43d}{%
\subsection{Независимость случайных
величин}\label{ux43dux435ux437ux430ux432ux438ux441ux438ux43cux43eux441ux442ux44c-ux441ux43bux443ux447ux430ux439ux43dux44bux445-ux432ux435ux43bux438ux447ux438ux43d}}

\textbf{Определение.} Случайные величины \(\xi_1, \dots, \xi_n\)
называются \emph{независимыми} (в совокупности), если для любых
\(x_1, \dots, x_n\) справедливо равенство
\[ F_{\xi_1, \dots, \xi_n}(x_1, \dots, x_n) = F_{\xi_1}(x_1) \cdot \ldots \cdot F_{\xi_n}(x_n).\]

\textbf{Дискретный случай}\\
Случайные величины \(\xi_1, \dots, \xi_n\) с дискретными распределениями
независимы (в совокупности), если для любых чисел \(a_1, \dots , a_n\)
имеет место равенство
\[ \mathrm{P}\{\xi_1=a_1, \dots, \xi_n=a_n\} = \mathrm{P}\{\xi_1=a_1\} \cdot \ldots \cdot \mathrm{P}\{\xi_n=a_n\}.\]

\textbf{Абсолютно непрерывный случай}\\
Случайные величины \(\xi_1, \dots, \xi_n\) с абсолютно непрерывными
распределениями независимы (в совокупности) тогда и только тогда, когда
плотность их совместного распределения существует и равна произведению
плотностей каждой величины, т. е. \(\forall x_1, \dots , x_n\):

\[ f_{\xi_1, \dots, \xi_n}(x_1, \dots, x_n) = f_{\xi_1}(x_1) \cdot \ldots \cdot f_{\xi_n}(x_n). \]

    \begin{center}\rule{0.5\linewidth}{0.5pt}\end{center}

    \hypertarget{ux447ux438ux441ux43bux43eux432ux44bux435-ux445ux430ux440ux430ux43aux442ux435ux440ux438ux441ux442ux438ux43aux438-ux441ux43bux443ux447ux430ux439ux43dux44bux445-ux432ux435ux43bux438ux447ux438ux43d}{%
\section{Числовые характеристики случайных
величин}\label{ux447ux438ux441ux43bux43eux432ux44bux435-ux445ux430ux440ux430ux43aux442ux435ux440ux438ux441ux442ux438ux43aux438-ux441ux43bux443ux447ux430ux439ux43dux44bux445-ux432ux435ux43bux438ux447ux438ux43d}}

\hypertarget{ux43cux430ux442ux435ux43cux430ux442ux438ux447ux435ux441ux43aux43eux435-ux43eux436ux438ux434ux430ux43dux438ux435}{%
\subsection{Математическое
ожидание}\label{ux43cux430ux442ux435ux43cux430ux442ux438ux447ux435ux441ux43aux43eux435-ux43eux436ux438ux434ux430ux43dux438ux435}}

\textbf{Определение (дискретный случай).} Пусть \(\xi\) --- дискретная
случайная величина на пространстве
\((\Omega, \mathcal{F}, \mathrm{P})\), а \(X\) --- множество её
значений. Тогда \emph{математическим ожиданием} \(\xi\) называется
число, равное
\[ \mathrm{E}\xi = \sum\limits_{x \in X}x\mathrm{P}(\xi = x), \] если
этот ряд сходится абсолютно.

\textbf{Определение (абсолютно непрерывный случай).} Пусть \(\xi\) ---
абсолютно непрерывная случайная величина на пространстве
\((\Omega, \mathcal{F}, \mathrm{P})\), а \(F_\xi\), \(f_\xi\) --- её
функция распределения и плотность. Тогда математическим ожиданием
\(\xi\) называется число, равное
\[ \mathrm{E}\xi = \int\limits_{-\infty}^{\infty} xdF_\xi(x) = \int\limits_{-\infty}^{\infty} xf_\xi(x)dx, \]
если этот интеграл сходится абсолютно.

\textbf{Свойства математического ожидания:}

\begin{enumerate}
\def\labelenumi{\arabic{enumi}.}
\tightlist
\item
  Если \(\xi \ge 0\), то \(\mathrm{E}\xi \ge 0\).
\item
  \(\mathrm{E}(a\xi +b\eta) = a\mathrm{E}\xi +b\mathrm{E}\eta\),
  \(\hspace{0.5em}\) \(a\), \(b\) --- постоянные (\emph{линейность}).
\item
  Если \(\xi \ge \eta\), то \(\mathrm{E}\xi \ge \mathrm{E}\eta\).
\item
  \(|\mathrm{E}\xi| \le \mathrm{E}|\xi|\).
\item
  Если \(\xi\) и \(\eta\) независимы, то
  \(\mathrm{E}\xi\eta = \mathrm{E}\xi \cdot \mathrm{E}\eta\).
\item
  \((\mathrm{E}|\xi\eta|)^2 \le \mathrm{E}\xi^2 \cdot \mathrm{E}\eta^2\)
  (\emph{неравенство Коши--Буняковского--Шварца}).
\end{enumerate}

    \hypertarget{ux434ux438ux441ux43fux435ux440ux441ux438ux44f}{%
\subsection{Дисперсия}\label{ux434ux438ux441ux43fux435ux440ux441ux438ux44f}}

\textbf{Определение.} \emph{Дисперсией} случайной величины \(\xi\)
называется величина
\[ \mathrm{D} \xi = \mathrm{E} \left( \xi - \mathrm{E} \xi \right)^2. \]

Величина \(\sigma_\xi = +\sqrt{\mathrm{D} \xi}\) называется
\emph{стандартным отклонением} значений случайной величины \(\xi\) от её
среднего значения \(\mathrm{E} \xi\).

\textbf{Свойства дисперсии:}

\begin{enumerate}
\def\labelenumi{\arabic{enumi}.}
\tightlist
\item
  Дисперсию случайной величины \(\xi\) можно вычислить как разность
  математического ожидания кавдрата величины и квадрата математического
  ожидания:
  \(\mathrm{D}\xi = \mathrm{E} \xi^2 - \left( \mathrm{E} \xi \right)^2\).
\item
  \(\mathrm{D}\xi \ge 0\).
\item
  \(\mathrm{D}(a + b\xi) = b^2 \mathrm{D}\xi\), \(\hspace{0.5em}\)
  \(a\), \(b\) --- постоянные.
\item
  \(\mathrm{D}(\xi + \eta) = \mathrm{E} \left[ (\xi-\mathrm{E}\xi) + (\eta-\mathrm{E}\eta) \right]^2 = \mathrm{D}\xi + \mathrm{D}\eta + 2\mathrm{E}(\xi-\mathrm{E}\xi)(\eta-\mathrm{E}\eta)\)
\end{enumerate}

    \hypertarget{ux43aux43eux432ux430ux440ux438ux430ux446ux438ux44f}{%
\subsection{Ковариация}\label{ux43aux43eux432ux430ux440ux438ux430ux446ux438ux44f}}

\textbf{Определение.} Пусть \(\xi\) и \(\eta\) --- две случайные
величины. Их \emph{ковариацией} называется величина \[
    \mathrm{cov}(\xi, \eta) = \mathrm{E} (\xi-\mathrm{E}\xi)(\eta-\mathrm{E}\eta) = \mathrm{E}(\xi \eta) - \mathrm{E}\xi \mathrm{E}\eta.
\]

С учётом введённого обозначения для ковариации находим, что
\[ \mathrm{D}(\xi+\eta) = \mathrm{D}\xi  + \mathrm{D}\eta +  2\mathrm{cov}(\xi, \eta).\]

Если \(\mathrm{cov}\left( \xi, \eta \right) = 0\), то говорят, что
случайные величины \(\xi\) и \(\eta\) \emph{некоррелированы}.\\
Если \(\xi\) и \(\eta\) некоррелированы, то дисперсия суммы
\(\mathrm{D}(\xi+\eta)\) равна сумме дисперсий:
\[ \mathrm{D}(\xi+\eta) = \mathrm{D}\xi + \mathrm{D}\eta. \]

\textbf{Замечание.} Из некоррелированности \(\xi\) и \(\eta\), вообще
говоря, не следует их независимость. Проиллюстрируем этот факт следующим
примером.

\begin{quote}
\textbf{Пример.} Пусть случайная величина \(\alpha\) принимает значения
0, \(\pi/2\) и \(\pi\) с вероятностями 1/3. Рассмотрим две случайные
величины \(\xi = \sin \alpha\) и \(\eta = \cos \alpha\).\\
Величины \(\xi\) и \(\eta\) некоррелированы, однако они не только
зависимы относительно вероятности, но и \emph{функционально зависимы}:
\(\xi^2 + \eta^2 = 1\).
\end{quote}

    \hypertarget{ux43aux43eux44dux444ux444ux438ux446ux438ux435ux43dux442-ux43aux43eux440ux440ux435ux43bux44fux446ux438ux438}{%
\subsection{Коэффициент
корреляции}\label{ux43aux43eux44dux444ux444ux438ux446ux438ux435ux43dux442-ux43aux43eux440ux440ux435ux43bux44fux446ux438ux438}}

\textbf{Определение.} Если \(\mathrm{D}\xi > 0\),
\(\mathrm{D}\eta > 0\), то величина \[
    \rho(\xi, \eta) = \dfrac{\mathrm{cov}(\xi, \eta)}{\sqrt{\mathrm{D}\xi \cdot \mathrm{D}\eta}} = \dfrac{\mathrm{cov}(\xi, \eta)}{\sigma_\xi \cdot \sigma_\eta}
\] называется \emph{коэффициентом корреляции} случайных величин \(\xi\)
и \(\eta\).

Eсли \(\rho(\xi, \eta) = \pm 1\), то величины \(\xi\) и \(\eta\) линейно
зависимы: \[ \eta =a \xi + b, \] где \(a>0\), если
\(\rho(\xi, \eta) = 1\), \(a<0\), если \(\rho(\xi, \eta) = -1\).

\textbf{Определение.} Говорят, что \(\xi\) и \(\eta\) отрицательно
коррелированы, если \(\rho(\xi, \eta) < 0\); положительно коррелированы,
если \(\rho(\xi, \eta) > 0\); некоррелированы, если
\(\rho(\xi, \eta) = 0\).

\begin{quote}
\textbf{Замечание.} Коэффициент корреляции можно интерпретировать как косинус угла в
некотором пространстве случайных величин, в котором скалярным
произведением является ковариация.
\end{quote}

    \begin{center}\rule{0.5\linewidth}{0.5pt}\end{center}

    \hypertarget{ux43eux43fux442ux438ux43cux430ux43bux44cux43dux430ux44f-ux43bux438ux43dux435ux439ux43dux430ux44f-ux43eux446ux435ux43dux43aux430}{%
\section{Оптимальная линейная
оценка}\label{ux43eux43fux442ux438ux43cux430ux43bux44cux43dux430ux44f-ux43bux438ux43dux435ux439ux43dux430ux44f-ux43eux446ux435ux43dux43aux430}}

Рассмотрим две случайные величины \(\xi\) и \(\eta\). Предположим, что
наблюдению подлежит лишь случайная величина \(\xi\). Если величины
\(\xi\) и \(\eta\) коррелированы, то можно ожидать, что знание значений
\(\xi\) позволит вынести некоторые суждения и о значениях ненаблюдаемой
величины \(\eta\).

Всякую функцию \(f = f(\xi)\) от \(\xi\) будем называть \emph{оценкой}
для \(\eta\). Будем говорить также, что \emph{оценка}
\(f^\ast = f^\ast(\xi)\) \emph{оптимальна в среднеквадратическом
смысле}, если
\[ \mathrm{E}(\eta − f^\ast(\xi))^2 = \inf_f \mathrm{E}(\eta − f(\xi))^2. \]

Покажем, как найти оптимальную оценку в классе \emph{линейных} оценок
\(\lambda(\xi) = a + b\xi\). Для этого рассмотрим функцию
\(g(a, b) = \mathrm{E}(\eta − (a+b\xi))^2\). Дифференцируя \(g(a, b)\)
по \(a\) и \(b\), получаем \[
\begin{split}
    \frac{\partial g(a, b)}{\partial a} &= −2 \mathrm{E} \left[ \eta − (a+b\xi) \right], \\
    \frac{\partial g(a, b)}{\partial b} &= −2 \mathrm{E} \left[ (\eta − (a+b\xi))\xi \right],
\end{split}
\] откуда, приравнивая производные к нулю, находим, что
\textbf{оптимальная в среднеквадратическом смысле линейная} оценка есть
\(\lambda^\ast (\xi) = a^\ast + b^\ast \xi\), где \[
    a^\ast = \mathrm{E}\eta − b^\ast\mathrm{E}\xi, \quad b^\ast = \frac{\mathrm{cov}(\xi, \eta)}{\mathrm{D}\xi}.
\]

Иначе говоря, \[
    \lambda^\ast(\xi) = \mathrm{E}\eta + \frac{\mathrm{cov}(\xi, \eta)}{\mathrm{D}\xi} (\xi - \mathrm{E}\xi).
\]

Величина \(\mathrm{E}(\eta − \lambda^\ast(\xi))^2\) называется
\emph{среднеквадратической ошибкой} оценивания. Простой подсчёт
показывает, что эта ошибка равна \[
\Delta^\ast = \mathrm{E}(\eta − \lambda^\ast(\xi))^2 = \mathrm{D}\eta − \frac{\mathrm{cov}^2(\xi, \eta)}{\mathrm{D}\xi} = \mathrm{D}\eta \cdot [1 - \rho^2(\xi, \eta)].
\]

Таким образом, чем больше по модулю коэффициент корреляции
\(\rho(\xi, \eta)\) между \(\xi\) и \(\eta\), тем меньше
среднеквадратическая ошибка оценивания \(\Delta^\ast\). В частности,
если \(|\rho(\xi, \eta)|=1\), то \(\Delta^\ast = 0\). Если же случайные
величины \(\xi\) и \(\eta\) не коррелированы (т. е.
\(\rho(\xi, \eta)=0\)), то \(\lambda^\ast(\xi) = \mathrm{E}\eta\). Таким
образом, в случае отсутствия корреляции между \(\xi\) и \(\eta\) лучшей
оценкой \(\eta\) по \(\xi\) является просто \(\mathrm{E}\eta\).

    \begin{center}\rule{0.5\linewidth}{0.5pt}\end{center}

    \hypertarget{ux438ux441ux442ux43eux447ux43dux438ux43aux438}{%
\section{Источники}\label{ux438ux441ux442ux43eux447ux43dux438ux43aux438}}

\begin{enumerate}
\def\labelenumi{\arabic{enumi}.}
\tightlist
\item
  \emph{Ширяев А.Н.} Вероятность --- 1. --- М.: МЦНМО, 2007. --- 517 с.
\item
  \emph{Чернова Н. И.} Теория вероятностей. Учебное пособие. ---
  Новосибирск, 2007. --- 160 с.
\item
  \emph{Феллер В.} Введение в теорию вероятностей и её приложения. ---
  М.: Мир, 1964. --- 498 с.
\item
  \emph{Шпигельхалтер Д.} Искусство статистики. Как находить ответы в
  данных. --- М.: Манн, Иванов и Фербер, 2021. --- 448 с.
\end{enumerate}


    % Add a bibliography block to the postdoc
    
    
    
\end{document}
