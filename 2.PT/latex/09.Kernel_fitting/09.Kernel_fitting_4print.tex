\documentclass[11pt,a4paper]{article}

    \usepackage[breakable]{tcolorbox}
    \usepackage{parskip} % Stop auto-indenting (to mimic markdown behaviour)
    
    \usepackage{iftex}
    \ifPDFTeX
      \usepackage[T2A]{fontenc}
      \usepackage{mathpazo}
      \usepackage[russian,english]{babel}
    \else
      \usepackage{fontspec}
      \usepackage{polyglossia}
      \setmainlanguage[babelshorthands=true]{russian}    % Язык по-умолчанию русский с поддержкой приятных команд пакета babel
      \setotherlanguage{english}                         % Дополнительный язык = английский (в американской вариации по-умолчанию)

      \defaultfontfeatures{Ligatures=TeX}
      \setmainfont[BoldFont={STIX Two Text SemiBold}]%
      {STIX Two Text}                                    % Шрифт с засечками
      \newfontfamily\cyrillicfont[BoldFont={STIX Two Text SemiBold}]%
      {STIX Two Text}                                    % Шрифт с засечками для кириллицы
      \setsansfont{Fira Sans}                            % Шрифт без засечек
      \newfontfamily\cyrillicfontsf{Fira Sans}           % Шрифт без засечек для кириллицы
      \setmonofont[Scale=0.87,BoldFont={Fira Mono Medium},ItalicFont=[FiraMono-Oblique]]%
      {Fira Mono}%                                       % Моноширинный шрифт
      \newfontfamily\cyrillicfonttt[Scale=0.87,BoldFont={Fira Mono Medium},ItalicFont=[FiraMono-Oblique]]%
      {Fira Mono}                                        % Моноширинный шрифт для кириллицы

      %%% Математические пакеты %%%
      \usepackage{amsthm,amsmath,amscd}   % Математические дополнения от AMS
      \usepackage{amsfonts,amssymb}       % Математические дополнения от AMS
      \usepackage{mathtools}              % Добавляет окружение multlined
      \usepackage{unicode-math}           % Для шрифта STIX Two Math
      \setmathfont{STIX Two Math}         % Математический шрифт
    \fi

    % Basic figure setup, for now with no caption control since it's done
    % automatically by Pandoc (which extracts ![](path) syntax from Markdown).
    \usepackage{graphicx}
    % Maintain compatibility with old templates. Remove in nbconvert 6.0
    \let\Oldincludegraphics\includegraphics
    % Ensure that by default, figures have no caption (until we provide a
    % proper Figure object with a Caption API and a way to capture that
    % in the conversion process - todo).
    \usepackage{caption}
    \DeclareCaptionFormat{nocaption}{}
    \captionsetup{format=nocaption,aboveskip=0pt,belowskip=0pt}

    \usepackage{float}
    \floatplacement{figure}{H} % forces figures to be placed at the correct location
    \usepackage{xcolor} % Allow colors to be defined
    \usepackage{enumerate} % Needed for markdown enumerations to work
    \usepackage{geometry} % Used to adjust the document margins
    \usepackage{amsmath} % Equations
    \usepackage{amssymb} % Equations
    \usepackage{textcomp} % defines textquotesingle
    % Hack from http://tex.stackexchange.com/a/47451/13684:
    \AtBeginDocument{%
        \def\PYZsq{\textquotesingle}% Upright quotes in Pygmentized code
    }
    \usepackage{upquote} % Upright quotes for verbatim code
    \usepackage{eurosym} % defines \euro
    \usepackage[mathletters]{ucs} % Extended unicode (utf-8) support
    \usepackage{fancyvrb} % verbatim replacement that allows latex
    \usepackage{grffile} % extends the file name processing of package graphics 
                         % to support a larger range
    \makeatletter % fix for old versions of grffile with XeLaTeX
    \@ifpackagelater{grffile}{2019/11/01}
    {
      % Do nothing on new versions
    }
    {
      \def\Gread@@xetex#1{%
        \IfFileExists{"\Gin@base".bb}%
        {\Gread@eps{\Gin@base.bb}}%
        {\Gread@@xetex@aux#1}%
      }
    }
    \makeatother
    \usepackage[Export]{adjustbox} % Used to constrain images to a maximum size
    \adjustboxset{max size={0.9\linewidth}{0.9\paperheight}}

    % The hyperref package gives us a pdf with properly built
    % internal navigation ('pdf bookmarks' for the table of contents,
    % internal cross-reference links, web links for URLs, etc.)
    \usepackage{hyperref}
    % The default LaTeX title has an obnoxious amount of whitespace. By default,
    % titling removes some of it. It also provides customization options.
    \usepackage{titling}
    \usepackage{longtable} % longtable support required by pandoc >1.10
    \usepackage{booktabs}  % table support for pandoc > 1.12.2
    \usepackage[inline]{enumitem} % IRkernel/repr support (it uses the enumerate* environment)
    \usepackage[normalem]{ulem} % ulem is needed to support strikethroughs (\sout)
                                % normalem makes italics be italics, not underlines
    \usepackage{mathrsfs}
    

    
    % Colors for the hyperref package
    \definecolor{urlcolor}{rgb}{0,.145,.698}
    \definecolor{linkcolor}{rgb}{.71,0.21,0.01}
    \definecolor{citecolor}{rgb}{.12,.54,.11}

    % ANSI colors
    \definecolor{ansi-black}{HTML}{3E424D}
    \definecolor{ansi-black-intense}{HTML}{282C36}
    \definecolor{ansi-red}{HTML}{E75C58}
    \definecolor{ansi-red-intense}{HTML}{B22B31}
    \definecolor{ansi-green}{HTML}{00A250}
    \definecolor{ansi-green-intense}{HTML}{007427}
    \definecolor{ansi-yellow}{HTML}{DDB62B}
    \definecolor{ansi-yellow-intense}{HTML}{B27D12}
    \definecolor{ansi-blue}{HTML}{208FFB}
    \definecolor{ansi-blue-intense}{HTML}{0065CA}
    \definecolor{ansi-magenta}{HTML}{D160C4}
    \definecolor{ansi-magenta-intense}{HTML}{A03196}
    \definecolor{ansi-cyan}{HTML}{60C6C8}
    \definecolor{ansi-cyan-intense}{HTML}{258F8F}
    \definecolor{ansi-white}{HTML}{C5C1B4}
    \definecolor{ansi-white-intense}{HTML}{A1A6B2}
    \definecolor{ansi-default-inverse-fg}{HTML}{FFFFFF}
    \definecolor{ansi-default-inverse-bg}{HTML}{000000}

    % common color for the border for error outputs.
    \definecolor{outerrorbackground}{HTML}{FFDFDF}

    % commands and environments needed by pandoc snippets
    % extracted from the output of `pandoc -s`
    \providecommand{\tightlist}{%
      \setlength{\itemsep}{0pt}\setlength{\parskip}{0pt}}
    \DefineVerbatimEnvironment{Highlighting}{Verbatim}{commandchars=\\\{\}}
    % Add ',fontsize=\small' for more characters per line
    \newenvironment{Shaded}{}{}
    \newcommand{\KeywordTok}[1]{\textcolor[rgb]{0.00,0.44,0.13}{\textbf{{#1}}}}
    \newcommand{\DataTypeTok}[1]{\textcolor[rgb]{0.56,0.13,0.00}{{#1}}}
    \newcommand{\DecValTok}[1]{\textcolor[rgb]{0.25,0.63,0.44}{{#1}}}
    \newcommand{\BaseNTok}[1]{\textcolor[rgb]{0.25,0.63,0.44}{{#1}}}
    \newcommand{\FloatTok}[1]{\textcolor[rgb]{0.25,0.63,0.44}{{#1}}}
    \newcommand{\CharTok}[1]{\textcolor[rgb]{0.25,0.44,0.63}{{#1}}}
    \newcommand{\StringTok}[1]{\textcolor[rgb]{0.25,0.44,0.63}{{#1}}}
    \newcommand{\CommentTok}[1]{\textcolor[rgb]{0.38,0.63,0.69}{\textit{{#1}}}}
    \newcommand{\OtherTok}[1]{\textcolor[rgb]{0.00,0.44,0.13}{{#1}}}
    \newcommand{\AlertTok}[1]{\textcolor[rgb]{1.00,0.00,0.00}{\textbf{{#1}}}}
    \newcommand{\FunctionTok}[1]{\textcolor[rgb]{0.02,0.16,0.49}{{#1}}}
    \newcommand{\RegionMarkerTok}[1]{{#1}}
    \newcommand{\ErrorTok}[1]{\textcolor[rgb]{1.00,0.00,0.00}{\textbf{{#1}}}}
    \newcommand{\NormalTok}[1]{{#1}}
    
    % Additional commands for more recent versions of Pandoc
    \newcommand{\ConstantTok}[1]{\textcolor[rgb]{0.53,0.00,0.00}{{#1}}}
    \newcommand{\SpecialCharTok}[1]{\textcolor[rgb]{0.25,0.44,0.63}{{#1}}}
    \newcommand{\VerbatimStringTok}[1]{\textcolor[rgb]{0.25,0.44,0.63}{{#1}}}
    \newcommand{\SpecialStringTok}[1]{\textcolor[rgb]{0.73,0.40,0.53}{{#1}}}
    \newcommand{\ImportTok}[1]{{#1}}
    \newcommand{\DocumentationTok}[1]{\textcolor[rgb]{0.73,0.13,0.13}{\textit{{#1}}}}
    \newcommand{\AnnotationTok}[1]{\textcolor[rgb]{0.38,0.63,0.69}{\textbf{\textit{{#1}}}}}
    \newcommand{\CommentVarTok}[1]{\textcolor[rgb]{0.38,0.63,0.69}{\textbf{\textit{{#1}}}}}
    \newcommand{\VariableTok}[1]{\textcolor[rgb]{0.10,0.09,0.49}{{#1}}}
    \newcommand{\ControlFlowTok}[1]{\textcolor[rgb]{0.00,0.44,0.13}{\textbf{{#1}}}}
    \newcommand{\OperatorTok}[1]{\textcolor[rgb]{0.40,0.40,0.40}{{#1}}}
    \newcommand{\BuiltInTok}[1]{{#1}}
    \newcommand{\ExtensionTok}[1]{{#1}}
    \newcommand{\PreprocessorTok}[1]{\textcolor[rgb]{0.74,0.48,0.00}{{#1}}}
    \newcommand{\AttributeTok}[1]{\textcolor[rgb]{0.49,0.56,0.16}{{#1}}}
    \newcommand{\InformationTok}[1]{\textcolor[rgb]{0.38,0.63,0.69}{\textbf{\textit{{#1}}}}}
    \newcommand{\WarningTok}[1]{\textcolor[rgb]{0.38,0.63,0.69}{\textbf{\textit{{#1}}}}}
    
    
    % Define a nice break command that doesn't care if a line doesn't already
    % exist.
    \def\br{\hspace*{\fill} \\* }
    % Math Jax compatibility definitions
    \def\gt{>}
    \def\lt{<}
    \let\Oldtex\TeX
    \let\Oldlatex\LaTeX
    \renewcommand{\TeX}{\textrm{\Oldtex}}
    \renewcommand{\LaTeX}{\textrm{\Oldlatex}}
    % Document parameters
    % Document title
    \title{
      {\Large Лекция 9} \\
      Выбор ковариационной функции и её параметров
    }
    % \date{20 апреля 2022\,г.}
    \date{}
    
    
    
% Pygments definitions
\makeatletter
\def\PY@reset{\let\PY@it=\relax \let\PY@bf=\relax%
    \let\PY@ul=\relax \let\PY@tc=\relax%
    \let\PY@bc=\relax \let\PY@ff=\relax}
\def\PY@tok#1{\csname PY@tok@#1\endcsname}
\def\PY@toks#1+{\ifx\relax#1\empty\else%
    \PY@tok{#1}\expandafter\PY@toks\fi}
\def\PY@do#1{\PY@bc{\PY@tc{\PY@ul{%
    \PY@it{\PY@bf{\PY@ff{#1}}}}}}}
\def\PY#1#2{\PY@reset\PY@toks#1+\relax+\PY@do{#2}}

\@namedef{PY@tok@w}{\def\PY@tc##1{\textcolor[rgb]{0.73,0.73,0.73}{##1}}}
\@namedef{PY@tok@c}{\let\PY@it=\textit\def\PY@tc##1{\textcolor[rgb]{0.24,0.48,0.48}{##1}}}
\@namedef{PY@tok@cp}{\def\PY@tc##1{\textcolor[rgb]{0.61,0.40,0.00}{##1}}}
\@namedef{PY@tok@k}{\let\PY@bf=\textbf\def\PY@tc##1{\textcolor[rgb]{0.00,0.50,0.00}{##1}}}
\@namedef{PY@tok@kp}{\def\PY@tc##1{\textcolor[rgb]{0.00,0.50,0.00}{##1}}}
\@namedef{PY@tok@kt}{\def\PY@tc##1{\textcolor[rgb]{0.69,0.00,0.25}{##1}}}
\@namedef{PY@tok@o}{\def\PY@tc##1{\textcolor[rgb]{0.40,0.40,0.40}{##1}}}
\@namedef{PY@tok@ow}{\let\PY@bf=\textbf\def\PY@tc##1{\textcolor[rgb]{0.67,0.13,1.00}{##1}}}
\@namedef{PY@tok@nb}{\def\PY@tc##1{\textcolor[rgb]{0.00,0.50,0.00}{##1}}}
\@namedef{PY@tok@nf}{\def\PY@tc##1{\textcolor[rgb]{0.00,0.00,1.00}{##1}}}
\@namedef{PY@tok@nc}{\let\PY@bf=\textbf\def\PY@tc##1{\textcolor[rgb]{0.00,0.00,1.00}{##1}}}
\@namedef{PY@tok@nn}{\let\PY@bf=\textbf\def\PY@tc##1{\textcolor[rgb]{0.00,0.00,1.00}{##1}}}
\@namedef{PY@tok@ne}{\let\PY@bf=\textbf\def\PY@tc##1{\textcolor[rgb]{0.80,0.25,0.22}{##1}}}
\@namedef{PY@tok@nv}{\def\PY@tc##1{\textcolor[rgb]{0.10,0.09,0.49}{##1}}}
\@namedef{PY@tok@no}{\def\PY@tc##1{\textcolor[rgb]{0.53,0.00,0.00}{##1}}}
\@namedef{PY@tok@nl}{\def\PY@tc##1{\textcolor[rgb]{0.46,0.46,0.00}{##1}}}
\@namedef{PY@tok@ni}{\let\PY@bf=\textbf\def\PY@tc##1{\textcolor[rgb]{0.44,0.44,0.44}{##1}}}
\@namedef{PY@tok@na}{\def\PY@tc##1{\textcolor[rgb]{0.41,0.47,0.13}{##1}}}
\@namedef{PY@tok@nt}{\let\PY@bf=\textbf\def\PY@tc##1{\textcolor[rgb]{0.00,0.50,0.00}{##1}}}
\@namedef{PY@tok@nd}{\def\PY@tc##1{\textcolor[rgb]{0.67,0.13,1.00}{##1}}}
\@namedef{PY@tok@s}{\def\PY@tc##1{\textcolor[rgb]{0.73,0.13,0.13}{##1}}}
\@namedef{PY@tok@sd}{\let\PY@it=\textit\def\PY@tc##1{\textcolor[rgb]{0.73,0.13,0.13}{##1}}}
\@namedef{PY@tok@si}{\let\PY@bf=\textbf\def\PY@tc##1{\textcolor[rgb]{0.64,0.35,0.47}{##1}}}
\@namedef{PY@tok@se}{\let\PY@bf=\textbf\def\PY@tc##1{\textcolor[rgb]{0.67,0.36,0.12}{##1}}}
\@namedef{PY@tok@sr}{\def\PY@tc##1{\textcolor[rgb]{0.64,0.35,0.47}{##1}}}
\@namedef{PY@tok@ss}{\def\PY@tc##1{\textcolor[rgb]{0.10,0.09,0.49}{##1}}}
\@namedef{PY@tok@sx}{\def\PY@tc##1{\textcolor[rgb]{0.00,0.50,0.00}{##1}}}
\@namedef{PY@tok@m}{\def\PY@tc##1{\textcolor[rgb]{0.40,0.40,0.40}{##1}}}
\@namedef{PY@tok@gh}{\let\PY@bf=\textbf\def\PY@tc##1{\textcolor[rgb]{0.00,0.00,0.50}{##1}}}
\@namedef{PY@tok@gu}{\let\PY@bf=\textbf\def\PY@tc##1{\textcolor[rgb]{0.50,0.00,0.50}{##1}}}
\@namedef{PY@tok@gd}{\def\PY@tc##1{\textcolor[rgb]{0.63,0.00,0.00}{##1}}}
\@namedef{PY@tok@gi}{\def\PY@tc##1{\textcolor[rgb]{0.00,0.52,0.00}{##1}}}
\@namedef{PY@tok@gr}{\def\PY@tc##1{\textcolor[rgb]{0.89,0.00,0.00}{##1}}}
\@namedef{PY@tok@ge}{\let\PY@it=\textit}
\@namedef{PY@tok@gs}{\let\PY@bf=\textbf}
\@namedef{PY@tok@gp}{\let\PY@bf=\textbf\def\PY@tc##1{\textcolor[rgb]{0.00,0.00,0.50}{##1}}}
\@namedef{PY@tok@go}{\def\PY@tc##1{\textcolor[rgb]{0.44,0.44,0.44}{##1}}}
\@namedef{PY@tok@gt}{\def\PY@tc##1{\textcolor[rgb]{0.00,0.27,0.87}{##1}}}
\@namedef{PY@tok@err}{\def\PY@bc##1{{\setlength{\fboxsep}{\string -\fboxrule}\fcolorbox[rgb]{1.00,0.00,0.00}{1,1,1}{\strut ##1}}}}
\@namedef{PY@tok@kc}{\let\PY@bf=\textbf\def\PY@tc##1{\textcolor[rgb]{0.00,0.50,0.00}{##1}}}
\@namedef{PY@tok@kd}{\let\PY@bf=\textbf\def\PY@tc##1{\textcolor[rgb]{0.00,0.50,0.00}{##1}}}
\@namedef{PY@tok@kn}{\let\PY@bf=\textbf\def\PY@tc##1{\textcolor[rgb]{0.00,0.50,0.00}{##1}}}
\@namedef{PY@tok@kr}{\let\PY@bf=\textbf\def\PY@tc##1{\textcolor[rgb]{0.00,0.50,0.00}{##1}}}
\@namedef{PY@tok@bp}{\def\PY@tc##1{\textcolor[rgb]{0.00,0.50,0.00}{##1}}}
\@namedef{PY@tok@fm}{\def\PY@tc##1{\textcolor[rgb]{0.00,0.00,1.00}{##1}}}
\@namedef{PY@tok@vc}{\def\PY@tc##1{\textcolor[rgb]{0.10,0.09,0.49}{##1}}}
\@namedef{PY@tok@vg}{\def\PY@tc##1{\textcolor[rgb]{0.10,0.09,0.49}{##1}}}
\@namedef{PY@tok@vi}{\def\PY@tc##1{\textcolor[rgb]{0.10,0.09,0.49}{##1}}}
\@namedef{PY@tok@vm}{\def\PY@tc##1{\textcolor[rgb]{0.10,0.09,0.49}{##1}}}
\@namedef{PY@tok@sa}{\def\PY@tc##1{\textcolor[rgb]{0.73,0.13,0.13}{##1}}}
\@namedef{PY@tok@sb}{\def\PY@tc##1{\textcolor[rgb]{0.73,0.13,0.13}{##1}}}
\@namedef{PY@tok@sc}{\def\PY@tc##1{\textcolor[rgb]{0.73,0.13,0.13}{##1}}}
\@namedef{PY@tok@dl}{\def\PY@tc##1{\textcolor[rgb]{0.73,0.13,0.13}{##1}}}
\@namedef{PY@tok@s2}{\def\PY@tc##1{\textcolor[rgb]{0.73,0.13,0.13}{##1}}}
\@namedef{PY@tok@sh}{\def\PY@tc##1{\textcolor[rgb]{0.73,0.13,0.13}{##1}}}
\@namedef{PY@tok@s1}{\def\PY@tc##1{\textcolor[rgb]{0.73,0.13,0.13}{##1}}}
\@namedef{PY@tok@mb}{\def\PY@tc##1{\textcolor[rgb]{0.40,0.40,0.40}{##1}}}
\@namedef{PY@tok@mf}{\def\PY@tc##1{\textcolor[rgb]{0.40,0.40,0.40}{##1}}}
\@namedef{PY@tok@mh}{\def\PY@tc##1{\textcolor[rgb]{0.40,0.40,0.40}{##1}}}
\@namedef{PY@tok@mi}{\def\PY@tc##1{\textcolor[rgb]{0.40,0.40,0.40}{##1}}}
\@namedef{PY@tok@il}{\def\PY@tc##1{\textcolor[rgb]{0.40,0.40,0.40}{##1}}}
\@namedef{PY@tok@mo}{\def\PY@tc##1{\textcolor[rgb]{0.40,0.40,0.40}{##1}}}
\@namedef{PY@tok@ch}{\let\PY@it=\textit\def\PY@tc##1{\textcolor[rgb]{0.24,0.48,0.48}{##1}}}
\@namedef{PY@tok@cm}{\let\PY@it=\textit\def\PY@tc##1{\textcolor[rgb]{0.24,0.48,0.48}{##1}}}
\@namedef{PY@tok@cpf}{\let\PY@it=\textit\def\PY@tc##1{\textcolor[rgb]{0.24,0.48,0.48}{##1}}}
\@namedef{PY@tok@c1}{\let\PY@it=\textit\def\PY@tc##1{\textcolor[rgb]{0.24,0.48,0.48}{##1}}}
\@namedef{PY@tok@cs}{\let\PY@it=\textit\def\PY@tc##1{\textcolor[rgb]{0.24,0.48,0.48}{##1}}}

\def\PYZbs{\char`\\}
\def\PYZus{\char`\_}
\def\PYZob{\char`\{}
\def\PYZcb{\char`\}}
\def\PYZca{\char`\^}
\def\PYZam{\char`\&}
\def\PYZlt{\char`\<}
\def\PYZgt{\char`\>}
\def\PYZsh{\char`\#}
\def\PYZpc{\char`\%}
\def\PYZdl{\char`\$}
\def\PYZhy{\char`\-}
\def\PYZsq{\char`\'}
\def\PYZdq{\char`\"}
\def\PYZti{\char`\~}
% for compatibility with earlier versions
\def\PYZat{@}
\def\PYZlb{[}
\def\PYZrb{]}
\makeatother


    % For linebreaks inside Verbatim environment from package fancyvrb. 
    \makeatletter
        \newbox\Wrappedcontinuationbox 
        \newbox\Wrappedvisiblespacebox 
        \newcommand*\Wrappedvisiblespace {\textcolor{red}{\textvisiblespace}} 
        \newcommand*\Wrappedcontinuationsymbol {\textcolor{red}{\llap{\tiny$\m@th\hookrightarrow$}}} 
        \newcommand*\Wrappedcontinuationindent {3ex } 
        \newcommand*\Wrappedafterbreak {\kern\Wrappedcontinuationindent\copy\Wrappedcontinuationbox} 
        % Take advantage of the already applied Pygments mark-up to insert 
        % potential linebreaks for TeX processing. 
        %        {, <, #, %, $, ' and ": go to next line. 
        %        _, }, ^, &, >, - and ~: stay at end of broken line. 
        % Use of \textquotesingle for straight quote. 
        \newcommand*\Wrappedbreaksatspecials {% 
            \def\PYGZus{\discretionary{\char`\_}{\Wrappedafterbreak}{\char`\_}}% 
            \def\PYGZob{\discretionary{}{\Wrappedafterbreak\char`\{}{\char`\{}}% 
            \def\PYGZcb{\discretionary{\char`\}}{\Wrappedafterbreak}{\char`\}}}% 
            \def\PYGZca{\discretionary{\char`\^}{\Wrappedafterbreak}{\char`\^}}% 
            \def\PYGZam{\discretionary{\char`\&}{\Wrappedafterbreak}{\char`\&}}% 
            \def\PYGZlt{\discretionary{}{\Wrappedafterbreak\char`\<}{\char`\<}}% 
            \def\PYGZgt{\discretionary{\char`\>}{\Wrappedafterbreak}{\char`\>}}% 
            \def\PYGZsh{\discretionary{}{\Wrappedafterbreak\char`\#}{\char`\#}}% 
            \def\PYGZpc{\discretionary{}{\Wrappedafterbreak\char`\%}{\char`\%}}% 
            \def\PYGZdl{\discretionary{}{\Wrappedafterbreak\char`\$}{\char`\$}}% 
            \def\PYGZhy{\discretionary{\char`\-}{\Wrappedafterbreak}{\char`\-}}% 
            \def\PYGZsq{\discretionary{}{\Wrappedafterbreak\textquotesingle}{\textquotesingle}}% 
            \def\PYGZdq{\discretionary{}{\Wrappedafterbreak\char`\"}{\char`\"}}% 
            \def\PYGZti{\discretionary{\char`\~}{\Wrappedafterbreak}{\char`\~}}% 
        } 
        % Some characters . , ; ? ! / are not pygmentized. 
        % This macro makes them "active" and they will insert potential linebreaks 
        \newcommand*\Wrappedbreaksatpunct {% 
            \lccode`\~`\.\lowercase{\def~}{\discretionary{\hbox{\char`\.}}{\Wrappedafterbreak}{\hbox{\char`\.}}}% 
            \lccode`\~`\,\lowercase{\def~}{\discretionary{\hbox{\char`\,}}{\Wrappedafterbreak}{\hbox{\char`\,}}}% 
            \lccode`\~`\;\lowercase{\def~}{\discretionary{\hbox{\char`\;}}{\Wrappedafterbreak}{\hbox{\char`\;}}}% 
            \lccode`\~`\:\lowercase{\def~}{\discretionary{\hbox{\char`\:}}{\Wrappedafterbreak}{\hbox{\char`\:}}}% 
            \lccode`\~`\?\lowercase{\def~}{\discretionary{\hbox{\char`\?}}{\Wrappedafterbreak}{\hbox{\char`\?}}}% 
            \lccode`\~`\!\lowercase{\def~}{\discretionary{\hbox{\char`\!}}{\Wrappedafterbreak}{\hbox{\char`\!}}}% 
            \lccode`\~`\/\lowercase{\def~}{\discretionary{\hbox{\char`\/}}{\Wrappedafterbreak}{\hbox{\char`\/}}}% 
            \catcode`\.\active
            \catcode`\,\active 
            \catcode`\;\active
            \catcode`\:\active
            \catcode`\?\active
            \catcode`\!\active
            \catcode`\/\active 
            \lccode`\~`\~ 	
        }
    \makeatother

    \let\OriginalVerbatim=\Verbatim
    \makeatletter
    \renewcommand{\Verbatim}[1][1]{%
        %\parskip\z@skip
        \sbox\Wrappedcontinuationbox {\Wrappedcontinuationsymbol}%
        \sbox\Wrappedvisiblespacebox {\FV@SetupFont\Wrappedvisiblespace}%
        \def\FancyVerbFormatLine ##1{\hsize\linewidth
            \vtop{\raggedright\hyphenpenalty\z@\exhyphenpenalty\z@
                \doublehyphendemerits\z@\finalhyphendemerits\z@
                \strut ##1\strut}%
        }%
        % If the linebreak is at a space, the latter will be displayed as visible
        % space at end of first line, and a continuation symbol starts next line.
        % Stretch/shrink are however usually zero for typewriter font.
        \def\FV@Space {%
            \nobreak\hskip\z@ plus\fontdimen3\font minus\fontdimen4\font
            \discretionary{\copy\Wrappedvisiblespacebox}{\Wrappedafterbreak}
            {\kern\fontdimen2\font}%
        }%
        
        % Allow breaks at special characters using \PYG... macros.
        \Wrappedbreaksatspecials
        % Breaks at punctuation characters . , ; ? ! and / need catcode=\active 	
        \OriginalVerbatim[#1,codes*=\Wrappedbreaksatpunct]%
    }
    \makeatother

    % Exact colors from NB
    \definecolor{incolor}{HTML}{303F9F}
    \definecolor{outcolor}{HTML}{D84315}
    \definecolor{cellborder}{HTML}{CFCFCF}
    \definecolor{cellbackground}{HTML}{F7F7F7}
    
    % prompt
    \makeatletter
    \newcommand{\boxspacing}{\kern\kvtcb@left@rule\kern\kvtcb@boxsep}
    \makeatother
    \newcommand{\prompt}[4]{
        {\ttfamily\llap{{\color{#2}[#3]:\hspace{3pt}#4}}\vspace{-\baselineskip}}
    }
    

    
    % Prevent overflowing lines due to hard-to-break entities
    \sloppy 
    % Setup hyperref package
    \hypersetup{
      breaklinks=true,  % so long urls are correctly broken across lines
      colorlinks=true,
      urlcolor=urlcolor,
      linkcolor=linkcolor,
      citecolor=citecolor,
      }
    % Slightly bigger margins than the latex defaults
    
    \geometry{verbose,tmargin=1in,bmargin=1in,lmargin=1in,rmargin=1in}
    
    

\begin{document}

  \maketitle
  \thispagestyle{empty}
  \tableofcontents

%\let\thefootnote\relax\footnote{
%  \textit{День 20 апреля в истории:
%    \begin{itemize}[topsep=2pt,itemsep=1pt]
%      \item 1832 г. --- петербургский акушер Андрей Вольф сделал первое в России успешное переливание крови;
%      \item 1841 г. --- публикация первого в истории литературы детективного рассказа --- <<Убийство на улице Морг>> американского писателя Эдгара Алана По;
%      \item 1902 г. --- супруги Мария и Пьер Кюри получили чистый радий;
%      \item 1934 г. --- первое присвоение звания Герой Советского Союза семерым лётчикам (М.\,В. Водопьянов, И.\,В. Доронин, Н.\,П. Каманин, С.\,А. Леваневский, А.\,В. Ляпидевский, В.\,С. Молоков, М.\,Т. Слепнёв), спасшим экипаж парохода <<Челюскин>>;
%      \item 1972 г. --- прилунился пилотируемый космический корабль <<Аполлон-16>> (21 апреля --- пятая высадка людей на Луну).
%    \end{itemize}
%  }
%}

  \newpage


%    \begin{tcolorbox}[breakable, size=fbox, boxrule=1pt, pad at break*=1mm,colback=cellbackground, colframe=cellborder]
%\prompt{In}{incolor}{1}{\boxspacing}
%\begin{Verbatim}[commandchars=\\\{\}]
%\PY{c+c1}{\PYZsh{} Imports}
%\PY{k+kn}{import} \PY{n+nn}{numpy} \PY{k}{as} \PY{n+nn}{np}
%\PY{k+kn}{import} \PY{n+nn}{numpy}\PY{n+nn}{.}\PY{n+nn}{linalg} \PY{k}{as} \PY{n+nn}{LA}
%\PY{n}{np}\PY{o}{.}\PY{n}{random}\PY{o}{.}\PY{n}{seed}\PY{p}{(}\PY{l+m+mi}{42}\PY{p}{)}
%\PY{k+kn}{from} \PY{n+nn}{scipy}\PY{n+nn}{.}\PY{n+nn}{optimize} \PY{k+kn}{import} \PY{n}{minimize}
%
%\PY{k+kn}{import} \PY{n+nn}{sys}
%\PY{n}{sys}\PY{o}{.}\PY{n}{path}\PY{o}{.}\PY{n}{append}\PY{p}{(}\PY{l+s+s1}{\PYZsq{}}\PY{l+s+s1}{./scripts}\PY{l+s+s1}{\PYZsq{}}\PY{p}{)}
%\PY{k+kn}{import} \PY{n+nn}{GP\PYZus{}kernels}
%\PY{k+kn}{from} \PY{n+nn}{GP\PYZus{}utils} \PY{k+kn}{import} \PY{n}{plot\PYZus{}GP}\PY{p}{,} \PY{n}{GP\PYZus{}predictor}
%\end{Verbatim}
%\end{tcolorbox}
%
%    \begin{tcolorbox}[breakable, size=fbox, boxrule=1pt, pad at break*=1mm,colback=cellbackground, colframe=cellborder]
%\prompt{In}{incolor}{2}{\boxspacing}
%\begin{Verbatim}[commandchars=\\\{\}]
%\PY{c+c1}{\PYZsh{} Styles, fonts}
%\PY{k+kn}{import} \PY{n+nn}{matplotlib}
%\PY{n}{matplotlib}\PY{o}{.}\PY{n}{rcParams}\PY{p}{[}\PY{l+s+s1}{\PYZsq{}}\PY{l+s+s1}{font.size}\PY{l+s+s1}{\PYZsq{}}\PY{p}{]} \PY{o}{=} \PY{l+m+mi}{12}
%\PY{k+kn}{import} \PY{n+nn}{matplotlib}\PY{n+nn}{.}\PY{n+nn}{pyplot} \PY{k}{as} \PY{n+nn}{plt}
%\PY{k+kn}{import} \PY{n+nn}{matplotlib}\PY{n+nn}{.}\PY{n+nn}{gridspec} \PY{k}{as} \PY{n+nn}{gridspec}
%\PY{k+kn}{from} \PY{n+nn}{matplotlib} \PY{k+kn}{import} \PY{n}{cm} \PY{c+c1}{\PYZsh{} Colormaps}
%
%\PY{k+kn}{import} \PY{n+nn}{seaborn}
%\PY{n}{seaborn}\PY{o}{.}\PY{n}{set\PYZus{}style}\PY{p}{(}\PY{l+s+s1}{\PYZsq{}}\PY{l+s+s1}{whitegrid}\PY{l+s+s1}{\PYZsq{}}\PY{p}{)}
%
%\PY{k+kn}{from} \PY{n+nn}{IPython}\PY{n+nn}{.}\PY{n+nn}{display} \PY{k+kn}{import} \PY{n}{display}\PY{p}{,} \PY{n}{Markdown}
%\end{Verbatim}
%\end{tcolorbox}
%
%    \begin{tcolorbox}[breakable, size=fbox, boxrule=1pt, pad at break*=1mm,colback=cellbackground, colframe=cellborder]
%\prompt{In}{incolor}{3}{\boxspacing}
%\begin{Verbatim}[commandchars=\\\{\}]
%\PY{c+c1}{\PYZsh{} \PYZpc{}config InlineBackend.figure\PYZus{}formats = [\PYZsq{}pdf\PYZsq{}]}
%\PY{c+c1}{\PYZsh{} \PYZpc{}config Completer.use\PYZus{}jedi = False}
%\end{Verbatim}
%\end{tcolorbox}
%
%    \begin{center}\rule{0.5\linewidth}{0.5pt}\end{center}

    На прошлом занятии мы научились делать регрессию, используя гауссовские
процессы с заданной фиксированной ковариационной функцией. Однако во
многих практических приложениях указать ковариационную функцию может
быть затруднительно. Обычно мы имеем достаточно расплывчатую информацию
о свойствах, например, о значении свободных (гипер-) параметров,
например, о длинах. Таким образом, необходимо разработать методы,
решающие проблему выбора модели и значений гиперпараметров.

    \hypertarget{ux432ux43bux438ux44fux43dux438ux435-ux432ux438ux434ux430-ux43aux43eux432ux430ux440ux438ux430ux446ux438ux43eux43dux43dux43eux439-ux444ux443ux43dux43aux446ux438ux438}{%
\section{Влияние вида ковариационной
функции}\label{ux432ux43bux438ux44fux43dux438ux435-ux432ux438ux434ux430-ux43aux43eux432ux430ux440ux438ux430ux446ux438ux43eux43dux43dux43eux439-ux444ux443ux43dux43aux446ux438ux438}}

    \hypertarget{ux438ux441ux43fux43eux43bux44cux437ux443ux435ux43cux44bux435-ux43aux43eux432ux430ux440ux438ux430ux446ux438ux43eux43dux43dux44bux439-ux444ux443ux43dux43aux446ux438ux438}{%
\subsection{Используемые ковариационный
функции}\label{ux438ux441ux43fux43eux43bux44cux437ux443ux435ux43cux44bux435-ux43aux43eux432ux430ux440ux438ux430ux446ux438ux43eux43dux43dux44bux439-ux444ux443ux43dux43aux446ux438ux438}}

Рассмотрим следующие ковариационные функции:

\begin{enumerate}
\def\labelenumi{\arabic{enumi}.}
\tightlist
\item
  Функция Гаусса (квадратичная экспоненциальная функция) \[
    k(x, x') = \sigma_k^2 \exp \left( -\frac{\lVert x - x' \rVert^2}{2\ell^2}  \right)
  \]
\item
  Рациональная квадратичная функция \[
    k(x, x') = \sigma_k^2 \left( 1 + \frac{ \left\Vert x - x' \right\Vert^2}{2 \alpha \ell^2} \right)^{-\alpha}
  \]
\item
  Периодическая функция \[
    k(x, x') = \sigma_k^2 \exp \left(-\frac{2}{\ell^2}\sin^2 \left( \pi \frac{\left\Vert x - x' \right\Vert}{p}\right) \right)
  \]
\end{enumerate}

%    \begin{tcolorbox}[breakable, size=fbox, boxrule=1pt, pad at break*=1mm,colback=cellbackground, colframe=cellborder]
%\prompt{In}{incolor}{4}{\boxspacing}
%\begin{Verbatim}[commandchars=\\\{\}]
%\PY{n}{xlim} \PY{o}{=} \PY{p}{(}\PY{o}{\PYZhy{}}\PY{l+m+mi}{4}\PY{p}{,} \PY{l+m+mi}{4}\PY{p}{)}
%\PY{n}{X} \PY{o}{=} \PY{n}{np}\PY{o}{.}\PY{n}{expand\PYZus{}dims}\PY{p}{(}\PY{n}{np}\PY{o}{.}\PY{n}{linspace}\PY{p}{(}\PY{o}{*}\PY{n}{xlim}\PY{p}{,} \PY{n}{num}\PY{o}{=}\PY{l+m+mi}{100}\PY{p}{)}\PY{p}{,} \PY{l+m+mi}{1}\PY{p}{)}
%\PY{n}{zero} \PY{o}{=} \PY{n}{np}\PY{o}{.}\PY{n}{array}\PY{p}{(}\PY{p}{[}\PY{p}{[}\PY{l+m+mf}{0.}\PY{p}{]}\PY{p}{]}\PY{p}{)}
%\PY{n}{kernels} \PY{o}{=} \PY{p}{[}\PY{n}{GP\PYZus{}kernels}\PY{o}{.}\PY{n}{gauss}\PY{p}{,} \PY{n}{GP\PYZus{}kernels}\PY{o}{.}\PY{n}{rational\PYZus{}quadratic}\PY{p}{]}
%\PY{n}{kernel\PYZus{}args} \PY{o}{=} \PY{p}{[}
%    \PY{p}{(}\PY{l+m+mf}{1.0}\PY{p}{,} \PY{l+m+mf}{1.0}\PY{p}{,} \PY{l+m+mf}{1e6}\PY{p}{)}\PY{p}{,}
%    \PY{p}{(}\PY{l+m+mf}{0.5}\PY{p}{,} \PY{l+m+mf}{1.0}\PY{p}{,} \PY{l+m+mf}{1.0}\PY{p}{)}\PY{p}{,}
%    \PY{p}{(}\PY{l+m+mf}{1.0}\PY{p}{,} \PY{l+m+mf}{0.5}\PY{p}{,} \PY{l+m+mf}{0.1}\PY{p}{)}\PY{p}{,}
%\PY{p}{]}
%\end{Verbatim}
%\end{tcolorbox}
%
%    \begin{tcolorbox}[breakable, size=fbox, boxrule=1pt, pad at break*=1mm,colback=cellbackground, colframe=cellborder]
%\prompt{In}{incolor}{5}{\boxspacing}
%\begin{Verbatim}[commandchars=\\\{\}]
%\PY{c+c1}{\PYZsh{} Plot exponentiated quadratic distance}
%\PY{n}{fig}\PY{p}{,} \PY{n}{axes} \PY{o}{=} \PY{n}{plt}\PY{o}{.}\PY{n}{subplots}\PY{p}{(}\PY{l+m+mi}{1}\PY{p}{,} \PY{l+m+mi}{2}\PY{p}{,} \PY{n}{figsize}\PY{o}{=}\PY{p}{(}\PY{l+m+mi}{12}\PY{p}{,} \PY{l+m+mi}{5}\PY{p}{)}\PY{p}{)}
%
%\PY{k}{for} \PY{n}{i}\PY{p}{,} \PY{n}{kernel} \PY{o+ow}{in} \PY{n+nb}{enumerate}\PY{p}{(}\PY{n}{kernels}\PY{p}{)}\PY{p}{:}
%    \PY{k}{for} \PY{p}{(}\PY{n}{l}\PY{p}{,} \PY{n}{sigma\PYZus{}k}\PY{p}{,} \PY{n}{alpha}\PY{p}{)} \PY{o+ow}{in} \PY{n}{kernel\PYZus{}args}\PY{p}{:}
%        \PY{k}{if} \PY{l+s+s1}{\PYZsq{}}\PY{l+s+s1}{gauss}\PY{l+s+s1}{\PYZsq{}} \PY{o+ow}{in} \PY{n+nb}{str}\PY{p}{(}\PY{n}{kernel}\PY{p}{)}\PY{p}{:}
%            \PY{n}{axes}\PY{p}{[}\PY{n}{i}\PY{p}{]}\PY{o}{.}\PY{n}{set\PYZus{}title}\PY{p}{(}\PY{l+s+s1}{\PYZsq{}}\PY{l+s+s1}{Exponentiated quadratic kernel}\PY{l+s+s1}{\PYZsq{}}\PY{p}{)}
%            \PY{n}{kernel\PYZus{}arg} \PY{o}{=} \PY{p}{\PYZob{}}\PY{l+s+s1}{\PYZsq{}}\PY{l+s+s1}{l}\PY{l+s+s1}{\PYZsq{}}\PY{p}{:}\PY{n}{l}\PY{p}{,} \PY{l+s+s1}{\PYZsq{}}\PY{l+s+s1}{sigma\PYZus{}k}\PY{l+s+s1}{\PYZsq{}}\PY{p}{:}\PY{n}{sigma\PYZus{}k}\PY{p}{\PYZcb{}}
%            \PY{n}{label} \PY{o}{=} \PY{l+s+sa}{f}\PY{l+s+s1}{\PYZsq{}}\PY{l+s+s1}{\PYZdl{}}\PY{l+s+se}{\PYZbs{}\PYZbs{}}\PY{l+s+s1}{ell = }\PY{l+s+si}{\PYZob{}}\PY{n}{l}\PY{l+s+si}{\PYZcb{}}\PY{l+s+s1}{\PYZdl{}, \PYZdl{}}\PY{l+s+se}{\PYZbs{}\PYZbs{}}\PY{l+s+s1}{sigma\PYZus{}k = }\PY{l+s+si}{\PYZob{}}\PY{n}{sigma\PYZus{}k}\PY{l+s+si}{\PYZcb{}}\PY{l+s+s1}{\PYZdl{}}\PY{l+s+s1}{\PYZsq{}}
%        \PY{k}{else}\PY{p}{:}
%            \PY{n}{axes}\PY{p}{[}\PY{n}{i}\PY{p}{]}\PY{o}{.}\PY{n}{set\PYZus{}title}\PY{p}{(}\PY{l+s+s1}{\PYZsq{}}\PY{l+s+s1}{Rational quadratic kernel}\PY{l+s+s1}{\PYZsq{}}\PY{p}{)}
%            \PY{n}{kernel\PYZus{}arg} \PY{o}{=} \PY{p}{\PYZob{}}\PY{l+s+s1}{\PYZsq{}}\PY{l+s+s1}{alpha}\PY{l+s+s1}{\PYZsq{}}\PY{p}{:}\PY{n}{alpha}\PY{p}{\PYZcb{}}
%            \PY{n}{label} \PY{o}{=} \PY{l+s+sa}{f}\PY{l+s+s1}{\PYZsq{}}\PY{l+s+s1}{\PYZdl{}}\PY{l+s+se}{\PYZbs{}\PYZbs{}}\PY{l+s+s1}{alpha = }\PY{l+s+si}{\PYZob{}}\PY{n}{alpha}\PY{l+s+si}{:}\PY{l+s+s1}{g}\PY{l+s+si}{\PYZcb{}}\PY{l+s+s1}{\PYZdl{}}\PY{l+s+s1}{\PYZsq{}}
%        \PY{n}{K} \PY{o}{=} \PY{n}{kernel}\PY{p}{(}\PY{n}{zero}\PY{p}{,} \PY{n}{X}\PY{p}{,} \PY{n}{kernel\PYZus{}arg}\PY{p}{)}
%        \PY{n}{axes}\PY{p}{[}\PY{n}{i}\PY{p}{]}\PY{o}{.}\PY{n}{plot}\PY{p}{(}\PY{n}{X}\PY{p}{[}\PY{p}{:}\PY{p}{,}\PY{l+m+mi}{0}\PY{p}{]}\PY{p}{,} \PY{n}{K}\PY{p}{[}\PY{l+m+mi}{0}\PY{p}{,}\PY{p}{:}\PY{p}{]}\PY{p}{,} \PY{n}{label}\PY{o}{=}\PY{n}{label}\PY{p}{)}
%
%    \PY{n}{axes}\PY{p}{[}\PY{n}{i}\PY{p}{]}\PY{o}{.}\PY{n}{set\PYZus{}xlabel}\PY{p}{(}\PY{l+s+s1}{\PYZsq{}}\PY{l+s+s1}{\PYZdl{}x \PYZhy{} x}\PY{l+s+se}{\PYZbs{}\PYZsq{}}\PY{l+s+s1}{\PYZdl{}}\PY{l+s+s1}{\PYZsq{}}\PY{p}{)}
%    \PY{n}{axes}\PY{p}{[}\PY{n}{i}\PY{p}{]}\PY{o}{.}\PY{n}{set\PYZus{}ylabel}\PY{p}{(}\PY{l+s+s1}{\PYZsq{}}\PY{l+s+s1}{\PYZdl{}K(x,x}\PY{l+s+se}{\PYZbs{}\PYZsq{}}\PY{l+s+s1}{)\PYZdl{}}\PY{l+s+s1}{\PYZsq{}}\PY{p}{)}
%    \PY{n}{axes}\PY{p}{[}\PY{n}{i}\PY{p}{]}\PY{o}{.}\PY{n}{set\PYZus{}ylim}\PY{p}{(}\PY{p}{[}\PY{l+m+mi}{0}\PY{p}{,} \PY{l+m+mf}{1.1}\PY{p}{]}\PY{p}{)}
%    \PY{n}{axes}\PY{p}{[}\PY{n}{i}\PY{p}{]}\PY{o}{.}\PY{n}{set\PYZus{}xlim}\PY{p}{(}\PY{o}{*}\PY{n}{xlim}\PY{p}{)}
%    \PY{n}{axes}\PY{p}{[}\PY{n}{i}\PY{p}{]}\PY{o}{.}\PY{n}{legend}\PY{p}{(}\PY{n}{loc}\PY{o}{=}\PY{l+m+mi}{1}\PY{p}{)}
%
%\PY{n}{plt}\PY{o}{.}\PY{n}{tight\PYZus{}layout}\PY{p}{(}\PY{p}{)}
%\PY{n}{plt}\PY{o}{.}\PY{n}{show}\PY{p}{(}\PY{p}{)}
%\end{Verbatim}
%\end{tcolorbox}

    \begin{center}
    \adjustimage{max size={0.9\linewidth}{0.9\paperheight}}{Kernels_EQ_vs_RQ.pdf}
    \end{center}
%    { \hspace*{\fill} \\}

    \hypertarget{ux43eux431ux443ux447ux430ux44eux449ux438ux435-ux434ux430ux43dux43dux44bux435}{%
\subsection{Обучающие
данные}\label{ux43eux431ux443ux447ux430ux44eux449ux438ux435-ux434ux430ux43dux43dux44bux435}}

В качестве обучающей выборки будем использовать данные из предыдущего
занятия.

%    \begin{tcolorbox}[breakable, size=fbox, boxrule=1pt, pad at break*=1mm,colback=cellbackground, colframe=cellborder]
%\prompt{In}{incolor}{6}{\boxspacing}
%\begin{Verbatim}[commandchars=\\\{\}]
%\PY{c+c1}{\PYZsh{} Data}
%\PY{n}{xlim} \PY{o}{=} \PY{p}{[}\PY{l+m+mf}{0.}\PY{p}{,} \PY{l+m+mf}{10.}\PY{p}{]}
%\PY{n}{N\PYZus{}test} \PY{o}{=} \PY{l+m+mi}{501}
%\PY{n}{X\PYZus{}test} \PY{o}{=} \PY{n}{np}\PY{o}{.}\PY{n}{linspace}\PY{p}{(}\PY{o}{*}\PY{n}{xlim}\PY{p}{,} \PY{n}{N\PYZus{}test}\PY{p}{)}\PY{o}{.}\PY{n}{reshape}\PY{p}{(}\PY{o}{\PYZhy{}}\PY{l+m+mi}{1}\PY{p}{,} \PY{l+m+mi}{1}\PY{p}{)}
%
%\PY{n}{X\PYZus{}train} \PY{o}{=} \PY{n}{np}\PY{o}{.}\PY{n}{array}\PY{p}{(}\PY{p}{[}\PY{l+m+mf}{2.}\PY{p}{,}  \PY{l+m+mf}{6.}\PY{p}{,}  \PY{l+m+mf}{7.}\PY{p}{,} \PY{l+m+mf}{8.}\PY{p}{,}  \PY{l+m+mf}{4.}\PY{p}{,} \PY{l+m+mf}{3.} \PY{p}{]}\PY{p}{)}\PY{o}{.}\PY{n}{reshape}\PY{p}{(}\PY{o}{\PYZhy{}}\PY{l+m+mi}{1}\PY{p}{,} \PY{l+m+mi}{1}\PY{p}{)}
%\PY{n}{Y\PYZus{}train} \PY{o}{=} \PY{n}{np}\PY{o}{.}\PY{n}{array}\PY{p}{(}\PY{p}{[}\PY{l+m+mf}{1.}\PY{p}{,} \PY{o}{\PYZhy{}}\PY{l+m+mf}{1.}\PY{p}{,} \PY{o}{\PYZhy{}}\PY{l+m+mf}{1.}\PY{p}{,} \PY{l+m+mf}{0.5}\PY{p}{,} \PY{l+m+mf}{1.}\PY{p}{,} \PY{l+m+mf}{0.5}\PY{p}{]}\PY{p}{)}\PY{o}{.}\PY{n}{reshape}\PY{p}{(}\PY{o}{\PYZhy{}}\PY{l+m+mi}{1}\PY{p}{,} \PY{l+m+mi}{1}\PY{p}{)}
%\end{Verbatim}
%\end{tcolorbox}

%    \begin{tcolorbox}[breakable, size=fbox, boxrule=1pt, pad at break*=1mm,colback=cellbackground, colframe=cellborder]
%\prompt{In}{incolor}{7}{\boxspacing}
%\begin{Verbatim}[commandchars=\\\{\}]
%\PY{k}{def} \PY{n+nf}{plot\PYZus{}regression}\PY{p}{(}\PY{n}{ax}\PY{p}{,} \PY{n}{X\PYZus{}test}\PY{p}{,} \PY{n}{mu}\PY{p}{,} \PY{n}{cov}\PY{p}{,} \PY{n}{X\PYZus{}train}\PY{o}{=}\PY{k+kc}{None}\PY{p}{,} \PY{n}{Y\PYZus{}train}\PY{o}{=}\PY{k+kc}{None}\PY{p}{)}\PY{p}{:}
%    \PY{l+s+sd}{\PYZsq{}\PYZsq{}\PYZsq{}Plot samples\PYZsq{}\PYZsq{}\PYZsq{}}
%    \PY{n}{X\PYZus{}test} \PY{o}{=} \PY{n}{X\PYZus{}test}\PY{o}{.}\PY{n}{flatten}\PY{p}{(}\PY{p}{)}
%    \PY{n}{mu} \PY{o}{=} \PY{n}{mu}\PY{o}{.}\PY{n}{flatten}\PY{p}{(}\PY{p}{)}
%    \PY{n}{std} \PY{o}{=} \PY{n}{np}\PY{o}{.}\PY{n}{sqrt}\PY{p}{(}\PY{n}{np}\PY{o}{.}\PY{n}{diag}\PY{p}{(}\PY{n}{cov}\PY{p}{)}\PY{p}{)}
%    
%    \PY{k}{for} \PY{n}{std\PYZus{}i} \PY{o+ow}{in} \PY{n}{np}\PY{o}{.}\PY{n}{linspace}\PY{p}{(}\PY{l+m+mi}{2}\PY{o}{*}\PY{n}{std}\PY{p}{,}\PY{l+m+mi}{0}\PY{p}{,}\PY{l+m+mi}{21}\PY{p}{)}\PY{p}{:}
%        \PY{n}{ax}\PY{o}{.}\PY{n}{fill\PYZus{}between}\PY{p}{(}\PY{n}{X\PYZus{}test}\PY{p}{,} \PY{n}{mu}\PY{o}{\PYZhy{}}\PY{n}{std\PYZus{}i}\PY{p}{,} \PY{n}{mu}\PY{o}{+}\PY{n}{std\PYZus{}i}\PY{p}{,}
%                        \PY{n}{color}\PY{o}{=}\PY{n}{cm}\PY{o}{.}\PY{n}{tab10}\PY{p}{(}\PY{l+m+mi}{4}\PY{p}{)}\PY{p}{,} \PY{n}{alpha}\PY{o}{=}\PY{l+m+mf}{0.02}\PY{p}{)}
%    \PY{n}{ax}\PY{o}{.}\PY{n}{plot}\PY{p}{(}\PY{n}{X\PYZus{}test}\PY{p}{,} \PY{n}{mu}\PY{p}{,} \PY{l+s+s1}{\PYZsq{}}\PY{l+s+s1}{k}\PY{l+s+s1}{\PYZsq{}}\PY{p}{)}
%    \PY{k}{if} \PY{n}{X\PYZus{}train} \PY{o+ow}{is} \PY{o+ow}{not} \PY{k+kc}{None}\PY{p}{:}
%        \PY{n}{ax}\PY{o}{.}\PY{n}{plot}\PY{p}{(}\PY{n}{X\PYZus{}train}\PY{p}{,} \PY{n}{Y\PYZus{}train}\PY{p}{,} \PY{l+s+s1}{\PYZsq{}}\PY{l+s+s1}{kx}\PY{l+s+s1}{\PYZsq{}}\PY{p}{,} \PY{n}{mew}\PY{o}{=}\PY{l+m+mf}{1.0}\PY{p}{)}
%        
%    \PY{n}{ax}\PY{o}{.}\PY{n}{set\PYZus{}xlim}\PY{p}{(}\PY{p}{[}\PY{n}{X\PYZus{}test}\PY{o}{.}\PY{n}{min}\PY{p}{(}\PY{p}{)}\PY{p}{,} \PY{n}{X\PYZus{}test}\PY{o}{.}\PY{n}{max}\PY{p}{(}\PY{p}{)}\PY{p}{]}\PY{p}{)}
%    \PY{n}{ax}\PY{o}{.}\PY{n}{set\PYZus{}ylim}\PY{p}{(}\PY{p}{[}\PY{p}{(}\PY{n}{mu}\PY{o}{\PYZhy{}}\PY{l+m+mi}{3}\PY{o}{*}\PY{n}{std}\PY{p}{)}\PY{o}{.}\PY{n}{min}\PY{p}{(}\PY{p}{)}\PY{p}{,} \PY{p}{(}\PY{n}{mu}\PY{o}{+}\PY{l+m+mi}{3}\PY{o}{*}\PY{n}{std}\PY{p}{)}\PY{o}{.}\PY{n}{max}\PY{p}{(}\PY{p}{)}\PY{p}{]} \PY{p}{)}
%    \PY{n}{ax}\PY{o}{.}\PY{n}{set\PYZus{}xlabel}\PY{p}{(}\PY{l+s+s1}{\PYZsq{}}\PY{l+s+s1}{\PYZdl{}x\PYZdl{}}\PY{l+s+s1}{\PYZsq{}}\PY{p}{)}
%    \PY{n}{ax}\PY{o}{.}\PY{n}{set\PYZus{}ylabel}\PY{p}{(}\PY{l+s+s1}{\PYZsq{}}\PY{l+s+s1}{\PYZdl{}y\PYZdl{}}\PY{l+s+s1}{\PYZsq{}}\PY{p}{)}
%    \PY{n}{ax}\PY{o}{.}\PY{n}{grid}\PY{p}{(}\PY{k+kc}{True}\PY{p}{)}
%\end{Verbatim}
%\end{tcolorbox}

%    \begin{tcolorbox}[breakable, size=fbox, boxrule=1pt, pad at break*=1mm,colback=cellbackground, colframe=cellborder]
%\prompt{In}{incolor}{8}{\boxspacing}
%\begin{Verbatim}[commandchars=\\\{\}]
%\PY{k}{def} \PY{n+nf}{plot\PYZus{}matrix} \PY{p}{(}\PY{n}{ax}\PY{p}{,} \PY{n}{cov}\PY{p}{)}\PY{p}{:}
%    \PY{l+s+sd}{\PYZsq{}\PYZsq{}\PYZsq{}Plot covariance matrix\PYZsq{}\PYZsq{}\PYZsq{}}
%    \PY{n}{seaborn}\PY{o}{.}\PY{n}{set\PYZus{}style}\PY{p}{(}\PY{l+s+s1}{\PYZsq{}}\PY{l+s+s1}{white}\PY{l+s+s1}{\PYZsq{}}\PY{p}{)}
%    \PY{n}{im} \PY{o}{=} \PY{n}{ax}\PY{o}{.}\PY{n}{imshow}\PY{p}{(}\PY{n}{cov}\PY{p}{,} \PY{n}{cmap}\PY{o}{=}\PY{n}{cm}\PY{o}{.}\PY{n}{YlGnBu}\PY{p}{)}
%    \PY{n}{cbar} \PY{o}{=} \PY{n}{plt}\PY{o}{.}\PY{n}{colorbar}\PY{p}{(}\PY{n}{im}\PY{p}{,} \PY{n}{ax}\PY{o}{=}\PY{n}{ax}\PY{p}{,} \PY{n}{fraction}\PY{o}{=}\PY{l+m+mf}{0.045}\PY{p}{,} \PY{n}{pad}\PY{o}{=}\PY{l+m+mf}{0.05}\PY{p}{)}
%    \PY{n}{cbar}\PY{o}{.}\PY{n}{ax}\PY{o}{.}\PY{n}{set\PYZus{}ylabel}\PY{p}{(}\PY{l+s+s1}{\PYZsq{}}\PY{l+s+s1}{\PYZdl{}k(X,X)\PYZdl{}}\PY{l+s+s1}{\PYZsq{}}\PY{p}{,} \PY{n}{fontsize}\PY{o}{=}\PY{l+m+mi}{10}\PY{p}{)}
%    \PY{n}{ax}\PY{o}{.}\PY{n}{set\PYZus{}xticks}\PY{p}{(}\PY{p}{[}\PY{p}{]}\PY{p}{)}
%    \PY{n}{ax}\PY{o}{.}\PY{n}{set\PYZus{}yticks}\PY{p}{(}\PY{p}{[}\PY{p}{]}\PY{p}{)}
%    \PY{n}{ax}\PY{o}{.}\PY{n}{grid}\PY{p}{(}\PY{k+kc}{False}\PY{p}{)}
%    \PY{n}{seaborn}\PY{o}{.}\PY{n}{set\PYZus{}style}\PY{p}{(}\PY{l+s+s1}{\PYZsq{}}\PY{l+s+s1}{whitegrid}\PY{l+s+s1}{\PYZsq{}}\PY{p}{)}
%\end{Verbatim}
%\end{tcolorbox}

    Рассмотрим четыре ковариационных функции.

%    \begin{tcolorbox}[breakable, size=fbox, boxrule=1pt, pad at break*=1mm,colback=cellbackground, colframe=cellborder]
%\prompt{In}{incolor}{9}{\boxspacing}
%\begin{Verbatim}[commandchars=\\\{\}]
%\PY{n}{kernels} \PY{o}{=} \PY{p}{[} \PY{n}{GP\PYZus{}kernels}\PY{o}{.}\PY{n}{gauss}\PY{p}{,} \PY{n}{GP\PYZus{}kernels}\PY{o}{.}\PY{n}{brownian}\PY{p}{,}
%            \PY{n}{GP\PYZus{}kernels}\PY{o}{.}\PY{n}{rational\PYZus{}quadratic}\PY{p}{,} \PY{n}{GP\PYZus{}kernels}\PY{o}{.}\PY{n}{periodic} \PY{p}{]}
%\PY{n}{kernel\PYZus{}names} \PY{o}{=} \PY{p}{[}\PY{l+s+s1}{\PYZsq{}}\PY{l+s+s1}{Gauss kernel}\PY{l+s+s1}{\PYZsq{}}\PY{p}{,} \PY{l+s+s1}{\PYZsq{}}\PY{l+s+s1}{Brownian kernel}\PY{l+s+s1}{\PYZsq{}}\PY{p}{,}
%                \PY{l+s+s1}{\PYZsq{}}\PY{l+s+s1}{Rational quadratic kernel}\PY{l+s+s1}{\PYZsq{}}\PY{p}{,} \PY{l+s+s1}{\PYZsq{}}\PY{l+s+s1}{Periodic kernel}\PY{l+s+s1}{\PYZsq{}}\PY{p}{]}
%\PY{n}{kernel\PYZus{}args} \PY{o}{=} \PY{p}{\PYZob{}}\PY{l+s+s1}{\PYZsq{}}\PY{l+s+s1}{l}\PY{l+s+s1}{\PYZsq{}}\PY{p}{:}\PY{l+m+mf}{1.}\PY{p}{,} \PY{l+s+s1}{\PYZsq{}}\PY{l+s+s1}{sigma\PYZus{}k}\PY{l+s+s1}{\PYZsq{}}\PY{p}{:}\PY{l+m+mf}{1.}\PY{p}{,} \PY{l+s+s1}{\PYZsq{}}\PY{l+s+s1}{alpha}\PY{l+s+s1}{\PYZsq{}}\PY{p}{:}\PY{l+m+mf}{0.1}\PY{p}{,} \PY{l+s+s1}{\PYZsq{}}\PY{l+s+s1}{period}\PY{l+s+s1}{\PYZsq{}}\PY{p}{:}\PY{l+m+mf}{5.}\PY{p}{\PYZcb{}}
%\PY{n}{sigma\PYZus{}n} \PY{o}{=} \PY{l+m+mf}{1e\PYZhy{}6}
%\end{Verbatim}
%\end{tcolorbox}

%    \begin{tcolorbox}[breakable, size=fbox, boxrule=1pt, pad at break*=1mm,colback=cellbackground, colframe=cellborder]
%\prompt{In}{incolor}{10}{\boxspacing}
%\begin{Verbatim}[commandchars=\\\{\}]
%\PY{k}{def} \PY{n+nf}{make\PYZus{}kernel\PYZus{}title}\PY{p}{(}\PY{n}{kernel\PYZus{}number}\PY{p}{)}\PY{p}{:}
%    \PY{l+s+sd}{\PYZsq{}\PYZsq{}\PYZsq{}Make title for kernel plot\PYZsq{}\PYZsq{}\PYZsq{}}
%    \PY{n}{label}  \PY{o}{=} \PY{n}{kernel\PYZus{}names}\PY{p}{[}\PY{n}{kernel\PYZus{}number}\PY{p}{]}
%    \PY{n}{label} \PY{o}{+}\PY{o}{=} \PY{l+s+sa}{f}\PY{l+s+s2}{\PYZdq{}}\PY{l+s+s2}{: \PYZdl{}}\PY{l+s+se}{\PYZbs{}\PYZbs{}}\PY{l+s+s2}{ell = }\PY{l+s+si}{\PYZob{}}\PY{n}{kernel\PYZus{}args}\PY{p}{[}\PY{l+s+s1}{\PYZsq{}}\PY{l+s+s1}{l}\PY{l+s+s1}{\PYZsq{}}\PY{p}{]}\PY{l+s+si}{:}\PY{l+s+s2}{.2}\PY{l+s+si}{\PYZcb{}}\PY{l+s+s2}{\PYZdl{}}\PY{l+s+s2}{\PYZdq{}}
%    \PY{n}{label} \PY{o}{+}\PY{o}{=} \PY{l+s+sa}{f}\PY{l+s+s2}{\PYZdq{}}\PY{l+s+s2}{\PYZdl{},}\PY{l+s+s2}{\PYZbs{}}\PY{l+s+s2}{;}\PY{l+s+se}{\PYZbs{}\PYZbs{}}\PY{l+s+s2}{sigma\PYZus{}k = }\PY{l+s+si}{\PYZob{}}\PY{n}{kernel\PYZus{}args}\PY{p}{[}\PY{l+s+s1}{\PYZsq{}}\PY{l+s+s1}{sigma\PYZus{}k}\PY{l+s+s1}{\PYZsq{}}\PY{p}{]}\PY{l+s+si}{:}\PY{l+s+s2}{.2}\PY{l+s+si}{\PYZcb{}}\PY{l+s+s2}{\PYZdl{}}\PY{l+s+s2}{\PYZdq{}}
%    \PY{k}{if} \PY{n}{kernel\PYZus{}names}\PY{p}{[}\PY{n}{i}\PY{p}{]} \PY{o}{==} \PY{l+s+s1}{\PYZsq{}}\PY{l+s+s1}{Rational quadratic kernel}\PY{l+s+s1}{\PYZsq{}}\PY{p}{:}
%        \PY{n}{label} \PY{o}{+}\PY{o}{=} \PY{l+s+sa}{f}\PY{l+s+s2}{\PYZdq{}}\PY{l+s+s2}{\PYZdl{},}\PY{l+s+s2}{\PYZbs{}}\PY{l+s+s2}{;}\PY{l+s+se}{\PYZbs{}\PYZbs{}}\PY{l+s+s2}{alpha = }\PY{l+s+si}{\PYZob{}}\PY{n}{kernel\PYZus{}args}\PY{p}{[}\PY{l+s+s1}{\PYZsq{}}\PY{l+s+s1}{alpha}\PY{l+s+s1}{\PYZsq{}}\PY{p}{]}\PY{l+s+si}{:}\PY{l+s+s2}{.2}\PY{l+s+si}{\PYZcb{}}\PY{l+s+s2}{\PYZdl{}}\PY{l+s+s2}{\PYZdq{}}
%    \PY{k}{elif} \PY{n}{kernel\PYZus{}names}\PY{p}{[}\PY{n}{i}\PY{p}{]} \PY{o}{==} \PY{l+s+s1}{\PYZsq{}}\PY{l+s+s1}{Periodic kernel}\PY{l+s+s1}{\PYZsq{}}\PY{p}{:}
%        \PY{n}{label} \PY{o}{+}\PY{o}{=} \PY{l+s+sa}{f}\PY{l+s+s2}{\PYZdq{}}\PY{l+s+s2}{\PYZdl{},}\PY{l+s+s2}{\PYZbs{}}\PY{l+s+s2}{;period = }\PY{l+s+si}{\PYZob{}}\PY{n}{kernel\PYZus{}args}\PY{p}{[}\PY{l+s+s1}{\PYZsq{}}\PY{l+s+s1}{period}\PY{l+s+s1}{\PYZsq{}}\PY{p}{]}\PY{l+s+si}{:}\PY{l+s+s2}{.2}\PY{l+s+si}{\PYZcb{}}\PY{l+s+s2}{\PYZdl{}}\PY{l+s+s2}{\PYZdq{}}
%    \PY{k}{return} \PY{n}{label}
%\end{Verbatim}
%\end{tcolorbox}

%    \begin{tcolorbox}[breakable, size=fbox, boxrule=1pt, pad at break*=1mm,colback=cellbackground, colframe=cellborder]
%\prompt{In}{incolor}{11}{\boxspacing}
%\begin{Verbatim}[commandchars=\\\{\}]
%\PY{k}{for} \PY{n}{i}\PY{p}{,} \PY{n}{kernel} \PY{o+ow}{in} \PY{n+nb}{enumerate}\PY{p}{(}\PY{n}{kernels}\PY{p}{)}\PY{p}{:}
%    \PY{n}{fig} \PY{o}{=} \PY{n}{plt}\PY{o}{.}\PY{n}{figure}\PY{p}{(}\PY{n}{figsize}\PY{o}{=}\PY{p}{(}\PY{l+m+mi}{14}\PY{p}{,} \PY{l+m+mi}{4}\PY{p}{)}\PY{p}{)}
%    \PY{n}{gs} \PY{o}{=} \PY{n}{gridspec}\PY{o}{.}\PY{n}{GridSpec}\PY{p}{(}\PY{l+m+mi}{1}\PY{p}{,} \PY{l+m+mi}{2}\PY{p}{,} \PY{n}{width\PYZus{}ratios}\PY{o}{=}\PY{p}{[}\PY{l+m+mi}{2}\PY{p}{,}\PY{l+m+mi}{1}\PY{p}{]}\PY{p}{,} \PY{n}{wspace}\PY{o}{=}\PY{l+m+mf}{0.1}\PY{p}{,} \PY{n}{hspace}\PY{o}{=}\PY{l+m+mf}{0.3}\PY{p}{)}
%    \PY{c+c1}{\PYZsh{} Compute mean and covariance of the posterior predictive distribution}
%    \PY{n}{K} \PY{o}{=} \PY{n}{kernel}\PY{p}{(}\PY{n}{X\PYZus{}test}\PY{p}{,} \PY{n}{X\PYZus{}test}\PY{p}{,} \PY{n}{kernel\PYZus{}args}\PY{p}{)}
%    \PY{n}{mu}\PY{p}{,} \PY{n}{cov} \PY{o}{=} \PY{n}{GP\PYZus{}predictor}\PY{p}{(}\PY{n}{X\PYZus{}test}\PY{p}{,} \PY{n}{X\PYZus{}train}\PY{p}{,} \PY{n}{Y\PYZus{}train}\PY{p}{,}
%                           \PY{n}{kernel}\PY{p}{,} \PY{n}{kernel\PYZus{}args}\PY{p}{,} \PY{n}{sigma\PYZus{}n}\PY{p}{)}
%    
%    \PY{c+c1}{\PYZsh{} plot}
%    \PY{n}{ax1} \PY{o}{=} \PY{n}{fig}\PY{o}{.}\PY{n}{add\PYZus{}subplot}\PY{p}{(}\PY{n}{gs}\PY{p}{[}\PY{l+m+mi}{0}\PY{p}{]}\PY{p}{)}
%    \PY{n}{ax2} \PY{o}{=} \PY{n}{fig}\PY{o}{.}\PY{n}{add\PYZus{}subplot}\PY{p}{(}\PY{n}{gs}\PY{p}{[}\PY{l+m+mi}{1}\PY{p}{]}\PY{p}{)}
%    \PY{n}{plot\PYZus{}regression}\PY{p}{(}\PY{n}{ax1}\PY{p}{,} \PY{n}{X\PYZus{}test}\PY{p}{,} \PY{n}{mu}\PY{p}{,} \PY{n}{cov}\PY{p}{,} \PY{n}{X\PYZus{}train}\PY{p}{,} \PY{n}{Y\PYZus{}train}\PY{p}{)}
%    \PY{n}{plot\PYZus{}matrix}\PY{p}{(}\PY{n}{ax2}\PY{p}{,} \PY{n}{K}\PY{p}{)}
%    \PY{n}{title} \PY{o}{=} \PY{n}{make\PYZus{}kernel\PYZus{}title}\PY{p}{(}\PY{n}{i}\PY{p}{)}
%    \PY{n}{fig}\PY{o}{.}\PY{n}{suptitle}\PY{p}{(}\PY{n}{title}\PY{p}{,} \PY{n}{y}\PY{o}{=}\PY{l+m+mf}{0.97}\PY{p}{)}
%
%    \PY{n}{plt}\PY{o}{.}\PY{n}{show}\PY{p}{(}\PY{p}{)}
%\end{Verbatim}
%\end{tcolorbox}

    \begin{center}
    \adjustimage{max size={0.9\linewidth}{0.9\paperheight}}{Kernel_gauss.pdf}
    \end{center}
%    { \hspace*{\fill} \\}

    \begin{center}
    \adjustimage{max size={0.9\linewidth}{0.9\paperheight}}{Kernel_brownian.pdf}
    \end{center}
%    { \hspace*{\fill} \\}

    \begin{center}
    \adjustimage{max size={0.9\linewidth}{0.9\paperheight}}{Kernel_RQ.pdf}
    \end{center}
%    { \hspace*{\fill} \\}

    \begin{center}
    \adjustimage{max size={0.9\linewidth}{0.9\paperheight}}{Kernel_periodic.pdf}
    \end{center}
%    { \hspace*{\fill} \\}

    \begin{center}\rule{0.5\linewidth}{0.5pt}\end{center}

    \hypertarget{ux432ux43bux438ux44fux43dux438ux435-ux43fux430ux440ux430ux43cux435ux442ux440ux43eux432-ux43aux43eux432ux430ux440ux438ux430ux446ux438ux43eux43dux43dux43eux439-ux444ux443ux43dux43aux446ux438ux438}{%
\section{Влияние параметров ковариационной
функции}\label{ux432ux43bux438ux44fux43dux438ux435-ux43fux430ux440ux430ux43cux435ux442ux440ux43eux432-ux43aux43eux432ux430ux440ux438ux430ux446ux438ux43eux43dux43dux43eux439-ux444ux443ux43dux43aux446ux438ux438}}

    \hypertarget{ux438ux441ux43fux43eux43bux44cux437ux443ux435ux43cux44bux435-ux444ux443ux43dux43aux446ux438ux438}{%
\subsection{Используемые
функции}\label{ux438ux441ux43fux43eux43bux44cux437ux443ux435ux43cux44bux435-ux444ux443ux43dux43aux446ux438ux438}}

На предыдущих занятиях мы использовали квадратичное экспоненциальное
ядро:

\[
  k(x, x') = \sigma_k^2 \exp{ \left( -\frac{\lVert x - x' \rVert^2}{2l^2} \right) }.
\]

Теперь пришло время поиграть с параметрами ядра: ширины ядра \(l\) и
амплитуды \(\sigma_f\).

Мы будем использовать функции \texttt{gauss()} из
\texttt{GP\_kernels.py}, а также \texttt{plot\_GP()} и
\texttt{GP\_predictor()} из \texttt{GP\_utils.py}.

    \hypertarget{ux432ux430ux440ux44cux438ux440ux43eux432ux430ux43dux438ux435-ux43fux430ux440ux430ux43cux435ux442ux440ux43eux432-ux44fux434ux440ux430-ux438-ux430ux43cux43fux43bux438ux442ux443ux434ux44b-ux448ux443ux43cux430}{%
\subsection{Варьирование параметров ядра и амплитуды
шума}\label{ux432ux430ux440ux44cux438ux440ux43eux432ux430ux43dux438ux435-ux43fux430ux440ux430ux43cux435ux442ux440ux43eux432-ux44fux434ux440ux430-ux438-ux430ux43cux43fux43bux438ux442ux443ux434ux44b-ux448ux443ux43cux430}}

В следующем примере показано влияние параметров ядра \(l\) и
\(\sigma_k\), а также амплитуды шума \(\sigma_n\).

%    \begin{tcolorbox}[breakable, size=fbox, boxrule=1pt, pad at break*=1mm,colback=cellbackground, colframe=cellborder]
%\prompt{In}{incolor}{12}{\boxspacing}
%\begin{Verbatim}[commandchars=\\\{\}]
%\PY{n}{params} \PY{o}{=} \PY{p}{[}
%\PY{c+c1}{\PYZsh{}     (1.0, 1.0, 0.1),}
%    \PY{p}{(}\PY{l+m+mf}{0.2}\PY{p}{,} \PY{l+m+mf}{1.0}\PY{p}{,} \PY{l+m+mf}{0.1}\PY{p}{)}\PY{p}{,}
%    \PY{p}{(}\PY{l+m+mf}{3.0}\PY{p}{,} \PY{l+m+mf}{1.0}\PY{p}{,} \PY{l+m+mf}{0.1}\PY{p}{)}\PY{p}{,}
%    \PY{p}{(}\PY{l+m+mf}{1.0}\PY{p}{,} \PY{l+m+mf}{0.2}\PY{p}{,} \PY{l+m+mf}{0.1}\PY{p}{)}\PY{p}{,}
%    \PY{p}{(}\PY{l+m+mf}{1.0}\PY{p}{,} \PY{l+m+mf}{5.0}\PY{p}{,} \PY{l+m+mf}{0.1}\PY{p}{)}\PY{p}{,}
%    \PY{p}{(}\PY{l+m+mf}{1.0}\PY{p}{,} \PY{l+m+mf}{1.0}\PY{p}{,} \PY{l+m+mf}{1e\PYZhy{}6}\PY{p}{)}\PY{p}{,}
%    \PY{p}{(}\PY{l+m+mf}{1.0}\PY{p}{,} \PY{l+m+mf}{1.0}\PY{p}{,} \PY{l+m+mf}{2.0}\PY{p}{)}\PY{p}{,}
%\PY{p}{]}
%\end{Verbatim}
%\end{tcolorbox}
%
%    \begin{tcolorbox}[breakable, size=fbox, boxrule=1pt, pad at break*=1mm,colback=cellbackground, colframe=cellborder]
%\prompt{In}{incolor}{13}{\boxspacing}
%\begin{Verbatim}[commandchars=\\\{\}]
%\PY{n}{plt}\PY{o}{.}\PY{n}{figure}\PY{p}{(}\PY{n}{figsize}\PY{o}{=}\PY{p}{(}\PY{l+m+mi}{12}\PY{p}{,} \PY{l+m+mi}{12}\PY{p}{)}\PY{p}{)}
%\PY{n}{y\PYZus{}lim} \PY{o}{=} \PY{p}{[}\PY{o}{\PYZhy{}}\PY{l+m+mi}{3}\PY{p}{,} \PY{l+m+mi}{3}\PY{p}{]}
%
%\PY{k}{for} \PY{n}{i}\PY{p}{,} \PY{p}{(}\PY{n}{l}\PY{p}{,} \PY{n}{sigma\PYZus{}k}\PY{p}{,} \PY{n}{sigma\PYZus{}n}\PY{p}{)} \PY{o+ow}{in} \PY{n+nb}{enumerate}\PY{p}{(}\PY{n}{params}\PY{p}{)}\PY{p}{:}
%    \PY{n}{kernel\PYZus{}fun} \PY{o}{=} \PY{n}{GP\PYZus{}kernels}\PY{o}{.}\PY{n}{gauss}
%    \PY{n}{kernel\PYZus{}args} \PY{o}{=} \PY{p}{\PYZob{}}\PY{l+s+s1}{\PYZsq{}}\PY{l+s+s1}{l}\PY{l+s+s1}{\PYZsq{}}\PY{p}{:}\PY{n}{l}\PY{p}{,} \PY{l+s+s1}{\PYZsq{}}\PY{l+s+s1}{sigma\PYZus{}k}\PY{l+s+s1}{\PYZsq{}}\PY{p}{:}\PY{n}{sigma\PYZus{}k}\PY{p}{\PYZcb{}}
%    \PY{n}{mu}\PY{p}{,} \PY{n}{cov} \PY{o}{=} \PY{n}{GP\PYZus{}predictor}\PY{p}{(}\PY{n}{X\PYZus{}test}\PY{p}{,} \PY{n}{X\PYZus{}train}\PY{p}{,} \PY{n}{Y\PYZus{}train}\PY{p}{,}
%                           \PY{n}{kernel\PYZus{}fun}\PY{p}{,} \PY{n}{kernel\PYZus{}args}\PY{p}{,} \PY{n}{sigma\PYZus{}n}\PY{p}{)}
%    
%    \PY{n}{plt}\PY{o}{.}\PY{n}{subplot}\PY{p}{(}\PY{l+m+mi}{3}\PY{p}{,} \PY{l+m+mi}{2}\PY{p}{,} \PY{n}{i}\PY{o}{+}\PY{l+m+mi}{1}\PY{p}{)}
%    \PY{n}{plt}\PY{o}{.}\PY{n}{subplots\PYZus{}adjust}\PY{p}{(}\PY{n}{top}\PY{o}{=}\PY{l+m+mi}{2}\PY{p}{)}
%    \PY{n}{plt}\PY{o}{.}\PY{n}{title}\PY{p}{(}\PY{l+s+sa}{f}\PY{l+s+s1}{\PYZsq{}}\PY{l+s+s1}{\PYZdl{}}\PY{l+s+s1}{\PYZbs{}}\PY{l+s+s1}{ell = }\PY{l+s+si}{\PYZob{}}\PY{n}{l}\PY{l+s+si}{\PYZcb{}}\PY{l+s+s1}{,}\PY{l+s+s1}{\PYZbs{}}\PY{l+s+s1}{;}\PY{l+s+s1}{\PYZbs{}}\PY{l+s+s1}{sigma\PYZus{}k = }\PY{l+s+si}{\PYZob{}}\PY{n}{sigma\PYZus{}k}\PY{l+s+si}{\PYZcb{}}\PY{l+s+s1}{,}\PY{l+s+s1}{\PYZbs{}}\PY{l+s+s1}{;}\PY{l+s+s1}{\PYZbs{}}\PY{l+s+s1}{sigma\PYZus{}n = }\PY{l+s+si}{\PYZob{}}\PY{n}{sigma\PYZus{}n}\PY{l+s+si}{\PYZcb{}}\PY{l+s+s1}{\PYZdl{}}\PY{l+s+s1}{\PYZsq{}}\PY{p}{)}
%    \PY{n}{plot\PYZus{}GP}\PY{p}{(}\PY{n}{X\PYZus{}test}\PY{p}{,} \PY{n}{mu}\PY{p}{,} \PY{n}{cov}\PY{p}{,} \PY{n}{X\PYZus{}train}\PY{p}{,} \PY{n}{Y\PYZus{}train}\PY{p}{,} \PY{n}{draw\PYZus{}ci}\PY{o}{=}\PY{k+kc}{True}\PY{p}{)}
%    \PY{n}{plt}\PY{o}{.}\PY{n}{ylim}\PY{p}{(}\PY{n}{y\PYZus{}lim}\PY{p}{)}
%    
%\PY{n}{plt}\PY{o}{.}\PY{n}{tight\PYZus{}layout}\PY{p}{(}\PY{p}{)}
%\PY{n}{plt}\PY{o}{.}\PY{n}{show}\PY{p}{(}\PY{p}{)}
%\end{Verbatim}
%\end{tcolorbox}

%    \begin{center}
%    \adjustimage{max size={0.9\linewidth}{0.9\paperheight}}{HP_variation.pdf}
%    \end{center}
%%    { \hspace*{\fill} \\}

    По рисункам можно сделать следующие выводы:

\begin{enumerate}
\def\labelenumi{\arabic{enumi}.}
\tightlist
\item
  Малое значения \(l\) приводит к достаточно «изогнутой» средней функции
  с большими доверительными интервалами между точками обучающей выборки.
  Большое значение ширины ядра \(l\) даёт более гладкую регрессионную
  функцию, но более грубую аппроксимацию обучающих данных.
\item
  Параметр \(\sigma_k\) контролирует вертикальную вариативность функций,
  взятых из GP. Это видно по большим доверительным интервалам за
  пределами области тренировочных данных на правом рисунке второй
  строки.
\item
  Параметр \(\sigma_n\) представляет собой уровень шума в обучающих
  данных. Более высокое значение \(\sigma_n\) приводит к более грубой
  аппроксимации, но позволяет избежать подгонки под шумные данные.
\end{enumerate}

    \begin{center}\rule{0.5\linewidth}{0.5pt}\end{center}

    \hypertarget{ux43eux43fux442ux438ux43cux438ux437ux430ux446ux438ux44f-ux43fux430ux440ux430ux43cux435ux442ux440ux43eux432-ux44fux434ux440ux430}{%
\section{Оптимизация параметров
ядра}\label{ux43eux43fux442ux438ux43cux438ux437ux430ux446ux438ux44f-ux43fux430ux440ux430ux43cux435ux442ux440ux43eux432-ux44fux434ux440ux430}}

    \hypertarget{ux43eux434ux43dux43eux43cux435ux440ux43dux44bux439-ux441ux43bux443ux447ux430ux439}{%
\subsection{Одномерный
случай}\label{ux43eux434ux43dux43eux43cux435ux440ux43dux44bux439-ux441ux43bux443ux447ux430ux439}}

    Мы можем настроить гиперпараметры \(\theta\) нашей гауссовской модели
процесса на основе полученных данных. Настройка делается путём
максимизации функции правдоподобия \(p(Y \mid X, \theta)\) распределения
гауссовского процесса на основе данных наблюдений \((X, Y)\). \[
  \hat{\theta}  = \underset{\theta}{\text{argmax}} \left[ p(Y \mid X, \theta) \right].
\]

    Функция правдоподобия гауссовского процесса определяется следующим
образом: \[
  p(y \mid 0, \Sigma) = \mathcal{N}(y \mid 0, \Sigma) = 
  \frac{1}{\sqrt{(2\pi)^d \lvert\Sigma\rvert}} \exp{ \left( -\frac{1}{2}Y^\top \Sigma^{-1} Y \right)}.
\]

    Оптимальные значения гиперпарамтров ядра (гиперпарамтров) могут быть
оценены путём максимизации логарифма правдоподобия \[
  \log p(Y \mid X) = - \frac{1}{2} Y^\top \Sigma^{-1} Y -\frac{1}{2} \ln \lvert \Sigma \rvert - \frac{d}{2} \ln(2\pi). \tag{1}
\]

%    \begin{tcolorbox}[breakable, size=fbox, boxrule=1pt, pad at break*=1mm,colback=cellbackground, colframe=cellborder]
%\prompt{In}{incolor}{14}{\boxspacing}
%\begin{Verbatim}[commandchars=\\\{\}]
%\PY{k}{def} \PY{n+nf}{nll\PYZus{}fn\PYZus{}1}\PY{p}{(}\PY{n}{X\PYZus{}train}\PY{p}{,} \PY{n}{Y\PYZus{}train}\PY{p}{)}\PY{p}{:}
%    \PY{l+s+sd}{\PYZsq{}\PYZsq{}\PYZsq{}}
%\PY{l+s+sd}{    Returns a function that computes the negative marginal log\PYZhy{}}
%\PY{l+s+sd}{    likelihood for training data X\PYZus{}train and Y\PYZus{}train.}
%\PY{l+s+sd}{    }
%\PY{l+s+sd}{    Args:}
%\PY{l+s+sd}{        X\PYZus{}train: training locations (m x d)}
%\PY{l+s+sd}{        Y\PYZus{}train: training targets (m x 1)}
%\PY{l+s+sd}{        }
%\PY{l+s+sd}{    Returns:}
%\PY{l+s+sd}{        Minimization objective}
%\PY{l+s+sd}{    \PYZsq{}\PYZsq{}\PYZsq{}}
%    \PY{k}{def} \PY{n+nf}{nll\PYZus{}1}\PY{p}{(}\PY{n}{theta}\PY{p}{)}\PY{p}{:}
%        \PY{n}{kernel\PYZus{}args} \PY{o}{=} \PY{p}{\PYZob{}}\PY{l+s+s1}{\PYZsq{}}\PY{l+s+s1}{l}\PY{l+s+s1}{\PYZsq{}}\PY{p}{:}\PY{n}{theta}\PY{p}{[}\PY{l+m+mi}{0}\PY{p}{]}\PY{p}{,} \PY{l+s+s1}{\PYZsq{}}\PY{l+s+s1}{sigma\PYZus{}k}\PY{l+s+s1}{\PYZsq{}}\PY{p}{:}\PY{n}{theta}\PY{p}{[}\PY{l+m+mi}{1}\PY{p}{]}\PY{p}{\PYZcb{}}
%        \PY{n}{Sigma} \PY{o}{=} \PY{n}{GP\PYZus{}kernels}\PY{o}{.}\PY{n}{gauss}\PY{p}{(}\PY{n}{X\PYZus{}train}\PY{p}{,} \PY{n}{X\PYZus{}train}\PY{p}{,} \PY{n}{kernel\PYZus{}args}\PY{p}{)} \PY{o}{+} \PYZbs{}
%            \PY{n}{theta}\PY{p}{[}\PY{l+m+mi}{2}\PY{p}{]}\PY{o}{*}\PY{o}{*}\PY{l+m+mi}{2} \PY{o}{*} \PY{n}{np}\PY{o}{.}\PY{n}{eye}\PY{p}{(}\PY{n+nb}{len}\PY{p}{(}\PY{n}{X\PYZus{}train}\PY{p}{)}\PY{p}{)}
%        \PY{n}{y} \PY{o}{=} \PY{n}{Y\PYZus{}train}\PY{o}{.}\PY{n}{T} \PY{o}{@} \PY{n}{LA}\PY{o}{.}\PY{n}{inv}\PY{p}{(}\PY{n}{Sigma}\PY{p}{)} \PY{o}{@} \PY{n}{Y\PYZus{}train} \PY{o}{+} \PY{n}{np}\PY{o}{.}\PY{n}{log}\PY{p}{(}\PY{n}{LA}\PY{o}{.}\PY{n}{det}\PY{p}{(}\PY{n}{Sigma}\PY{p}{)}\PY{p}{)}
%        \PY{k}{return} \PY{n}{y}\PY{o}{.}\PY{n}{flatten}\PY{p}{(}\PY{p}{)}
%    
%    \PY{k}{return} \PY{n}{nll\PYZus{}1}
%\end{Verbatim}
%\end{tcolorbox}
%
%    \begin{tcolorbox}[breakable, size=fbox, boxrule=1pt, pad at break*=1mm,colback=cellbackground, colframe=cellborder]
%\prompt{In}{incolor}{15}{\boxspacing}
%\begin{Verbatim}[commandchars=\\\{\}]
%\PY{n}{plt}\PY{o}{.}\PY{n}{figure}\PY{p}{(}\PY{n}{figsize}\PY{o}{=}\PY{p}{(}\PY{l+m+mi}{12}\PY{p}{,} \PY{l+m+mi}{12}\PY{p}{)}\PY{p}{)}
%\PY{n}{y\PYZus{}lim} \PY{o}{=} \PY{p}{[}\PY{o}{\PYZhy{}}\PY{l+m+mi}{3}\PY{p}{,} \PY{l+m+mi}{3}\PY{p}{]}
%
%\PY{k}{for} \PY{n}{i}\PY{p}{,} \PY{p}{(}\PY{n}{l}\PY{p}{,} \PY{n}{sigma\PYZus{}k}\PY{p}{,} \PY{n}{sigma\PYZus{}n}\PY{p}{)} \PY{o+ow}{in} \PY{n+nb}{enumerate}\PY{p}{(}\PY{n}{params}\PY{p}{)}\PY{p}{:}
%    \PY{n}{kernel\PYZus{}fun} \PY{o}{=} \PY{n}{GP\PYZus{}kernels}\PY{o}{.}\PY{n}{gauss}
%    \PY{n}{kernel\PYZus{}args} \PY{o}{=} \PY{p}{\PYZob{}}\PY{l+s+s1}{\PYZsq{}}\PY{l+s+s1}{l}\PY{l+s+s1}{\PYZsq{}}\PY{p}{:}\PY{n}{l}\PY{p}{,} \PY{l+s+s1}{\PYZsq{}}\PY{l+s+s1}{sigma\PYZus{}k}\PY{l+s+s1}{\PYZsq{}}\PY{p}{:}\PY{n}{sigma\PYZus{}k}\PY{p}{\PYZcb{}}
%    \PY{n}{mu}\PY{p}{,} \PY{n}{cov} \PY{o}{=} \PY{n}{GP\PYZus{}predictor}\PY{p}{(}\PY{n}{X\PYZus{}test}\PY{p}{,} \PY{n}{X\PYZus{}train}\PY{p}{,} \PY{n}{Y\PYZus{}train}\PY{p}{,}
%                           \PY{n}{kernel\PYZus{}fun}\PY{p}{,} \PY{n}{kernel\PYZus{}args}\PY{p}{,} \PY{n}{sigma\PYZus{}n}\PY{p}{)}
%    \PY{n}{nll} \PY{o}{=} \PY{n}{nll\PYZus{}fn\PYZus{}1}\PY{p}{(}\PY{n}{X\PYZus{}train}\PY{p}{,} \PY{n}{Y\PYZus{}train}\PY{p}{)}\PY{p}{(}\PY{p}{[}\PY{n}{l}\PY{p}{,} \PY{n}{sigma\PYZus{}k}\PY{p}{,} \PY{n}{sigma\PYZus{}n}\PY{p}{]}\PY{p}{)}\PY{p}{[}\PY{l+m+mi}{0}\PY{p}{]}
%    
%    \PY{n}{plt}\PY{o}{.}\PY{n}{subplot}\PY{p}{(}\PY{l+m+mi}{3}\PY{p}{,} \PY{l+m+mi}{2}\PY{p}{,} \PY{n}{i}\PY{o}{+}\PY{l+m+mi}{1}\PY{p}{)}
%    \PY{n}{plt}\PY{o}{.}\PY{n}{subplots\PYZus{}adjust}\PY{p}{(}\PY{n}{top}\PY{o}{=}\PY{l+m+mi}{2}\PY{p}{)}
%    \PY{n}{plt}\PY{o}{.}\PY{n}{title}\PY{p}{(}\PY{l+s+sa}{f}\PY{l+s+s1}{\PYZsq{}}\PY{l+s+s1}{\PYZdl{}}\PY{l+s+s1}{\PYZbs{}}\PY{l+s+s1}{ell = }\PY{l+s+si}{\PYZob{}}\PY{n}{l}\PY{l+s+si}{\PYZcb{}}\PY{l+s+s1}{,}\PY{l+s+s1}{\PYZbs{}}\PY{l+s+s1}{;}\PY{l+s+s1}{\PYZbs{}}\PY{l+s+s1}{sigma\PYZus{}k = }\PY{l+s+si}{\PYZob{}}\PY{n}{sigma\PYZus{}k}\PY{l+s+si}{\PYZcb{}}\PY{l+s+s1}{,}\PY{l+s+s1}{\PYZbs{}}\PY{l+s+s1}{;}\PY{l+s+se}{\PYZbs{}}
%\PY{l+s+s1}{                 }\PY{l+s+s1}{\PYZbs{}}\PY{l+s+s1}{sigma\PYZus{}n = }\PY{l+s+si}{\PYZob{}}\PY{n}{sigma\PYZus{}n}\PY{l+s+si}{\PYZcb{}}\PY{l+s+s1}{,}\PY{l+s+s1}{\PYZbs{}}\PY{l+s+s1}{;NLL = }\PY{l+s+si}{\PYZob{}}\PY{n}{nll}\PY{l+s+si}{:}\PY{l+s+s1}{.3}\PY{l+s+si}{\PYZcb{}}\PY{l+s+s1}{\PYZdl{}}\PY{l+s+s1}{\PYZsq{}}\PY{p}{)}
%    \PY{n}{plot\PYZus{}GP}\PY{p}{(}\PY{n}{X\PYZus{}test}\PY{p}{,} \PY{n}{mu}\PY{p}{,} \PY{n}{cov}\PY{p}{,} \PY{n}{X\PYZus{}train}\PY{p}{,} \PY{n}{Y\PYZus{}train}\PY{p}{,} \PY{n}{draw\PYZus{}ci}\PY{o}{=}\PY{k+kc}{True}\PY{p}{)}
%    \PY{n}{plt}\PY{o}{.}\PY{n}{ylim}\PY{p}{(}\PY{n}{y\PYZus{}lim}\PY{p}{)}
%    
%\PY{n}{plt}\PY{o}{.}\PY{n}{tight\PYZus{}layout}\PY{p}{(}\PY{p}{)}
%\PY{n}{plt}\PY{o}{.}\PY{n}{show}\PY{p}{(}\PY{p}{)}
%\end{Verbatim}
%\end{tcolorbox}

    \begin{center}
    \adjustimage{max size={0.9\linewidth}{0.9\paperheight}}{HP_variation_NLL.pdf}
    \end{center}
%    { \hspace*{\fill} \\}

%    \begin{tcolorbox}[breakable, size=fbox, boxrule=1pt, pad at break*=1mm,colback=cellbackground, colframe=cellborder]
%\prompt{In}{incolor}{16}{\boxspacing}
%\begin{Verbatim}[commandchars=\\\{\}]
%\PY{n}{res} \PY{o}{=} \PY{n}{minimize}\PY{p}{(}\PY{n}{nll\PYZus{}fn\PYZus{}1}\PY{p}{(}\PY{n}{X\PYZus{}train}\PY{p}{,} \PY{n}{Y\PYZus{}train}\PY{p}{)}\PY{p}{,} \PY{p}{[}\PY{l+m+mi}{1}\PY{p}{,} \PY{l+m+mi}{1}\PY{p}{,} \PY{l+m+mi}{1}\PY{p}{]}\PY{p}{,} 
%               \PY{n}{bounds}\PY{o}{=}\PY{p}{(}\PY{p}{(}\PY{l+m+mf}{1e\PYZhy{}3}\PY{p}{,} \PY{k+kc}{None}\PY{p}{)}\PY{p}{,} \PY{p}{(}\PY{l+m+mf}{1e\PYZhy{}3}\PY{p}{,} \PY{k+kc}{None}\PY{p}{)}\PY{p}{,} \PY{p}{(}\PY{l+m+mf}{1e\PYZhy{}6}\PY{p}{,} \PY{k+kc}{None}\PY{p}{)}\PY{p}{)}\PY{p}{,}
%               \PY{n}{method}\PY{o}{=}\PY{l+s+s1}{\PYZsq{}}\PY{l+s+s1}{L\PYZhy{}BFGS\PYZhy{}B}\PY{l+s+s1}{\PYZsq{}}\PY{p}{)}
%\PY{n}{l\PYZus{}opt}\PY{p}{,} \PY{n}{sigma\PYZus{}k\PYZus{}opt}\PY{p}{,} \PY{n}{sigma\PYZus{}n\PYZus{}opt} \PY{o}{=} \PY{n}{res}\PY{o}{.}\PY{n}{x}
%\PY{p}{[}\PY{n}{nll}\PY{p}{]} \PY{o}{=} \PY{n}{res}\PY{o}{.}\PY{n}{fun}
%
%\PY{n}{display}\PY{p}{(}\PY{n}{Markdown}\PY{p}{(}\PY{l+s+sa}{f}\PY{l+s+s1}{\PYZsq{}}\PY{l+s+s1}{\PYZdl{}}\PY{l+s+s1}{\PYZbs{}}\PY{l+s+s1}{ell = }\PY{l+s+si}{\PYZob{}}\PY{n}{l\PYZus{}opt}\PY{l+s+si}{:}\PY{l+s+s1}{.3}\PY{l+s+si}{\PYZcb{}}\PY{l+s+s1}{,}\PY{l+s+s1}{\PYZbs{}}\PY{l+s+s1}{;}\PY{l+s+s1}{\PYZbs{}}\PY{l+s+s1}{sigma\PYZus{}k = }\PY{l+s+si}{\PYZob{}}\PY{n}{sigma\PYZus{}k\PYZus{}opt}\PY{l+s+si}{:}\PY{l+s+s1}{.3}\PY{l+s+si}{\PYZcb{}}\PY{l+s+s1}{,}\PY{l+s+s1}{\PYZbs{}}\PY{l+s+s1}{;}\PY{l+s+se}{\PYZbs{}}
%\PY{l+s+s1}{                    }\PY{l+s+s1}{\PYZbs{}}\PY{l+s+s1}{sigma\PYZus{}n = }\PY{l+s+si}{\PYZob{}}\PY{n}{sigma\PYZus{}n\PYZus{}opt}\PY{l+s+si}{:}\PY{l+s+s1}{.3}\PY{l+s+si}{\PYZcb{}}\PY{l+s+s1}{,}\PY{l+s+s1}{\PYZbs{}}\PY{l+s+s1}{;NLL = }\PY{l+s+si}{\PYZob{}}\PY{n}{nll}\PY{l+s+si}{:}\PY{l+s+s1}{.6}\PY{l+s+si}{\PYZcb{}}\PY{l+s+s1}{\PYZdl{}}\PY{l+s+s1}{\PYZsq{}}\PY{p}{)}\PY{p}{)}
%\end{Verbatim}
%\end{tcolorbox}

%    \(\ell = 0.771,\;\sigma_k = 0.858,\; \sigma_n = 1e-06,\;NLL = 3.2661\)

    
%    \begin{tcolorbox}[breakable, size=fbox, boxrule=1pt, pad at break*=1mm,colback=cellbackground, colframe=cellborder]
%\prompt{In}{incolor}{17}{\boxspacing}
%\begin{Verbatim}[commandchars=\\\{\}]
%\PY{c+c1}{\PYZsh{} Compute the prosterior predictive statistics with optimized kernel parameters and plot the results}
%\PY{n}{kernel} \PY{o}{=} \PY{n}{GP\PYZus{}kernels}\PY{o}{.}\PY{n}{gauss}
%\PY{n}{kernel\PYZus{}args} \PY{o}{=} \PY{p}{\PYZob{}}\PY{l+s+s1}{\PYZsq{}}\PY{l+s+s1}{l}\PY{l+s+s1}{\PYZsq{}}\PY{p}{:}\PY{n}{l\PYZus{}opt}\PY{p}{,} \PY{l+s+s1}{\PYZsq{}}\PY{l+s+s1}{sigma\PYZus{}k}\PY{l+s+s1}{\PYZsq{}}\PY{p}{:}\PY{n}{sigma\PYZus{}k\PYZus{}opt}\PY{p}{\PYZcb{}}
%\PY{n}{mu}\PY{p}{,} \PY{n}{cov} \PY{o}{=} \PY{n}{GP\PYZus{}predictor}\PY{p}{(}\PY{n}{X\PYZus{}test}\PY{p}{,} \PY{n}{X\PYZus{}train}\PY{p}{,} \PY{n}{Y\PYZus{}train}\PY{p}{,}
%                           \PY{n}{kernel}\PY{p}{,} \PY{n}{kernel\PYZus{}args}\PY{p}{,} \PY{n}{sigma\PYZus{}n\PYZus{}opt}\PY{p}{)}
%
%\PY{n}{plt}\PY{o}{.}\PY{n}{figure}\PY{p}{(}\PY{n}{figsize}\PY{o}{=}\PY{p}{(}\PY{l+m+mi}{8}\PY{p}{,} \PY{l+m+mi}{5}\PY{p}{)}\PY{p}{)}
%\PY{n}{plt}\PY{o}{.}\PY{n}{title}\PY{p}{(}\PY{l+s+sa}{f}\PY{l+s+s1}{\PYZsq{}}\PY{l+s+s1}{\PYZdl{}}\PY{l+s+s1}{\PYZbs{}}\PY{l+s+s1}{ell = }\PY{l+s+si}{\PYZob{}}\PY{n}{l\PYZus{}opt}\PY{l+s+si}{:}\PY{l+s+s1}{.3}\PY{l+s+si}{\PYZcb{}}\PY{l+s+s1}{,}\PY{l+s+s1}{\PYZbs{}}\PY{l+s+s1}{;}\PY{l+s+s1}{\PYZbs{}}\PY{l+s+s1}{sigma\PYZus{}k = }\PY{l+s+si}{\PYZob{}}\PY{n}{sigma\PYZus{}k\PYZus{}opt}\PY{l+s+si}{:}\PY{l+s+s1}{.3}\PY{l+s+si}{\PYZcb{}}\PY{l+s+s1}{,}\PY{l+s+s1}{\PYZbs{}}\PY{l+s+s1}{;}\PY{l+s+se}{\PYZbs{}}
%\PY{l+s+s1}{             }\PY{l+s+s1}{\PYZbs{}}\PY{l+s+s1}{sigma\PYZus{}n = }\PY{l+s+si}{\PYZob{}}\PY{n}{sigma\PYZus{}n\PYZus{}opt}\PY{l+s+si}{:}\PY{l+s+s1}{.3}\PY{l+s+si}{\PYZcb{}}\PY{l+s+s1}{,}\PY{l+s+s1}{\PYZbs{}}\PY{l+s+s1}{;NLL = }\PY{l+s+si}{\PYZob{}}\PY{n}{nll}\PY{l+s+si}{:}\PY{l+s+s1}{.3}\PY{l+s+si}{\PYZcb{}}\PY{l+s+s1}{\PYZdl{}}\PY{l+s+s1}{\PYZsq{}}\PY{p}{)}
%\PY{n}{plot\PYZus{}GP}\PY{p}{(}\PY{n}{X\PYZus{}test}\PY{p}{,} \PY{n}{mu}\PY{p}{,} \PY{n}{cov}\PY{p}{,} \PY{n}{X\PYZus{}train}\PY{p}{,} \PY{n}{Y\PYZus{}train}\PY{p}{,} \PY{n}{draw\PYZus{}ci}\PY{o}{=}\PY{k+kc}{True}\PY{p}{)}
%\PY{n}{plt}\PY{o}{.}\PY{n}{tight\PYZus{}layout}\PY{p}{(}\PY{p}{)}
%\PY{n}{plt}\PY{o}{.}\PY{n}{show}\PY{p}{(}\PY{p}{)}
%\end{Verbatim}
%\end{tcolorbox}

    \begin{center}
    \adjustimage{max size={0.65\linewidth}{0.65\paperheight}}{HP_opt.pdf}
    \end{center}
%    { \hspace*{\fill} \\}

    Уровень шума минимизировался. Зафиксируем уровень шума.

%    \begin{tcolorbox}[breakable, size=fbox, boxrule=1pt, pad at break*=1mm,colback=cellbackground, colframe=cellborder]
%\prompt{In}{incolor}{18}{\boxspacing}
%\begin{Verbatim}[commandchars=\\\{\}]
%\PY{k}{def} \PY{n+nf}{nll\PYZus{}fn}\PY{p}{(}\PY{n}{X\PYZus{}train}\PY{p}{,} \PY{n}{Y\PYZus{}train}\PY{p}{,} \PY{n}{sigma\PYZus{}n}\PY{p}{)}\PY{p}{:}
%    \PY{l+s+sd}{\PYZsq{}\PYZsq{}\PYZsq{}}
%\PY{l+s+sd}{    Returns a function that computes the negative marginal log\PYZhy{}}
%\PY{l+s+sd}{    likelihood for training data X\PYZus{}train and Y\PYZus{}train and given }
%\PY{l+s+sd}{    noise level.}
%\PY{l+s+sd}{    }
%\PY{l+s+sd}{    Args:}
%\PY{l+s+sd}{        X\PYZus{}train: training locations (m x d)}
%\PY{l+s+sd}{        Y\PYZus{}train: training targets (m x 1)}
%\PY{l+s+sd}{        sigma\PYZus{}n: known noise level of Y\PYZus{}train}
%\PY{l+s+sd}{        }
%\PY{l+s+sd}{    Returns:}
%\PY{l+s+sd}{        Minimization objective}
%\PY{l+s+sd}{    \PYZsq{}\PYZsq{}\PYZsq{}}
%    \PY{k}{def} \PY{n+nf}{nll}\PY{p}{(}\PY{n}{theta}\PY{p}{)}\PY{p}{:}
%        \PY{n}{kernel\PYZus{}args} \PY{o}{=} \PY{p}{\PYZob{}}\PY{l+s+s1}{\PYZsq{}}\PY{l+s+s1}{l}\PY{l+s+s1}{\PYZsq{}}\PY{p}{:}\PY{n}{theta}\PY{p}{[}\PY{l+m+mi}{0}\PY{p}{]}\PY{p}{,} \PY{l+s+s1}{\PYZsq{}}\PY{l+s+s1}{sigma\PYZus{}k}\PY{l+s+s1}{\PYZsq{}}\PY{p}{:}\PY{n}{theta}\PY{p}{[}\PY{l+m+mi}{1}\PY{p}{]}\PY{p}{\PYZcb{}}
%        \PY{n}{K} \PY{o}{=} \PY{n}{GP\PYZus{}kernels}\PY{o}{.}\PY{n}{gauss}\PY{p}{(}\PY{n}{X\PYZus{}train}\PY{p}{,} \PY{n}{X\PYZus{}train}\PY{p}{,} \PY{n}{kernel\PYZus{}args}\PY{p}{)} \PY{o}{+} \PYZbs{}
%            \PY{n}{sigma\PYZus{}n}\PY{o}{*}\PY{o}{*}\PY{l+m+mi}{2} \PY{o}{*} \PY{n}{np}\PY{o}{.}\PY{n}{eye}\PY{p}{(}\PY{n+nb}{len}\PY{p}{(}\PY{n}{X\PYZus{}train}\PY{p}{)}\PY{p}{)}
%        \PY{n}{y} \PY{o}{=} \PY{n}{Y\PYZus{}train}\PY{o}{.}\PY{n}{T} \PY{o}{@} \PY{n}{LA}\PY{o}{.}\PY{n}{inv}\PY{p}{(}\PY{n}{K}\PY{p}{)} \PY{o}{@} \PY{n}{Y\PYZus{}train} \PY{o}{+} \PY{n}{np}\PY{o}{.}\PY{n}{log}\PY{p}{(}\PY{n}{LA}\PY{o}{.}\PY{n}{det}\PY{p}{(}\PY{n}{K}\PY{p}{)}\PY{p}{)}
%        \PY{k}{return} \PY{n}{y}\PY{o}{.}\PY{n}{flatten}\PY{p}{(}\PY{p}{)}
%    
%    \PY{k}{return} \PY{n}{nll}
%\end{Verbatim}
%\end{tcolorbox}
%
%    \begin{tcolorbox}[breakable, size=fbox, boxrule=1pt, pad at break*=1mm,colback=cellbackground, colframe=cellborder]
%\prompt{In}{incolor}{19}{\boxspacing}
%\begin{Verbatim}[commandchars=\\\{\}]
%\PY{n}{sigma\PYZus{}n} \PY{o}{=} \PY{l+m+mf}{0.1}
%\PY{n}{res} \PY{o}{=} \PY{n}{minimize}\PY{p}{(}\PY{n}{nll\PYZus{}fn}\PY{p}{(}\PY{n}{X\PYZus{}train}\PY{p}{,} \PY{n}{Y\PYZus{}train}\PY{p}{,} \PY{n}{sigma\PYZus{}n}\PY{p}{)}\PY{p}{,} \PY{p}{[}\PY{l+m+mi}{1}\PY{p}{,} \PY{l+m+mi}{1}\PY{p}{]}\PY{p}{,}
%               \PY{n}{bounds}\PY{o}{=}\PY{p}{(}\PY{p}{(}\PY{l+m+mf}{1e\PYZhy{}5}\PY{p}{,} \PY{k+kc}{None}\PY{p}{)}\PY{p}{,} \PY{p}{(}\PY{l+m+mf}{1e\PYZhy{}5}\PY{p}{,} \PY{k+kc}{None}\PY{p}{)}\PY{p}{)}\PY{p}{,}
%               \PY{n}{method}\PY{o}{=}\PY{l+s+s1}{\PYZsq{}}\PY{l+s+s1}{L\PYZhy{}BFGS\PYZhy{}B}\PY{l+s+s1}{\PYZsq{}}\PY{p}{)}
%\PY{n}{l\PYZus{}opt}\PY{p}{,} \PY{n}{sigma\PYZus{}k\PYZus{}opt} \PY{o}{=} \PY{n}{res}\PY{o}{.}\PY{n}{x}
%\PY{p}{[}\PY{n}{nll}\PY{p}{]} \PY{o}{=} \PY{n}{res}\PY{o}{.}\PY{n}{fun}
%
%\PY{n}{display}\PY{p}{(}\PY{n}{Markdown}\PY{p}{(}\PY{l+s+sa}{f}\PY{l+s+s1}{\PYZsq{}}\PY{l+s+s1}{\PYZdl{}}\PY{l+s+s1}{\PYZbs{}}\PY{l+s+s1}{ell = }\PY{l+s+si}{\PYZob{}}\PY{n}{l\PYZus{}opt}\PY{l+s+si}{:}\PY{l+s+s1}{.3}\PY{l+s+si}{\PYZcb{}}\PY{l+s+s1}{,}\PY{l+s+s1}{\PYZbs{}}\PY{l+s+s1}{;}\PY{l+s+s1}{\PYZbs{}}\PY{l+s+s1}{sigma\PYZus{}k = }\PY{l+s+si}{\PYZob{}}\PY{n}{sigma\PYZus{}k\PYZus{}opt}\PY{l+s+si}{:}\PY{l+s+s1}{.3}\PY{l+s+si}{\PYZcb{}}\PY{l+s+s1}{,}\PY{l+s+s1}{\PYZbs{}}\PY{l+s+s1}{;}\PY{l+s+se}{\PYZbs{}}
%\PY{l+s+s1}{                    }\PY{l+s+s1}{\PYZbs{}}\PY{l+s+s1}{sigma\PYZus{}n = }\PY{l+s+si}{\PYZob{}}\PY{n}{sigma\PYZus{}n}\PY{l+s+si}{:}\PY{l+s+s1}{.3}\PY{l+s+si}{\PYZcb{}}\PY{l+s+s1}{,}\PY{l+s+s1}{\PYZbs{}}\PY{l+s+s1}{;NLL = }\PY{l+s+si}{\PYZob{}}\PY{n}{nll}\PY{l+s+si}{:}\PY{l+s+s1}{.6}\PY{l+s+si}{\PYZcb{}}\PY{l+s+s1}{\PYZdl{}}\PY{l+s+s1}{\PYZsq{}}\PY{p}{)}\PY{p}{)}
%\end{Verbatim}
%\end{tcolorbox}
%
%    \(\ell = 0.778,\;\sigma_k = 0.852,\; \sigma_n = 0.1,\;NLL = 3.26679\)
%
%    
%    \begin{tcolorbox}[breakable, size=fbox, boxrule=1pt, pad at break*=1mm,colback=cellbackground, colframe=cellborder]
%\prompt{In}{incolor}{20}{\boxspacing}
%\begin{Verbatim}[commandchars=\\\{\}]
%\PY{c+c1}{\PYZsh{} Compute the prosterior predictive statistics with optimized kernel parameters and plot the results}
%\PY{n}{kernel} \PY{o}{=} \PY{n}{GP\PYZus{}kernels}\PY{o}{.}\PY{n}{gauss}
%\PY{n}{kernel\PYZus{}args} \PY{o}{=} \PY{p}{\PYZob{}}\PY{l+s+s1}{\PYZsq{}}\PY{l+s+s1}{l}\PY{l+s+s1}{\PYZsq{}}\PY{p}{:}\PY{n}{l\PYZus{}opt}\PY{p}{,} \PY{l+s+s1}{\PYZsq{}}\PY{l+s+s1}{sigma\PYZus{}k}\PY{l+s+s1}{\PYZsq{}}\PY{p}{:}\PY{n}{sigma\PYZus{}k\PYZus{}opt}\PY{p}{\PYZcb{}}
%\PY{n}{mu\PYZus{}s}\PY{p}{,} \PY{n}{cov\PYZus{}s} \PY{o}{=} \PY{n}{GP\PYZus{}predictor}\PY{p}{(}\PY{n}{X\PYZus{}test}\PY{p}{,} \PY{n}{X\PYZus{}train}\PY{p}{,} \PY{n}{Y\PYZus{}train}\PY{p}{,}
%                           \PY{n}{kernel}\PY{p}{,} \PY{n}{kernel\PYZus{}args}\PY{p}{,} \PY{n}{sigma\PYZus{}n}\PY{p}{)}
%
%\PY{n}{plt}\PY{o}{.}\PY{n}{figure}\PY{p}{(}\PY{n}{figsize}\PY{o}{=}\PY{p}{(}\PY{l+m+mi}{8}\PY{p}{,} \PY{l+m+mi}{5}\PY{p}{)}\PY{p}{)}
%\PY{n}{plt}\PY{o}{.}\PY{n}{title}\PY{p}{(}\PY{l+s+sa}{f}\PY{l+s+s1}{\PYZsq{}}\PY{l+s+s1}{\PYZdl{}}\PY{l+s+s1}{\PYZbs{}}\PY{l+s+s1}{ell = }\PY{l+s+si}{\PYZob{}}\PY{n}{l\PYZus{}opt}\PY{l+s+si}{:}\PY{l+s+s1}{.3}\PY{l+s+si}{\PYZcb{}}\PY{l+s+s1}{,}\PY{l+s+s1}{\PYZbs{}}\PY{l+s+s1}{;}\PY{l+s+s1}{\PYZbs{}}\PY{l+s+s1}{sigma\PYZus{}k = }\PY{l+s+si}{\PYZob{}}\PY{n}{sigma\PYZus{}k\PYZus{}opt}\PY{l+s+si}{:}\PY{l+s+s1}{.3}\PY{l+s+si}{\PYZcb{}}\PY{l+s+s1}{,}\PY{l+s+s1}{\PYZbs{}}\PY{l+s+s1}{;}\PY{l+s+se}{\PYZbs{}}
%\PY{l+s+s1}{             }\PY{l+s+s1}{\PYZbs{}}\PY{l+s+s1}{sigma\PYZus{}n = }\PY{l+s+si}{\PYZob{}}\PY{n}{sigma\PYZus{}n}\PY{l+s+si}{:}\PY{l+s+s1}{.3}\PY{l+s+si}{\PYZcb{}}\PY{l+s+s1}{,}\PY{l+s+s1}{\PYZbs{}}\PY{l+s+s1}{;NLL = }\PY{l+s+si}{\PYZob{}}\PY{n}{nll}\PY{l+s+si}{:}\PY{l+s+s1}{.3}\PY{l+s+si}{\PYZcb{}}\PY{l+s+s1}{\PYZdl{}}\PY{l+s+s1}{\PYZsq{}}\PY{p}{)}
%\PY{n}{plot\PYZus{}GP}\PY{p}{(}\PY{n}{X\PYZus{}test}\PY{p}{,} \PY{n}{mu\PYZus{}s}\PY{p}{,} \PY{n}{cov\PYZus{}s}\PY{p}{,} \PY{n}{X\PYZus{}train}\PY{p}{,} \PY{n}{Y\PYZus{}train}\PY{p}{,} \PY{n}{draw\PYZus{}ci}\PY{o}{=}\PY{k+kc}{True}\PY{p}{)}
%\PY{n}{plt}\PY{o}{.}\PY{n}{tight\PYZus{}layout}\PY{p}{(}\PY{p}{)}
%\PY{n}{plt}\PY{o}{.}\PY{n}{show}\PY{p}{(}\PY{p}{)}
%\end{Verbatim}
%\end{tcolorbox}

    \begin{center}
    \adjustimage{max size={0.65\linewidth}{0.65\paperheight}}{HP_opt_noise.pdf}
    \end{center}
%    { \hspace*{\fill} \\}

    \hypertarget{ux434ux432ux443ux43cux435ux440ux43dux44bux439-ux441ux43bux443ux447ux430ux439}{%
\subsection{Двумерный
случай}\label{ux434ux432ux443ux43cux435ux440ux43dux44bux439-ux441ux43bux443ux447ux430ux439}}

%    \begin{tcolorbox}[breakable, size=fbox, boxrule=1pt, pad at break*=1mm,colback=cellbackground, colframe=cellborder]
%\prompt{In}{incolor}{21}{\boxspacing}
%\begin{Verbatim}[commandchars=\\\{\}]
%\PY{k+kn}{from} \PY{n+nn}{mpl\PYZus{}toolkits}\PY{n+nn}{.}\PY{n+nn}{mplot3d} \PY{k+kn}{import} \PY{n}{Axes3D}
%
%\PY{k}{def} \PY{n+nf}{plot\PYZus{}gp\PYZus{}2D}\PY{p}{(}\PY{n}{ax}\PY{p}{,} \PY{n}{gx}\PY{p}{,} \PY{n}{gy}\PY{p}{,} \PY{n}{mu}\PY{p}{,} \PY{n}{X\PYZus{}train}\PY{p}{,} \PY{n}{Y\PYZus{}train}\PY{p}{,} \PY{n}{title}\PY{p}{)}\PY{p}{:}
%    \PY{n}{ax}\PY{o}{.}\PY{n}{plot\PYZus{}surface}\PY{p}{(}\PY{n}{gx}\PY{p}{,} \PY{n}{gy}\PY{p}{,} \PY{n}{mu}\PY{o}{.}\PY{n}{reshape}\PY{p}{(}\PY{n}{gx}\PY{o}{.}\PY{n}{shape}\PY{p}{)}\PY{p}{,} \PY{n}{cmap}\PY{o}{=}\PY{n}{cm}\PY{o}{.}\PY{n}{coolwarm}\PY{p}{,}
%                    \PY{n}{linewidth}\PY{o}{=}\PY{l+m+mi}{0}\PY{p}{,} \PY{n}{alpha}\PY{o}{=}\PY{l+m+mf}{0.2}\PY{p}{,} \PY{n}{antialiased}\PY{o}{=}\PY{k+kc}{False}\PY{p}{)}
%    \PY{n}{ax}\PY{o}{.}\PY{n}{scatter}\PY{p}{(}\PY{n}{X\PYZus{}train}\PY{p}{[}\PY{p}{:}\PY{p}{,}\PY{l+m+mi}{0}\PY{p}{]}\PY{p}{,} \PY{n}{X\PYZus{}train}\PY{p}{[}\PY{p}{:}\PY{p}{,}\PY{l+m+mi}{1}\PY{p}{]}\PY{p}{,} \PY{n}{Y\PYZus{}train}\PY{p}{,} \PY{n}{c}\PY{o}{=}\PY{n}{Y\PYZus{}train}\PY{p}{,} \PY{n}{cmap}\PY{o}{=}\PY{n}{cm}\PY{o}{.}\PY{n}{coolwarm}\PY{p}{)}
%    \PY{n}{ax}\PY{o}{.}\PY{n}{set\PYZus{}title}\PY{p}{(}\PY{n}{title}\PY{p}{)}
%\end{Verbatim}
%\end{tcolorbox}
%
%    \begin{tcolorbox}[breakable, size=fbox, boxrule=1pt, pad at break*=1mm,colback=cellbackground, colframe=cellborder]
%\prompt{In}{incolor}{22}{\boxspacing}
%\begin{Verbatim}[commandchars=\\\{\}]
%\PY{n}{N\PYZus{}train} \PY{o}{=} \PY{l+m+mi}{100}
%\PY{n}{noise\PYZus{}2D\PYZus{}train} \PY{o}{=} \PY{l+m+mf}{0.05}
%\PY{n}{sigma\PYZus{}n} \PY{o}{=} \PY{l+m+mf}{0.05}
%
%\PY{n}{rx}\PY{p}{,} \PY{n}{ry} \PY{o}{=} \PY{n}{np}\PY{o}{.}\PY{n}{arange}\PY{p}{(}\PY{o}{\PYZhy{}}\PY{l+m+mi}{5}\PY{p}{,} \PY{l+m+mi}{5}\PY{p}{,} \PY{l+m+mf}{0.2}\PY{p}{)}\PY{p}{,} \PY{n}{np}\PY{o}{.}\PY{n}{arange}\PY{p}{(}\PY{o}{\PYZhy{}}\PY{l+m+mi}{5}\PY{p}{,} \PY{l+m+mi}{5}\PY{p}{,} \PY{l+m+mf}{0.2}\PY{p}{)}
%\PY{n}{gx}\PY{p}{,} \PY{n}{gy} \PY{o}{=} \PY{n}{np}\PY{o}{.}\PY{n}{meshgrid}\PY{p}{(}\PY{n}{rx}\PY{p}{,} \PY{n}{rx}\PY{p}{)}
%
%\PY{n}{X\PYZus{}2D} \PY{o}{=} \PY{n}{np}\PY{o}{.}\PY{n}{c\PYZus{}}\PY{p}{[}\PY{n}{gx}\PY{o}{.}\PY{n}{ravel}\PY{p}{(}\PY{p}{)}\PY{p}{,} \PY{n}{gy}\PY{o}{.}\PY{n}{ravel}\PY{p}{(}\PY{p}{)}\PY{p}{]}
%
%\PY{n}{X\PYZus{}2D\PYZus{}train} \PY{o}{=} \PY{n}{np}\PY{o}{.}\PY{n}{random}\PY{o}{.}\PY{n}{uniform}\PY{p}{(}\PY{o}{\PYZhy{}}\PY{l+m+mi}{4}\PY{p}{,} \PY{l+m+mi}{4}\PY{p}{,} \PY{p}{(}\PY{n}{N\PYZus{}train}\PY{p}{,} \PY{l+m+mi}{2}\PY{p}{)}\PY{p}{)}
%\PY{n}{Y\PYZus{}2D\PYZus{}train} \PY{o}{=} \PY{n}{np}\PY{o}{.}\PY{n}{sin}\PY{p}{(}\PY{l+m+mf}{0.5} \PY{o}{*} \PY{n}{np}\PY{o}{.}\PY{n}{linalg}\PY{o}{.}\PY{n}{norm}\PY{p}{(}\PY{n}{X\PYZus{}2D\PYZus{}train}\PY{p}{,} \PY{n}{axis}\PY{o}{=}\PY{l+m+mi}{1}\PY{p}{)}\PY{p}{)} \PY{o}{+} \PYZbs{}
%             \PY{n}{noise\PYZus{}2D\PYZus{}train} \PY{o}{*} \PY{n}{np}\PY{o}{.}\PY{n}{random}\PY{o}{.}\PY{n}{randn}\PY{p}{(}\PY{n+nb}{len}\PY{p}{(}\PY{n}{X\PYZus{}2D\PYZus{}train}\PY{p}{)}\PY{p}{)}
%\end{Verbatim}
%\end{tcolorbox}
%
%    \begin{tcolorbox}[breakable, size=fbox, boxrule=1pt, pad at break*=1mm,colback=cellbackground, colframe=cellborder]
%\prompt{In}{incolor}{23}{\boxspacing}
%\begin{Verbatim}[commandchars=\\\{\}]
%\PY{n}{kernel} \PY{o}{=} \PY{n}{GP\PYZus{}kernels}\PY{o}{.}\PY{n}{gauss}
%\PY{n}{kernel\PYZus{}args} \PY{o}{=} \PY{p}{\PYZob{}}\PY{l+s+s1}{\PYZsq{}}\PY{l+s+s1}{l}\PY{l+s+s1}{\PYZsq{}}\PY{p}{:}\PY{l+m+mf}{1.}\PY{p}{,} \PY{l+s+s1}{\PYZsq{}}\PY{l+s+s1}{sigma\PYZus{}k}\PY{l+s+s1}{\PYZsq{}}\PY{p}{:}\PY{l+m+mf}{1.}\PY{p}{\PYZcb{}}
%\PY{n}{mu}\PY{p}{,} \PY{n}{\PYZus{}} \PY{o}{=} \PY{n}{GP\PYZus{}predictor}\PY{p}{(}\PY{n}{X\PYZus{}2D}\PY{p}{,} \PY{n}{X\PYZus{}2D\PYZus{}train}\PY{p}{,} \PY{n}{Y\PYZus{}2D\PYZus{}train}\PY{p}{,}
%                       \PY{n}{kernel}\PY{p}{,} \PY{n}{kernel\PYZus{}args}\PY{p}{,} \PY{n}{sigma\PYZus{}n}\PY{p}{)}
%\end{Verbatim}
%\end{tcolorbox}
%
%    \begin{tcolorbox}[breakable, size=fbox, boxrule=1pt, pad at break*=1mm,colback=cellbackground, colframe=cellborder]
%\prompt{In}{incolor}{24}{\boxspacing}
%\begin{Verbatim}[commandchars=\\\{\}]
%\PY{n}{res} \PY{o}{=} \PY{n}{minimize}\PY{p}{(}\PY{n}{nll\PYZus{}fn}\PY{p}{(}\PY{n}{X\PYZus{}2D\PYZus{}train}\PY{p}{,} \PY{n}{Y\PYZus{}2D\PYZus{}train}\PY{p}{,} \PY{n}{sigma\PYZus{}n}\PY{p}{)}\PY{p}{,} \PY{p}{[}\PY{l+m+mi}{1}\PY{p}{,} \PY{l+m+mi}{1}\PY{p}{]}\PY{p}{,} 
%               \PY{n}{bounds}\PY{o}{=}\PY{p}{(}\PY{p}{(}\PY{l+m+mf}{1e\PYZhy{}5}\PY{p}{,} \PY{k+kc}{None}\PY{p}{)}\PY{p}{,} \PY{p}{(}\PY{l+m+mf}{1e\PYZhy{}5}\PY{p}{,} \PY{k+kc}{None}\PY{p}{)}\PY{p}{)}\PY{p}{,}
%               \PY{n}{method}\PY{o}{=}\PY{l+s+s1}{\PYZsq{}}\PY{l+s+s1}{L\PYZhy{}BFGS\PYZhy{}B}\PY{l+s+s1}{\PYZsq{}}\PY{p}{)}
%\PY{n}{l\PYZus{}opt}\PY{p}{,} \PY{n}{sigma\PYZus{}k\PYZus{}opt} \PY{o}{=} \PY{n}{res}\PY{o}{.}\PY{n}{x}
%\PY{n}{kernel\PYZus{}args} \PY{o}{=} \PY{p}{\PYZob{}}\PY{l+s+s1}{\PYZsq{}}\PY{l+s+s1}{l}\PY{l+s+s1}{\PYZsq{}}\PY{p}{:}\PY{n}{l\PYZus{}opt}\PY{p}{,} \PY{l+s+s1}{\PYZsq{}}\PY{l+s+s1}{sigma\PYZus{}k}\PY{l+s+s1}{\PYZsq{}}\PY{p}{:}\PY{n}{sigma\PYZus{}k\PYZus{}opt}\PY{p}{\PYZcb{}}
%\PY{n}{mu\PYZus{}opt}\PY{p}{,} \PY{n}{\PYZus{}} \PY{o}{=} \PY{n}{GP\PYZus{}predictor}\PY{p}{(}\PY{n}{X\PYZus{}2D}\PY{p}{,} \PY{n}{X\PYZus{}2D\PYZus{}train}\PY{p}{,} \PY{n}{Y\PYZus{}2D\PYZus{}train}\PY{p}{,}
%                       \PY{n}{kernel}\PY{p}{,} \PY{n}{kernel\PYZus{}args}\PY{p}{,} \PY{n}{sigma\PYZus{}n}\PY{p}{)}
%\end{Verbatim}
%\end{tcolorbox}
%
%    \begin{tcolorbox}[breakable, size=fbox, boxrule=1pt, pad at break*=1mm,colback=cellbackground, colframe=cellborder]
%\prompt{In}{incolor}{25}{\boxspacing}
%\begin{Verbatim}[commandchars=\\\{\}]
%\PY{n}{fig} \PY{o}{=} \PY{n}{plt}\PY{o}{.}\PY{n}{figure}\PY{p}{(}\PY{n}{figsize}\PY{o}{=}\PY{p}{(}\PY{l+m+mi}{14}\PY{p}{,}\PY{l+m+mi}{7}\PY{p}{)}\PY{p}{)}
%
%\PY{n}{ax1} \PY{o}{=} \PY{n}{fig}\PY{o}{.}\PY{n}{add\PYZus{}subplot}\PY{p}{(}\PY{l+m+mi}{121}\PY{p}{,} \PY{n}{projection}\PY{o}{=}\PY{l+s+s1}{\PYZsq{}}\PY{l+s+s1}{3d}\PY{l+s+s1}{\PYZsq{}}\PY{p}{)}
%\PY{n}{plot\PYZus{}gp\PYZus{}2D}\PY{p}{(}\PY{n}{ax1}\PY{p}{,} \PY{n}{gx}\PY{p}{,} \PY{n}{gy}\PY{p}{,} \PY{n}{mu}\PY{p}{,} \PY{n}{X\PYZus{}2D\PYZus{}train}\PY{p}{,} \PY{n}{Y\PYZus{}2D\PYZus{}train}\PY{p}{,} 
%           \PY{l+s+sa}{f}\PY{l+s+s1}{\PYZsq{}}\PY{l+s+s1}{До оптимизации: \PYZdl{}l=}\PY{l+s+si}{\PYZob{}}\PY{l+m+mf}{1.00}\PY{l+s+si}{\PYZcb{}}\PY{l+s+s1}{,}\PY{l+s+s1}{\PYZbs{}}\PY{l+s+s1}{;}\PY{l+s+s1}{\PYZbs{}}\PY{l+s+s1}{sigma\PYZus{}k=}\PY{l+s+si}{\PYZob{}}\PY{l+m+mf}{1.00}\PY{l+s+si}{\PYZcb{}}\PY{l+s+s1}{\PYZdl{}}\PY{l+s+s1}{\PYZsq{}}\PY{p}{)}
%
%\PY{n}{ax2} \PY{o}{=} \PY{n}{fig}\PY{o}{.}\PY{n}{add\PYZus{}subplot}\PY{p}{(}\PY{l+m+mi}{122}\PY{p}{,} \PY{n}{projection}\PY{o}{=}\PY{l+s+s1}{\PYZsq{}}\PY{l+s+s1}{3d}\PY{l+s+s1}{\PYZsq{}}\PY{p}{)}
%\PY{n}{plot\PYZus{}gp\PYZus{}2D}\PY{p}{(}\PY{n}{ax2}\PY{p}{,} \PY{n}{gx}\PY{p}{,} \PY{n}{gy}\PY{p}{,} \PY{n}{mu\PYZus{}opt}\PY{p}{,} \PY{n}{X\PYZus{}2D\PYZus{}train}\PY{p}{,} \PY{n}{Y\PYZus{}2D\PYZus{}train}\PY{p}{,}
%           \PY{l+s+sa}{f}\PY{l+s+s1}{\PYZsq{}}\PY{l+s+s1}{После оптимизации: \PYZdl{}l=}\PY{l+s+si}{\PYZob{}}\PY{n}{res}\PY{o}{.}\PY{n}{x}\PY{p}{[}\PY{l+m+mi}{0}\PY{p}{]}\PY{l+s+si}{:}\PY{l+s+s1}{.2f}\PY{l+s+si}{\PYZcb{}}\PY{l+s+s1}{,}\PY{l+s+s1}{\PYZbs{}}\PY{l+s+s1}{;}\PY{l+s+s1}{\PYZbs{}}\PY{l+s+s1}{sigma\PYZus{}k=}\PY{l+s+si}{\PYZob{}}\PY{n}{res}\PY{o}{.}\PY{n}{x}\PY{p}{[}\PY{l+m+mi}{1}\PY{p}{]}\PY{l+s+si}{:}\PY{l+s+s1}{.2f}\PY{l+s+si}{\PYZcb{}}\PY{l+s+s1}{\PYZdl{}}\PY{l+s+s1}{\PYZsq{}}\PY{p}{)}
%\end{Verbatim}
%\end{tcolorbox}

    \begin{center}
    \adjustimage{max size={1.0\linewidth}{1.0\paperheight}}{HP_opt_2D.pdf}
    \end{center}
%    { \hspace*{\fill} \\}

    \begin{center}\rule{0.5\linewidth}{0.5pt}\end{center}

    \hypertarget{ux438ux441ux442ux43eux447ux43dux438ux43aux438}{%
\section{Источники}\label{ux438ux441ux442ux43eux447ux43dux438ux43aux438}}

\begin{enumerate}
\def\labelenumi{\arabic{enumi}.}
\tightlist
\item
  \emph{Roelants P.}
  \href{https://peterroelants.github.io/posts/gaussian-process-tutorial/}{Understanding
  Gaussian processes}.
\item
  \emph{Krasser M.}
  \href{http://krasserm.github.io/2018/03/19/gaussian-processes/}{Gaussian
  processes}.
\end{enumerate}

%    \begin{tcolorbox}[breakable, size=fbox, boxrule=1pt, pad at break*=1mm,colback=cellbackground, colframe=cellborder]
%\prompt{In}{incolor}{26}{\boxspacing}
%\begin{Verbatim}[commandchars=\\\{\}]
%\PY{c+c1}{\PYZsh{} Versions used}
%\PY{n+nb}{print}\PY{p}{(}\PY{l+s+s1}{\PYZsq{}}\PY{l+s+s1}{Python: }\PY{l+s+si}{\PYZob{}\PYZcb{}}\PY{l+s+s1}{.}\PY{l+s+si}{\PYZob{}\PYZcb{}}\PY{l+s+s1}{.}\PY{l+s+si}{\PYZob{}\PYZcb{}}\PY{l+s+s1}{\PYZsq{}}\PY{o}{.}\PY{n}{format}\PY{p}{(}\PY{o}{*}\PY{n}{sys}\PY{o}{.}\PY{n}{version\PYZus{}info}\PY{p}{[}\PY{p}{:}\PY{l+m+mi}{3}\PY{p}{]}\PY{p}{)}\PY{p}{)}
%\PY{n+nb}{print}\PY{p}{(}\PY{l+s+s1}{\PYZsq{}}\PY{l+s+s1}{numpy: }\PY{l+s+si}{\PYZob{}\PYZcb{}}\PY{l+s+s1}{\PYZsq{}}\PY{o}{.}\PY{n}{format}\PY{p}{(}\PY{n}{np}\PY{o}{.}\PY{n}{\PYZus{}\PYZus{}version\PYZus{}\PYZus{}}\PY{p}{)}\PY{p}{)}
%\PY{n+nb}{print}\PY{p}{(}\PY{l+s+s1}{\PYZsq{}}\PY{l+s+s1}{matplotlib: }\PY{l+s+si}{\PYZob{}\PYZcb{}}\PY{l+s+s1}{\PYZsq{}}\PY{o}{.}\PY{n}{format}\PY{p}{(}\PY{n}{matplotlib}\PY{o}{.}\PY{n}{\PYZus{}\PYZus{}version\PYZus{}\PYZus{}}\PY{p}{)}\PY{p}{)}
%\PY{n+nb}{print}\PY{p}{(}\PY{l+s+s1}{\PYZsq{}}\PY{l+s+s1}{seaborn: }\PY{l+s+si}{\PYZob{}\PYZcb{}}\PY{l+s+s1}{\PYZsq{}}\PY{o}{.}\PY{n}{format}\PY{p}{(}\PY{n}{seaborn}\PY{o}{.}\PY{n}{\PYZus{}\PYZus{}version\PYZus{}\PYZus{}}\PY{p}{)}\PY{p}{)}
%\end{Verbatim}
%\end{tcolorbox}
%
%    \begin{Verbatim}[commandchars=\\\{\}]
%Python: 3.7.16
%numpy: 1.20.3
%matplotlib: 3.5.1
%seaborn: 0.12.2
%    \end{Verbatim}


    % Add a bibliography block to the postdoc
    
    
    
\end{document}
