\documentclass[11pt,a4paper]{article}

    \usepackage[breakable]{tcolorbox}
    \usepackage{parskip} % Stop auto-indenting (to mimic markdown behaviour)
    
    \usepackage{iftex}
    \ifPDFTeX
      \usepackage[T2A]{fontenc}
      \usepackage{mathpazo}
      \usepackage[russian,english]{babel}
    \else
      \usepackage{fontspec}
      \usepackage{polyglossia}
      \setmainlanguage[babelshorthands=true]{russian}    % Язык по-умолчанию русский с поддержкой приятных команд пакета babel
      \setotherlanguage{english}                         % Дополнительный язык = английский (в американской вариации по-умолчанию)
      \newfontfamily\cyrillicfonttt[Scale=0.87,BoldFont={Fira Mono Medium}] {Fira Mono}  % Моноширинный шрифт для кириллицы
      \defaultfontfeatures{Ligatures=TeX}
      \newfontfamily\cyrillicfont{STIX Two Text}         % Шрифт с засечками для кириллицы
    \fi
    \renewcommand{\linethickness}{0.1ex}

    % Basic figure setup, for now with no caption control since it's done
    % automatically by Pandoc (which extracts ![](path) syntax from Markdown).
    \usepackage{graphicx}
    % Maintain compatibility with old templates. Remove in nbconvert 6.0
    \let\Oldincludegraphics\includegraphics
    % Ensure that by default, figures have no caption (until we provide a
    % proper Figure object with a Caption API and a way to capture that
    % in the conversion process - todo).
    \usepackage{caption}
    \DeclareCaptionFormat{nocaption}{}
    \captionsetup{format=nocaption,aboveskip=0pt,belowskip=0pt}

    \usepackage{float}
    \floatplacement{figure}{H} % forces figures to be placed at the correct location
    \usepackage{xcolor} % Allow colors to be defined
    \usepackage{enumerate} % Needed for markdown enumerations to work
    \usepackage{geometry} % Used to adjust the document margins
    \usepackage{amsmath} % Equations
    \usepackage{amssymb} % Equations
    \usepackage{textcomp} % defines textquotesingle
    % Hack from http://tex.stackexchange.com/a/47451/13684:
    \AtBeginDocument{%
        \def\PYZsq{\textquotesingle}% Upright quotes in Pygmentized code
    }
    \usepackage{upquote} % Upright quotes for verbatim code
    \usepackage{eurosym} % defines \euro
    \usepackage[mathletters]{ucs} % Extended unicode (utf-8) support
    \usepackage{fancyvrb} % verbatim replacement that allows latex
    \usepackage{grffile} % extends the file name processing of package graphics 
                         % to support a larger range
    \makeatletter % fix for old versions of grffile with XeLaTeX
    \@ifpackagelater{grffile}{2019/11/01}
    {
      % Do nothing on new versions
    }
    {
      \def\Gread@@xetex#1{%
        \IfFileExists{"\Gin@base".bb}%
        {\Gread@eps{\Gin@base.bb}}%
        {\Gread@@xetex@aux#1}%
      }
    }
    \makeatother
    \usepackage[Export]{adjustbox} % Used to constrain images to a maximum size
    \adjustboxset{max size={0.9\linewidth}{0.9\paperheight}}

    % The hyperref package gives us a pdf with properly built
    % internal navigation ('pdf bookmarks' for the table of contents,
    % internal cross-reference links, web links for URLs, etc.)
    \usepackage{hyperref}
    % The default LaTeX title has an obnoxious amount of whitespace. By default,
    % titling removes some of it. It also provides customization options.
    \usepackage{titling}
    \usepackage{longtable} % longtable support required by pandoc >1.10
    \usepackage{booktabs}  % table support for pandoc > 1.12.2
    \usepackage[inline]{enumitem} % IRkernel/repr support (it uses the enumerate* environment)
    \usepackage[normalem]{ulem} % ulem is needed to support strikethroughs (\sout)
                                % normalem makes italics be italics, not underlines
    \usepackage{mathrsfs}
    

    
    % Colors for the hyperref package
    \definecolor{urlcolor}{rgb}{0,.145,.698}
    \definecolor{linkcolor}{rgb}{.71,0.21,0.01}
    \definecolor{citecolor}{rgb}{.12,.54,.11}

    % ANSI colors
    \definecolor{ansi-black}{HTML}{3E424D}
    \definecolor{ansi-black-intense}{HTML}{282C36}
    \definecolor{ansi-red}{HTML}{E75C58}
    \definecolor{ansi-red-intense}{HTML}{B22B31}
    \definecolor{ansi-green}{HTML}{00A250}
    \definecolor{ansi-green-intense}{HTML}{007427}
    \definecolor{ansi-yellow}{HTML}{DDB62B}
    \definecolor{ansi-yellow-intense}{HTML}{B27D12}
    \definecolor{ansi-blue}{HTML}{208FFB}
    \definecolor{ansi-blue-intense}{HTML}{0065CA}
    \definecolor{ansi-magenta}{HTML}{D160C4}
    \definecolor{ansi-magenta-intense}{HTML}{A03196}
    \definecolor{ansi-cyan}{HTML}{60C6C8}
    \definecolor{ansi-cyan-intense}{HTML}{258F8F}
    \definecolor{ansi-white}{HTML}{C5C1B4}
    \definecolor{ansi-white-intense}{HTML}{A1A6B2}
    \definecolor{ansi-default-inverse-fg}{HTML}{FFFFFF}
    \definecolor{ansi-default-inverse-bg}{HTML}{000000}

    % common color for the border for error outputs.
    \definecolor{outerrorbackground}{HTML}{FFDFDF}

    % commands and environments needed by pandoc snippets
    % extracted from the output of `pandoc -s`
    \providecommand{\tightlist}{%
      \setlength{\itemsep}{0pt}\setlength{\parskip}{0pt}}
    \DefineVerbatimEnvironment{Highlighting}{Verbatim}{commandchars=\\\{\}}
    % Add ',fontsize=\small' for more characters per line
    \newenvironment{Shaded}{}{}
    \newcommand{\KeywordTok}[1]{\textcolor[rgb]{0.00,0.44,0.13}{\textbf{{#1}}}}
    \newcommand{\DataTypeTok}[1]{\textcolor[rgb]{0.56,0.13,0.00}{{#1}}}
    \newcommand{\DecValTok}[1]{\textcolor[rgb]{0.25,0.63,0.44}{{#1}}}
    \newcommand{\BaseNTok}[1]{\textcolor[rgb]{0.25,0.63,0.44}{{#1}}}
    \newcommand{\FloatTok}[1]{\textcolor[rgb]{0.25,0.63,0.44}{{#1}}}
    \newcommand{\CharTok}[1]{\textcolor[rgb]{0.25,0.44,0.63}{{#1}}}
    \newcommand{\StringTok}[1]{\textcolor[rgb]{0.25,0.44,0.63}{{#1}}}
    \newcommand{\CommentTok}[1]{\textcolor[rgb]{0.38,0.63,0.69}{\textit{{#1}}}}
    \newcommand{\OtherTok}[1]{\textcolor[rgb]{0.00,0.44,0.13}{{#1}}}
    \newcommand{\AlertTok}[1]{\textcolor[rgb]{1.00,0.00,0.00}{\textbf{{#1}}}}
    \newcommand{\FunctionTok}[1]{\textcolor[rgb]{0.02,0.16,0.49}{{#1}}}
    \newcommand{\RegionMarkerTok}[1]{{#1}}
    \newcommand{\ErrorTok}[1]{\textcolor[rgb]{1.00,0.00,0.00}{\textbf{{#1}}}}
    \newcommand{\NormalTok}[1]{{#1}}
    
    % Additional commands for more recent versions of Pandoc
    \newcommand{\ConstantTok}[1]{\textcolor[rgb]{0.53,0.00,0.00}{{#1}}}
    \newcommand{\SpecialCharTok}[1]{\textcolor[rgb]{0.25,0.44,0.63}{{#1}}}
    \newcommand{\VerbatimStringTok}[1]{\textcolor[rgb]{0.25,0.44,0.63}{{#1}}}
    \newcommand{\SpecialStringTok}[1]{\textcolor[rgb]{0.73,0.40,0.53}{{#1}}}
    \newcommand{\ImportTok}[1]{{#1}}
    \newcommand{\DocumentationTok}[1]{\textcolor[rgb]{0.73,0.13,0.13}{\textit{{#1}}}}
    \newcommand{\AnnotationTok}[1]{\textcolor[rgb]{0.38,0.63,0.69}{\textbf{\textit{{#1}}}}}
    \newcommand{\CommentVarTok}[1]{\textcolor[rgb]{0.38,0.63,0.69}{\textbf{\textit{{#1}}}}}
    \newcommand{\VariableTok}[1]{\textcolor[rgb]{0.10,0.09,0.49}{{#1}}}
    \newcommand{\ControlFlowTok}[1]{\textcolor[rgb]{0.00,0.44,0.13}{\textbf{{#1}}}}
    \newcommand{\OperatorTok}[1]{\textcolor[rgb]{0.40,0.40,0.40}{{#1}}}
    \newcommand{\BuiltInTok}[1]{{#1}}
    \newcommand{\ExtensionTok}[1]{{#1}}
    \newcommand{\PreprocessorTok}[1]{\textcolor[rgb]{0.74,0.48,0.00}{{#1}}}
    \newcommand{\AttributeTok}[1]{\textcolor[rgb]{0.49,0.56,0.16}{{#1}}}
    \newcommand{\InformationTok}[1]{\textcolor[rgb]{0.38,0.63,0.69}{\textbf{\textit{{#1}}}}}
    \newcommand{\WarningTok}[1]{\textcolor[rgb]{0.38,0.63,0.69}{\textbf{\textit{{#1}}}}}
    
    
    % Define a nice break command that doesn't care if a line doesn't already
    % exist.
    \def\br{\hspace*{\fill} \\* }
    % Math Jax compatibility definitions
    \def\gt{>}
    \def\lt{<}
    \let\Oldtex\TeX
    \let\Oldlatex\LaTeX
    \renewcommand{\TeX}{\textrm{\Oldtex}}
    \renewcommand{\LaTeX}{\textrm{\Oldlatex}}
    % Document parameters
    % Document title
    \title{
      {\Large Лекция 5} \\
      Распределение Гаусса
    }
    \date{16 марта 2022\,г.}
    
    
    
% Pygments definitions
\makeatletter
\def\PY@reset{\let\PY@it=\relax \let\PY@bf=\relax%
    \let\PY@ul=\relax \let\PY@tc=\relax%
    \let\PY@bc=\relax \let\PY@ff=\relax}
\def\PY@tok#1{\csname PY@tok@#1\endcsname}
\def\PY@toks#1+{\ifx\relax#1\empty\else%
    \PY@tok{#1}\expandafter\PY@toks\fi}
\def\PY@do#1{\PY@bc{\PY@tc{\PY@ul{%
    \PY@it{\PY@bf{\PY@ff{#1}}}}}}}
\def\PY#1#2{\PY@reset\PY@toks#1+\relax+\PY@do{#2}}

\@namedef{PY@tok@w}{\def\PY@tc##1{\textcolor[rgb]{0.73,0.73,0.73}{##1}}}
\@namedef{PY@tok@c}{\let\PY@it=\textit\def\PY@tc##1{\textcolor[rgb]{0.24,0.48,0.48}{##1}}}
\@namedef{PY@tok@cp}{\def\PY@tc##1{\textcolor[rgb]{0.61,0.40,0.00}{##1}}}
\@namedef{PY@tok@k}{\let\PY@bf=\textbf\def\PY@tc##1{\textcolor[rgb]{0.00,0.50,0.00}{##1}}}
\@namedef{PY@tok@kp}{\def\PY@tc##1{\textcolor[rgb]{0.00,0.50,0.00}{##1}}}
\@namedef{PY@tok@kt}{\def\PY@tc##1{\textcolor[rgb]{0.69,0.00,0.25}{##1}}}
\@namedef{PY@tok@o}{\def\PY@tc##1{\textcolor[rgb]{0.40,0.40,0.40}{##1}}}
\@namedef{PY@tok@ow}{\let\PY@bf=\textbf\def\PY@tc##1{\textcolor[rgb]{0.67,0.13,1.00}{##1}}}
\@namedef{PY@tok@nb}{\def\PY@tc##1{\textcolor[rgb]{0.00,0.50,0.00}{##1}}}
\@namedef{PY@tok@nf}{\def\PY@tc##1{\textcolor[rgb]{0.00,0.00,1.00}{##1}}}
\@namedef{PY@tok@nc}{\let\PY@bf=\textbf\def\PY@tc##1{\textcolor[rgb]{0.00,0.00,1.00}{##1}}}
\@namedef{PY@tok@nn}{\let\PY@bf=\textbf\def\PY@tc##1{\textcolor[rgb]{0.00,0.00,1.00}{##1}}}
\@namedef{PY@tok@ne}{\let\PY@bf=\textbf\def\PY@tc##1{\textcolor[rgb]{0.80,0.25,0.22}{##1}}}
\@namedef{PY@tok@nv}{\def\PY@tc##1{\textcolor[rgb]{0.10,0.09,0.49}{##1}}}
\@namedef{PY@tok@no}{\def\PY@tc##1{\textcolor[rgb]{0.53,0.00,0.00}{##1}}}
\@namedef{PY@tok@nl}{\def\PY@tc##1{\textcolor[rgb]{0.46,0.46,0.00}{##1}}}
\@namedef{PY@tok@ni}{\let\PY@bf=\textbf\def\PY@tc##1{\textcolor[rgb]{0.44,0.44,0.44}{##1}}}
\@namedef{PY@tok@na}{\def\PY@tc##1{\textcolor[rgb]{0.41,0.47,0.13}{##1}}}
\@namedef{PY@tok@nt}{\let\PY@bf=\textbf\def\PY@tc##1{\textcolor[rgb]{0.00,0.50,0.00}{##1}}}
\@namedef{PY@tok@nd}{\def\PY@tc##1{\textcolor[rgb]{0.67,0.13,1.00}{##1}}}
\@namedef{PY@tok@s}{\def\PY@tc##1{\textcolor[rgb]{0.73,0.13,0.13}{##1}}}
\@namedef{PY@tok@sd}{\let\PY@it=\textit\def\PY@tc##1{\textcolor[rgb]{0.73,0.13,0.13}{##1}}}
\@namedef{PY@tok@si}{\let\PY@bf=\textbf\def\PY@tc##1{\textcolor[rgb]{0.64,0.35,0.47}{##1}}}
\@namedef{PY@tok@se}{\let\PY@bf=\textbf\def\PY@tc##1{\textcolor[rgb]{0.67,0.36,0.12}{##1}}}
\@namedef{PY@tok@sr}{\def\PY@tc##1{\textcolor[rgb]{0.64,0.35,0.47}{##1}}}
\@namedef{PY@tok@ss}{\def\PY@tc##1{\textcolor[rgb]{0.10,0.09,0.49}{##1}}}
\@namedef{PY@tok@sx}{\def\PY@tc##1{\textcolor[rgb]{0.00,0.50,0.00}{##1}}}
\@namedef{PY@tok@m}{\def\PY@tc##1{\textcolor[rgb]{0.40,0.40,0.40}{##1}}}
\@namedef{PY@tok@gh}{\let\PY@bf=\textbf\def\PY@tc##1{\textcolor[rgb]{0.00,0.00,0.50}{##1}}}
\@namedef{PY@tok@gu}{\let\PY@bf=\textbf\def\PY@tc##1{\textcolor[rgb]{0.50,0.00,0.50}{##1}}}
\@namedef{PY@tok@gd}{\def\PY@tc##1{\textcolor[rgb]{0.63,0.00,0.00}{##1}}}
\@namedef{PY@tok@gi}{\def\PY@tc##1{\textcolor[rgb]{0.00,0.52,0.00}{##1}}}
\@namedef{PY@tok@gr}{\def\PY@tc##1{\textcolor[rgb]{0.89,0.00,0.00}{##1}}}
\@namedef{PY@tok@ge}{\let\PY@it=\textit}
\@namedef{PY@tok@gs}{\let\PY@bf=\textbf}
\@namedef{PY@tok@gp}{\let\PY@bf=\textbf\def\PY@tc##1{\textcolor[rgb]{0.00,0.00,0.50}{##1}}}
\@namedef{PY@tok@go}{\def\PY@tc##1{\textcolor[rgb]{0.44,0.44,0.44}{##1}}}
\@namedef{PY@tok@gt}{\def\PY@tc##1{\textcolor[rgb]{0.00,0.27,0.87}{##1}}}
\@namedef{PY@tok@err}{\def\PY@bc##1{{\setlength{\fboxsep}{\string -\fboxrule}\fcolorbox[rgb]{1.00,0.00,0.00}{1,1,1}{\strut ##1}}}}
\@namedef{PY@tok@kc}{\let\PY@bf=\textbf\def\PY@tc##1{\textcolor[rgb]{0.00,0.50,0.00}{##1}}}
\@namedef{PY@tok@kd}{\let\PY@bf=\textbf\def\PY@tc##1{\textcolor[rgb]{0.00,0.50,0.00}{##1}}}
\@namedef{PY@tok@kn}{\let\PY@bf=\textbf\def\PY@tc##1{\textcolor[rgb]{0.00,0.50,0.00}{##1}}}
\@namedef{PY@tok@kr}{\let\PY@bf=\textbf\def\PY@tc##1{\textcolor[rgb]{0.00,0.50,0.00}{##1}}}
\@namedef{PY@tok@bp}{\def\PY@tc##1{\textcolor[rgb]{0.00,0.50,0.00}{##1}}}
\@namedef{PY@tok@fm}{\def\PY@tc##1{\textcolor[rgb]{0.00,0.00,1.00}{##1}}}
\@namedef{PY@tok@vc}{\def\PY@tc##1{\textcolor[rgb]{0.10,0.09,0.49}{##1}}}
\@namedef{PY@tok@vg}{\def\PY@tc##1{\textcolor[rgb]{0.10,0.09,0.49}{##1}}}
\@namedef{PY@tok@vi}{\def\PY@tc##1{\textcolor[rgb]{0.10,0.09,0.49}{##1}}}
\@namedef{PY@tok@vm}{\def\PY@tc##1{\textcolor[rgb]{0.10,0.09,0.49}{##1}}}
\@namedef{PY@tok@sa}{\def\PY@tc##1{\textcolor[rgb]{0.73,0.13,0.13}{##1}}}
\@namedef{PY@tok@sb}{\def\PY@tc##1{\textcolor[rgb]{0.73,0.13,0.13}{##1}}}
\@namedef{PY@tok@sc}{\def\PY@tc##1{\textcolor[rgb]{0.73,0.13,0.13}{##1}}}
\@namedef{PY@tok@dl}{\def\PY@tc##1{\textcolor[rgb]{0.73,0.13,0.13}{##1}}}
\@namedef{PY@tok@s2}{\def\PY@tc##1{\textcolor[rgb]{0.73,0.13,0.13}{##1}}}
\@namedef{PY@tok@sh}{\def\PY@tc##1{\textcolor[rgb]{0.73,0.13,0.13}{##1}}}
\@namedef{PY@tok@s1}{\def\PY@tc##1{\textcolor[rgb]{0.73,0.13,0.13}{##1}}}
\@namedef{PY@tok@mb}{\def\PY@tc##1{\textcolor[rgb]{0.40,0.40,0.40}{##1}}}
\@namedef{PY@tok@mf}{\def\PY@tc##1{\textcolor[rgb]{0.40,0.40,0.40}{##1}}}
\@namedef{PY@tok@mh}{\def\PY@tc##1{\textcolor[rgb]{0.40,0.40,0.40}{##1}}}
\@namedef{PY@tok@mi}{\def\PY@tc##1{\textcolor[rgb]{0.40,0.40,0.40}{##1}}}
\@namedef{PY@tok@il}{\def\PY@tc##1{\textcolor[rgb]{0.40,0.40,0.40}{##1}}}
\@namedef{PY@tok@mo}{\def\PY@tc##1{\textcolor[rgb]{0.40,0.40,0.40}{##1}}}
\@namedef{PY@tok@ch}{\let\PY@it=\textit\def\PY@tc##1{\textcolor[rgb]{0.24,0.48,0.48}{##1}}}
\@namedef{PY@tok@cm}{\let\PY@it=\textit\def\PY@tc##1{\textcolor[rgb]{0.24,0.48,0.48}{##1}}}
\@namedef{PY@tok@cpf}{\let\PY@it=\textit\def\PY@tc##1{\textcolor[rgb]{0.24,0.48,0.48}{##1}}}
\@namedef{PY@tok@c1}{\let\PY@it=\textit\def\PY@tc##1{\textcolor[rgb]{0.24,0.48,0.48}{##1}}}
\@namedef{PY@tok@cs}{\let\PY@it=\textit\def\PY@tc##1{\textcolor[rgb]{0.24,0.48,0.48}{##1}}}

\def\PYZbs{\char`\\}
\def\PYZus{\char`\_}
\def\PYZob{\char`\{}
\def\PYZcb{\char`\}}
\def\PYZca{\char`\^}
\def\PYZam{\char`\&}
\def\PYZlt{\char`\<}
\def\PYZgt{\char`\>}
\def\PYZsh{\char`\#}
\def\PYZpc{\char`\%}
\def\PYZdl{\char`\$}
\def\PYZhy{\char`\-}
\def\PYZsq{\char`\'}
\def\PYZdq{\char`\"}
\def\PYZti{\char`\~}
% for compatibility with earlier versions
\def\PYZat{@}
\def\PYZlb{[}
\def\PYZrb{]}
\makeatother


    % For linebreaks inside Verbatim environment from package fancyvrb. 
    \makeatletter
        \newbox\Wrappedcontinuationbox 
        \newbox\Wrappedvisiblespacebox 
        \newcommand*\Wrappedvisiblespace {\textcolor{red}{\textvisiblespace}} 
        \newcommand*\Wrappedcontinuationsymbol {\textcolor{red}{\llap{\tiny$\m@th\hookrightarrow$}}} 
        \newcommand*\Wrappedcontinuationindent {3ex } 
        \newcommand*\Wrappedafterbreak {\kern\Wrappedcontinuationindent\copy\Wrappedcontinuationbox} 
        % Take advantage of the already applied Pygments mark-up to insert 
        % potential linebreaks for TeX processing. 
        %        {, <, #, %, $, ' and ": go to next line. 
        %        _, }, ^, &, >, - and ~: stay at end of broken line. 
        % Use of \textquotesingle for straight quote. 
        \newcommand*\Wrappedbreaksatspecials {% 
            \def\PYGZus{\discretionary{\char`\_}{\Wrappedafterbreak}{\char`\_}}% 
            \def\PYGZob{\discretionary{}{\Wrappedafterbreak\char`\{}{\char`\{}}% 
            \def\PYGZcb{\discretionary{\char`\}}{\Wrappedafterbreak}{\char`\}}}% 
            \def\PYGZca{\discretionary{\char`\^}{\Wrappedafterbreak}{\char`\^}}% 
            \def\PYGZam{\discretionary{\char`\&}{\Wrappedafterbreak}{\char`\&}}% 
            \def\PYGZlt{\discretionary{}{\Wrappedafterbreak\char`\<}{\char`\<}}% 
            \def\PYGZgt{\discretionary{\char`\>}{\Wrappedafterbreak}{\char`\>}}% 
            \def\PYGZsh{\discretionary{}{\Wrappedafterbreak\char`\#}{\char`\#}}% 
            \def\PYGZpc{\discretionary{}{\Wrappedafterbreak\char`\%}{\char`\%}}% 
            \def\PYGZdl{\discretionary{}{\Wrappedafterbreak\char`\$}{\char`\$}}% 
            \def\PYGZhy{\discretionary{\char`\-}{\Wrappedafterbreak}{\char`\-}}% 
            \def\PYGZsq{\discretionary{}{\Wrappedafterbreak\textquotesingle}{\textquotesingle}}% 
            \def\PYGZdq{\discretionary{}{\Wrappedafterbreak\char`\"}{\char`\"}}% 
            \def\PYGZti{\discretionary{\char`\~}{\Wrappedafterbreak}{\char`\~}}% 
        } 
        % Some characters . , ; ? ! / are not pygmentized. 
        % This macro makes them "active" and they will insert potential linebreaks 
        \newcommand*\Wrappedbreaksatpunct {% 
            \lccode`\~`\.\lowercase{\def~}{\discretionary{\hbox{\char`\.}}{\Wrappedafterbreak}{\hbox{\char`\.}}}% 
            \lccode`\~`\,\lowercase{\def~}{\discretionary{\hbox{\char`\,}}{\Wrappedafterbreak}{\hbox{\char`\,}}}% 
            \lccode`\~`\;\lowercase{\def~}{\discretionary{\hbox{\char`\;}}{\Wrappedafterbreak}{\hbox{\char`\;}}}% 
            \lccode`\~`\:\lowercase{\def~}{\discretionary{\hbox{\char`\:}}{\Wrappedafterbreak}{\hbox{\char`\:}}}% 
            \lccode`\~`\?\lowercase{\def~}{\discretionary{\hbox{\char`\?}}{\Wrappedafterbreak}{\hbox{\char`\?}}}% 
            \lccode`\~`\!\lowercase{\def~}{\discretionary{\hbox{\char`\!}}{\Wrappedafterbreak}{\hbox{\char`\!}}}% 
            \lccode`\~`\/\lowercase{\def~}{\discretionary{\hbox{\char`\/}}{\Wrappedafterbreak}{\hbox{\char`\/}}}% 
            \catcode`\.\active
            \catcode`\,\active 
            \catcode`\;\active
            \catcode`\:\active
            \catcode`\?\active
            \catcode`\!\active
            \catcode`\/\active 
            \lccode`\~`\~ 	
        }
    \makeatother

    \let\OriginalVerbatim=\Verbatim
    \makeatletter
    \renewcommand{\Verbatim}[1][1]{%
        %\parskip\z@skip
        \sbox\Wrappedcontinuationbox {\Wrappedcontinuationsymbol}%
        \sbox\Wrappedvisiblespacebox {\FV@SetupFont\Wrappedvisiblespace}%
        \def\FancyVerbFormatLine ##1{\hsize\linewidth
            \vtop{\raggedright\hyphenpenalty\z@\exhyphenpenalty\z@
                \doublehyphendemerits\z@\finalhyphendemerits\z@
                \strut ##1\strut}%
        }%
        % If the linebreak is at a space, the latter will be displayed as visible
        % space at end of first line, and a continuation symbol starts next line.
        % Stretch/shrink are however usually zero for typewriter font.
        \def\FV@Space {%
            \nobreak\hskip\z@ plus\fontdimen3\font minus\fontdimen4\font
            \discretionary{\copy\Wrappedvisiblespacebox}{\Wrappedafterbreak}
            {\kern\fontdimen2\font}%
        }%
        
        % Allow breaks at special characters using \PYG... macros.
        \Wrappedbreaksatspecials
        % Breaks at punctuation characters . , ; ? ! and / need catcode=\active 	
        \OriginalVerbatim[#1,codes*=\Wrappedbreaksatpunct]%
    }
    \makeatother

    % Exact colors from NB
    \definecolor{incolor}{HTML}{303F9F}
    \definecolor{outcolor}{HTML}{D84315}
    \definecolor{cellborder}{HTML}{CFCFCF}
    \definecolor{cellbackground}{HTML}{F7F7F7}
    
    % prompt
    \makeatletter
    \newcommand{\boxspacing}{\kern\kvtcb@left@rule\kern\kvtcb@boxsep}
    \makeatother
    \newcommand{\prompt}[4]{
        {\ttfamily\llap{{\color{#2}[#3]:\hspace{3pt}#4}}\vspace{-\baselineskip}}
    }
    

    
    % Prevent overflowing lines due to hard-to-break entities
    \sloppy 
    % Setup hyperref package
    \hypersetup{
      breaklinks=true,  % so long urls are correctly broken across lines
      colorlinks=true,
      urlcolor=urlcolor,
      linkcolor=linkcolor,
      citecolor=citecolor,
      }
    % Slightly bigger margins than the latex defaults
    
    \geometry{verbose,tmargin=1in,bmargin=1in,lmargin=1in,rmargin=1in}
    
    

\begin{document}
    
  \maketitle
  \thispagestyle{empty}
  \tableofcontents
  \pagebreak


    \hypertarget{ux433ux430ux443ux441ux441ux43eux432ux441ux43aux438ux435-ux441ux43bux443ux447ux430ux439ux43dux44bux435-ux432ux435ux43bux438ux447ux438ux43dux44b}{%
\section{Гауссовские случайные
величины}\label{ux433ux430ux443ux441ux441ux43eux432ux441ux43aux438ux435-ux441ux43bux443ux447ux430ux439ux43dux44bux435-ux432ux435ux43bux438ux447ux438ux43dux44b}}

\hypertarget{ux43eux43fux440ux435ux434ux435ux43bux435ux43dux438ux435}{%
\subsection{Определение}\label{ux43eux43fux440ux435ux434ux435ux43bux435ux43dux438ux435}}

Если \(\xi\) --- случайная величина с гауссовской (нормальной)
плотностью (probability density function, pdf)
\[
f_\xi(x|\mu,\sigma) = \frac{1}{\sqrt{2\pi}\sigma} \exp{ \left( -\frac{(x - \mu)^2}{2\sigma^2}\right)}, \quad \sigma>0, \quad -\infty < \mu < \infty,
\]

то смысл параметров \(\mu\) и \(\sigma\) оказывается очень простым:
\[ \mu = \mathrm{E} \xi, \quad \sigma^2 = \mathrm{D} \xi . \]

Таким образом, рапределение вероятностей этой случайной величины
\(\xi\), называемой \emph{гауссовской} или \emph{нормально
распределённой}, полностью определяется её средним значением \(\mu\) и
дисперсией \(\sigma^2\). В этой связи часто используется запись
\[ \xi \sim \mathcal{N}\left( \mu, \sigma^2 \right). \]

    \hypertarget{ux441ux432ux43eux439ux441ux442ux432ux430}{%
\subsection{Свойства}\label{ux441ux432ux43eux439ux441ux442ux432ux430}}

\begin{enumerate}
\def\labelenumi{\arabic{enumi}.}
\item
  Если \(\xi\) и \(\eta\) --- гауссовские случайные величины, то из их
  \emph{некоррелированности} следует их \emph{независимость}.
\item
  Пусть \(\xi\) и \(\eta\) --- две независимые гауссовские случайные
  величины: \(\xi \sim \mathcal{N}\left( \mu_1, \sigma_1^2 \right)\),
  \(\eta \sim \mathcal{N}\left( \mu_2, \sigma_2^2 \right)\). Тогда их
  сумма снова есть гауссовская случайная величина со средним
  \(\mu_1 + \mu_2\) и дисперсией \(\sigma_1^2 + \sigma_2^2\).
\item
  \textbf{Центральная предельная теорема:} распределение суммы большого
  числа независимых случайных величин или случайных векторов,
  подчиняющихся не слишком стеснительным условиям, хорошо
  аппроксимируется нормальным распределением.
\end{enumerate}

\textbf{Замечание.} Строго говоря, свойство 1 выполняется, если у
\(\xi\) и \(\eta\) существует \emph{совместная нормальная плотность}
(см. \emph{Чернова Н. И.} Математическая статистика. с. 124). Например,
случайные величины \(\xi\) и \(\xi^2\) некоррелированы, но функционально
зависимы.

    Докажем свойство 2 для случая, когда \(\xi\) и \(\eta\) имеют
\emph{стандартное} нормальное распределение.\\
Применим формулу свёртки:
\begin{align*}
  f_{\xi+\eta}(t)
  &= \dfrac{1}{2\pi} \int\limits_{-\infty}^{\infty} e^{-{u^2}/{2}} e^{-{(t-u)^2}/{2}} du
  = \dfrac{1}{2\pi} \int\limits_{-\infty}^{\infty} e^{-\left( u^2 - ut + t^2/4 + t^2/4 \right)} du \\
  &= \dfrac{1}{2\pi} e^{-t^2/4} \int\limits_{-\infty}^{\infty} e^{-\left( u - t/2 \right)^2} du
  = \dfrac{1}{2\pi} e^{-t^2/4} \int\limits_{-\infty}^{\infty} e^{-v^2}dv
  = \dfrac{1}{2\sqrt{\pi}} e^{-t^2/4}.
\end{align*}

    \hypertarget{ux43cux43eux43cux435ux43dux442ux44b-ux43dux43eux440ux43cux430ux43bux44cux43dux43eux433ux43e-ux440ux430ux441ux43fux440ux435ux434ux435ux43bux435ux43dux438ux44f}{%
\subsection{Моменты нормального
распределения}\label{ux43cux43eux43cux435ux43dux442ux44b-ux43dux43eux440ux43cux430ux43bux44cux43dux43eux433ux43e-ux440ux430ux441ux43fux440ux435ux434ux435ux43bux435ux43dux438ux44f}}

Выведем общую формулу для центрального момента любого порядка:
\[
  M_n = \dfrac{1}{\sqrt{2\pi}\sigma} \int\limits_{-\infty}^{\infty}(x-\mu)^n e^{-\frac{(x-\mu)^2}{2\sigma^2}}dx.
\]

Делая замену переменной \(t = \dfrac{x-\mu}{\sqrt{2}\sigma}\), получим:
\[
  M_n = \dfrac{(\sqrt{2}\sigma)^n}{\sqrt{\pi}} \int\limits_{-\infty}^{\infty}t^n e^{-t^2}dt.
\]

Интегрируем по частям:
\[
  M_n = \dfrac{(\sqrt{2}\sigma)^n}{\sqrt{\pi}} \int\limits_{-\infty}^{\infty}t^{n-1}t e^{-t^2}dt
  = \dfrac{(\sqrt{2}\sigma)^n}{\sqrt{\pi}} \left( \left.-\dfrac{1}{2}t^{n-1}e^{-t^2} \right|_{-\infty}^{+\infty}
  + \dfrac{n-1}{2} \int\limits_{-\infty}^{\infty}t^{n-2} e^{-t^2}dt \right).
\]

Имея в виду, что первое слагаемое в скобках равно нулю, получим
следующее рекуррентное соотношение:
\[
  M_n = \dfrac{(\sqrt{2}\sigma)^n (n-1)}{2\sqrt{\pi}} \int\limits_{-\infty}^{\infty}t^{n-2} e^{-t^2}dt 
  = (n-1) \sigma^2 M_{n-2}.
\]

    Принимая во внимание, что \(M_0=1\) и \(M_1=0\), получим итоговый
результат для \emph{центральных} моментов:

\begin{enumerate}
\def\labelenumi{\arabic{enumi}.}
\tightlist
\item
  \(M_n = 0\) для нечётных \(n\),
\item
  \(M_n = (n-1)!!\,\sigma^n\) для чётных \(n\).
\end{enumerate}

    \hypertarget{ux43fux440ux438ux43cux435ux440ux44b}{%
\subsection{Примеры}\label{ux43fux440ux438ux43cux435ux440ux44b}}

\textbf{Пример 1}\\
Графики плотностей трёх одномерных нормальных распределений:
\(\mathcal{N}(0, 1)\),
\(\mathcal{N}(2, 3)\),
\(\mathcal{N}(0, 0.2)\).


    \begin{center}
    \adjustimage{max size={0.45\linewidth}{0.45\paperheight}}{output_13_0.png}
    \end{center}
%    { \hspace*{\fill} \\}
    
    \textbf{Пример 2}\\
Графики плотностей \(\mathcal{N}(-1, 1)\), \(\mathcal{N}(2, 0.64)\) и их
суммы.


    \begin{center}
    \adjustimage{max size={0.5\linewidth}{0.5\paperheight}}{output_16_0.png}
    \end{center}
%    { \hspace*{\fill} \\}
    
%    \begin{center}\rule{0.5\linewidth}{0.5pt}\end{center}

    \hypertarget{ux441ux43bux443ux447ux430ux439ux43dux44bux439-ux432ux435ux43aux442ux43eux440}{%
\section{Случайный
вектор}\label{ux441ux43bux443ux447ux430ux439ux43dux44bux439-ux432ux435ux43aux442ux43eux440}}

    \hypertarget{ux43eux43fux440ux435ux434ux435ux43bux435ux43dux438ux435}{%
\subsection{Определение}\label{ux43eux43fux440ux435ux434ux435ux43bux435ux43dux438ux435}}

\textbf{Определение.} Всякий упорядоченный набор случайных величин
\(\vec{\xi} = (\xi_1, \ldots, \xi_n)\) будем называть \emph{\(n\)-мерным
случайным вектором}.

\textbf{Определение.} Математическим ожиданием случайного вектора будем
называть вектор математических ожиданий его каждой компоненты:
\(\mathrm{E}\vec{\xi} = (\mathrm{E}\xi_1, \ldots, \mathrm{E}\xi_n)\).

Для математического ожидания случайного вектора справедливы все свойства
математического ожидания случайной величины. В том числе
\emph{линейность:}
\(\mathrm{E}(A\vec{\xi} + B\vec{\eta}) = A \cdot \mathrm{E}\vec{\xi} + B \cdot \mathrm{E}\vec{\eta}\).

    \hypertarget{ux43aux43eux432ux430ux440ux438ux430ux446ux438ux43eux43dux43dux430ux44f-ux43cux430ux442ux440ux438ux446ux430}{%
\subsection{Ковариационная
матрица}\label{ux43aux43eux432ux430ux440ux438ux430ux446ux438ux43eux43dux43dux430ux44f-ux43cux430ux442ux440ux438ux446ux430}}

Пусть \(\vec\xi = \left( \xi_1, \dots, \xi_n \right)\) --- случайный
вектор, компоненты которого имеют конечный второй момент. Назовём
\emph{матрицей ковариаций} (ковариационной матрицей) вектора \(\xi\)
матрицу (порядка \(n \times n\)) \(\Sigma = ||\Sigma_{ij}||\), где
\(\Sigma_{ij} = \text{cov}\left( \xi_i, \xi_j \right)\).

Ковариационная матрица случайного вектора является многомерным аналогом
дисперсии случайной величины для случайных векторов. На диагонали
\(\Sigma\) располагаются дисперсии компонент вектора, а внедиагональные
элементы --- ковариации между компонентами.

    \textbf{Свойства ковариационной матрицы:}

\begin{enumerate}
\def\labelenumi{\arabic{enumi}.}
\tightlist
\item
  \(\mathrm{cov}(\vec\xi) = \mathrm{E} \left[ (\vec\xi -\mathrm{E}\vec\xi) \cdot (\vec\xi -\mathrm{E}\vec\xi)^\top \right]\)
\item
  \(\mathrm{cov}(\vec\xi) = \mathrm{E} \vec\xi \vec\xi^\top - \mathrm{E} \vec\xi \cdot \mathrm{E} \vec\xi^\top\)
\item
  Положительная определённость: \(\mathrm{cov}(\vec\xi) > 0\)
\item
  Аффинное преобразование:
  \(\mathrm{cov}(A\vec\xi + \vec{b}) = A \cdot \mathrm{cov}(\vec\xi) \cdot A^\top\)
\end{enumerate}

    \textbf{Утверждение}. Ковариационная матрица случайного вектора является
\emph{симметричной} и \emph{положительно определённой}.

\emph{Доказательство}. Обозначим ковариационную матрицу \(\mathbf{K}\).
Положительная определёнонсть означает, что
\(\forall \mathbf{a} \in \mathbb{R}^n: \mathbf{a}^\top \mathbf{K} \mathbf{a} > 0\).

Действительно,
\[\begin{split}
\mathbf{a}^\top \mathbf{K} \mathbf{a} = \sum\limits_{i,j=1}^n a_i K_{ij} a_j
  &= \sum\limits_{i,j=1}^n a_i \mathrm{E}\left[ (\xi_i - \mathrm{E}\xi_i) (\xi_j - \mathrm{E}\xi_j) \right] a_j = \sum\limits_{i,j=1}^n \mathrm{E}\left[ a_i (\xi_i - \mathrm{E}\xi_i) (\xi_j - \mathrm{E}\xi_j) a_j \right] \\
  &= \mathrm{E}\left[ \sum\limits_{i,j=1}^n a_i (\xi_i - \mathrm{E}\xi_i) (\xi_j - \mathrm{E}\xi_j) a_j \right] = \mathrm{E}\left[ \sum\limits_{i=1}^n a_i (\xi_i - \mathrm{E}\xi_i) \right]^2. \mathrm{\square}
\end{split}
\]

    Справедлив и обратный результат.

\textbf{Утверждение.} Для того, чтобы матрица \(\Sigma\) порядка
\(n \times n\) была ковариационной матрицей некоторого случайного
вектора \(\vec\xi = \left( \xi_1, \dots, \xi_n \right)\), необходимо и
достаточно, чтобы эта матрица была симметричной и положительно
определённой.

\emph{Доказательство}. Тот факт, что всякая ковариационная матрица
является симетричной и положительно определённой будем считать
доказанным. Покажем теперь обратное, что \(\Sigma\) является
ковариационной матрицей некоторого случайного вектора.

Воспользуемся \emph{разложением Холецкого} --- представлением
симметричной положительно определённой матрицы в виде произведения
нижнетреугольной матрицы \(L\) и верхнетреугольной матрицы \(L^\top\).

Пусть \(\vec\eta\) --- вектор нормально распределённых случайных величин
\(\vec\eta \sim \mathcal{N}(0, 1)\). Пусть \(\Sigma = L L^\top\).
Покажем, что вектор \(\vec\xi = L\vec\eta\) имеет ковариационную матрицу
\(\Sigma\).
\[
\mathrm{cov}(\vec\xi)
= \mathrm{E} \left[\vec\xi \cdot \vec\xi^\top \right]
= \mathrm{E} \left[(L\vec\eta)(L\vec\eta)^\top \right]
= L \cdot \mathrm{E} \left[\vec\eta \cdot \vec\eta^\top \right] \cdot L^\top
= L I_n L^\top = LL^\top = \Sigma. \; \mathrm{\square}
\]

    \begin{center}\rule{0.5\linewidth}{0.5pt}\end{center}

    \hypertarget{ux43cux43dux43eux433ux43eux43cux435ux440ux43dux43eux435-ux43dux43eux440ux43cux430ux43bux44cux43dux43eux435-ux440ux430ux441ux43fux440ux435ux434ux435ux43bux435ux43dux438ux435}{%
\section{Многомерное нормальное
распределение}\label{ux43cux43dux43eux433ux43eux43cux435ux440ux43dux43eux435-ux43dux43eux440ux43cux430ux43bux44cux43dux43eux435-ux440ux430ux441ux43fux440ux435ux434ux435ux43bux435ux43dux438ux435}}

    \hypertarget{ux43eux43fux440ux435ux434ux435ux43bux435ux43dux438ux435}{%
\subsection{Определение}\label{ux43eux43fux440ux435ux434ux435ux43bux435ux43dux438ux435}}

Многомерное нормальное распределение представляет собой многомерное
обобщение одномерного нормального распределения. Оно представляет собой
распределение многомерной случайной величины, состоящей из нескольких
случайных величин, которые могут быть скоррелированы друг с другом.

Как и одномерное, многомерное нормальное распределение определяется
набором параметров: вектором средних значений \(\mathbf{\mu}\), который
является вектором математических ожиданий распределения, и
ковариационной матрицей \(\Sigma\), которая измеряет степень зависимости
двух случайных величин и их совместного изменения.

Многомерное нормальное распределение случайного вектора
\(\overline{\xi}\) размерностью \(n\) имеет следующую функцию плотности
совместной вероятности:
\[
f_n(\vec{x}|\vec{\mu}, \Sigma) = \frac{1}{\sqrt{(2\pi)^n |\Sigma|}} \exp{ \left( -\frac{1}{2}(\vec{x} - \vec{\mu})^\top \Sigma^{-1} (\vec{x} - \vec{\mu}) \right)}.
\]

Здесь \(\vec{x}\) --- случайный вектор размерностью \(n\), \(\vec{\mu}\)
--- вектор математического ожидания, \(\Sigma\) --- ковариационная
матрица (симметричная, положительно определённая матрица с размерностью
\(n \times n\), \(\Sigma_{ij} = \text{cov}(\xi_i, \xi_j)\)), а
\(\lvert\Sigma\rvert\) --- её определитель. Многомерное нормальное
распределение принято обозначать следующим образом:
\[
    \vec{\xi} \sim \mathcal{N}(\vec{\mu}, \Sigma)
\]

\begin{quote}
Далее для простоты записи стрелка над вектором будет опускаться, т. е.
вместо \(\vec{\xi}\) будем писать просто \(\xi\).
\end{quote}

    \textbf{Замечание.} Вектор, составленный из нормальных случайных
величин, не обязательно имеет многомерное нормальное распределение. Так,
для \(\xi \sim \mathcal{N}(0,1)\) вектор \((\xi, c\xi)\) имеет
вырожденную матрицу ковариаций
\(\begin{pmatrix}  1 & c \\  c & c^2 \end{pmatrix}\) и не имеет
плотности в \(\mathbb{R}^2\).

    \hypertarget{ux434ux432ux443ux43cux435ux440ux43dux43eux435-ux43dux43eux440ux43cux430ux43bux44cux43dux43eux435-ux440ux430ux441ux43fux440ux435ux434ux435ux43bux435ux43dux438ux435}{%
\subsection{Двумерное нормальное
распределение}\label{ux434ux432ux443ux43cux435ux440ux43dux43eux435-ux43dux43eux440ux43cux430ux43bux44cux43dux43eux435-ux440ux430ux441ux43fux440ux435ux434ux435ux43bux435ux43dux438ux435}}

В качестве примера рассмотрим двумерный случайный вектор. В этом случае
ковариационная матрица имеет вид \[
\Sigma = 
\begin{pmatrix}
    \sigma_1^2 & \rho \sigma_1 \sigma_2 \\
    \rho \sigma_1 \sigma_2 & \sigma_2^2 
\end{pmatrix}.
\]

    Двумерная нормальная плотность \(p(x_1, x_2)\) имеет вид \[
  f_{\xi,\eta}(x_1, x_2) = \frac{1}{2\pi\sigma_1\sigma_2\sqrt{1-\rho^2}} \times \\
  \exp \left\{-\frac{1}{2(1-\rho^2)} \left[ \frac{(x_1-m_1)^2}{\sigma_1^2} - 2\rho\frac{(x_1-m_1)(x_2-m_2)}{\sigma_1\sigma_2} + \frac{(x_2-m_2)^2}{\sigma_2^2} \right]\right\},
\]

где\\
\(m_1 = \mathrm{E} \xi\), \(m_2 = \mathrm{E} \eta\) --- математические
ожидания,\\
\(\sigma_1^2 = \mathrm{D} \xi\), \(\sigma_2^2 = \mathrm{D} \eta\) ---
стандартное отклонение \(x_i\),\\
\(\rho = \dfrac{\mathrm{cov}(\xi, \eta)}{\sigma_1 \cdot \sigma_2}\) ---
коэффициент корреляции.

\textbf{Замечание.} Можно убедиться, что если пара (\(\xi\), \(\eta\))
--- гауссовская, то из некоррелированности \(\xi\) и \(\eta\) следует их
независимость.\\
Действительно, если \(\rho=0\), то
\[
  f_{\xi,\eta}(x_1, x_2) = \frac{1}{2\pi\sigma_1\sigma_2} \, \exp\left\{-\frac{(x_1-m_1)^2}{2\sigma_1^2}\right\} \, \exp\left\{-\frac{(x_2-m_2)^2}{2\sigma_2^2}\right\}
  = f_\xi(x_1) \cdot f_{\eta}(x_2).
\]

    \textbf{Примеры}

Примеры двумерных нормальных распределений приведены ниже.

\begin{enumerate}
\def\labelenumi{\arabic{enumi}.}
\tightlist
\item
  Двумерное распределение независимых случайных величин
\item
  Двумерное распределение положительно коррелированных случайных величин
  (возрастание \(x_1\) увеличивает вероятность возрастания \(x_2\))
\end{enumerate}


    \begin{center}
    \adjustimage{max size={0.9\linewidth}{0.9\paperheight}}{output_35_0.png}
    \end{center}
%    { \hspace*{\fill} \\}
    
    \begin{center}\rule{0.5\linewidth}{0.5pt}\end{center}

    \hypertarget{ux433ux435ux43dux435ux440ux430ux446ux438ux44f-ux432ux44bux431ux43eux440ux43aux438-ux433ux430ux443ux441ux441ux43eux432ux441ux43aux438ux445-ux432ux435ux43aux442ux43eux440ux43eux432}{%
\section{Генерация выборки гауссовских
векторов}\label{ux433ux435ux43dux435ux440ux430ux446ux438ux44f-ux432ux44bux431ux43eux440ux43aux438-ux433ux430ux443ux441ux441ux43eux432ux441ux43aux438ux445-ux432ux435ux43aux442ux43eux440ux43eux432}}

    \hypertarget{ux430ux444ux444ux438ux43dux43dux43eux435-ux43fux440ux435ux43eux431ux440ux430ux437ux43eux432ux430ux43dux438ux435-ux43cux43dux43eux433ux43eux43cux435ux440ux43dux43eux433ux43e-ux43dux43eux440ux43cux430ux43bux44cux43dux43eux433ux43e-ux440ux430ux441ux43fux440ux435ux434ux435ux43bux435ux43dux438ux44f}{%
\subsection{Аффинное преобразование многомерного нормального
распределения}\label{ux430ux444ux444ux438ux43dux43dux43eux435-ux43fux440ux435ux43eux431ux440ux430ux437ux43eux432ux430ux43dux438ux435-ux43cux43dux43eux433ux43eux43cux435ux440ux43dux43eux433ux43e-ux43dux43eux440ux43cux430ux43bux44cux43dux43eux433ux43e-ux440ux430ux441ux43fux440ux435ux434ux435ux43bux435ux43dux438ux44f}}

Многомерное нормальное распределение можно преобразовать с помощью
аффинного преобразования. Так, если \(X\) --- нормально распределённый
случайный вектор, а \(Y = u + LX\) --- аффинное преобразованием \(X\) с
матрицей \(L\) и вектором \(u\), то \(Y\) также нормально распределён со
средним значением \(\mu_{Y} = u + L\mu_{X}\) и ковариационной матрицей \(\Sigma_{Y} = L\Sigma_{X}L^\top\).
\begin{align*}
X \sim \mathcal{N}(\mu_{X}, \Sigma_{X}), \quad\quad Y \sim \mathcal{N}(\mu_{Y}, \Sigma_{Y}) \\
\mathcal{N}(\mu_{Y}, \Sigma_{Y}) = \mathcal{N}(u + L\mu_{X}, L\Sigma_{X}L^\top) = u + L\mathcal{N}(\mu_{X}, \Sigma_{X})
\end{align*}

Это можно доказать следующим образом:
\[\mu_{Y} = \mathrm{E}[Y] = \mathrm{E}[u + LX] = u + \mathrm{E}[LX] = u + L\mu_{X}\]
\[\begin{split}
\Sigma_{Y} & = \mathrm{E}[(Y-\mu_{Y})(Y-\mu_{Y})^\top]
           = \mathrm{E}[(u+LX - u-L\mu_{X})(u+LX - u-L\mu_{X})^\top] \\
           & = \mathrm{E}[(L(X-\mu_{X})) (L(X-\mu_{X}))^\top]
           = \mathrm{E}[L(X-\mu_{X}) (X-\mu_{X})^\top L^\top] \\
           & = L\mathrm{E}[(X-\mu_{X})(X-\mu_{X})^\top]L^\top
           = L\Sigma_{X}L^\top.
\end{split}\]

    Итак, можно сформулировать следующий результат:

\begin{itemize}
\tightlist
\item
  \emph{любой нормальный вектор} в \(\mathbb{R}^m\) со сколь угодно
  зависимыми координатами может быть умножением на подходящую
  невырожденную матрицу превращён в вектор, состоящий из
  \emph{независимых стандартных нормальных случайных величин};
\item
  и наоборот, \emph{стандартный нормальный случайный вектор} можно
  линейным преобразованием превратить в вектор с \emph{заданным
  многомерным нормальным распределением}.
\end{itemize}

    \hypertarget{ux433ux435ux43dux435ux440ux430ux446ux438ux44f-ux432ux44bux431ux43eux440ux43aux438-ux433ux430ux443ux441ux441ux43eux432ux441ux43aux438ux445-ux432ux435ux43aux442ux43eux440ux43eux432}{%
\subsection{Генерация выборки гауссовских
векторов}\label{ux433ux435ux43dux435ux440ux430ux446ux438ux44f-ux432ux44bux431ux43eux440ux43aux438-ux433ux430ux443ux441ux441ux43eux432ux441ux43aux438ux445-ux432ux435ux43aux442ux43eux440ux43eux432}}

Предыдущая формула поможет нам сгенерировать гауссовский вектор с
заданными вектором средних значений и ковариационной матрицей.\\
Для этого сгенерируем вектор \(X\), подчиняющийся стандартному
нормальному распределению \(X \sim \mathcal{N}(0, I)\) со средним
значением \(\mu_{X} = 0\) и единичной ковариационной матрицей
\(\Sigma_{X} = I\). Генерация такого вектора не представляет труда, так
как каждая переменная в \(X\) независима от всех других переменных, и мы
можем просто генерировать каждую переменную отдельно, пользуясь
одномерным распределением Гаусса.

Для генерации \(Y \sim \mathcal{N}(\mu_{Y}, \Sigma_{Y})\) возьмём \(X\)
и применим к нему аффинное преобразование \(Y = u + LX\). Из предыдущего
раздела мы знаем, что ковариация \(Y\) будет
\(\Sigma_{Y} = L\Sigma_{X}L^\top\). Поскольку \(\Sigma_{X}=I\), а
\(\mu_{X} = 0\), то \(\Sigma_{Y} = L L^\top\) и \(\mu_{Y} = u\). В итоге
получаем, что искомое преобразование \(Y = \mu_{Y} + L_{Y}X\), где
матрица \(L_{Y}\) --- нижнетреугольная матрица, которую можно найти с
помощью разложения Холецкого матрицы \(\Sigma_{Y}\).

    В качестве иллюстрации сгенерируем выборку двумерных векторов для
следующего распределения:
\[
Y
\sim
\mathcal{N}\left(
\begin{bmatrix}
    0 \\ 
    0 
\end{bmatrix},
\begin{bmatrix}
    1 & 0.8 \\
    0.8 & 1
\end{bmatrix}\right).
\]


    \begin{center}
    \adjustimage{max size={0.45\linewidth}{0.45\paperheight}}{output_43_0.png}
    \end{center}
%    { \hspace*{\fill} \\}
    
    Дополнительно нарисуем область, в которую попадает 95 \% векторов
(область \(2\sigma\)). Для этого нам понадобится распределение
\emph{длин} гауссовских векторов.

    \hypertarget{ux440ux430ux441ux43fux440ux435ux434ux435ux43bux435ux43dux438ux435-ux43fux438ux440ux441ux43eux43dux430-mathbfchi2}{%
\subsection{\texorpdfstring{Распределение Пирсона
\(\mathbf{\chi^2}\)}{Распределение Пирсона \textbackslash mathbf\{\textbackslash chi\^{}2\}}}\label{ux440ux430ux441ux43fux440ux435ux434ux435ux43bux435ux43dux438ux435-ux43fux438ux440ux441ux43eux43dux430-mathbfchi2}}

\textbf{Определение.} Распределение суммы \(k\) квадратов независимых
случайных величин со стандартным нормальным распределением называется
распределением \(\chi^2 = \xi_1^2 + \ldots + \xi_k^2\) (хи-квадрат) с
\(k\) степенями свободы и обозначается \(H_k\).

Плотность распределения \(H_k\) для \(x>0\) равна \[
  H_k(x) = \dfrac{1}{2^{k/2}\Gamma(k/2)} x^{k/2-1} e^{-x/2}.
\]

    \textbf{Свойства:}

\begin{enumerate}
\def\labelenumi{\arabic{enumi}.}
\tightlist
\item
  При \(k\ge2\) максимум плотности распределения \(H_k\) достигается в
  точке \(x=k-2\).
\item
  Если случайные величины \(\chi^2 \sim H_k\) и \(\psi^2 \sim H_m\)
  независимы, то их сумма \(\chi^2 + \psi^2\) имеет распределение
  \(H_{k+m}\).
\item
  Если величина \(\chi^2\) имеет распределение \(H_k\) то
  \(\mathrm{E} \chi^2 = k\) и \(\mathrm{D} \chi^2 = 2k\).
\item
  Если случайные величины \(\xi_1, \ldots, \xi_k\) независимы и имеют
  нормальное распределение \(\mathcal{N}(\mu, \sigma^2)\), то случайная
  величины
  \(\chi_k^2 = \sum\limits_{i=1}^{k} \left( \dfrac{\xi_i-\mu}{\sigma} \right)^2\)
  имеет распределение \(H_k\).
\end{enumerate}


    \begin{center}
    \adjustimage{max size={0.5\linewidth}{0.5\paperheight}}{output_47_0.png}
    \end{center}
%    { \hspace*{\fill} \\}
    

    \begin{center}
    \adjustimage{max size={0.5\linewidth}{0.5\paperheight}}{output_49_0.png}
    \end{center}
%    { \hspace*{\fill} \\}
    
    \begin{center}\rule{0.5\linewidth}{0.5pt}\end{center}

    \hypertarget{ux438ux441ux442ux43eux447ux43dux438ux43aux438}{%
\section{Источники}\label{ux438ux441ux442ux43eux447ux43dux438ux43aux438}}

\begin{enumerate}
\def\labelenumi{\arabic{enumi}.}
\tightlist
\item
  \emph{Ширяев А.Н.} Вероятность --- 1. --- М.: МЦНМО, 2007. --- 517 с.
\item
  \emph{Чернова Н.И.} Математическая статистика: Учебное пособие ---
  Новосиб. гос. ун-т, 2007. --- 148 с.
\item
  \emph{Roelants P.}
  \href{https://peterroelants.github.io/posts/multivariate-normal-primer/}{Multivariate
  normal distribution}.
\end{enumerate}



    % Add a bibliography block to the postdoc
    
    
    
\end{document}
