\documentclass[11pt,a4paper]{article}

    \usepackage[breakable]{tcolorbox}
    \usepackage{parskip} % Stop auto-indenting (to mimic markdown behaviour)
    
    \usepackage{iftex}
    \ifPDFTeX
      \usepackage[T2A]{fontenc}
      \usepackage{mathpazo}
      \usepackage[russian,english]{babel}
    \else
      \usepackage{fontspec}
      \usepackage{polyglossia}
      \setmainlanguage[babelshorthands=true]{russian}    % Язык по-умолчанию русский с поддержкой приятных команд пакета babel
      \setotherlanguage{english}                         % Дополнительный язык = английский (в американской вариации по-умолчанию)

      \defaultfontfeatures{Ligatures=TeX}
      \setmainfont[BoldFont={STIX Two Text SemiBold}]%
      {STIX Two Text}                                    % Шрифт с засечками
      \newfontfamily\cyrillicfont[BoldFont={STIX Two Text SemiBold}]%
      {STIX Two Text}                                    % Шрифт с засечками для кириллицы
      \setsansfont{Fira Sans}                            % Шрифт без засечек
      \newfontfamily\cyrillicfontsf{Fira Sans}           % Шрифт без засечек для кириллицы
      \setmonofont[Scale=0.87,BoldFont={Fira Mono Medium},ItalicFont=[FiraMono-Oblique]]%
      {Fira Mono}%                                       % Моноширинный шрифт
      \newfontfamily\cyrillicfonttt[Scale=0.87,BoldFont={Fira Mono Medium},ItalicFont=[FiraMono-Oblique]]%
      {Fira Mono}                                        % Моноширинный шрифт для кириллицы

      %%% Математические пакеты %%%
      \usepackage{amsthm,amsmath,amscd}   % Математические дополнения от AMS
      \usepackage{amsfonts,amssymb}       % Математические дополнения от AMS
      \usepackage{mathtools}              % Добавляет окружение multlined
      \usepackage{unicode-math}           % Для шрифта STIX Two Math
      \setmathfont{STIX Two Math}         % Математический шрифт
    \fi

    % Basic figure setup, for now with no caption control since it's done
    % automatically by Pandoc (which extracts ![](path) syntax from Markdown).
    \usepackage{graphicx}
    % Maintain compatibility with old templates. Remove in nbconvert 6.0
    \let\Oldincludegraphics\includegraphics
    % Ensure that by default, figures have no caption (until we provide a
    % proper Figure object with a Caption API and a way to capture that
    % in the conversion process - todo).
    \usepackage{caption}
    \DeclareCaptionFormat{nocaption}{}
    \captionsetup{format=nocaption,aboveskip=0pt,belowskip=0pt}

    \usepackage{float}
    \floatplacement{figure}{H} % forces figures to be placed at the correct location
    \usepackage{xcolor} % Allow colors to be defined
    \usepackage{enumerate} % Needed for markdown enumerations to work
    \usepackage{geometry} % Used to adjust the document margins
    \usepackage{amsmath} % Equations
    \usepackage{amssymb} % Equations
    \usepackage{textcomp} % defines textquotesingle
    % Hack from http://tex.stackexchange.com/a/47451/13684:
    \AtBeginDocument{%
        \def\PYZsq{\textquotesingle}% Upright quotes in Pygmentized code
    }
    \usepackage{upquote} % Upright quotes for verbatim code
    \usepackage{eurosym} % defines \euro
    \usepackage[mathletters]{ucs} % Extended unicode (utf-8) support
    \usepackage{fancyvrb} % verbatim replacement that allows latex
    \usepackage{grffile} % extends the file name processing of package graphics 
                         % to support a larger range
    \makeatletter % fix for old versions of grffile with XeLaTeX
    \@ifpackagelater{grffile}{2019/11/01}
    {
      % Do nothing on new versions
    }
    {
      \def\Gread@@xetex#1{%
        \IfFileExists{"\Gin@base".bb}%
        {\Gread@eps{\Gin@base.bb}}%
        {\Gread@@xetex@aux#1}%
      }
    }
    \makeatother
    \usepackage[Export]{adjustbox} % Used to constrain images to a maximum size
    \adjustboxset{max size={0.9\linewidth}{0.9\paperheight}}

    % The hyperref package gives us a pdf with properly built
    % internal navigation ('pdf bookmarks' for the table of contents,
    % internal cross-reference links, web links for URLs, etc.)
    \usepackage{hyperref}
    % The default LaTeX title has an obnoxious amount of whitespace. By default,
    % titling removes some of it. It also provides customization options.
    \usepackage{titling}
    \usepackage{longtable} % longtable support required by pandoc >1.10
    \usepackage{booktabs}  % table support for pandoc > 1.12.2
    \usepackage[inline]{enumitem} % IRkernel/repr support (it uses the enumerate* environment)
    \usepackage[normalem]{ulem} % ulem is needed to support strikethroughs (\sout)
                                % normalem makes italics be italics, not underlines
    \usepackage{mathrsfs}
    

    
    % Colors for the hyperref package
    \definecolor{urlcolor}{rgb}{0,.145,.698}
    \definecolor{linkcolor}{rgb}{.71,0.21,0.01}
    \definecolor{citecolor}{rgb}{.12,.54,.11}

    % ANSI colors
    \definecolor{ansi-black}{HTML}{3E424D}
    \definecolor{ansi-black-intense}{HTML}{282C36}
    \definecolor{ansi-red}{HTML}{E75C58}
    \definecolor{ansi-red-intense}{HTML}{B22B31}
    \definecolor{ansi-green}{HTML}{00A250}
    \definecolor{ansi-green-intense}{HTML}{007427}
    \definecolor{ansi-yellow}{HTML}{DDB62B}
    \definecolor{ansi-yellow-intense}{HTML}{B27D12}
    \definecolor{ansi-blue}{HTML}{208FFB}
    \definecolor{ansi-blue-intense}{HTML}{0065CA}
    \definecolor{ansi-magenta}{HTML}{D160C4}
    \definecolor{ansi-magenta-intense}{HTML}{A03196}
    \definecolor{ansi-cyan}{HTML}{60C6C8}
    \definecolor{ansi-cyan-intense}{HTML}{258F8F}
    \definecolor{ansi-white}{HTML}{C5C1B4}
    \definecolor{ansi-white-intense}{HTML}{A1A6B2}
    \definecolor{ansi-default-inverse-fg}{HTML}{FFFFFF}
    \definecolor{ansi-default-inverse-bg}{HTML}{000000}

    % common color for the border for error outputs.
    \definecolor{outerrorbackground}{HTML}{FFDFDF}

    % commands and environments needed by pandoc snippets
    % extracted from the output of `pandoc -s`
    \providecommand{\tightlist}{%
      \setlength{\itemsep}{0pt}\setlength{\parskip}{0pt}}
    \DefineVerbatimEnvironment{Highlighting}{Verbatim}{commandchars=\\\{\}}
    % Add ',fontsize=\small' for more characters per line
    \newenvironment{Shaded}{}{}
    \newcommand{\KeywordTok}[1]{\textcolor[rgb]{0.00,0.44,0.13}{\textbf{{#1}}}}
    \newcommand{\DataTypeTok}[1]{\textcolor[rgb]{0.56,0.13,0.00}{{#1}}}
    \newcommand{\DecValTok}[1]{\textcolor[rgb]{0.25,0.63,0.44}{{#1}}}
    \newcommand{\BaseNTok}[1]{\textcolor[rgb]{0.25,0.63,0.44}{{#1}}}
    \newcommand{\FloatTok}[1]{\textcolor[rgb]{0.25,0.63,0.44}{{#1}}}
    \newcommand{\CharTok}[1]{\textcolor[rgb]{0.25,0.44,0.63}{{#1}}}
    \newcommand{\StringTok}[1]{\textcolor[rgb]{0.25,0.44,0.63}{{#1}}}
    \newcommand{\CommentTok}[1]{\textcolor[rgb]{0.38,0.63,0.69}{\textit{{#1}}}}
    \newcommand{\OtherTok}[1]{\textcolor[rgb]{0.00,0.44,0.13}{{#1}}}
    \newcommand{\AlertTok}[1]{\textcolor[rgb]{1.00,0.00,0.00}{\textbf{{#1}}}}
    \newcommand{\FunctionTok}[1]{\textcolor[rgb]{0.02,0.16,0.49}{{#1}}}
    \newcommand{\RegionMarkerTok}[1]{{#1}}
    \newcommand{\ErrorTok}[1]{\textcolor[rgb]{1.00,0.00,0.00}{\textbf{{#1}}}}
    \newcommand{\NormalTok}[1]{{#1}}
    
    % Additional commands for more recent versions of Pandoc
    \newcommand{\ConstantTok}[1]{\textcolor[rgb]{0.53,0.00,0.00}{{#1}}}
    \newcommand{\SpecialCharTok}[1]{\textcolor[rgb]{0.25,0.44,0.63}{{#1}}}
    \newcommand{\VerbatimStringTok}[1]{\textcolor[rgb]{0.25,0.44,0.63}{{#1}}}
    \newcommand{\SpecialStringTok}[1]{\textcolor[rgb]{0.73,0.40,0.53}{{#1}}}
    \newcommand{\ImportTok}[1]{{#1}}
    \newcommand{\DocumentationTok}[1]{\textcolor[rgb]{0.73,0.13,0.13}{\textit{{#1}}}}
    \newcommand{\AnnotationTok}[1]{\textcolor[rgb]{0.38,0.63,0.69}{\textbf{\textit{{#1}}}}}
    \newcommand{\CommentVarTok}[1]{\textcolor[rgb]{0.38,0.63,0.69}{\textbf{\textit{{#1}}}}}
    \newcommand{\VariableTok}[1]{\textcolor[rgb]{0.10,0.09,0.49}{{#1}}}
    \newcommand{\ControlFlowTok}[1]{\textcolor[rgb]{0.00,0.44,0.13}{\textbf{{#1}}}}
    \newcommand{\OperatorTok}[1]{\textcolor[rgb]{0.40,0.40,0.40}{{#1}}}
    \newcommand{\BuiltInTok}[1]{{#1}}
    \newcommand{\ExtensionTok}[1]{{#1}}
    \newcommand{\PreprocessorTok}[1]{\textcolor[rgb]{0.74,0.48,0.00}{{#1}}}
    \newcommand{\AttributeTok}[1]{\textcolor[rgb]{0.49,0.56,0.16}{{#1}}}
    \newcommand{\InformationTok}[1]{\textcolor[rgb]{0.38,0.63,0.69}{\textbf{\textit{{#1}}}}}
    \newcommand{\WarningTok}[1]{\textcolor[rgb]{0.38,0.63,0.69}{\textbf{\textit{{#1}}}}}
    
    
    % Define a nice break command that doesn't care if a line doesn't already
    % exist.
    \def\br{\hspace*{\fill} \\* }
    % Math Jax compatibility definitions
    \def\gt{>}
    \def\lt{<}
    \let\Oldtex\TeX
    \let\Oldlatex\LaTeX
    \renewcommand{\TeX}{\textrm{\Oldtex}}
    \renewcommand{\LaTeX}{\textrm{\Oldlatex}}
    % Document parameters
    % Document title
    \title{
      {\Large Лекция 7} \\
      Гауссовские процессы
    }
    % \date{29 марта 2023\,г.}
    \date{}
    
    
    
% Pygments definitions
\makeatletter
\def\PY@reset{\let\PY@it=\relax \let\PY@bf=\relax%
    \let\PY@ul=\relax \let\PY@tc=\relax%
    \let\PY@bc=\relax \let\PY@ff=\relax}
\def\PY@tok#1{\csname PY@tok@#1\endcsname}
\def\PY@toks#1+{\ifx\relax#1\empty\else%
    \PY@tok{#1}\expandafter\PY@toks\fi}
\def\PY@do#1{\PY@bc{\PY@tc{\PY@ul{%
    \PY@it{\PY@bf{\PY@ff{#1}}}}}}}
\def\PY#1#2{\PY@reset\PY@toks#1+\relax+\PY@do{#2}}

\@namedef{PY@tok@w}{\def\PY@tc##1{\textcolor[rgb]{0.73,0.73,0.73}{##1}}}
\@namedef{PY@tok@c}{\let\PY@it=\textit\def\PY@tc##1{\textcolor[rgb]{0.24,0.48,0.48}{##1}}}
\@namedef{PY@tok@cp}{\def\PY@tc##1{\textcolor[rgb]{0.61,0.40,0.00}{##1}}}
\@namedef{PY@tok@k}{\let\PY@bf=\textbf\def\PY@tc##1{\textcolor[rgb]{0.00,0.50,0.00}{##1}}}
\@namedef{PY@tok@kp}{\def\PY@tc##1{\textcolor[rgb]{0.00,0.50,0.00}{##1}}}
\@namedef{PY@tok@kt}{\def\PY@tc##1{\textcolor[rgb]{0.69,0.00,0.25}{##1}}}
\@namedef{PY@tok@o}{\def\PY@tc##1{\textcolor[rgb]{0.40,0.40,0.40}{##1}}}
\@namedef{PY@tok@ow}{\let\PY@bf=\textbf\def\PY@tc##1{\textcolor[rgb]{0.67,0.13,1.00}{##1}}}
\@namedef{PY@tok@nb}{\def\PY@tc##1{\textcolor[rgb]{0.00,0.50,0.00}{##1}}}
\@namedef{PY@tok@nf}{\def\PY@tc##1{\textcolor[rgb]{0.00,0.00,1.00}{##1}}}
\@namedef{PY@tok@nc}{\let\PY@bf=\textbf\def\PY@tc##1{\textcolor[rgb]{0.00,0.00,1.00}{##1}}}
\@namedef{PY@tok@nn}{\let\PY@bf=\textbf\def\PY@tc##1{\textcolor[rgb]{0.00,0.00,1.00}{##1}}}
\@namedef{PY@tok@ne}{\let\PY@bf=\textbf\def\PY@tc##1{\textcolor[rgb]{0.80,0.25,0.22}{##1}}}
\@namedef{PY@tok@nv}{\def\PY@tc##1{\textcolor[rgb]{0.10,0.09,0.49}{##1}}}
\@namedef{PY@tok@no}{\def\PY@tc##1{\textcolor[rgb]{0.53,0.00,0.00}{##1}}}
\@namedef{PY@tok@nl}{\def\PY@tc##1{\textcolor[rgb]{0.46,0.46,0.00}{##1}}}
\@namedef{PY@tok@ni}{\let\PY@bf=\textbf\def\PY@tc##1{\textcolor[rgb]{0.44,0.44,0.44}{##1}}}
\@namedef{PY@tok@na}{\def\PY@tc##1{\textcolor[rgb]{0.41,0.47,0.13}{##1}}}
\@namedef{PY@tok@nt}{\let\PY@bf=\textbf\def\PY@tc##1{\textcolor[rgb]{0.00,0.50,0.00}{##1}}}
\@namedef{PY@tok@nd}{\def\PY@tc##1{\textcolor[rgb]{0.67,0.13,1.00}{##1}}}
\@namedef{PY@tok@s}{\def\PY@tc##1{\textcolor[rgb]{0.73,0.13,0.13}{##1}}}
\@namedef{PY@tok@sd}{\let\PY@it=\textit\def\PY@tc##1{\textcolor[rgb]{0.73,0.13,0.13}{##1}}}
\@namedef{PY@tok@si}{\let\PY@bf=\textbf\def\PY@tc##1{\textcolor[rgb]{0.64,0.35,0.47}{##1}}}
\@namedef{PY@tok@se}{\let\PY@bf=\textbf\def\PY@tc##1{\textcolor[rgb]{0.67,0.36,0.12}{##1}}}
\@namedef{PY@tok@sr}{\def\PY@tc##1{\textcolor[rgb]{0.64,0.35,0.47}{##1}}}
\@namedef{PY@tok@ss}{\def\PY@tc##1{\textcolor[rgb]{0.10,0.09,0.49}{##1}}}
\@namedef{PY@tok@sx}{\def\PY@tc##1{\textcolor[rgb]{0.00,0.50,0.00}{##1}}}
\@namedef{PY@tok@m}{\def\PY@tc##1{\textcolor[rgb]{0.40,0.40,0.40}{##1}}}
\@namedef{PY@tok@gh}{\let\PY@bf=\textbf\def\PY@tc##1{\textcolor[rgb]{0.00,0.00,0.50}{##1}}}
\@namedef{PY@tok@gu}{\let\PY@bf=\textbf\def\PY@tc##1{\textcolor[rgb]{0.50,0.00,0.50}{##1}}}
\@namedef{PY@tok@gd}{\def\PY@tc##1{\textcolor[rgb]{0.63,0.00,0.00}{##1}}}
\@namedef{PY@tok@gi}{\def\PY@tc##1{\textcolor[rgb]{0.00,0.52,0.00}{##1}}}
\@namedef{PY@tok@gr}{\def\PY@tc##1{\textcolor[rgb]{0.89,0.00,0.00}{##1}}}
\@namedef{PY@tok@ge}{\let\PY@it=\textit}
\@namedef{PY@tok@gs}{\let\PY@bf=\textbf}
\@namedef{PY@tok@gp}{\let\PY@bf=\textbf\def\PY@tc##1{\textcolor[rgb]{0.00,0.00,0.50}{##1}}}
\@namedef{PY@tok@go}{\def\PY@tc##1{\textcolor[rgb]{0.44,0.44,0.44}{##1}}}
\@namedef{PY@tok@gt}{\def\PY@tc##1{\textcolor[rgb]{0.00,0.27,0.87}{##1}}}
\@namedef{PY@tok@err}{\def\PY@bc##1{{\setlength{\fboxsep}{\string -\fboxrule}\fcolorbox[rgb]{1.00,0.00,0.00}{1,1,1}{\strut ##1}}}}
\@namedef{PY@tok@kc}{\let\PY@bf=\textbf\def\PY@tc##1{\textcolor[rgb]{0.00,0.50,0.00}{##1}}}
\@namedef{PY@tok@kd}{\let\PY@bf=\textbf\def\PY@tc##1{\textcolor[rgb]{0.00,0.50,0.00}{##1}}}
\@namedef{PY@tok@kn}{\let\PY@bf=\textbf\def\PY@tc##1{\textcolor[rgb]{0.00,0.50,0.00}{##1}}}
\@namedef{PY@tok@kr}{\let\PY@bf=\textbf\def\PY@tc##1{\textcolor[rgb]{0.00,0.50,0.00}{##1}}}
\@namedef{PY@tok@bp}{\def\PY@tc##1{\textcolor[rgb]{0.00,0.50,0.00}{##1}}}
\@namedef{PY@tok@fm}{\def\PY@tc##1{\textcolor[rgb]{0.00,0.00,1.00}{##1}}}
\@namedef{PY@tok@vc}{\def\PY@tc##1{\textcolor[rgb]{0.10,0.09,0.49}{##1}}}
\@namedef{PY@tok@vg}{\def\PY@tc##1{\textcolor[rgb]{0.10,0.09,0.49}{##1}}}
\@namedef{PY@tok@vi}{\def\PY@tc##1{\textcolor[rgb]{0.10,0.09,0.49}{##1}}}
\@namedef{PY@tok@vm}{\def\PY@tc##1{\textcolor[rgb]{0.10,0.09,0.49}{##1}}}
\@namedef{PY@tok@sa}{\def\PY@tc##1{\textcolor[rgb]{0.73,0.13,0.13}{##1}}}
\@namedef{PY@tok@sb}{\def\PY@tc##1{\textcolor[rgb]{0.73,0.13,0.13}{##1}}}
\@namedef{PY@tok@sc}{\def\PY@tc##1{\textcolor[rgb]{0.73,0.13,0.13}{##1}}}
\@namedef{PY@tok@dl}{\def\PY@tc##1{\textcolor[rgb]{0.73,0.13,0.13}{##1}}}
\@namedef{PY@tok@s2}{\def\PY@tc##1{\textcolor[rgb]{0.73,0.13,0.13}{##1}}}
\@namedef{PY@tok@sh}{\def\PY@tc##1{\textcolor[rgb]{0.73,0.13,0.13}{##1}}}
\@namedef{PY@tok@s1}{\def\PY@tc##1{\textcolor[rgb]{0.73,0.13,0.13}{##1}}}
\@namedef{PY@tok@mb}{\def\PY@tc##1{\textcolor[rgb]{0.40,0.40,0.40}{##1}}}
\@namedef{PY@tok@mf}{\def\PY@tc##1{\textcolor[rgb]{0.40,0.40,0.40}{##1}}}
\@namedef{PY@tok@mh}{\def\PY@tc##1{\textcolor[rgb]{0.40,0.40,0.40}{##1}}}
\@namedef{PY@tok@mi}{\def\PY@tc##1{\textcolor[rgb]{0.40,0.40,0.40}{##1}}}
\@namedef{PY@tok@il}{\def\PY@tc##1{\textcolor[rgb]{0.40,0.40,0.40}{##1}}}
\@namedef{PY@tok@mo}{\def\PY@tc##1{\textcolor[rgb]{0.40,0.40,0.40}{##1}}}
\@namedef{PY@tok@ch}{\let\PY@it=\textit\def\PY@tc##1{\textcolor[rgb]{0.24,0.48,0.48}{##1}}}
\@namedef{PY@tok@cm}{\let\PY@it=\textit\def\PY@tc##1{\textcolor[rgb]{0.24,0.48,0.48}{##1}}}
\@namedef{PY@tok@cpf}{\let\PY@it=\textit\def\PY@tc##1{\textcolor[rgb]{0.24,0.48,0.48}{##1}}}
\@namedef{PY@tok@c1}{\let\PY@it=\textit\def\PY@tc##1{\textcolor[rgb]{0.24,0.48,0.48}{##1}}}
\@namedef{PY@tok@cs}{\let\PY@it=\textit\def\PY@tc##1{\textcolor[rgb]{0.24,0.48,0.48}{##1}}}

\def\PYZbs{\char`\\}
\def\PYZus{\char`\_}
\def\PYZob{\char`\{}
\def\PYZcb{\char`\}}
\def\PYZca{\char`\^}
\def\PYZam{\char`\&}
\def\PYZlt{\char`\<}
\def\PYZgt{\char`\>}
\def\PYZsh{\char`\#}
\def\PYZpc{\char`\%}
\def\PYZdl{\char`\$}
\def\PYZhy{\char`\-}
\def\PYZsq{\char`\'}
\def\PYZdq{\char`\"}
\def\PYZti{\char`\~}
% for compatibility with earlier versions
\def\PYZat{@}
\def\PYZlb{[}
\def\PYZrb{]}
\makeatother


    % For linebreaks inside Verbatim environment from package fancyvrb. 
    \makeatletter
        \newbox\Wrappedcontinuationbox 
        \newbox\Wrappedvisiblespacebox 
        \newcommand*\Wrappedvisiblespace {\textcolor{red}{\textvisiblespace}} 
        \newcommand*\Wrappedcontinuationsymbol {\textcolor{red}{\llap{\tiny$\m@th\hookrightarrow$}}} 
        \newcommand*\Wrappedcontinuationindent {3ex } 
        \newcommand*\Wrappedafterbreak {\kern\Wrappedcontinuationindent\copy\Wrappedcontinuationbox} 
        % Take advantage of the already applied Pygments mark-up to insert 
        % potential linebreaks for TeX processing. 
        %        {, <, #, %, $, ' and ": go to next line. 
        %        _, }, ^, &, >, - and ~: stay at end of broken line. 
        % Use of \textquotesingle for straight quote. 
        \newcommand*\Wrappedbreaksatspecials {% 
            \def\PYGZus{\discretionary{\char`\_}{\Wrappedafterbreak}{\char`\_}}% 
            \def\PYGZob{\discretionary{}{\Wrappedafterbreak\char`\{}{\char`\{}}% 
            \def\PYGZcb{\discretionary{\char`\}}{\Wrappedafterbreak}{\char`\}}}% 
            \def\PYGZca{\discretionary{\char`\^}{\Wrappedafterbreak}{\char`\^}}% 
            \def\PYGZam{\discretionary{\char`\&}{\Wrappedafterbreak}{\char`\&}}% 
            \def\PYGZlt{\discretionary{}{\Wrappedafterbreak\char`\<}{\char`\<}}% 
            \def\PYGZgt{\discretionary{\char`\>}{\Wrappedafterbreak}{\char`\>}}% 
            \def\PYGZsh{\discretionary{}{\Wrappedafterbreak\char`\#}{\char`\#}}% 
            \def\PYGZpc{\discretionary{}{\Wrappedafterbreak\char`\%}{\char`\%}}% 
            \def\PYGZdl{\discretionary{}{\Wrappedafterbreak\char`\$}{\char`\$}}% 
            \def\PYGZhy{\discretionary{\char`\-}{\Wrappedafterbreak}{\char`\-}}% 
            \def\PYGZsq{\discretionary{}{\Wrappedafterbreak\textquotesingle}{\textquotesingle}}% 
            \def\PYGZdq{\discretionary{}{\Wrappedafterbreak\char`\"}{\char`\"}}% 
            \def\PYGZti{\discretionary{\char`\~}{\Wrappedafterbreak}{\char`\~}}% 
        } 
        % Some characters . , ; ? ! / are not pygmentized. 
        % This macro makes them "active" and they will insert potential linebreaks 
        \newcommand*\Wrappedbreaksatpunct {% 
            \lccode`\~`\.\lowercase{\def~}{\discretionary{\hbox{\char`\.}}{\Wrappedafterbreak}{\hbox{\char`\.}}}% 
            \lccode`\~`\,\lowercase{\def~}{\discretionary{\hbox{\char`\,}}{\Wrappedafterbreak}{\hbox{\char`\,}}}% 
            \lccode`\~`\;\lowercase{\def~}{\discretionary{\hbox{\char`\;}}{\Wrappedafterbreak}{\hbox{\char`\;}}}% 
            \lccode`\~`\:\lowercase{\def~}{\discretionary{\hbox{\char`\:}}{\Wrappedafterbreak}{\hbox{\char`\:}}}% 
            \lccode`\~`\?\lowercase{\def~}{\discretionary{\hbox{\char`\?}}{\Wrappedafterbreak}{\hbox{\char`\?}}}% 
            \lccode`\~`\!\lowercase{\def~}{\discretionary{\hbox{\char`\!}}{\Wrappedafterbreak}{\hbox{\char`\!}}}% 
            \lccode`\~`\/\lowercase{\def~}{\discretionary{\hbox{\char`\/}}{\Wrappedafterbreak}{\hbox{\char`\/}}}% 
            \catcode`\.\active
            \catcode`\,\active 
            \catcode`\;\active
            \catcode`\:\active
            \catcode`\?\active
            \catcode`\!\active
            \catcode`\/\active 
            \lccode`\~`\~ 	
        }
    \makeatother

    \let\OriginalVerbatim=\Verbatim
    \makeatletter
    \renewcommand{\Verbatim}[1][1]{%
        %\parskip\z@skip
        \sbox\Wrappedcontinuationbox {\Wrappedcontinuationsymbol}%
        \sbox\Wrappedvisiblespacebox {\FV@SetupFont\Wrappedvisiblespace}%
        \def\FancyVerbFormatLine ##1{\hsize\linewidth
            \vtop{\raggedright\hyphenpenalty\z@\exhyphenpenalty\z@
                \doublehyphendemerits\z@\finalhyphendemerits\z@
                \strut ##1\strut}%
        }%
        % If the linebreak is at a space, the latter will be displayed as visible
        % space at end of first line, and a continuation symbol starts next line.
        % Stretch/shrink are however usually zero for typewriter font.
        \def\FV@Space {%
            \nobreak\hskip\z@ plus\fontdimen3\font minus\fontdimen4\font
            \discretionary{\copy\Wrappedvisiblespacebox}{\Wrappedafterbreak}
            {\kern\fontdimen2\font}%
        }%
        
        % Allow breaks at special characters using \PYG... macros.
        \Wrappedbreaksatspecials
        % Breaks at punctuation characters . , ; ? ! and / need catcode=\active 	
        \OriginalVerbatim[#1,codes*=\Wrappedbreaksatpunct]%
    }
    \makeatother

    % Exact colors from NB
    \definecolor{incolor}{HTML}{303F9F}
    \definecolor{outcolor}{HTML}{D84315}
    \definecolor{cellborder}{HTML}{CFCFCF}
    \definecolor{cellbackground}{HTML}{F7F7F7}
    
    % prompt
    \makeatletter
    \newcommand{\boxspacing}{\kern\kvtcb@left@rule\kern\kvtcb@boxsep}
    \makeatother
    \newcommand{\prompt}[4]{
        {\ttfamily\llap{{\color{#2}[#3]:\hspace{3pt}#4}}\vspace{-\baselineskip}}
    }
    

    
    % Prevent overflowing lines due to hard-to-break entities
    \sloppy 
    % Setup hyperref package
    \hypersetup{
      breaklinks=true,  % so long urls are correctly broken across lines
      colorlinks=true,
      urlcolor=urlcolor,
      linkcolor=linkcolor,
      citecolor=citecolor,
      }
    % Slightly bigger margins than the latex defaults
    
    \geometry{verbose,tmargin=1in,bmargin=1in,lmargin=1in,rmargin=1in}
    
    

\begin{document}
    
  \maketitle
  \thispagestyle{empty}
  \tableofcontents

%\let\thefootnote\relax\footnote{
%  \textit{День 6 апреля в истории:
%    \begin{itemize}[topsep=2pt,itemsep=1pt]
%      \item в 1199 г. во время войны с Францией погиб Ричард Львиное Сердце, английский король;
%      \item в 1772 г. Екатерина II отменила введённую Петром I пошлину на ношение бороды;
%      \item в 1814 г. Наполеон I отрёкся от престола, восстановилась династия Бурбонов;
%      \item в 1896 г. в Афинах открылись первые современные Олимпийские игры.
%    \end{itemize}
%  }
%}

  \newpage


%    \begin{tcolorbox}[breakable, size=fbox, boxrule=1pt, pad at break*=1mm,colback=cellbackground, colframe=cellborder]
%\prompt{In}{incolor}{1}{\boxspacing}
%\begin{Verbatim}[commandchars=\\\{\}]
%\PY{c+c1}{\PYZsh{} Imports}
%\PY{k+kn}{import} \PY{n+nn}{numpy} \PY{k}{as} \PY{n+nn}{np}
%\PY{n}{np}\PY{o}{.}\PY{n}{random}\PY{o}{.}\PY{n}{seed}\PY{p}{(}\PY{l+m+mi}{42}\PY{p}{)}
%\PY{k+kn}{from} \PY{n+nn}{scipy} \PY{k+kn}{import} \PY{n}{stats}
%
%\PY{k+kn}{import} \PY{n+nn}{sys}
%\PY{n}{sys}\PY{o}{.}\PY{n}{path}\PY{o}{.}\PY{n}{append}\PY{p}{(}\PY{l+s+s1}{\PYZsq{}}\PY{l+s+s1}{./scripts}\PY{l+s+s1}{\PYZsq{}}\PY{p}{)}
%\PY{k+kn}{from} \PY{n+nn}{GP\PYZus{}utils} \PY{k+kn}{import} \PY{n}{generate\PYZus{}gauss\PYZus{}surface}
%\end{Verbatim}
%\end{tcolorbox}
%
%    \begin{tcolorbox}[breakable, size=fbox, boxrule=1pt, pad at break*=1mm,colback=cellbackground, colframe=cellborder]
%\prompt{In}{incolor}{2}{\boxspacing}
%\begin{Verbatim}[commandchars=\\\{\}]
%\PY{c+c1}{\PYZsh{} Styles, fonts}
%\PY{k+kn}{import} \PY{n+nn}{matplotlib}
%\PY{n}{matplotlib}\PY{o}{.}\PY{n}{rcParams}\PY{p}{[}\PY{l+s+s1}{\PYZsq{}}\PY{l+s+s1}{font.size}\PY{l+s+s1}{\PYZsq{}}\PY{p}{]} \PY{o}{=} \PY{l+m+mi}{12}
%\PY{k+kn}{import} \PY{n+nn}{matplotlib}\PY{n+nn}{.}\PY{n+nn}{pyplot} \PY{k}{as} \PY{n+nn}{plt}
%\PY{k+kn}{from} \PY{n+nn}{matplotlib} \PY{k+kn}{import} \PY{n}{cm} \PY{c+c1}{\PYZsh{} Colormaps}
%
%\PY{k+kn}{import} \PY{n+nn}{seaborn}
%\PY{n}{seaborn}\PY{o}{.}\PY{n}{set\PYZus{}style}\PY{p}{(}\PY{l+s+s1}{\PYZsq{}}\PY{l+s+s1}{whitegrid}\PY{l+s+s1}{\PYZsq{}}\PY{p}{)}
%\end{Verbatim}
%\end{tcolorbox}

%    \begin{tcolorbox}[breakable, size=fbox, boxrule=1pt, pad at break*=1mm,colback=cellbackground, colframe=cellborder]
%\prompt{In}{incolor}{3}{\boxspacing}
%\begin{Verbatim}[commandchars=\\\{\}]
%\PY{c+c1}{\PYZsh{} \PYZpc{}config InlineBackend.figure\PYZus{}formats = [\PYZsq{}pdf\PYZsq{}]}
%\PY{c+c1}{\PYZsh{} \PYZpc{}config Completer.use\PYZus{}jedi = False}
%\end{Verbatim}
%\end{tcolorbox}

%    \begin{center}\rule{0.5\linewidth}{0.5pt}\end{center}

    \hypertarget{ux441ux43bux443ux447ux430ux439ux43dux44bux435-ux43fux440ux43eux446ux435ux441ux441ux44b}{%
\section{Случайные
процессы}\label{ux441ux43bux443ux447ux430ux439ux43dux44bux435-ux43fux440ux43eux446ux435ux441ux441ux44b}}

%Что такое гауссовский процесс? Как можно догадаться из названия, это
%процесс, состоящий из случайных величин, распределённых по Гауссу.
%Точное определение гласит, что гауссовский процесс --- это случайный
%процесс, все конечномерные распределения которого гауссовские. Данное
%определение (хотя оно и абсолютно верное) всё же не до конца проясняет
%суть, поэтому давайте разбираться по порядку.

    \hypertarget{ux431ux430ux437ux43eux432ux44bux435-ux43fux43eux43dux44fux442ux438ux44f-ux438-ux43eux43fux440ux435ux434ux435ux43bux435ux43dux438ux44f}{%
\subsection{Базовые понятия и
определения}\label{ux431ux430ux437ux43eux432ux44bux435-ux43fux43eux43dux44fux442ux438ux44f-ux438-ux43eux43fux440ux435ux434ux435ux43bux435ux43dux438ux44f}}

Вначале теория вероятностей имела дело со \emph{случайными
экспериментами} (подбрасывание монеты, игральной кости и т. п.), для
которых подсчитывались вероятности, в которыми может произойти то или
иное событие. Затем возникло понятие \emph{случайной величины},
позволившее количественно описывать результаты проводимых экспериментов,
например, размер выигрыша в лотерее. Наконец, в случайные эксперименты
был явно введён \emph{фактор времени}, что дало возможность строить
стохастические модели, в основу которых легло понятие \emph{случайного
процесса}, описывающего динамику развития изучаемого случайного явления.

Случайные (или стохастические) процессы обычно описывают системы,
случайно меняющиеся с течением времени. Процессы являются
стохастическими из-за наличия в системе неопределённости. Даже если
исходная точка известна, существует несколько направлений, в которых
такие процессы могут развиваться.

    \textbf{Определение.} \emph{Случайным процессом} называется семейство
случайных величин \(X(\omega, t)\), \(\omega \in \Omega\), заданных на
одном вероятностном пространстве \((\Omega, \mathcal{F}, \mathrm{P})\) и
зависящих от параметра \(t\), принимающего значения из некоторого
множества \(T \in \mathbb{R}\). Параметр \(t\) обычно называют
\emph{временем}.
К случайному процессу всегда следует относиться как к функции двух
переменных: исхода \(\omega\) и времени \(t\). Это независимые
переменные.

\textbf{Определение.} При фиксированном времени \(t = t_0\) случайная
величина \(X(\omega, t_0)\) называется \emph{сечением процесса} в точке
\(t_0\). При фиксированном исходе \(\omega = \omega_0\) функция времени
\(X(\omega_0, t)\) называется \emph{траекторией} (\emph{реализацией},
\emph{выборочной функцией}) процесса.

    \textbf{Пример.}
Известным примером стохастического процесса является броуновское
движение (известное также как винеровский процесс). Броуновское движение
--- это случайное движение частиц, взвешенных в жидкости. Такое движение
может рассматриваться как непрерывное случайное движение, при котором
частица перемещается в жидкости из-за случайного столкновения с ней
других частиц.

Мы можем смоделировать этот процесс во времени \(t\) в одном измерении
\(d\), начиная с точки \(t_0 = 0\) и перемещая частицу за определенное
количество времени \(\Delta t\) на случайное расстояние \(\Delta d\) от
предыдущего положения. Случайное расстояние выбирается из нормального
распределения со средним \(\mu = 0\) и дисперсией
\(\sigma^2 = \Delta t\): \(\Delta d \sim \mathcal{N}(0, \Delta t)\).
Позиция \(d(t)\) изменяется со временем по следующему закону
\(d(t + \Delta t) = d(t) + \Delta d\).

%    \begin{tcolorbox}[breakable, size=fbox, boxrule=1pt, pad at break*=1mm,colback=cellbackground, colframe=cellborder]
%\prompt{In}{incolor}{4}{\boxspacing}
%\begin{Verbatim}[commandchars=\\\{\}]
%\PY{c+c1}{\PYZsh{} 1D simulation of the Brownian motion process}
%\PY{n}{t\PYZus{}max}\PY{p}{,} \PY{n}{n\PYZus{}steps} \PY{o}{=} \PY{l+m+mf}{1.}\PY{p}{,} \PY{l+m+mi}{500}
%\PY{n}{delta\PYZus{}t} \PY{o}{=} \PY{n}{t\PYZus{}max} \PY{o}{/} \PY{n}{n\PYZus{}steps}
%\PY{n}{n\PYZus{}processes} \PY{o}{=} \PY{n+nb}{int}\PY{p}{(}\PY{l+m+mf}{1e5}\PY{p}{)}  \PY{c+c1}{\PYZsh{} Simulate n\PYZus{}processes different motions}
%\PY{n}{mean} \PY{o}{=} \PY{l+m+mf}{0.}     \PY{c+c1}{\PYZsh{} Mean of each movement}
%\PY{n}{sigma\PYZus{}k} \PY{o}{=} \PY{l+m+mf}{1.}  \PY{c+c1}{\PYZsh{} Scale parameter of each movement}
%\PY{n}{std} \PY{o}{=} \PY{n}{sigma\PYZus{}k}\PY{o}{*}\PY{n}{np}\PY{o}{.}\PY{n}{sqrt}\PY{p}{(}\PY{n}{delta\PYZus{}t}\PY{p}{)}  \PY{c+c1}{\PYZsh{} Standard deviation of each movement}
%
%\PY{c+c1}{\PYZsh{} Simulate the brownian motions in a 1D space by cumulatively making a new movement delta\PYZus{}d}
%\PY{c+c1}{\PYZsh{} Move randomly from current location to N(0, delta\PYZus{}t)}
%\PY{n}{t} \PY{o}{=} \PY{n}{np}\PY{o}{.}\PY{n}{linspace}\PY{p}{(}\PY{l+m+mi}{0}\PY{p}{,} \PY{n}{t\PYZus{}max}\PY{p}{,} \PY{n}{n\PYZus{}steps}\PY{o}{+}\PY{l+m+mi}{1}\PY{p}{)}
%\PY{n}{delta\PYZus{}d} \PY{o}{=} \PY{n}{np}\PY{o}{.}\PY{n}{random}\PY{o}{.}\PY{n}{normal}\PY{p}{(}\PY{n}{mean}\PY{p}{,} \PY{n}{std}\PY{p}{,} \PY{p}{(}\PY{n}{n\PYZus{}steps}\PY{p}{,} \PY{n}{n\PYZus{}processes}\PY{p}{)}\PY{p}{)}
%\PY{n}{distances} \PY{o}{=} \PY{n}{np}\PY{o}{.}\PY{n}{zeros}\PY{p}{(}\PY{p}{(}\PY{p}{(}\PY{n}{n\PYZus{}steps}\PY{o}{+}\PY{l+m+mi}{1}\PY{p}{,} \PY{n}{n\PYZus{}processes}\PY{p}{)}\PY{p}{)}\PY{p}{)}
%\PY{n}{distances}\PY{p}{[}\PY{l+m+mi}{1}\PY{p}{:}\PY{p}{]} \PY{o}{=} \PY{n}{np}\PY{o}{.}\PY{n}{cumsum}\PY{p}{(}\PY{n}{delta\PYZus{}d}\PY{p}{,} \PY{n}{axis}\PY{o}{=}\PY{l+m+mi}{0}\PY{p}{)}
%\end{Verbatim}
%\end{tcolorbox}

    На следующем рисунке приводится несколько различных траекторий
броуновского движения, проиллюстрированных разным цветом. Чёрной линией
нарисована зависимость дисперсии процесса от времени:
\(D(t) = \sqrt{t}.\)

%    \begin{tcolorbox}[breakable, size=fbox, boxrule=1pt, pad at break*=1mm,colback=cellbackground, colframe=cellborder]
%\prompt{In}{incolor}{5}{\boxspacing}
%\begin{Verbatim}[commandchars=\\\{\}]
%\PY{c+c1}{\PYZsh{} Make the plots}
%\PY{n}{n\PYZus{}draw} \PY{o}{=} \PY{l+m+mi}{200}
%\PY{n}{std\PYZus{}t} \PY{o}{=} \PY{n}{sigma\PYZus{}k}\PY{o}{*}\PY{n}{t}\PY{o}{*}\PY{o}{*}\PY{l+m+mf}{0.5}
%\PY{n}{plt}\PY{o}{.}\PY{n}{figure}\PY{p}{(}\PY{n}{figsize}\PY{o}{=}\PY{p}{(}\PY{l+m+mi}{8}\PY{p}{,} \PY{l+m+mi}{5}\PY{p}{)}\PY{p}{)}
%\PY{n}{plt}\PY{o}{.}\PY{n}{title}\PY{p}{(}\PY{l+s+sa}{f}\PY{l+s+s1}{\PYZsq{}}\PY{l+s+s1}{Процесс броуновского движения,}\PY{l+s+se}{\PYZbs{}n}\PY{l+s+se}{\PYZbs{}}
%\PY{l+s+s1}{    траектории }\PY{l+s+si}{\PYZob{}}\PY{n}{n\PYZus{}draw}\PY{l+s+si}{\PYZcb{}}\PY{l+s+s1}{ реализаций процесса}\PY{l+s+s1}{\PYZsq{}}\PY{p}{)}
%
%\PY{k}{for} \PY{n}{i} \PY{o+ow}{in} \PY{n+nb}{range}\PY{p}{(}\PY{n}{n\PYZus{}draw}\PY{p}{)}\PY{p}{:}
%    \PY{n}{plt}\PY{o}{.}\PY{n}{plot}\PY{p}{(}\PY{n}{t}\PY{p}{,} \PY{n}{distances}\PY{p}{[}\PY{p}{:}\PY{p}{,}\PY{n}{i}\PY{p}{]}\PY{p}{,} \PY{n}{lw}\PY{o}{=}\PY{l+m+mf}{.3}\PY{p}{)}
%\PY{n}{plt}\PY{o}{.}\PY{n}{plot}\PY{p}{(}\PY{n}{t}\PY{p}{,} \PY{o}{\PYZhy{}}\PY{l+m+mi}{2}\PY{o}{*}\PY{n}{std\PYZus{}t}\PY{p}{,} \PY{l+s+s1}{\PYZsq{}}\PY{l+s+s1}{k\PYZhy{}}\PY{l+s+s1}{\PYZsq{}}\PY{p}{,} \PY{n}{lw}\PY{o}{=}\PY{l+m+mf}{1.}\PY{p}{)}
%\PY{n}{plt}\PY{o}{.}\PY{n}{plot}\PY{p}{(}\PY{n}{t}\PY{p}{,}  \PY{l+m+mi}{2}\PY{o}{*}\PY{n}{std\PYZus{}t}\PY{p}{,} \PY{l+s+s1}{\PYZsq{}}\PY{l+s+s1}{k\PYZhy{}}\PY{l+s+s1}{\PYZsq{}}\PY{p}{,} \PY{n}{lw}\PY{o}{=}\PY{l+m+mf}{1.}\PY{p}{)}
%
%\PY{n}{plt}\PY{o}{.}\PY{n}{xlabel}\PY{p}{(}\PY{l+s+s1}{\PYZsq{}}\PY{l+s+s1}{\PYZdl{}t\PYZdl{}}\PY{l+s+s1}{\PYZsq{}}\PY{p}{)}
%\PY{n}{plt}\PY{o}{.}\PY{n}{ylabel}\PY{p}{(}\PY{l+s+s1}{\PYZsq{}}\PY{l+s+s1}{\PYZdl{}d(t)\PYZdl{}}\PY{l+s+s1}{\PYZsq{}}\PY{p}{)}
%\PY{n}{plt}\PY{o}{.}\PY{n}{xlim}\PY{p}{(}\PY{p}{[}\PY{l+m+mi}{0}\PY{p}{,} \PY{n}{t\PYZus{}max}\PY{p}{]}\PY{p}{)}
%\PY{n}{plt}\PY{o}{.}\PY{n}{ylim}\PY{p}{(}\PY{p}{[}\PY{o}{\PYZhy{}}\PY{l+m+mi}{3}\PY{o}{*}\PY{n}{sigma\PYZus{}k}\PY{o}{*}\PY{n}{t\PYZus{}max}\PY{o}{*}\PY{o}{*}\PY{l+m+mf}{0.5}\PY{p}{,} \PY{l+m+mi}{3}\PY{o}{*}\PY{n}{sigma\PYZus{}k}\PY{o}{*}\PY{n}{t\PYZus{}max}\PY{o}{*}\PY{o}{*}\PY{l+m+mf}{0.5}\PY{p}{]}\PY{p}{)}
%\PY{n}{plt}\PY{o}{.}\PY{n}{tight\PYZus{}layout}\PY{p}{(}\PY{p}{)}
%\PY{n}{plt}\PY{o}{.}\PY{n}{show}\PY{p}{(}\PY{p}{)}
%\end{Verbatim}
%\end{tcolorbox}

    \begin{center}
    \adjustimage{max size={0.6\linewidth}{0.6\paperheight}}{Brownian_motion.pdf}
    \end{center}
%    { \hspace*{\fill} \\}

%    Нарисуем гистограмму значений \(d(t^\ast)\) в каком-либо сечении
%\(t^\ast = \mathrm{const}\). На рисунке видно, что распределение
%\(d(t^\ast)\) является гауссовым.
%
%    \begin{tcolorbox}[breakable, size=fbox, boxrule=1pt, pad at break*=1mm,colback=cellbackground, colframe=cellborder]
%\prompt{In}{incolor}{6}{\boxspacing}
%\begin{Verbatim}[commandchars=\\\{\}]
%\PY{c+c1}{\PYZsh{} Make section}
%\PY{n}{t\PYZus{}star} \PY{o}{=} \PY{l+m+mf}{0.49}
%\PY{n}{i\PYZus{}sec} \PY{o}{=} \PY{n}{np}\PY{o}{.}\PY{n}{flatnonzero}\PY{p}{(}\PY{n}{t} \PY{o}{==} \PY{n}{t\PYZus{}star}\PY{p}{)}\PY{p}{[}\PY{l+m+mi}{0}\PY{p}{]}
%\PY{n}{d\PYZus{}star} \PY{o}{=} \PY{n}{distances}\PY{p}{[}\PY{n}{i\PYZus{}sec}\PY{p}{]}
%\PY{n}{sigma} \PY{o}{=} \PY{n}{sigma\PYZus{}k}\PY{o}{*}\PY{n}{t\PYZus{}star}\PY{o}{*}\PY{o}{*}\PY{l+m+mf}{0.5}
%
%\PY{n}{xlim} \PY{o}{=} \PY{p}{[}\PY{o}{\PYZhy{}}\PY{l+m+mf}{3.5}\PY{o}{*}\PY{n}{sigma}\PY{p}{,} \PY{l+m+mf}{3.5}\PY{o}{*}\PY{n}{sigma}\PY{p}{]}
%\PY{n}{x} \PY{o}{=} \PY{n}{np}\PY{o}{.}\PY{n}{linspace}\PY{p}{(}\PY{o}{*}\PY{n}{xlim}\PY{p}{,}\PY{l+m+mi}{101}\PY{p}{)}
%\PY{n}{y} \PY{o}{=} \PY{n}{stats}\PY{o}{.}\PY{n}{norm}\PY{o}{.}\PY{n}{pdf}\PY{p}{(}\PY{n}{x}\PY{p}{,} \PY{n}{loc}\PY{o}{=}\PY{n}{mean}\PY{p}{,} \PY{n}{scale}\PY{o}{=}\PY{n}{sigma}\PY{p}{)}
%\end{Verbatim}
%\end{tcolorbox}
%
%    \begin{tcolorbox}[breakable, size=fbox, boxrule=1pt, pad at break*=1mm,colback=cellbackground, colframe=cellborder]
%\prompt{In}{incolor}{7}{\boxspacing}
%\begin{Verbatim}[commandchars=\\\{\}]
%\PY{c+c1}{\PYZsh{} Draw section histogram}
%\PY{n}{plt}\PY{o}{.}\PY{n}{figure}\PY{p}{(}\PY{n}{figsize}\PY{o}{=}\PY{p}{(}\PY{l+m+mi}{7}\PY{p}{,} \PY{l+m+mi}{4}\PY{p}{)}\PY{p}{)}
%\PY{n}{plt}\PY{o}{.}\PY{n}{title}\PY{p}{(}\PY{l+s+sa}{f}\PY{l+s+s1}{\PYZsq{}}\PY{l+s+s1}{Гистограмма значений \PYZdl{}d(t)\PYZdl{} в сечении \PYZdl{}t=}\PY{l+s+si}{\PYZob{}}\PY{n}{t\PYZus{}star}\PY{l+s+si}{\PYZcb{}}\PY{l+s+s1}{\PYZdl{}}\PY{l+s+s1}{\PYZsq{}}\PY{p}{)}
%\PY{n}{plt}\PY{o}{.}\PY{n}{hist}\PY{p}{(}\PY{n}{d\PYZus{}star}\PY{p}{,} \PY{n}{bins}\PY{o}{=}\PY{l+m+mi}{100}\PY{p}{,} \PY{n}{density}\PY{o}{=}\PY{k+kc}{True}\PY{p}{)}
%\PY{n}{plt}\PY{o}{.}\PY{n}{plot}\PY{p}{(}\PY{n}{x}\PY{p}{,} \PY{n}{y}\PY{p}{,} \PY{l+s+s1}{\PYZsq{}}\PY{l+s+s1}{k\PYZhy{}}\PY{l+s+s1}{\PYZsq{}}\PY{p}{,} \PY{n}{label}\PY{o}{=}\PY{l+s+sa}{f}\PY{l+s+s1}{\PYZsq{}}\PY{l+s+s1}{\PYZdl{}}\PY{l+s+s1}{\PYZbs{}}\PY{l+s+s1}{mathcal}\PY{l+s+se}{\PYZob{}\PYZob{}}\PY{l+s+s1}{N}\PY{l+s+se}{\PYZcb{}\PYZcb{}}\PY{l+s+s1}{(}\PY{l+s+si}{\PYZob{}}\PY{n}{mean}\PY{l+s+si}{:}\PY{l+s+s1}{.1g}\PY{l+s+si}{\PYZcb{}}\PY{l+s+s1}{, }\PY{l+s+si}{\PYZob{}}\PY{n}{sigma}\PY{l+s+si}{:}\PY{l+s+s1}{.3g}\PY{l+s+si}{\PYZcb{}}\PY{l+s+s1}{)\PYZdl{}}\PY{l+s+s1}{\PYZsq{}}\PY{p}{)}
%\PY{n}{plt}\PY{o}{.}\PY{n}{xlim}\PY{p}{(}\PY{n}{xlim}\PY{p}{)}
%\PY{n}{plt}\PY{o}{.}\PY{n}{legend}\PY{p}{(}\PY{p}{)}
%\PY{n}{plt}\PY{o}{.}\PY{n}{tight\PYZus{}layout}\PY{p}{(}\PY{p}{)}
%\PY{n}{plt}\PY{o}{.}\PY{n}{show}\PY{p}{(}\PY{p}{)}
%\end{Verbatim}
%\end{tcolorbox}
%
%    \begin{center}
%    \adjustimage{max size={0.65\linewidth}{0.65\paperheight}}{Hist_t049.pdf}
%    \end{center}
%%    { \hspace*{\fill} \\}

    \hypertarget{ux43cux43eux43cux435ux43dux442ux43dux44bux435-ux445ux430ux440ux430ux43aux442ux435ux440ux438ux441ux442ux438ux43aux438-ux43fux440ux43eux446ux435ux441ux441ux43eux432}{%
\subsection{Моментные характеристики
процессов}\label{ux43cux43eux43cux435ux43dux442ux43dux44bux435-ux445ux430ux440ux430ux43aux442ux435ux440ux438ux441ux442ux438ux43aux438-ux43fux440ux43eux446ux435ux441ux441ux43eux432}}

На рисунке выше можно видеть несколько траекторий стохастического
процесса. Каждая реализация определяет позицию \(d\) для каждого
возможного временного шага \(t\). Таким образом, каждая реализация
соответствует функции \(f(t) = d\).

Это означает, что случайный процесс можно интерпретировать как случайное
распределение функции. Мы можем получить реализацию функции с помощью
стохастического процесса и каждая реализация будет различной из-за
случайности этого процесса.

\textbf{Определение.} \emph{Математическим ожиданием} случайного
процесса \(X(t)\) называется функция \(m_x : T \rightarrow \mathbb{R}\),
значение который в каждый момент времени \(t \in T\) равно
математическому ожиданию соответствующего сечения:
\(m_x(t) = \mathrm{E}[X(t)]\).

\textbf{Определение.} \emph{Ковариационной функцией} случайного процесса
\(X(t)\) называется функция двух переменных
\(k : T \times T \rightarrow \mathbb{R}\), которая каждой паре моментов
времени сопоставляет корреляционный момент соответствующих сечений
процесса: \[
  k(t_1, t_2) = \mathrm{E} \left[ \left(X(t_1) - \mathrm{E}X(t_1)\right) \cdot \left(X(t_2) - \mathrm{E}X(t_2)\right) \right].
\]

    \begin{center}\rule{0.5\linewidth}{0.5pt}\end{center}

    \hypertarget{ux433ux430ux443ux441ux441ux43eux432ux441ux43aux438ux435-ux43fux440ux43eux446ux435ux441ux441ux44b}{%
\section{Гауссовские
процессы}\label{ux433ux430ux443ux441ux441ux43eux432ux441ux43aux438ux435-ux43fux440ux43eux446ux435ux441ux441ux44b}}

    \hypertarget{ux431ux430ux437ux43eux432ux44bux435-ux43fux43eux43dux44fux442ux438ux44f-ux438-ux43eux43fux440ux435ux434ux435ux43bux435ux43dux438ux44f}{%
\subsection{Базовые понятия и
определения}\label{ux431ux430ux437ux43eux432ux44bux435-ux43fux43eux43dux44fux442ux438ux44f-ux438-ux43eux43fux440ux435ux434ux435ux43bux435ux43dux438ux44f}}

\textbf{Определение.} Случайный процесс \(Y(x)\) называется
\emph{гауссовским}, если для любого \(n \ge 1\) и точек
\(x_1 < \ldots < x_n\) вектор \((Y(x_1), \, \ldots \, , Y(x_n))\)
является нормальным случайным вектором.

Другими словами, гауссовский процесс --- это процесс, все конечномерные
распределения которого нормальные.

Гауссовский процесс --- это распределение функций \(f(x)\), которое
определяется средней функцией \(m(x)\) и положительной ковариационной
функцией \(k(x,x')\), где \(x\) --- параметр функции, а \((x,x')\) ---
все возможные пары из области определения. Обозначаются гауссовский
процесс так: \[
  f(x) \sim \mathcal{GP}(m(x), k(x,x')).
\]

    Для любого конечного подмножества \(X=\{x_1 \ldots x_n \}\) области
определения \(x\) распределение \(f(X)\) представляет собой многомерное
гауссовское распределение \[
  f(X) \sim \mathcal{N}(m(X), k(X, X))
\] со средним вектором \(\mathbf{\mu} = m(X)\) и ковариационной матрицей
\(\Sigma = k(X, X)\).

В то время как многомерное гауссовское распределение задаёт конечное
количество совместно распределённых по Гауссу величин, гауссовский
процесс не имеет этого ограничения. Его среднее и ковариация
определяются функциями. Каждый вход в эту функцию является переменной,
коррелирующей с другими переменными входного домена в соответствии с
ковариационной функцией. Поскольку функции могут иметь бесконечный
входной домен, гауссовский процесс можно интерпретировать как
бесконечную размерную гауссовскую случайную величину.

    \hypertarget{ux43aux43eux432ux430ux440ux438ux430ux446ux438ux43eux43dux43dux430ux44f-ux444ux443ux43dux43aux446ux438ux44f}{%
\subsection{Ковариационная
функция}\label{ux43aux43eux432ux430ux440ux438ux430ux446ux438ux43eux43dux43dux430ux44f-ux444ux443ux43dux43aux446ux438ux44f}}

Гауссовский процесс полностью определяется функцией среднего и
ковариационной функцией. Ковариационная функция \(k(x, x')\) моделирует
совместную изменчивость случайных переменных гауссовского процесса, она
возвращает значение ковариации между каждой парой \((x, x')\).

Спецификация ковариационной функции (также известной как функция ядра)
неявно задаёт распределение по функциям \(f(x)\). Выбирая конкретный вид
функции ядра \(k\), мы задаём априорную информацию о данном
распределении. Функция ядра должна быть симметричной и
положительно-определённой.

    Рассмотрим квадратичное экспоненциальное (гауссовское) ядро: \[
  k(x, x') = \sigma_k^2 \exp{ \left( -\frac{\lVert x - x' \rVert^2}{{2\ell^2}} \right) }.
\]

Параметр длины \(l\) контролирует гладкость функции, а параметр амплитуды
\(\sigma_k\) --- вертикальную вариацию. В многомерном случае обычно
используется один и тот же параметр длины \(l\) для всех компонент
вектора \(x\) (изотропное ядро). Могут быть определены и другие функции
ядра, приводящие к различным свойствам гауссовского процесса.

%    \begin{tcolorbox}[breakable, size=fbox, boxrule=1pt, pad at break*=1mm,colback=cellbackground, colframe=cellborder]
%\prompt{In}{incolor}{8}{\boxspacing}
%\begin{Verbatim}[commandchars=\\\{\}]
%\PY{c+c1}{\PYZsh{} Isotropic squared exponential kernel}
%\PY{k}{def} \PY{n+nf}{gauss\PYZus{}kernel}\PY{p}{(}\PY{n}{X1}\PY{p}{,} \PY{n}{X2}\PY{p}{,} \PY{n}{l}\PY{o}{=}\PY{l+m+mf}{1.0}\PY{p}{,} \PY{n}{sigma\PYZus{}k}\PY{o}{=}\PY{l+m+mf}{1.0}\PY{p}{)}\PY{p}{:}
%    \PY{l+s+sd}{\PYZsq{}\PYZsq{}\PYZsq{}}
%\PY{l+s+sd}{    Isotropic squared exponential kernel.}
%\PY{l+s+sd}{    Computes a covariance matrix from points in X1 and X2.}
%\PY{l+s+sd}{    }
%\PY{l+s+sd}{    Args:}
%\PY{l+s+sd}{        X1: Array of m points (m x d)}
%\PY{l+s+sd}{        X2: Array of n points (n x d)}
%\PY{l+s+sd}{        sigma\PYZus{}k: Kernel vertical variation parameter}
%
%\PY{l+s+sd}{    Returns:}
%\PY{l+s+sd}{        Covariance matrix (m x n)}
%\PY{l+s+sd}{    \PYZsq{}\PYZsq{}\PYZsq{}}
%    
%    \PY{n}{sqdist} \PY{o}{=} \PY{n}{np}\PY{o}{.}\PY{n}{sum}\PY{p}{(}\PY{n}{X1}\PY{o}{*}\PY{o}{*}\PY{l+m+mi}{2}\PY{p}{,}\PY{l+m+mi}{1}\PY{p}{)}\PY{o}{.}\PY{n}{reshape}\PY{p}{(}\PY{o}{\PYZhy{}}\PY{l+m+mi}{1}\PY{p}{,}\PY{l+m+mi}{1}\PY{p}{)} \PY{o}{+} \PY{n}{np}\PY{o}{.}\PY{n}{sum}\PY{p}{(}\PY{n}{X2}\PY{o}{*}\PY{o}{*}\PY{l+m+mi}{2}\PY{p}{,}\PY{l+m+mi}{1}\PY{p}{)} \PY{o}{\PYZhy{}} \PY{l+m+mi}{2}\PY{o}{*}\PY{n}{np}\PY{o}{.}\PY{n}{dot}\PY{p}{(}\PY{n}{X1}\PY{p}{,}\PY{n}{X2}\PY{o}{.}\PY{n}{T}\PY{p}{)}
%    \PY{k}{return} \PY{n}{sigma\PYZus{}k}\PY{o}{*}\PY{o}{*}\PY{l+m+mi}{2} \PY{o}{*} \PY{n}{np}\PY{o}{.}\PY{n}{exp}\PY{p}{(}\PY{o}{\PYZhy{}}\PY{l+m+mf}{0.5} \PY{o}{/} \PY{n}{l}\PY{o}{*}\PY{o}{*}\PY{l+m+mi}{2} \PY{o}{*} \PY{n}{sqdist}\PY{p}{)}
%\end{Verbatim}
%\end{tcolorbox}

    Пример ковариационной матрицы с гауссовским ядром приведён на рисунке
слева внизу. Справа показано одномерное сечение ковариационной функции
\(k(x,0)\).

%    \begin{tcolorbox}[breakable, size=fbox, boxrule=1pt, pad at break*=1mm,colback=cellbackground, colframe=cellborder]
%\prompt{In}{incolor}{9}{\boxspacing}
%\begin{Verbatim}[commandchars=\\\{\}]
%\PY{n}{xlim} \PY{o}{=} \PY{p}{(}\PY{o}{\PYZhy{}}\PY{l+m+mi}{3}\PY{p}{,} \PY{l+m+mi}{3}\PY{p}{)}
%\PY{n}{X} \PY{o}{=} \PY{n}{np}\PY{o}{.}\PY{n}{reshape}\PY{p}{(}\PY{n}{np}\PY{o}{.}\PY{n}{linspace}\PY{p}{(}\PY{o}{*}\PY{n}{xlim}\PY{p}{,} \PY{l+m+mi}{25}\PY{p}{)}\PY{p}{,} \PY{p}{(}\PY{o}{\PYZhy{}}\PY{l+m+mi}{1}\PY{p}{,} \PY{l+m+mi}{1}\PY{p}{)}\PY{p}{)}
%\PY{n}{Sigma} \PY{o}{=} \PY{n}{gauss\PYZus{}kernel}\PY{p}{(}\PY{n}{X}\PY{p}{,} \PY{n}{X}\PY{p}{,} \PY{n}{l}\PY{o}{=}\PY{l+m+mf}{1.}\PY{p}{)}
%
%\PY{n}{zero} \PY{o}{=} \PY{n}{np}\PY{o}{.}\PY{n}{reshape}\PY{p}{(}\PY{p}{[}\PY{l+m+mi}{0}\PY{p}{]}\PY{p}{,} \PY{p}{(}\PY{o}{\PYZhy{}}\PY{l+m+mi}{1}\PY{p}{,} \PY{l+m+mi}{1}\PY{p}{)}\PY{p}{)}
%\PY{n}{Sigma\PYZus{}0} \PY{o}{=} \PY{n}{gauss\PYZus{}kernel}\PY{p}{(}\PY{n}{X}\PY{p}{,} \PY{n}{zero}\PY{p}{,} \PY{n}{l}\PY{o}{=}\PY{l+m+mf}{1.}\PY{p}{)}
%\end{Verbatim}
%\end{tcolorbox}

%    \begin{tcolorbox}[breakable, size=fbox, boxrule=1pt, pad at break*=1mm,colback=cellbackground, colframe=cellborder]
%\prompt{In}{incolor}{10}{\boxspacing}
%\begin{Verbatim}[commandchars=\\\{\}]
%\PY{c+c1}{\PYZsh{} Show covariance matrix example from exponentiated quadratic}
%\PY{n}{seaborn}\PY{o}{.}\PY{n}{set\PYZus{}style}\PY{p}{(}\PY{l+s+s1}{\PYZsq{}}\PY{l+s+s1}{white}\PY{l+s+s1}{\PYZsq{}}\PY{p}{)}
%\PY{n}{fig}\PY{p}{,} \PY{p}{(}\PY{n}{ax1}\PY{p}{,} \PY{n}{ax2}\PY{p}{)} \PY{o}{=} \PY{n}{plt}\PY{o}{.}\PY{n}{subplots}\PY{p}{(}\PY{l+m+mi}{1}\PY{p}{,} \PY{l+m+mi}{2}\PY{p}{,} \PY{n}{figsize}\PY{o}{=}\PY{p}{(}\PY{l+m+mi}{10}\PY{p}{,} \PY{l+m+mf}{4.5}\PY{p}{)}\PY{p}{)}
%
%\PY{c+c1}{\PYZsh{} Plot covariance matrix}
%\PY{n}{im} \PY{o}{=} \PY{n}{ax1}\PY{o}{.}\PY{n}{imshow}\PY{p}{(}\PY{n}{Sigma}\PY{p}{,} \PY{n}{cmap}\PY{o}{=}\PY{n}{cm}\PY{o}{.}\PY{n}{YlGnBu}\PY{p}{)}
%\PY{n}{cbar} \PY{o}{=} \PY{n}{fig}\PY{o}{.}\PY{n}{colorbar}\PY{p}{(}\PY{n}{im}\PY{p}{,} \PY{n}{ax}\PY{o}{=}\PY{n}{ax1}\PY{p}{,} \PY{n}{fraction}\PY{o}{=}\PY{l+m+mf}{0.045}\PY{p}{,} \PY{n}{pad}\PY{o}{=}\PY{l+m+mf}{0.05}\PY{p}{)}
%\PY{n}{cbar}\PY{o}{.}\PY{n}{ax}\PY{o}{.}\PY{n}{set\PYZus{}ylabel}\PY{p}{(}\PY{l+s+s1}{\PYZsq{}}\PY{l+s+s1}{\PYZdl{}k(x\PYZus{}1,x\PYZus{}2)\PYZdl{}}\PY{l+s+s1}{\PYZsq{}}\PY{p}{)}
%\PY{n}{ax1}\PY{o}{.}\PY{n}{set\PYZus{}title}\PY{p}{(}\PY{l+s+s1}{\PYZsq{}}\PY{l+s+s1}{Ковариационная матрица}\PY{l+s+se}{\PYZbs{}n}\PY{l+s+se}{\PYZbs{}}
%\PY{l+s+s1}{  для гауссовского ядра}\PY{l+s+s1}{\PYZsq{}}\PY{p}{,} \PY{n}{pad}\PY{o}{=}\PY{l+m+mi}{10}\PY{p}{)}
%\PY{n}{ax1}\PY{o}{.}\PY{n}{set\PYZus{}xlabel}\PY{p}{(}\PY{l+s+s1}{\PYZsq{}}\PY{l+s+s1}{\PYZdl{}x\PYZus{}1\PYZdl{}}\PY{l+s+s1}{\PYZsq{}}\PY{p}{)}
%\PY{n}{ax1}\PY{o}{.}\PY{n}{set\PYZus{}ylabel}\PY{p}{(}\PY{l+s+s1}{\PYZsq{}}\PY{l+s+s1}{\PYZdl{}x\PYZus{}2\PYZdl{}}\PY{l+s+s1}{\PYZsq{}}\PY{p}{)}
%\PY{n}{labels} \PY{o}{=} \PY{n+nb}{list}\PY{p}{(}\PY{n+nb}{range}\PY{p}{(}\PY{n}{xlim}\PY{p}{[}\PY{l+m+mi}{0}\PY{p}{]}\PY{p}{,} \PY{n}{xlim}\PY{p}{[}\PY{l+m+mi}{1}\PY{p}{]}\PY{o}{+}\PY{l+m+mi}{1}\PY{p}{)}\PY{p}{)}
%\PY{n}{ticks} \PY{o}{=} \PY{n}{np}\PY{o}{.}\PY{n}{linspace}\PY{p}{(}\PY{l+m+mi}{0}\PY{p}{,} \PY{n+nb}{len}\PY{p}{(}\PY{n}{X}\PY{p}{)}\PY{o}{\PYZhy{}}\PY{l+m+mi}{1}\PY{p}{,} \PY{n+nb}{len}\PY{p}{(}\PY{n}{labels}\PY{p}{)}\PY{p}{)}
%\PY{n}{ax1}\PY{o}{.}\PY{n}{set\PYZus{}xticks}\PY{p}{(}\PY{n}{ticks}\PY{p}{)}
%\PY{n}{ax1}\PY{o}{.}\PY{n}{set\PYZus{}yticks}\PY{p}{(}\PY{n}{ticks}\PY{p}{)}
%\PY{n}{ax1}\PY{o}{.}\PY{n}{set\PYZus{}xticklabels}\PY{p}{(}\PY{n}{labels}\PY{p}{)}
%\PY{n}{ax1}\PY{o}{.}\PY{n}{set\PYZus{}yticklabels}\PY{p}{(}\PY{n}{labels}\PY{p}{)}
%\PY{n}{ax1}\PY{o}{.}\PY{n}{grid}\PY{p}{(}\PY{k+kc}{False}\PY{p}{)}
%
%\PY{c+c1}{\PYZsh{} Plot covariance with X=0}
%\PY{n}{ax2}\PY{o}{.}\PY{n}{plot}\PY{p}{(}\PY{n}{X}\PY{p}{,} \PY{n}{Sigma\PYZus{}0}\PY{p}{,} \PY{n}{label}\PY{o}{=}\PY{l+s+s1}{\PYZsq{}}\PY{l+s+s1}{\PYZdl{}k(x,0)\PYZdl{}}\PY{l+s+s1}{\PYZsq{}}\PY{p}{)}
%\PY{n}{ax2}\PY{o}{.}\PY{n}{set\PYZus{}xlabel}\PY{p}{(}\PY{l+s+s1}{\PYZsq{}}\PY{l+s+s1}{\PYZdl{}x\PYZdl{}}\PY{l+s+s1}{\PYZsq{}}\PY{p}{)}
%\PY{n}{ax2}\PY{o}{.}\PY{n}{set\PYZus{}ylabel}\PY{p}{(}\PY{l+s+s1}{\PYZsq{}}\PY{l+s+s1}{\PYZdl{}k(x)\PYZdl{}}\PY{l+s+s1}{\PYZsq{}}\PY{p}{)}
%\PY{n}{ax2}\PY{o}{.}\PY{n}{set\PYZus{}title}\PY{p}{(}\PY{l+s+s1}{\PYZsq{}}\PY{l+s+s1}{Ковариация между \PYZdl{}x\PYZdl{} и \PYZdl{}0\PYZdl{}}\PY{l+s+s1}{\PYZsq{}}\PY{p}{,}\PY{n}{pad}\PY{o}{=}\PY{l+m+mi}{10}\PY{p}{)}
%\PY{c+c1}{\PYZsh{} ax2.set\PYZus{}ylim([0, 1.1])}
%\PY{n}{ax2}\PY{o}{.}\PY{n}{set\PYZus{}xlim}\PY{p}{(}\PY{o}{*}\PY{n}{xlim}\PY{p}{)}
%\PY{n}{ax2}\PY{o}{.}\PY{n}{legend}\PY{p}{(}\PY{n}{loc}\PY{o}{=}\PY{l+m+mi}{1}\PY{p}{)}
%\PY{n}{ax2}\PY{o}{.}\PY{n}{grid}\PY{p}{(}\PY{k+kc}{True}\PY{p}{)}
%
%\PY{n}{plt}\PY{o}{.}\PY{n}{tight\PYZus{}layout}\PY{p}{(}\PY{p}{)}
%\PY{n}{plt}\PY{o}{.}\PY{n}{show}\PY{p}{(}\PY{p}{)}
%\end{Verbatim}
%\end{tcolorbox}

    \begin{center}
    \adjustimage{max size={0.75\linewidth}{0.75\paperheight}}{CovMatrix_gauss_kernel.pdf}
    \end{center}
%    { \hspace*{\fill} \\}

    \begin{center}\rule{0.5\linewidth}{0.5pt}\end{center}

    \hypertarget{ux433ux435ux43dux435ux440ux430ux446ux438ux44f-ux440ux435ux430ux43bux438ux437ux430ux446ux438ux439-ux433ux430ux443ux441ux441ux43eux432ux441ux43aux43eux433ux43e-ux43fux440ux43eux446ux435ux441ux441ux430}{%
\section{Генерация реализаций гауссовского
процесса}\label{ux433ux435ux43dux435ux440ux430ux446ux438ux44f-ux440ux435ux430ux43bux438ux437ux430ux446ux438ux439-ux433ux430ux443ux441ux441ux43eux432ux441ux43aux43eux433ux43e-ux43fux440ux43eux446ux435ux441ux441ux430}}

    \hypertarget{ux433ux435ux43dux435ux440ux430ux446ux438ux44f-ux43cux43dux43eux433ux43eux43cux435ux440ux43dux44bux445-ux441ux435ux447ux435ux43dux438ux439-ux43fux440ux43eux446ux435ux441ux441ux430}{%
\subsection{Генерация многомерных сечений
процесса}\label{ux433ux435ux43dux435ux440ux430ux446ux438ux44f-ux43cux43dux43eux433ux43eux43cux435ux440ux43dux44bux445-ux441ux435ux447ux435ux43dux438ux439-ux43fux440ux43eux446ux435ux441ux441ux430}}

На практике мы не можем сгенерировать полную функциональную реализацию
\(f\) гауссовского процесса, так как для этого потребовалось бы
вычислить значения \(m(x)\) и \(k(x,x')\) в бесконечном количестве
точек. Однако мы можем построить сколь угодно близкую дискретную
аппроксимацию \(y\) функции \(f\). Для этого необходимо вычислить
значение \(f\) на сколь угодно большом, но конечном наборе точек \(X\):
\(y = f(X)\).

Полученное конечномерное сечение гауссовского процесса является
гауссовским вектором \(y \sim \mathcal{N}(\mu, \Sigma)\) с
математическим ожиданием \(\mu = m(X)\) и ковариационной матрицей
\(\Sigma = k(X, X)\).

На рисунке ниже приведена выборка из 5 различных реализаций гауссовского
процесса с гауссовским ядром. Фактически, на рисунке представлены 5
векторов, подчиняющихся 51-мерному гауссовскому распределению
\(\mathcal{N}(0, k(X, X))\) при \(X = [x_1, \ldots, x_{51}]\).

%    \begin{tcolorbox}[breakable, size=fbox, boxrule=1pt, pad at break*=1mm,colback=cellbackground, colframe=cellborder]
%\prompt{In}{incolor}{11}{\boxspacing}
%\begin{Verbatim}[commandchars=\\\{\}]
%\PY{k}{def} \PY{n+nf}{kernel}\PY{p}{(}\PY{n}{X}\PY{p}{,} \PY{n}{Y}\PY{p}{)}\PY{p}{:}
%    \PY{k}{return} \PY{n}{gauss\PYZus{}kernel}\PY{p}{(}\PY{n}{X}\PY{p}{,} \PY{n}{Y}\PY{p}{,} \PY{n}{l}\PY{o}{=}\PY{l+m+mf}{1.}\PY{p}{,} \PY{n}{sigma\PYZus{}k}\PY{o}{=}\PY{l+m+mf}{1.}\PY{p}{)}
%\end{Verbatim}
%\end{tcolorbox}

%    \begin{tcolorbox}[breakable, size=fbox, boxrule=1pt, pad at break*=1mm,colback=cellbackground, colframe=cellborder]
%\prompt{In}{incolor}{12}{\boxspacing}
%\begin{Verbatim}[commandchars=\\\{\}]
%\PY{c+c1}{\PYZsh{} Sample from the Gaussian process distribution}
%\PY{n}{n\PYZus{}samples}   \PY{o}{=} \PY{l+m+mi}{51}  \PY{c+c1}{\PYZsh{} Number of points in each function}
%\PY{n}{n\PYZus{}functions} \PY{o}{=}  \PY{l+m+mi}{5}  \PY{c+c1}{\PYZsh{} Number of functions to sample}
%\PY{c+c1}{\PYZsh{} Independent variable samples}
%\PY{n}{X} \PY{o}{=} \PY{n}{np}\PY{o}{.}\PY{n}{reshape}\PY{p}{(}\PY{n}{np}\PY{o}{.}\PY{n}{linspace}\PY{p}{(}\PY{l+m+mi}{0}\PY{p}{,} \PY{l+m+mi}{10}\PY{p}{,} \PY{n}{n\PYZus{}samples}\PY{p}{)}\PY{p}{,} \PY{p}{(}\PY{o}{\PYZhy{}}\PY{l+m+mi}{1}\PY{p}{,} \PY{l+m+mi}{1}\PY{p}{)}\PY{p}{)}
%\PY{n}{Sigma} \PY{o}{=} \PY{n}{kernel}\PY{p}{(}\PY{n}{X}\PY{p}{,} \PY{n}{X}\PY{p}{)}  \PY{c+c1}{\PYZsh{} Kernel of data points}
%
%\PY{c+c1}{\PYZsh{} Draw samples from the prior at our data points.}
%\PY{c+c1}{\PYZsh{} Assume a mean of 0 for simplicity}
%\PY{n}{ys} \PY{o}{=} \PY{n}{np}\PY{o}{.}\PY{n}{random}\PY{o}{.}\PY{n}{multivariate\PYZus{}normal}\PY{p}{(}
%    \PY{n}{mean}\PY{o}{=}\PY{n}{np}\PY{o}{.}\PY{n}{zeros}\PY{p}{(}\PY{n}{n\PYZus{}samples}\PY{p}{)}\PY{p}{,} \PY{n}{cov}\PY{o}{=}\PY{n}{Sigma}\PY{p}{,} \PY{n}{size}\PY{o}{=}\PY{n}{n\PYZus{}functions}\PY{p}{)}
%\end{Verbatim}
%\end{tcolorbox}

%    \begin{tcolorbox}[breakable, size=fbox, boxrule=1pt, pad at break*=1mm,colback=cellbackground, colframe=cellborder]
%\prompt{In}{incolor}{13}{\boxspacing}
%\begin{Verbatim}[commandchars=\\\{\}]
%\PY{c+c1}{\PYZsh{} Plot the sampled functions}
%\PY{n}{seaborn}\PY{o}{.}\PY{n}{set\PYZus{}style}\PY{p}{(}\PY{l+s+s1}{\PYZsq{}}\PY{l+s+s1}{whitegrid}\PY{l+s+s1}{\PYZsq{}}\PY{p}{)}
%\PY{n}{plt}\PY{o}{.}\PY{n}{figure}\PY{p}{(}\PY{n}{figsize}\PY{o}{=}\PY{p}{(}\PY{l+m+mi}{8}\PY{p}{,} \PY{l+m+mi}{5}\PY{p}{)}\PY{p}{)}
%\PY{k}{for} \PY{n}{i} \PY{o+ow}{in} \PY{n+nb}{range}\PY{p}{(}\PY{n}{n\PYZus{}functions}\PY{p}{)}\PY{p}{:}
%    \PY{n}{plt}\PY{o}{.}\PY{n}{plot}\PY{p}{(}\PY{n}{X}\PY{p}{,} \PY{n}{ys}\PY{p}{[}\PY{n}{i}\PY{p}{]}\PY{p}{,} \PY{n}{linestyle}\PY{o}{=}\PY{l+s+s1}{\PYZsq{}}\PY{l+s+s1}{\PYZhy{}}\PY{l+s+s1}{\PYZsq{}}\PY{p}{,} \PY{n}{marker}\PY{o}{=}\PY{l+s+s1}{\PYZsq{}}\PY{l+s+s1}{o}\PY{l+s+s1}{\PYZsq{}}\PY{p}{,} \PY{n}{markersize}\PY{o}{=}\PY{l+m+mi}{3}\PY{p}{)}
%\PY{n}{plt}\PY{o}{.}\PY{n}{xlabel}\PY{p}{(}\PY{l+s+s1}{\PYZsq{}}\PY{l+s+s1}{\PYZdl{}x\PYZdl{}}\PY{l+s+s1}{\PYZsq{}}\PY{p}{)}
%\PY{n}{plt}\PY{o}{.}\PY{n}{ylabel}\PY{p}{(}\PY{l+s+s1}{\PYZsq{}}\PY{l+s+s1}{\PYZdl{}y = f(x)\PYZdl{}}\PY{l+s+s1}{\PYZsq{}}\PY{p}{)}
%\PY{n}{plt}\PY{o}{.}\PY{n}{title}\PY{p}{(}\PY{l+s+sa}{f}\PY{l+s+s1}{\PYZsq{}}\PY{l+s+si}{\PYZob{}}\PY{n}{n\PYZus{}functions}\PY{l+s+si}{\PYZcb{}}\PY{l+s+s1}{ реализаций гауссовского процесса,}\PY{l+s+se}{\PYZbs{}n}\PY{l+s+se}{\PYZbs{}}
%\PY{l+s+s1}{    построенных по }\PY{l+s+si}{\PYZob{}}\PY{n}{n\PYZus{}samples}\PY{l+s+si}{\PYZcb{}}\PY{l+s+s1}{ точке}\PY{l+s+s1}{\PYZsq{}}\PY{p}{)}
%\PY{n}{plt}\PY{o}{.}\PY{n}{xlim}\PY{p}{(}\PY{p}{[} \PY{l+m+mi}{0}\PY{p}{,} \PY{l+m+mi}{10}\PY{p}{]}\PY{p}{)}
%\PY{n}{plt}\PY{o}{.}\PY{n}{ylim}\PY{p}{(}\PY{p}{[}\PY{o}{\PYZhy{}}\PY{l+m+mi}{3}\PY{p}{,}  \PY{l+m+mi}{3}\PY{p}{]}\PY{p}{)}
%\PY{n}{plt}\PY{o}{.}\PY{n}{tight\PYZus{}layout}\PY{p}{(}\PY{p}{)}
%\PY{n}{plt}\PY{o}{.}\PY{n}{show}\PY{p}{(}\PY{p}{)}
%\end{Verbatim}
%\end{tcolorbox}

    \begin{center}
    \adjustimage{max size={0.65\linewidth}{0.65\paperheight}}{Samples_5x51.pdf}
    \end{center}
%    { \hspace*{\fill} \\}

    Другой способ визуализации --- сделать двумерное сечения процесса в
точке \(X = (x_1, x_2)\). Мы получим двумерную нормальную плотность с
математическим ожиданием \(\mu(X)\) и ковариационной матрицей
\(k(X,X)\).

Следующий рисунок слева визуализирует два двумерных сечения:
\(X = [0, 0.2]\) с ковариацией \(k(0, 0.2) = 0.98\) и \(X = [0, 2]\) с
ковариацией \(k(0, 2) = 0.14\). Частные реализации процесса, приведённые
выше, представлены на рисунке цветными точками.

На рисунках видно, что близкорасположенные точки сильно скоррелированны,
в то время как точки, находящиеся далеко друг от друга, практически
независимы.

%    \begin{tcolorbox}[breakable, size=fbox, boxrule=1pt, pad at break*=1mm,colback=cellbackground, colframe=cellborder]
%\prompt{In}{incolor}{14}{\boxspacing}
%\begin{Verbatim}[commandchars=\\\{\}]
%\PY{n}{mu} \PY{o}{=} \PY{n}{np}\PY{o}{.}\PY{n}{array}\PY{p}{(}\PY{p}{[}\PY{l+m+mf}{0.}\PY{p}{,} \PY{l+m+mf}{0.}\PY{p}{]}\PY{p}{)}
%
%\PY{c+c1}{\PYZsh{} Strong correlation}
%\PY{n}{x1}\PY{p}{,} \PY{n}{x2} \PY{o}{=} \PY{l+m+mf}{0.}\PY{p}{,} \PY{l+m+mf}{0.2}
%\PY{n}{X\PYZus{}strong} \PY{o}{=} \PY{n}{np}\PY{o}{.}\PY{n}{reshape}\PY{p}{(}\PY{p}{[}\PY{n}{x1}\PY{p}{,} \PY{n}{x2}\PY{p}{]}\PY{p}{,} \PY{p}{(}\PY{o}{\PYZhy{}}\PY{l+m+mi}{1}\PY{p}{,} \PY{l+m+mi}{1}\PY{p}{)}\PY{p}{)}
%\PY{n}{Sigma\PYZus{}strong} \PY{o}{=} \PY{n}{kernel}\PY{p}{(}\PY{n}{X\PYZus{}strong}\PY{p}{,} \PY{n}{X\PYZus{}strong}\PY{p}{)}
%\PY{c+c1}{\PYZsh{} Select samples}
%\PY{n}{X\PYZus{}00\PYZus{}index} \PY{o}{=} \PY{n}{np}\PY{o}{.}\PY{n}{where}\PY{p}{(}\PY{n}{np}\PY{o}{.}\PY{n}{isclose}\PY{p}{(}\PY{n}{X}\PY{p}{,} \PY{n}{x1}\PY{p}{)}\PY{p}{)}
%\PY{n}{X\PYZus{}02\PYZus{}index} \PY{o}{=} \PY{n}{np}\PY{o}{.}\PY{n}{where}\PY{p}{(}\PY{n}{np}\PY{o}{.}\PY{n}{isclose}\PY{p}{(}\PY{n}{X}\PY{p}{,} \PY{n}{x2}\PY{p}{)}\PY{p}{)}
%\PY{n}{y\PYZus{}strong} \PY{o}{=} \PY{n}{ys}\PY{p}{[}\PY{p}{:}\PY{p}{,}\PY{p}{[}\PY{n}{X\PYZus{}00\PYZus{}index}\PY{p}{[}\PY{l+m+mi}{0}\PY{p}{]}\PY{p}{[}\PY{l+m+mi}{0}\PY{p}{]}\PY{p}{,} \PY{n}{X\PYZus{}02\PYZus{}index}\PY{p}{[}\PY{l+m+mi}{0}\PY{p}{]}\PY{p}{[}\PY{l+m+mi}{0}\PY{p}{]}\PY{p}{]}\PY{p}{]}
%
%\PY{c+c1}{\PYZsh{} Strong correlation}
%\PY{n}{x1}\PY{p}{,} \PY{n}{x2} \PY{o}{=} \PY{l+m+mf}{0.}\PY{p}{,} \PY{l+m+mf}{2.}
%\PY{n}{X\PYZus{}weak} \PY{o}{=} \PY{n}{np}\PY{o}{.}\PY{n}{reshape}\PY{p}{(}\PY{p}{[}\PY{n}{x1}\PY{p}{,} \PY{n}{x2}\PY{p}{]}\PY{p}{,} \PY{p}{(}\PY{o}{\PYZhy{}}\PY{l+m+mi}{1}\PY{p}{,} \PY{l+m+mi}{1}\PY{p}{)}\PY{p}{)}
%\PY{n}{Sigma\PYZus{}weak} \PY{o}{=} \PY{n}{kernel}\PY{p}{(}\PY{n}{X\PYZus{}weak}\PY{p}{,} \PY{n}{X\PYZus{}weak}\PY{p}{)}
%\PY{c+c1}{\PYZsh{} Select samples}
%\PY{n}{X\PYZus{}0\PYZus{}index} \PY{o}{=} \PY{n}{np}\PY{o}{.}\PY{n}{where}\PY{p}{(}\PY{n}{np}\PY{o}{.}\PY{n}{isclose}\PY{p}{(}\PY{n}{X}\PY{p}{,} \PY{n}{x1}\PY{p}{)}\PY{p}{)}
%\PY{n}{X\PYZus{}2\PYZus{}index} \PY{o}{=} \PY{n}{np}\PY{o}{.}\PY{n}{where}\PY{p}{(}\PY{n}{np}\PY{o}{.}\PY{n}{isclose}\PY{p}{(}\PY{n}{X}\PY{p}{,} \PY{n}{x2}\PY{p}{)}\PY{p}{)}
%\PY{n}{y\PYZus{}weak} \PY{o}{=} \PY{n}{ys}\PY{p}{[}\PY{p}{:}\PY{p}{,}\PY{p}{[}\PY{n}{X\PYZus{}0\PYZus{}index}\PY{p}{[}\PY{l+m+mi}{0}\PY{p}{]}\PY{p}{[}\PY{l+m+mi}{0}\PY{p}{]}\PY{p}{,} \PY{n}{X\PYZus{}2\PYZus{}index}\PY{p}{[}\PY{l+m+mi}{0}\PY{p}{]}\PY{p}{[}\PY{l+m+mi}{0}\PY{p}{]}\PY{p}{]}\PY{p}{]}
%\end{Verbatim}
%\end{tcolorbox}

%    \begin{tcolorbox}[breakable, size=fbox, boxrule=1pt, pad at break*=1mm,colback=cellbackground, colframe=cellborder]
%\prompt{In}{incolor}{15}{\boxspacing}
%\begin{Verbatim}[commandchars=\\\{\}]
%\PY{n+nb}{print}\PY{p}{(}\PY{n}{Sigma\PYZus{}weak}\PY{p}{)}
%\end{Verbatim}
%\end{tcolorbox}

%    \begin{Verbatim}[commandchars=\\\{\}]
%[[1.         0.13533528]
% [0.13533528 1.        ]]
%    \end{Verbatim}

%    \begin{tcolorbox}[breakable, size=fbox, boxrule=1pt, pad at break*=1mm,colback=cellbackground, colframe=cellborder]
%\prompt{In}{incolor}{16}{\boxspacing}
%\begin{Verbatim}[commandchars=\\\{\}]
%\PY{c+c1}{\PYZsh{} Show marginal 2D Gaussians}
%\PY{n}{seaborn}\PY{o}{.}\PY{n}{set\PYZus{}style}\PY{p}{(}\PY{l+s+s1}{\PYZsq{}}\PY{l+s+s1}{white}\PY{l+s+s1}{\PYZsq{}}\PY{p}{)}
%\PY{n}{fig}\PY{p}{,} \PY{p}{(}\PY{n}{ax1}\PY{p}{,} \PY{n}{ax2}\PY{p}{)} \PY{o}{=} \PY{n}{plt}\PY{o}{.}\PY{n}{subplots}\PY{p}{(}\PY{n}{nrows}\PY{o}{=}\PY{l+m+mi}{1}\PY{p}{,} \PY{n}{ncols}\PY{o}{=}\PY{l+m+mi}{2}\PY{p}{,} \PY{n}{figsize}\PY{o}{=}\PY{p}{(}\PY{l+m+mi}{12}\PY{p}{,}\PY{l+m+mi}{6}\PY{p}{)}\PY{p}{)}
%
%\PY{c+c1}{\PYZsh{} Plot of strong correlation}
%\PY{n}{y1}\PY{p}{,} \PY{n}{y2}\PY{p}{,} \PY{n}{p} \PY{o}{=} \PY{n}{generate\PYZus{}gauss\PYZus{}surface}\PY{p}{(}\PY{n}{mu}\PY{p}{,} \PY{n}{Sigma\PYZus{}strong}\PY{p}{)}
%\PY{c+c1}{\PYZsh{} Plot bivariate distribution}
%\PY{n}{con1} \PY{o}{=} \PY{n}{ax1}\PY{o}{.}\PY{n}{contourf}\PY{p}{(}\PY{n}{y1}\PY{p}{,} \PY{n}{y2}\PY{p}{,} \PY{n}{p}\PY{p}{,} \PY{l+m+mi}{100}\PY{p}{,} \PY{n}{cmap}\PY{o}{=}\PY{n}{cm}\PY{o}{.}\PY{n}{magma\PYZus{}r}\PY{p}{)}
%\PY{n}{ax1}\PY{o}{.}\PY{n}{set\PYZus{}xlabel}\PY{p}{(}\PY{l+s+sa}{f}\PY{l+s+s1}{\PYZsq{}}\PY{l+s+s1}{\PYZdl{}y\PYZus{}1 = f(x=}\PY{l+s+si}{\PYZob{}}\PY{n}{X\PYZus{}strong}\PY{p}{[}\PY{l+m+mi}{0}\PY{p}{,}\PY{l+m+mi}{0}\PY{p}{]}\PY{l+s+si}{\PYZcb{}}\PY{l+s+s1}{)\PYZdl{}}\PY{l+s+s1}{\PYZsq{}}\PY{p}{)}
%\PY{n}{ax1}\PY{o}{.}\PY{n}{set\PYZus{}ylabel}\PY{p}{(}\PY{l+s+sa}{f}\PY{l+s+s1}{\PYZsq{}}\PY{l+s+s1}{\PYZdl{}y\PYZus{}2 = f(x=}\PY{l+s+si}{\PYZob{}}\PY{n}{X\PYZus{}strong}\PY{p}{[}\PY{l+m+mi}{1}\PY{p}{,}\PY{l+m+mi}{0}\PY{p}{]}\PY{l+s+si}{\PYZcb{}}\PY{l+s+s1}{)\PYZdl{}}\PY{l+s+s1}{\PYZsq{}}\PY{p}{)}
%\PY{n}{ax1}\PY{o}{.}\PY{n}{axis}\PY{p}{(}\PY{p}{[}\PY{o}{\PYZhy{}}\PY{l+m+mf}{2.7}\PY{p}{,} \PY{l+m+mf}{2.7}\PY{p}{,} \PY{o}{\PYZhy{}}\PY{l+m+mf}{2.7}\PY{p}{,} \PY{l+m+mf}{2.7}\PY{p}{]}\PY{p}{)}
%\PY{n}{ax1}\PY{o}{.}\PY{n}{set\PYZus{}aspect}\PY{p}{(}\PY{l+s+s1}{\PYZsq{}}\PY{l+s+s1}{equal}\PY{l+s+s1}{\PYZsq{}}\PY{p}{)}
%\PY{n}{ax1}\PY{o}{.}\PY{n}{text}\PY{p}{(}\PY{o}{\PYZhy{}}\PY{l+m+mf}{2.4}\PY{p}{,} \PY{l+m+mf}{2.3}\PY{p}{,}
%    \PY{p}{(}\PY{l+s+sa}{f}\PY{l+s+s1}{\PYZsq{}}\PY{l+s+s1}{\PYZdl{}k(}\PY{l+s+si}{\PYZob{}}\PY{n}{X\PYZus{}strong}\PY{p}{[}\PY{l+m+mi}{0}\PY{p}{,}\PY{l+m+mi}{0}\PY{p}{]}\PY{l+s+si}{\PYZcb{}}\PY{l+s+s1}{, }\PY{l+s+si}{\PYZob{}}\PY{n}{X\PYZus{}strong}\PY{p}{[}\PY{l+m+mi}{1}\PY{p}{,}\PY{l+m+mi}{0}\PY{p}{]}\PY{l+s+si}{\PYZcb{}}\PY{l+s+s1}{) = }\PY{l+s+si}{\PYZob{}}\PY{n}{Sigma\PYZus{}strong}\PY{p}{[}\PY{l+m+mi}{0}\PY{p}{,}\PY{l+m+mi}{1}\PY{p}{]}\PY{l+s+si}{:}\PY{l+s+s1}{.2f}\PY{l+s+si}{\PYZcb{}}\PY{l+s+s1}{\PYZdl{}}\PY{l+s+s1}{\PYZsq{}}\PY{p}{)}\PY{p}{,} 
%    \PY{n}{fontsize}\PY{o}{=}\PY{l+m+mi}{12}\PY{p}{)}
%\PY{n}{ax1}\PY{o}{.}\PY{n}{set\PYZus{}title}\PY{p}{(}\PY{l+s+sa}{f}\PY{l+s+s1}{\PYZsq{}}\PY{l+s+s1}{\PYZdl{}X = [}\PY{l+s+si}{\PYZob{}}\PY{n}{X\PYZus{}strong}\PY{p}{[}\PY{l+m+mi}{0}\PY{p}{,}\PY{l+m+mi}{0}\PY{p}{]}\PY{l+s+si}{\PYZcb{}}\PY{l+s+s1}{, }\PY{l+s+si}{\PYZob{}}\PY{n}{X\PYZus{}strong}\PY{p}{[}\PY{l+m+mi}{1}\PY{p}{,}\PY{l+m+mi}{0}\PY{p}{]}\PY{l+s+si}{\PYZcb{}}\PY{l+s+s1}{]\PYZdl{} }\PY{l+s+s1}{\PYZsq{}}\PY{p}{,} \PY{n}{fontsize}\PY{o}{=}\PY{l+m+mi}{12}\PY{p}{)}
%\PY{c+c1}{\PYZsh{} Show samples on surface}
%\PY{k}{for} \PY{n}{i} \PY{o+ow}{in} \PY{n+nb}{range}\PY{p}{(}\PY{n}{y\PYZus{}strong}\PY{o}{.}\PY{n}{shape}\PY{p}{[}\PY{l+m+mi}{0}\PY{p}{]}\PY{p}{)}\PY{p}{:}
%    \PY{n}{ax1}\PY{o}{.}\PY{n}{plot}\PY{p}{(}\PY{n}{y\PYZus{}strong}\PY{p}{[}\PY{n}{i}\PY{p}{,}\PY{l+m+mi}{0}\PY{p}{]}\PY{p}{,} \PY{n}{y\PYZus{}strong}\PY{p}{[}\PY{n}{i}\PY{p}{,}\PY{l+m+mi}{1}\PY{p}{]}\PY{p}{,} \PY{l+s+s1}{\PYZsq{}}\PY{l+s+s1}{o}\PY{l+s+s1}{\PYZsq{}}\PY{p}{,} \PY{n}{ms}\PY{o}{=}\PY{l+m+mi}{8}\PY{p}{)}
%
%\PY{c+c1}{\PYZsh{} Plot weak correlation}
%\PY{n}{y1}\PY{p}{,} \PY{n}{y2}\PY{p}{,} \PY{n}{p} \PY{o}{=} \PY{n}{generate\PYZus{}gauss\PYZus{}surface}\PY{p}{(}\PY{n}{mu}\PY{p}{,} \PY{n}{Sigma\PYZus{}weak}\PY{p}{)}
%\PY{c+c1}{\PYZsh{} Plot bivariate distribution}
%\PY{n}{con2} \PY{o}{=} \PY{n}{ax2}\PY{o}{.}\PY{n}{contourf}\PY{p}{(}\PY{n}{y1}\PY{p}{,} \PY{n}{y2}\PY{p}{,} \PY{n}{p}\PY{p}{,} \PY{l+m+mi}{100}\PY{p}{,} \PY{n}{cmap}\PY{o}{=}\PY{n}{cm}\PY{o}{.}\PY{n}{magma\PYZus{}r}\PY{p}{)}
%\PY{n}{con2}\PY{o}{.}\PY{n}{set\PYZus{}cmap}\PY{p}{(}\PY{n}{con1}\PY{o}{.}\PY{n}{get\PYZus{}cmap}\PY{p}{(}\PY{p}{)}\PY{p}{)}
%\PY{n}{con2}\PY{o}{.}\PY{n}{set\PYZus{}clim}\PY{p}{(}\PY{n}{con1}\PY{o}{.}\PY{n}{get\PYZus{}clim}\PY{p}{(}\PY{p}{)}\PY{p}{)}
%\PY{n}{ax2}\PY{o}{.}\PY{n}{set\PYZus{}xlabel}\PY{p}{(}\PY{l+s+sa}{f}\PY{l+s+s1}{\PYZsq{}}\PY{l+s+s1}{\PYZdl{}y\PYZus{}1 = f(x=}\PY{l+s+si}{\PYZob{}}\PY{n}{X\PYZus{}weak}\PY{p}{[}\PY{l+m+mi}{0}\PY{p}{,}\PY{l+m+mi}{0}\PY{p}{]}\PY{l+s+si}{\PYZcb{}}\PY{l+s+s1}{)\PYZdl{}}\PY{l+s+s1}{\PYZsq{}}\PY{p}{)}
%\PY{n}{ax2}\PY{o}{.}\PY{n}{set\PYZus{}ylabel}\PY{p}{(}\PY{l+s+sa}{f}\PY{l+s+s1}{\PYZsq{}}\PY{l+s+s1}{\PYZdl{}y\PYZus{}2 = f(x=}\PY{l+s+si}{\PYZob{}}\PY{n}{X\PYZus{}weak}\PY{p}{[}\PY{l+m+mi}{1}\PY{p}{,}\PY{l+m+mi}{0}\PY{p}{]}\PY{l+s+si}{\PYZcb{}}\PY{l+s+s1}{)\PYZdl{}}\PY{l+s+s1}{\PYZsq{}}\PY{p}{)}
%\PY{n}{ax2}\PY{o}{.}\PY{n}{axis}\PY{p}{(}\PY{p}{[}\PY{o}{\PYZhy{}}\PY{l+m+mf}{2.7}\PY{p}{,} \PY{l+m+mf}{2.7}\PY{p}{,} \PY{o}{\PYZhy{}}\PY{l+m+mf}{2.7}\PY{p}{,} \PY{l+m+mf}{2.7}\PY{p}{]}\PY{p}{)}
%\PY{n}{ax2}\PY{o}{.}\PY{n}{set\PYZus{}aspect}\PY{p}{(}\PY{l+s+s1}{\PYZsq{}}\PY{l+s+s1}{equal}\PY{l+s+s1}{\PYZsq{}}\PY{p}{)}
%\PY{n}{ax2}\PY{o}{.}\PY{n}{text}\PY{p}{(}\PY{o}{\PYZhy{}}\PY{l+m+mf}{2.4}\PY{p}{,} \PY{l+m+mf}{2.3}\PY{p}{,} 
%    \PY{p}{(}\PY{l+s+sa}{f}\PY{l+s+s1}{\PYZsq{}}\PY{l+s+s1}{\PYZdl{}k(}\PY{l+s+si}{\PYZob{}}\PY{n}{X\PYZus{}weak}\PY{p}{[}\PY{l+m+mi}{0}\PY{p}{,}\PY{l+m+mi}{0}\PY{p}{]}\PY{l+s+si}{\PYZcb{}}\PY{l+s+s1}{, }\PY{l+s+si}{\PYZob{}}\PY{n}{X\PYZus{}weak}\PY{p}{[}\PY{l+m+mi}{1}\PY{p}{,}\PY{l+m+mi}{0}\PY{p}{]}\PY{l+s+si}{\PYZcb{}}\PY{l+s+s1}{) = }\PY{l+s+si}{\PYZob{}}\PY{n}{Sigma\PYZus{}weak}\PY{p}{[}\PY{l+m+mi}{0}\PY{p}{,}\PY{l+m+mi}{1}\PY{p}{]}\PY{l+s+si}{:}\PY{l+s+s1}{.2f}\PY{l+s+si}{\PYZcb{}}\PY{l+s+s1}{\PYZdl{}}\PY{l+s+s1}{\PYZsq{}}\PY{p}{)}\PY{p}{,}
%    \PY{n}{fontsize}\PY{o}{=}\PY{l+m+mi}{12}\PY{p}{)}
%\PY{n}{ax2}\PY{o}{.}\PY{n}{set\PYZus{}title}\PY{p}{(}\PY{l+s+sa}{f}\PY{l+s+s1}{\PYZsq{}}\PY{l+s+s1}{\PYZdl{}X = [}\PY{l+s+si}{\PYZob{}}\PY{n}{X\PYZus{}weak}\PY{p}{[}\PY{l+m+mi}{0}\PY{p}{,}\PY{l+m+mi}{0}\PY{p}{]}\PY{l+s+si}{\PYZcb{}}\PY{l+s+s1}{, }\PY{l+s+si}{\PYZob{}}\PY{n}{X\PYZus{}weak}\PY{p}{[}\PY{l+m+mi}{1}\PY{p}{,}\PY{l+m+mi}{0}\PY{p}{]}\PY{l+s+si}{\PYZcb{}}\PY{l+s+s1}{]\PYZdl{}}\PY{l+s+s1}{\PYZsq{}}\PY{p}{,} \PY{n}{fontsize}\PY{o}{=}\PY{l+m+mi}{12}\PY{p}{)}
%\PY{c+c1}{\PYZsh{} Show samples on surface}
%\PY{k}{for} \PY{n}{i} \PY{o+ow}{in} \PY{n+nb}{range}\PY{p}{(}\PY{n}{y\PYZus{}weak}\PY{o}{.}\PY{n}{shape}\PY{p}{[}\PY{l+m+mi}{0}\PY{p}{]}\PY{p}{)}\PY{p}{:}
%    \PY{n}{ax2}\PY{o}{.}\PY{n}{plot}\PY{p}{(}\PY{n}{y\PYZus{}weak}\PY{p}{[}\PY{n}{i}\PY{p}{,}\PY{l+m+mi}{0}\PY{p}{]}\PY{p}{,} \PY{n}{y\PYZus{}weak}\PY{p}{[}\PY{n}{i}\PY{p}{,}\PY{l+m+mi}{1}\PY{p}{]}\PY{p}{,} \PY{l+s+s1}{\PYZsq{}}\PY{l+s+s1}{o}\PY{l+s+s1}{\PYZsq{}}\PY{p}{,} \PY{n}{ms}\PY{o}{=}\PY{l+m+mi}{8}\PY{p}{)}
%
%\PY{c+c1}{\PYZsh{} Add colorbar and title}
%\PY{n}{fig}\PY{o}{.}\PY{n}{subplots\PYZus{}adjust}\PY{p}{(}\PY{n}{right}\PY{o}{=}\PY{l+m+mf}{0.8}\PY{p}{)}
%\PY{n}{cbar\PYZus{}ax} \PY{o}{=} \PY{n}{fig}\PY{o}{.}\PY{n}{add\PYZus{}axes}\PY{p}{(}\PY{p}{[}\PY{l+m+mf}{0.85}\PY{p}{,} \PY{l+m+mf}{0.15}\PY{p}{,} \PY{l+m+mf}{0.02}\PY{p}{,} \PY{l+m+mf}{0.7}\PY{p}{]}\PY{p}{)}
%\PY{n}{cbar} \PY{o}{=} \PY{n}{fig}\PY{o}{.}\PY{n}{colorbar}\PY{p}{(}\PY{n}{con1}\PY{p}{,} \PY{n}{cax}\PY{o}{=}\PY{n}{cbar\PYZus{}ax}\PY{p}{)}
%\PY{n}{cbar}\PY{o}{.}\PY{n}{ax}\PY{o}{.}\PY{n}{set\PYZus{}ylabel}\PY{p}{(}\PY{l+s+s1}{\PYZsq{}}\PY{l+s+s1}{density: \PYZdl{}p(y\PYZus{}1, y\PYZus{}2)\PYZdl{}}\PY{l+s+s1}{\PYZsq{}}\PY{p}{,} \PY{n}{fontsize}\PY{o}{=}\PY{l+m+mi}{11}\PY{p}{)}
%\PY{n}{fig}\PY{o}{.}\PY{n}{suptitle}\PY{p}{(}\PY{l+s+s1}{\PYZsq{}}\PY{l+s+s1}{Частные 2D распределения: \PYZdl{}y }\PY{l+s+s1}{\PYZbs{}}\PY{l+s+s1}{sim }\PY{l+s+s1}{\PYZbs{}}\PY{l+s+s1}{mathcal}\PY{l+s+si}{\PYZob{}N\PYZcb{}}\PY{l+s+s1}{(0, k(X, X))\PYZdl{}}\PY{l+s+s1}{\PYZsq{}}\PY{p}{,} \PY{n}{y}\PY{o}{=}\PY{l+m+mf}{0.92}\PY{p}{)}
%\PY{n}{plt}\PY{o}{.}\PY{n}{show}\PY{p}{(}\PY{p}{)}
%\end{Verbatim}
%\end{tcolorbox}

    \begin{center}
    \adjustimage{max size={0.9\linewidth}{0.9\paperheight}}{Conditional_2D.pdf}
    \end{center}
%    { \hspace*{\fill} \\}

    \hypertarget{ux43fux440ux438ux43cux435ux440ux44b-ux442ux440ux430ux435ux43aux442ux43eux440ux438ux439}{%
\subsection{Примеры
траекторий}\label{ux43fux440ux438ux43cux435ux440ux44b-ux442ux440ux430ux435ux43aux442ux43eux440ux438ux439}}

Сгенерируем выборку реализаций гауссовского процесса с нулевой средней
функцией \(\mu(x) = 0\) и гауссовской функцией ядра.

На рисунке ниже приведены примеры таких реализаций.

%    \begin{tcolorbox}[breakable, size=fbox, boxrule=1pt, pad at break*=1mm,colback=cellbackground, colframe=cellborder]
%\prompt{In}{incolor}{17}{\boxspacing}
%\begin{Verbatim}[commandchars=\\\{\}]
%\PY{k}{def} \PY{n+nf}{plot\PYZus{}gp}\PY{p}{(}\PY{n}{mu}\PY{p}{,} \PY{n}{cov}\PY{p}{,} \PY{n}{X\PYZus{}test}\PY{p}{,} \PY{n}{X\PYZus{}train}\PY{o}{=}\PY{p}{[}\PY{p}{]}\PY{p}{,} \PY{n}{Y\PYZus{}train}\PY{o}{=}\PY{p}{[}\PY{p}{]}\PY{p}{,}
%            \PY{n}{samples}\PY{o}{=}\PY{p}{[}\PY{p}{]}\PY{p}{,} \PY{n}{draw\PYZus{}ci}\PY{o}{=}\PY{k+kc}{False}\PY{p}{)}\PY{p}{:}
%    \PY{l+s+sd}{\PYZsq{}\PYZsq{}\PYZsq{}Plot gaussian process\PYZsq{}\PYZsq{}\PYZsq{}}
%    \PY{n}{X\PYZus{}test} \PY{o}{=} \PY{n}{X\PYZus{}test}\PY{o}{.}\PY{n}{ravel}\PY{p}{(}\PY{p}{)}
%    \PY{n}{mu} \PY{o}{=} \PY{n}{mu}\PY{o}{.}\PY{n}{ravel}\PY{p}{(}\PY{p}{)}
%    \PY{n}{std} \PY{o}{=} \PY{n}{np}\PY{o}{.}\PY{n}{sqrt}\PY{p}{(}\PY{n}{np}\PY{o}{.}\PY{n}{diag}\PY{p}{(}\PY{n}{cov}\PY{p}{)}\PY{p}{)}
%    
%    \PY{k}{if} \PY{n}{draw\PYZus{}ci}\PY{p}{:}
%        \PY{k}{for} \PY{n}{std\PYZus{}i} \PY{o+ow}{in} \PY{n}{np}\PY{o}{.}\PY{n}{linspace}\PY{p}{(}\PY{l+m+mi}{2}\PY{o}{*}\PY{n}{std}\PY{p}{,}\PY{l+m+mi}{0}\PY{p}{,}\PY{l+m+mi}{11}\PY{p}{)}\PY{p}{:}
%            \PY{n}{plt}\PY{o}{.}\PY{n}{fill\PYZus{}between}\PY{p}{(}\PY{n}{X\PYZus{}test}\PY{p}{,} \PY{n}{mu}\PY{o}{\PYZhy{}}\PY{n}{std\PYZus{}i}\PY{p}{,} \PY{n}{mu}\PY{o}{+}\PY{n}{std\PYZus{}i}\PY{p}{,}
%                             \PY{n}{color}\PY{o}{=}\PY{l+s+s1}{\PYZsq{}}\PY{l+s+s1}{grey}\PY{l+s+s1}{\PYZsq{}}\PY{p}{,} \PY{n}{alpha}\PY{o}{=}\PY{l+m+mf}{0.02}\PY{p}{)}
%    \PY{k}{if} \PY{n+nb}{len}\PY{p}{(}\PY{n}{samples}\PY{p}{)}\PY{p}{:}
%        \PY{n}{plt}\PY{o}{.}\PY{n}{plot}\PY{p}{(}\PY{n}{X\PYZus{}test}\PY{p}{,} \PY{n}{samples}\PY{p}{,} \PY{l+s+s1}{\PYZsq{}}\PY{l+s+s1}{\PYZhy{}}\PY{l+s+s1}{\PYZsq{}}\PY{p}{,} \PY{n}{lw}\PY{o}{=}\PY{l+m+mf}{.5}\PY{p}{)}
%    \PY{n}{plt}\PY{o}{.}\PY{n}{plot}\PY{p}{(}\PY{n}{X\PYZus{}test}\PY{p}{,} \PY{n}{mu}\PY{p}{,} \PY{l+s+s1}{\PYZsq{}}\PY{l+s+s1}{k}\PY{l+s+s1}{\PYZsq{}}\PY{p}{)}
%    \PY{k}{if} \PY{n+nb}{len}\PY{p}{(}\PY{n}{X\PYZus{}train}\PY{p}{)}\PY{p}{:}
%        \PY{n}{plt}\PY{o}{.}\PY{n}{plot}\PY{p}{(}\PY{n}{X\PYZus{}train}\PY{p}{,} \PY{n}{Y\PYZus{}train}\PY{p}{,} \PY{l+s+s1}{\PYZsq{}}\PY{l+s+s1}{kx}\PY{l+s+s1}{\PYZsq{}}\PY{p}{,} \PY{n}{mew}\PY{o}{=}\PY{l+m+mf}{1.0}\PY{p}{)}
%    \PY{n}{plt}\PY{o}{.}\PY{n}{xlim}\PY{p}{(}\PY{p}{[}\PY{n}{X\PYZus{}test}\PY{o}{.}\PY{n}{min}\PY{p}{(}\PY{p}{)}\PY{p}{,} \PY{n}{X\PYZus{}test}\PY{o}{.}\PY{n}{max}\PY{p}{(}\PY{p}{)}\PY{p}{]}\PY{p}{)}
%    \PY{n}{plt}\PY{o}{.}\PY{n}{ylim}\PY{p}{(}\PY{p}{[}\PY{p}{(}\PY{n}{mu}\PY{o}{\PYZhy{}}\PY{l+m+mi}{3}\PY{o}{*}\PY{n}{std}\PY{p}{)}\PY{o}{.}\PY{n}{min}\PY{p}{(}\PY{p}{)}\PY{p}{,} \PY{p}{(}\PY{n}{mu}\PY{o}{+}\PY{l+m+mi}{3}\PY{o}{*}\PY{n}{std}\PY{p}{)}\PY{o}{.}\PY{n}{max}\PY{p}{(}\PY{p}{)}\PY{p}{]}\PY{p}{)}
%    \PY{n}{plt}\PY{o}{.}\PY{n}{xlabel}\PY{p}{(}\PY{l+s+s1}{\PYZsq{}}\PY{l+s+s1}{\PYZdl{}x\PYZdl{}}\PY{l+s+s1}{\PYZsq{}}\PY{p}{)}
%    \PY{n}{plt}\PY{o}{.}\PY{n}{ylabel}\PY{p}{(}\PY{l+s+s1}{\PYZsq{}}\PY{l+s+s1}{\PYZdl{}f(x)\PYZdl{}}\PY{l+s+s1}{\PYZsq{}}\PY{p}{,} \PY{n}{rotation}\PY{o}{=}\PY{l+m+mi}{0}\PY{p}{)}
%\end{Verbatim}
%\end{tcolorbox}

%    \begin{tcolorbox}[breakable, size=fbox, boxrule=1pt, pad at break*=1mm,colback=cellbackground, colframe=cellborder]
%\prompt{In}{incolor}{18}{\boxspacing}
%\begin{Verbatim}[commandchars=\\\{\}]
%\PY{c+c1}{\PYZsh{} Test data}
%\PY{n}{x\PYZus{}min}\PY{p}{,} \PY{n}{x\PYZus{}max} \PY{o}{=} \PY{l+m+mi}{0}\PY{p}{,} \PY{l+m+mi}{10}
%\PY{n}{n\PYZus{}test} \PY{o}{=} \PY{l+m+mi}{101}
%\PY{n}{X\PYZus{}test} \PY{o}{=} \PY{n}{np}\PY{o}{.}\PY{n}{linspace}\PY{p}{(}\PY{n}{x\PYZus{}min}\PY{p}{,} \PY{n}{x\PYZus{}max}\PY{p}{,} \PY{n}{n\PYZus{}test}\PY{p}{)}\PY{o}{.}\PY{n}{reshape}\PY{p}{(}\PY{o}{\PYZhy{}}\PY{l+m+mi}{1}\PY{p}{,} \PY{l+m+mi}{1}\PY{p}{)}
%
%\PY{c+c1}{\PYZsh{} Set mean and covariance}
%\PY{n}{M} \PY{o}{=} \PY{n}{np}\PY{o}{.}\PY{n}{zeros\PYZus{}like}\PY{p}{(}\PY{n}{X\PYZus{}test}\PY{p}{)}\PY{o}{.}\PY{n}{reshape}\PY{p}{(}\PY{o}{\PYZhy{}}\PY{l+m+mi}{1}\PY{p}{,} \PY{l+m+mi}{1}\PY{p}{)}
%\PY{n}{l} \PY{o}{=} \PY{l+m+mf}{1e\PYZhy{}1}\PY{o}{*}\PY{p}{(}\PY{n}{x\PYZus{}max}\PY{o}{\PYZhy{}}\PY{n}{x\PYZus{}min}\PY{p}{)}
%\PY{n}{K} \PY{o}{=} \PY{n}{gauss\PYZus{}kernel}\PY{p}{(}\PY{n}{X\PYZus{}test}\PY{p}{,} \PY{n}{X\PYZus{}test}\PY{p}{,} \PY{n}{l}\PY{o}{=}\PY{n}{l}\PY{p}{)}
%
%\PY{c+c1}{\PYZsh{} Generate samples from the prior}
%\PY{n}{n\PYZus{}p} \PY{o}{=} \PY{n+nb}{int}\PY{p}{(}\PY{l+m+mf}{1e5}\PY{p}{)}
%\PY{n}{L} \PY{o}{=} \PY{n}{np}\PY{o}{.}\PY{n}{linalg}\PY{o}{.}\PY{n}{cholesky}\PY{p}{(}\PY{n}{K} \PY{o}{+} \PY{l+m+mf}{1e\PYZhy{}6}\PY{o}{*}\PY{n}{np}\PY{o}{.}\PY{n}{eye}\PY{p}{(}\PY{n}{n\PYZus{}test}\PY{p}{)}\PY{p}{)}
%\PY{n}{gp} \PY{o}{=} \PY{n}{M} \PY{o}{+} \PY{n}{np}\PY{o}{.}\PY{n}{dot}\PY{p}{(}\PY{n}{L}\PY{p}{,} \PY{n}{np}\PY{o}{.}\PY{n}{random}\PY{o}{.}\PY{n}{normal}\PY{p}{(}\PY{n}{size}\PY{o}{=}\PY{p}{(}\PY{n}{n\PYZus{}test}\PY{p}{,}\PY{n}{n\PYZus{}p}\PY{p}{)}\PY{p}{)}\PY{p}{)}
%\end{Verbatim}
%\end{tcolorbox}

%    \begin{tcolorbox}[breakable, size=fbox, boxrule=1pt, pad at break*=1mm,colback=cellbackground, colframe=cellborder]
%\prompt{In}{incolor}{19}{\boxspacing}
%\begin{Verbatim}[commandchars=\\\{\}]
%\PY{c+c1}{\PYZsh{} Draw samples from the prior}
%\PY{n}{x\PYZus{}i}\PY{p}{,} \PY{n}{n\PYZus{}draw} \PY{o}{=} \PY{l+m+mf}{2.}\PY{p}{,} \PY{l+m+mi}{100}
%\PY{n}{seaborn}\PY{o}{.}\PY{n}{set\PYZus{}style}\PY{p}{(}\PY{l+s+s1}{\PYZsq{}}\PY{l+s+s1}{whitegrid}\PY{l+s+s1}{\PYZsq{}}\PY{p}{)}
%
%\PY{n}{plt}\PY{o}{.}\PY{n}{figure}\PY{p}{(}\PY{n}{figsize}\PY{o}{=}\PY{p}{(}\PY{l+m+mi}{8}\PY{p}{,} \PY{l+m+mi}{5}\PY{p}{)}\PY{p}{)}
%\PY{n}{plot\PYZus{}gp}\PY{p}{(}\PY{n}{M}\PY{p}{,} \PY{n}{K}\PY{p}{,} \PY{n}{X\PYZus{}test}\PY{p}{,} \PY{n}{samples}\PY{o}{=}\PY{n}{gp}\PY{p}{[}\PY{p}{:}\PY{p}{,}\PY{p}{:}\PY{n}{n\PYZus{}draw}\PY{p}{]}\PY{p}{)}
%\PY{n}{plt}\PY{o}{.}\PY{n}{axvline}\PY{p}{(}\PY{n}{x\PYZus{}i}\PY{p}{,} \PY{n}{c}\PY{o}{=}\PY{l+s+s1}{\PYZsq{}}\PY{l+s+s1}{k}\PY{l+s+s1}{\PYZsq{}}\PY{p}{,} \PY{n}{ls}\PY{o}{=}\PY{l+s+s1}{\PYZsq{}}\PY{l+s+s1}{:}\PY{l+s+s1}{\PYZsq{}}\PY{p}{)}
%\PY{n}{plt}\PY{o}{.}\PY{n}{tight\PYZus{}layout}\PY{p}{(}\PY{p}{)}
%\PY{n}{plt}\PY{o}{.}\PY{n}{show}\PY{p}{(}\PY{p}{)}
%\end{Verbatim}
%\end{tcolorbox}

    \begin{center}
    \adjustimage{max size={0.65\linewidth}{0.65\paperheight}}{Traj_examples.pdf}
    \end{center}
%    { \hspace*{\fill} \\}

%    Убедимся в правильных статистических характеристиках нашей выборки. Для
%этого нарисуем гистограмму значений \(f(x)\) в каком-либо сечении
%\(x = \mathrm{const}\). Согласно определению гауссовского процесса
%распределение \(f(x)\) должно быть гауссовым.
%
%    \begin{tcolorbox}[breakable, size=fbox, boxrule=1pt, pad at break*=1mm,colback=cellbackground, colframe=cellborder]
%\prompt{In}{incolor}{20}{\boxspacing}
%\begin{Verbatim}[commandchars=\\\{\}]
%\PY{c+c1}{\PYZsh{} Make section}
%\PY{n}{i\PYZus{}sec} \PY{o}{=} \PY{n}{np}\PY{o}{.}\PY{n}{flatnonzero}\PY{p}{(}\PY{n}{X\PYZus{}test}\PY{o}{.}\PY{n}{ravel}\PY{p}{(}\PY{p}{)} \PY{o}{==} \PY{n}{x\PYZus{}i}\PY{p}{)}\PY{p}{[}\PY{l+m+mi}{0}\PY{p}{]}
%\PY{n}{gp\PYZus{}i} \PY{o}{=} \PY{n}{gp}\PY{p}{[}\PY{n}{i\PYZus{}sec}\PY{p}{]}
%\PY{n}{mean}  \PY{o}{=} \PY{n}{M}\PY{o}{.}\PY{n}{flatten}\PY{p}{(}\PY{p}{)}\PY{p}{[}\PY{n}{i\PYZus{}sec}\PY{p}{]}
%\PY{n}{sigma} \PY{o}{=} \PY{n}{np}\PY{o}{.}\PY{n}{sqrt}\PY{p}{(}\PY{n}{np}\PY{o}{.}\PY{n}{diag}\PY{p}{(}\PY{n}{K}\PY{p}{)}\PY{p}{[}\PY{n}{i\PYZus{}sec}\PY{p}{]}\PY{p}{)}
%
%\PY{n}{xlim} \PY{o}{=} \PY{p}{[}\PY{o}{\PYZhy{}}\PY{l+m+mf}{3.5}\PY{o}{*}\PY{n}{sigma}\PY{p}{,} \PY{l+m+mf}{3.5}\PY{o}{*}\PY{n}{sigma}\PY{p}{]}
%\PY{n}{x} \PY{o}{=} \PY{n}{np}\PY{o}{.}\PY{n}{linspace}\PY{p}{(}\PY{o}{*}\PY{n}{xlim}\PY{p}{,} \PY{l+m+mi}{101}\PY{p}{)}
%\PY{n}{y} \PY{o}{=} \PY{n}{stats}\PY{o}{.}\PY{n}{norm}\PY{o}{.}\PY{n}{pdf}\PY{p}{(}\PY{n}{x}\PY{p}{,} \PY{n}{loc}\PY{o}{=}\PY{n}{mean}\PY{p}{,} \PY{n}{scale}\PY{o}{=}\PY{n}{sigma}\PY{p}{)}
%\end{Verbatim}
%\end{tcolorbox}
%
%    \begin{tcolorbox}[breakable, size=fbox, boxrule=1pt, pad at break*=1mm,colback=cellbackground, colframe=cellborder]
%\prompt{In}{incolor}{21}{\boxspacing}
%\begin{Verbatim}[commandchars=\\\{\}]
%\PY{c+c1}{\PYZsh{} Draw section histogram}
%\PY{n}{plt}\PY{o}{.}\PY{n}{figure}\PY{p}{(}\PY{n}{figsize}\PY{o}{=}\PY{p}{(}\PY{l+m+mi}{7}\PY{p}{,} \PY{l+m+mi}{4}\PY{p}{)}\PY{p}{)}
%\PY{n}{plt}\PY{o}{.}\PY{n}{title}\PY{p}{(}\PY{l+s+sa}{f}\PY{l+s+s1}{\PYZsq{}}\PY{l+s+s1}{Гистограмма значений \PYZdl{}f(x)\PYZdl{} в сечении \PYZdl{}x=}\PY{l+s+si}{\PYZob{}}\PY{n}{x\PYZus{}i}\PY{l+s+si}{\PYZcb{}}\PY{l+s+s1}{\PYZdl{}}\PY{l+s+s1}{\PYZsq{}}\PY{p}{)}
%\PY{n}{plt}\PY{o}{.}\PY{n}{hist}\PY{p}{(}\PY{n}{gp\PYZus{}i}\PY{p}{,} \PY{n}{bins}\PY{o}{=}\PY{l+m+mi}{100}\PY{p}{,} \PY{n}{density}\PY{o}{=}\PY{k+kc}{True}\PY{p}{)}
%\PY{n}{plt}\PY{o}{.}\PY{n}{plot}\PY{p}{(}\PY{n}{x}\PY{p}{,} \PY{n}{y}\PY{p}{,} \PY{l+s+s1}{\PYZsq{}}\PY{l+s+s1}{k\PYZhy{}}\PY{l+s+s1}{\PYZsq{}}\PY{p}{,} \PY{n}{label}\PY{o}{=}\PY{l+s+sa}{f}\PY{l+s+s1}{\PYZsq{}}\PY{l+s+s1}{\PYZdl{}}\PY{l+s+s1}{\PYZbs{}}\PY{l+s+s1}{mathcal}\PY{l+s+se}{\PYZob{}\PYZob{}}\PY{l+s+s1}{N}\PY{l+s+se}{\PYZcb{}\PYZcb{}}\PY{l+s+s1}{(}\PY{l+s+si}{\PYZob{}}\PY{n}{mean}\PY{l+s+si}{:}\PY{l+s+s1}{.1g}\PY{l+s+si}{\PYZcb{}}\PY{l+s+s1}{, }\PY{l+s+si}{\PYZob{}}\PY{n}{sigma}\PY{l+s+si}{:}\PY{l+s+s1}{.3g}\PY{l+s+si}{\PYZcb{}}\PY{l+s+s1}{)\PYZdl{}}\PY{l+s+s1}{\PYZsq{}}\PY{p}{)}
%\PY{n}{plt}\PY{o}{.}\PY{n}{xlim}\PY{p}{(}\PY{n}{xlim}\PY{p}{)}
%\PY{n}{plt}\PY{o}{.}\PY{n}{legend}\PY{p}{(}\PY{p}{)}
%\PY{n}{plt}\PY{o}{.}\PY{n}{tight\PYZus{}layout}\PY{p}{(}\PY{p}{)}
%\PY{n}{plt}\PY{o}{.}\PY{n}{show}\PY{p}{(}\PY{p}{)}
%\end{Verbatim}
%\end{tcolorbox}
%
%    \begin{center}
%    \adjustimage{max size={0.65\linewidth}{0.65\paperheight}}{Hist_x20.pdf}
%    \end{center}
%%    { \hspace*{\fill} \\}

    \begin{center}\rule{0.5\linewidth}{0.5pt}\end{center}

    \hypertarget{ux431ux440ux43eux443ux43dux43eux432ux441ux43aux43eux435-ux434ux432ux438ux436ux435ux43dux438ux435-ux43aux430ux43a-ux433ux430ux443ux441ux441ux43eux432ux441ux43aux438ux439-ux43fux440ux43eux446ux435ux441ux441}{%
\section{Броуновское движение как гауссовский
процесс}\label{ux431ux440ux43eux443ux43dux43eux432ux441ux43aux43eux435-ux434ux432ux438ux436ux435ux43dux438ux435-ux43aux430ux43a-ux433ux430ux443ux441ux441ux43eux432ux441ux43aux438ux439-ux43fux440ux43eux446ux435ux441ux441}}

    Ранее, построив гистограмму сечения процесса броуновского движения, мы
увидели, что это гауссовский процесс. Тогда процесс броуновского
движения можно реализовать, задав подходящую ковариационную функцию.
Несложно убедится, что в качестве такой функции подходит
\(k(x, x') = \min(x, x')\).

%    \begin{tcolorbox}[breakable, size=fbox, boxrule=1pt, pad at break*=1mm,colback=cellbackground, colframe=cellborder]
%\prompt{In}{incolor}{22}{\boxspacing}
%\begin{Verbatim}[commandchars=\\\{\}]
%\PY{k}{def} \PY{n+nf}{brownian\PYZus{}kernel}\PY{p}{(}\PY{n}{X1}\PY{p}{,} \PY{n}{X2}\PY{p}{,} \PY{n}{sigma\PYZus{}k}\PY{o}{=}\PY{l+m+mf}{1.}\PY{p}{)}\PY{p}{:}
%    \PY{l+s+sd}{\PYZsq{}\PYZsq{}\PYZsq{}}
%\PY{l+s+sd}{    Brownian motion kernel}
%\PY{l+s+sd}{    }
%\PY{l+s+sd}{    Args:}
%\PY{l+s+sd}{        X1: Array of m points (m x d)}
%\PY{l+s+sd}{        X2: Array of n points (n x d)}
%
%\PY{l+s+sd}{    Returns:}
%\PY{l+s+sd}{        Covariance matrix (m x n)}
%\PY{l+s+sd}{    \PYZsq{}\PYZsq{}\PYZsq{}}
%    
%    \PY{n}{cov} \PY{o}{=} \PY{n}{np}\PY{o}{.}\PY{n}{min}\PY{p}{(}\PY{n}{np}\PY{o}{.}\PY{n}{dstack}\PY{p}{(}\PY{n}{np}\PY{o}{.}\PY{n}{meshgrid}\PY{p}{(}\PY{n}{X1}\PY{p}{,} \PY{n}{X2}\PY{p}{)}\PY{p}{)}\PY{p}{,} \PY{n}{axis}\PY{o}{=}\PY{o}{\PYZhy{}}\PY{l+m+mi}{1}\PY{p}{)}
%    \PY{k}{return} \PY{n}{sigma\PYZus{}k}\PY{o}{*}\PY{o}{*}\PY{l+m+mi}{2} \PY{o}{*} \PY{n}{cov}\PY{o}{.}\PY{n}{T}
%\end{Verbatim}
%\end{tcolorbox}

%    \begin{tcolorbox}[breakable, size=fbox, boxrule=1pt, pad at break*=1mm,colback=cellbackground, colframe=cellborder]
%\prompt{In}{incolor}{23}{\boxspacing}
%\begin{Verbatim}[commandchars=\\\{\}]
%\PY{c+c1}{\PYZsh{} Test data}
%\PY{n}{x\PYZus{}min}\PY{p}{,} \PY{n}{x\PYZus{}max} \PY{o}{=} \PY{l+m+mi}{0}\PY{p}{,} \PY{l+m+mi}{1}
%\PY{n}{n\PYZus{}test} \PY{o}{=} \PY{l+m+mi}{500}
%\PY{n}{X\PYZus{}test} \PY{o}{=} \PY{n}{np}\PY{o}{.}\PY{n}{linspace}\PY{p}{(}\PY{n}{x\PYZus{}min}\PY{p}{,} \PY{n}{x\PYZus{}max}\PY{p}{,} \PY{n}{n\PYZus{}test}\PY{o}{+}\PY{l+m+mi}{1}\PY{p}{)}\PY{o}{.}\PY{n}{reshape}\PY{p}{(}\PY{o}{\PYZhy{}}\PY{l+m+mi}{1}\PY{p}{,} \PY{l+m+mi}{1}\PY{p}{)}
%
%\PY{c+c1}{\PYZsh{} Set mean and covariance}
%\PY{n}{sigma\PYZus{}k} \PY{o}{=} \PY{l+m+mf}{1.0}
%\PY{n}{M} \PY{o}{=} \PY{n}{np}\PY{o}{.}\PY{n}{zeros\PYZus{}like}\PY{p}{(}\PY{n}{X\PYZus{}test}\PY{p}{[}\PY{l+m+mi}{1}\PY{p}{:}\PY{p}{]}\PY{p}{)}\PY{o}{.}\PY{n}{reshape}\PY{p}{(}\PY{o}{\PYZhy{}}\PY{l+m+mi}{1}\PY{p}{,} \PY{l+m+mi}{1}\PY{p}{)}
%\PY{n}{K} \PY{o}{=} \PY{n}{brownian\PYZus{}kernel}\PY{p}{(}\PY{n}{X\PYZus{}test}\PY{p}{[}\PY{l+m+mi}{1}\PY{p}{:}\PY{p}{]}\PY{p}{,} \PY{n}{X\PYZus{}test}\PY{p}{[}\PY{l+m+mi}{1}\PY{p}{:}\PY{p}{]}\PY{p}{,} \PY{n}{sigma\PYZus{}k}\PY{p}{)}
%
%\PY{c+c1}{\PYZsh{} Generate samples from the prior}
%\PY{n}{n\PYZus{}p} \PY{o}{=} \PY{n+nb}{int}\PY{p}{(}\PY{l+m+mf}{1e5}\PY{p}{)}
%\PY{n}{L} \PY{o}{=} \PY{n}{np}\PY{o}{.}\PY{n}{linalg}\PY{o}{.}\PY{n}{cholesky}\PY{p}{(}\PY{n}{K} \PY{o}{+} \PY{l+m+mf}{1e\PYZhy{}6}\PY{o}{*}\PY{n}{np}\PY{o}{.}\PY{n}{eye}\PY{p}{(}\PY{n}{n\PYZus{}test}\PY{p}{)}\PY{p}{)}
%\PY{n}{bp} \PY{o}{=} \PY{n}{np}\PY{o}{.}\PY{n}{zeros}\PY{p}{(}\PY{p}{(}\PY{p}{(}\PY{n}{n\PYZus{}test}\PY{o}{+}\PY{l+m+mi}{1}\PY{p}{,} \PY{n}{n\PYZus{}p}\PY{p}{)}\PY{p}{)}\PY{p}{)}
%\PY{n}{bp}\PY{p}{[}\PY{l+m+mi}{1}\PY{p}{:}\PY{p}{]} \PY{o}{=} \PY{n}{M} \PY{o}{+} \PY{n}{np}\PY{o}{.}\PY{n}{dot}\PY{p}{(}\PY{n}{L}\PY{p}{,} \PY{n}{np}\PY{o}{.}\PY{n}{random}\PY{o}{.}\PY{n}{normal}\PY{p}{(}\PY{n}{size}\PY{o}{=}\PY{p}{(}\PY{n}{n\PYZus{}test}\PY{p}{,}\PY{n}{n\PYZus{}p}\PY{p}{)}\PY{p}{)}\PY{p}{)}
%\end{Verbatim}
%\end{tcolorbox}

%    \begin{tcolorbox}[breakable, size=fbox, boxrule=1pt, pad at break*=1mm,colback=cellbackground, colframe=cellborder]
%\prompt{In}{incolor}{24}{\boxspacing}
%\begin{Verbatim}[commandchars=\\\{\}]
%\PY{c+c1}{\PYZsh{} Make the plots}
%\PY{n}{n\PYZus{}draw} \PY{o}{=} \PY{l+m+mi}{200}
%\PY{n}{x} \PY{o}{=} \PY{n}{X\PYZus{}test}\PY{o}{.}\PY{n}{flatten}\PY{p}{(}\PY{p}{)}
%\PY{n}{std\PYZus{}x} \PY{o}{=} \PY{n}{sigma\PYZus{}k}\PY{o}{*}\PY{n}{x}\PY{o}{*}\PY{o}{*}\PY{l+m+mf}{0.5}
%\PY{n}{plt}\PY{o}{.}\PY{n}{figure}\PY{p}{(}\PY{n}{figsize}\PY{o}{=}\PY{p}{(}\PY{l+m+mi}{8}\PY{p}{,} \PY{l+m+mi}{5}\PY{p}{)}\PY{p}{)}
%\PY{n}{plt}\PY{o}{.}\PY{n}{title}\PY{p}{(}\PY{l+s+sa}{f}\PY{l+s+s1}{\PYZsq{}}\PY{l+s+s1}{Процесс броуновского движения,}\PY{l+s+se}{\PYZbs{}n}\PY{l+s+se}{\PYZbs{}}
%\PY{l+s+s1}{    траектории }\PY{l+s+si}{\PYZob{}}\PY{n}{n\PYZus{}draw}\PY{l+s+si}{\PYZcb{}}\PY{l+s+s1}{ реализаций процесса}\PY{l+s+s1}{\PYZsq{}}\PY{p}{)}
%
%\PY{k}{for} \PY{n}{i} \PY{o+ow}{in} \PY{n+nb}{range}\PY{p}{(}\PY{n}{n\PYZus{}draw}\PY{p}{)}\PY{p}{:}
%    \PY{n}{plt}\PY{o}{.}\PY{n}{plot}\PY{p}{(}\PY{n}{x}\PY{p}{,} \PY{n}{bp}\PY{p}{[}\PY{p}{:}\PY{p}{,}\PY{n}{i}\PY{p}{]}\PY{p}{,} \PY{n}{lw}\PY{o}{=}\PY{l+m+mf}{.4}\PY{p}{)}
%\PY{n}{plt}\PY{o}{.}\PY{n}{fill\PYZus{}between}\PY{p}{(}\PY{n}{x}\PY{p}{,} \PY{o}{\PYZhy{}}\PY{l+m+mi}{2}\PY{o}{*}\PY{n}{std\PYZus{}x}\PY{p}{,} \PY{l+m+mi}{2}\PY{o}{*}\PY{n}{std\PYZus{}x}\PY{p}{,} \PY{n}{fc}\PY{o}{=}\PY{l+s+s1}{\PYZsq{}}\PY{l+s+s1}{grey}\PY{l+s+s1}{\PYZsq{}}\PY{p}{,} \PY{n}{alpha}\PY{o}{=}\PY{l+m+mf}{0.1}\PY{p}{)}
%\PY{n}{plt}\PY{o}{.}\PY{n}{plot}\PY{p}{(}\PY{n}{x}\PY{p}{,} \PY{o}{\PYZhy{}}\PY{l+m+mi}{2}\PY{o}{*}\PY{n}{std\PYZus{}x}\PY{p}{,} \PY{l+s+s1}{\PYZsq{}}\PY{l+s+s1}{k\PYZhy{}}\PY{l+s+s1}{\PYZsq{}}\PY{p}{,} \PY{n}{lw}\PY{o}{=}\PY{l+m+mf}{1.}\PY{p}{)}
%\PY{n}{plt}\PY{o}{.}\PY{n}{plot}\PY{p}{(}\PY{n}{x}\PY{p}{,}  \PY{l+m+mi}{2}\PY{o}{*}\PY{n}{std\PYZus{}x}\PY{p}{,} \PY{l+s+s1}{\PYZsq{}}\PY{l+s+s1}{k\PYZhy{}}\PY{l+s+s1}{\PYZsq{}}\PY{p}{,} \PY{n}{lw}\PY{o}{=}\PY{l+m+mf}{1.}\PY{p}{)}
%
%\PY{n}{plt}\PY{o}{.}\PY{n}{xlabel}\PY{p}{(}\PY{l+s+s1}{\PYZsq{}}\PY{l+s+s1}{\PYZdl{}t\PYZdl{}}\PY{l+s+s1}{\PYZsq{}}\PY{p}{)}
%\PY{n}{plt}\PY{o}{.}\PY{n}{ylabel}\PY{p}{(}\PY{l+s+s1}{\PYZsq{}}\PY{l+s+s1}{\PYZdl{}d(t)\PYZdl{}}\PY{l+s+s1}{\PYZsq{}}\PY{p}{)}
%\PY{n}{plt}\PY{o}{.}\PY{n}{xlim}\PY{p}{(}\PY{p}{[}\PY{l+m+mi}{0}\PY{p}{,} \PY{n}{x\PYZus{}max}\PY{p}{]}\PY{p}{)}
%\PY{n}{plt}\PY{o}{.}\PY{n}{ylim}\PY{p}{(}\PY{p}{[}\PY{o}{\PYZhy{}}\PY{l+m+mi}{3}\PY{o}{*}\PY{n}{sigma\PYZus{}k}\PY{o}{*}\PY{n}{x\PYZus{}max}\PY{o}{*}\PY{o}{*}\PY{l+m+mf}{0.5}\PY{p}{,} \PY{l+m+mi}{3}\PY{o}{*}\PY{n}{sigma\PYZus{}k}\PY{o}{*}\PY{n}{x\PYZus{}max}\PY{o}{*}\PY{o}{*}\PY{l+m+mf}{0.5}\PY{p}{]}\PY{p}{)}
%\PY{n}{plt}\PY{o}{.}\PY{n}{tight\PYZus{}layout}\PY{p}{(}\PY{p}{)}
%\PY{n}{plt}\PY{o}{.}\PY{n}{show}\PY{p}{(}\PY{p}{)}
%\end{Verbatim}
%\end{tcolorbox}

    \begin{center}
    \adjustimage{max size={0.75\linewidth}{0.75\paperheight}}{BrMo_as_GP.pdf}
    \end{center}
%    { \hspace*{\fill} \\}

    \begin{center}\rule{0.5\linewidth}{0.5pt}\end{center}

    \hypertarget{ux438ux441ux442ux43eux447ux43dux438ux43aux438}{%
\section{Источники}\label{ux438ux441ux442ux43eux447ux43dux438ux43aux438}}

\begin{enumerate}
\def\labelenumi{\arabic{enumi}.}
\tightlist
\item
  Лекции по случайным процессам / под ред. А.В. Гасникова. М.: МФТИ,
  2019.
\item
  \emph{Roelants P.}
  \href{https://peterroelants.github.io/posts/gaussian-process-tutorial/}{Understanding
  Gaussian processes}.
\item
  \emph{Krasser M.}
  \href{http://krasserm.github.io/2018/03/19/gaussian-processes/}{Gaussian
  processes}.
\end{enumerate}

%    \begin{tcolorbox}[breakable, size=fbox, boxrule=1pt, pad at break*=1mm,colback=cellbackground, colframe=cellborder]
%\prompt{In}{incolor}{25}{\boxspacing}
%\begin{Verbatim}[commandchars=\\\{\}]
%\PY{c+c1}{\PYZsh{} Versions used}
%\PY{n+nb}{print}\PY{p}{(}\PY{l+s+s1}{\PYZsq{}}\PY{l+s+s1}{Python: }\PY{l+s+si}{\PYZob{}\PYZcb{}}\PY{l+s+s1}{.}\PY{l+s+si}{\PYZob{}\PYZcb{}}\PY{l+s+s1}{.}\PY{l+s+si}{\PYZob{}\PYZcb{}}\PY{l+s+s1}{\PYZsq{}}\PY{o}{.}\PY{n}{format}\PY{p}{(}\PY{o}{*}\PY{n}{sys}\PY{o}{.}\PY{n}{version\PYZus{}info}\PY{p}{[}\PY{p}{:}\PY{l+m+mi}{3}\PY{p}{]}\PY{p}{)}\PY{p}{)}
%\PY{n+nb}{print}\PY{p}{(}\PY{l+s+s1}{\PYZsq{}}\PY{l+s+s1}{numpy: }\PY{l+s+si}{\PYZob{}\PYZcb{}}\PY{l+s+s1}{\PYZsq{}}\PY{o}{.}\PY{n}{format}\PY{p}{(}\PY{n}{np}\PY{o}{.}\PY{n}{\PYZus{}\PYZus{}version\PYZus{}\PYZus{}}\PY{p}{)}\PY{p}{)}
%\PY{n+nb}{print}\PY{p}{(}\PY{l+s+s1}{\PYZsq{}}\PY{l+s+s1}{matplotlib: }\PY{l+s+si}{\PYZob{}\PYZcb{}}\PY{l+s+s1}{\PYZsq{}}\PY{o}{.}\PY{n}{format}\PY{p}{(}\PY{n}{matplotlib}\PY{o}{.}\PY{n}{\PYZus{}\PYZus{}version\PYZus{}\PYZus{}}\PY{p}{)}\PY{p}{)}
%\PY{n+nb}{print}\PY{p}{(}\PY{l+s+s1}{\PYZsq{}}\PY{l+s+s1}{seaborn: }\PY{l+s+si}{\PYZob{}\PYZcb{}}\PY{l+s+s1}{\PYZsq{}}\PY{o}{.}\PY{n}{format}\PY{p}{(}\PY{n}{seaborn}\PY{o}{.}\PY{n}{\PYZus{}\PYZus{}version\PYZus{}\PYZus{}}\PY{p}{)}\PY{p}{)}
%\end{Verbatim}
%\end{tcolorbox}
%
%    \begin{Verbatim}[commandchars=\\\{\}]
%Python: 3.7.16
%numpy: 1.20.3
%matplotlib: 3.5.1
%seaborn: 0.12.2
%    \end{Verbatim}


    % Add a bibliography block to the postdoc
    
    
    
\end{document}
