\documentclass[11pt,a4paper]{article}

    \usepackage[breakable]{tcolorbox}
    \usepackage{parskip} % Stop auto-indenting (to mimic markdown behaviour)
    
    \usepackage{iftex}
    \ifPDFTeX
      \usepackage[T2A]{fontenc}
      \usepackage{mathpazo}
      \usepackage[russian,english]{babel}
    \else
      \usepackage{fontspec}
      \usepackage{polyglossia}
      \setmainlanguage[babelshorthands=true]{russian}    % Язык по-умолчанию русский с поддержкой приятных команд пакета babel
      \setotherlanguage{english}                         % Дополнительный язык = английский (в американской вариации по-умолчанию)
      \newfontfamily\cyrillicfonttt[Scale=0.87,BoldFont={Fira Mono Medium}] {Fira Mono}  % Моноширинный шрифт для кириллицы
      \defaultfontfeatures{Ligatures=TeX}
      \newfontfamily\cyrillicfont{STIX Two Text}         % Шрифт с засечками для кириллицы
    \fi
    \renewcommand{\linethickness}{0.1ex}

    % Basic figure setup, for now with no caption control since it's done
    % automatically by Pandoc (which extracts ![](path) syntax from Markdown).
    \usepackage{graphicx}
    % Maintain compatibility with old templates. Remove in nbconvert 6.0
    \let\Oldincludegraphics\includegraphics
    % Ensure that by default, figures have no caption (until we provide a
    % proper Figure object with a Caption API and a way to capture that
    % in the conversion process - todo).
    \usepackage{caption}
    \DeclareCaptionFormat{nocaption}{}
    \captionsetup{format=nocaption,aboveskip=0pt,belowskip=0pt}

    \usepackage[Export]{adjustbox} % Used to constrain images to a maximum size
    \adjustboxset{max size={0.9\linewidth}{0.9\paperheight}}
    \usepackage{float}
    \floatplacement{figure}{H} % forces figures to be placed at the correct location
    \usepackage{xcolor} % Allow colors to be defined
    \usepackage{enumerate} % Needed for markdown enumerations to work
    \usepackage{geometry} % Used to adjust the document margins
    \usepackage{amsmath} % Equations
    \usepackage{amssymb} % Equations
    \usepackage{textcomp} % defines textquotesingle
    % Hack from http://tex.stackexchange.com/a/47451/13684:
    \AtBeginDocument{%
        \def\PYZsq{\textquotesingle}% Upright quotes in Pygmentized code
    }
    \usepackage{upquote} % Upright quotes for verbatim code
    \usepackage{eurosym} % defines \euro
    \usepackage[mathletters]{ucs} % Extended unicode (utf-8) support
    \usepackage{fancyvrb} % verbatim replacement that allows latex
    \usepackage{grffile} % extends the file name processing of package graphics 
                         % to support a larger range
    \makeatletter % fix for grffile with XeLaTeX
    \def\Gread@@xetex#1{%
      \IfFileExists{"\Gin@base".bb}%
      {\Gread@eps{\Gin@base.bb}}%
      {\Gread@@xetex@aux#1}%
    }
    \makeatother

    % The hyperref package gives us a pdf with properly built
    % internal navigation ('pdf bookmarks' for the table of contents,
    % internal cross-reference links, web links for URLs, etc.)
    \usepackage{hyperref}
    % The default LaTeX title has an obnoxious amount of whitespace. By default,
    % titling removes some of it. It also provides customization options.
    \usepackage{titling}
    \usepackage{longtable} % longtable support required by pandoc >1.10
    \usepackage{booktabs}  % table support for pandoc > 1.12.2
    \usepackage[inline]{enumitem} % IRkernel/repr support (it uses the enumerate* environment)
    \usepackage[normalem]{ulem} % ulem is needed to support strikethroughs (\sout)
                                % normalem makes italics be italics, not underlines
    \usepackage{mathrsfs}
    

    
    % Colors for the hyperref package
    \definecolor{urlcolor}{rgb}{0,.145,.698}
    \definecolor{linkcolor}{rgb}{.71,0.21,0.01}
    \definecolor{citecolor}{rgb}{.12,.54,.11}

    % ANSI colors
    \definecolor{ansi-black}{HTML}{3E424D}
    \definecolor{ansi-black-intense}{HTML}{282C36}
    \definecolor{ansi-red}{HTML}{E75C58}
    \definecolor{ansi-red-intense}{HTML}{B22B31}
    \definecolor{ansi-green}{HTML}{00A250}
    \definecolor{ansi-green-intense}{HTML}{007427}
    \definecolor{ansi-yellow}{HTML}{DDB62B}
    \definecolor{ansi-yellow-intense}{HTML}{B27D12}
    \definecolor{ansi-blue}{HTML}{208FFB}
    \definecolor{ansi-blue-intense}{HTML}{0065CA}
    \definecolor{ansi-magenta}{HTML}{D160C4}
    \definecolor{ansi-magenta-intense}{HTML}{A03196}
    \definecolor{ansi-cyan}{HTML}{60C6C8}
    \definecolor{ansi-cyan-intense}{HTML}{258F8F}
    \definecolor{ansi-white}{HTML}{C5C1B4}
    \definecolor{ansi-white-intense}{HTML}{A1A6B2}
    \definecolor{ansi-default-inverse-fg}{HTML}{FFFFFF}
    \definecolor{ansi-default-inverse-bg}{HTML}{000000}

    % commands and environments needed by pandoc snippets
    % extracted from the output of `pandoc -s`
    \providecommand{\tightlist}{%
      \setlength{\itemsep}{0pt}\setlength{\parskip}{0pt}}
    \DefineVerbatimEnvironment{Highlighting}{Verbatim}{commandchars=\\\{\}}
    % Add ',fontsize=\small' for more characters per line
    \newenvironment{Shaded}{}{}
    \newcommand{\KeywordTok}[1]{\textcolor[rgb]{0.00,0.44,0.13}{\textbf{{#1}}}}
    \newcommand{\DataTypeTok}[1]{\textcolor[rgb]{0.56,0.13,0.00}{{#1}}}
    \newcommand{\DecValTok}[1]{\textcolor[rgb]{0.25,0.63,0.44}{{#1}}}
    \newcommand{\BaseNTok}[1]{\textcolor[rgb]{0.25,0.63,0.44}{{#1}}}
    \newcommand{\FloatTok}[1]{\textcolor[rgb]{0.25,0.63,0.44}{{#1}}}
    \newcommand{\CharTok}[1]{\textcolor[rgb]{0.25,0.44,0.63}{{#1}}}
    \newcommand{\StringTok}[1]{\textcolor[rgb]{0.25,0.44,0.63}{{#1}}}
    \newcommand{\CommentTok}[1]{\textcolor[rgb]{0.38,0.63,0.69}{\textit{{#1}}}}
    \newcommand{\OtherTok}[1]{\textcolor[rgb]{0.00,0.44,0.13}{{#1}}}
    \newcommand{\AlertTok}[1]{\textcolor[rgb]{1.00,0.00,0.00}{\textbf{{#1}}}}
    \newcommand{\FunctionTok}[1]{\textcolor[rgb]{0.02,0.16,0.49}{{#1}}}
    \newcommand{\RegionMarkerTok}[1]{{#1}}
    \newcommand{\ErrorTok}[1]{\textcolor[rgb]{1.00,0.00,0.00}{\textbf{{#1}}}}
    \newcommand{\NormalTok}[1]{{#1}}
    
    % Additional commands for more recent versions of Pandoc
    \newcommand{\ConstantTok}[1]{\textcolor[rgb]{0.53,0.00,0.00}{{#1}}}
    \newcommand{\SpecialCharTok}[1]{\textcolor[rgb]{0.25,0.44,0.63}{{#1}}}
    \newcommand{\VerbatimStringTok}[1]{\textcolor[rgb]{0.25,0.44,0.63}{{#1}}}
    \newcommand{\SpecialStringTok}[1]{\textcolor[rgb]{0.73,0.40,0.53}{{#1}}}
    \newcommand{\ImportTok}[1]{{#1}}
    \newcommand{\DocumentationTok}[1]{\textcolor[rgb]{0.73,0.13,0.13}{\textit{{#1}}}}
    \newcommand{\AnnotationTok}[1]{\textcolor[rgb]{0.38,0.63,0.69}{\textbf{\textit{{#1}}}}}
    \newcommand{\CommentVarTok}[1]{\textcolor[rgb]{0.38,0.63,0.69}{\textbf{\textit{{#1}}}}}
    \newcommand{\VariableTok}[1]{\textcolor[rgb]{0.10,0.09,0.49}{{#1}}}
    \newcommand{\ControlFlowTok}[1]{\textcolor[rgb]{0.00,0.44,0.13}{\textbf{{#1}}}}
    \newcommand{\OperatorTok}[1]{\textcolor[rgb]{0.40,0.40,0.40}{{#1}}}
    \newcommand{\BuiltInTok}[1]{{#1}}
    \newcommand{\ExtensionTok}[1]{{#1}}
    \newcommand{\PreprocessorTok}[1]{\textcolor[rgb]{0.74,0.48,0.00}{{#1}}}
    \newcommand{\AttributeTok}[1]{\textcolor[rgb]{0.49,0.56,0.16}{{#1}}}
    \newcommand{\InformationTok}[1]{\textcolor[rgb]{0.38,0.63,0.69}{\textbf{\textit{{#1}}}}}
    \newcommand{\WarningTok}[1]{\textcolor[rgb]{0.38,0.63,0.69}{\textbf{\textit{{#1}}}}}
    
    
    % Define a nice break command that doesn't care if a line doesn't already
    % exist.
    \def\br{\hspace*{\fill} \\* }
    % Math Jax compatibility definitions
    \def\gt{>}
    \def\lt{<}
    \let\Oldtex\TeX
    \let\Oldlatex\LaTeX
    \renewcommand{\TeX}{\textrm{\Oldtex}}
    \renewcommand{\LaTeX}{\textrm{\Oldlatex}}
    % Document parameters
    % Document title
    \title{
    {\Large Лекция 2} \\
    Аксиоматика теории вероятностей.
    Условная вероятность
  }
    
    
    
    
    
% Pygments definitions
\makeatletter
\def\PY@reset{\let\PY@it=\relax \let\PY@bf=\relax%
    \let\PY@ul=\relax \let\PY@tc=\relax%
    \let\PY@bc=\relax \let\PY@ff=\relax}
\def\PY@tok#1{\csname PY@tok@#1\endcsname}
\def\PY@toks#1+{\ifx\relax#1\empty\else%
    \PY@tok{#1}\expandafter\PY@toks\fi}
\def\PY@do#1{\PY@bc{\PY@tc{\PY@ul{%
    \PY@it{\PY@bf{\PY@ff{#1}}}}}}}
\def\PY#1#2{\PY@reset\PY@toks#1+\relax+\PY@do{#2}}

\expandafter\def\csname PY@tok@w\endcsname{\def\PY@tc##1{\textcolor[rgb]{0.73,0.73,0.73}{##1}}}
\expandafter\def\csname PY@tok@c\endcsname{\let\PY@it=\textit\def\PY@tc##1{\textcolor[rgb]{0.25,0.50,0.50}{##1}}}
\expandafter\def\csname PY@tok@cp\endcsname{\def\PY@tc##1{\textcolor[rgb]{0.74,0.48,0.00}{##1}}}
\expandafter\def\csname PY@tok@k\endcsname{\let\PY@bf=\textbf\def\PY@tc##1{\textcolor[rgb]{0.00,0.50,0.00}{##1}}}
\expandafter\def\csname PY@tok@kp\endcsname{\def\PY@tc##1{\textcolor[rgb]{0.00,0.50,0.00}{##1}}}
\expandafter\def\csname PY@tok@kt\endcsname{\def\PY@tc##1{\textcolor[rgb]{0.69,0.00,0.25}{##1}}}
\expandafter\def\csname PY@tok@o\endcsname{\def\PY@tc##1{\textcolor[rgb]{0.40,0.40,0.40}{##1}}}
\expandafter\def\csname PY@tok@ow\endcsname{\let\PY@bf=\textbf\def\PY@tc##1{\textcolor[rgb]{0.67,0.13,1.00}{##1}}}
\expandafter\def\csname PY@tok@nb\endcsname{\def\PY@tc##1{\textcolor[rgb]{0.00,0.50,0.00}{##1}}}
\expandafter\def\csname PY@tok@nf\endcsname{\def\PY@tc##1{\textcolor[rgb]{0.00,0.00,1.00}{##1}}}
\expandafter\def\csname PY@tok@nc\endcsname{\let\PY@bf=\textbf\def\PY@tc##1{\textcolor[rgb]{0.00,0.00,1.00}{##1}}}
\expandafter\def\csname PY@tok@nn\endcsname{\let\PY@bf=\textbf\def\PY@tc##1{\textcolor[rgb]{0.00,0.00,1.00}{##1}}}
\expandafter\def\csname PY@tok@ne\endcsname{\let\PY@bf=\textbf\def\PY@tc##1{\textcolor[rgb]{0.82,0.25,0.23}{##1}}}
\expandafter\def\csname PY@tok@nv\endcsname{\def\PY@tc##1{\textcolor[rgb]{0.10,0.09,0.49}{##1}}}
\expandafter\def\csname PY@tok@no\endcsname{\def\PY@tc##1{\textcolor[rgb]{0.53,0.00,0.00}{##1}}}
\expandafter\def\csname PY@tok@nl\endcsname{\def\PY@tc##1{\textcolor[rgb]{0.63,0.63,0.00}{##1}}}
\expandafter\def\csname PY@tok@ni\endcsname{\let\PY@bf=\textbf\def\PY@tc##1{\textcolor[rgb]{0.60,0.60,0.60}{##1}}}
\expandafter\def\csname PY@tok@na\endcsname{\def\PY@tc##1{\textcolor[rgb]{0.49,0.56,0.16}{##1}}}
\expandafter\def\csname PY@tok@nt\endcsname{\let\PY@bf=\textbf\def\PY@tc##1{\textcolor[rgb]{0.00,0.50,0.00}{##1}}}
\expandafter\def\csname PY@tok@nd\endcsname{\def\PY@tc##1{\textcolor[rgb]{0.67,0.13,1.00}{##1}}}
\expandafter\def\csname PY@tok@s\endcsname{\def\PY@tc##1{\textcolor[rgb]{0.73,0.13,0.13}{##1}}}
\expandafter\def\csname PY@tok@sd\endcsname{\let\PY@it=\textit\def\PY@tc##1{\textcolor[rgb]{0.73,0.13,0.13}{##1}}}
\expandafter\def\csname PY@tok@si\endcsname{\let\PY@bf=\textbf\def\PY@tc##1{\textcolor[rgb]{0.73,0.40,0.53}{##1}}}
\expandafter\def\csname PY@tok@se\endcsname{\let\PY@bf=\textbf\def\PY@tc##1{\textcolor[rgb]{0.73,0.40,0.13}{##1}}}
\expandafter\def\csname PY@tok@sr\endcsname{\def\PY@tc##1{\textcolor[rgb]{0.73,0.40,0.53}{##1}}}
\expandafter\def\csname PY@tok@ss\endcsname{\def\PY@tc##1{\textcolor[rgb]{0.10,0.09,0.49}{##1}}}
\expandafter\def\csname PY@tok@sx\endcsname{\def\PY@tc##1{\textcolor[rgb]{0.00,0.50,0.00}{##1}}}
\expandafter\def\csname PY@tok@m\endcsname{\def\PY@tc##1{\textcolor[rgb]{0.40,0.40,0.40}{##1}}}
\expandafter\def\csname PY@tok@gh\endcsname{\let\PY@bf=\textbf\def\PY@tc##1{\textcolor[rgb]{0.00,0.00,0.50}{##1}}}
\expandafter\def\csname PY@tok@gu\endcsname{\let\PY@bf=\textbf\def\PY@tc##1{\textcolor[rgb]{0.50,0.00,0.50}{##1}}}
\expandafter\def\csname PY@tok@gd\endcsname{\def\PY@tc##1{\textcolor[rgb]{0.63,0.00,0.00}{##1}}}
\expandafter\def\csname PY@tok@gi\endcsname{\def\PY@tc##1{\textcolor[rgb]{0.00,0.63,0.00}{##1}}}
\expandafter\def\csname PY@tok@gr\endcsname{\def\PY@tc##1{\textcolor[rgb]{1.00,0.00,0.00}{##1}}}
\expandafter\def\csname PY@tok@ge\endcsname{\let\PY@it=\textit}
\expandafter\def\csname PY@tok@gs\endcsname{\let\PY@bf=\textbf}
\expandafter\def\csname PY@tok@gp\endcsname{\let\PY@bf=\textbf\def\PY@tc##1{\textcolor[rgb]{0.00,0.00,0.50}{##1}}}
\expandafter\def\csname PY@tok@go\endcsname{\def\PY@tc##1{\textcolor[rgb]{0.53,0.53,0.53}{##1}}}
\expandafter\def\csname PY@tok@gt\endcsname{\def\PY@tc##1{\textcolor[rgb]{0.00,0.27,0.87}{##1}}}
\expandafter\def\csname PY@tok@err\endcsname{\def\PY@bc##1{\setlength{\fboxsep}{0pt}\fcolorbox[rgb]{1.00,0.00,0.00}{1,1,1}{\strut ##1}}}
\expandafter\def\csname PY@tok@kc\endcsname{\let\PY@bf=\textbf\def\PY@tc##1{\textcolor[rgb]{0.00,0.50,0.00}{##1}}}
\expandafter\def\csname PY@tok@kd\endcsname{\let\PY@bf=\textbf\def\PY@tc##1{\textcolor[rgb]{0.00,0.50,0.00}{##1}}}
\expandafter\def\csname PY@tok@kn\endcsname{\let\PY@bf=\textbf\def\PY@tc##1{\textcolor[rgb]{0.00,0.50,0.00}{##1}}}
\expandafter\def\csname PY@tok@kr\endcsname{\let\PY@bf=\textbf\def\PY@tc##1{\textcolor[rgb]{0.00,0.50,0.00}{##1}}}
\expandafter\def\csname PY@tok@bp\endcsname{\def\PY@tc##1{\textcolor[rgb]{0.00,0.50,0.00}{##1}}}
\expandafter\def\csname PY@tok@fm\endcsname{\def\PY@tc##1{\textcolor[rgb]{0.00,0.00,1.00}{##1}}}
\expandafter\def\csname PY@tok@vc\endcsname{\def\PY@tc##1{\textcolor[rgb]{0.10,0.09,0.49}{##1}}}
\expandafter\def\csname PY@tok@vg\endcsname{\def\PY@tc##1{\textcolor[rgb]{0.10,0.09,0.49}{##1}}}
\expandafter\def\csname PY@tok@vi\endcsname{\def\PY@tc##1{\textcolor[rgb]{0.10,0.09,0.49}{##1}}}
\expandafter\def\csname PY@tok@vm\endcsname{\def\PY@tc##1{\textcolor[rgb]{0.10,0.09,0.49}{##1}}}
\expandafter\def\csname PY@tok@sa\endcsname{\def\PY@tc##1{\textcolor[rgb]{0.73,0.13,0.13}{##1}}}
\expandafter\def\csname PY@tok@sb\endcsname{\def\PY@tc##1{\textcolor[rgb]{0.73,0.13,0.13}{##1}}}
\expandafter\def\csname PY@tok@sc\endcsname{\def\PY@tc##1{\textcolor[rgb]{0.73,0.13,0.13}{##1}}}
\expandafter\def\csname PY@tok@dl\endcsname{\def\PY@tc##1{\textcolor[rgb]{0.73,0.13,0.13}{##1}}}
\expandafter\def\csname PY@tok@s2\endcsname{\def\PY@tc##1{\textcolor[rgb]{0.73,0.13,0.13}{##1}}}
\expandafter\def\csname PY@tok@sh\endcsname{\def\PY@tc##1{\textcolor[rgb]{0.73,0.13,0.13}{##1}}}
\expandafter\def\csname PY@tok@s1\endcsname{\def\PY@tc##1{\textcolor[rgb]{0.73,0.13,0.13}{##1}}}
\expandafter\def\csname PY@tok@mb\endcsname{\def\PY@tc##1{\textcolor[rgb]{0.40,0.40,0.40}{##1}}}
\expandafter\def\csname PY@tok@mf\endcsname{\def\PY@tc##1{\textcolor[rgb]{0.40,0.40,0.40}{##1}}}
\expandafter\def\csname PY@tok@mh\endcsname{\def\PY@tc##1{\textcolor[rgb]{0.40,0.40,0.40}{##1}}}
\expandafter\def\csname PY@tok@mi\endcsname{\def\PY@tc##1{\textcolor[rgb]{0.40,0.40,0.40}{##1}}}
\expandafter\def\csname PY@tok@il\endcsname{\def\PY@tc##1{\textcolor[rgb]{0.40,0.40,0.40}{##1}}}
\expandafter\def\csname PY@tok@mo\endcsname{\def\PY@tc##1{\textcolor[rgb]{0.40,0.40,0.40}{##1}}}
\expandafter\def\csname PY@tok@ch\endcsname{\let\PY@it=\textit\def\PY@tc##1{\textcolor[rgb]{0.25,0.50,0.50}{##1}}}
\expandafter\def\csname PY@tok@cm\endcsname{\let\PY@it=\textit\def\PY@tc##1{\textcolor[rgb]{0.25,0.50,0.50}{##1}}}
\expandafter\def\csname PY@tok@cpf\endcsname{\let\PY@it=\textit\def\PY@tc##1{\textcolor[rgb]{0.25,0.50,0.50}{##1}}}
\expandafter\def\csname PY@tok@c1\endcsname{\let\PY@it=\textit\def\PY@tc##1{\textcolor[rgb]{0.25,0.50,0.50}{##1}}}
\expandafter\def\csname PY@tok@cs\endcsname{\let\PY@it=\textit\def\PY@tc##1{\textcolor[rgb]{0.25,0.50,0.50}{##1}}}

\def\PYZbs{\char`\\}
\def\PYZus{\char`\_}
\def\PYZob{\char`\{}
\def\PYZcb{\char`\}}
\def\PYZca{\char`\^}
\def\PYZam{\char`\&}
\def\PYZlt{\char`\<}
\def\PYZgt{\char`\>}
\def\PYZsh{\char`\#}
\def\PYZpc{\char`\%}
\def\PYZdl{\char`\$}
\def\PYZhy{\char`\-}
\def\PYZsq{\char`\'}
\def\PYZdq{\char`\"}
\def\PYZti{\char`\~}
% for compatibility with earlier versions
\def\PYZat{@}
\def\PYZlb{[}
\def\PYZrb{]}
\makeatother


    % For linebreaks inside Verbatim environment from package fancyvrb. 
    \makeatletter
        \newbox\Wrappedcontinuationbox 
        \newbox\Wrappedvisiblespacebox 
        \newcommand*\Wrappedvisiblespace {\textcolor{red}{\textvisiblespace}} 
        \newcommand*\Wrappedcontinuationsymbol {\textcolor{red}{\llap{\tiny$\m@th\hookrightarrow$}}} 
        \newcommand*\Wrappedcontinuationindent {3ex } 
        \newcommand*\Wrappedafterbreak {\kern\Wrappedcontinuationindent\copy\Wrappedcontinuationbox} 
        % Take advantage of the already applied Pygments mark-up to insert 
        % potential linebreaks for TeX processing. 
        %        {, <, #, %, $, ' and ": go to next line. 
        %        _, }, ^, &, >, - and ~: stay at end of broken line. 
        % Use of \textquotesingle for straight quote. 
        \newcommand*\Wrappedbreaksatspecials {% 
            \def\PYGZus{\discretionary{\char`\_}{\Wrappedafterbreak}{\char`\_}}% 
            \def\PYGZob{\discretionary{}{\Wrappedafterbreak\char`\{}{\char`\{}}% 
            \def\PYGZcb{\discretionary{\char`\}}{\Wrappedafterbreak}{\char`\}}}% 
            \def\PYGZca{\discretionary{\char`\^}{\Wrappedafterbreak}{\char`\^}}% 
            \def\PYGZam{\discretionary{\char`\&}{\Wrappedafterbreak}{\char`\&}}% 
            \def\PYGZlt{\discretionary{}{\Wrappedafterbreak\char`\<}{\char`\<}}% 
            \def\PYGZgt{\discretionary{\char`\>}{\Wrappedafterbreak}{\char`\>}}% 
            \def\PYGZsh{\discretionary{}{\Wrappedafterbreak\char`\#}{\char`\#}}% 
            \def\PYGZpc{\discretionary{}{\Wrappedafterbreak\char`\%}{\char`\%}}% 
            \def\PYGZdl{\discretionary{}{\Wrappedafterbreak\char`\$}{\char`\$}}% 
            \def\PYGZhy{\discretionary{\char`\-}{\Wrappedafterbreak}{\char`\-}}% 
            \def\PYGZsq{\discretionary{}{\Wrappedafterbreak\textquotesingle}{\textquotesingle}}% 
            \def\PYGZdq{\discretionary{}{\Wrappedafterbreak\char`\"}{\char`\"}}% 
            \def\PYGZti{\discretionary{\char`\~}{\Wrappedafterbreak}{\char`\~}}% 
        } 
        % Some characters . , ; ? ! / are not pygmentized. 
        % This macro makes them "active" and they will insert potential linebreaks 
        \newcommand*\Wrappedbreaksatpunct {% 
            \lccode`\~`\.\lowercase{\def~}{\discretionary{\hbox{\char`\.}}{\Wrappedafterbreak}{\hbox{\char`\.}}}% 
            \lccode`\~`\,\lowercase{\def~}{\discretionary{\hbox{\char`\,}}{\Wrappedafterbreak}{\hbox{\char`\,}}}% 
            \lccode`\~`\;\lowercase{\def~}{\discretionary{\hbox{\char`\;}}{\Wrappedafterbreak}{\hbox{\char`\;}}}% 
            \lccode`\~`\:\lowercase{\def~}{\discretionary{\hbox{\char`\:}}{\Wrappedafterbreak}{\hbox{\char`\:}}}% 
            \lccode`\~`\?\lowercase{\def~}{\discretionary{\hbox{\char`\?}}{\Wrappedafterbreak}{\hbox{\char`\?}}}% 
            \lccode`\~`\!\lowercase{\def~}{\discretionary{\hbox{\char`\!}}{\Wrappedafterbreak}{\hbox{\char`\!}}}% 
            \lccode`\~`\/\lowercase{\def~}{\discretionary{\hbox{\char`\/}}{\Wrappedafterbreak}{\hbox{\char`\/}}}% 
            \catcode`\.\active
            \catcode`\,\active 
            \catcode`\;\active
            \catcode`\:\active
            \catcode`\?\active
            \catcode`\!\active
            \catcode`\/\active 
            \lccode`\~`\~ 	
        }
    \makeatother

    \let\OriginalVerbatim=\Verbatim
    \makeatletter
    \renewcommand{\Verbatim}[1][1]{%
        %\parskip\z@skip
        \sbox\Wrappedcontinuationbox {\Wrappedcontinuationsymbol}%
        \sbox\Wrappedvisiblespacebox {\FV@SetupFont\Wrappedvisiblespace}%
        \def\FancyVerbFormatLine ##1{\hsize\linewidth
            \vtop{\raggedright\hyphenpenalty\z@\exhyphenpenalty\z@
                \doublehyphendemerits\z@\finalhyphendemerits\z@
                \strut ##1\strut}%
        }%
        % If the linebreak is at a space, the latter will be displayed as visible
        % space at end of first line, and a continuation symbol starts next line.
        % Stretch/shrink are however usually zero for typewriter font.
        \def\FV@Space {%
            \nobreak\hskip\z@ plus\fontdimen3\font minus\fontdimen4\font
            \discretionary{\copy\Wrappedvisiblespacebox}{\Wrappedafterbreak}
            {\kern\fontdimen2\font}%
        }%
        
        % Allow breaks at special characters using \PYG... macros.
        \Wrappedbreaksatspecials
        % Breaks at punctuation characters . , ; ? ! and / need catcode=\active 	
        \OriginalVerbatim[#1,codes*=\Wrappedbreaksatpunct]%
    }
    \makeatother

    % Exact colors from NB
    \definecolor{incolor}{HTML}{303F9F}
    \definecolor{outcolor}{HTML}{D84315}
    \definecolor{cellborder}{HTML}{CFCFCF}
    \definecolor{cellbackground}{HTML}{F7F7F7}
    
    % prompt
    \makeatletter
    \newcommand{\boxspacing}{\kern\kvtcb@left@rule\kern\kvtcb@boxsep}
    \makeatother
    \newcommand{\prompt}[4]{
        \ttfamily\llap{{\color{#2}[#3]:\hspace{3pt}#4}}\vspace{-\baselineskip}
    }
    

    
    % Prevent overflowing lines due to hard-to-break entities
    \sloppy 
    % Setup hyperref package
    \hypersetup{
      breaklinks=true,  % so long urls are correctly broken across lines
      colorlinks=true,
      urlcolor=urlcolor,
      linkcolor=linkcolor,
      citecolor=citecolor,
      }
    % Slightly bigger margins than the latex defaults
    
    \geometry{verbose,tmargin=1in,bmargin=1in,lmargin=1in,rmargin=1in}
    
    

\begin{document}

\maketitle
\tableofcontents
\pagebreak



    \begin{tcolorbox}[breakable, size=fbox, boxrule=1pt, pad at break*=1mm,colback=cellbackground, colframe=cellborder]
\prompt{In}{incolor}{1}{\boxspacing}
\begin{Verbatim}[commandchars=\\\{\}]
\PY{c+c1}{\PYZsh{} Imports}
\PY{k+kn}{import} \PY{n+nn}{numpy} \PY{k}{as} \PY{n+nn}{np}
\PY{k+kn}{import} \PY{n+nn}{matplotlib}\PY{n+nn}{.}\PY{n+nn}{pyplot} \PY{k}{as} \PY{n+nn}{plt}
\end{Verbatim}
\end{tcolorbox}

    \begin{tcolorbox}[breakable, size=fbox, boxrule=1pt, pad at break*=1mm,colback=cellbackground, colframe=cellborder]
\prompt{In}{incolor}{2}{\boxspacing}
\begin{Verbatim}[commandchars=\\\{\}]
\PY{c+c1}{\PYZsh{} Styles}
\PY{k+kn}{import} \PY{n+nn}{warnings}
\PY{c+c1}{\PYZsh{} warnings.simplefilter(action=\PYZsq{}ignore\PYZsq{}, category=FutureWarning)}
\PY{n}{warnings}\PY{o}{.}\PY{n}{filterwarnings}\PY{p}{(}\PY{l+s+s1}{\PYZsq{}}\PY{l+s+s1}{ignore}\PY{l+s+s1}{\PYZsq{}}\PY{p}{)}

\PY{k+kn}{import} \PY{n+nn}{matplotlib}
\PY{n}{matplotlib}\PY{o}{.}\PY{n}{rcParams}\PY{p}{[}\PY{l+s+s1}{\PYZsq{}}\PY{l+s+s1}{font.size}\PY{l+s+s1}{\PYZsq{}}\PY{p}{]} \PY{o}{=} \PY{l+m+mi}{14}
\PY{n}{matplotlib}\PY{o}{.}\PY{n}{rcParams}\PY{p}{[}\PY{l+s+s1}{\PYZsq{}}\PY{l+s+s1}{lines.linewidth}\PY{l+s+s1}{\PYZsq{}}\PY{p}{]} \PY{o}{=} \PY{l+m+mf}{1.5}
\PY{n}{matplotlib}\PY{o}{.}\PY{n}{rcParams}\PY{p}{[}\PY{l+s+s1}{\PYZsq{}}\PY{l+s+s1}{lines.markersize}\PY{l+s+s1}{\PYZsq{}}\PY{p}{]} \PY{o}{=} \PY{l+m+mi}{4}
\PY{n}{cm} \PY{o}{=} \PY{n}{plt}\PY{o}{.}\PY{n}{cm}\PY{o}{.}\PY{n}{tab10}  \PY{c+c1}{\PYZsh{} Colormap}

\PY{k+kn}{import} \PY{n+nn}{seaborn}
\PY{n}{seaborn}\PY{o}{.}\PY{n}{set\PYZus{}style}\PY{p}{(}\PY{l+s+s1}{\PYZsq{}}\PY{l+s+s1}{whitegrid}\PY{l+s+s1}{\PYZsq{}}\PY{p}{)}
\end{Verbatim}
\end{tcolorbox}

    \begin{center}\rule{0.5\linewidth}{\linethickness}\end{center}

    \hypertarget{ux43fux440ux435ux434ux43cux435ux442-ux442ux435ux43eux440ux438ux438-ux432ux435ux440ux43eux44fux442ux43dux43eux441ux442ux435ux439}{%
\section{Предмет теории
вероятностей}\label{ux43fux440ux435ux434ux43cux435ux442-ux442ux435ux43eux440ux438ux438-ux432ux435ux440ux43eux44fux442ux43dux43eux441ux442ux435ux439}}

Предметом теории вероятностей является математический анализ случайных
явлений --- эмпирических феноменов, которые (при заданном «комплексе
условий») могут быть охарактеризованы тем, что

\begin{itemize}
\tightlist
\item
  для них отсутствует \emph{детерминистическая регулярность} (наблюдения
  над ними не всегда приводят к одним и тем же исходам)
\end{itemize}

и в то же самое время

\begin{itemize}
\tightlist
\item
  они обладают некоторой \emph{статистической регулярностью}
  (проявляющейся в статистической устойчивости частот).
\end{itemize}

Поясним сказанное на классическом примере «честного» подбрасывания
«правильной» монеты. Ясно, что заранее невозможно с определенностью
предсказать исход каждого подбрасывания. Результаты отдельных
экспериментов носят крайне нерегулярный характер (то «герб», то
«решетка»), и кажется, что это лишает нас возможности познать какие-либо
закономерности, связанные с этими экспериментами. Однако, если провести
большое число «независимых» подбрасываний, то можно заметить, что для
«правильной» монеты будет наблюдаться вполне определенная статистической
регулярность, проявляющаяся в том, что частота выпадания «герба» будет
«близка» к \(1/2\).

    \begin{center}\rule{0.5\linewidth}{\linethickness}\end{center}

    \hypertarget{ux432ux435ux440ux43eux44fux442ux43dux43eux441ux442ux43dux430ux44f-ux43cux43eux434ux435ux43bux44c}{%
\section{Вероятностная
модель}\label{ux432ux435ux440ux43eux44fux442ux43dux43eux441ux442ux43dux430ux44f-ux43cux43eux434ux435ux43bux44c}}

Согласно аксиоматике Колмогорова первоначальным объектом теории
вероятностей является \emph{вероятностное пространство}
\((\Omega, \mathcal{F}, \mathrm{P})\). Здесь \(\Omega\) --- это
множество, состоящее из элементарных событий \(\omega\), с выделенной на
нём системой его подмножеств (событий) \(\mathcal{F}\), образующих
\(\sigma\)-алгебру, а \(\mathrm{P}\) --- вероятностная мера
(вероятность), определённая на множествах из \(\mathcal{F}\).

А теперь подробнее.

    \hypertarget{ux43fux440ux43eux441ux442ux440ux430ux43dux441ux442ux432ux43e-ux44dux43bux435ux43cux435ux43dux442ux430ux440ux43dux44bux445-ux438ux441ux445ux43eux434ux43eux432}{%
\subsection{Пространство элементарных
исходов}\label{ux43fux440ux43eux441ux442ux440ux430ux43dux441ux442ux432ux43e-ux44dux43bux435ux43cux435ux43dux442ux430ux440ux43dux44bux445-ux438ux441ux445ux43eux434ux43eux432}}

\textbf{Определение}. \emph{Пространством элементарных исходов}
называется множество \(\Omega\), содержащее все возможные
\emph{взаимоисключающие} результаты данного случайного эксперимента.
Элементы множества \(\Omega\) называются элементарными исходами и
обозначаются буквой \(\omega\).

Элементарный исход --- это мельчайший неделимый результат
эксперимента.\\
Выделение пространства элементарных исходов представляет собой первый
шаг в формулировании понятия \emph{вероятностной модели} (вероятностной
«теории») того или иного эксперимента.

    \textbf{Примеры:}

\begin{itemize}
\tightlist
\item
  однократное подбрасывание монеты (пространство исходов состоит из двух
  точек: Г --- «герб», Р --- «решетка»);
\item
  n-кратное подбрасывание монеты;
\item
  выбор шаров с возвращением (упорядоченные и неупорядоченные выборки);
\item
  выбор шаров без возвращения (упорядоченные и неупорядоченные выборки).
\end{itemize}

    \hypertarget{ux430ux43bux433ux435ux431ux440ux430-ux438-mathbfsigma-ux430ux43bux433ux435ux431ux440ux430-ux441ux43eux431ux44bux442ux438ux439}{%
\subsection{\texorpdfstring{Алгебра и \(\mathbf{\sigma}\)-алгебра
событий}{Алгебра и \textbackslash{}mathbf\{\textbackslash{}sigma\}-алгебра событий}}\label{ux430ux43bux433ux435ux431ux440ux430-ux438-mathbfsigma-ux430ux43bux433ux435ux431ux440ux430-ux441ux43eux431ux44bux442ux438ux439}}

Наряду с понятием пространства элементарных исходов введём теперь важное
понятие события, лежащее в основе построения всякой вероятностной модели
рассматриваемого эксперимента.

Мы собираемся определить набор подмножеств множества \(\Omega\), которые
будут называться событиями, и затем задать вероятность как функцию,
определённую \emph{только} на множестве событий.

\emph{Событиями} мы будем называть не любые подмножества \(\Omega\), а
лишь элементы некоторого выделенного набора подмножеств множества
\(\Omega\). В качестве наборов событий целесообразно рассматривать
системы множеств, являющиеся \emph{алгебрами}. Для этого необходимо
позаботиться, чтобы этот набор подмножеств был \emph{замкнут}
относительно операций объединения, пересечения и дополнения.

\textbf{Определение.} Множество \(\mathcal{F}\), элементами которого
являются подмножества множества \(\Omega\) называется
\(\sigma\)-\emph{алгеброй}, если оно удовлетворяет следующим аксиомам:

\begin{enumerate}
\def\labelenumi{\arabic{enumi}.}
\tightlist
\item
  \(\Omega \in \mathcal{F}\) (\(\sigma\)-алгебра содержит
  \emph{достоверное событие});
\item
  если \(A \in \mathcal{F}\), то \(\overline{A} \in \mathcal{F}\)
  (вместе с любым множеством \(\sigma\)-алгебра содержит противоположное
  к нему);
\item
  если \(A_1,\,A_2,\,\ldots \in \mathcal{F}\), то
  \(\bigcup\limits_{i=1}^{\infty}A_i \in \mathcal{F}\) (вместе с любым
  \emph{счётным} набором событий \(\sigma\)-алгебра содержит их
  объединение).
\end{enumerate}

    \textbf{Свойства:}

\begin{enumerate}
\def\labelenumi{\arabic{enumi}.}
\tightlist
\item
  Из аксиом 1 и 2 следует, что пустое множество
  \(\emptyset = \overline{\Omega}\) также содержится в \(\mathcal{F}\),
  т. е. алгебра содержит и \emph{невозможное событие}.\\
\item
  Из аксиом 2 и 3 следует, что вместе с любым счётным набором событий
  \(\sigma\)-алгебра содержит не только их объединение
  \(\bigcup\limits_{i=1}^{\infty}A_i\), но и их пересечение
  \(\bigcap\limits_{i=1}^{\infty}A_i\).
\end{enumerate}

\textbf{Примеры:}

\begin{enumerate}
\def\labelenumi{\arabic{enumi}.}
\tightlist
\item
  \(\mathcal{F} = \{ \Omega, \emptyset \}\) --- система, состоящая из
  \(\Omega\) и пустого множества (так называемая тривиальная алгебра);
\item
  \(\mathcal{F} = \{ A, \overline{A}, \Omega, \emptyset \}\) ---
  система, порождённая событием \(A\);
\item
  \(\mathcal{F} = \{ A: A \subseteq \Omega \}\) --- совокупность всех
  подмножеств \(\Omega\) (обозначается \(2^\Omega\)).
\end{enumerate}

\begin{quote}
\emph{Упражнение.} Доказать, что если \(\Omega\) состоит из \(n\)
элементов, то в множестве всех его подмножеств ровно \(2^n\) элементов.
\end{quote}

    \hypertarget{ux432ux435ux440ux43eux44fux442ux43dux43eux441ux442ux43dux430ux44f-ux43cux435ux440ux430}{%
\subsection{Вероятностная
мера}\label{ux432ux435ux440ux43eux44fux442ux43dux43eux441ux442ux43dux430ux44f-ux43cux435ux440ux430}}

Пока мы сделали два первых шага к построению вероятностной модели
эксперимента: выделили пространство исходов \(\Omega\) и некоторую
систему \(\mathcal{F}\) его подмножеств, образующих \(\sigma\)-алгебру и
называемых событиями. Сделаем теперь следующий шаг, а именно введём
вероятностную меру.

\textbf{Определение.} Пусть \(\Omega\) --- непустое множество, а
\(\mathcal{F}\) --- \(\sigma\)-алгебра его подмножеств. Функция
\(\mathrm{P}: \mathcal{F} \rightarrow \mathbb{R}\) называется
\emph{вероятностной мерой}, если она удовлетворяет следующим аксиомам:

\begin{enumerate}
\def\labelenumi{\arabic{enumi}.}
\tightlist
\item
  \(\mathrm{P}(A) \ge 0\), \(A \in \mathcal{F}\) (неотрицательность);
\item
  \(\mathrm{P}\left( \bigcup\limits_{i=1}^{\infty}A_i \right) = \sum\limits_{i=1}^{\infty}\mathrm{P}(A_i)\),
  где \(A_i \in \mathcal{F}\), \(A_i \cap A_j = \emptyset\), \(i \ne j\)
  (счётная или \(\sigma\)-аддитивность)\\
  (для любого счётного набора попарно несовместных событий мера их
  объединения равна сумме их мер);
\item
  \(\mathrm{P}(\Omega) = 1\) (нормированность).
\end{enumerate}

\emph{Замечание 1.} Аксиомы 1 и 2 задают \emph{меру} как неотрицательную
\(\sigma\)-аддитивную функцию множеств, аксиома 3 определяет вероятность
как \emph{нормированную меру}.

\emph{Замечание 2.} Существуют примеры неизмеримых множеств, например,
\emph{множество Витали}.

    \textbf{Свойства:}

\begin{enumerate}
\def\labelenumi{\arabic{enumi}.}
\tightlist
\item
  \(\mathrm{P}(\emptyset) = 0\);
\item
  \(\mathrm{P}(\overline{A}) = 1 - \mathrm{P}(A)\);
\item
  \(\mathrm{P}(A \cup B) = \mathrm{P}(A) + \mathrm{P}(B) - \mathrm{P}(A \cap B)\);
\item
  \(\mathrm{P}\left( \bigcup\limits_{i=1}^{n}A_i \right) = \sum\limits_{i}\mathrm{P}(A_i) - \sum\limits_{i<j}\mathrm{P}(A_i \cap A_j) + \sum\limits_{i<j<k}\mathrm{P}(A_i \cap A_j \cap A_k) + \ldots + (-1)^{n-1} \mathrm{P}\left( \bigcap\limits_{i=1}^{n}A_i \right)\)
  --- формула включения-исключения.
\end{enumerate}

    \hypertarget{ux432ux435ux440ux43eux44fux442ux43dux43eux441ux442ux43dux43eux435-ux43fux440ux43eux441ux442ux440ux430ux43dux441ux442ux432ux43e}{%
\subsection{Вероятностное
пространство}\label{ux432ux435ux440ux43eux44fux442ux43dux43eux441ux442ux43dux43eux435-ux43fux440ux43eux441ux442ux440ux430ux43dux441ux442ux432ux43e}}

\textbf{Определение.} Тройка объектов

\[ \left( \Omega, \mathcal{F}, \mathrm{P} \right), \]

где \(\Omega\) --- множество элементарных исходов, \(\mathcal{F}\) ---
\(\sigma\)-алгебра его подмножеств и \(\mathrm{P}\) --- вероятностная
мера на \(\mathcal{F}\), называется \emph{вероятностным пространством}.

    \hypertarget{ux437ux430ux43cux435ux447ux430ux43dux438ux44f}{%
\subsection{Замечания}\label{ux437ux430ux43cux435ux447ux430ux43dux438ux44f}}

\hypertarget{ux43fux43eux441ux442ux440ux43eux435ux43dux438ux435-ux432ux435ux440ux43eux44fux442ux43dux43eux441ux442ux43dux43eux433ux43e-ux43fux440ux43eux441ux442ux440ux430ux43dux441ux442ux432ux430}{%
\paragraph{Построение вероятностного
пространства}\label{ux43fux43eux441ux442ux440ux43eux435ux43dux438ux435-ux432ux435ux440ux43eux44fux442ux43dux43eux441ux442ux43dux43eux433ux43e-ux43fux440ux43eux441ux442ux440ux430ux43dux441ux442ux432ux430}}

При построении вероятностных моделей в конкретных ситуациях выделение
пространства элементарных событий \(\Omega\) и алгебры событий
\(\mathcal{F}\), как правило, не является сложной задачей. При этом в
элементарной теории вероятностей в качестве алгебры \(\mathcal{F}\)
обычно берется алгебра \emph{всех} подмножеств \(\Omega\). Труднее
обстоит дело с вопросом о том, как задавать вероятности элементарных
событий. В сущности, ответ на этот вопрос лежит вне рамок теории
вероятностей, и мы его подробно не рассматриваем, считая, что основной
нашей задачей является не вопрос о том, как приписывать исходам те или
иные вероятности, а \emph{вычисление} вероятностей сложных событий
(событий из \(\mathcal{F}\)) по вероятностям элементарных событий.

С математической точки зрения ясно, что в случае конечного пространства
элементарных событий с помощью приписывания исходам
\(\omega_1, \ldots , \omega_N\) неотрицательных чисел
\(p_1, \ldots , p_N\), удовлетворяющих условию
\(p_1 + \ldots + p_N = 1\), мы получаем все мыслимые (конечные)
вероятностные пространства.

\emph{Правильность} же назначенных для конкретной ситуации значений
\(p_1, \ldots , p_N\) может быть до известной степени проверена с
помощью \emph{закона больших чисел}, согласно которому в длинных сериях
«независимых» экспериментов, происходящих при одинаковых условиях,
частоты появления элементарных событий «близки» к их вероятностям.

    \hypertarget{ux442ux430ux43a-ux447ux442ux43e-ux436ux435-ux442ux430ux43aux43eux435-ux432ux435ux440ux43eux44fux442ux43dux43eux441ux442ux44c}{%
\paragraph{Так что же такое
вероятность?}\label{ux442ux430ux43a-ux447ux442ux43e-ux436ux435-ux442ux430ux43aux43eux435-ux432ux435ux440ux43eux44fux442ux43dux43eux441ux442ux44c}}

\textbf{Классическое определение вероятности.} Это то, чему нас учат в
школе. Оно основано на симметрии монет, костей, перетасованных колод
карт и так далее и может быть сформулировано как «отношение числа
благоприятных исходов к числу всех исходов, если все исходы
равновозможны». Например, вероятность выпадения единицы на правильной
кости равна 1/6, потому что возможны 6 исходов, а нас устраивает один.
Однако это определение в какой-то степени носит круговой характер,
поскольку прежде мы должны уяснить, что значит равновозможны.

\textbf{«Перечислительная» вероятность.} Предположим, в ящике лежат три
белых и четыре черных носка. Если вытаскивать носок случайным образом,
то чему равна вероятность, что он белый? Ответ 3/7 можно получить путем
простого перечисления всех возможностей. Многие из нас страдали от таких
вопросов в школе, и здесь мы фактически имеем дело с расширением
рассмотренной выше классической идеи, где требуется случайный выбор из
группы физических объектов. Мы уже использовали эту идею при описании
случайного выбора элемента данных из общей генеральной совокупности.

\textbf{Вероятность как частота.} Такое определение говорит о
вероятности как о доле случаев, когда интересующее нас событие наступает
в бесконечной последовательности идентичных экспериментов --- в точности
так как при моделировании двух вариантов игры шевалье де Мере. Для
бесконечно повторяющихся событий это может быть разумно (хотя бы
теоретически), но как насчет уникальных одноразовых событий, например
скачек или завтрашней погоды? На деле практически любая реальная
ситуация даже в принципе не может быть бесконечно воспроизводимой.

\textbf{Субъективная, или «личная», вероятность.} Это степень веры
конкретного человека в какое-либо событие, основанная на его нынешних
знаниях. Субъективная вероятность означает, что любая численная
вероятность фактически \emph{строится} в соответствии с тем, что
известно в нынешней ситуации, --- и на самом деле вероятность вообще не
«существует» (за исключением, возможно, субатомного уровня). Такой
подход лежит в основе \textbf{байесовской} школы статистики.

    \begin{center}\rule{0.5\linewidth}{\linethickness}\end{center}

    \hypertarget{ux43fux440ux438ux43cux435ux440ux44b}{%
\section{Примеры}\label{ux43fux440ux438ux43cux435ux440ux44b}}

    \hypertarget{ux431ux438ux43dux43eux43cux438ux430ux43bux44cux43dux43eux435-ux440ux430ux441ux43fux440ux435ux434ux435ux43bux435ux43dux438ux435}{%
\subsection{Биномиальное
распределение}\label{ux431ux438ux43dux43eux43cux438ux430ux43bux44cux43dux43eux435-ux440ux430ux441ux43fux440ux435ux434ux435ux43bux435ux43dux438ux435}}

Предположим, что монета подбрасывается \(n\) раз и результат наблюдений
записывается в виде упорядоченного набора \((a_1, \ldots, a_n)\), где
\(a_i = 1\) в случае появления «герба» («успех») и \(a_i = 0\) в случае
появления «решетки» («неуспех»). Пространство всех исходов имеет
следующую структуру:
\[ \Omega= \left\{ \omega: \omega = (a_1, \ldots, a_n), \quad a_i = 0 \; \mathrm{или} \; 1 \right\}. \]

Припишем каждому элементарному событию \(\omega = (a_1, \ldots, a_n)\)
вероятность («вес») \[ p(\omega) = p^{\sum a_i} q^{n-\sum a_i}, \] где
неотрицательные числа \(p\) и \(q\) таковы, что \(p + q = 1\).

Итак, пространство \(\Omega\) вместе с системой \(\mathcal{A}\) всех его
подмножеств и вероятностями
\(\mathrm{P}(A) = \sum\limits_{\omega \in A}p(\omega), \; A \in \mathcal{A}\)
(в частности,
\(\mathrm{P}(\{\omega\}) = p(\omega), \; \omega \in \Omega\)) определяет
некоторую вероятностную модель. Естественно назвать её
\emph{вероятностной моделью, описывающей \(n\)-кратное подбрасывание
монеты}.

Введём в рассмотрение события \[ 
    A_k = \left\{\omega: \omega=(a_1, \ldots, a_n), a_1 + \ldots + a_n = k\right\}, \quad k = 0, 1, \ldots, n,
\] означающие, что произойдет в точности \(k\) «успехов». Тогда
вероятность события \(A_k\) равна
\[ \mathrm{P}(A_k) = C_n^k p^k q^{n-k}, \] причём
\(\sum\limits_{k=0}^n \mathrm{P}(A_k) = 1\).

Набор вероятностей \(\{\mathrm{P}(A_0), \ldots,\mathrm{P}(A_n)\}\)
называется \emph{биномиальным распределением} (числа «успехов» в выборке
объёма \(n\)).

    \hypertarget{ux433ux438ux43fux435ux440ux433ux435ux43eux43cux435ux442ux440ux438ux447ux435ux441ux43aux43eux435-ux440ux430ux441ux43fux440ux435ux434ux435ux43bux435ux43dux438ux435}{%
\subsection{Гипергеометрическое
распределение}\label{ux433ux438ux43fux435ux440ux433ux435ux43eux43cux435ux442ux440ux438ux447ux435ux441ux43aux43eux435-ux440ux430ux441ux43fux440ux435ux434ux435ux43bux435ux43dux438ux435}}

Рассмотрим урну, содержащую \(N\) шаров, из которых \(M\) шаров имеют
белый цвет. Предположим, что осуществляется выбор без возвращения объёма
\(n < N\). Вероятность события \(B_m\), состоящего в том, что \(m\)
шаров из выборки имеют белый цвет равна

\[ \mathrm{P}(B_m) = \dfrac{C_M^m C_{N-M}^{n-m}}{C_N^n}. \]

Набор вероятностей \(\{\mathrm{P}(B_0), \ldots,\mathrm{P}(B_n)\}\) носит
название многомерного гипергеометрического распределения.

    \hypertarget{ux43eux446ux435ux43dux43aux430-ux43cux430ux43aux441ux438ux43cux430ux43bux44cux43dux43eux433ux43e-ux43fux440ux430ux432ux434ux43eux43fux43eux434ux43eux431ux438ux44f}{%
\section{Оценка максимального
правдоподобия}\label{ux43eux446ux435ux43dux43aux430-ux43cux430ux43aux441ux438ux43cux430ux43bux44cux43dux43eux433ux43e-ux43fux440ux430ux432ux434ux43eux43fux43eux434ux43eux431ux438ux44f}}

Пусть \(N\) --- размер некоторой популяции, который требуется оценить
«минимальными средствами» без простого пересчета всех элементов этой
совокупности. Подобного рода вопрос интересен, например, при оценке
числа жителей в той или иной стране, городе и т. д.

В 1786 г. Лаплас для оценки числа \(N\) жителей Франции предложил
следующий метод.

\begin{enumerate}
\def\labelenumi{\arabic{enumi}.}
\tightlist
\item
  Выберем некоторое число \(M\), элементов популяции и пометим их.
\item
  Возвратим их в основную совокупность и предположим, что они «хорошо
  перемешаны» с немаркированными элементами.
\item
  После этого возьмём из «перемешанной» популяции \(n\) элементов.
\item
  Пусть среди них \(m\) элементов оказались маркированными.
\end{enumerate}

Вероятность \(\mathrm{P}(B_m(N))\) задается формулой
гипергеометрического распределения: \[
    \mathrm{P}(B_m(N)) = \frac{C_M^m C_{N-M}^{n-m}}{C_N^n}. \tag{1}\label{eq:prob}
\]

Нам известны числа \(M\), \(n\) и \(m\), а \(N\) (размер популяциии) ---
нет, его требуется оценить.

Для каждого частного набора наблюдений \(M\), \(n\) и \(m\) значение
\(N\), при котором вероятность \(\mathrm{P}(B_m(N))\) максимальна,
называется \textbf{оценкой максимального правдоподобия}. Обозначим
наиболее правдоподобное значение через \(\hat{N}\).

Можно показать, что \(\hat{N}\) определяется формулой (\([\cdot]\) ---
целая часть):
\[ \hat{N} = \left[\dfrac{Mn}{m}\right]. \tag{2}\label{eq:max} \]

\begin{quote}
\emph{Задание.} Получить формулу \(\eqref{eq:max}\).\\
Подсказка: можно воспользоваться формулой Стирлинга
\(n! \sim \sqrt{2 \pi n}\left( \dfrac{n}{e} \right)^n\).
\end{quote}

    \hypertarget{ux437ux430ux434ux430ux447ux430-ux43eux431-ux43eux446ux435ux43dux43aux435-ux433ux435ux43dux435ux440ux430ux43bux44cux43dux43eux439-ux441ux43eux432ux43eux43aux443ux43fux43dux43eux441ux442ux438-ux43fux43e-ux432ux44bux431ux43eux440ux43aux435}{%
\subsection{Задача об оценке генеральной совокупности по
выборке}\label{ux437ux430ux434ux430ux447ux430-ux43eux431-ux43eux446ux435ux43dux43aux435-ux433ux435ux43dux435ux440ux430ux43bux44cux43dux43eux439-ux441ux43eux432ux43eux43aux443ux43fux43dux43eux441ux442ux438-ux43fux43e-ux432ux44bux431ux43eux440ux43aux435}}

Применим метод максимального правдоподобия для оценки количества рыб в
озере. Пусть, например, \(M=1000\), \(n=1000\), а \(m=100\). Тогда всё,
что нам достоверно известно о количестве рыб, это
\(N \ge n + M - m = 1900\). Вообще говоря, не исключено, что в озере их
ровно \(1900\). Однако, отправляясь от этой гипотезы, мы придём к
выводу, что случилось событие фантастически малой вероятности.
Действительно, вероятность того, что выборка объёмом \(n=1000\) из
генеральной совокупности объёма \(N=1900\) будет содержать \(m=100\)
маркированных объектов, если общее число маркированных объектов
\(M=1000\), по формуле \eqref{eq:prob} равна \[
    \mathrm{P}(B_{100}(1900)) = \frac{C_{1000}^{100} C_{900}^{900}}{C_{1900}^{1000}} = \frac{(1000!)^2}{100! \, 1900!} \sim 10^{-430}.
\]

Аналогичное рассуждение заставляет нас откинуть гипотезу о том, что
\(N\) очень велико, скажем равно миллиону
(\(\mathrm{P}(B_{100}(10^6)) \sim 10^{-163}\)).

    \begin{tcolorbox}[breakable, size=fbox, boxrule=1pt, pad at break*=1mm,colback=cellbackground, colframe=cellborder]
\prompt{In}{incolor}{3}{\boxspacing}
\begin{Verbatim}[commandchars=\\\{\}]
\PY{k+kn}{import} \PY{n+nn}{mpmath}
\PY{k+kn}{from} \PY{n+nn}{scipy}\PY{n+nn}{.}\PY{n+nn}{stats} \PY{k+kn}{import} \PY{n}{hypergeom}

\PY{n}{P\PYZus{}1e2} \PY{o}{=} \PY{n}{mpmath}\PY{o}{.}\PY{n}{fac}\PY{p}{(}\PY{l+m+mi}{1000}\PY{p}{)}\PY{o}{*}\PY{o}{*}\PY{l+m+mi}{2} \PY{o}{/} \PY{n}{mpmath}\PY{o}{.}\PY{n}{fac}\PY{p}{(}\PY{l+m+mi}{100}\PY{p}{)} \PY{o}{/} \PY{n}{mpmath}\PY{o}{.}\PY{n}{fac}\PY{p}{(}\PY{l+m+mi}{1900}\PY{p}{)}
\PY{n+nb}{print}\PY{p}{(}\PY{n}{P\PYZus{}1e2}\PY{p}{)}

\PY{n}{P\PYZus{}1e6} \PY{o}{=} \PY{n}{hypergeom}\PY{o}{.}\PY{n}{pmf}\PY{p}{(}\PY{l+m+mi}{100}\PY{p}{,} \PY{l+m+mf}{1e6}\PY{p}{,} \PY{l+m+mi}{1000}\PY{p}{,} \PY{l+m+mi}{1000}\PY{p}{)}
\PY{n+nb}{print}\PY{p}{(}\PY{n}{P\PYZus{}1e6}\PY{p}{)}
\end{Verbatim}
\end{tcolorbox}

    \begin{Verbatim}[commandchars=\\\{\}]
5.35146651961926e-430
1.6996195526413255e-163
    \end{Verbatim}

    Обозначим вероятность события \(B_{100}(N)\) через \(P(N)\) и построим
её зависимость от \(N\).

    \begin{tcolorbox}[breakable, size=fbox, boxrule=1pt, pad at break*=1mm,colback=cellbackground, colframe=cellborder]
\prompt{In}{incolor}{4}{\boxspacing}
\begin{Verbatim}[commandchars=\\\{\}]
\PY{k}{def} \PY{n+nf}{P}\PY{p}{(}\PY{n}{x}\PY{p}{)}\PY{p}{:}
    \PY{k}{return} \PY{n}{hypergeom}\PY{o}{.}\PY{n}{pmf}\PY{p}{(}\PY{l+m+mi}{100}\PY{p}{,} \PY{n}{x}\PY{p}{,} \PY{l+m+mi}{1000}\PY{p}{,} \PY{l+m+mi}{1000}\PY{p}{)}

\PY{c+c1}{\PYZsh{} Generate data}
\PY{n}{X} \PY{o}{=} \PY{n}{np}\PY{o}{.}\PY{n}{arange}\PY{p}{(}\PY{l+m+mi}{5000}\PY{p}{,} \PY{l+m+mi}{28000}\PY{p}{,} \PY{l+m+mi}{100}\PY{p}{)}
\PY{n}{Y} \PY{o}{=} \PY{n}{P}\PY{p}{(}\PY{n}{X}\PY{p}{)}
\end{Verbatim}
\end{tcolorbox}

    \begin{tcolorbox}[breakable, size=fbox, boxrule=1pt, pad at break*=1mm,colback=cellbackground, colframe=cellborder]
\prompt{In}{incolor}{6}{\boxspacing}
\begin{Verbatim}[commandchars=\\\{\}]
\PY{c+c1}{\PYZsh{} Show data}
\PY{n}{plt}\PY{o}{.}\PY{n}{figure}\PY{p}{(}\PY{n}{figsize}\PY{o}{=}\PY{p}{(}\PY{l+m+mi}{8}\PY{p}{,} \PY{l+m+mi}{6}\PY{p}{)}\PY{p}{)}
\PY{n}{plt}\PY{o}{.}\PY{n}{plot}\PY{p}{(}\PY{n}{X}\PY{p}{,} \PY{n}{Y}\PY{p}{,} \PY{l+s+s1}{\PYZsq{}}\PY{l+s+s1}{\PYZhy{}}\PY{l+s+s1}{\PYZsq{}}\PY{p}{)}
\PY{n}{plt}\PY{o}{.}\PY{n}{yscale}\PY{p}{(}\PY{l+s+s1}{\PYZsq{}}\PY{l+s+s1}{log}\PY{l+s+s1}{\PYZsq{}}\PY{p}{)}
\PY{n}{plt}\PY{o}{.}\PY{n}{xlabel}\PY{p}{(}\PY{l+s+s1}{\PYZsq{}}\PY{l+s+s1}{\PYZdl{}N\PYZdl{}}\PY{l+s+s1}{\PYZsq{}}\PY{p}{)}
\PY{n}{plt}\PY{o}{.}\PY{n}{ylabel}\PY{p}{(}\PY{l+s+s1}{\PYZsq{}}\PY{l+s+s1}{\PYZdl{}P(N)\PYZdl{}}\PY{l+s+s1}{\PYZsq{}}\PY{p}{,} \PY{n}{rotation}\PY{o}{=}\PY{l+m+mi}{0}\PY{p}{,} \PY{n}{ha}\PY{o}{=}\PY{l+s+s1}{\PYZsq{}}\PY{l+s+s1}{right}\PY{l+s+s1}{\PYZsq{}}\PY{p}{)}
\PY{n}{plt}\PY{o}{.}\PY{n}{show}\PY{p}{(}\PY{p}{)}
\end{Verbatim}
\end{tcolorbox}

    \begin{center}
    \adjustimage{max size={0.7\linewidth}{0.7\paperheight}}{output_26_0.png}
    \end{center}
    { \hspace*{\fill} \\}
    
    \begin{tcolorbox}[breakable, size=fbox, boxrule=1pt, pad at break*=1mm,colback=cellbackground, colframe=cellborder]
\prompt{In}{incolor}{7}{\boxspacing}
\begin{Verbatim}[commandchars=\\\{\}]
\PY{c+c1}{\PYZsh{} Find maximum likelihood estimation}
\PY{n}{mle\PYZus{}idx} \PY{o}{=} \PY{n}{np}\PY{o}{.}\PY{n}{argmax}\PY{p}{(}\PY{n}{Y}\PY{p}{)}
\PY{n}{x\PYZus{}mle} \PY{o}{=} \PY{n}{X}\PY{p}{[}\PY{n}{mle\PYZus{}idx}\PY{p}{]}
\PY{n}{y\PYZus{}mle} \PY{o}{=} \PY{n}{Y}\PY{p}{[}\PY{n}{mle\PYZus{}idx}\PY{p}{]}

\PY{n+nb}{print}\PY{p}{(}\PY{n}{x\PYZus{}mle}\PY{p}{,} \PY{n}{y\PYZus{}mle}\PY{p}{)}
\end{Verbatim}
\end{tcolorbox}

    \begin{Verbatim}[commandchars=\\\{\}]
10000 0.044289171719937244
    \end{Verbatim}

    В нашем примере оценкой максимального правдоподобия для количества рыб в
озере является число \(\hat{N} = 10^4\), а вероятность соответствующего
события \(\mathrm{P}(B_{100}(10^4)) \approx 0.044\).

    \begin{center}\rule{0.5\linewidth}{\linethickness}\end{center}

    \hypertarget{ux443ux441ux43bux43eux432ux43dux430ux44f-ux432ux435ux440ux43eux44fux442ux43dux43eux441ux442ux44c.-ux43dux435ux437ux430ux432ux438ux441ux438ux43cux43eux441ux442ux44c}{%
\section{Условная вероятность.
Независимость}\label{ux443ux441ux43bux43eux432ux43dux430ux44f-ux432ux435ux440ux43eux44fux442ux43dux43eux441ux442ux44c.-ux43dux435ux437ux430ux432ux438ux441ux438ux43cux43eux441ux442ux44c}}

\hypertarget{ux443ux441ux43bux43eux432ux43dux430ux44f-ux432ux435ux440ux43eux44fux442ux43dux43eux441ux442ux44c}{%
\subsection{Условная
вероятность}\label{ux443ux441ux43bux43eux432ux43dux430ux44f-ux432ux435ux440ux43eux44fux442ux43dux43eux441ux442ux44c}}

Игральная кость подбрасывается один раз. Известно, что выпало более трёх
очков. Какова при этом вероятность того, что выпало нечётное число
очков?

Пусть событие \(A = \{4, 5, 6\}\) означает, что выпало более трёх очков,
событие \(B = \{1, 3, 5\}\) --- выпало нечётное число очков. Как
понимать вероятность события \(B\), если известно, что \(A\) случилось?
Знаем, что произошло событие \(A\), но всё равно не знаем, что именно
выпало на кости. Однако теперь возможностей осталось только три : могло
выпасть 4, 5 или 6 очков. Событию \(B\) из этих равновозможных исходов
благоприятен единственный исход: выпадение пяти очков. Поэтому искомая
вероятность равна \(1/3\).

Итак, при вычислении условной вероятности события \(B\) при случившемся
событии \(A\) мы ищем долю исходов, благоприятствующих \(B\), среди всех
исходов события \(A\). Эту условную вероятность будем обозначать
\(\mathrm{P}(B|A)\).

Дадим теперь определение условной вероятности, согласующееся с
интуитивными представлениями о ней.

\textbf{Определение.} Условной вероятностью события \(B\) при условии
события \(A\) с \(\mathrm{P}(A)>0\) называется величина
\[ \mathrm{P}(B|A) = \dfrac{\mathrm{P}(AB)}{\mathrm{P}(A)}. \]

    \hypertarget{ux441ux432ux43eux439ux441ux442ux432ux430-ux443ux441ux43bux43eux432ux43dux43eux439-ux432ux435ux440ux43eux44fux442ux43dux43eux441ux442ux438}{%
\paragraph{Свойства условной
вероятности}\label{ux441ux432ux43eux439ux441ux442ux432ux430-ux443ux441ux43bux43eux432ux43dux43eux439-ux432ux435ux440ux43eux44fux442ux43dux43eux441ux442ux438}}

Все общие теоремы о вероятностях верны также и для условных
вероятностей, если эти условные вероятности берутся при одном и том же
условии \(A\).

\begin{enumerate}
\def\labelenumi{\arabic{enumi}.}
\tightlist
\item
  \(\mathrm{P}(A|A) = 1\),
\item
  \(\mathrm{P}(\emptyset|A) = 0\),
\item
  \(\mathrm{P}(B|A) = 1\), \(B \supseteq A\),
\item
  \(\mathrm{P}(B_1 + B_2|A) = \mathrm{P}(B_1|A) + \mathrm{P}(B_2|A) - \mathrm{P}(B_1 B_2|A)\).
\end{enumerate}

\begin{quote}
\textbf{Пример.} Рассмотрим семьи, имеющие двух детей. Спрашивается,
какова вероятность того, что в семье оба ребенка мальчики, в
предположении, что: 1. старший ребенок --- мальчик (\emph{1/2}); 2. по
крайней мере один из детей --- мальчик (\emph{1/3})?
\end{quote}

    \hypertarget{ux444ux43eux440ux43cux443ux43bux430-ux43fux43eux43bux43dux43eux439-ux432ux435ux440ux43eux44fux442ux43dux43eux441ux442ux438}{%
\subsection{Формула полной
вероятности}\label{ux444ux43eux440ux43cux443ux43bux430-ux43fux43eux43bux43dux43eux439-ux432ux435ux440ux43eux44fux442ux43dux43eux441ux442ux438}}

Рассмотрим \emph{полную группу несовместимых событий}
\(\mathcal{D} = \{A_1, \dots, A_n\}\). Имеет место \textbf{формула
полной вероятности}
\[ \mathrm{P}(B) = \sum_{i=1}^n \mathrm{P}(B|A_i) \mathrm{P}(A_i). \]

\begin{quote}
\textbf{Пример.} В урне имеется \(M\) шаров, среди которых \(m\)
«счастливых». Спрашивается, какова вероятность извлечь на втором шаге
«счастливый» шар (предполагается, что качество первого извлеченного шара
неизвестно).
\end{quote}

Справедлива \textbf{формула умножения вероятностей}:
\[ \mathrm{P}(AB) = \mathrm{P}(B|A) \mathrm{P}(A). \]

Формула умножения вероятностей обобщается по индукции:
\[ \mathrm{P}(A_1 \dots A_n) = \mathrm{P}(A_1) \mathrm{P}(A_2|A_1) \dots \mathrm{P}(A_n|A_1 \dots A_{n-1}). \]

    \hypertarget{ux43dux435ux437ux430ux432ux438ux441ux438ux43cux43eux441ux442ux44c}{%
\paragraph{Независимость}\label{ux43dux435ux437ux430ux432ux438ux441ux438ux43cux43eux441ux442ux44c}}

\textbf{Определение.} События \(A\) и \(B\) называются
\emph{независимыми} или \emph{статистически независимыми} (относительно
вероятности \(\mathrm{P}\)), если
\[ \mathrm{P}(AB) = \mathrm{P}(A) \cdot \mathrm{P}(B). \]

    \hypertarget{ux442ux435ux43eux440ux435ux43cux430-ux431ux430ux439ux435ux441ux430}{%
\subsection{Теорема
Байеса}\label{ux442ux435ux43eux440ux435ux43cux430-ux431ux430ux439ux435ux441ux430}}

    Из формул \(\mathrm{P}(B|A) = \dfrac{\mathrm{P}(AB)}{\mathrm{P}(A)}\) и
\(\mathrm{P}(A|B) = \dfrac{\mathrm{P}(AB)}{\mathrm{P}(B)}\) получаем
\textbf{формулу Байеса}:
\[ \mathrm{P}(A|B) = \dfrac{\mathrm{P}(A) \mathrm{P}(B|A)}{\mathrm{P}(B)}. \]

Если события \(A_1, \dots, A_n\) образуют разбиение \(\Omega\), то из
формул полной вероятности и Байеса следует \textbf{теорема Байеса}:
\[ \mathrm{P}(A_i|B) = \frac{\mathrm{P}(A_i) \mathrm{P}(B|A_i)}{\sum_{j=1}^{n} \mathrm{P}(A_j) \mathrm{P}(B|A_j)}. \]

    В статистических применениях события \(H_1, \dots, H_n\) образующие
«полную группу событий» (\(H_1 + \dots + H_n = \Omega\)), часто называют
«гипотезами», а \(\mathrm{P}(H_i)\) --- \emph{априорной} вероятностью
гипотезы \(H_i\). Условная вероятность \(\mathrm{P}(H_i|D)\) трактуется
как \emph{апостериорная} вероятность гипотезы \(H_i\) после наступления
события (наблюдения данных) \(D\).

Формула Байеса даёт возможность получить ответ на вопрос о том, как
изменится постериорная вероятность \(\mathrm{P}(H|D)\) гипотезы \(H\)
при данных наблюдения \(D\) (т.е. оценка, полученная после изучения
новых экспериментальных данных), при условии, что известны априорная
вероятность этой гипотезы \(\mathrm{P}(H)\) (т.е. оценка, полученная без
учета данных \(D\)), вероятность получения данных \(\mathrm{P}(D)\) и
условная вероятность \(\mathrm{P}(D|H)\) получить данные \(D\), если
гипотеза \(H\) выполнена.\\
В этом случае величину \(\mathrm{P}(D|H)\) называют
\emph{правдоподобием} гипотезы \(H\) с учётом данных \(D\).

    \hypertarget{ux43fux440ux438ux43cux435ux440ux44b}{%
\subsection{Примеры}\label{ux43fux440ux438ux43cux435ux440ux44b}}

\textbf{Пример 1.}

У вас в кармане три монеты: на одной два орла, на другой две решки,
третья обычная. Вы наугад вытаскиваете монету, подбрасываете ее, и
выпадает орёл. Какова вероятность, что на другой стороне монеты тоже
орёл?

Многие бы решили, что ответ --- \(1/2\), поскольку монета либо обычная,
либо с двумя орлами, и вероятность выбрать одну из них одинакова.
Существует много способов это проверить, но проще всего использовать
идею с ожидаемыми количествами.

Подумаем, чего можно ожидать, если проделать такой эксперимент шесть
раз. В среднем каждая монета будет выбрана дважды, и каждая из сторон
выпадет по разу. Орёл выпадает в трех случаях, причем в двух на второй
стороне также будет орёл. Поэтому вероятность того, что на второй
стороне монеты тоже орёл, равна \(2/3\), а не \(1/2\). По сути,
выпадение орла повышает вероятность выбора монеты с двумя орлами, ведь у
такой монеты есть два варианта упасть орлом вверх, а у симметричной ---
только один.

    \textbf{Пример 2.}

Используемый в эксперименте детектор сложных молекул правильно различает
молекулы \(A\) и \(B\) в \(90\,\%\) случаев, т.е.
\(\mathrm{P}(a|A) = 0.9\), \(\mathrm{P}(b|B) = 0.9\), где \(a\) и \(b\)
означают результат тестирования для соответствующей молекулы.
Следовательно, вероятности ошибочных результатов равны
\(\mathrm{P}(b|A) = \mathrm{P}(a|B) = 0.1\). Эти цифры характеризуют
аппаратную надежность тестов. Предположим, что молекулы типов \(A\) и
\(B\) в потоке частиц встречаются с вероятностями
\(\mathrm{P}(A) = 0.95\) и \(\mathrm{P}(B) = 0.05\) соответственно.
Какова вероятность \(\mathrm{P}(B|b)\) того, что молекула действительно
имеет тип \(B\) при условии, что об этом свидетельствует результат
тестирования \(b\)?

\begin{align*}
  \mathrm{P}(B|b) = \mathrm{P}(B) \dfrac{\mathrm{P}(b|B)}{\mathrm{P}(b)} 
  &= \mathrm{P}(B) \dfrac{\mathrm{P}(b|B)}{\mathrm{P}(B)\mathrm{P}(b|B) + \mathrm{P}(A)\mathrm{P}(b|A)} \\
  &= 0.05\dfrac{0.9}{0.05 \cdot 0.9 + 0.95 \cdot 0.1} = 0.321.
\end{align*}

    \begin{tcolorbox}[breakable, size=fbox, boxrule=1pt, pad at break*=1mm,colback=cellbackground, colframe=cellborder]
\prompt{In}{incolor}{8}{\boxspacing}
\begin{Verbatim}[commandchars=\\\{\}]
\PY{n}{p1} \PY{o}{=} \PY{l+m+mf}{0.05}\PY{o}{*}\PY{l+m+mf}{0.9}\PY{o}{/}\PY{p}{(}\PY{l+m+mf}{0.05}\PY{o}{*}\PY{l+m+mf}{0.9} \PY{o}{+} \PY{l+m+mf}{0.95}\PY{o}{*}\PY{l+m+mf}{0.1}\PY{p}{)}
\PY{n+nb}{print}\PY{p}{(}\PY{n}{p1}\PY{p}{)} 
\end{Verbatim}
\end{tcolorbox}

    \begin{Verbatim}[commandchars=\\\{\}]
0.32142857142857145
    \end{Verbatim}

    \textbf{Пример 3.}

В задаче из примера 2 после первого детектора поставили такой же
второй.\\
Пусть оба детектора показывают результат «b»: \(b_1\) и \(b_2\). Как при
этом изменится вероятность ложноположительного результата? Вероятность
ошибки должна уменьшиться, но насколько именно?

Применим формулу Байеса для события \(\{b_1 \cap b_2\}\):
\[ \mathrm{P}(B|b_1 b_2) = \mathrm{P}(B) \dfrac{\mathrm{P}(b_1 b_2|B)}{\mathrm{P}(b_1 b_2)}. \]

Обратим внимание, что события \(b_1\) и \(b_2\), не являясь независимыми
сами по себе, являются независимыми для каждой из частиц по отдельности:
\(\mathrm{P}(b_1 b_2) \ne \mathrm{P}(b_1) \mathrm{P}(b_2)\), но
\(\mathrm{P}(b_1 b_2|B) = \mathrm{P}(b_1|B) \mathrm{P}(b_2|B)\).
Поэтому
\begin{align*}
  \mathrm{P}(B|b_1 b_2) 
  &= \mathrm{P}(B) \dfrac{\mathrm{P}(b_1|B) \mathrm{P}(b_2|B)}{\mathrm{P}(B)\mathrm{P}(b_1|B)\mathrm{P}(b_2|B) + \mathrm{P}(A)\mathrm{P}(b_1|A)\mathrm{P}(b_2|A)} \\
  &= 0.05 \dfrac{0.9 \cdot 0.9}{0.05 \cdot 0.9 \cdot 0.9 + 0.95 \cdot 0.1 \cdot 0.1} = 0.81.
\end{align*}



    \begin{tcolorbox}[breakable, size=fbox, boxrule=1pt, pad at break*=1mm,colback=cellbackground, colframe=cellborder]
\prompt{In}{incolor}{9}{\boxspacing}
\begin{Verbatim}[commandchars=\\\{\}]
\PY{n}{p2} \PY{o}{=} \PY{l+m+mf}{0.05}\PY{o}{*}\PY{l+m+mf}{0.9}\PY{o}{*}\PY{l+m+mf}{0.9}\PY{o}{/}\PY{p}{(}\PY{l+m+mf}{0.05}\PY{o}{*}\PY{l+m+mf}{0.9}\PY{o}{*}\PY{l+m+mf}{0.9} \PY{o}{+} \PY{l+m+mf}{0.95}\PY{o}{*}\PY{l+m+mf}{0.1}\PY{o}{*}\PY{l+m+mf}{0.1}\PY{p}{)}
\PY{n+nb}{print}\PY{p}{(}\PY{n}{p2}\PY{p}{)} 
\end{Verbatim}
\end{tcolorbox}

    \begin{Verbatim}[commandchars=\\\{\}]
0.81
    \end{Verbatim}

    \hypertarget{ux43eux442ux43dux43eux448ux435ux43dux438ux435-ux448ux430ux43dux441ux43eux432-ux438-ux43eux442ux43dux43eux448ux435ux43dux438ux435-ux43fux440ux430ux432ux434ux43eux43fux43eux434ux43eux431ux438ux44f}{%
\subsection{Отношение шансов и отношение
правдоподобия}\label{ux43eux442ux43dux43eux448ux435ux43dux438ux435-ux448ux430ux43dux441ux43eux432-ux438-ux43eux442ux43dux43eux448ux435ux43dux438ux435-ux43fux440ux430ux432ux434ux43eux43fux43eux434ux43eux431ux438ux44f}}

Рассмотрим две конкурирующие гипотезы \(H_1\) и \(H_2\).\\
\emph{Отношение правдоподобия} --- это вероятность некоторого факта
\(D\) при условии гипотезы \(H_1\), делённая на вероятность того же
факта при условии гипотезы \(H_2\):
\(LR = \dfrac{\mathrm{P}(D|H_1)}{\mathrm{P}(D|H_2)}\).

Отношение правдоподобия фактически сравнивает относительную поддержку,
предоставляемую неким фактом (свидетельством, доказательством) для двух
конкурирующих гипотез, например, означающих виновность и невиновность.

\emph{Шансы} на какое-то событие --- это вероятность того, что оно
произойдет, делённая на вероятность того, что оно не произойдет.
Например, если мы бросаем игральную кость, то шансы на выпадение
шестерки --- 1 к 5.

    Вычислим шансы в эксперименте из \textbf{примера 2}: \[
  \dfrac{\mathrm{P}(B|b)}{\mathrm{P}(A|b)} = \dfrac{\frac{\mathrm{P}(B)\mathrm{P}(b|B)}{\mathrm{P}(b)}} {\frac{\mathrm{P}(A)\mathrm{P}(b|A)}{\mathrm{P}(b)}} = \frac{\mathrm{P}(B)}{\mathrm{P}(A)} \cdot \dfrac{\mathrm{P}(b|B)}{\mathrm{P}(b|A)}. \tag{3}\label{eq:chances} 
\]

Множитель \(\mathrm{P}(b)\) удачно сократился и формула
\eqref{eq:chances} в итоге выражает апостериорные шансы гипотезы через её начальные шансы и отношение правдоподобия.

\[
  c_1 = \dfrac{\mathrm{P}(B|b)}{\mathrm{P}(A|b)} = \frac{0.05}{0.95} \cdot \dfrac{0.9}{0.1} = 0.05263 \cdot 9 = 0.474, \\
  \mathrm{P}(B|b) = \dfrac{c_1}{1+c_1} = 0.321.
\]

    \begin{tcolorbox}[breakable, size=fbox, boxrule=1pt, pad at break*=1mm,colback=cellbackground, colframe=cellborder]
\prompt{In}{incolor}{10}{\boxspacing}
\begin{Verbatim}[commandchars=\\\{\}]
\PY{n}{c1} \PY{o}{=} \PY{l+m+mf}{0.05}\PY{o}{/}\PY{l+m+mf}{0.95} \PY{o}{*} \PY{l+m+mf}{0.9}\PY{o}{/}\PY{l+m+mf}{0.1}
\PY{n+nb}{print}\PY{p}{(}\PY{n}{c1}\PY{p}{)}
\PY{n+nb}{print}\PY{p}{(}\PY{n}{c1}\PY{o}{/}\PY{p}{(}\PY{l+m+mi}{1}\PY{o}{+}\PY{n}{c1}\PY{p}{)}\PY{p}{)}
\end{Verbatim}
\end{tcolorbox}

    \begin{Verbatim}[commandchars=\\\{\}]
0.47368421052631576
0.32142857142857145
    \end{Verbatim}

    Продолжим с экспериментом из \textbf{примера 3}.

Формулу \eqref{eq:chances} можно применить ещё раз, тогда апостериорные шансы после первого факта станут априорными перед учётом второго факта (при условии его независимости).

В итоге получаем красивый результат: при объединении всех этапов процесс
эквивалентен \emph{умножению независимых отношений правдоподобия} и
формированию общего составного отношения правдоподобия. А апостериорные
шансы равны произведению априорных шансов на общее отношение
правдоподобия всех фактов.

\[
  c_2 = \dfrac{\mathrm{P}(B|b_1 b_2)}{\mathrm{P}(A|b_1 b_2)} = \dfrac{\mathrm{P}(B)}{\mathrm{P}(A)} \cdot \dfrac{\mathrm{P}(b_1|B)}{\mathrm{P}(b_1|A)} \cdot \dfrac{\mathrm{P}(b_2|B)}{\mathrm{P}(b_2|A)}
  = \dfrac{0.05}{0.95} \cdot \frac{0.9}{0.1} \cdot \dfrac{0.9}{0.1} = 0.05263 \cdot 81 = 4.263.
\]

В итоге искомая условная вероятность равна
\[ \mathrm{P}(B|b_1 b_2) = \dfrac{c_2}{1+c_2} = 0.81. \]

    \begin{tcolorbox}[breakable, size=fbox, boxrule=1pt, pad at break*=1mm,colback=cellbackground, colframe=cellborder]
\prompt{In}{incolor}{11}{\boxspacing}
\begin{Verbatim}[commandchars=\\\{\}]
\PY{n}{c2} \PY{o}{=} \PY{n}{c1} \PY{o}{*} \PY{l+m+mf}{0.9}\PY{o}{/}\PY{l+m+mf}{0.1}
\PY{n+nb}{print}\PY{p}{(}\PY{n}{c2}\PY{p}{)}
\PY{n+nb}{print}\PY{p}{(}\PY{n}{c2}\PY{o}{/}\PY{p}{(}\PY{l+m+mi}{1}\PY{o}{+}\PY{n}{c2}\PY{p}{)}\PY{p}{)}
\end{Verbatim}
\end{tcolorbox}

    \begin{Verbatim}[commandchars=\\\{\}]
4.263157894736842
0.8099999999999999
    \end{Verbatim}

    \begin{center}\rule{0.5\linewidth}{\linethickness}\end{center}

    \hypertarget{ux438ux441ux442ux43eux447ux43dux438ux43aux438}{%
\section{Источники}\label{ux438ux441ux442ux43eux447ux43dux438ux43aux438}}

\begin{enumerate}
\def\labelenumi{\arabic{enumi}.}
\tightlist
\item
  \emph{Ширяев А.Н.} Вероятность --- 1. --- М.: МЦНМО, 2007. --- 517 с.
\item
  \emph{Чернова Н. И.} Теория вероятностей. Учебное пособие. ---
  Новосибирск, 2007. --- 160 с.
\item
  \emph{Феллер В.} Введение в теорию вероятностей и её приложения. ---
  М.: Мир, 1964. --- 498 с.
\item
  \emph{Шпигельхалтер Д.} Искусство статистики. Как находить ответы в
  данных. --- М.: Манн, Иванов и Фербер, 2021. --- 448 с.
\end{enumerate}


    % Add a bibliography block to the postdoc
    
    
    
\end{document}
