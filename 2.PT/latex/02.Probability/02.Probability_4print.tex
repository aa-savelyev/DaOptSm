\documentclass[11pt,a4paper]{article}

    \usepackage[breakable]{tcolorbox}
    \usepackage{parskip} % Stop auto-indenting (to mimic markdown behaviour)
    
    \usepackage{iftex}
    \ifPDFTeX
      \usepackage[T2A]{fontenc}
      \usepackage{mathpazo}
      \usepackage[russian,english]{babel}
    \else
      \usepackage{fontspec}
      \usepackage{polyglossia}
      \setmainlanguage[babelshorthands=true]{russian}    % Язык по-умолчанию русский с поддержкой приятных команд пакета babel
      \setotherlanguage{english}                         % Дополнительный язык = английский (в американской вариации по-умолчанию)

      \defaultfontfeatures{Ligatures=TeX}
      \setmainfont[BoldFont={STIX Two Text SemiBold}]%
      {STIX Two Text}                                    % Шрифт с засечками
      \newfontfamily\cyrillicfont[BoldFont={STIX Two Text SemiBold}]%
      {STIX Two Text}                                    % Шрифт с засечками для кириллицы
      \setsansfont{Fira Sans}                            % Шрифт без засечек
      \newfontfamily\cyrillicfontsf{Fira Sans}           % Шрифт без засечек для кириллицы
      \setmonofont[Scale=0.87,BoldFont={Fira Mono Medium},ItalicFont=[FiraMono-Oblique]]%
      {Fira Mono}%                                       % Моноширинный шрифт
      \newfontfamily\cyrillicfonttt[Scale=0.87,BoldFont={Fira Mono Medium},ItalicFont=[FiraMono-Oblique]]%
      {Fira Mono}                                        % Моноширинный шрифт для кириллицы

      %%% Математические пакеты %%%
      \usepackage{amsthm,amsmath,amscd}   % Математические дополнения от AMS
      \usepackage{amsfonts,amssymb}       % Математические дополнения от AMS
      \usepackage{mathtools}              % Добавляет окружение multlined
      \usepackage{unicode-math}           % Для шрифта STIX Two Math
      \setmathfont{STIX Two Math}         % Математический шрифт
    \fi

    % Basic figure setup, for now with no caption control since it's done
    % automatically by Pandoc (which extracts ![](path) syntax from Markdown).
    \usepackage{graphicx}
    % Maintain compatibility with old templates. Remove in nbconvert 6.0
    \let\Oldincludegraphics\includegraphics
    % Ensure that by default, figures have no caption (until we provide a
    % proper Figure object with a Caption API and a way to capture that
    % in the conversion process - todo).
    \usepackage{caption}
    \DeclareCaptionFormat{nocaption}{}
    \captionsetup{format=nocaption,aboveskip=0pt,belowskip=0pt}

    \usepackage{float}
    \floatplacement{figure}{H} % forces figures to be placed at the correct location
    \usepackage{xcolor} % Allow colors to be defined
    \usepackage{enumerate} % Needed for markdown enumerations to work
    \usepackage{geometry} % Used to adjust the document margins
    \usepackage{amsmath} % Equations
    \usepackage{amssymb} % Equations
    \usepackage{textcomp} % defines textquotesingle
    % Hack from http://tex.stackexchange.com/a/47451/13684:
    \AtBeginDocument{%
        \def\PYZsq{\textquotesingle}% Upright quotes in Pygmentized code
    }
    \usepackage{upquote} % Upright quotes for verbatim code
    \usepackage{eurosym} % defines \euro
    \usepackage[mathletters]{ucs} % Extended unicode (utf-8) support
    \usepackage{fancyvrb} % verbatim replacement that allows latex
    \usepackage{grffile} % extends the file name processing of package graphics 
                         % to support a larger range
    \makeatletter % fix for old versions of grffile with XeLaTeX
    \@ifpackagelater{grffile}{2019/11/01}
    {
      % Do nothing on new versions
    }
    {
      \def\Gread@@xetex#1{%
        \IfFileExists{"\Gin@base".bb}%
        {\Gread@eps{\Gin@base.bb}}%
        {\Gread@@xetex@aux#1}%
      }
    }
    \makeatother
    \usepackage[Export]{adjustbox} % Used to constrain images to a maximum size
    \adjustboxset{max size={0.9\linewidth}{0.9\paperheight}}

    % The hyperref package gives us a pdf with properly built
    % internal navigation ('pdf bookmarks' for the table of contents,
    % internal cross-reference links, web links for URLs, etc.)
    \usepackage{hyperref}
    % The default LaTeX title has an obnoxious amount of whitespace. By default,
    % titling removes some of it. It also provides customization options.
    \usepackage{titling}
    \usepackage{longtable} % longtable support required by pandoc >1.10
    \usepackage{booktabs}  % table support for pandoc > 1.12.2
    \usepackage[inline]{enumitem} % IRkernel/repr support (it uses the enumerate* environment)
    \usepackage[normalem]{ulem} % ulem is needed to support strikethroughs (\sout)
                                % normalem makes italics be italics, not underlines
    \usepackage{mathrsfs}
    

    
    % Colors for the hyperref package
    \definecolor{urlcolor}{rgb}{0,.145,.698}
    \definecolor{linkcolor}{rgb}{.71,0.21,0.01}
    \definecolor{citecolor}{rgb}{.12,.54,.11}

    % ANSI colors
    \definecolor{ansi-black}{HTML}{3E424D}
    \definecolor{ansi-black-intense}{HTML}{282C36}
    \definecolor{ansi-red}{HTML}{E75C58}
    \definecolor{ansi-red-intense}{HTML}{B22B31}
    \definecolor{ansi-green}{HTML}{00A250}
    \definecolor{ansi-green-intense}{HTML}{007427}
    \definecolor{ansi-yellow}{HTML}{DDB62B}
    \definecolor{ansi-yellow-intense}{HTML}{B27D12}
    \definecolor{ansi-blue}{HTML}{208FFB}
    \definecolor{ansi-blue-intense}{HTML}{0065CA}
    \definecolor{ansi-magenta}{HTML}{D160C4}
    \definecolor{ansi-magenta-intense}{HTML}{A03196}
    \definecolor{ansi-cyan}{HTML}{60C6C8}
    \definecolor{ansi-cyan-intense}{HTML}{258F8F}
    \definecolor{ansi-white}{HTML}{C5C1B4}
    \definecolor{ansi-white-intense}{HTML}{A1A6B2}
    \definecolor{ansi-default-inverse-fg}{HTML}{FFFFFF}
    \definecolor{ansi-default-inverse-bg}{HTML}{000000}

    % common color for the border for error outputs.
    \definecolor{outerrorbackground}{HTML}{FFDFDF}

    % commands and environments needed by pandoc snippets
    % extracted from the output of `pandoc -s`
    \providecommand{\tightlist}{%
      \setlength{\itemsep}{0pt}\setlength{\parskip}{0pt}}
    \DefineVerbatimEnvironment{Highlighting}{Verbatim}{commandchars=\\\{\}}
    % Add ',fontsize=\small' for more characters per line
    \newenvironment{Shaded}{}{}
    \newcommand{\KeywordTok}[1]{\textcolor[rgb]{0.00,0.44,0.13}{\textbf{{#1}}}}
    \newcommand{\DataTypeTok}[1]{\textcolor[rgb]{0.56,0.13,0.00}{{#1}}}
    \newcommand{\DecValTok}[1]{\textcolor[rgb]{0.25,0.63,0.44}{{#1}}}
    \newcommand{\BaseNTok}[1]{\textcolor[rgb]{0.25,0.63,0.44}{{#1}}}
    \newcommand{\FloatTok}[1]{\textcolor[rgb]{0.25,0.63,0.44}{{#1}}}
    \newcommand{\CharTok}[1]{\textcolor[rgb]{0.25,0.44,0.63}{{#1}}}
    \newcommand{\StringTok}[1]{\textcolor[rgb]{0.25,0.44,0.63}{{#1}}}
    \newcommand{\CommentTok}[1]{\textcolor[rgb]{0.38,0.63,0.69}{\textit{{#1}}}}
    \newcommand{\OtherTok}[1]{\textcolor[rgb]{0.00,0.44,0.13}{{#1}}}
    \newcommand{\AlertTok}[1]{\textcolor[rgb]{1.00,0.00,0.00}{\textbf{{#1}}}}
    \newcommand{\FunctionTok}[1]{\textcolor[rgb]{0.02,0.16,0.49}{{#1}}}
    \newcommand{\RegionMarkerTok}[1]{{#1}}
    \newcommand{\ErrorTok}[1]{\textcolor[rgb]{1.00,0.00,0.00}{\textbf{{#1}}}}
    \newcommand{\NormalTok}[1]{{#1}}
    
    % Additional commands for more recent versions of Pandoc
    \newcommand{\ConstantTok}[1]{\textcolor[rgb]{0.53,0.00,0.00}{{#1}}}
    \newcommand{\SpecialCharTok}[1]{\textcolor[rgb]{0.25,0.44,0.63}{{#1}}}
    \newcommand{\VerbatimStringTok}[1]{\textcolor[rgb]{0.25,0.44,0.63}{{#1}}}
    \newcommand{\SpecialStringTok}[1]{\textcolor[rgb]{0.73,0.40,0.53}{{#1}}}
    \newcommand{\ImportTok}[1]{{#1}}
    \newcommand{\DocumentationTok}[1]{\textcolor[rgb]{0.73,0.13,0.13}{\textit{{#1}}}}
    \newcommand{\AnnotationTok}[1]{\textcolor[rgb]{0.38,0.63,0.69}{\textbf{\textit{{#1}}}}}
    \newcommand{\CommentVarTok}[1]{\textcolor[rgb]{0.38,0.63,0.69}{\textbf{\textit{{#1}}}}}
    \newcommand{\VariableTok}[1]{\textcolor[rgb]{0.10,0.09,0.49}{{#1}}}
    \newcommand{\ControlFlowTok}[1]{\textcolor[rgb]{0.00,0.44,0.13}{\textbf{{#1}}}}
    \newcommand{\OperatorTok}[1]{\textcolor[rgb]{0.40,0.40,0.40}{{#1}}}
    \newcommand{\BuiltInTok}[1]{{#1}}
    \newcommand{\ExtensionTok}[1]{{#1}}
    \newcommand{\PreprocessorTok}[1]{\textcolor[rgb]{0.74,0.48,0.00}{{#1}}}
    \newcommand{\AttributeTok}[1]{\textcolor[rgb]{0.49,0.56,0.16}{{#1}}}
    \newcommand{\InformationTok}[1]{\textcolor[rgb]{0.38,0.63,0.69}{\textbf{\textit{{#1}}}}}
    \newcommand{\WarningTok}[1]{\textcolor[rgb]{0.38,0.63,0.69}{\textbf{\textit{{#1}}}}}
    
    
    % Define a nice break command that doesn't care if a line doesn't already
    % exist.
    \def\br{\hspace*{\fill} \\* }
    % Math Jax compatibility definitions
    \def\gt{>}
    \def\lt{<}
    \let\Oldtex\TeX
    \let\Oldlatex\LaTeX
    \renewcommand{\TeX}{\textrm{\Oldtex}}
    \renewcommand{\LaTeX}{\textrm{\Oldlatex}}
    % Document parameters
    % Document title
    \title{
      {\Large Лекция 2} \\
      Аксиоматика теории вероятностей
    }
    % \date{16 февраля 2022\,г.}
    \date{}
    
    
    
% Pygments definitions
\makeatletter
\def\PY@reset{\let\PY@it=\relax \let\PY@bf=\relax%
    \let\PY@ul=\relax \let\PY@tc=\relax%
    \let\PY@bc=\relax \let\PY@ff=\relax}
\def\PY@tok#1{\csname PY@tok@#1\endcsname}
\def\PY@toks#1+{\ifx\relax#1\empty\else%
    \PY@tok{#1}\expandafter\PY@toks\fi}
\def\PY@do#1{\PY@bc{\PY@tc{\PY@ul{%
    \PY@it{\PY@bf{\PY@ff{#1}}}}}}}
\def\PY#1#2{\PY@reset\PY@toks#1+\relax+\PY@do{#2}}

\@namedef{PY@tok@w}{\def\PY@tc##1{\textcolor[rgb]{0.73,0.73,0.73}{##1}}}
\@namedef{PY@tok@c}{\let\PY@it=\textit\def\PY@tc##1{\textcolor[rgb]{0.24,0.48,0.48}{##1}}}
\@namedef{PY@tok@cp}{\def\PY@tc##1{\textcolor[rgb]{0.61,0.40,0.00}{##1}}}
\@namedef{PY@tok@k}{\let\PY@bf=\textbf\def\PY@tc##1{\textcolor[rgb]{0.00,0.50,0.00}{##1}}}
\@namedef{PY@tok@kp}{\def\PY@tc##1{\textcolor[rgb]{0.00,0.50,0.00}{##1}}}
\@namedef{PY@tok@kt}{\def\PY@tc##1{\textcolor[rgb]{0.69,0.00,0.25}{##1}}}
\@namedef{PY@tok@o}{\def\PY@tc##1{\textcolor[rgb]{0.40,0.40,0.40}{##1}}}
\@namedef{PY@tok@ow}{\let\PY@bf=\textbf\def\PY@tc##1{\textcolor[rgb]{0.67,0.13,1.00}{##1}}}
\@namedef{PY@tok@nb}{\def\PY@tc##1{\textcolor[rgb]{0.00,0.50,0.00}{##1}}}
\@namedef{PY@tok@nf}{\def\PY@tc##1{\textcolor[rgb]{0.00,0.00,1.00}{##1}}}
\@namedef{PY@tok@nc}{\let\PY@bf=\textbf\def\PY@tc##1{\textcolor[rgb]{0.00,0.00,1.00}{##1}}}
\@namedef{PY@tok@nn}{\let\PY@bf=\textbf\def\PY@tc##1{\textcolor[rgb]{0.00,0.00,1.00}{##1}}}
\@namedef{PY@tok@ne}{\let\PY@bf=\textbf\def\PY@tc##1{\textcolor[rgb]{0.80,0.25,0.22}{##1}}}
\@namedef{PY@tok@nv}{\def\PY@tc##1{\textcolor[rgb]{0.10,0.09,0.49}{##1}}}
\@namedef{PY@tok@no}{\def\PY@tc##1{\textcolor[rgb]{0.53,0.00,0.00}{##1}}}
\@namedef{PY@tok@nl}{\def\PY@tc##1{\textcolor[rgb]{0.46,0.46,0.00}{##1}}}
\@namedef{PY@tok@ni}{\let\PY@bf=\textbf\def\PY@tc##1{\textcolor[rgb]{0.44,0.44,0.44}{##1}}}
\@namedef{PY@tok@na}{\def\PY@tc##1{\textcolor[rgb]{0.41,0.47,0.13}{##1}}}
\@namedef{PY@tok@nt}{\let\PY@bf=\textbf\def\PY@tc##1{\textcolor[rgb]{0.00,0.50,0.00}{##1}}}
\@namedef{PY@tok@nd}{\def\PY@tc##1{\textcolor[rgb]{0.67,0.13,1.00}{##1}}}
\@namedef{PY@tok@s}{\def\PY@tc##1{\textcolor[rgb]{0.73,0.13,0.13}{##1}}}
\@namedef{PY@tok@sd}{\let\PY@it=\textit\def\PY@tc##1{\textcolor[rgb]{0.73,0.13,0.13}{##1}}}
\@namedef{PY@tok@si}{\let\PY@bf=\textbf\def\PY@tc##1{\textcolor[rgb]{0.64,0.35,0.47}{##1}}}
\@namedef{PY@tok@se}{\let\PY@bf=\textbf\def\PY@tc##1{\textcolor[rgb]{0.67,0.36,0.12}{##1}}}
\@namedef{PY@tok@sr}{\def\PY@tc##1{\textcolor[rgb]{0.64,0.35,0.47}{##1}}}
\@namedef{PY@tok@ss}{\def\PY@tc##1{\textcolor[rgb]{0.10,0.09,0.49}{##1}}}
\@namedef{PY@tok@sx}{\def\PY@tc##1{\textcolor[rgb]{0.00,0.50,0.00}{##1}}}
\@namedef{PY@tok@m}{\def\PY@tc##1{\textcolor[rgb]{0.40,0.40,0.40}{##1}}}
\@namedef{PY@tok@gh}{\let\PY@bf=\textbf\def\PY@tc##1{\textcolor[rgb]{0.00,0.00,0.50}{##1}}}
\@namedef{PY@tok@gu}{\let\PY@bf=\textbf\def\PY@tc##1{\textcolor[rgb]{0.50,0.00,0.50}{##1}}}
\@namedef{PY@tok@gd}{\def\PY@tc##1{\textcolor[rgb]{0.63,0.00,0.00}{##1}}}
\@namedef{PY@tok@gi}{\def\PY@tc##1{\textcolor[rgb]{0.00,0.52,0.00}{##1}}}
\@namedef{PY@tok@gr}{\def\PY@tc##1{\textcolor[rgb]{0.89,0.00,0.00}{##1}}}
\@namedef{PY@tok@ge}{\let\PY@it=\textit}
\@namedef{PY@tok@gs}{\let\PY@bf=\textbf}
\@namedef{PY@tok@gp}{\let\PY@bf=\textbf\def\PY@tc##1{\textcolor[rgb]{0.00,0.00,0.50}{##1}}}
\@namedef{PY@tok@go}{\def\PY@tc##1{\textcolor[rgb]{0.44,0.44,0.44}{##1}}}
\@namedef{PY@tok@gt}{\def\PY@tc##1{\textcolor[rgb]{0.00,0.27,0.87}{##1}}}
\@namedef{PY@tok@err}{\def\PY@bc##1{{\setlength{\fboxsep}{\string -\fboxrule}\fcolorbox[rgb]{1.00,0.00,0.00}{1,1,1}{\strut ##1}}}}
\@namedef{PY@tok@kc}{\let\PY@bf=\textbf\def\PY@tc##1{\textcolor[rgb]{0.00,0.50,0.00}{##1}}}
\@namedef{PY@tok@kd}{\let\PY@bf=\textbf\def\PY@tc##1{\textcolor[rgb]{0.00,0.50,0.00}{##1}}}
\@namedef{PY@tok@kn}{\let\PY@bf=\textbf\def\PY@tc##1{\textcolor[rgb]{0.00,0.50,0.00}{##1}}}
\@namedef{PY@tok@kr}{\let\PY@bf=\textbf\def\PY@tc##1{\textcolor[rgb]{0.00,0.50,0.00}{##1}}}
\@namedef{PY@tok@bp}{\def\PY@tc##1{\textcolor[rgb]{0.00,0.50,0.00}{##1}}}
\@namedef{PY@tok@fm}{\def\PY@tc##1{\textcolor[rgb]{0.00,0.00,1.00}{##1}}}
\@namedef{PY@tok@vc}{\def\PY@tc##1{\textcolor[rgb]{0.10,0.09,0.49}{##1}}}
\@namedef{PY@tok@vg}{\def\PY@tc##1{\textcolor[rgb]{0.10,0.09,0.49}{##1}}}
\@namedef{PY@tok@vi}{\def\PY@tc##1{\textcolor[rgb]{0.10,0.09,0.49}{##1}}}
\@namedef{PY@tok@vm}{\def\PY@tc##1{\textcolor[rgb]{0.10,0.09,0.49}{##1}}}
\@namedef{PY@tok@sa}{\def\PY@tc##1{\textcolor[rgb]{0.73,0.13,0.13}{##1}}}
\@namedef{PY@tok@sb}{\def\PY@tc##1{\textcolor[rgb]{0.73,0.13,0.13}{##1}}}
\@namedef{PY@tok@sc}{\def\PY@tc##1{\textcolor[rgb]{0.73,0.13,0.13}{##1}}}
\@namedef{PY@tok@dl}{\def\PY@tc##1{\textcolor[rgb]{0.73,0.13,0.13}{##1}}}
\@namedef{PY@tok@s2}{\def\PY@tc##1{\textcolor[rgb]{0.73,0.13,0.13}{##1}}}
\@namedef{PY@tok@sh}{\def\PY@tc##1{\textcolor[rgb]{0.73,0.13,0.13}{##1}}}
\@namedef{PY@tok@s1}{\def\PY@tc##1{\textcolor[rgb]{0.73,0.13,0.13}{##1}}}
\@namedef{PY@tok@mb}{\def\PY@tc##1{\textcolor[rgb]{0.40,0.40,0.40}{##1}}}
\@namedef{PY@tok@mf}{\def\PY@tc##1{\textcolor[rgb]{0.40,0.40,0.40}{##1}}}
\@namedef{PY@tok@mh}{\def\PY@tc##1{\textcolor[rgb]{0.40,0.40,0.40}{##1}}}
\@namedef{PY@tok@mi}{\def\PY@tc##1{\textcolor[rgb]{0.40,0.40,0.40}{##1}}}
\@namedef{PY@tok@il}{\def\PY@tc##1{\textcolor[rgb]{0.40,0.40,0.40}{##1}}}
\@namedef{PY@tok@mo}{\def\PY@tc##1{\textcolor[rgb]{0.40,0.40,0.40}{##1}}}
\@namedef{PY@tok@ch}{\let\PY@it=\textit\def\PY@tc##1{\textcolor[rgb]{0.24,0.48,0.48}{##1}}}
\@namedef{PY@tok@cm}{\let\PY@it=\textit\def\PY@tc##1{\textcolor[rgb]{0.24,0.48,0.48}{##1}}}
\@namedef{PY@tok@cpf}{\let\PY@it=\textit\def\PY@tc##1{\textcolor[rgb]{0.24,0.48,0.48}{##1}}}
\@namedef{PY@tok@c1}{\let\PY@it=\textit\def\PY@tc##1{\textcolor[rgb]{0.24,0.48,0.48}{##1}}}
\@namedef{PY@tok@cs}{\let\PY@it=\textit\def\PY@tc##1{\textcolor[rgb]{0.24,0.48,0.48}{##1}}}

\def\PYZbs{\char`\\}
\def\PYZus{\char`\_}
\def\PYZob{\char`\{}
\def\PYZcb{\char`\}}
\def\PYZca{\char`\^}
\def\PYZam{\char`\&}
\def\PYZlt{\char`\<}
\def\PYZgt{\char`\>}
\def\PYZsh{\char`\#}
\def\PYZpc{\char`\%}
\def\PYZdl{\char`\$}
\def\PYZhy{\char`\-}
\def\PYZsq{\char`\'}
\def\PYZdq{\char`\"}
\def\PYZti{\char`\~}
% for compatibility with earlier versions
\def\PYZat{@}
\def\PYZlb{[}
\def\PYZrb{]}
\makeatother


    % For linebreaks inside Verbatim environment from package fancyvrb. 
    \makeatletter
        \newbox\Wrappedcontinuationbox 
        \newbox\Wrappedvisiblespacebox 
        \newcommand*\Wrappedvisiblespace {\textcolor{red}{\textvisiblespace}} 
        \newcommand*\Wrappedcontinuationsymbol {\textcolor{red}{\llap{\tiny$\m@th\hookrightarrow$}}} 
        \newcommand*\Wrappedcontinuationindent {3ex } 
        \newcommand*\Wrappedafterbreak {\kern\Wrappedcontinuationindent\copy\Wrappedcontinuationbox} 
        % Take advantage of the already applied Pygments mark-up to insert 
        % potential linebreaks for TeX processing. 
        %        {, <, #, %, $, ' and ": go to next line. 
        %        _, }, ^, &, >, - and ~: stay at end of broken line. 
        % Use of \textquotesingle for straight quote. 
        \newcommand*\Wrappedbreaksatspecials {% 
            \def\PYGZus{\discretionary{\char`\_}{\Wrappedafterbreak}{\char`\_}}% 
            \def\PYGZob{\discretionary{}{\Wrappedafterbreak\char`\{}{\char`\{}}% 
            \def\PYGZcb{\discretionary{\char`\}}{\Wrappedafterbreak}{\char`\}}}% 
            \def\PYGZca{\discretionary{\char`\^}{\Wrappedafterbreak}{\char`\^}}% 
            \def\PYGZam{\discretionary{\char`\&}{\Wrappedafterbreak}{\char`\&}}% 
            \def\PYGZlt{\discretionary{}{\Wrappedafterbreak\char`\<}{\char`\<}}% 
            \def\PYGZgt{\discretionary{\char`\>}{\Wrappedafterbreak}{\char`\>}}% 
            \def\PYGZsh{\discretionary{}{\Wrappedafterbreak\char`\#}{\char`\#}}% 
            \def\PYGZpc{\discretionary{}{\Wrappedafterbreak\char`\%}{\char`\%}}% 
            \def\PYGZdl{\discretionary{}{\Wrappedafterbreak\char`\$}{\char`\$}}% 
            \def\PYGZhy{\discretionary{\char`\-}{\Wrappedafterbreak}{\char`\-}}% 
            \def\PYGZsq{\discretionary{}{\Wrappedafterbreak\textquotesingle}{\textquotesingle}}% 
            \def\PYGZdq{\discretionary{}{\Wrappedafterbreak\char`\"}{\char`\"}}% 
            \def\PYGZti{\discretionary{\char`\~}{\Wrappedafterbreak}{\char`\~}}% 
        } 
        % Some characters . , ; ? ! / are not pygmentized. 
        % This macro makes them "active" and they will insert potential linebreaks 
        \newcommand*\Wrappedbreaksatpunct {% 
            \lccode`\~`\.\lowercase{\def~}{\discretionary{\hbox{\char`\.}}{\Wrappedafterbreak}{\hbox{\char`\.}}}% 
            \lccode`\~`\,\lowercase{\def~}{\discretionary{\hbox{\char`\,}}{\Wrappedafterbreak}{\hbox{\char`\,}}}% 
            \lccode`\~`\;\lowercase{\def~}{\discretionary{\hbox{\char`\;}}{\Wrappedafterbreak}{\hbox{\char`\;}}}% 
            \lccode`\~`\:\lowercase{\def~}{\discretionary{\hbox{\char`\:}}{\Wrappedafterbreak}{\hbox{\char`\:}}}% 
            \lccode`\~`\?\lowercase{\def~}{\discretionary{\hbox{\char`\?}}{\Wrappedafterbreak}{\hbox{\char`\?}}}% 
            \lccode`\~`\!\lowercase{\def~}{\discretionary{\hbox{\char`\!}}{\Wrappedafterbreak}{\hbox{\char`\!}}}% 
            \lccode`\~`\/\lowercase{\def~}{\discretionary{\hbox{\char`\/}}{\Wrappedafterbreak}{\hbox{\char`\/}}}% 
            \catcode`\.\active
            \catcode`\,\active 
            \catcode`\;\active
            \catcode`\:\active
            \catcode`\?\active
            \catcode`\!\active
            \catcode`\/\active 
            \lccode`\~`\~ 	
        }
    \makeatother

    \let\OriginalVerbatim=\Verbatim
    \makeatletter
    \renewcommand{\Verbatim}[1][1]{%
        %\parskip\z@skip
        \sbox\Wrappedcontinuationbox {\Wrappedcontinuationsymbol}%
        \sbox\Wrappedvisiblespacebox {\FV@SetupFont\Wrappedvisiblespace}%
        \def\FancyVerbFormatLine ##1{\hsize\linewidth
            \vtop{\raggedright\hyphenpenalty\z@\exhyphenpenalty\z@
                \doublehyphendemerits\z@\finalhyphendemerits\z@
                \strut ##1\strut}%
        }%
        % If the linebreak is at a space, the latter will be displayed as visible
        % space at end of first line, and a continuation symbol starts next line.
        % Stretch/shrink are however usually zero for typewriter font.
        \def\FV@Space {%
            \nobreak\hskip\z@ plus\fontdimen3\font minus\fontdimen4\font
            \discretionary{\copy\Wrappedvisiblespacebox}{\Wrappedafterbreak}
            {\kern\fontdimen2\font}%
        }%
        
        % Allow breaks at special characters using \PYG... macros.
        \Wrappedbreaksatspecials
        % Breaks at punctuation characters . , ; ? ! and / need catcode=\active 	
        \OriginalVerbatim[#1,codes*=\Wrappedbreaksatpunct]%
    }
    \makeatother

    % Exact colors from NB
    \definecolor{incolor}{HTML}{303F9F}
    \definecolor{outcolor}{HTML}{D84315}
    \definecolor{cellborder}{HTML}{CFCFCF}
    \definecolor{cellbackground}{HTML}{F7F7F7}
    
    % prompt
    \makeatletter
    \newcommand{\boxspacing}{\kern\kvtcb@left@rule\kern\kvtcb@boxsep}
    \makeatother
    \newcommand{\prompt}[4]{
        {\ttfamily\llap{{\color{#2}[#3]:\hspace{3pt}#4}}\vspace{-\baselineskip}}
    }
    

    
    % Prevent overflowing lines due to hard-to-break entities
    \sloppy 
    % Setup hyperref package
    \hypersetup{
      breaklinks=true,  % so long urls are correctly broken across lines
      colorlinks=true,
      urlcolor=urlcolor,
      linkcolor=linkcolor,
      citecolor=citecolor,
      }
    % Slightly bigger margins than the latex defaults
    
    \geometry{verbose,tmargin=1in,bmargin=1in,lmargin=1in,rmargin=1in}
    
    

\begin{document}
    
  \maketitle
%  \thispagestyle{empty}
%  \tableofcontents
%  \pagebreak


%    \begin{tcolorbox}[breakable, size=fbox, boxrule=1pt, pad at break*=1mm,colback=cellbackground, colframe=cellborder]
%\prompt{In}{incolor}{1}{\boxspacing}
%\begin{Verbatim}[commandchars=\\\{\}]
%\PY{c+c1}{\PYZsh{} Imports}
%\PY{k+kn}{import} \PY{n+nn}{numpy} \PY{k}{as} \PY{n+nn}{np}
%\PY{k+kn}{import} \PY{n+nn}{matplotlib}\PY{n+nn}{.}\PY{n+nn}{pyplot} \PY{k}{as} \PY{n+nn}{plt}
%
%\PY{k+kn}{import} \PY{n+nn}{warnings}
%\PY{n}{warnings}\PY{o}{.}\PY{n}{filterwarnings}\PY{p}{(}\PY{l+s+s1}{\PYZsq{}}\PY{l+s+s1}{ignore}\PY{l+s+s1}{\PYZsq{}}\PY{p}{)}
%\end{Verbatim}
%\end{tcolorbox}
%
%    \begin{tcolorbox}[breakable, size=fbox, boxrule=1pt, pad at break*=1mm,colback=cellbackground, colframe=cellborder]
%\prompt{In}{incolor}{2}{\boxspacing}
%\begin{Verbatim}[commandchars=\\\{\}]
%\PY{c+c1}{\PYZsh{} Styles}
%\PY{k+kn}{import} \PY{n+nn}{matplotlib}
%\PY{n}{matplotlib}\PY{o}{.}\PY{n}{rcParams}\PY{p}{[}\PY{l+s+s1}{\PYZsq{}}\PY{l+s+s1}{font.size}\PY{l+s+s1}{\PYZsq{}}\PY{p}{]} \PY{o}{=} \PY{l+m+mi}{14}
%\PY{n}{matplotlib}\PY{o}{.}\PY{n}{rcParams}\PY{p}{[}\PY{l+s+s1}{\PYZsq{}}\PY{l+s+s1}{lines.linewidth}\PY{l+s+s1}{\PYZsq{}}\PY{p}{]} \PY{o}{=} \PY{l+m+mf}{1.5}
%\PY{n}{matplotlib}\PY{o}{.}\PY{n}{rcParams}\PY{p}{[}\PY{l+s+s1}{\PYZsq{}}\PY{l+s+s1}{lines.markersize}\PY{l+s+s1}{\PYZsq{}}\PY{p}{]} \PY{o}{=} \PY{l+m+mi}{4}
%\PY{n}{cm} \PY{o}{=} \PY{n}{matplotlib}\PY{o}{.}\PY{n}{pyplot}\PY{o}{.}\PY{n}{cm}\PY{o}{.}\PY{n}{tab10}  \PY{c+c1}{\PYZsh{} Colormap}
%
%\PY{k+kn}{import} \PY{n+nn}{seaborn}
%\PY{n}{seaborn}\PY{o}{.}\PY{n}{set\PYZus{}style}\PY{p}{(}\PY{l+s+s1}{\PYZsq{}}\PY{l+s+s1}{whitegrid}\PY{l+s+s1}{\PYZsq{}}\PY{p}{)}
%\end{Verbatim}
%\end{tcolorbox}
%
%    \begin{tcolorbox}[breakable, size=fbox, boxrule=1pt, pad at break*=1mm,colback=cellbackground, colframe=cellborder]
%\prompt{In}{incolor}{3}{\boxspacing}
%\begin{Verbatim}[commandchars=\\\{\}]
%\PY{c+c1}{\PYZsh{} \PYZpc{}config InlineBackend.figure\PYZus{}formats = [\PYZsq{}pdf\PYZsq{}]}
%\PY{c+c1}{\PYZsh{} \PYZpc{}config Completer.use\PYZus{}jedi = False}
%\end{Verbatim}
%\end{tcolorbox}
%
%    \begin{center}\rule{0.5\linewidth}{0.5pt}\end{center}

    \hypertarget{ux43fux440ux435ux434ux43cux435ux442-ux442ux435ux43eux440ux438ux438-ux432ux435ux440ux43eux44fux442ux43dux43eux441ux442ux435ux439}{%
\section{Предмет теории
вероятностей}\label{ux43fux440ux435ux434ux43cux435ux442-ux442ux435ux43eux440ux438ux438-ux432ux435ux440ux43eux44fux442ux43dux43eux441ux442ux435ux439}}

Предметом теории вероятностей является математический анализ случайных
явлений --- эмпирических феноменов, которые (при заданном «комплексе
условий») могут быть охарактеризованы тем, что

\begin{itemize}
\tightlist
\item
  для них отсутствует \emph{детерминистическая регулярность} (наблюдения
  над ними не всегда приводят к одним и тем же исходам)
\end{itemize}

и в то же самое время

\begin{itemize}
\tightlist
\item
  они обладают некоторой \emph{статистической регулярностью}
  (проявляющейся в статистической устойчивости частот).
\end{itemize}

Поясним сказанное на классическом примере «честного» подбрасывания
«правильной» монеты. Ясно, что заранее невозможно с определённостью
предсказать исход каждого подбрасывания. Результаты отдельных
экспериментов носят крайне нерегулярный характер (то «герб», то
«решетка»), и кажется, что это лишает нас возможности познать какие-либо
закономерности, связанные с этими экспериментами. Однако, если провести
большое число «независимых» подбрасываний, то можно заметить, что для
«правильной» монеты будет наблюдаться вполне определенная статистической
регулярность, проявляющаяся в том, что частота выпадания «герба» будет
«близка» к \(1/2\).

    \begin{center}\rule{0.5\linewidth}{0.5pt}\end{center}

    \hypertarget{ux432ux435ux440ux43eux44fux442ux43dux43eux441ux442ux43dux430ux44f-ux43cux43eux434ux435ux43bux44c}{%
\section{Вероятностная
модель}\label{ux432ux435ux440ux43eux44fux442ux43dux43eux441ux442ux43dux430ux44f-ux43cux43eux434ux435ux43bux44c}}

Согласно аксиоматике Колмогорова первоначальным объектом теории
вероятностей является \emph{вероятностное пространство}
\((\Omega, \mathcal{F}, \mathrm{P})\). Здесь \(\Omega\) --- это
множество, состоящее из элементарных событий \(\omega\), с выделенной на
нём системой его подмножеств (событий) \(\mathcal{F}\), образующих
\(\sigma\)-алгебру, а \(\mathrm{P}\) --- вероятностная мера
(вероятность), определённая на множествах из \(\mathcal{F}\).

    \hypertarget{ux43fux440ux43eux441ux442ux440ux430ux43dux441ux442ux432ux43e-ux44dux43bux435ux43cux435ux43dux442ux430ux440ux43dux44bux445-ux438ux441ux445ux43eux434ux43eux432}{%
\subsection{Пространство элементарных
исходов}\label{ux43fux440ux43eux441ux442ux440ux430ux43dux441ux442ux432ux43e-ux44dux43bux435ux43cux435ux43dux442ux430ux440ux43dux44bux445-ux438ux441ux445ux43eux434ux43eux432}}

\textbf{Определение}. \emph{Пространством элементарных исходов}
называется множество \(\Omega\), содержащее все возможные
\emph{взаимоисключающие} результаты данного случайного эксперимента.
Элементы множества \(\Omega\) называются элементарными исходами и
обозначаются буквой \(\omega\).

Элементарный исход --- это мельчайший неделимый результат
эксперимента.\\
Выделение пространства элементарных исходов представляет собой первый
шаг в формулировании понятия \emph{вероятностной модели} (вероятностной
«теории») того или иного эксперимента.

    \textbf{Примеры:}

\begin{itemize}
\tightlist
\item
  однократное подбрасывание монеты (пространство исходов состоит из двух
  точек: Г --- «герб», Р --- «решетка»);
\item
  n-кратное подбрасывание монеты;
\item
  выбор шаров с возвращением (упорядоченные и неупорядоченные выборки);
\item
  выбор шаров без возвращения (упорядоченные и неупорядоченные выборки).
\end{itemize}

    \hypertarget{ux430ux43bux433ux435ux431ux440ux430-ux438-mathbfsigma-ux430ux43bux433ux435ux431ux440ux430-ux441ux43eux431ux44bux442ux438ux439}{%
\subsection{\texorpdfstring{Алгебра и \(\sigma\)-алгебра
событий}{Алгебра и \{\textbackslash sigma\}-алгебра событий}}\label{ux430ux43bux433ux435ux431ux440ux430-ux438-mathbfsigma-ux430ux43bux433ux435ux431ux440ux430-ux441ux43eux431ux44bux442ux438ux439}}

Наряду с понятием пространства элементарных исходов введём теперь важное
понятие события, лежащее в основе построения всякой вероятностной модели
рассматриваемого эксперимента.
Мы собираемся определить набор подмножеств множества \(\Omega\), которые
будут называться событиями, и затем задать вероятность как функцию,
определённую \emph{только} на множестве событий.

\emph{Событиями} мы будем называть не любые подмножества \(\Omega\), а
лишь элементы некоторого выделенного набора подмножеств множества
\(\Omega\). В качестве наборов событий целесообразно рассматривать
системы множеств, являющиеся \emph{алгебрами}. Для этого необходимо
позаботиться, чтобы этот набор подмножеств был \emph{замкнут}
относительно операций объединения, пересечения и дополнения.

\textbf{Определение.} Множество \(\mathcal{F}\), элементами которого
являются подмножества множества \(\Omega\) называется
\(\sigma\)-\emph{алгеброй}, если оно удовлетворяет следующим аксиомам:

\begin{enumerate}
\def\labelenumi{\arabic{enumi}.}
\tightlist
\item
  \(\Omega \in \mathcal{F}\) (\(\sigma\)-алгебра содержит
  \emph{достоверное событие});
\item
  если \(A \in \mathcal{F}\), то \(\overline{A} \in \mathcal{F}\)
  (вместе с любым множеством \(\sigma\)-алгебра содержит противоположное
  к нему);
\item
  если \(A_1,\,A_2,\,\ldots \in \mathcal{F}\), то
  \(\bigcup\limits_{i=1}^{\infty}A_i \in \mathcal{F}\) (вместе с любым
  \emph{счётным} набором событий \(\sigma\)-алгебра содержит их
  объединение).
\end{enumerate}

    \textbf{Свойства:}

\begin{enumerate}
\def\labelenumi{\arabic{enumi}.}
\tightlist
\item
  Из аксиом 1 и 2 следует, что пустое множество
  \(\emptyset = \overline{\Omega}\) также содержится в \(\mathcal{F}\),
  т. е. алгебра содержит и \emph{невозможное событие}.
\item
  Из аксиом 2 и 3 следует, что вместе с любым счётным набором событий
  \(\sigma\)-алгебра содержит не только их объединение
  \(\bigcup\limits_{i=1}^{\infty}A_i\), но и их пересечение
  \(\bigcap\limits_{i=1}^{\infty}A_i\).
\end{enumerate}

\textbf{Примеры:}

\begin{enumerate}
\def\labelenumi{\arabic{enumi}.}
\tightlist
\item
  \(\mathcal{F} = \{ \Omega, \emptyset \}\) --- система, состоящая из
  \(\Omega\) и пустого множества (так называемая тривиальная алгебра);
\item
  \(\mathcal{F} = \{ A, \overline{A}, \Omega, \emptyset \}\) ---
  система, порождённая событием \(A\);
\item
  \(\mathcal{F} = \{ A: A \subseteq \Omega \}\) --- совокупность всех
  подмножеств \(\Omega\) (обозначается \(2^\Omega\)).
\end{enumerate}

\begin{quote}
\emph{Упражнение.} Доказать, что если \(\Omega\) состоит из \(n\)
элементов, то в множестве всех его подмножеств ровно \(2^n\) элементов.
\end{quote}

    \hypertarget{ux432ux435ux440ux43eux44fux442ux43dux43eux441ux442ux43dux430ux44f-ux43cux435ux440ux430}{%
\subsection{Вероятностная
мера}\label{ux432ux435ux440ux43eux44fux442ux43dux43eux441ux442ux43dux430ux44f-ux43cux435ux440ux430}}

Пока мы сделали два первых шага к построению вероятностной модели
эксперимента: выделили пространство исходов \(\Omega\) и некоторую
систему \(\mathcal{F}\) его подмножеств, образующих \(\sigma\)-алгебру и
называемых событиями. Сделаем теперь следующий шаг, а именно введём
вероятностную меру.

\textbf{Определение.} Пусть \(\Omega\) --- непустое множество, а
\(\mathcal{F}\) --- \(\sigma\)-алгебра его подмножеств. Функция
\(\mathrm{P}: \mathcal{F} \rightarrow \mathbb{R}\) называется
\emph{вероятностной мерой}, если она удовлетворяет следующим аксиомам:

\begin{enumerate}
\def\labelenumi{\arabic{enumi}.}
\tightlist
\item
  \(\mathrm{P}(A) \ge 0\), \(A \in \mathcal{F}\) (неотрицательность);
\item
  \(\mathrm{P}\left( \bigcup\limits_{i=1}^{\infty}A_i \right) = \sum\limits_{i=1}^{\infty}\mathrm{P}(A_i)\),
  где \(A_i \in \mathcal{F}\), \(A_i \cap A_j = \emptyset\), \(i \ne j\)
  (счётная или \(\sigma\)-аддитивность)\\
  (для любого счётного набора попарно несовместных событий мера их
  объединения равна сумме их мер);
\item
  \(\mathrm{P}(\Omega) = 1\) (нормированность).
\end{enumerate}

\textbf{Замечание 1.} Аксиомы 1 и 2 задают \emph{меру} как
неотрицательную \(\sigma\)-аддитивную функцию множеств, аксиома 3
определяет вероятность как \emph{нормированную меру}.

\textbf{Замечание 2.} Существуют примеры неизмеримых множеств, например,
\emph{множество Витали}.

    \textbf{Свойства:}

\begin{enumerate}
\def\labelenumi{\arabic{enumi}.}
\tightlist
\item
  \(\mathrm{P}: \mathcal{F} \rightarrow [0,1]\)
\item
  \(\mathrm{P}(\emptyset) = 0\);
\item
  \(\mathrm{P}(\overline{A}) = 1 - \mathrm{P}(A)\);
\item
  \(\mathrm{P}(A \cup B) = \mathrm{P}(A) + \mathrm{P}(B) - \mathrm{P}(A \cap B)\);
\item
  \(\mathrm{P}\left( \bigcup\limits_{i=1}^{n}A_i \right) = \sum\limits_{i}\mathrm{P}(A_i) - \sum\limits_{i<j}\mathrm{P}(A_i \cap A_j) + \sum\limits_{i<j<k}\mathrm{P}(A_i \cap A_j \cap A_k) + \ldots + (-1)^{n-1} \mathrm{P}\left( \bigcap\limits_{i=1}^{n}A_i \right)\)
  --- формула включения-исключения.
\end{enumerate}

    \hypertarget{ux432ux435ux440ux43eux44fux442ux43dux43eux441ux442ux43dux43eux435-ux43fux440ux43eux441ux442ux440ux430ux43dux441ux442ux432ux43e}{%
\subsection{Вероятностное
пространство}\label{ux432ux435ux440ux43eux44fux442ux43dux43eux441ux442ux43dux43eux435-ux43fux440ux43eux441ux442ux440ux430ux43dux441ux442ux432ux43e}}

\textbf{Определение.} Тройка объектов

\[ \left( \Omega, \mathcal{F}, \mathrm{P} \right), \]

где \(\Omega\) --- множество элементарных исходов, \(\mathcal{F}\) ---
\(\sigma\)-алгебра его подмножеств и \(\mathrm{P}\) --- вероятностная
мера на \(\mathcal{F}\), называется \emph{вероятностным пространством}.

    \hypertarget{ux437ux430ux43cux435ux447ux430ux43dux438ux44f}{%
\subsection{Замечания}\label{ux437ux430ux43cux435ux447ux430ux43dux438ux44f}}

\hypertarget{ux43fux43eux441ux442ux440ux43eux435ux43dux438ux435-ux432ux435ux440ux43eux44fux442ux43dux43eux441ux442ux43dux43eux433ux43e-ux43fux440ux43eux441ux442ux440ux430ux43dux441ux442ux432ux430}{%
\subsubsection{Построение вероятностного
пространства}\label{ux43fux43eux441ux442ux440ux43eux435ux43dux438ux435-ux432ux435ux440ux43eux44fux442ux43dux43eux441ux442ux43dux43eux433ux43e-ux43fux440ux43eux441ux442ux440ux430ux43dux441ux442ux432ux430}}

При построении вероятностных моделей в конкретных ситуациях выделение
пространства элементарных событий \(\Omega\) и алгебры событий
\(\mathcal{F}\), как правило, не является сложной задачей. При этом в
элементарной теории вероятностей в качестве алгебры \(\mathcal{F}\)
обычно берется алгебра \emph{всех} подмножеств \(\Omega\). Труднее
обстоит дело с вопросом о том, как задавать вероятности элементарных
событий. В сущности, ответ на этот вопрос лежит вне рамок теории
вероятностей, и мы его подробно не рассматриваем, считая, что основной
нашей задачей является не вопрос о том, как приписывать исходам те или
иные вероятности, а \emph{вычисление} вероятностей сложных событий
(событий из \(\mathcal{F}\)) по вероятностям элементарных событий.

С математической точки зрения ясно, что в случае конечного пространства
элементарных событий с помощью приписывания исходам
\(\omega_1, \ldots , \omega_N\) неотрицательных чисел
\(p_1, \ldots , p_N\), удовлетворяющих условию
\(p_1 + \ldots + p_N = 1\), мы получаем все мыслимые (конечные)
вероятностные пространства.

\emph{Правильность} же назначенных для конкретной ситуации значений
\(p_1, \ldots , p_N\) может быть до известной степени проверена с
помощью \emph{закона больших чисел}, согласно которому в длинных сериях
«независимых» экспериментов, происходящих при одинаковых условиях,
частоты появления элементарных событий «близки» к их вероятностям.

    \hypertarget{ux442ux430ux43a-ux447ux442ux43e-ux436ux435-ux442ux430ux43aux43eux435-ux432ux435ux440ux43eux44fux442ux43dux43eux441ux442ux44c}{%
\subsubsection{Так что же такое
вероятность?}\label{ux442ux430ux43a-ux447ux442ux43e-ux436ux435-ux442ux430ux43aux43eux435-ux432ux435ux440ux43eux44fux442ux43dux43eux441ux442ux44c}}

\textbf{Классическое определение вероятности.} Это то, чему нас учат в
школе. Оно основано на симметрии монет, костей, перетасованных колод
карт и так далее и может быть сформулировано как «отношение числа
благоприятных исходов к числу всех исходов, если все исходы
равновозможны». Например, вероятность выпадения единицы на правильной
кости равна 1/6, потому что возможны 6 исходов, а нас устраивает один.
Однако это определение в какой-то степени носит круговой характер,
поскольку прежде мы должны уяснить, что значит равновозможны.

\textbf{«Перечислительная» вероятность.} Предположим, в ящике лежат три
белых и четыре черных носка. Если вытаскивать носок случайным образом,
то чему равна вероятность, что он белый? Ответ 3/7 можно получить путем
простого перечисления всех возможностей. Многие из нас страдали от таких
вопросов в школе, и здесь мы фактически имеем дело с расширением
рассмотренной выше классической идеи, где требуется случайный выбор из
группы физических объектов. Мы уже использовали эту идею при описании
случайного выбора элемента данных из общей генеральной совокупности.

\textbf{Вероятность как частота.} Такое определение говорит о
вероятности как о доле случаев, когда интересующее нас событие наступает
в бесконечной последовательности идентичных экспериментов. Для
бесконечно повторяющихся событий это может быть разумно (хотя бы
теоретически), но как насчет уникальных одноразовых событий, например
скачек или завтрашней погоды? На деле практически любая реальная
ситуация даже в принципе не может быть бесконечно воспроизводимой.

\textbf{Субъективная, или «личная», вероятность.} Это степень веры
конкретного человека в какое-либо событие, основанная на его нынешних
знаниях. Субъективная вероятность означает, что любая численная
вероятность фактически \emph{строится} в соответствии с тем, что
известно в нынешней ситуации, --- и на самом деле вероятность вообще не
«существует» (за исключением, возможно, субатомного уровня). Такой
подход лежит в основе \textbf{байесовской} школы статистики.

    \begin{center}\rule{0.5\linewidth}{0.5pt}\end{center}

    \hypertarget{ux43fux440ux438ux43cux435ux440ux44b}{%
\section{Примеры}\label{ux43fux440ux438ux43cux435ux440ux44b}}

    \hypertarget{ux431ux438ux43dux43eux43cux438ux430ux43bux44cux43dux43eux435-ux440ux430ux441ux43fux440ux435ux434ux435ux43bux435ux43dux438ux435}{%
\subsection{Биномиальное
распределение}\label{ux431ux438ux43dux43eux43cux438ux430ux43bux44cux43dux43eux435-ux440ux430ux441ux43fux440ux435ux434ux435ux43bux435ux43dux438ux435}}

Предположим, что монета подбрасывается \(n\) раз и результат наблюдений
записывается в виде упорядоченного набора \((a_1, \ldots, a_n)\), где
\(a_i = 1\) в случае появления «герба» («успех») и \(a_i = 0\) в случае
появления «решетки» («неуспех»). Пространство всех исходов имеет
следующую структуру:
\[ \Omega= \left\{ \omega: \omega = (a_1, \ldots, a_n), \quad a_i = 0 \; \mathrm{или} \; 1 \right\}. \]

Припишем каждому элементарному событию \(\omega = (a_1, \ldots, a_n)\)
вероятность («вес») \[ p(\omega) = p^{\sum a_i} q^{n-\sum a_i}, \] где
неотрицательные числа \(p\) и \(q\) таковы, что \(p + q = 1\).

Итак, пространство \(\Omega\) вместе с системой \(\mathcal{A}\) всех его
подмножеств и вероятностями
\(\mathrm{P}(A) = \sum\limits_{\omega \in A}p(\omega), \; A \in \mathcal{A}\)
(в частности,
\(\mathrm{P}(\{\omega\}) = p(\omega), \; \omega \in \Omega\)) определяет
некоторую вероятностную модель. Естественно назвать её
\emph{вероятностной моделью, описывающей \(n\)-кратное подбрасывание
монеты}.

Введём в рассмотрение события \[ 
    A_k = \left\{\omega: \omega=(a_1, \ldots, a_n), a_1 + \ldots + a_n = k\right\}, \quad k = 0, 1, \ldots, n,
\] означающие, что произойдет в точности \(k\) «успехов». Тогда
вероятность события \(A_k\) равна
\[ \mathrm{P}(A_k) = C_n^k p^k q^{n-k}, \] причём
\(\sum\limits_{k=0}^n \mathrm{P}(A_k) = 1\).

Набор вероятностей \(\{\mathrm{P}(A_0), \ldots,\mathrm{P}(A_n)\}\)
называется \emph{биномиальным распределением} (числа «успехов» в выборке
объёма \(n\)).

    \hypertarget{ux433ux438ux43fux435ux440ux433ux435ux43eux43cux435ux442ux440ux438ux447ux435ux441ux43aux43eux435-ux440ux430ux441ux43fux440ux435ux434ux435ux43bux435ux43dux438ux435}{%
\subsection{Гипергеометрическое
распределение}\label{ux433ux438ux43fux435ux440ux433ux435ux43eux43cux435ux442ux440ux438ux447ux435ux441ux43aux43eux435-ux440ux430ux441ux43fux440ux435ux434ux435ux43bux435ux43dux438ux435}}

Рассмотрим урну, содержащую \(N\) шаров, из которых \(M\) шаров имеют
белый цвет. Предположим, что осуществляется выбор без возвращения объёма
\(n < N\). Вероятность события \(B_m\), состоящего в том, что \(m\)
шаров из выборки имеют белый цвет равна

\[ \mathrm{P}(B_m) = \dfrac{C_M^m C_{N-M}^{n-m}}{C_N^n}. \]

Набор вероятностей \(\{\mathrm{P}(B_0), \ldots,\mathrm{P}(B_n)\}\) носит
название многомерного гипергеометрического распределения.

    \hypertarget{ux43eux446ux435ux43dux43aux430-ux43cux430ux43aux441ux438ux43cux430ux43bux44cux43dux43eux433ux43e-ux43fux440ux430ux432ux434ux43eux43fux43eux434ux43eux431ux438ux44f}{%
\section{Оценка максимального
правдоподобия}\label{ux43eux446ux435ux43dux43aux430-ux43cux430ux43aux441ux438ux43cux430ux43bux44cux43dux43eux433ux43e-ux43fux440ux430ux432ux434ux43eux43fux43eux434ux43eux431ux438ux44f}}

Пусть \(N\) --- размер некоторой популяции, который требуется оценить
«минимальными средствами» без простого пересчета всех элементов этой
совокупности. Подобного рода вопрос интересен, например, при оценке
числа жителей в той или иной стране, городе и т. д.

В 1786 г. Лаплас для оценки числа \(N\) жителей Франции предложил
следующий метод.

\begin{enumerate}
\def\labelenumi{\arabic{enumi}.}
\tightlist
\item
  Выберем некоторое число \(M\), элементов популяции и пометим их.
\item
  Возвратим их в основную совокупность и предположим, что они «хорошо
  перемешаны» с немаркированными элементами.
\item
  После этого возьмём из «перемешанной» популяции \(n\) элементов.
\item
  Пусть среди них \(m\) элементов оказались маркированными.
\end{enumerate}

Вероятность \(\mathrm{P}(B_m(N))\) задается формулой
гипергеометрического распределения: \[
    \mathrm{P}(B_m(N)) = \frac{C_M^m C_{N-M}^{n-m}}{C_N^n}. \tag{1}\label{eq:prob}
\]

Нам известны числа \(M\), \(n\) и \(m\), а \(N\) (размер популяции) ---
нет, его требуется оценить.

Для каждого частного набора наблюдений \(M\), \(n\) и \(m\) значение
\(N\), при котором вероятность \(\mathrm{P}(B_m(N))\) максимальна,
называется \textbf{оценкой максимального правдоподобия}. Обозначим
наиболее правдоподобное значение через \(\hat{N}\).

Можно показать, что \(\hat{N}\) определяется следующей формулой
(\([\cdot]\) --- целая часть):
\[ \hat{N} = \left[\dfrac{Mn}{m}\right]. \tag{2}\label{eq:max} \]

\begin{quote}
\emph{Задание.} Получить формулу \(\eqref{eq:max}\).\\
Подсказка: можно воспользоваться формулой Стирлинга
\(n! \sim \sqrt{2 \pi n}\left( \dfrac{n}{e} \right)^n\).
\end{quote}

    \hypertarget{ux437ux430ux434ux430ux447ux430-ux43eux431-ux43eux446ux435ux43dux43aux435-ux433ux435ux43dux435ux440ux430ux43bux44cux43dux43eux439-ux441ux43eux432ux43eux43aux443ux43fux43dux43eux441ux442ux438-ux43fux43e-ux432ux44bux431ux43eux440ux43aux435}{%
\subsection{Задача об оценке генеральной совокупности по
выборке}\label{ux437ux430ux434ux430ux447ux430-ux43eux431-ux43eux446ux435ux43dux43aux435-ux433ux435ux43dux435ux440ux430ux43bux44cux43dux43eux439-ux441ux43eux432ux43eux43aux443ux43fux43dux43eux441ux442ux438-ux43fux43e-ux432ux44bux431ux43eux440ux43aux435}}

Применим метод максимального правдоподобия для оценки количества рыб в
озере. Пусть, например, \(M=1000\), \(n=1000\), а \(m=100\). Тогда всё,
что нам достоверно известно о количестве рыб, это
\(N \ge n + M - m = 1900\). Вообще говоря, не исключено, что в озере их
ровно \(1900\). Однако, отправляясь от этой гипотезы, мы придём к
выводу, что случилось событие фантастически малой вероятности.
Действительно, вероятность того, что выборка объёмом \(n=1000\) из
генеральной совокупности объёма \(N=1900\) будет содержать \(m=100\)
маркированных объектов, если общее число маркированных объектов
\(M=1000\), по формуле (\hyperref[mjx-eqn-eqprob]{1}) равна \[
    \mathrm{P}(B_{100}(1900)) = \frac{C_{1000}^{100} C_{900}^{900}}{C_{1900}^{1000}} = \frac{(1000!)^2}{100! \, 1900!} \sim 10^{-430}.
\]

Аналогичное рассуждение заставляет нас откинуть гипотезу о том, что
\(N\) очень велико, скажем равно миллиону
(\(\mathrm{P}(B_{100}(10^6)) \sim 10^{-163}\)).

%    \begin{tcolorbox}[breakable, size=fbox, boxrule=1pt, pad at break*=1mm,colback=cellbackground, colframe=cellborder]
%\prompt{In}{incolor}{4}{\boxspacing}
%\begin{Verbatim}[commandchars=\\\{\}]
%\PY{k+kn}{import} \PY{n+nn}{mpmath}
%\PY{k+kn}{from} \PY{n+nn}{scipy}\PY{n+nn}{.}\PY{n+nn}{stats} \PY{k+kn}{import} \PY{n}{hypergeom}
%
%\PY{n}{P\PYZus{}1e2} \PY{o}{=} \PY{n}{mpmath}\PY{o}{.}\PY{n}{fac}\PY{p}{(}\PY{l+m+mi}{1000}\PY{p}{)}\PY{o}{*}\PY{o}{*}\PY{l+m+mi}{2} \PY{o}{/} \PY{n}{mpmath}\PY{o}{.}\PY{n}{fac}\PY{p}{(}\PY{l+m+mi}{100}\PY{p}{)} \PY{o}{/} \PY{n}{mpmath}\PY{o}{.}\PY{n}{fac}\PY{p}{(}\PY{l+m+mi}{1900}\PY{p}{)}
%\PY{n+nb}{print}\PY{p}{(}\PY{n}{P\PYZus{}1e2}\PY{p}{)}
%
%\PY{n}{P\PYZus{}1e6} \PY{o}{=} \PY{n}{hypergeom}\PY{o}{.}\PY{n}{pmf}\PY{p}{(}\PY{l+m+mi}{100}\PY{p}{,} \PY{l+m+mf}{1e6}\PY{p}{,} \PY{l+m+mi}{1000}\PY{p}{,} \PY{l+m+mi}{1000}\PY{p}{)}
%\PY{n+nb}{print}\PY{p}{(}\PY{n}{P\PYZus{}1e6}\PY{p}{)}
%\end{Verbatim}
%\end{tcolorbox}
%
%    \begin{Verbatim}[commandchars=\\\{\}]
%5.35146651961926e-430
%1.6996195526413255e-163
%    \end{Verbatim}

    Обозначим вероятность события \(B_{100}(N)\) через \(P(N)\) и построим
её зависимость от \(N\).

%    \begin{tcolorbox}[breakable, size=fbox, boxrule=1pt, pad at break*=1mm,colback=cellbackground, colframe=cellborder]
%\prompt{In}{incolor}{5}{\boxspacing}
%\begin{Verbatim}[commandchars=\\\{\}]
%\PY{k}{def} \PY{n+nf}{P}\PY{p}{(}\PY{n}{x}\PY{p}{)}\PY{p}{:}
%    \PY{k}{return} \PY{n}{hypergeom}\PY{o}{.}\PY{n}{pmf}\PY{p}{(}\PY{l+m+mi}{100}\PY{p}{,} \PY{n}{x}\PY{p}{,} \PY{l+m+mi}{1000}\PY{p}{,} \PY{l+m+mi}{1000}\PY{p}{)}
%
%\PY{c+c1}{\PYZsh{} Generate data}
%\PY{n}{X} \PY{o}{=} \PY{n}{np}\PY{o}{.}\PY{n}{arange}\PY{p}{(}\PY{l+m+mi}{5000}\PY{p}{,} \PY{l+m+mi}{28000}\PY{p}{,} \PY{l+m+mi}{100}\PY{p}{)}
%\PY{n}{Y} \PY{o}{=} \PY{n}{P}\PY{p}{(}\PY{n}{X}\PY{p}{)}
%\end{Verbatim}
%\end{tcolorbox}
%
%    \begin{tcolorbox}[breakable, size=fbox, boxrule=1pt, pad at break*=1mm,colback=cellbackground, colframe=cellborder]
%\prompt{In}{incolor}{6}{\boxspacing}
%\begin{Verbatim}[commandchars=\\\{\}]
%\PY{c+c1}{\PYZsh{} Show data}
%\PY{n}{plt}\PY{o}{.}\PY{n}{figure}\PY{p}{(}\PY{n}{figsize}\PY{o}{=}\PY{p}{(}\PY{l+m+mi}{8}\PY{p}{,} \PY{l+m+mi}{6}\PY{p}{)}\PY{p}{)}
%\PY{n}{plt}\PY{o}{.}\PY{n}{plot}\PY{p}{(}\PY{n}{X}\PY{p}{,} \PY{n}{Y}\PY{p}{,} \PY{l+s+s1}{\PYZsq{}}\PY{l+s+s1}{\PYZhy{}}\PY{l+s+s1}{\PYZsq{}}\PY{p}{)}
%\PY{n}{plt}\PY{o}{.}\PY{n}{yscale}\PY{p}{(}\PY{l+s+s1}{\PYZsq{}}\PY{l+s+s1}{log}\PY{l+s+s1}{\PYZsq{}}\PY{p}{)}
%\PY{n}{plt}\PY{o}{.}\PY{n}{xlabel}\PY{p}{(}\PY{l+s+s1}{\PYZsq{}}\PY{l+s+s1}{\PYZdl{}N\PYZdl{}}\PY{l+s+s1}{\PYZsq{}}\PY{p}{)}
%\PY{n}{plt}\PY{o}{.}\PY{n}{ylabel}\PY{p}{(}\PY{l+s+s1}{\PYZsq{}}\PY{l+s+s1}{\PYZdl{}P(N)\PYZdl{}}\PY{l+s+s1}{\PYZsq{}}\PY{p}{,} \PY{n}{rotation}\PY{o}{=}\PY{l+m+mi}{0}\PY{p}{,} \PY{n}{ha}\PY{o}{=}\PY{l+s+s1}{\PYZsq{}}\PY{l+s+s1}{right}\PY{l+s+s1}{\PYZsq{}}\PY{p}{)}
%\PY{n}{plt}\PY{o}{.}\PY{n}{show}\PY{p}{(}\PY{p}{)}
%\end{Verbatim}
%\end{tcolorbox}

    \begin{center}
    \adjustimage{max size={0.7\linewidth}{0.7\paperheight}}{PNvsN.pdf}
    \end{center}
%    { \hspace*{\fill} \\}
    
%    \begin{tcolorbox}[breakable, size=fbox, boxrule=1pt, pad at break*=1mm,colback=cellbackground, colframe=cellborder]
%\prompt{In}{incolor}{7}{\boxspacing}
%\begin{Verbatim}[commandchars=\\\{\}]
%\PY{c+c1}{\PYZsh{} Find maximum likelihood estimation}
%\PY{n}{mle\PYZus{}idx} \PY{o}{=} \PY{n}{np}\PY{o}{.}\PY{n}{argmax}\PY{p}{(}\PY{n}{Y}\PY{p}{)}
%\PY{n}{x\PYZus{}mle} \PY{o}{=} \PY{n}{X}\PY{p}{[}\PY{n}{mle\PYZus{}idx}\PY{p}{]}
%\PY{n}{y\PYZus{}mle} \PY{o}{=} \PY{n}{Y}\PY{p}{[}\PY{n}{mle\PYZus{}idx}\PY{p}{]}
%
%\PY{n+nb}{print}\PY{p}{(}\PY{l+s+sa}{f}\PY{l+s+s1}{\PYZsq{}}\PY{l+s+s1}{P(}\PY{l+s+si}{\PYZob{}}\PY{n}{x\PYZus{}mle}\PY{l+s+si}{\PYZcb{}}\PY{l+s+s1}{) = }\PY{l+s+si}{\PYZob{}}\PY{n}{y\PYZus{}mle}\PY{l+s+si}{:}\PY{l+s+s1}{.3}\PY{l+s+si}{\PYZcb{}}\PY{l+s+s1}{\PYZsq{}}\PY{p}{)}
%\end{Verbatim}
%\end{tcolorbox}
%
%    \begin{Verbatim}[commandchars=\\\{\}]
%P(10000) = 0.0443
%    \end{Verbatim}

    В нашем примере оценкой максимального правдоподобия для количества рыб в
озере является число \(\hat{N} = 10^4\), а вероятность соответствующего
события \(\mathrm{P}(B_{100}(10^4)) \approx 0.044\).

    \begin{center}\rule{0.5\linewidth}{0.5pt}\end{center}

    \hypertarget{ux438ux441ux442ux43eux447ux43dux438ux43aux438}{%
\section{Источники}\label{ux438ux441ux442ux43eux447ux43dux438ux43aux438}}

\begin{enumerate}
\def\labelenumi{\arabic{enumi}.}
\tightlist
\item
  \emph{Ширяев А. Н.} Вероятность --- 1. --- М.: МЦНМО, 2007. --- 517 с.
\item
  \emph{Чернова Н. И.} Теория вероятностей. Учебное пособие. ---
  Новосиб. гос. ун-т, 2007. --- 160~с.
\item
  \emph{Феллер В.} Введение в теорию вероятностей и её приложения. ---
  М.: Мир, 1964. --- 498 с.
\item
  \emph{Шпигельхалтер Д.} Искусство статистики. Как находить ответы в
  данных. --- М.: Манн, Иванов и Фербер, 2021. --- 448 с.
\end{enumerate}

%    \begin{tcolorbox}[breakable, size=fbox, boxrule=1pt, pad at break*=1mm,colback=cellbackground, colframe=cellborder]
%\prompt{In}{incolor}{8}{\boxspacing}
%\begin{Verbatim}[commandchars=\\\{\}]
%\PY{c+c1}{\PYZsh{} Versions used}
%\PY{k+kn}{import} \PY{n+nn}{sys}
%\PY{n+nb}{print}\PY{p}{(}\PY{l+s+s1}{\PYZsq{}}\PY{l+s+s1}{Python: }\PY{l+s+si}{\PYZob{}\PYZcb{}}\PY{l+s+s1}{.}\PY{l+s+si}{\PYZob{}\PYZcb{}}\PY{l+s+s1}{.}\PY{l+s+si}{\PYZob{}\PYZcb{}}\PY{l+s+s1}{\PYZsq{}}\PY{o}{.}\PY{n}{format}\PY{p}{(}\PY{o}{*}\PY{n}{sys}\PY{o}{.}\PY{n}{version\PYZus{}info}\PY{p}{[}\PY{p}{:}\PY{l+m+mi}{3}\PY{p}{]}\PY{p}{)}\PY{p}{)}
%\PY{n+nb}{print}\PY{p}{(}\PY{l+s+s1}{\PYZsq{}}\PY{l+s+s1}{numpy: }\PY{l+s+si}{\PYZob{}\PYZcb{}}\PY{l+s+s1}{\PYZsq{}}\PY{o}{.}\PY{n}{format}\PY{p}{(}\PY{n}{np}\PY{o}{.}\PY{n}{\PYZus{}\PYZus{}version\PYZus{}\PYZus{}}\PY{p}{)}\PY{p}{)}
%\PY{n+nb}{print}\PY{p}{(}\PY{l+s+s1}{\PYZsq{}}\PY{l+s+s1}{matplotlib: }\PY{l+s+si}{\PYZob{}\PYZcb{}}\PY{l+s+s1}{\PYZsq{}}\PY{o}{.}\PY{n}{format}\PY{p}{(}\PY{n}{matplotlib}\PY{o}{.}\PY{n}{\PYZus{}\PYZus{}version\PYZus{}\PYZus{}}\PY{p}{)}\PY{p}{)}
%\PY{n+nb}{print}\PY{p}{(}\PY{l+s+s1}{\PYZsq{}}\PY{l+s+s1}{seaborn: }\PY{l+s+si}{\PYZob{}\PYZcb{}}\PY{l+s+s1}{\PYZsq{}}\PY{o}{.}\PY{n}{format}\PY{p}{(}\PY{n}{seaborn}\PY{o}{.}\PY{n}{\PYZus{}\PYZus{}version\PYZus{}\PYZus{}}\PY{p}{)}\PY{p}{)}
%\end{Verbatim}
%\end{tcolorbox}
%
%    \begin{Verbatim}[commandchars=\\\{\}]
%Python: 3.7.16
%numpy: 1.20.3
%matplotlib: 3.5.1
%seaborn: 0.12.2
%    \end{Verbatim}


    % Add a bibliography block to the postdoc
    
    
    
\end{document}
