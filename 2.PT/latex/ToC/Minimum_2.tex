%%%%%%%%%%%%%%%%%%%%%%%%%%%%%%%%%%%%%%%%%%%%%%%%%%%%%%%%%%%%%%%%%%%%%%%%%%%
%% Packages
\documentclass[12pt,oneside,openany]{article}
\usepackage{polyglossia}                       % загружает пакет многоязыковой вёрстки
\usepackage{geometry}                          % для последующего задания полей

\usepackage{indentfirst}                       % первый абзац с красной строки
\usepackage[dvipsnames]{xcolor}
%\usepackage{savetrees}
\usepackage{hyperref}                          % гиперссылки

%%%%%%%%%%%%%%%%%%%%%%%%%%%%%%%%%%%%%%%%%%%%%%%%%%%%%%%%%%%%%%%%%%%%%%%%%%%
%% Styles
\geometry{a4paper, top=1cm, bottom=1cm, left=2.5cm, right=1cm, nofoot, nomarginpar} %, heightrounded, showframe

%% Languages
\setmainlanguage[babelshorthands=true]{russian} % устанавливает главный язык документа русский с поддержкой приятных команд пакета babel
\setotherlanguage{english}                     % устанавливает второй язык документа

%% Fonts
\defaultfontfeatures{Ligatures={TeX},Renderer=Basic} %% задаёт свойства шрифтов по умолчанию
\setmainfont{STIX Two Text}                    % задаёт основной шрифт документа
\setsansfont{Tahoma}                           % задаёт шрифт без засечек
\setmonofont{Courier New}                      % задаёт моноширинный шрифт
\newfontfamily\cyrillicfonttt[Script=Cyrillic]{Courier New}
\usepackage{unicode-math}                      % использование Unicode-шрифтов для формул
\setmathfont{STIX Two Math}

%% Выравнивание и переносы
\tolerance 1414
\hbadness 1414
\emergencystretch 1.5em % В случае проблем регулировать в первую очередь
\hfuzz 0.3pt
\vfuzz \hfuzz
%\raggedbottom
%\sloppy                % Избавляемся от переполнений
\clubpenalty=10000      % Запрещаем разрыв страницы после первой строки абзаца
\widowpenalty=10000     % Запрещаем разрыв страницы после последней строки абзаца
\brokenpenalty=4991     % Ограничение на разрыв страницы, если строка заканчивается переносом

\hypersetup{pdfstartview=FitH, colorlinks=true, urlcolor=blue}
%%%%%%%%%%%%%%%%%%%%%%%%%%%%%%%%%%%%%%%%%%%%%%%%%%%%%%%%%%%%%%%%%%%%%%%%%%%
%% Commands
\newcommand*{\todo}[1]{\textcolor{magenta}{\textbf{#1}}}

%%%%%%%%%%%%%%%%%%%%%%%%%%%%%%%%%%%%%%%%%%%%%%%%%%%%%%%%%%%%%%%%%%%%%%%%%%%


\begin{document}

\title{
  \large
  \textbf{Суррогатное моделирование и~оптимизация в~прикладных задачах} \\
  Минимальный набор знаний, весенний семестр
%  (оценка <<удовлетворительно>>)
}

\author{}
\date{}

\maketitle
\thispagestyle{empty}

\vspace{-10ex}


\begin{enumerate}
  
%  \item \textbf{Числовые характеристики данных}
    \item Среднее, среднеквадратичное отклонение, медиана, интерквартильный размах
    \item Коэффициент корреляции Пирсона, ранговый коэффициент корреляции Спирмена

%  \item \textbf{Основы теории вероятностей}
    \item Вероятностное пространство $\left( \Omega, \mathcal{F}, \mathrm{P} \right)$
    \item Оценка максимального правдоподобия
    \item Условная вероятность, формула Байеса, формула полной вероятности
    \item Отношение шансов, отношение правдоподобия
    

%  \item \textbf{Случайные величины и их распределения}
    \item Случайная величина (СВЛ), дискретные и абсолютно непрерывные СВЛ
    \item Числовые характеристики СВЛ: математическое ожидание, дисперсия, ковариация, коэффициент корреляции
    \item Оптимальная линейная оценка СВЛ

%  \item \textbf{Распределение Гаусса}
    \item Гауссовская СВЛ, функция плотности
    \item Случайный вектор (СВК), гауссовский СВК, функция плотности совместной вероятности, ковариационная матрица
    \item Генерация выборки гауссовских СВК

%  \item \textbf{Условное математическое ожидание}
    \item Условное математическое ожидание и дисперсия, формулы полного мат. ожидания и полной дисперсии
    \item Условное распределение, обобщение формулы Байеса
    \item Теорема о нормальной корреляции, кривая регрессии
    \item Частное и условное распределения гауссовского СВК

%  \item \textbf{Гауссовские случайные процессы}
    \item Случайный процесс (СП), его сечение и траектория
    \item Математическое ожидание и ковариационная функция СП
    \item Гауссовский СП
    \item Генерация выборки реализаций гауссовского СП

%  \item \textbf{Регрессия на основе гауссовских процессов}
    \item Построение регрессионной модели с помощью гауссовского СП
    \item Формулы для апостериорного математического ожидания и апостериорной ковариационной матрицы
    \item Влияние параметров ядра и амплитуды шума на регрессионную кривую

%  \item \textbf{Байесовская оптимизация}
    \item Байесовская оптимизация, алгоритм
    \item Функции продвижения: нижняя граница доверительного интервала, вероятность улучшения, ожидаемое улучшение

\end{enumerate}



\end{document}
%%%%%%%%%%%%%%%%%%%%%%%%%%%%%%%%%%%%%%%%%%%%%%%%%%%%%%%%%%%%%%%%%%%%%%%%%%%