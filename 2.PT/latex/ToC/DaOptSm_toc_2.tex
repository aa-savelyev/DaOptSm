%%%%%%%%%%%%%%%%%%%%%%%%%%%%%%%%%%%%%%%%%%%%%%%%%%%%%%%%%%%%%%%%%%%%%%%%%%%
%% Packages
\documentclass[12pt,oneside,openany]{article}
\usepackage{polyglossia}                       % загружает пакет многоязыковой вёрстки
\usepackage{geometry}                          % для последующего задания полей

\usepackage{indentfirst}                       % первый абзац с красной строки
\usepackage[dvipsnames]{xcolor}
%\usepackage{savetrees}
\usepackage{hyperref}                          % гиперссылки

%%%%%%%%%%%%%%%%%%%%%%%%%%%%%%%%%%%%%%%%%%%%%%%%%%%%%%%%%%%%%%%%%%%%%%%%%%%
%% Styles
\geometry{a4paper, top=1cm, bottom=1cm, left=2.5cm, right=1cm, nofoot, nomarginpar} %, heightrounded, showframe

%% Languages
\setmainlanguage[babelshorthands=true]{russian} % устанавливает главный язык документа русский с поддержкой приятных команд пакета babel
\setotherlanguage{english}                     % устанавливает второй язык документа

%% Fonts
\defaultfontfeatures{Ligatures={TeX},Renderer=Basic} %% задаёт свойства шрифтов по умолчанию
\setmainfont{STIX Two Text}                    % задаёт основной шрифт документа
\setsansfont{Tahoma}                           % задаёт шрифт без засечек
\setmonofont{Courier New}                      % задаёт моноширинный шрифт
\newfontfamily\cyrillicfonttt[Script=Cyrillic]{Courier New}
\usepackage{unicode-math}                      % использование Unicode-шрифтов для формул
\setmathfont{STIX Two Math}

%% Выравнивание и переносы
\tolerance 1414
\hbadness 1414
\emergencystretch 1.5em % В случае проблем регулировать в первую очередь
\hfuzz 0.3pt
\vfuzz \hfuzz
%\raggedbottom
%\sloppy                % Избавляемся от переполнений
\clubpenalty=10000      % Запрещаем разрыв страницы после первой строки абзаца
\widowpenalty=10000     % Запрещаем разрыв страницы после последней строки абзаца
\brokenpenalty=4991     % Ограничение на разрыв страницы, если строка заканчивается переносом

\hypersetup{pdfstartview=FitH, colorlinks=true, urlcolor=blue}
%%%%%%%%%%%%%%%%%%%%%%%%%%%%%%%%%%%%%%%%%%%%%%%%%%%%%%%%%%%%%%%%%%%%%%%%%%%
%% Commands
\newcommand*{\todo}[1]{\textcolor{magenta}{\textbf{#1}}}
\newcommand*{\Rey}{\mathrm{Re}}

%%%%%%%%%%%%%%%%%%%%%%%%%%%%%%%%%%%%%%%%%%%%%%%%%%%%%%%%%%%%%%%%%%%%%%%%%%%


\begin{document}

\title{
  \large
  \textbf{Программа курса <<Анализ данных, суррогатное моделирование и~оптимизация в~прикладных задачах>>} \\
  2022 год, весенний семестр
}

\author{}
\date{}

\maketitle
\thispagestyle{empty}

\vspace{-10ex}
%\section{Отчётность}


\begin{enumerate}

  \item \textbf{Данные и методы работы с ними} \\
  Три парадигмы научных исследований, четвёртая парадигма Джима Грея. Постановка задачи статистического исследования, цикл PPDAC. Числовые характеристики выборки: среднее, медиана, среднеквадратичное отклонение, интерквартильный размах. Коэффициент корреляции Пирсона, ранговый коэффициент корреляции Спирмена.

  \item \textbf{Основы теории вероятностей} \\
  Аксиоматика теории вероятностей, вероятностное пространство. Условная вероятность, формула полной вероятности, формула Байеса. Отношение шансов и отношение правдоподобия. Оценка максимального правдоподобия.

  \item \textbf{Случайные величины и их распределения} \\
  Понятие случайной величины. Распределения случайной величины, многомерные распределения. Числовые характеристики случайных величин: математическое ожидание, дисперсия, ковариация, коэффициент корреляции.

  \item \textbf{Многомерное нормальное распределение} \\
  Гауссовские случайные величины, их свойства. Ковариационная матрица. Линейное преобразование многомерного нормального распределения, генерация выборки гауссовских векторов. Маргинальные и условные распределения.

  \item \textbf{Гауссовские случайные процессы} \\
  Случайные процессы: базовые понятие и примеры. Моментные характеристики случайных процессов. Гауссовские случайные процессы. Ковариационные функции. Генерация случайной выборки гауссовских процессов.

  \item \textbf{Регрессия на основе гауссовских процессов} \\
  Методы восстановление регрессии, параметрические и непараметрические модели. Ядерные методы. Построение регрессионной модели с помощью гауссовских процессов. Вычисление параметров ковариационной функции апостериорного процесса: апостериорное среднее и апостериорная дисперсия.

  \item \textbf{Оптимизация регрессионной кривой} \\
  Обучающая выборка с шумом. Влияние параметров ядра и амплитуды шума на регрессионную кривую. Оптимизация параметров ядра. Многомерный случай.

  \item \textbf{Алгоритм эффективной глобальной оптимизации} \\
  Методы планирования экспериментов, метод оптимальных латинских гиперкубов. Подход эксплуатации и эксплорации, функция ожидаемого улучшения. Алгоритм эффективной глобальной оптимизации.

\end{enumerate}





\end{document}
%%%%%%%%%%%%%%%%%%%%%%%%%%%%%%%%%%%%%%%%%%%%%%%%%%%%%%%%%%%%%%%%%%%%%%%%%%%