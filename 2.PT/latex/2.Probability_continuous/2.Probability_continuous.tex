\documentclass[11pt,a4paper]{article}

    \usepackage[breakable]{tcolorbox}
    \usepackage{parskip} % Stop auto-indenting (to mimic markdown behaviour)
    
    \usepackage{iftex}
    \ifPDFTeX
      \usepackage[T2A]{fontenc}
      \usepackage{mathpazo}
      \usepackage[russian,english]{babel}
    \else
      \usepackage{fontspec}
      \usepackage{polyglossia}
      \setmainlanguage[babelshorthands=true]{russian}    % Язык по-умолчанию русский с поддержкой приятных команд пакета babel
      \setotherlanguage{english}                         % Дополнительный язык = английский (в американской вариации по-умолчанию)
      \newfontfamily\cyrillicfonttt[Scale=0.87,BoldFont={Fira Mono Medium}] {Fira Mono}  % Моноширинный шрифт для кириллицы
      \defaultfontfeatures{Ligatures=TeX}
      \newfontfamily\cyrillicfont{STIX Two Text}         % Шрифт с засечками для кириллицы
    \fi
    \renewcommand{\linethickness}{0.1ex}

    % Basic figure setup, for now with no caption control since it's done
    % automatically by Pandoc (which extracts ![](path) syntax from Markdown).
    \usepackage{graphicx}
    % Maintain compatibility with old templates. Remove in nbconvert 6.0
    \let\Oldincludegraphics\includegraphics
    % Ensure that by default, figures have no caption (until we provide a
    % proper Figure object with a Caption API and a way to capture that
    % in the conversion process - todo).
    \usepackage{caption}
    \DeclareCaptionFormat{nocaption}{}
    \captionsetup{format=nocaption,aboveskip=0pt,belowskip=0pt}

    \usepackage[Export]{adjustbox} % Used to constrain images to a maximum size
    \adjustboxset{max size={0.9\linewidth}{0.9\paperheight}}
    \usepackage{float}
    \floatplacement{figure}{H} % forces figures to be placed at the correct location
    \usepackage{xcolor} % Allow colors to be defined
    \usepackage{enumerate} % Needed for markdown enumerations to work
    \usepackage{geometry} % Used to adjust the document margins
    \usepackage{amsmath} % Equations
    \usepackage{amssymb} % Equations
    \usepackage{textcomp} % defines textquotesingle
    % Hack from http://tex.stackexchange.com/a/47451/13684:
    \AtBeginDocument{%
        \def\PYZsq{\textquotesingle}% Upright quotes in Pygmentized code
    }
    \usepackage{upquote} % Upright quotes for verbatim code
    \usepackage{eurosym} % defines \euro
    \usepackage[mathletters]{ucs} % Extended unicode (utf-8) support
    \usepackage{fancyvrb} % verbatim replacement that allows latex
    \usepackage{grffile} % extends the file name processing of package graphics 
                         % to support a larger range
    \makeatletter % fix for grffile with XeLaTeX
    \def\Gread@@xetex#1{%
      \IfFileExists{"\Gin@base".bb}%
      {\Gread@eps{\Gin@base.bb}}%
      {\Gread@@xetex@aux#1}%
    }
    \makeatother

    % The hyperref package gives us a pdf with properly built
    % internal navigation ('pdf bookmarks' for the table of contents,
    % internal cross-reference links, web links for URLs, etc.)
    \usepackage{hyperref}
    % The default LaTeX title has an obnoxious amount of whitespace. By default,
    % titling removes some of it. It also provides customization options.
    \usepackage{titling}
    \usepackage{longtable} % longtable support required by pandoc >1.10
    \usepackage{booktabs}  % table support for pandoc > 1.12.2
    \usepackage[inline]{enumitem} % IRkernel/repr support (it uses the enumerate* environment)
    \usepackage[normalem]{ulem} % ulem is needed to support strikethroughs (\sout)
                                % normalem makes italics be italics, not underlines
    \usepackage{mathrsfs}
    

    
    % Colors for the hyperref package
    \definecolor{urlcolor}{rgb}{0,.145,.698}
    \definecolor{linkcolor}{rgb}{.71,0.21,0.01}
    \definecolor{citecolor}{rgb}{.12,.54,.11}

    % ANSI colors
    \definecolor{ansi-black}{HTML}{3E424D}
    \definecolor{ansi-black-intense}{HTML}{282C36}
    \definecolor{ansi-red}{HTML}{E75C58}
    \definecolor{ansi-red-intense}{HTML}{B22B31}
    \definecolor{ansi-green}{HTML}{00A250}
    \definecolor{ansi-green-intense}{HTML}{007427}
    \definecolor{ansi-yellow}{HTML}{DDB62B}
    \definecolor{ansi-yellow-intense}{HTML}{B27D12}
    \definecolor{ansi-blue}{HTML}{208FFB}
    \definecolor{ansi-blue-intense}{HTML}{0065CA}
    \definecolor{ansi-magenta}{HTML}{D160C4}
    \definecolor{ansi-magenta-intense}{HTML}{A03196}
    \definecolor{ansi-cyan}{HTML}{60C6C8}
    \definecolor{ansi-cyan-intense}{HTML}{258F8F}
    \definecolor{ansi-white}{HTML}{C5C1B4}
    \definecolor{ansi-white-intense}{HTML}{A1A6B2}
    \definecolor{ansi-default-inverse-fg}{HTML}{FFFFFF}
    \definecolor{ansi-default-inverse-bg}{HTML}{000000}

    % commands and environments needed by pandoc snippets
    % extracted from the output of `pandoc -s`
    \providecommand{\tightlist}{%
      \setlength{\itemsep}{0pt}\setlength{\parskip}{0pt}}
    \DefineVerbatimEnvironment{Highlighting}{Verbatim}{commandchars=\\\{\}}
    % Add ',fontsize=\small' for more characters per line
    \newenvironment{Shaded}{}{}
    \newcommand{\KeywordTok}[1]{\textcolor[rgb]{0.00,0.44,0.13}{\textbf{{#1}}}}
    \newcommand{\DataTypeTok}[1]{\textcolor[rgb]{0.56,0.13,0.00}{{#1}}}
    \newcommand{\DecValTok}[1]{\textcolor[rgb]{0.25,0.63,0.44}{{#1}}}
    \newcommand{\BaseNTok}[1]{\textcolor[rgb]{0.25,0.63,0.44}{{#1}}}
    \newcommand{\FloatTok}[1]{\textcolor[rgb]{0.25,0.63,0.44}{{#1}}}
    \newcommand{\CharTok}[1]{\textcolor[rgb]{0.25,0.44,0.63}{{#1}}}
    \newcommand{\StringTok}[1]{\textcolor[rgb]{0.25,0.44,0.63}{{#1}}}
    \newcommand{\CommentTok}[1]{\textcolor[rgb]{0.38,0.63,0.69}{\textit{{#1}}}}
    \newcommand{\OtherTok}[1]{\textcolor[rgb]{0.00,0.44,0.13}{{#1}}}
    \newcommand{\AlertTok}[1]{\textcolor[rgb]{1.00,0.00,0.00}{\textbf{{#1}}}}
    \newcommand{\FunctionTok}[1]{\textcolor[rgb]{0.02,0.16,0.49}{{#1}}}
    \newcommand{\RegionMarkerTok}[1]{{#1}}
    \newcommand{\ErrorTok}[1]{\textcolor[rgb]{1.00,0.00,0.00}{\textbf{{#1}}}}
    \newcommand{\NormalTok}[1]{{#1}}
    
    % Additional commands for more recent versions of Pandoc
    \newcommand{\ConstantTok}[1]{\textcolor[rgb]{0.53,0.00,0.00}{{#1}}}
    \newcommand{\SpecialCharTok}[1]{\textcolor[rgb]{0.25,0.44,0.63}{{#1}}}
    \newcommand{\VerbatimStringTok}[1]{\textcolor[rgb]{0.25,0.44,0.63}{{#1}}}
    \newcommand{\SpecialStringTok}[1]{\textcolor[rgb]{0.73,0.40,0.53}{{#1}}}
    \newcommand{\ImportTok}[1]{{#1}}
    \newcommand{\DocumentationTok}[1]{\textcolor[rgb]{0.73,0.13,0.13}{\textit{{#1}}}}
    \newcommand{\AnnotationTok}[1]{\textcolor[rgb]{0.38,0.63,0.69}{\textbf{\textit{{#1}}}}}
    \newcommand{\CommentVarTok}[1]{\textcolor[rgb]{0.38,0.63,0.69}{\textbf{\textit{{#1}}}}}
    \newcommand{\VariableTok}[1]{\textcolor[rgb]{0.10,0.09,0.49}{{#1}}}
    \newcommand{\ControlFlowTok}[1]{\textcolor[rgb]{0.00,0.44,0.13}{\textbf{{#1}}}}
    \newcommand{\OperatorTok}[1]{\textcolor[rgb]{0.40,0.40,0.40}{{#1}}}
    \newcommand{\BuiltInTok}[1]{{#1}}
    \newcommand{\ExtensionTok}[1]{{#1}}
    \newcommand{\PreprocessorTok}[1]{\textcolor[rgb]{0.74,0.48,0.00}{{#1}}}
    \newcommand{\AttributeTok}[1]{\textcolor[rgb]{0.49,0.56,0.16}{{#1}}}
    \newcommand{\InformationTok}[1]{\textcolor[rgb]{0.38,0.63,0.69}{\textbf{\textit{{#1}}}}}
    \newcommand{\WarningTok}[1]{\textcolor[rgb]{0.38,0.63,0.69}{\textbf{\textit{{#1}}}}}
    
    
    % Define a nice break command that doesn't care if a line doesn't already
    % exist.
    \def\br{\hspace*{\fill} \\* }
    % Math Jax compatibility definitions
    \def\gt{>}
    \def\lt{<}
    \let\Oldtex\TeX
    \let\Oldlatex\LaTeX
    \renewcommand{\TeX}{\textrm{\Oldtex}}
    \renewcommand{\LaTeX}{\textrm{\Oldlatex}}
    % Document parameters
    % Document title
    \title{
    Лекция 2 \\
    Необходимые сведения из теории вероятностей --- 2 \\
    \emph{\large Математические основания теории вероятностей}
  }
    
    
    
    
    
% Pygments definitions
\makeatletter
\def\PY@reset{\let\PY@it=\relax \let\PY@bf=\relax%
    \let\PY@ul=\relax \let\PY@tc=\relax%
    \let\PY@bc=\relax \let\PY@ff=\relax}
\def\PY@tok#1{\csname PY@tok@#1\endcsname}
\def\PY@toks#1+{\ifx\relax#1\empty\else%
    \PY@tok{#1}\expandafter\PY@toks\fi}
\def\PY@do#1{\PY@bc{\PY@tc{\PY@ul{%
    \PY@it{\PY@bf{\PY@ff{#1}}}}}}}
\def\PY#1#2{\PY@reset\PY@toks#1+\relax+\PY@do{#2}}

\expandafter\def\csname PY@tok@w\endcsname{\def\PY@tc##1{\textcolor[rgb]{0.73,0.73,0.73}{##1}}}
\expandafter\def\csname PY@tok@c\endcsname{\let\PY@it=\textit\def\PY@tc##1{\textcolor[rgb]{0.25,0.50,0.50}{##1}}}
\expandafter\def\csname PY@tok@cp\endcsname{\def\PY@tc##1{\textcolor[rgb]{0.74,0.48,0.00}{##1}}}
\expandafter\def\csname PY@tok@k\endcsname{\let\PY@bf=\textbf\def\PY@tc##1{\textcolor[rgb]{0.00,0.50,0.00}{##1}}}
\expandafter\def\csname PY@tok@kp\endcsname{\def\PY@tc##1{\textcolor[rgb]{0.00,0.50,0.00}{##1}}}
\expandafter\def\csname PY@tok@kt\endcsname{\def\PY@tc##1{\textcolor[rgb]{0.69,0.00,0.25}{##1}}}
\expandafter\def\csname PY@tok@o\endcsname{\def\PY@tc##1{\textcolor[rgb]{0.40,0.40,0.40}{##1}}}
\expandafter\def\csname PY@tok@ow\endcsname{\let\PY@bf=\textbf\def\PY@tc##1{\textcolor[rgb]{0.67,0.13,1.00}{##1}}}
\expandafter\def\csname PY@tok@nb\endcsname{\def\PY@tc##1{\textcolor[rgb]{0.00,0.50,0.00}{##1}}}
\expandafter\def\csname PY@tok@nf\endcsname{\def\PY@tc##1{\textcolor[rgb]{0.00,0.00,1.00}{##1}}}
\expandafter\def\csname PY@tok@nc\endcsname{\let\PY@bf=\textbf\def\PY@tc##1{\textcolor[rgb]{0.00,0.00,1.00}{##1}}}
\expandafter\def\csname PY@tok@nn\endcsname{\let\PY@bf=\textbf\def\PY@tc##1{\textcolor[rgb]{0.00,0.00,1.00}{##1}}}
\expandafter\def\csname PY@tok@ne\endcsname{\let\PY@bf=\textbf\def\PY@tc##1{\textcolor[rgb]{0.82,0.25,0.23}{##1}}}
\expandafter\def\csname PY@tok@nv\endcsname{\def\PY@tc##1{\textcolor[rgb]{0.10,0.09,0.49}{##1}}}
\expandafter\def\csname PY@tok@no\endcsname{\def\PY@tc##1{\textcolor[rgb]{0.53,0.00,0.00}{##1}}}
\expandafter\def\csname PY@tok@nl\endcsname{\def\PY@tc##1{\textcolor[rgb]{0.63,0.63,0.00}{##1}}}
\expandafter\def\csname PY@tok@ni\endcsname{\let\PY@bf=\textbf\def\PY@tc##1{\textcolor[rgb]{0.60,0.60,0.60}{##1}}}
\expandafter\def\csname PY@tok@na\endcsname{\def\PY@tc##1{\textcolor[rgb]{0.49,0.56,0.16}{##1}}}
\expandafter\def\csname PY@tok@nt\endcsname{\let\PY@bf=\textbf\def\PY@tc##1{\textcolor[rgb]{0.00,0.50,0.00}{##1}}}
\expandafter\def\csname PY@tok@nd\endcsname{\def\PY@tc##1{\textcolor[rgb]{0.67,0.13,1.00}{##1}}}
\expandafter\def\csname PY@tok@s\endcsname{\def\PY@tc##1{\textcolor[rgb]{0.73,0.13,0.13}{##1}}}
\expandafter\def\csname PY@tok@sd\endcsname{\let\PY@it=\textit\def\PY@tc##1{\textcolor[rgb]{0.73,0.13,0.13}{##1}}}
\expandafter\def\csname PY@tok@si\endcsname{\let\PY@bf=\textbf\def\PY@tc##1{\textcolor[rgb]{0.73,0.40,0.53}{##1}}}
\expandafter\def\csname PY@tok@se\endcsname{\let\PY@bf=\textbf\def\PY@tc##1{\textcolor[rgb]{0.73,0.40,0.13}{##1}}}
\expandafter\def\csname PY@tok@sr\endcsname{\def\PY@tc##1{\textcolor[rgb]{0.73,0.40,0.53}{##1}}}
\expandafter\def\csname PY@tok@ss\endcsname{\def\PY@tc##1{\textcolor[rgb]{0.10,0.09,0.49}{##1}}}
\expandafter\def\csname PY@tok@sx\endcsname{\def\PY@tc##1{\textcolor[rgb]{0.00,0.50,0.00}{##1}}}
\expandafter\def\csname PY@tok@m\endcsname{\def\PY@tc##1{\textcolor[rgb]{0.40,0.40,0.40}{##1}}}
\expandafter\def\csname PY@tok@gh\endcsname{\let\PY@bf=\textbf\def\PY@tc##1{\textcolor[rgb]{0.00,0.00,0.50}{##1}}}
\expandafter\def\csname PY@tok@gu\endcsname{\let\PY@bf=\textbf\def\PY@tc##1{\textcolor[rgb]{0.50,0.00,0.50}{##1}}}
\expandafter\def\csname PY@tok@gd\endcsname{\def\PY@tc##1{\textcolor[rgb]{0.63,0.00,0.00}{##1}}}
\expandafter\def\csname PY@tok@gi\endcsname{\def\PY@tc##1{\textcolor[rgb]{0.00,0.63,0.00}{##1}}}
\expandafter\def\csname PY@tok@gr\endcsname{\def\PY@tc##1{\textcolor[rgb]{1.00,0.00,0.00}{##1}}}
\expandafter\def\csname PY@tok@ge\endcsname{\let\PY@it=\textit}
\expandafter\def\csname PY@tok@gs\endcsname{\let\PY@bf=\textbf}
\expandafter\def\csname PY@tok@gp\endcsname{\let\PY@bf=\textbf\def\PY@tc##1{\textcolor[rgb]{0.00,0.00,0.50}{##1}}}
\expandafter\def\csname PY@tok@go\endcsname{\def\PY@tc##1{\textcolor[rgb]{0.53,0.53,0.53}{##1}}}
\expandafter\def\csname PY@tok@gt\endcsname{\def\PY@tc##1{\textcolor[rgb]{0.00,0.27,0.87}{##1}}}
\expandafter\def\csname PY@tok@err\endcsname{\def\PY@bc##1{\setlength{\fboxsep}{0pt}\fcolorbox[rgb]{1.00,0.00,0.00}{1,1,1}{\strut ##1}}}
\expandafter\def\csname PY@tok@kc\endcsname{\let\PY@bf=\textbf\def\PY@tc##1{\textcolor[rgb]{0.00,0.50,0.00}{##1}}}
\expandafter\def\csname PY@tok@kd\endcsname{\let\PY@bf=\textbf\def\PY@tc##1{\textcolor[rgb]{0.00,0.50,0.00}{##1}}}
\expandafter\def\csname PY@tok@kn\endcsname{\let\PY@bf=\textbf\def\PY@tc##1{\textcolor[rgb]{0.00,0.50,0.00}{##1}}}
\expandafter\def\csname PY@tok@kr\endcsname{\let\PY@bf=\textbf\def\PY@tc##1{\textcolor[rgb]{0.00,0.50,0.00}{##1}}}
\expandafter\def\csname PY@tok@bp\endcsname{\def\PY@tc##1{\textcolor[rgb]{0.00,0.50,0.00}{##1}}}
\expandafter\def\csname PY@tok@fm\endcsname{\def\PY@tc##1{\textcolor[rgb]{0.00,0.00,1.00}{##1}}}
\expandafter\def\csname PY@tok@vc\endcsname{\def\PY@tc##1{\textcolor[rgb]{0.10,0.09,0.49}{##1}}}
\expandafter\def\csname PY@tok@vg\endcsname{\def\PY@tc##1{\textcolor[rgb]{0.10,0.09,0.49}{##1}}}
\expandafter\def\csname PY@tok@vi\endcsname{\def\PY@tc##1{\textcolor[rgb]{0.10,0.09,0.49}{##1}}}
\expandafter\def\csname PY@tok@vm\endcsname{\def\PY@tc##1{\textcolor[rgb]{0.10,0.09,0.49}{##1}}}
\expandafter\def\csname PY@tok@sa\endcsname{\def\PY@tc##1{\textcolor[rgb]{0.73,0.13,0.13}{##1}}}
\expandafter\def\csname PY@tok@sb\endcsname{\def\PY@tc##1{\textcolor[rgb]{0.73,0.13,0.13}{##1}}}
\expandafter\def\csname PY@tok@sc\endcsname{\def\PY@tc##1{\textcolor[rgb]{0.73,0.13,0.13}{##1}}}
\expandafter\def\csname PY@tok@dl\endcsname{\def\PY@tc##1{\textcolor[rgb]{0.73,0.13,0.13}{##1}}}
\expandafter\def\csname PY@tok@s2\endcsname{\def\PY@tc##1{\textcolor[rgb]{0.73,0.13,0.13}{##1}}}
\expandafter\def\csname PY@tok@sh\endcsname{\def\PY@tc##1{\textcolor[rgb]{0.73,0.13,0.13}{##1}}}
\expandafter\def\csname PY@tok@s1\endcsname{\def\PY@tc##1{\textcolor[rgb]{0.73,0.13,0.13}{##1}}}
\expandafter\def\csname PY@tok@mb\endcsname{\def\PY@tc##1{\textcolor[rgb]{0.40,0.40,0.40}{##1}}}
\expandafter\def\csname PY@tok@mf\endcsname{\def\PY@tc##1{\textcolor[rgb]{0.40,0.40,0.40}{##1}}}
\expandafter\def\csname PY@tok@mh\endcsname{\def\PY@tc##1{\textcolor[rgb]{0.40,0.40,0.40}{##1}}}
\expandafter\def\csname PY@tok@mi\endcsname{\def\PY@tc##1{\textcolor[rgb]{0.40,0.40,0.40}{##1}}}
\expandafter\def\csname PY@tok@il\endcsname{\def\PY@tc##1{\textcolor[rgb]{0.40,0.40,0.40}{##1}}}
\expandafter\def\csname PY@tok@mo\endcsname{\def\PY@tc##1{\textcolor[rgb]{0.40,0.40,0.40}{##1}}}
\expandafter\def\csname PY@tok@ch\endcsname{\let\PY@it=\textit\def\PY@tc##1{\textcolor[rgb]{0.25,0.50,0.50}{##1}}}
\expandafter\def\csname PY@tok@cm\endcsname{\let\PY@it=\textit\def\PY@tc##1{\textcolor[rgb]{0.25,0.50,0.50}{##1}}}
\expandafter\def\csname PY@tok@cpf\endcsname{\let\PY@it=\textit\def\PY@tc##1{\textcolor[rgb]{0.25,0.50,0.50}{##1}}}
\expandafter\def\csname PY@tok@c1\endcsname{\let\PY@it=\textit\def\PY@tc##1{\textcolor[rgb]{0.25,0.50,0.50}{##1}}}
\expandafter\def\csname PY@tok@cs\endcsname{\let\PY@it=\textit\def\PY@tc##1{\textcolor[rgb]{0.25,0.50,0.50}{##1}}}

\def\PYZbs{\char`\\}
\def\PYZus{\char`\_}
\def\PYZob{\char`\{}
\def\PYZcb{\char`\}}
\def\PYZca{\char`\^}
\def\PYZam{\char`\&}
\def\PYZlt{\char`\<}
\def\PYZgt{\char`\>}
\def\PYZsh{\char`\#}
\def\PYZpc{\char`\%}
\def\PYZdl{\char`\$}
\def\PYZhy{\char`\-}
\def\PYZsq{\char`\'}
\def\PYZdq{\char`\"}
\def\PYZti{\char`\~}
% for compatibility with earlier versions
\def\PYZat{@}
\def\PYZlb{[}
\def\PYZrb{]}
\makeatother


    % For linebreaks inside Verbatim environment from package fancyvrb. 
    \makeatletter
        \newbox\Wrappedcontinuationbox 
        \newbox\Wrappedvisiblespacebox 
        \newcommand*\Wrappedvisiblespace {\textcolor{red}{\textvisiblespace}} 
        \newcommand*\Wrappedcontinuationsymbol {\textcolor{red}{\llap{\tiny$\m@th\hookrightarrow$}}} 
        \newcommand*\Wrappedcontinuationindent {3ex } 
        \newcommand*\Wrappedafterbreak {\kern\Wrappedcontinuationindent\copy\Wrappedcontinuationbox} 
        % Take advantage of the already applied Pygments mark-up to insert 
        % potential linebreaks for TeX processing. 
        %        {, <, #, %, $, ' and ": go to next line. 
        %        _, }, ^, &, >, - and ~: stay at end of broken line. 
        % Use of \textquotesingle for straight quote. 
        \newcommand*\Wrappedbreaksatspecials {% 
            \def\PYGZus{\discretionary{\char`\_}{\Wrappedafterbreak}{\char`\_}}% 
            \def\PYGZob{\discretionary{}{\Wrappedafterbreak\char`\{}{\char`\{}}% 
            \def\PYGZcb{\discretionary{\char`\}}{\Wrappedafterbreak}{\char`\}}}% 
            \def\PYGZca{\discretionary{\char`\^}{\Wrappedafterbreak}{\char`\^}}% 
            \def\PYGZam{\discretionary{\char`\&}{\Wrappedafterbreak}{\char`\&}}% 
            \def\PYGZlt{\discretionary{}{\Wrappedafterbreak\char`\<}{\char`\<}}% 
            \def\PYGZgt{\discretionary{\char`\>}{\Wrappedafterbreak}{\char`\>}}% 
            \def\PYGZsh{\discretionary{}{\Wrappedafterbreak\char`\#}{\char`\#}}% 
            \def\PYGZpc{\discretionary{}{\Wrappedafterbreak\char`\%}{\char`\%}}% 
            \def\PYGZdl{\discretionary{}{\Wrappedafterbreak\char`\$}{\char`\$}}% 
            \def\PYGZhy{\discretionary{\char`\-}{\Wrappedafterbreak}{\char`\-}}% 
            \def\PYGZsq{\discretionary{}{\Wrappedafterbreak\textquotesingle}{\textquotesingle}}% 
            \def\PYGZdq{\discretionary{}{\Wrappedafterbreak\char`\"}{\char`\"}}% 
            \def\PYGZti{\discretionary{\char`\~}{\Wrappedafterbreak}{\char`\~}}% 
        } 
        % Some characters . , ; ? ! / are not pygmentized. 
        % This macro makes them "active" and they will insert potential linebreaks 
        \newcommand*\Wrappedbreaksatpunct {% 
            \lccode`\~`\.\lowercase{\def~}{\discretionary{\hbox{\char`\.}}{\Wrappedafterbreak}{\hbox{\char`\.}}}% 
            \lccode`\~`\,\lowercase{\def~}{\discretionary{\hbox{\char`\,}}{\Wrappedafterbreak}{\hbox{\char`\,}}}% 
            \lccode`\~`\;\lowercase{\def~}{\discretionary{\hbox{\char`\;}}{\Wrappedafterbreak}{\hbox{\char`\;}}}% 
            \lccode`\~`\:\lowercase{\def~}{\discretionary{\hbox{\char`\:}}{\Wrappedafterbreak}{\hbox{\char`\:}}}% 
            \lccode`\~`\?\lowercase{\def~}{\discretionary{\hbox{\char`\?}}{\Wrappedafterbreak}{\hbox{\char`\?}}}% 
            \lccode`\~`\!\lowercase{\def~}{\discretionary{\hbox{\char`\!}}{\Wrappedafterbreak}{\hbox{\char`\!}}}% 
            \lccode`\~`\/\lowercase{\def~}{\discretionary{\hbox{\char`\/}}{\Wrappedafterbreak}{\hbox{\char`\/}}}% 
            \catcode`\.\active
            \catcode`\,\active 
            \catcode`\;\active
            \catcode`\:\active
            \catcode`\?\active
            \catcode`\!\active
            \catcode`\/\active 
            \lccode`\~`\~ 	
        }
    \makeatother

    \let\OriginalVerbatim=\Verbatim
    \makeatletter
    \renewcommand{\Verbatim}[1][1]{%
        %\parskip\z@skip
        \sbox\Wrappedcontinuationbox {\Wrappedcontinuationsymbol}%
        \sbox\Wrappedvisiblespacebox {\FV@SetupFont\Wrappedvisiblespace}%
        \def\FancyVerbFormatLine ##1{\hsize\linewidth
            \vtop{\raggedright\hyphenpenalty\z@\exhyphenpenalty\z@
                \doublehyphendemerits\z@\finalhyphendemerits\z@
                \strut ##1\strut}%
        }%
        % If the linebreak is at a space, the latter will be displayed as visible
        % space at end of first line, and a continuation symbol starts next line.
        % Stretch/shrink are however usually zero for typewriter font.
        \def\FV@Space {%
            \nobreak\hskip\z@ plus\fontdimen3\font minus\fontdimen4\font
            \discretionary{\copy\Wrappedvisiblespacebox}{\Wrappedafterbreak}
            {\kern\fontdimen2\font}%
        }%
        
        % Allow breaks at special characters using \PYG... macros.
        \Wrappedbreaksatspecials
        % Breaks at punctuation characters . , ; ? ! and / need catcode=\active 	
        \OriginalVerbatim[#1,codes*=\Wrappedbreaksatpunct]%
    }
    \makeatother

    % Exact colors from NB
    \definecolor{incolor}{HTML}{303F9F}
    \definecolor{outcolor}{HTML}{D84315}
    \definecolor{cellborder}{HTML}{CFCFCF}
    \definecolor{cellbackground}{HTML}{F7F7F7}
    
    % prompt
    \makeatletter
    \newcommand{\boxspacing}{\kern\kvtcb@left@rule\kern\kvtcb@boxsep}
    \makeatother
    \newcommand{\prompt}[4]{
        \ttfamily\llap{{\color{#2}[#3]:\hspace{3pt}#4}}\vspace{-\baselineskip}
    }
    

    
    % Prevent overflowing lines due to hard-to-break entities
    \sloppy 
    % Setup hyperref package
    \hypersetup{
      breaklinks=true,  % so long urls are correctly broken across lines
      colorlinks=true,
      urlcolor=urlcolor,
      linkcolor=linkcolor,
      citecolor=citecolor,
      }
    % Slightly bigger margins than the latex defaults
    
    \geometry{verbose,tmargin=1in,bmargin=1in,lmargin=1in,rmargin=1in}
    
    

\begin{document}
    
\maketitle
%\thispagestyle{empty}


\begin{quote}
Теория вероятностей как математическая дисциплина может и должна быть
аксиоматизирована совершенно в том же смысле, как геометрия и алгебра.
Это означает, что, после того как даны названия изучаемым объектам и их
основным отношениям, а также аксиомы, которым эти соотношения должны
подчиняться, все дальнейшее изложение должно основываться исключительно
лишь на этих аксиомах, не опираясь на обычное конкретное значение этих
объектов и их отношений.\\
А. Н. Колмогоров. «Основные понятия теории вероятностей»
\end{quote}

%\tableofcontents
%\pagebreak



    \hypertarget{ux432ux435ux440ux43eux44fux442ux43dux43eux441ux442ux43dux43eux435-ux43fux440ux43eux441ux442ux440ux430ux43dux441ux442ux432ux43e}{%
\section{Вероятностное
пространство}\label{ux432ux435ux440ux43eux44fux442ux43dux43eux441ux442ux43dux43eux435-ux43fux440ux43eux441ux442ux440ux430ux43dux441ux442ux432ux43e}}

Согласно аксиоматике Колмогорова первоначальным объектом теории
вероятностей является вероятностное пространство
\((\Omega, \mathcal{F}, \mathrm{P})\). Здесь \((\Omega, \mathcal{F})\)
--- измеримое пространство, т. е. множество \(\Omega\), состоящее из
элементарных событий \(\omega\), с выделенной на нём системой
\(\mathcal{F}\) его подмножеств (событий), образующих
\(\sigma\)-алгебру, а \(\mathrm{P}\) --- вероятностная мера
(вероятность), определённая на множествах из \(\mathcal{F}\).

А теперь подробнее.

    \hypertarget{ux430ux43bux433ux435ux431ux440ux430}{%
\subsection{Алгебра}\label{ux430ux43bux433ux435ux431ux440ux430}}

В данной главе нам потребуется расширить понятие \emph{алгебры событий},
которое мы использовали в рамках элементарной теории веротяностей.

\hypertarget{sigma-ux430ux43bux433ux435ux431ux440ux430-ux441ux43eux431ux44bux442ux438ux439}{%
\paragraph{\texorpdfstring{\(\sigma\)-алгебра
событий}{\textbackslash{}sigma-алгебра событий}}\label{sigma-ux430ux43bux433ux435ux431ux440ux430-ux441ux43eux431ux44bux442ux438ux439}}

\textbf{Определение.} Множество \(\mathcal{F}\), элементами которого
являются подмножества множества \(\Omega\) называется
\(\sigma\)-\emph{алгеброй}, если оно удовлетворяет следующим условиям:
1. \(\Omega \in \mathcal{F}\) (\(\sigma\)-алгебра содержит достоверное
событие); 2. если \(A \in \mathcal{F}\), то
\(\overline{A} \in \mathcal{F}\) (вместе с любым множеством
\(\sigma\)-алгебра содержит противоположное к нему); 3. если
\(A_1,\,A_2,\,\ldots \in \mathcal{F}\), то
\(\bigcup\limits_{i=1}^{\infty}A_i \in \mathcal{F}\) (вместе с любым
\emph{счётным} набором событий \(\sigma\)-алгебра содержит их
объединение)

\hypertarget{ux431ux43eux440ux435ux43bux435ux432ux441ux43aux430ux44f-sigma-ux430ux43bux433ux435ux431ux440ux430-ux441ux43eux431ux44bux442ux438ux439}{%
\paragraph{\texorpdfstring{борелевская \(\sigma\)-алгебра
событий}{борелевская \textbackslash{}sigma-алгебра событий}}\label{ux431ux43eux440ux435ux43bux435ux432ux441ux43aux430ux44f-sigma-ux430ux43bux433ux435ux431ux440ux430-ux441ux43eux431ux44bux442ux438ux439}}

Борелевской \(\sigma\)-алгеброй в \(\mathbb{R}\) называется \emph{самая
маленькая} среди всех возможных \(\sigma\)-алгебр, содержащих
\emph{любые} интервалы на прямой.

\textbf{Определение.} Минимальной \(\sigma\)-алгебр, содержащей набор
множеств \(\mathcal{A}\), называется пересечение всех \(\sigma\)-алгебр,
содержащих \(\mathcal{A}\).

Пусть \(\Omega = \mathbb{R}\), а множество \(\mathcal{A}\) состоит из
всевозможных открытых интервалов \((a, b)\), где
\(a < b : \mathcal{A} = \left\{ (a, b) | -\infty < a < b < \infty \right\}\).

\textbf{Определение.} Минимальная \(\sigma\)-алгебра, содержащая
множество \(\mathcal{A}\) всех интервалов на вещественной прямой,
называется \emph{борелевской} \(\sigma\)-алгеброй в \(\mathbb{R}\) и
обозначается \(\mathcal{B}(\mathbb{R})\).

Если множество \(B\) содержится в \(\mathcal{B}(\mathbb{R}^n)\), то
говорят, что \(B\) --- \emph{борелевское множество}. Перечислим
некоторые множества на прямой, содержащиеся в
\(\mathcal{B}(\mathbb{R})\). Таковы все привычные нам множества:

\begin{enumerate}
\def\labelenumi{\arabic{enumi}.}
\tightlist
\item
  \(\mathbb{R} \in \mathcal{B}(\mathbb{R})\);
\item
  Все интервалы на прямой принадлежат \(\mathcal{B}(\mathbb{R})\);
\item
  Все одноточечные множества \(\{x\}\), где \(x \in \mathbb{R}\),
  принадлежат \(\mathcal{B}(\mathbb{R})\);
\item
  \(\mathbb{N} \in \mathcal{B}(\mathbb{R})\),
  \(\mathbb{Q} \in \mathcal{B}(\mathbb{R})\).
\end{enumerate}

Другими словами, получить множество, не содержащееся в
\(\mathcal{B}(\mathbb{R})\) не так-то просто, для этого требуются
специальные построения.

    \hypertarget{ux432ux435ux440ux43eux44fux442ux43dux43eux441ux442ux43dux430ux44f-ux43cux435ux440ux430}{%
\subsection{Вероятностная
мера}\label{ux432ux435ux440ux43eux44fux442ux43dux43eux441ux442ux43dux430ux44f-ux43cux435ux440ux430}}

\textbf{Определение.} Пусть \(\Omega\) --- непустое множество, а
\(\mathcal{F}\) --- \(\sigma\)-алгебра его подмножеств. Функция
\(\mathrm{P}: \mathcal{F} \rightarrow \mathbb{R}\) называется
\emph{вероятностной мерой}, если она обладает следующими свойствами:

\begin{enumerate}
  \item \(\mathrm{P}(A) \ge 0\), \(A \in \mathcal{F}\) (неотрицательность);
  \item \(\mathrm{P}\left( \bigcup\limits_{i=1}^{\infty}A_i \right) = \sum\limits_{i=1}^{\infty}\mathrm{P}(A_i)\), где \(A_i \in \mathcal{F}\), \(A_i \cap A_j = \emptyset\), \(i \ne j\) (счётная или \(\sigma\)-аддитивность)\\
(для любого счётного набора попарно несовместных событий мера их
объединения равна сумме их мер);
  \item \(\mathrm{P}(\Omega) = 1\) (нормированность).
\end{enumerate}

\begin{quote}
\emph{Замечание.} Существуют примеры неизмеримых множеств, например,
\emph{множество Витали}.
\end{quote}

    \begin{center}\rule{0.5\linewidth}{\linethickness}\end{center}

    \hypertarget{ux441ux43bux443ux447ux430ux439ux43dux44bux435-ux432ux435ux43bux438ux447ux438ux43dux44b-ux438-ux438ux445-ux440ux430ux441ux43fux440ux435ux434ux435ux43bux435ux43dux438ux44f}{%
\section{Случайные величины и их
распределения}\label{ux441ux43bux443ux447ux430ux439ux43dux44bux435-ux432ux435ux43bux438ux447ux438ux43dux44b-ux438-ux438ux445-ux440ux430ux441ux43fux440ux435ux434ux435ux43bux435ux43dux438ux44f}}

\hypertarget{ux43fux43eux43dux44fux442ux438ux435-ux441ux43bux443ux447ux430ux439ux43dux43eux439-ux432ux435ux43bux438ux447ux438ux43dux44b}{%
\subsection{Понятие случайной
величины}\label{ux43fux43eux43dux44fux442ux438ux435-ux441ux43bux443ux447ux430ux439ux43dux43eux439-ux432ux435ux43bux438ux447ux438ux43dux44b}}

Пусть и \((\Omega, \mathcal{F})\) и \((E, \mathcal{B})\) --- два
измеримых пространства. Функция \(\xi = \xi(\omega)\), определённая на
\((\Omega, \mathcal{F})\) со значениями в \(E\), называется
\(\mathcal{F}/\mathcal{B}\)-\emph{измеримой}, если множество
\(\{\omega: \xi(\omega) \in \mathcal{B}\} \in \mathcal{F}\) для всякого
\(B \in \mathcal{B}\). В теории вероятностей такие функции называют
\emph{случайными элементами} со значениями в \(E\). В том случае, когда
\(E = \mathbb{R}\) --- действительная прямая, а \(\mathcal{B}\) ---
\(\sigma\)-алгебра борелевских подмножеств \(\mathbb{R}\),
\(\mathcal{F}/\mathcal{B}\)-измеримые функции \(\xi=\xi(\omega)\)
называют (действительными) случайными величинами. В этом специальном
случае \(\mathcal{F}/\mathcal{B}\)-измеримые функции для краткости
называют просто \(\mathcal{F}\)-измеримыми.

Ещё раз другими словами:\\
\textbf{Определение.} Функция \(\xi: \Omega \rightarrow \mathbb{R}\)
называется \emph{случайной величиной}, если для любого борелевского
множества \(B \in \mathcal{B}(\mathbb{R})\) множество \(\xi^{-1}(B)\)
является событием, т. е. принадлежит \(\sigma\)-алгебре \(\mathcal{F}\).

\begin{quote}
\emph{Замечание.} В рамках нашего курса можно смело считать, что любое
множество элементарных исходов есть событие, и, следовательно, случайная
величина есть \emph{произвольная функция} из \(\Omega\) в
\(\mathbb{R}\).
\end{quote}

    \hypertarget{ux440ux430ux441ux43fux440ux435ux434ux435ux43bux435ux43dux438ux44f-ux441ux43bux443ux447ux430ux439ux43dux44bux445-ux432ux435ux43bux438ux447ux438ux43d}{%
\subsection{Распределения случайных
величин}\label{ux440ux430ux441ux43fux440ux435ux434ux435ux43bux435ux43dux438ux44f-ux441ux43bux443ux447ux430ux439ux43dux44bux445-ux432ux435ux43bux438ux447ux438ux43d}}

Существуют различные типы распределений случайных величин. Вся
вероятностная мера может быть сосредоточена в нескольких точках прямой,
а может быть «размазана» по некоторому интервалу или по всей прямой. В
зависимости от типа множества, на котором сосредоточена вероятностная
мера, распределения делят на дискретные, абсолютно непрерывные,
сингулярные и их смеси. Нас будут интересовать дискретные и абсолютно
непрерывные случайные величины.

\textbf{Определение.} Вероятностная мера \(P_\xi\) на
\((\mathbb{R}, \mathcal{B}(\mathbb{R}))\) c
\[ P_\xi(B) = \mathrm{P}\{\omega: \xi(\omega) \in B\}, \quad B \in \mathcal{B}(\mathbb{R}), \]
называется \emph{распределением вероятностей} случайной величины \(\xi\)
на \((\mathbb{R}, \mathcal{B}(\mathbb{R}))\).

\textbf{Определение.} Функция
\[ F_\xi(x) = \mathrm{P}\left\{ \omega: \xi(\omega) \le x \right\}, \quad x \in \mathbb{R} \]
называется \emph{функцией распределения} случайной величины \(\xi\).

\textbf{Определение.} Случайная величина \(\xi\) называется
\emph{дискретной} (имеет \emph{дискретное распределение}), если она
принимает не более чем счётное число значений, т. е. если существует
конечный или счётный набор чисел \(a_1, a_2, \ldots\) такой, что
\[ \mathrm{P}\{\xi=a_i\} > 0 \quad \forall i. \] Если число значений
\emph{конечно}, то такая случайная величина называется \emph{простой}.

Для дискретной случайной величины \(\xi\) мера \(P_\xi\) сосредоточена
не более чем в счётном числе точек и может быть представлена в виде
\[ \mathrm{P}\{\xi \in B\} \equiv P_\xi(B) = \sum\limits_{k:x_k \in B}p(x_k), \]
где \(p(x_k) = \mathrm{P}\{\xi=x_k\}\).

\textbf{Определение.} Случайная величина \(\xi\) называется
\emph{абсолютно непрерывной}, если существует неотрицательная функция
\(f=f_\xi(x)\), называемая \emph{плотностью}, такая, что
\[ F_\xi(x) = \int\limits_{-\infty}^x f_\xi(t) dt, \quad x \in \mathbb{R}. \]

Если \(\xi\) --- абсолютно непрерывная случайная величина с плотностью
\(f_\xi(x)\), то для любого борелевского множества \(B\) имеет место
равенство: \[ P_\xi(B) = \int\limits_B f_\xi(x)dx, \] где интеграл
понимается как интеграл Лебега от функции \(f_\xi(x)\) по мере Лебега
множества \(B\).

    \hypertarget{ux43cux43dux43eux433ux43eux43cux435ux440ux43dux44bux435-ux440ux430ux441ux43fux440ux435ux434ux435ux43bux435ux43dux438ux44f}{%
\subsection{Многомерные
распределения}\label{ux43cux43dux43eux433ux43eux43cux435ux440ux43dux44bux435-ux440ux430ux441ux43fux440ux435ux434ux435ux43bux435ux43dux438ux44f}}

Пусть случайные величины \(\xi_1, \ldots, \xi_n\) заданы на одном
вероятностном пространстве \((\Omega, \mathcal{F}, \mathrm{P})\).

\textbf{Определение.} Функция
\[ F_{\xi_1, \ldots, \xi_n}(x_1, \ldots, x_n) = \mathrm{P}(\xi_1<x_1, \ldots, \xi_n<x_n) \]
называется \emph{функцией совместного распределения} случайных величин
\(\xi_1, \ldots , \xi_n\).

Далее для простоты обозначений ограничимся вектором из двух величин
\((\xi, \eta)\).

\textbf{Определение.} Говорят, что случайные величины \(\xi, \eta\)
имеют \emph{абсолютно непрерывное совместное распределение}, если
существует неотрицательная функция \(f_{\xi, \eta}(x, y)\), называемая
\emph{плотностью}, такая, что
\[ F_{\xi, \eta}(x, y) = \int\limits_{-\infty}^{x} \int\limits_{-\infty}^{y} f_{\xi, \eta}(t_1, t_2) dt_1 dt_2. \]

\textbf{Теорема.} Если случайные величины \(\xi\) и \(\eta\) имеют
абсолютно непрерывное совместное распределение с плотностью \(f(x, y)\),
то \(\xi\) и \(\eta\) в отдельности также имеют абсолютно непрерывное
распределение с плотностями:
\[ f_{\xi}(x) = \int\limits_{-\infty}^{\infty} f_{\xi, \eta}(x, y)dy \quad \mathrm{и} \quad f_{\eta}(y) = \int\limits_{-\infty}^{\infty} f_{\xi, \eta}(x, y)dx. \]
Соотвествующие распределения называются \emph{частными} или
\emph{маргинальными}.

    \begin{center}\rule{0.5\linewidth}{\linethickness}\end{center}

    \hypertarget{ux447ux438ux441ux43bux43eux432ux44bux435-ux445ux430ux440ux430ux43aux442ux435ux440ux438ux441ux442ux438ux43aux438-ux441ux43bux443ux447ux430ux439ux43dux44bux445-ux432ux435ux43bux438ux447ux438ux43d}{%
\section{Числовые характеристики случайных
величин}\label{ux447ux438ux441ux43bux43eux432ux44bux435-ux445ux430ux440ux430ux43aux442ux435ux440ux438ux441ux442ux438ux43aux438-ux441ux43bux443ux447ux430ux439ux43dux44bux445-ux432ux435ux43bux438ux447ux438ux43d}}

\hypertarget{ux43cux430ux442ux435ux43cux430ux442ux438ux447ux435ux441ux43aux43eux435-ux43eux436ux438ux434ux430ux43dux438ux435}{%
\subsection{Математическое
ожидание}\label{ux43cux430ux442ux435ux43cux430ux442ux438ux447ux435ux441ux43aux43eux435-ux43eux436ux438ux434ux430ux43dux438ux435}}

\textbf{Определение (дискретный случай).} Пусть \(\xi\) --- дискретная
случайная величина на пространстве
\((\Omega, \mathcal{F}, \mathrm{P})\), а \(X\) --- множество её
значений. Тогда \emph{математическим ожиданием} \(\xi\) называется
число, равное
\[ \mathrm{E}\xi = \sum\limits_{x \in X}x\mathrm{P}(\xi = x), \] если
этот ряд сходится абсолютно.

\textbf{Определение (абсолютно непрерывный случай).} Пусть \(\xi\) ---
абсолютно непрерывная случайная величина на пространстве
\((\Omega, \mathcal{F}, \mathrm{P})\), а \(F_\xi\), \(f_\xi\) --- её
функция распределения и плотность. Тогда математическим ожиданием
\(\xi\) называется число, равное
\[ \mathrm{E}\xi = \int\limits_{-\infty}^{\infty} xdF_\xi(x) = \int\limits_{-\infty}^{\infty} xf_\xi(x)dx, \]
если этот интеграл сходится абсолютно.

\begin{quote}
\emph{Замечание.} Определения и свойства дисперсии, ковариации и
коэффициента корреляции случайных величин в \emph{общем случае} такие
же, как и в случае \emph{простых} случайных величин (см. предыдущую
главу).
\end{quote}

    \begin{center}\rule{0.5\linewidth}{\linethickness}\end{center}

%    \hypertarget{ux43bux438ux442ux435ux440ux430ux442ux443ux440ux430}{%
%\section{Литература}\label{ux43bux438ux442ux435ux440ux430ux442ux443ux440ux430}}
%
%\begin{enumerate}
%\def\labelenumi{\arabic{enumi}.}
%\tightlist
%\item
%  Ширяев А.Н. Вероятность --- 1. М.: МЦНМО, 2007. 517 с.
%\item
%  Чернова Н.И. Теория вероятностей. Учебное пособие. Новосибирск, 2007.
%  160 с.
%\item
%  Липцер Р.Ш., Ширяев А.Н. Статистика случайных процессов (нелинейная
%  фильтрация и смежные вопросы). М.: Наука, 1974. 697 с.
%\end{enumerate}


    % Add a bibliography block to the postdoc
    
    
    
\end{document}
